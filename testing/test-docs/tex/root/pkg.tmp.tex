\chapter*{\color{headtoc} root}
\hypertarget{ecldoc:toc:root}{}
\hyperlink{ecldoc:toc:}{Go Up}


\section*{Table of Contents}
{\renewcommand{\arraystretch}{1.5}
\begin{longtable}{|p{\textwidth}|}
\hline
\hyperlink{ecldoc:toc:example}{example.ecl} \\
Basic Example with : \\
\hline
\hyperlink{ecldoc:toc:example_10}{example\_10.ecl} \\
\hline
\hyperlink{ecldoc:toc:example_11}{example\_11.ecl} \\
\hline
\hyperlink{ecldoc:toc:example_2}{example\_2.ecl} \\
Basic Inheritance documentation : mod\_3 inherits both mod\_1 and mod\_2 \\
\hline
\hyperlink{ecldoc:toc:example_3}{example\_3.ecl} \\
Documentation Testing Multiline Title \\
\hline
\hyperlink{ecldoc:toc:example_4}{example\_4.ecl} \\
Example : Inheritance across files \\
\hline
\hyperlink{ecldoc:toc:example_7}{example\_7.ecl} \\
Basic Type Example \\
\hline
\hyperlink{ecldoc:toc:Math}{Math.ecl} \\
\hline
\hyperlink{ecldoc:toc:test}{test.ecl} \\
test module \\
\hline
\hyperlink{ecldoc:toc:types}{types.ecl} \\
\hline
\end{longtable}
}

\chapter*{\color{headfile}
example
}
\hypertarget{ecldoc:toc:example}{}
\hyperlink{ecldoc:toc:root}{Go Up}


\section*{\underline{\textsf{DESCRIPTIONS}}}
\subsection*{\textsf{\colorbox{headtoc}{\color{white} MODULE}
example}}

\hypertarget{ecldoc:example}{}

{\renewcommand{\arraystretch}{1.5}
\begin{tabularx}{\textwidth}{|>{\raggedright\arraybackslash}l|X|}
\hline
\hspace{0pt}\mytexttt{\color{red} } & \textbf{example} \\
\hline
\end{tabularx}
}

\par





Basic Example with : records, interface, function, modules, transform, embed, macros and functionmacro







\textbf{Children}
\begin{enumerate}
\item \hyperlink{ecldoc:example.rec_1}{rec\_1}
: No Documentation Found
\item \hyperlink{ecldoc:example.rec_2}{rec\_2}
: No Documentation Found
\item \hyperlink{ecldoc:example.interface_ex}{interface\_ex}
: No Documentation Found
\item \hyperlink{ecldoc:example.func_1}{func\_1}
: No Documentation Found
\item \hyperlink{ecldoc:example.func_2}{func\_2}
: No Documentation Found
\item \hyperlink{ecldoc:example.mod_1}{mod\_1}
: No Documentation Found
\item \hyperlink{ecldoc:example.mod_2}{mod\_2}
: No Documentation Found
\item \hyperlink{ecldoc:example.cpp_1}{cpp\_1}
: No Documentation Found
\item \hyperlink{ecldoc:example.funcmacro_1}{funcmacro\_1}
: No Documentation Found
\item \hyperlink{ecldoc:example.macro_1}{macro\_1}
: No Documentation Found
\item \hyperlink{ecldoc:example.macro_2}{macro\_2}
: No Documentation Found
\end{enumerate}

\rule{\linewidth}{0.5pt}

\subsection*{\textsf{\colorbox{headtoc}{\color{white} RECORD}
rec\_1}}

\hypertarget{ecldoc:example.rec_1}{}
\hspace{0pt} \hyperlink{ecldoc:example}{example} \textbackslash 

{\renewcommand{\arraystretch}{1.5}
\begin{tabularx}{\textwidth}{|>{\raggedright\arraybackslash}l|X|}
\hline
\hspace{0pt}\mytexttt{\color{red} } & \textbf{rec\_1} \\
\hline
\end{tabularx}
}

\par





No Documentation Found







\par
\begin{description}
\item [\colorbox{tagtype}{\color{white} \textbf{\textsf{FIELD}}}] \textbf{\underline{g}} ||| REAL8 --- No Doc
\item [\colorbox{tagtype}{\color{white} \textbf{\textsf{FIELD}}}] \textbf{\underline{f}} ||| REAL8 --- No Doc
\end{description}





\rule{\linewidth}{0.5pt}
\subsection*{\textsf{\colorbox{headtoc}{\color{white} RECORD}
rec\_2}}

\hypertarget{ecldoc:example.rec_2}{}
\hspace{0pt} \hyperlink{ecldoc:example}{example} \textbackslash 

{\renewcommand{\arraystretch}{1.5}
\begin{tabularx}{\textwidth}{|>{\raggedright\arraybackslash}l|X|}
\hline
\hspace{0pt}\mytexttt{\color{red} } & \textbf{rec\_2} \\
\hline
\end{tabularx}
}

\par





No Documentation Found







\par
\begin{description}
\item [\colorbox{tagtype}{\color{white} \textbf{\textsf{FIELD}}}] \textbf{\underline{g}} ||| REAL8 --- No Doc
\item [\colorbox{tagtype}{\color{white} \textbf{\textsf{FIELD}}}] \textbf{\underline{a}} ||| UNSIGNED4 --- No Doc
\item [\colorbox{tagtype}{\color{white} \textbf{\textsf{FIELD}}}] \textbf{\underline{b}} ||| REAL8 --- No Doc
\end{description}





\rule{\linewidth}{0.5pt}
\subsection*{\textsf{\colorbox{headtoc}{\color{white} INTERFACE}
interface\_ex}}

\hypertarget{ecldoc:example.interface_ex}{}
\hspace{0pt} \hyperlink{ecldoc:example}{example} \textbackslash 

{\renewcommand{\arraystretch}{1.5}
\begin{tabularx}{\textwidth}{|>{\raggedright\arraybackslash}l|X|}
\hline
\hspace{0pt}\mytexttt{\color{red} } & \textbf{interface\_ex} \\
\hline
\end{tabularx}
}

\par





No Documentation Found







\textbf{Children}
\begin{enumerate}
\item \hyperlink{ecldoc:example.interface_ex.iface_v3}{iface\_v3}
: No Documentation Found
\end{enumerate}

\rule{\linewidth}{0.5pt}

\subsection*{\textsf{\colorbox{headtoc}{\color{white} ATTRIBUTE}
iface\_v3}}

\hypertarget{ecldoc:example.interface_ex.iface_v3}{}
\hspace{0pt} \hyperlink{ecldoc:example}{example} \textbackslash 
\hspace{0pt} \hyperlink{ecldoc:example.interface_ex}{interface_ex} \textbackslash 

{\renewcommand{\arraystretch}{1.5}
\begin{tabularx}{\textwidth}{|>{\raggedright\arraybackslash}l|X|}
\hline
\hspace{0pt}\mytexttt{\color{red} STRING25} & \textbf{iface\_v3} \\
\hline
\end{tabularx}
}

\par





No Documentation Found








\par
\begin{description}
\item [\colorbox{tagtype}{\color{white} \textbf{\textsf{RETURN}}}] \textbf{STRING25} --- 
\end{description}




\rule{\linewidth}{0.5pt}


\subsection*{\textsf{\colorbox{headtoc}{\color{white} FUNCTION}
func\_1}}

\hypertarget{ecldoc:example.func_1}{}
\hspace{0pt} \hyperlink{ecldoc:example}{example} \textbackslash 

{\renewcommand{\arraystretch}{1.5}
\begin{tabularx}{\textwidth}{|>{\raggedright\arraybackslash}l|X|}
\hline
\hspace{0pt}\mytexttt{\color{red} } & \textbf{func\_1} \\
\hline
\multicolumn{2}{|>{\raggedright\arraybackslash}X|}{\hspace{0pt}\mytexttt{\color{param} (REAL8 x, STRING25 y)}} \\
\hline
\end{tabularx}
}

\par





No Documentation Found






\par
\begin{description}
\item [\colorbox{tagtype}{\color{white} \textbf{\textsf{PARAMETER}}}] \textbf{\underline{y}} ||| STRING25 --- No Doc
\item [\colorbox{tagtype}{\color{white} \textbf{\textsf{PARAMETER}}}] \textbf{\underline{x}} ||| REAL8 --- No Doc
\end{description}







\par
\begin{description}
\item [\colorbox{tagtype}{\color{white} \textbf{\textsf{RETURN}}}] \textbf{REAL8} --- 
\end{description}




\rule{\linewidth}{0.5pt}
\subsection*{\textsf{\colorbox{headtoc}{\color{white} FUNCTION}
func\_2}}

\hypertarget{ecldoc:example.func_2}{}
\hspace{0pt} \hyperlink{ecldoc:example}{example} \textbackslash 

{\renewcommand{\arraystretch}{1.5}
\begin{tabularx}{\textwidth}{|>{\raggedright\arraybackslash}l|X|}
\hline
\hspace{0pt}\mytexttt{\color{red} DATASET(rec\_2)} & \textbf{func\_2} \\
\hline
\multicolumn{2}{|>{\raggedright\arraybackslash}X|}{\hspace{0pt}\mytexttt{\color{param} (DATASET(rec\_1) d)}} \\
\hline
\end{tabularx}
}

\par





No Documentation Found






\par
\begin{description}
\item [\colorbox{tagtype}{\color{white} \textbf{\textsf{PARAMETER}}}] \textbf{\underline{d}} ||| TABLE ( rec\_1 ) --- No Doc
\end{description}







\par
\begin{description}
\item [\colorbox{tagtype}{\color{white} \textbf{\textsf{RETURN}}}] \textbf{TABLE ( \{ UNSIGNED4 a , REAL8 b , REAL8 g \} )} --- 
\end{description}




\rule{\linewidth}{0.5pt}
\subsection*{\textsf{\colorbox{headtoc}{\color{white} MODULE}
mod\_1}}

\hypertarget{ecldoc:example.mod_1}{}
\hspace{0pt} \hyperlink{ecldoc:example}{example} \textbackslash 

{\renewcommand{\arraystretch}{1.5}
\begin{tabularx}{\textwidth}{|>{\raggedright\arraybackslash}l|X|}
\hline
\hspace{0pt}\mytexttt{\color{red} } & \textbf{mod\_1} \\
\hline
\multicolumn{2}{|>{\raggedright\arraybackslash}X|}{\hspace{0pt}\mytexttt{\color{param} (REAL8 a)}} \\
\hline
\end{tabularx}
}

\par





No Documentation Found






\par
\begin{description}
\item [\colorbox{tagtype}{\color{white} \textbf{\textsf{PARAMETER}}}] \textbf{\underline{a}} ||| REAL8 --- No Doc
\end{description}






\textbf{Children}
\begin{enumerate}
\item \hyperlink{ecldoc:example.mod_1.pi_w}{pi\_w}
: No Documentation Found
\end{enumerate}

\rule{\linewidth}{0.5pt}

\subsection*{\textsf{\colorbox{headtoc}{\color{white} ATTRIBUTE}
pi\_w}}

\hypertarget{ecldoc:example.mod_1.pi_w}{}
\hspace{0pt} \hyperlink{ecldoc:example}{example} \textbackslash 
\hspace{0pt} \hyperlink{ecldoc:example.mod_1}{mod_1} \textbackslash 

{\renewcommand{\arraystretch}{1.5}
\begin{tabularx}{\textwidth}{|>{\raggedright\arraybackslash}l|X|}
\hline
\hspace{0pt}\mytexttt{\color{red} } & \textbf{pi\_w} \\
\hline
\end{tabularx}
}

\par





No Documentation Found








\par
\begin{description}
\item [\colorbox{tagtype}{\color{white} \textbf{\textsf{RETURN}}}] \textbf{REAL8} --- 
\end{description}




\rule{\linewidth}{0.5pt}


\subsection*{\textsf{\colorbox{headtoc}{\color{white} MODULE}
mod\_2}}

\hypertarget{ecldoc:example.mod_2}{}
\hspace{0pt} \hyperlink{ecldoc:example}{example} \textbackslash 

{\renewcommand{\arraystretch}{1.5}
\begin{tabularx}{\textwidth}{|>{\raggedright\arraybackslash}l|X|}
\hline
\hspace{0pt}\mytexttt{\color{red} } & \textbf{mod\_2} \\
\hline
\end{tabularx}
}

\par





No Documentation Found







\textbf{Children}
\begin{enumerate}
\item \hyperlink{ecldoc:example.mod_2.pi_wo}{pi\_wo}
: No Documentation Found
\end{enumerate}

\rule{\linewidth}{0.5pt}

\subsection*{\textsf{\colorbox{headtoc}{\color{white} ATTRIBUTE}
pi\_wo}}

\hypertarget{ecldoc:example.mod_2.pi_wo}{}
\hspace{0pt} \hyperlink{ecldoc:example}{example} \textbackslash 
\hspace{0pt} \hyperlink{ecldoc:example.mod_2}{mod_2} \textbackslash 

{\renewcommand{\arraystretch}{1.5}
\begin{tabularx}{\textwidth}{|>{\raggedright\arraybackslash}l|X|}
\hline
\hspace{0pt}\mytexttt{\color{red} } & \textbf{pi\_wo} \\
\hline
\end{tabularx}
}

\par





No Documentation Found








\par
\begin{description}
\item [\colorbox{tagtype}{\color{white} \textbf{\textsf{RETURN}}}] \textbf{REAL8} --- 
\end{description}




\rule{\linewidth}{0.5pt}


\subsection*{\textsf{\colorbox{headtoc}{\color{white} EMBED}
cpp\_1}}

\hypertarget{ecldoc:example.cpp_1}{}
\hspace{0pt} \hyperlink{ecldoc:example}{example} \textbackslash 

{\renewcommand{\arraystretch}{1.5}
\begin{tabularx}{\textwidth}{|>{\raggedright\arraybackslash}l|X|}
\hline
\hspace{0pt}\mytexttt{\color{red} DATA} & \textbf{cpp\_1} \\
\hline
\multicolumn{2}{|>{\raggedright\arraybackslash}X|}{\hspace{0pt}\mytexttt{\color{param} (REAL8 varcpp)}} \\
\hline
\end{tabularx}
}

\par





No Documentation Found






\par
\begin{description}
\item [\colorbox{tagtype}{\color{white} \textbf{\textsf{PARAMETER}}}] \textbf{\underline{varcpp}} ||| REAL8 --- No Doc
\end{description}







\par
\begin{description}
\item [\colorbox{tagtype}{\color{white} \textbf{\textsf{RETURN}}}] \textbf{DATA} --- 
\end{description}




\rule{\linewidth}{0.5pt}
\subsection*{\textsf{\colorbox{headtoc}{\color{white} MACRO}
funcmacro\_1}}

\hypertarget{ecldoc:example.funcmacro_1}{}
\hspace{0pt} \hyperlink{ecldoc:example}{example} \textbackslash 

{\renewcommand{\arraystretch}{1.5}
\begin{tabularx}{\textwidth}{|>{\raggedright\arraybackslash}l|X|}
\hline
\hspace{0pt}\mytexttt{\color{red} } & \textbf{funcmacro\_1} \\
\hline
\multicolumn{2}{|>{\raggedright\arraybackslash}X|}{\hspace{0pt}\mytexttt{\color{param} (num)}} \\
\hline
\end{tabularx}
}

\par





No Documentation Found






\par
\begin{description}
\item [\colorbox{tagtype}{\color{white} \textbf{\textsf{PARAMETER}}}] \textbf{\underline{num}} ||| INTEGER8 --- No Doc
\end{description}







\par
\begin{description}
\item [\colorbox{tagtype}{\color{white} \textbf{\textsf{RETURN}}}] \textbf{BOOLEAN} --- 
\end{description}




\rule{\linewidth}{0.5pt}
\subsection*{\textsf{\colorbox{headtoc}{\color{white} MACRO}
macro\_1}}

\hypertarget{ecldoc:example.macro_1}{}
\hspace{0pt} \hyperlink{ecldoc:example}{example} \textbackslash 

{\renewcommand{\arraystretch}{1.5}
\begin{tabularx}{\textwidth}{|>{\raggedright\arraybackslash}l|X|}
\hline
\hspace{0pt}\mytexttt{\color{red} } & \textbf{macro\_1} \\
\hline
\multicolumn{2}{|>{\raggedright\arraybackslash}X|}{\hspace{0pt}\mytexttt{\color{param} (num\_1, num\_2)}} \\
\hline
\end{tabularx}
}

\par





No Documentation Found






\par
\begin{description}
\item [\colorbox{tagtype}{\color{white} \textbf{\textsf{PARAMETER}}}] \textbf{\underline{num\_2}} ||| INTEGER8 --- No Doc
\item [\colorbox{tagtype}{\color{white} \textbf{\textsf{PARAMETER}}}] \textbf{\underline{num\_1}} ||| INTEGER8 --- No Doc
\end{description}







\par
\begin{description}
\item [\colorbox{tagtype}{\color{white} \textbf{\textsf{RETURN}}}] \textbf{} --- 
\end{description}




\rule{\linewidth}{0.5pt}
\subsection*{\textsf{\colorbox{headtoc}{\color{white} MACRO}
macro\_2}}

\hypertarget{ecldoc:example.macro_2}{}
\hspace{0pt} \hyperlink{ecldoc:example}{example} \textbackslash 

{\renewcommand{\arraystretch}{1.5}
\begin{tabularx}{\textwidth}{|>{\raggedright\arraybackslash}l|X|}
\hline
\hspace{0pt}\mytexttt{\color{red} } & \textbf{macro\_2} \\
\hline
\end{tabularx}
}

\par





No Documentation Found








\par
\begin{description}
\item [\colorbox{tagtype}{\color{white} \textbf{\textsf{RETURN}}}] \textbf{} --- 
\end{description}




\rule{\linewidth}{0.5pt}



\chapter*{example\_10}
\hypertarget{ecldoc:toc:example_10}{}

\section*{\underline{IMPORTS}}
\begin{itemize}
\item intest
\end{itemize}

\section*{\underline{DESCRIPTIONS}}
\subsection*{MODULE : example\_10}
\hypertarget{ecldoc:example_10}{}
\hyperlink{ecldoc:toc:root}{Up} :

{\renewcommand{\arraystretch}{1.5}
\begin{tabularx}{\textwidth}{|>{\raggedright\arraybackslash}l|X|}
\hline
\hspace{0pt} & example\_10 \\
\hline
\end{tabularx}
}

\par


\hyperlink{ecldoc:intest.Example_3.mod_1}{mod\_1}  |

\rule{\linewidth}{0.5pt}

\subsection*{MODULE : mod\_1}
\hypertarget{ecldoc:intest.Example_3.mod_1}{}
\hyperlink{ecldoc:example_10}{Up} :
\hspace{0pt} \hyperlink{ecldoc:example_10}{example_10} \textbackslash 

{\renewcommand{\arraystretch}{1.5}
\begin{tabularx}{\textwidth}{|>{\raggedright\arraybackslash}l|X|}
\hline
\hspace{0pt} & mod\_1 \\
\hline
\end{tabularx}
}

\par

\par
\begin{description}
\item [\textbf{INHERITED}] True
\end{description}

\rule{\linewidth}{0.5pt}



\chapter*{example\_11}
\hypertarget{ecldoc:toc:example_11}{}

\section*{\underline{IMPORTS}}
\begin{itemize}
\item Inintest
\item Example\_3
\item intest.Example\_3
\item intest.inintest.Example\_3
\item Inintest.Example\_3
\end{itemize}

\section*{\underline{DESCRIPTIONS}}
\subsection*{MODULE : example\_11}
\hypertarget{ecldoc:example_11}{}
\par
\begin{minipage}[t]{\textwidth}
\begin{flushleft}
  
\end{flushleft}
\end{minipage}
\hyperlink{ecldoc:toc:root}{Up} \\
\par
\par
\begin{enumerate}
\item \hyperlink{ecldoc:Inintest.Example_3}{Example\_3}
\end{enumerate}
\subsection*{MODULE : Example\_3}
\hypertarget{ecldoc:Inintest.Example_3}{}
\par
\begin{minipage}[t]{\textwidth}
\begin{flushleft}
  
\end{flushleft}
\end{minipage}
\hyperlink{ecldoc:example_11}{Up} \\
\par
\par
\textbf{OVERRIDE} : True \\
\begin{enumerate}
\item \hyperlink{ecldoc:Inintest.Example_3.mod_1}{mod\_1}
\end{enumerate}
\subsection*{MODULE : mod\_1}
\hypertarget{ecldoc:Inintest.Example_3.mod_1}{}
\par
\begin{minipage}[t]{\textwidth}
\begin{flushleft}
  
\end{flushleft}
\end{minipage}
\hyperlink{ecldoc:Inintest.Example_3}{Up} \\
\par
\par
\begin{enumerate}
\item \hyperlink{ecldoc:inintest.example_3.mod_1.v2_m1_ex3}{v2\_m1\_ex3}
\end{enumerate}
\subsection*{ATTRIBUTE : v2\_m1\_ex3}
\hypertarget{ecldoc:inintest.example_3.mod_1.v2_m1_ex3}{}
\par
\begin{minipage}[t]{\textwidth}
\begin{flushleft}
  
\end{flushleft}
\end{minipage}
\hyperlink{ecldoc:Inintest.Example_3.mod_1}{Up} \\
\par
\par




\chapter*{\color{headfile}
example_2
}
\hypertarget{ecldoc:toc:example_2}{}
\hyperlink{ecldoc:toc:root}{Go Up}


\section*{\underline{\textsf{DESCRIPTIONS}}}
\subsection*{\textsf{\colorbox{headtoc}{\color{white} MODULE}
example\_2}}

\hypertarget{ecldoc:example_2}{}

{\renewcommand{\arraystretch}{1.5}
\begin{tabularx}{\textwidth}{|>{\raggedright\arraybackslash}l|X|}
\hline
\hspace{0pt}\mytexttt{\color{red} } & \textbf{example\_2} \\
\hline
\end{tabularx}
}

\par
Basic Inheritance documentation : mod\_3 inherits both mod\_1 and mod\_2 . Inherits v2\_m1, v2\_m2, Overrides v1\_m1, new locals v2\_m3 . Interface Inheritance : mod\_4 inherits interface iface\_1, overrides v1\_i1


\textbf{Children}
\begin{enumerate}
\item \hyperlink{ecldoc:example_2.rec_1}{rec\_1}
\item \hyperlink{ecldoc:example_2.rec_2}{rec\_2}
\item \hyperlink{ecldoc:example_2.rec_3}{rec\_3}
\item \hyperlink{ecldoc:example_2.mod_1}{mod\_1}
\item \hyperlink{ecldoc:example_2.mod_2}{mod\_2}
\item \hyperlink{ecldoc:example_2.mod_3}{mod\_3}
\item \hyperlink{ecldoc:example_2.iface_1}{iface\_1}
\item \hyperlink{ecldoc:example_2.mod_4}{mod\_4}
\end{enumerate}

\rule{\linewidth}{0.5pt}

\subsection*{\textsf{\colorbox{headtoc}{\color{white} RECORD}
rec\_1}}

\hypertarget{ecldoc:example_2.rec_1}{}
\hspace{0pt} \hyperlink{ecldoc:example_2}{example_2} \textbackslash 

{\renewcommand{\arraystretch}{1.5}
\begin{tabularx}{\textwidth}{|>{\raggedright\arraybackslash}l|X|}
\hline
\hspace{0pt}\mytexttt{\color{red} } & \textbf{rec\_1} \\
\hline
\end{tabularx}
}

\par


\rule{\linewidth}{0.5pt}
\subsection*{\textsf{\colorbox{headtoc}{\color{white} RECORD}
rec\_2}}

\hypertarget{ecldoc:example_2.rec_2}{}
\hspace{0pt} \hyperlink{ecldoc:example_2}{example_2} \textbackslash 

{\renewcommand{\arraystretch}{1.5}
\begin{tabularx}{\textwidth}{|>{\raggedright\arraybackslash}l|X|}
\hline
\hspace{0pt}\mytexttt{\color{red} } & \textbf{rec\_2} \\
\hline
\end{tabularx}
}

\par


\rule{\linewidth}{0.5pt}
\subsection*{\textsf{\colorbox{headtoc}{\color{white} RECORD}
rec\_3}}

\hypertarget{ecldoc:example_2.rec_3}{}
\hspace{0pt} \hyperlink{ecldoc:example_2}{example_2} \textbackslash 

{\renewcommand{\arraystretch}{1.5}
\begin{tabularx}{\textwidth}{|>{\raggedright\arraybackslash}l|X|}
\hline
\hspace{0pt}\mytexttt{\color{red} } & \textbf{rec\_3} \\
\hline
\end{tabularx}
}

\par


\rule{\linewidth}{0.5pt}
\subsection*{\textsf{\colorbox{headtoc}{\color{white} MODULE}
mod\_1}}

\hypertarget{ecldoc:example_2.mod_1}{}
\hspace{0pt} \hyperlink{ecldoc:example_2}{example_2} \textbackslash 

{\renewcommand{\arraystretch}{1.5}
\begin{tabularx}{\textwidth}{|>{\raggedright\arraybackslash}l|X|}
\hline
\hspace{0pt}\mytexttt{\color{red} } & \textbf{mod\_1} \\
\hline
\end{tabularx}
}

\par


\textbf{Children}
\begin{enumerate}
\item \hyperlink{ecldoc:example_2.mod_1.v1_m1}{v1\_m1}
\item \hyperlink{ecldoc:example_2.mod_1.v2_m1}{v2\_m1}
\end{enumerate}

\rule{\linewidth}{0.5pt}

\subsection*{\textsf{\colorbox{headtoc}{\color{white} ATTRIBUTE}
v1\_m1}}

\hypertarget{ecldoc:example_2.mod_1.v1_m1}{}
\hspace{0pt} \hyperlink{ecldoc:example_2}{example_2} \textbackslash 
\hspace{0pt} \hyperlink{ecldoc:example_2.mod_1}{mod_1} \textbackslash 

{\renewcommand{\arraystretch}{1.5}
\begin{tabularx}{\textwidth}{|>{\raggedright\arraybackslash}l|X|}
\hline
\hspace{0pt}\mytexttt{\color{red} real8} & \textbf{v1\_m1} \\
\hline
\end{tabularx}
}

\par


\rule{\linewidth}{0.5pt}
\subsection*{\textsf{\colorbox{headtoc}{\color{white} ATTRIBUTE}
v2\_m1}}

\hypertarget{ecldoc:example_2.mod_1.v2_m1}{}
\hspace{0pt} \hyperlink{ecldoc:example_2}{example_2} \textbackslash 
\hspace{0pt} \hyperlink{ecldoc:example_2.mod_1}{mod_1} \textbackslash 

{\renewcommand{\arraystretch}{1.5}
\begin{tabularx}{\textwidth}{|>{\raggedright\arraybackslash}l|X|}
\hline
\hspace{0pt}\mytexttt{\color{red} } & \textbf{v2\_m1} \\
\hline
\end{tabularx}
}

\par


\rule{\linewidth}{0.5pt}


\subsection*{\textsf{\colorbox{headtoc}{\color{white} MODULE}
mod\_2}}

\hypertarget{ecldoc:example_2.mod_2}{}
\hspace{0pt} \hyperlink{ecldoc:example_2}{example_2} \textbackslash 

{\renewcommand{\arraystretch}{1.5}
\begin{tabularx}{\textwidth}{|>{\raggedright\arraybackslash}l|X|}
\hline
\hspace{0pt}\mytexttt{\color{red} } & \textbf{mod\_2} \\
\hline
\end{tabularx}
}

\par


\textbf{Children}
\begin{enumerate}
\item \hyperlink{ecldoc:example_2.mod_2.v1_m1}{v1\_m1}
\item \hyperlink{ecldoc:example_2.mod_2.v2_m2}{v2\_m2}
\end{enumerate}

\rule{\linewidth}{0.5pt}

\subsection*{\textsf{\colorbox{headtoc}{\color{white} ATTRIBUTE}
v1\_m1}}

\hypertarget{ecldoc:example_2.mod_2.v1_m1}{}
\hspace{0pt} \hyperlink{ecldoc:example_2}{example_2} \textbackslash 
\hspace{0pt} \hyperlink{ecldoc:example_2.mod_2}{mod_2} \textbackslash 

{\renewcommand{\arraystretch}{1.5}
\begin{tabularx}{\textwidth}{|>{\raggedright\arraybackslash}l|X|}
\hline
\hspace{0pt}\mytexttt{\color{red} } & \textbf{v1\_m1} \\
\hline
\end{tabularx}
}

\par


\rule{\linewidth}{0.5pt}
\subsection*{\textsf{\colorbox{headtoc}{\color{white} ATTRIBUTE}
v2\_m2}}

\hypertarget{ecldoc:example_2.mod_2.v2_m2}{}
\hspace{0pt} \hyperlink{ecldoc:example_2}{example_2} \textbackslash 
\hspace{0pt} \hyperlink{ecldoc:example_2.mod_2}{mod_2} \textbackslash 

{\renewcommand{\arraystretch}{1.5}
\begin{tabularx}{\textwidth}{|>{\raggedright\arraybackslash}l|X|}
\hline
\hspace{0pt}\mytexttt{\color{red} } & \textbf{v2\_m2} \\
\hline
\end{tabularx}
}

\par


\rule{\linewidth}{0.5pt}


\subsection*{\textsf{\colorbox{headtoc}{\color{white} MODULE}
mod\_3}}

\hypertarget{ecldoc:example_2.mod_3}{}
\hspace{0pt} \hyperlink{ecldoc:example_2}{example_2} \textbackslash 

{\renewcommand{\arraystretch}{1.5}
\begin{tabularx}{\textwidth}{|>{\raggedright\arraybackslash}l|X|}
\hline
\hspace{0pt}\mytexttt{\color{red} } & \textbf{mod\_3} \\
\hline
\end{tabularx}
}

\par


\textbf{Children}
\begin{enumerate}
\item \hyperlink{ecldoc:example_2.mod_1.v2_m1}{v2\_m1}
\item \hyperlink{ecldoc:example_2.mod_2.v2_m2}{v2\_m2}
\item \hyperlink{ecldoc:example_2.mod_3.v1_m1}{v1\_m1}
\item \hyperlink{ecldoc:example_2.mod_3.v2_m3}{v2\_m3}
\end{enumerate}

\rule{\linewidth}{0.5pt}

\subsection*{\textsf{\colorbox{headtoc}{\color{white} ATTRIBUTE}
v2\_m1}}

\hypertarget{ecldoc:example_2.mod_1.v2_m1}{}
\hspace{0pt} \hyperlink{ecldoc:example_2}{example_2} \textbackslash 
\hspace{0pt} \hyperlink{ecldoc:example_2.mod_3}{mod_3} \textbackslash 

{\renewcommand{\arraystretch}{1.5}
\begin{tabularx}{\textwidth}{|>{\raggedright\arraybackslash}l|X|}
\hline
\hspace{0pt}\mytexttt{\color{red} } & \textbf{v2\_m1} \\
\hline
\end{tabularx}
}

\par

\par
\begin{description}
\item [\colorbox{tagtype}{\color{white} \textbf{\textsf{INHERITED}}}] \textbf{\underline{}} True
\end{description}

\rule{\linewidth}{0.5pt}
\subsection*{\textsf{\colorbox{headtoc}{\color{white} ATTRIBUTE}
v2\_m2}}

\hypertarget{ecldoc:example_2.mod_2.v2_m2}{}
\hspace{0pt} \hyperlink{ecldoc:example_2}{example_2} \textbackslash 
\hspace{0pt} \hyperlink{ecldoc:example_2.mod_3}{mod_3} \textbackslash 

{\renewcommand{\arraystretch}{1.5}
\begin{tabularx}{\textwidth}{|>{\raggedright\arraybackslash}l|X|}
\hline
\hspace{0pt}\mytexttt{\color{red} } & \textbf{v2\_m2} \\
\hline
\end{tabularx}
}

\par

\par
\begin{description}
\item [\colorbox{tagtype}{\color{white} \textbf{\textsf{INHERITED}}}] \textbf{\underline{}} True
\end{description}

\rule{\linewidth}{0.5pt}
\subsection*{\textsf{\colorbox{headtoc}{\color{white} ATTRIBUTE}
v1\_m1}}

\hypertarget{ecldoc:example_2.mod_3.v1_m1}{}
\hspace{0pt} \hyperlink{ecldoc:example_2}{example_2} \textbackslash 
\hspace{0pt} \hyperlink{ecldoc:example_2.mod_3}{mod_3} \textbackslash 

{\renewcommand{\arraystretch}{1.5}
\begin{tabularx}{\textwidth}{|>{\raggedright\arraybackslash}l|X|}
\hline
\hspace{0pt}\mytexttt{\color{red} } & \textbf{v1\_m1} \\
\hline
\end{tabularx}
}

\par

\par
\begin{description}
\item [\colorbox{tagtype}{\color{white} \textbf{\textsf{OVERRIDE}}}] \textbf{\underline{}} True
\end{description}

\rule{\linewidth}{0.5pt}
\subsection*{\textsf{\colorbox{headtoc}{\color{white} ATTRIBUTE}
v2\_m3}}

\hypertarget{ecldoc:example_2.mod_3.v2_m3}{}
\hspace{0pt} \hyperlink{ecldoc:example_2}{example_2} \textbackslash 
\hspace{0pt} \hyperlink{ecldoc:example_2.mod_3}{mod_3} \textbackslash 

{\renewcommand{\arraystretch}{1.5}
\begin{tabularx}{\textwidth}{|>{\raggedright\arraybackslash}l|X|}
\hline
\hspace{0pt}\mytexttt{\color{red} } & \textbf{v2\_m3} \\
\hline
\end{tabularx}
}

\par


\rule{\linewidth}{0.5pt}


\subsection*{\textsf{\colorbox{headtoc}{\color{white} INTERFACE}
iface\_1}}

\hypertarget{ecldoc:example_2.iface_1}{}
\hspace{0pt} \hyperlink{ecldoc:example_2}{example_2} \textbackslash 

{\renewcommand{\arraystretch}{1.5}
\begin{tabularx}{\textwidth}{|>{\raggedright\arraybackslash}l|X|}
\hline
\hspace{0pt}\mytexttt{\color{red} } & \textbf{iface\_1} \\
\hline
\end{tabularx}
}

\par


\textbf{Children}
\begin{enumerate}
\item \hyperlink{ecldoc:example_2.iface_1.v1_i1}{v1\_i1}
\end{enumerate}

\rule{\linewidth}{0.5pt}

\subsection*{\textsf{\colorbox{headtoc}{\color{white} ATTRIBUTE}
v1\_i1}}

\hypertarget{ecldoc:example_2.iface_1.v1_i1}{}
\hspace{0pt} \hyperlink{ecldoc:example_2}{example_2} \textbackslash 
\hspace{0pt} \hyperlink{ecldoc:example_2.iface_1}{iface_1} \textbackslash 

{\renewcommand{\arraystretch}{1.5}
\begin{tabularx}{\textwidth}{|>{\raggedright\arraybackslash}l|X|}
\hline
\hspace{0pt}\mytexttt{\color{red} real8} & \textbf{v1\_i1} \\
\hline
\end{tabularx}
}

\par


\rule{\linewidth}{0.5pt}


\subsection*{\textsf{\colorbox{headtoc}{\color{white} MODULE}
mod\_4}}

\hypertarget{ecldoc:example_2.mod_4}{}
\hspace{0pt} \hyperlink{ecldoc:example_2}{example_2} \textbackslash 

{\renewcommand{\arraystretch}{1.5}
\begin{tabularx}{\textwidth}{|>{\raggedright\arraybackslash}l|X|}
\hline
\hspace{0pt}\mytexttt{\color{red} } & \textbf{mod\_4} \\
\hline
\end{tabularx}
}

\par


\textbf{Children}
\begin{enumerate}
\item \hyperlink{ecldoc:example_2.mod_4.v1_i1}{v1\_i1}
\item \hyperlink{ecldoc:example_2.mod_4.v2_m4}{v2\_m4}
\end{enumerate}

\rule{\linewidth}{0.5pt}

\subsection*{\textsf{\colorbox{headtoc}{\color{white} ATTRIBUTE}
v1\_i1}}

\hypertarget{ecldoc:example_2.mod_4.v1_i1}{}
\hspace{0pt} \hyperlink{ecldoc:example_2}{example_2} \textbackslash 
\hspace{0pt} \hyperlink{ecldoc:example_2.mod_4}{mod_4} \textbackslash 

{\renewcommand{\arraystretch}{1.5}
\begin{tabularx}{\textwidth}{|>{\raggedright\arraybackslash}l|X|}
\hline
\hspace{0pt}\mytexttt{\color{red} } & \textbf{v1\_i1} \\
\hline
\end{tabularx}
}

\par

\par
\begin{description}
\item [\colorbox{tagtype}{\color{white} \textbf{\textsf{OVERRIDE}}}] \textbf{\underline{}} True
\end{description}

\rule{\linewidth}{0.5pt}
\subsection*{\textsf{\colorbox{headtoc}{\color{white} ATTRIBUTE}
v2\_m4}}

\hypertarget{ecldoc:example_2.mod_4.v2_m4}{}
\hspace{0pt} \hyperlink{ecldoc:example_2}{example_2} \textbackslash 
\hspace{0pt} \hyperlink{ecldoc:example_2.mod_4}{mod_4} \textbackslash 

{\renewcommand{\arraystretch}{1.5}
\begin{tabularx}{\textwidth}{|>{\raggedright\arraybackslash}l|X|}
\hline
\hspace{0pt}\mytexttt{\color{red} STRING20} & \textbf{v2\_m4} \\
\hline
\end{tabularx}
}

\par


\rule{\linewidth}{0.5pt}





\chapter*{\color{headfile}
{\large intest\slash\hspace{0pt}}
{\large in1intest\slash\hspace{0pt}}
 \\
example_3
}
\hypertarget{ecldoc:toc:intest.in1intest.example_3}{}
\hyperlink{ecldoc:toc:root/intest/in1intest}{Go Up}


\section*{\underline{\textsf{DESCRIPTIONS}}}
\subsection*{\textsf{\colorbox{headtoc}{\color{white} MODULE}
Example\_3}}

\hypertarget{ecldoc:intest.in1intest.Example_3}{}

{\renewcommand{\arraystretch}{1.5}
\begin{tabularx}{\textwidth}{|>{\raggedright\arraybackslash}l|X|}
\hline
\hspace{0pt}\mytexttt{\color{red} } & \textbf{Example\_3} \\
\hline
\end{tabularx}
}

\par
Example : Inheritance across files mod\_1 in Example\_4 inherits mod\_1 in Example\_3


\textbf{Children}
\begin{enumerate}
\item \hyperlink{ecldoc:intest.in1intest.Example_3.mod_1}{mod\_1}
\end{enumerate}

\rule{\linewidth}{0.5pt}

\subsection*{\textsf{\colorbox{headtoc}{\color{white} MODULE}
mod\_1}}

\hypertarget{ecldoc:intest.in1intest.Example_3.mod_1}{}
\hspace{0pt} \hyperlink{ecldoc:intest.in1intest.Example_3}{Example_3} \textbackslash 

{\renewcommand{\arraystretch}{1.5}
\begin{tabularx}{\textwidth}{|>{\raggedright\arraybackslash}l|X|}
\hline
\hspace{0pt}\mytexttt{\color{red} } & \textbf{mod\_1} \\
\hline
\end{tabularx}
}

\par


\textbf{Children}
\begin{enumerate}
\item \hyperlink{ecldoc:intest.in1intest.example_3.mod_1.v1_m1}{v1\_m1}
\item \hyperlink{ecldoc:intest.in1intest.example_3.mod_1.v2_m1_ex3}{v2\_m1\_ex3}
\end{enumerate}

\rule{\linewidth}{0.5pt}

\subsection*{\textsf{\colorbox{headtoc}{\color{white} ATTRIBUTE}
v1\_m1}}

\hypertarget{ecldoc:intest.in1intest.example_3.mod_1.v1_m1}{}
\hspace{0pt} \hyperlink{ecldoc:intest.in1intest.Example_3}{Example_3} \textbackslash 
\hspace{0pt} \hyperlink{ecldoc:intest.in1intest.Example_3.mod_1}{mod_1} \textbackslash 

{\renewcommand{\arraystretch}{1.5}
\begin{tabularx}{\textwidth}{|>{\raggedright\arraybackslash}l|X|}
\hline
\hspace{0pt}\mytexttt{\color{red} } & \textbf{v1\_m1} \\
\hline
\end{tabularx}
}

\par


\rule{\linewidth}{0.5pt}
\subsection*{\textsf{\colorbox{headtoc}{\color{white} ATTRIBUTE}
v2\_m1\_ex3}}

\hypertarget{ecldoc:intest.in1intest.example_3.mod_1.v2_m1_ex3}{}
\hspace{0pt} \hyperlink{ecldoc:intest.in1intest.Example_3}{Example_3} \textbackslash 
\hspace{0pt} \hyperlink{ecldoc:intest.in1intest.Example_3.mod_1}{mod_1} \textbackslash 

{\renewcommand{\arraystretch}{1.5}
\begin{tabularx}{\textwidth}{|>{\raggedright\arraybackslash}l|X|}
\hline
\hspace{0pt}\mytexttt{\color{red} } & \textbf{v2\_m1\_ex3} \\
\hline
\end{tabularx}
}

\par


\rule{\linewidth}{0.5pt}





\chapter*{\color{headfile}
{\large intest\slash\hspace{0pt}}
 \\
example_4
}
\hypertarget{ecldoc:toc:intest.example_4}{}
\hyperlink{ecldoc:toc:root/intest}{Go Up}

\section*{\underline{\textsf{IMPORTS}}}
\begin{doublespace}
{\large
Example\_3.mod\_1 |
}
\end{doublespace}

\section*{\underline{\textsf{DESCRIPTIONS}}}
\subsection*{\textsf{\colorbox{headtoc}{\color{white} MODULE}
example\_4}}

\hypertarget{ecldoc:intest.example_4}{}

{\renewcommand{\arraystretch}{1.5}
\begin{tabularx}{\textwidth}{|>{\raggedright\arraybackslash}l|X|}
\hline
\hspace{0pt}\mytexttt{\color{red} } & \textbf{example\_4} \\
\hline
\end{tabularx}
}

\par





Example : Inheritance across files mod\_1 in Example\_4 inherits mod\_1 in Example\_3







\textbf{Children}
\begin{enumerate}
\item \hyperlink{ecldoc:intest.example_4.mod_1}{mod\_1}
: No Documentation Found
\end{enumerate}

\rule{\linewidth}{0.5pt}

\subsection*{\textsf{\colorbox{headtoc}{\color{white} MODULE}
mod\_1}}

\hypertarget{ecldoc:intest.example_4.mod_1}{}
\hspace{0pt} \hyperlink{ecldoc:intest.example_4}{example_4} \textbackslash 

{\renewcommand{\arraystretch}{1.5}
\begin{tabularx}{\textwidth}{|>{\raggedright\arraybackslash}l|X|}
\hline
\hspace{0pt}\mytexttt{\color{red} } & \textbf{mod\_1} \\
\hline
\end{tabularx}
}

\par





No Documentation Found










\par
\begin{description}
\item [\colorbox{tagtype}{\color{white} \textbf{\textsf{PARENT}}}] \textbf{Example\_3.mod\_1} <../example\_3.ecl.tex>
\end{description}


\textbf{Children}
\begin{enumerate}
\item \hyperlink{ecldoc:intest.example_4.mod_1.v2_m1_ex4}{v2\_m1\_ex4}
: No Documentation Found
\item \hyperlink{ecldoc:example_3.mod_1.v1_m1}{v1\_m1}
: Doc test 2
\item \hyperlink{ecldoc:example_3.mod_1.v2_m1_ex3}{v2\_m1\_ex3}
: DOC Test 3
\item \hyperlink{ecldoc:example_3.mod_1.abc}{abc}
: No Documentation Found
\item \hyperlink{ecldoc:example_3.mod_1.long_name}{long\_name}
: No Documentation Found
\end{enumerate}

\rule{\linewidth}{0.5pt}

\subsection*{\textsf{\colorbox{headtoc}{\color{white} ATTRIBUTE}
v2\_m1\_ex4}}

\hypertarget{ecldoc:intest.example_4.mod_1.v2_m1_ex4}{}
\hspace{0pt} \hyperlink{ecldoc:intest.example_4}{example_4} \textbackslash 
\hspace{0pt} \hyperlink{ecldoc:intest.example_4.mod_1}{mod_1} \textbackslash 

{\renewcommand{\arraystretch}{1.5}
\begin{tabularx}{\textwidth}{|>{\raggedright\arraybackslash}l|X|}
\hline
\hspace{0pt}\mytexttt{\color{red} } & \textbf{v2\_m1\_ex4} \\
\hline
\end{tabularx}
}

\par





No Documentation Found








\par
\begin{description}
\item [\colorbox{tagtype}{\color{white} \textbf{\textsf{RETURN}}}] \textbf{REAL8} --- 
\end{description}




\rule{\linewidth}{0.5pt}
\subsection*{\textsf{\colorbox{headtoc}{\color{white} ATTRIBUTE}
v1\_m1}}

\hypertarget{ecldoc:example_3.mod_1.v1_m1}{}
\hspace{0pt} \hyperlink{ecldoc:intest.example_4}{example_4} \textbackslash 
\hspace{0pt} \hyperlink{ecldoc:intest.example_4.mod_1}{mod_1} \textbackslash 

{\renewcommand{\arraystretch}{1.5}
\begin{tabularx}{\textwidth}{|>{\raggedright\arraybackslash}l|X|}
\hline
\hspace{0pt}\mytexttt{\color{red} } & \textbf{v1\_m1} \\
\hline
\end{tabularx}
}

\par





Doc test 2. Title end by period not newline 
\begin{verbatim}

 ABCD ||||
 CDEF ||||\end{verbatim}










\par
\begin{description}
\item [\colorbox{tagtype}{\color{white} \textbf{\textsf{RETURN}}}] \textbf{REAL8} --- 
\end{description}






\par
\begin{description}
\item [\colorbox{tagtype}{\color{white} \textbf{\textsf{INHERITED}}}] 
\end{description}



\rule{\linewidth}{0.5pt}
\subsection*{\textsf{\colorbox{headtoc}{\color{white} ATTRIBUTE}
v2\_m1\_ex3}}

\hypertarget{ecldoc:example_3.mod_1.v2_m1_ex3}{}
\hspace{0pt} \hyperlink{ecldoc:intest.example_4}{example_4} \textbackslash 
\hspace{0pt} \hyperlink{ecldoc:intest.example_4.mod_1}{mod_1} \textbackslash 

{\renewcommand{\arraystretch}{1.5}
\begin{tabularx}{\textwidth}{|>{\raggedright\arraybackslash}l|X|}
\hline
\hspace{0pt}\mytexttt{\color{red} } & \textbf{v2\_m1\_ex3} \\
\hline
\end{tabularx}
}

\par





DOC Test 3 No Period title








\par
\begin{description}
\item [\colorbox{tagtype}{\color{white} \textbf{\textsf{RETURN}}}] \textbf{REAL8} --- 
\end{description}






\par
\begin{description}
\item [\colorbox{tagtype}{\color{white} \textbf{\textsf{INHERITED}}}] 
\end{description}



\rule{\linewidth}{0.5pt}
\subsection*{\textsf{\colorbox{headtoc}{\color{white} FUNCTION}
abc}}

\hypertarget{ecldoc:example_3.mod_1.abc}{}
\hspace{0pt} \hyperlink{ecldoc:intest.example_4}{example_4} \textbackslash 
\hspace{0pt} \hyperlink{ecldoc:intest.example_4.mod_1}{mod_1} \textbackslash 

{\renewcommand{\arraystretch}{1.5}
\begin{tabularx}{\textwidth}{|>{\raggedright\arraybackslash}l|X|}
\hline
\hspace{0pt}\mytexttt{\color{red} REAL8} & \textbf{abc} \\
\hline
\multicolumn{2}{|>{\raggedright\arraybackslash}X|}{\hspace{0pt}\mytexttt{\color{param} (REAL8 x)}} \\
\hline
\end{tabularx}
}

\par





No Documentation Found






\par
\begin{description}
\item [\colorbox{tagtype}{\color{white} \textbf{\textsf{PARAMETER}}}] \textbf{\underline{x}} ||| REAL8 --- No Doc
\end{description}







\par
\begin{description}
\item [\colorbox{tagtype}{\color{white} \textbf{\textsf{RETURN}}}] \textbf{REAL8} --- 
\end{description}






\par
\begin{description}
\item [\colorbox{tagtype}{\color{white} \textbf{\textsf{INHERITED}}}] 
\end{description}



\rule{\linewidth}{0.5pt}
\subsection*{\textsf{\colorbox{headtoc}{\color{white} FUNCTION}
long\_name}}

\hypertarget{ecldoc:example_3.mod_1.long_name}{}
\hspace{0pt} \hyperlink{ecldoc:intest.example_4}{example_4} \textbackslash 
\hspace{0pt} \hyperlink{ecldoc:intest.example_4.mod_1}{mod_1} \textbackslash 

{\renewcommand{\arraystretch}{1.5}
\begin{tabularx}{\textwidth}{|>{\raggedright\arraybackslash}l|X|}
\hline
\hspace{0pt}\mytexttt{\color{red} } & \textbf{long\_name} \\
\hline
\multicolumn{2}{|>{\raggedright\arraybackslash}X|}{\hspace{0pt}\mytexttt{\color{param} (DATASET(\{REAL8 u\}) X, DATASET(\{REAL8 u\}) IntW, DATASET(\{REAL8 u\}) Intb, REAL8 BETA=0.1, REAL8 sparsityParam=0.1 , REAL8 LAMBDA=0.001, REAL8 ALPHA=0.1, UNSIGNED2 MaxIter=100)}} \\
\hline
\end{tabularx}
}

\par





No Documentation Found






\par
\begin{description}
\item [\colorbox{tagtype}{\color{white} \textbf{\textsf{PARAMETER}}}] \textbf{\underline{sparsityparam}} ||| REAL8 --- No Doc
\item [\colorbox{tagtype}{\color{white} \textbf{\textsf{PARAMETER}}}] \textbf{\underline{alpha}} ||| REAL8 --- No Doc
\item [\colorbox{tagtype}{\color{white} \textbf{\textsf{PARAMETER}}}] \textbf{\underline{lambda}} ||| REAL8 --- No Doc
\item [\colorbox{tagtype}{\color{white} \textbf{\textsf{PARAMETER}}}] \textbf{\underline{maxiter}} ||| UNSIGNED2 --- No Doc
\item [\colorbox{tagtype}{\color{white} \textbf{\textsf{PARAMETER}}}] \textbf{\underline{intb}} ||| TABLE ( \{ REAL8 u \} ) --- No Doc
\item [\colorbox{tagtype}{\color{white} \textbf{\textsf{PARAMETER}}}] \textbf{\underline{beta}} ||| REAL8 --- No Doc
\item [\colorbox{tagtype}{\color{white} \textbf{\textsf{PARAMETER}}}] \textbf{\underline{x}} ||| TABLE ( \{ REAL8 u \} ) --- No Doc
\item [\colorbox{tagtype}{\color{white} \textbf{\textsf{PARAMETER}}}] \textbf{\underline{intw}} ||| TABLE ( \{ REAL8 u \} ) --- No Doc
\end{description}







\par
\begin{description}
\item [\colorbox{tagtype}{\color{white} \textbf{\textsf{RETURN}}}] \textbf{REAL8} --- 
\end{description}






\par
\begin{description}
\item [\colorbox{tagtype}{\color{white} \textbf{\textsf{INHERITED}}}] 
\end{description}



\rule{\linewidth}{0.5pt}





\chapter*{\color{headfile}
example_7
}
\hypertarget{ecldoc:toc:example_7}{}
\hyperlink{ecldoc:toc:root}{Go Up}


\section*{\underline{\textsf{DESCRIPTIONS}}}
\subsection*{\textsf{\colorbox{headtoc}{\color{white} MODULE}
example\_7}}

\hypertarget{ecldoc:example_7}{}

{\renewcommand{\arraystretch}{1.5}
\begin{tabularx}{\textwidth}{|>{\raggedright\arraybackslash}l|X|}
\hline
\hspace{0pt}\mytexttt{\color{red} } & \textbf{example\_7} \\
\hline
\end{tabularx}
}

\par





Basic Type Example Source Code copied from ECL Documentation







\textbf{Children}
\begin{enumerate}
\item \hyperlink{ecldoc:example_7.r}{R}
: No Documentation Found
\end{enumerate}

\rule{\linewidth}{0.5pt}

\subsection*{\textsf{\colorbox{headtoc}{\color{white} RECORD}
R}}

\hypertarget{ecldoc:example_7.r}{}
\hspace{0pt} \hyperlink{ecldoc:example_7}{example_7} \textbackslash 

{\renewcommand{\arraystretch}{1.5}
\begin{tabularx}{\textwidth}{|>{\raggedright\arraybackslash}l|X|}
\hline
\hspace{0pt}\mytexttt{\color{red} } & \textbf{R} \\
\hline
\end{tabularx}
}

\par





No Documentation Found







\par
\begin{description}
\item [\colorbox{tagtype}{\color{white} \textbf{\textsf{FIELD}}}] \textbf{\underline{f3}} ||| SCALEINT --- No Doc
\item [\colorbox{tagtype}{\color{white} \textbf{\textsf{FIELD}}}] \textbf{\underline{f1}} ||| REVERSESTRING4 --- No Doc
\item [\colorbox{tagtype}{\color{white} \textbf{\textsf{FIELD}}}] \textbf{\underline{f2}} ||| NEEDC --- No Doc
\end{description}





\rule{\linewidth}{0.5pt}



\chapter*{\color{headfile}
Math
}
\hypertarget{ecldoc:toc:Math}{}
\hyperlink{ecldoc:toc:root}{Go Up}


\section*{\underline{\textsf{DESCRIPTIONS}}}
\subsection*{\textsf{\colorbox{headtoc}{\color{white} MODULE}
Math}}

\hypertarget{ecldoc:Math}{}

{\renewcommand{\arraystretch}{1.5}
\begin{tabularx}{\textwidth}{|>{\raggedright\arraybackslash}l|X|}
\hline
\hspace{0pt}\mytexttt{\color{red} } & \textbf{Math} \\
\hline
\end{tabularx}
}

\par





No Documentation Found







\textbf{Children}
\begin{enumerate}
\item \hyperlink{ecldoc:math.infinity}{Infinity}
: Return a real ''infinity'' value
\item \hyperlink{ecldoc:math.nan}{NaN}
: Return a non-signalling NaN (Not a Number)value
\item \hyperlink{ecldoc:math.isinfinite}{isInfinite}
: Return whether a real value is infinite (positive or negative)
\item \hyperlink{ecldoc:math.isnan}{isNaN}
: Return whether a real value is a NaN (not a number) value
\item \hyperlink{ecldoc:math.isfinite}{isFinite}
: Return whether a real value is a valid value (neither infinite not NaN)
\item \hyperlink{ecldoc:math.fmod}{FMod}
: Returns the floating-point remainder of numer/denom (rounded towards zero)
\item \hyperlink{ecldoc:math.fmatch}{FMatch}
: Returns whether two floating point values are the same, within margin of error epsilon
\end{enumerate}

\rule{\linewidth}{0.5pt}

\subsection*{\textsf{\colorbox{headtoc}{\color{white} ATTRIBUTE}
Infinity}}

\hypertarget{ecldoc:math.infinity}{}
\hspace{0pt} \hyperlink{ecldoc:Math}{Math} \textbackslash 

{\renewcommand{\arraystretch}{1.5}
\begin{tabularx}{\textwidth}{|>{\raggedright\arraybackslash}l|X|}
\hline
\hspace{0pt}\mytexttt{\color{red} REAL8} & \textbf{Infinity} \\
\hline
\end{tabularx}
}

\par





Return a real ''infinity'' value.








\par
\begin{description}
\item [\colorbox{tagtype}{\color{white} \textbf{\textsf{RETURN}}}] \textbf{REAL8} --- 
\end{description}




\rule{\linewidth}{0.5pt}
\subsection*{\textsf{\colorbox{headtoc}{\color{white} ATTRIBUTE}
NaN}}

\hypertarget{ecldoc:math.nan}{}
\hspace{0pt} \hyperlink{ecldoc:Math}{Math} \textbackslash 

{\renewcommand{\arraystretch}{1.5}
\begin{tabularx}{\textwidth}{|>{\raggedright\arraybackslash}l|X|}
\hline
\hspace{0pt}\mytexttt{\color{red} REAL8} & \textbf{NaN} \\
\hline
\end{tabularx}
}

\par





Return a non-signalling NaN (Not a Number)value.








\par
\begin{description}
\item [\colorbox{tagtype}{\color{white} \textbf{\textsf{RETURN}}}] \textbf{REAL8} --- 
\end{description}




\rule{\linewidth}{0.5pt}
\subsection*{\textsf{\colorbox{headtoc}{\color{white} FUNCTION}
isInfinite}}

\hypertarget{ecldoc:math.isinfinite}{}
\hspace{0pt} \hyperlink{ecldoc:Math}{Math} \textbackslash 

{\renewcommand{\arraystretch}{1.5}
\begin{tabularx}{\textwidth}{|>{\raggedright\arraybackslash}l|X|}
\hline
\hspace{0pt}\mytexttt{\color{red} BOOLEAN} & \textbf{isInfinite} \\
\hline
\multicolumn{2}{|>{\raggedright\arraybackslash}X|}{\hspace{0pt}\mytexttt{\color{param} (REAL8 val)}} \\
\hline
\end{tabularx}
}

\par





Return whether a real value is infinite (positive or negative).






\par
\begin{description}
\item [\colorbox{tagtype}{\color{white} \textbf{\textsf{PARAMETER}}}] \textbf{\underline{val}} ||| REAL8 --- The value to test.
\end{description}







\par
\begin{description}
\item [\colorbox{tagtype}{\color{white} \textbf{\textsf{RETURN}}}] \textbf{BOOLEAN} --- 
\end{description}




\rule{\linewidth}{0.5pt}
\subsection*{\textsf{\colorbox{headtoc}{\color{white} FUNCTION}
isNaN}}

\hypertarget{ecldoc:math.isnan}{}
\hspace{0pt} \hyperlink{ecldoc:Math}{Math} \textbackslash 

{\renewcommand{\arraystretch}{1.5}
\begin{tabularx}{\textwidth}{|>{\raggedright\arraybackslash}l|X|}
\hline
\hspace{0pt}\mytexttt{\color{red} BOOLEAN} & \textbf{isNaN} \\
\hline
\multicolumn{2}{|>{\raggedright\arraybackslash}X|}{\hspace{0pt}\mytexttt{\color{param} (REAL8 val)}} \\
\hline
\end{tabularx}
}

\par





Return whether a real value is a NaN (not a number) value.






\par
\begin{description}
\item [\colorbox{tagtype}{\color{white} \textbf{\textsf{PARAMETER}}}] \textbf{\underline{val}} ||| REAL8 --- The value to test.
\end{description}







\par
\begin{description}
\item [\colorbox{tagtype}{\color{white} \textbf{\textsf{RETURN}}}] \textbf{BOOLEAN} --- 
\end{description}




\rule{\linewidth}{0.5pt}
\subsection*{\textsf{\colorbox{headtoc}{\color{white} FUNCTION}
isFinite}}

\hypertarget{ecldoc:math.isfinite}{}
\hspace{0pt} \hyperlink{ecldoc:Math}{Math} \textbackslash 

{\renewcommand{\arraystretch}{1.5}
\begin{tabularx}{\textwidth}{|>{\raggedright\arraybackslash}l|X|}
\hline
\hspace{0pt}\mytexttt{\color{red} BOOLEAN} & \textbf{isFinite} \\
\hline
\multicolumn{2}{|>{\raggedright\arraybackslash}X|}{\hspace{0pt}\mytexttt{\color{param} (REAL8 val)}} \\
\hline
\end{tabularx}
}

\par





Return whether a real value is a valid value (neither infinite not NaN).






\par
\begin{description}
\item [\colorbox{tagtype}{\color{white} \textbf{\textsf{PARAMETER}}}] \textbf{\underline{val}} ||| REAL8 --- The value to test.
\end{description}







\par
\begin{description}
\item [\colorbox{tagtype}{\color{white} \textbf{\textsf{RETURN}}}] \textbf{BOOLEAN} --- 
\end{description}




\rule{\linewidth}{0.5pt}
\subsection*{\textsf{\colorbox{headtoc}{\color{white} FUNCTION}
FMod}}

\hypertarget{ecldoc:math.fmod}{}
\hspace{0pt} \hyperlink{ecldoc:Math}{Math} \textbackslash 

{\renewcommand{\arraystretch}{1.5}
\begin{tabularx}{\textwidth}{|>{\raggedright\arraybackslash}l|X|}
\hline
\hspace{0pt}\mytexttt{\color{red} REAL8} & \textbf{FMod} \\
\hline
\multicolumn{2}{|>{\raggedright\arraybackslash}X|}{\hspace{0pt}\mytexttt{\color{param} (REAL8 numer, REAL8 denom)}} \\
\hline
\end{tabularx}
}

\par





Returns the floating-point remainder of numer/denom (rounded towards zero). If denom is zero, the result depends on the -fdivideByZero flag: 'zero' or unset: return zero. 'nan': return a non-signalling NaN value 'fail': throw an exception






\par
\begin{description}
\item [\colorbox{tagtype}{\color{white} \textbf{\textsf{PARAMETER}}}] \textbf{\underline{denom}} ||| REAL8 --- The numerator.
\item [\colorbox{tagtype}{\color{white} \textbf{\textsf{PARAMETER}}}] \textbf{\underline{numer}} ||| REAL8 --- The numerator.
\end{description}







\par
\begin{description}
\item [\colorbox{tagtype}{\color{white} \textbf{\textsf{RETURN}}}] \textbf{REAL8} --- 
\end{description}




\rule{\linewidth}{0.5pt}
\subsection*{\textsf{\colorbox{headtoc}{\color{white} FUNCTION}
FMatch}}

\hypertarget{ecldoc:math.fmatch}{}
\hspace{0pt} \hyperlink{ecldoc:Math}{Math} \textbackslash 

{\renewcommand{\arraystretch}{1.5}
\begin{tabularx}{\textwidth}{|>{\raggedright\arraybackslash}l|X|}
\hline
\hspace{0pt}\mytexttt{\color{red} BOOLEAN} & \textbf{FMatch} \\
\hline
\multicolumn{2}{|>{\raggedright\arraybackslash}X|}{\hspace{0pt}\mytexttt{\color{param} (REAL8 a, REAL8 b, REAL8 epsilon=0.0)}} \\
\hline
\end{tabularx}
}

\par





Returns whether two floating point values are the same, within margin of error epsilon.






\par
\begin{description}
\item [\colorbox{tagtype}{\color{white} \textbf{\textsf{PARAMETER}}}] \textbf{\underline{b}} ||| REAL8 --- The second value.
\item [\colorbox{tagtype}{\color{white} \textbf{\textsf{PARAMETER}}}] \textbf{\underline{a}} ||| REAL8 --- The first value.
\item [\colorbox{tagtype}{\color{white} \textbf{\textsf{PARAMETER}}}] \textbf{\underline{epsilon}} ||| REAL8 --- The allowable margin of error.
\end{description}







\par
\begin{description}
\item [\colorbox{tagtype}{\color{white} \textbf{\textsf{RETURN}}}] \textbf{BOOLEAN} --- 
\end{description}




\rule{\linewidth}{0.5pt}



\chapter*{\color{headfile}
test
}
\hypertarget{ecldoc:toc:test}{}
\hyperlink{ecldoc:toc:root}{Go Up}


\section*{\underline{\textsf{DESCRIPTIONS}}}
\subsection*{\textsf{\colorbox{headtoc}{\color{white} MODULE}
test}}

\hypertarget{ecldoc:test}{}

{\renewcommand{\arraystretch}{1.5}
\begin{tabularx}{\textwidth}{|>{\raggedright\arraybackslash}l|X|}
\hline
\hspace{0pt}\mytexttt{\color{red} } & \textbf{test} \\
\hline
\end{tabularx}
}

\par





test module







\rule{\linewidth}{0.5pt}

\chapter*{\color{headfile}
types
}
\hypertarget{ecldoc:toc:types}{}
\hyperlink{ecldoc:toc:root}{Go Up}


\section*{\underline{\textsf{DESCRIPTIONS}}}
\subsection*{\textsf{\colorbox{headtoc}{\color{white} MODULE}
types}}

\hypertarget{ecldoc:types}{}

{\renewcommand{\arraystretch}{1.5}
\begin{tabularx}{\textwidth}{|>{\raggedright\arraybackslash}l|X|}
\hline
\hspace{0pt}\mytexttt{\color{red} } & \textbf{types} \\
\hline
\end{tabularx}
}

\par





No Documentation Found







\textbf{Children}
\begin{enumerate}
\item \hyperlink{ecldoc:types.v1}{v1}
: No Documentation Found
\item \hyperlink{ecldoc:types.mod_1}{mod\_1}
: No Documentation Found
\item \hyperlink{ecldoc:types.mod_1_1}{mod\_1\_1}
: No Documentation Found
\item \hyperlink{ecldoc:types.mod_2}{mod\_2}
: No Documentation Found
\item \hyperlink{ecldoc:ecldoc-mod_3}{mod\_3}
: No Documentation Found
\item \hyperlink{ecldoc:ecldoc-mod_4}{mod\_4}
: No Documentation Found
\item \hyperlink{ecldoc:types.mod_41}{mod\_41}
: No Documentation Found
\item \hyperlink{ecldoc:types.mod_5}{mod\_5}
: mod\_5
\item \hyperlink{ecldoc:types.mod_6}{mod\_6}
: No Documentation Found
\item \hyperlink{ecldoc:types.mod_7}{mod\_7}
: No Documentation Found
\item \hyperlink{ecldoc:types.mod_8}{mod\_8}
: No Documentation Found
\item \hyperlink{ecldoc:types.mod_9}{mod\_9}
: No Documentation Found
\item \hyperlink{ecldoc:types.mod_90}{mod\_90}
: No Documentation Found
\item \hyperlink{ecldoc:types.mod_10}{mod\_10}
: No Documentation Found
\item \hyperlink{ecldoc:types.mod_11}{mod\_11}
: No Documentation Found
\item \hyperlink{ecldoc:types.d1}{D1}
: No Documentation Found
\item \hyperlink{ecldoc:types.mod_12}{mod\_12}
: No Documentation Found
\item \hyperlink{ecldoc:types.mod_13}{mod\_13}
: No Documentation Found
\item \hyperlink{ecldoc:types.v2}{v2}
: No Documentation Found
\item \hyperlink{ecldoc:types.v1tov2}{v1tov2}
: No Documentation Found
\item \hyperlink{ecldoc:types.mod_14}{mod\_14}
: No Documentation Found
\item \hyperlink{ecldoc:types.mod_15}{mod\_15}
: No Documentation Found
\item \hyperlink{ecldoc:types.v4}{v4}
: No Documentation Found
\item \hyperlink{ecldoc:types.v5}{v5}
: No Documentation Found
\item \hyperlink{ecldoc:types.v5_1}{v5\_1}
: No Documentation Found
\item \hyperlink{ecldoc:types.mod_17}{mod\_17}
: No Documentation Found
\item \hyperlink{ecldoc:types.mod_18}{mod\_18}
: No Documentation Found
\item \hyperlink{ecldoc:types.mod_19}{mod\_19}
: No Documentation Found
\item \hyperlink{ecldoc:types.mod_20}{mod\_20}
: No Documentation Found
\item \hyperlink{ecldoc:types.mod_21}{mod\_21}
: No Documentation Found
\item \hyperlink{ecldoc:types.mod_22}{mod\_22}
: No Documentation Found
\item \hyperlink{ecldoc:types.mod_23}{mod\_23}
: No Documentation Found
\item \hyperlink{ecldoc:types.mod_24}{mod\_24}
: No Documentation Found
\item \hyperlink{ecldoc:types.mod_25}{mod\_25}
: No Documentation Found
\item \hyperlink{ecldoc:types.mod_26}{mod\_26}
: No Documentation Found
\item \hyperlink{ecldoc:types.mod_27}{mod\_27}
: No Documentation Found
\end{enumerate}

\rule{\linewidth}{0.5pt}

\subsection*{\textsf{\colorbox{headtoc}{\color{white} RECORD}
v1}}

\hypertarget{ecldoc:types.v1}{}
\hspace{0pt} \hyperlink{ecldoc:types}{types} \textbackslash 

{\renewcommand{\arraystretch}{1.5}
\begin{tabularx}{\textwidth}{|>{\raggedright\arraybackslash}l|X|}
\hline
\hspace{0pt}\mytexttt{\color{red} } & \textbf{v1} \\
\hline
\end{tabularx}
}

\par





No Documentation Found







\par
\begin{description}
\item [\colorbox{tagtype}{\color{white} \textbf{\textsf{FIELD}}}] \textbf{\underline{v}} ||| REAL8 --- No Doc
\item [\colorbox{tagtype}{\color{white} \textbf{\textsf{FIELD}}}] \textbf{\underline{u}} ||| REAL8 --- No Doc
\end{description}





\rule{\linewidth}{0.5pt}
\subsection*{\textsf{\colorbox{headtoc}{\color{white} ATTRIBUTE}
mod\_1}}

\hypertarget{ecldoc:types.mod_1}{}
\hspace{0pt} \hyperlink{ecldoc:types}{types} \textbackslash 

{\renewcommand{\arraystretch}{1.5}
\begin{tabularx}{\textwidth}{|>{\raggedright\arraybackslash}l|X|}
\hline
\hspace{0pt}\mytexttt{\color{red} DATASET(v1)} & \textbf{mod\_1} \\
\hline
\end{tabularx}
}

\par





No Documentation Found








\par
\begin{description}
\item [\colorbox{tagtype}{\color{white} \textbf{\textsf{RETURN}}}] \textbf{TABLE ( \{ REAL8 u , REAL8 v \} )} --- 
\end{description}




\rule{\linewidth}{0.5pt}
\subsection*{\textsf{\colorbox{headtoc}{\color{white} FUNCTION}
mod\_1\_1}}

\hypertarget{ecldoc:types.mod_1_1}{}
\hspace{0pt} \hyperlink{ecldoc:types}{types} \textbackslash 

{\renewcommand{\arraystretch}{1.5}
\begin{tabularx}{\textwidth}{|>{\raggedright\arraybackslash}l|X|}
\hline
\hspace{0pt}\mytexttt{\color{red} } & \textbf{mod\_1\_1} \\
\hline
\multicolumn{2}{|>{\raggedright\arraybackslash}X|}{\hspace{0pt}\mytexttt{\color{param} (TYPEOF(mod\_1) x)}} \\
\hline
\end{tabularx}
}

\par





No Documentation Found






\par
\begin{description}
\item [\colorbox{tagtype}{\color{white} \textbf{\textsf{PARAMETER}}}] \textbf{\underline{x}} ||| TABLE ( \{ REAL8 u , REAL8 v \} ) --- No Doc
\end{description}







\par
\begin{description}
\item [\colorbox{tagtype}{\color{white} \textbf{\textsf{RETURN}}}] \textbf{TABLE ( \{ REAL8 u , REAL8 v \} )} --- 
\end{description}




\rule{\linewidth}{0.5pt}
\subsection*{\textsf{\colorbox{headtoc}{\color{white} ATTRIBUTE}
mod\_2}}

\hypertarget{ecldoc:types.mod_2}{}
\hspace{0pt} \hyperlink{ecldoc:types}{types} \textbackslash 

{\renewcommand{\arraystretch}{1.5}
\begin{tabularx}{\textwidth}{|>{\raggedright\arraybackslash}l|X|}
\hline
\hspace{0pt}\mytexttt{\color{red} DATASET(\{STRING20 a\})} & \textbf{mod\_2} \\
\hline
\end{tabularx}
}

\par





No Documentation Found








\par
\begin{description}
\item [\colorbox{tagtype}{\color{white} \textbf{\textsf{RETURN}}}] \textbf{TABLE ( \{ STRING20 a \} )} --- 
\end{description}




\rule{\linewidth}{0.5pt}
\subsection*{\textsf{\colorbox{headtoc}{\color{white} ATTRIBUTE}
mod\_3}}

\hypertarget{ecldoc:ecldoc-mod_3}{}
\hspace{0pt} \hyperlink{ecldoc:types}{types} \textbackslash 

{\renewcommand{\arraystretch}{1.5}
\begin{tabularx}{\textwidth}{|>{\raggedright\arraybackslash}l|X|}
\hline
\hspace{0pt}\mytexttt{\color{red} } & \textbf{mod\_3} \\
\hline
\end{tabularx}
}

\par





No Documentation Found








\par
\begin{description}
\item [\colorbox{tagtype}{\color{white} \textbf{\textsf{RETURN}}}] \textbf{UNSIGNED8} --- 
\end{description}




\rule{\linewidth}{0.5pt}
\subsection*{\textsf{\colorbox{headtoc}{\color{white} ATTRIBUTE}
mod\_4}}

\hypertarget{ecldoc:ecldoc-mod_4}{}
\hspace{0pt} \hyperlink{ecldoc:types}{types} \textbackslash 

{\renewcommand{\arraystretch}{1.5}
\begin{tabularx}{\textwidth}{|>{\raggedright\arraybackslash}l|X|}
\hline
\hspace{0pt}\mytexttt{\color{red} } & \textbf{mod\_4} \\
\hline
\end{tabularx}
}

\par





No Documentation Found








\par
\begin{description}
\item [\colorbox{tagtype}{\color{white} \textbf{\textsf{RETURN}}}] \textbf{UNSIGNED8} --- 
\end{description}




\rule{\linewidth}{0.5pt}
\subsection*{\textsf{\colorbox{headtoc}{\color{white} MODULE}
mod\_41}}

\hypertarget{ecldoc:types.mod_41}{}
\hspace{0pt} \hyperlink{ecldoc:types}{types} \textbackslash 

{\renewcommand{\arraystretch}{1.5}
\begin{tabularx}{\textwidth}{|>{\raggedright\arraybackslash}l|X|}
\hline
\hspace{0pt}\mytexttt{\color{red} } & \textbf{mod\_41} \\
\hline
\end{tabularx}
}

\par





No Documentation Found







\textbf{Children}
\begin{enumerate}
\item \hyperlink{ecldoc:types.mod_41.v41}{v41}
: No Documentation Found
\end{enumerate}

\rule{\linewidth}{0.5pt}

\subsection*{\textsf{\colorbox{headtoc}{\color{white} ATTRIBUTE}
v41}}

\hypertarget{ecldoc:types.mod_41.v41}{}
\hspace{0pt} \hyperlink{ecldoc:types}{types} \textbackslash 
\hspace{0pt} \hyperlink{ecldoc:types.mod_41}{mod_41} \textbackslash 

{\renewcommand{\arraystretch}{1.5}
\begin{tabularx}{\textwidth}{|>{\raggedright\arraybackslash}l|X|}
\hline
\hspace{0pt}\mytexttt{\color{red} } & \textbf{v41} \\
\hline
\end{tabularx}
}

\par





No Documentation Found








\par
\begin{description}
\item [\colorbox{tagtype}{\color{white} \textbf{\textsf{RETURN}}}] \textbf{REAL8} --- 
\end{description}




\rule{\linewidth}{0.5pt}


\subsection*{\textsf{\colorbox{headtoc}{\color{white} MODULE}
mod\_5}}

\hypertarget{ecldoc:types.mod_5}{}
\hspace{0pt} \hyperlink{ecldoc:types}{types} \textbackslash 

{\renewcommand{\arraystretch}{1.5}
\begin{tabularx}{\textwidth}{|>{\raggedright\arraybackslash}l|X|}
\hline
\hspace{0pt}\mytexttt{\color{red} } & \textbf{mod\_5} \\
\hline
\multicolumn{2}{|>{\raggedright\arraybackslash}X|}{\hspace{0pt}\mytexttt{\color{param} (REAL8 x)}} \\
\hline
\end{tabularx}
}

\par





mod\_5






\par
\begin{description}
\item [\colorbox{tagtype}{\color{white} \textbf{\textsf{PARAMETER}}}] \textbf{\underline{x}} ||| REAL8 --- abcd
\end{description}







\par
\begin{description}
\item [\colorbox{tagtype}{\color{white} \textbf{\textsf{RETURN}}}] \textbf{} --- module
\end{description}




\textbf{Children}
\begin{enumerate}
\item \hyperlink{ecldoc:types.mod_5.v6}{v6}
: No Documentation Found
\end{enumerate}

\rule{\linewidth}{0.5pt}

\subsection*{\textsf{\colorbox{headtoc}{\color{white} ATTRIBUTE}
v6}}

\hypertarget{ecldoc:types.mod_5.v6}{}
\hspace{0pt} \hyperlink{ecldoc:types}{types} \textbackslash 
\hspace{0pt} \hyperlink{ecldoc:types.mod_5}{mod_5} \textbackslash 

{\renewcommand{\arraystretch}{1.5}
\begin{tabularx}{\textwidth}{|>{\raggedright\arraybackslash}l|X|}
\hline
\hspace{0pt}\mytexttt{\color{red} } & \textbf{v6} \\
\hline
\end{tabularx}
}

\par





No Documentation Found








\par
\begin{description}
\item [\colorbox{tagtype}{\color{white} \textbf{\textsf{RETURN}}}] \textbf{REAL8} --- 
\end{description}




\rule{\linewidth}{0.5pt}


\subsection*{\textsf{\colorbox{headtoc}{\color{white} MODULE}
mod\_6}}

\hypertarget{ecldoc:types.mod_6}{}
\hspace{0pt} \hyperlink{ecldoc:types}{types} \textbackslash 

{\renewcommand{\arraystretch}{1.5}
\begin{tabularx}{\textwidth}{|>{\raggedright\arraybackslash}l|X|}
\hline
\hspace{0pt}\mytexttt{\color{red} } & \textbf{mod\_6} \\
\hline
\multicolumn{2}{|>{\raggedright\arraybackslash}X|}{\hspace{0pt}\mytexttt{\color{param} (REAL8 x)}} \\
\hline
\end{tabularx}
}

\par





No Documentation Found






\par
\begin{description}
\item [\colorbox{tagtype}{\color{white} \textbf{\textsf{PARAMETER}}}] \textbf{\underline{x}} ||| REAL8 --- No Doc
\end{description}






\textbf{Children}
\begin{enumerate}
\item \hyperlink{ecldoc:types.mod_5.v6}{v6}
: No Documentation Found
\end{enumerate}

\rule{\linewidth}{0.5pt}

\subsection*{\textsf{\colorbox{headtoc}{\color{white} ATTRIBUTE}
v6}}

\hypertarget{ecldoc:types.mod_5.v6}{}
\hspace{0pt} \hyperlink{ecldoc:types}{types} \textbackslash 
\hspace{0pt} \hyperlink{ecldoc:types.mod_6}{mod_6} \textbackslash 

{\renewcommand{\arraystretch}{1.5}
\begin{tabularx}{\textwidth}{|>{\raggedright\arraybackslash}l|X|}
\hline
\hspace{0pt}\mytexttt{\color{red} } & \textbf{v6} \\
\hline
\end{tabularx}
}

\par





No Documentation Found








\par
\begin{description}
\item [\colorbox{tagtype}{\color{white} \textbf{\textsf{RETURN}}}] \textbf{REAL8} --- 
\end{description}




\rule{\linewidth}{0.5pt}


\subsection*{\textsf{\colorbox{headtoc}{\color{white} FUNCTION}
mod\_7}}

\hypertarget{ecldoc:types.mod_7}{}
\hspace{0pt} \hyperlink{ecldoc:types}{types} \textbackslash 

{\renewcommand{\arraystretch}{1.5}
\begin{tabularx}{\textwidth}{|>{\raggedright\arraybackslash}l|X|}
\hline
\hspace{0pt}\mytexttt{\color{red} } & \textbf{mod\_7} \\
\hline
\multicolumn{2}{|>{\raggedright\arraybackslash}X|}{\hspace{0pt}\mytexttt{\color{param} (mod\_5 y, REAL8 z)}} \\
\hline
\end{tabularx}
}

\par





No Documentation Found






\par
\begin{description}
\item [\colorbox{tagtype}{\color{white} \textbf{\textsf{PARAMETER}}}] \textbf{\underline{y}} ||| FUNCTION [ REAL8 ] ( MODULE ( mod\_5 ) ) --- No Doc
\item [\colorbox{tagtype}{\color{white} \textbf{\textsf{PARAMETER}}}] \textbf{\underline{z}} ||| REAL8 --- No Doc
\end{description}







\par
\begin{description}
\item [\colorbox{tagtype}{\color{white} \textbf{\textsf{RETURN}}}] \textbf{REAL8} --- 
\end{description}




\rule{\linewidth}{0.5pt}
\subsection*{\textsf{\colorbox{headtoc}{\color{white} FUNCTION}
mod\_8}}

\hypertarget{ecldoc:types.mod_8}{}
\hspace{0pt} \hyperlink{ecldoc:types}{types} \textbackslash 

{\renewcommand{\arraystretch}{1.5}
\begin{tabularx}{\textwidth}{|>{\raggedright\arraybackslash}l|X|}
\hline
\hspace{0pt}\mytexttt{\color{red} } & \textbf{mod\_8} \\
\hline
\multicolumn{2}{|>{\raggedright\arraybackslash}X|}{\hspace{0pt}\mytexttt{\color{param} (mod\_3 a1)}} \\
\hline
\end{tabularx}
}

\par





No Documentation Found






\par
\begin{description}
\item [\colorbox{tagtype}{\color{white} \textbf{\textsf{PARAMETER}}}] \textbf{\underline{a1}} ||| UNSIGNED8 --- No Doc
\end{description}







\par
\begin{description}
\item [\colorbox{tagtype}{\color{white} \textbf{\textsf{RETURN}}}] \textbf{INTEGER8} --- 
\end{description}




\rule{\linewidth}{0.5pt}
\subsection*{\textsf{\colorbox{headtoc}{\color{white} FUNCTION}
mod\_9}}

\hypertarget{ecldoc:types.mod_9}{}
\hspace{0pt} \hyperlink{ecldoc:types}{types} \textbackslash 

{\renewcommand{\arraystretch}{1.5}
\begin{tabularx}{\textwidth}{|>{\raggedright\arraybackslash}l|X|}
\hline
\hspace{0pt}\mytexttt{\color{red} } & \textbf{mod\_9} \\
\hline
\multicolumn{2}{|>{\raggedright\arraybackslash}X|}{\hspace{0pt}\mytexttt{\color{param} (mod\_7 x)}} \\
\hline
\end{tabularx}
}

\par





No Documentation Found






\par
\begin{description}
\item [\colorbox{tagtype}{\color{white} \textbf{\textsf{PARAMETER}}}] \textbf{\underline{x}} ||| FUNCTION [ FUNCTION [ REAL8 ] ( MODULE ( mod\_5 ) ) , REAL8 ] ( REAL8 ) --- No Doc
\end{description}







\par
\begin{description}
\item [\colorbox{tagtype}{\color{white} \textbf{\textsf{RETURN}}}] \textbf{REAL8} --- 
\end{description}




\rule{\linewidth}{0.5pt}
\subsection*{\textsf{\colorbox{headtoc}{\color{white} FUNCTION}
mod\_90}}

\hypertarget{ecldoc:types.mod_90}{}
\hspace{0pt} \hyperlink{ecldoc:types}{types} \textbackslash 

{\renewcommand{\arraystretch}{1.5}
\begin{tabularx}{\textwidth}{|>{\raggedright\arraybackslash}l|X|}
\hline
\hspace{0pt}\mytexttt{\color{red} } & \textbf{mod\_90} \\
\hline
\multicolumn{2}{|>{\raggedright\arraybackslash}X|}{\hspace{0pt}\mytexttt{\color{param} (REAL8 z)}} \\
\hline
\end{tabularx}
}

\par





No Documentation Found






\par
\begin{description}
\item [\colorbox{tagtype}{\color{white} \textbf{\textsf{PARAMETER}}}] \textbf{\underline{z}} ||| REAL8 --- No Doc
\end{description}







\par
\begin{description}
\item [\colorbox{tagtype}{\color{white} \textbf{\textsf{RETURN}}}] \textbf{REAL8} --- 
\end{description}




\rule{\linewidth}{0.5pt}
\subsection*{\textsf{\colorbox{headtoc}{\color{white} RECORD}
mod\_10}}

\hypertarget{ecldoc:types.mod_10}{}
\hspace{0pt} \hyperlink{ecldoc:types}{types} \textbackslash 

{\renewcommand{\arraystretch}{1.5}
\begin{tabularx}{\textwidth}{|>{\raggedright\arraybackslash}l|X|}
\hline
\hspace{0pt}\mytexttt{\color{red} } & \textbf{mod\_10} \\
\hline
\end{tabularx}
}

\par





No Documentation Found







\par
\begin{description}
\item [\colorbox{tagtype}{\color{white} \textbf{\textsf{FIELD}}}] \textbf{\underline{u}} ||| REAL8 --- No Doc
\end{description}





\rule{\linewidth}{0.5pt}
\subsection*{\textsf{\colorbox{headtoc}{\color{white} FUNCTION}
mod\_11}}

\hypertarget{ecldoc:types.mod_11}{}
\hspace{0pt} \hyperlink{ecldoc:types}{types} \textbackslash 

{\renewcommand{\arraystretch}{1.5}
\begin{tabularx}{\textwidth}{|>{\raggedright\arraybackslash}l|X|}
\hline
\hspace{0pt}\mytexttt{\color{red} } & \textbf{mod\_11} \\
\hline
\multicolumn{2}{|>{\raggedright\arraybackslash}X|}{\hspace{0pt}\mytexttt{\color{param} (DATASET(mod\_10) y)}} \\
\hline
\end{tabularx}
}

\par





No Documentation Found






\par
\begin{description}
\item [\colorbox{tagtype}{\color{white} \textbf{\textsf{PARAMETER}}}] \textbf{\underline{y}} ||| TABLE ( mod\_10 ) --- No Doc
\end{description}







\par
\begin{description}
\item [\colorbox{tagtype}{\color{white} \textbf{\textsf{RETURN}}}] \textbf{TABLE ( mod\_10 )} --- 
\end{description}




\rule{\linewidth}{0.5pt}
\subsection*{\textsf{\colorbox{headtoc}{\color{white} ATTRIBUTE}
D1}}

\hypertarget{ecldoc:types.d1}{}
\hspace{0pt} \hyperlink{ecldoc:types}{types} \textbackslash 

{\renewcommand{\arraystretch}{1.5}
\begin{tabularx}{\textwidth}{|>{\raggedright\arraybackslash}l|X|}
\hline
\hspace{0pt}\mytexttt{\color{red} } & \textbf{D1} \\
\hline
\end{tabularx}
}

\par





No Documentation Found








\par
\begin{description}
\item [\colorbox{tagtype}{\color{white} \textbf{\textsf{RETURN}}}] \textbf{TABLE ( \{ UNSIGNED4 F1 , REAL8 f2 , REAL8 f3 \} )} --- 
\end{description}




\rule{\linewidth}{0.5pt}
\subsection*{\textsf{\colorbox{headtoc}{\color{white} RECORD}
mod\_12}}

\hypertarget{ecldoc:types.mod_12}{}
\hspace{0pt} \hyperlink{ecldoc:types}{types} \textbackslash 

{\renewcommand{\arraystretch}{1.5}
\begin{tabularx}{\textwidth}{|>{\raggedright\arraybackslash}l|X|}
\hline
\hspace{0pt}\mytexttt{\color{red} } & \textbf{mod\_12} \\
\hline
\end{tabularx}
}

\par





No Documentation Found







\par
\begin{description}
\item [\colorbox{tagtype}{\color{white} \textbf{\textsf{FIELD}}}] \textbf{\underline{gcount}} ||| INTEGER8 --- No Doc
\item [\colorbox{tagtype}{\color{white} \textbf{\textsf{FIELD}}}] \textbf{\underline{f1}} ||| UNSIGNED4 --- No Doc
\end{description}





\rule{\linewidth}{0.5pt}
\subsection*{\textsf{\colorbox{headtoc}{\color{white} ATTRIBUTE}
mod\_13}}

\hypertarget{ecldoc:types.mod_13}{}
\hspace{0pt} \hyperlink{ecldoc:types}{types} \textbackslash 

{\renewcommand{\arraystretch}{1.5}
\begin{tabularx}{\textwidth}{|>{\raggedright\arraybackslash}l|X|}
\hline
\hspace{0pt}\mytexttt{\color{red} } & \textbf{mod\_13} \\
\hline
\end{tabularx}
}

\par





No Documentation Found








\par
\begin{description}
\item [\colorbox{tagtype}{\color{white} \textbf{\textsf{RETURN}}}] \textbf{TABLE ( \{ UNSIGNED4 F1 , INTEGER8 gcount \} )} --- 
\end{description}




\rule{\linewidth}{0.5pt}
\subsection*{\textsf{\colorbox{headtoc}{\color{white} RECORD}
v2}}

\hypertarget{ecldoc:types.v2}{}
\hspace{0pt} \hyperlink{ecldoc:types}{types} \textbackslash 

{\renewcommand{\arraystretch}{1.5}
\begin{tabularx}{\textwidth}{|>{\raggedright\arraybackslash}l|X|}
\hline
\hspace{0pt}\mytexttt{\color{red} } & \textbf{v2} \\
\hline
\end{tabularx}
}

\par





No Documentation Found







\par
\begin{description}
\item [\colorbox{tagtype}{\color{white} \textbf{\textsf{FIELD}}}] \textbf{\underline{w3}} ||| UNSIGNED4 --- No Doc
\item [\colorbox{tagtype}{\color{white} \textbf{\textsf{FIELD}}}] \textbf{\underline{w2}} ||| UNSIGNED4 --- No Doc
\item [\colorbox{tagtype}{\color{white} \textbf{\textsf{FIELD}}}] \textbf{\underline{w1}} ||| UNSIGNED4 --- No Doc
\end{description}





\rule{\linewidth}{0.5pt}
\subsection*{\textsf{\colorbox{headtoc}{\color{white} TRANSFORM}
v1tov2}}

\hypertarget{ecldoc:types.v1tov2}{}
\hspace{0pt} \hyperlink{ecldoc:types}{types} \textbackslash 

{\renewcommand{\arraystretch}{1.5}
\begin{tabularx}{\textwidth}{|>{\raggedright\arraybackslash}l|X|}
\hline
\hspace{0pt}\mytexttt{\color{red} v2} & \textbf{v1tov2} \\
\hline
\multicolumn{2}{|>{\raggedright\arraybackslash}X|}{\hspace{0pt}\mytexttt{\color{param} (v1 x)}} \\
\hline
\end{tabularx}
}

\par





No Documentation Found






\par
\begin{description}
\item [\colorbox{tagtype}{\color{white} \textbf{\textsf{PARAMETER}}}] \textbf{\underline{x}} ||| ROW ( v1 ) --- No Doc
\end{description}







\par
\begin{description}
\item [\colorbox{tagtype}{\color{white} \textbf{\textsf{RETURN}}}] \textbf{v2} --- 
\end{description}




\rule{\linewidth}{0.5pt}
\subsection*{\textsf{\colorbox{headtoc}{\color{white} ATTRIBUTE}
mod\_14}}

\hypertarget{ecldoc:types.mod_14}{}
\hspace{0pt} \hyperlink{ecldoc:types}{types} \textbackslash 

{\renewcommand{\arraystretch}{1.5}
\begin{tabularx}{\textwidth}{|>{\raggedright\arraybackslash}l|X|}
\hline
\hspace{0pt}\mytexttt{\color{red} TYPEOF(mod\_1)} & \textbf{mod\_14} \\
\hline
\end{tabularx}
}

\par





No Documentation Found








\par
\begin{description}
\item [\colorbox{tagtype}{\color{white} \textbf{\textsf{RETURN}}}] \textbf{TABLE ( v1 )} --- 
\end{description}




\rule{\linewidth}{0.5pt}
\subsection*{\textsf{\colorbox{headtoc}{\color{white} ATTRIBUTE}
mod\_15}}

\hypertarget{ecldoc:types.mod_15}{}
\hspace{0pt} \hyperlink{ecldoc:types}{types} \textbackslash 

{\renewcommand{\arraystretch}{1.5}
\begin{tabularx}{\textwidth}{|>{\raggedright\arraybackslash}l|X|}
\hline
\hspace{0pt}\mytexttt{\color{red} DATASET(v3)} & \textbf{mod\_15} \\
\hline
\end{tabularx}
}

\par





No Documentation Found








\par
\begin{description}
\item [\colorbox{tagtype}{\color{white} \textbf{\textsf{RETURN}}}] \textbf{TABLE ( v3 )} --- 
\end{description}




\rule{\linewidth}{0.5pt}
\subsection*{\textsf{\colorbox{headtoc}{\color{white} RECORD}
v4}}

\hypertarget{ecldoc:types.v4}{}
\hspace{0pt} \hyperlink{ecldoc:types}{types} \textbackslash 

{\renewcommand{\arraystretch}{1.5}
\begin{tabularx}{\textwidth}{|>{\raggedright\arraybackslash}l|X|}
\hline
\hspace{0pt}\mytexttt{\color{red} } & \textbf{v4} \\
\hline
\end{tabularx}
}

\par





No Documentation Found







\par
\begin{description}
\item [\colorbox{tagtype}{\color{white} \textbf{\textsf{FIELD}}}] \textbf{\underline{w3}} ||| UNSIGNED4 --- No Doc
\item [\colorbox{tagtype}{\color{white} \textbf{\textsf{FIELD}}}] \textbf{\underline{w2}} ||| UNSIGNED4 --- No Doc
\item [\colorbox{tagtype}{\color{white} \textbf{\textsf{FIELD}}}] \textbf{\underline{w4}} ||| REAL8 --- No Doc
\item [\colorbox{tagtype}{\color{white} \textbf{\textsf{FIELD}}}] \textbf{\underline{w1}} ||| UNSIGNED4 --- No Doc
\end{description}





\rule{\linewidth}{0.5pt}
\subsection*{\textsf{\colorbox{headtoc}{\color{white} RECORD}
v5}}

\hypertarget{ecldoc:types.v5}{}
\hspace{0pt} \hyperlink{ecldoc:types}{types} \textbackslash 

{\renewcommand{\arraystretch}{1.5}
\begin{tabularx}{\textwidth}{|>{\raggedright\arraybackslash}l|X|}
\hline
\hspace{0pt}\mytexttt{\color{red} } & \textbf{v5} \\
\hline
\end{tabularx}
}

\par





No Documentation Found







\par
\begin{description}
\item [\colorbox{tagtype}{\color{white} \textbf{\textsf{FIELD}}}] \textbf{\underline{w3}} ||| UNSIGNED4 --- No Doc
\item [\colorbox{tagtype}{\color{white} \textbf{\textsf{FIELD}}}] \textbf{\underline{v}} ||| REAL8 --- No Doc
\item [\colorbox{tagtype}{\color{white} \textbf{\textsf{FIELD}}}] \textbf{\underline{u}} ||| REAL8 --- No Doc
\item [\colorbox{tagtype}{\color{white} \textbf{\textsf{FIELD}}}] \textbf{\underline{w5}} ||| TABLE ( v2 ) --- No Doc
\item [\colorbox{tagtype}{\color{white} \textbf{\textsf{FIELD}}}] \textbf{\underline{w2}} ||| UNSIGNED4 --- No Doc
\item [\colorbox{tagtype}{\color{white} \textbf{\textsf{FIELD}}}] \textbf{\underline{w4}} ||| REAL8 --- No Doc
\item [\colorbox{tagtype}{\color{white} \textbf{\textsf{FIELD}}}] \textbf{\underline{w1}} ||| UNSIGNED4 --- No Doc
\end{description}





\rule{\linewidth}{0.5pt}
\subsection*{\textsf{\colorbox{headtoc}{\color{white} FUNCTION}
v5\_1}}

\hypertarget{ecldoc:types.v5_1}{}
\hspace{0pt} \hyperlink{ecldoc:types}{types} \textbackslash 

{\renewcommand{\arraystretch}{1.5}
\begin{tabularx}{\textwidth}{|>{\raggedright\arraybackslash}l|X|}
\hline
\hspace{0pt}\mytexttt{\color{red} } & \textbf{v5\_1} \\
\hline
\multicolumn{2}{|>{\raggedright\arraybackslash}X|}{\hspace{0pt}\mytexttt{\color{param} (DATASET(\{v5, real8 y\}) x)}} \\
\hline
\end{tabularx}
}

\par





No Documentation Found






\par
\begin{description}
\item [\colorbox{tagtype}{\color{white} \textbf{\textsf{PARAMETER}}}] \textbf{\underline{x}} ||| TABLE ( \{ REAL8 u , REAL8 v , UNSIGNED4 w1 , UNSIGNED4 w2 , UNSIGNED4 w3 , REAL8 w4 , TABLE ( v2 ) w5 , REAL8 y \} ) --- No Doc
\end{description}







\par
\begin{description}
\item [\colorbox{tagtype}{\color{white} \textbf{\textsf{RETURN}}}] \textbf{TABLE ( \{ REAL8 u , REAL8 v , UNSIGNED4 w1 , UNSIGNED4 w2 , UNSIGNED4 w3 , REAL8 w4 , TABLE ( v2 ) w5 , REAL8 y \} )} --- 
\end{description}




\rule{\linewidth}{0.5pt}
\subsection*{\textsf{\colorbox{headtoc}{\color{white} TRANSFORM}
mod\_17}}

\hypertarget{ecldoc:types.mod_17}{}
\hspace{0pt} \hyperlink{ecldoc:types}{types} \textbackslash 

{\renewcommand{\arraystretch}{1.5}
\begin{tabularx}{\textwidth}{|>{\raggedright\arraybackslash}l|X|}
\hline
\hspace{0pt}\mytexttt{\color{red} \{ REAL8 a \}} & \textbf{mod\_17} \\
\hline
\multicolumn{2}{|>{\raggedright\arraybackslash}X|}{\hspace{0pt}\mytexttt{\color{param} (v1 x)}} \\
\hline
\end{tabularx}
}

\par





No Documentation Found






\par
\begin{description}
\item [\colorbox{tagtype}{\color{white} \textbf{\textsf{PARAMETER}}}] \textbf{\underline{x}} ||| ROW ( v1 ) --- No Doc
\end{description}







\par
\begin{description}
\item [\colorbox{tagtype}{\color{white} \textbf{\textsf{RETURN}}}] \textbf{\{ REAL8 a \}} --- 
\end{description}




\rule{\linewidth}{0.5pt}
\subsection*{\textsf{\colorbox{headtoc}{\color{white} FUNCTION}
mod\_18}}

\hypertarget{ecldoc:types.mod_18}{}
\hspace{0pt} \hyperlink{ecldoc:types}{types} \textbackslash 

{\renewcommand{\arraystretch}{1.5}
\begin{tabularx}{\textwidth}{|>{\raggedright\arraybackslash}l|X|}
\hline
\hspace{0pt}\mytexttt{\color{red} } & \textbf{mod\_18} \\
\hline
\multicolumn{2}{|>{\raggedright\arraybackslash}X|}{\hspace{0pt}\mytexttt{\color{param} (REAL8 x(REAL8 z), REAL8 y)}} \\
\hline
\end{tabularx}
}

\par





No Documentation Found






\par
\begin{description}
\item [\colorbox{tagtype}{\color{white} \textbf{\textsf{PARAMETER}}}] \textbf{\underline{y}} ||| REAL8 --- No Doc
\item [\colorbox{tagtype}{\color{white} \textbf{\textsf{PARAMETER}}}] \textbf{\underline{x}} ||| FUNCTION [ REAL8 ] ( REAL8 ) --- No Doc
\end{description}







\par
\begin{description}
\item [\colorbox{tagtype}{\color{white} \textbf{\textsf{RETURN}}}] \textbf{REAL8} --- 
\end{description}




\rule{\linewidth}{0.5pt}
\subsection*{\textsf{\colorbox{headtoc}{\color{white} FUNCTION}
mod\_19}}

\hypertarget{ecldoc:types.mod_19}{}
\hspace{0pt} \hyperlink{ecldoc:types}{types} \textbackslash 

{\renewcommand{\arraystretch}{1.5}
\begin{tabularx}{\textwidth}{|>{\raggedright\arraybackslash}l|X|}
\hline
\hspace{0pt}\mytexttt{\color{red} } & \textbf{mod\_19} \\
\hline
\multicolumn{2}{|>{\raggedright\arraybackslash}X|}{\hspace{0pt}\mytexttt{\color{param} (REAL8 w)}} \\
\hline
\end{tabularx}
}

\par





No Documentation Found






\par
\begin{description}
\item [\colorbox{tagtype}{\color{white} \textbf{\textsf{PARAMETER}}}] \textbf{\underline{w}} ||| REAL8 --- No Doc
\end{description}







\par
\begin{description}
\item [\colorbox{tagtype}{\color{white} \textbf{\textsf{RETURN}}}] \textbf{REAL8} --- 
\end{description}




\rule{\linewidth}{0.5pt}
\subsection*{\textsf{\colorbox{headtoc}{\color{white} FUNCTION}
mod\_20}}

\hypertarget{ecldoc:types.mod_20}{}
\hspace{0pt} \hyperlink{ecldoc:types}{types} \textbackslash 

{\renewcommand{\arraystretch}{1.5}
\begin{tabularx}{\textwidth}{|>{\raggedright\arraybackslash}l|X|}
\hline
\hspace{0pt}\mytexttt{\color{red} } & \textbf{mod\_20} \\
\hline
\multicolumn{2}{|>{\raggedright\arraybackslash}X|}{\hspace{0pt}\mytexttt{\color{param} (mod\_19 x)}} \\
\hline
\end{tabularx}
}

\par





No Documentation Found






\par
\begin{description}
\item [\colorbox{tagtype}{\color{white} \textbf{\textsf{PARAMETER}}}] \textbf{\underline{x}} ||| FUNCTION [ REAL8 ] ( REAL8 ) --- No Doc
\end{description}







\par
\begin{description}
\item [\colorbox{tagtype}{\color{white} \textbf{\textsf{RETURN}}}] \textbf{REAL8} --- 
\end{description}




\rule{\linewidth}{0.5pt}
\subsection*{\textsf{\colorbox{headtoc}{\color{white} ATTRIBUTE}
mod\_21}}

\hypertarget{ecldoc:types.mod_21}{}
\hspace{0pt} \hyperlink{ecldoc:types}{types} \textbackslash 

{\renewcommand{\arraystretch}{1.5}
\begin{tabularx}{\textwidth}{|>{\raggedright\arraybackslash}l|X|}
\hline
\hspace{0pt}\mytexttt{\color{red} } & \textbf{mod\_21} \\
\hline
\end{tabularx}
}

\par





No Documentation Found








\par
\begin{description}
\item [\colorbox{tagtype}{\color{white} \textbf{\textsf{RETURN}}}] \textbf{REAL8} --- 
\end{description}




\rule{\linewidth}{0.5pt}
\subsection*{\textsf{\colorbox{headtoc}{\color{white} FUNCTION}
mod\_22}}

\hypertarget{ecldoc:types.mod_22}{}
\hspace{0pt} \hyperlink{ecldoc:types}{types} \textbackslash 

{\renewcommand{\arraystretch}{1.5}
\begin{tabularx}{\textwidth}{|>{\raggedright\arraybackslash}l|X|}
\hline
\hspace{0pt}\mytexttt{\color{red} } & \textbf{mod\_22} \\
\hline
\multicolumn{2}{|>{\raggedright\arraybackslash}X|}{\hspace{0pt}\mytexttt{\color{param} (REAL8 w)}} \\
\hline
\end{tabularx}
}

\par





No Documentation Found






\par
\begin{description}
\item [\colorbox{tagtype}{\color{white} \textbf{\textsf{PARAMETER}}}] \textbf{\underline{w}} ||| REAL8 --- No Doc
\end{description}







\par
\begin{description}
\item [\colorbox{tagtype}{\color{white} \textbf{\textsf{RETURN}}}] \textbf{REAL8} --- 
\end{description}




\rule{\linewidth}{0.5pt}
\subsection*{\textsf{\colorbox{headtoc}{\color{white} ATTRIBUTE}
mod\_23}}

\hypertarget{ecldoc:types.mod_23}{}
\hspace{0pt} \hyperlink{ecldoc:types}{types} \textbackslash 

{\renewcommand{\arraystretch}{1.5}
\begin{tabularx}{\textwidth}{|>{\raggedright\arraybackslash}l|X|}
\hline
\hspace{0pt}\mytexttt{\color{red} } & \textbf{mod\_23} \\
\hline
\end{tabularx}
}

\par





No Documentation Found








\par
\begin{description}
\item [\colorbox{tagtype}{\color{white} \textbf{\textsf{RETURN}}}] \textbf{REAL8} --- 
\end{description}




\rule{\linewidth}{0.5pt}
\subsection*{\textsf{\colorbox{headtoc}{\color{white} FUNCTION}
mod\_24}}

\hypertarget{ecldoc:types.mod_24}{}
\hspace{0pt} \hyperlink{ecldoc:types}{types} \textbackslash 

{\renewcommand{\arraystretch}{1.5}
\begin{tabularx}{\textwidth}{|>{\raggedright\arraybackslash}l|X|}
\hline
\hspace{0pt}\mytexttt{\color{red} } & \textbf{mod\_24} \\
\hline
\multicolumn{2}{|>{\raggedright\arraybackslash}X|}{\hspace{0pt}\mytexttt{\color{param} (REAL8 y(REAL8 z(REAL8 u)), REAL8 x(REAL8 y))}} \\
\hline
\end{tabularx}
}

\par





No Documentation Found






\par
\begin{description}
\item [\colorbox{tagtype}{\color{white} \textbf{\textsf{PARAMETER}}}] \textbf{\underline{y}} ||| FUNCTION [ FUNCTION [ REAL8 ] ( REAL8 ) ] ( REAL8 ) --- No Doc
\item [\colorbox{tagtype}{\color{white} \textbf{\textsf{PARAMETER}}}] \textbf{\underline{x}} ||| FUNCTION [ REAL8 ] ( REAL8 ) --- No Doc
\end{description}







\par
\begin{description}
\item [\colorbox{tagtype}{\color{white} \textbf{\textsf{RETURN}}}] \textbf{REAL8} --- 
\end{description}




\rule{\linewidth}{0.5pt}
\subsection*{\textsf{\colorbox{headtoc}{\color{white} FUNCTION}
mod\_25}}

\hypertarget{ecldoc:types.mod_25}{}
\hspace{0pt} \hyperlink{ecldoc:types}{types} \textbackslash 

{\renewcommand{\arraystretch}{1.5}
\begin{tabularx}{\textwidth}{|>{\raggedright\arraybackslash}l|X|}
\hline
\hspace{0pt}\mytexttt{\color{red} REAL8} & \textbf{mod\_25} \\
\hline
\multicolumn{2}{|>{\raggedright\arraybackslash}X|}{\hspace{0pt}\mytexttt{\color{param} (REAL8 x(REAL8 y))}} \\
\hline
\end{tabularx}
}

\par





No Documentation Found






\par
\begin{description}
\item [\colorbox{tagtype}{\color{white} \textbf{\textsf{PARAMETER}}}] \textbf{\underline{x}} ||| FUNCTION [ REAL8 ] ( REAL8 ) --- No Doc
\end{description}







\par
\begin{description}
\item [\colorbox{tagtype}{\color{white} \textbf{\textsf{RETURN}}}] \textbf{REAL8} --- 
\end{description}




\rule{\linewidth}{0.5pt}
\subsection*{\textsf{\colorbox{headtoc}{\color{white} ATTRIBUTE}
mod\_26}}

\hypertarget{ecldoc:types.mod_26}{}
\hspace{0pt} \hyperlink{ecldoc:types}{types} \textbackslash 

{\renewcommand{\arraystretch}{1.5}
\begin{tabularx}{\textwidth}{|>{\raggedright\arraybackslash}l|X|}
\hline
\hspace{0pt}\mytexttt{\color{red} } & \textbf{mod\_26} \\
\hline
\end{tabularx}
}

\par





No Documentation Found








\par
\begin{description}
\item [\colorbox{tagtype}{\color{white} \textbf{\textsf{RETURN}}}] \textbf{REAL8} --- 
\end{description}




\rule{\linewidth}{0.5pt}
\subsection*{\textsf{\colorbox{headtoc}{\color{white} ATTRIBUTE}
mod\_27}}

\hypertarget{ecldoc:types.mod_27}{}
\hspace{0pt} \hyperlink{ecldoc:types}{types} \textbackslash 

{\renewcommand{\arraystretch}{1.5}
\begin{tabularx}{\textwidth}{|>{\raggedright\arraybackslash}l|X|}
\hline
\hspace{0pt}\mytexttt{\color{red} mod\_1} & \textbf{mod\_27} \\
\hline
\end{tabularx}
}

\par





No Documentation Found








\par
\begin{description}
\item [\colorbox{tagtype}{\color{white} \textbf{\textsf{RETURN}}}] \textbf{TABLE ( \{ REAL8 u , REAL8 v \} )} --- 
\end{description}




\rule{\linewidth}{0.5pt}



