\chapter*{\color{headfile}
Math
}
\hypertarget{ecldoc:toc:Math}{}
\hyperlink{ecldoc:toc:root}{Go Up}


\section*{\underline{\textsf{DESCRIPTIONS}}}
\subsection*{\textsf{\colorbox{headtoc}{\color{white} MODULE}
Math}}

\hypertarget{ecldoc:Math}{}

{\renewcommand{\arraystretch}{1.5}
\begin{tabularx}{\textwidth}{|>{\raggedright\arraybackslash}l|X|}
\hline
\hspace{0pt}\mytexttt{\color{red} } & \textbf{Math} \\
\hline
\end{tabularx}
}

\par


\textbf{Children}
\begin{enumerate}
\item \hyperlink{ecldoc:math.infinity}{Infinity}
: Return a real ''infinity'' value
\item \hyperlink{ecldoc:math.nan}{NaN}
: Return a non-signalling NaN (Not a Number)value
\item \hyperlink{ecldoc:math.isinfinite}{isInfinite}
: Return whether a real value is infinite (positive or negative)
\item \hyperlink{ecldoc:math.isnan}{isNaN}
: Return whether a real value is a NaN (not a number) value
\item \hyperlink{ecldoc:math.isfinite}{isFinite}
: Return whether a real value is a valid value (neither infinite not NaN)
\item \hyperlink{ecldoc:math.fmod}{FMod}
: Returns the floating-point remainder of numer/denom (rounded towards zero)
\item \hyperlink{ecldoc:math.fmatch}{FMatch}
: Returns whether two floating point values are the same, within margin of error epsilon
\end{enumerate}

\rule{\linewidth}{0.5pt}

\subsection*{\textsf{\colorbox{headtoc}{\color{white} ATTRIBUTE}
Infinity}}

\hypertarget{ecldoc:math.infinity}{}
\hspace{0pt} \hyperlink{ecldoc:Math}{Math} \textbackslash 

{\renewcommand{\arraystretch}{1.5}
\begin{tabularx}{\textwidth}{|>{\raggedright\arraybackslash}l|X|}
\hline
\hspace{0pt}\mytexttt{\color{red} REAL8} & \textbf{Infinity} \\
\hline
\end{tabularx}
}

\par
Return a real ''infinity'' value.


\rule{\linewidth}{0.5pt}
\subsection*{\textsf{\colorbox{headtoc}{\color{white} ATTRIBUTE}
NaN}}

\hypertarget{ecldoc:math.nan}{}
\hspace{0pt} \hyperlink{ecldoc:Math}{Math} \textbackslash 

{\renewcommand{\arraystretch}{1.5}
\begin{tabularx}{\textwidth}{|>{\raggedright\arraybackslash}l|X|}
\hline
\hspace{0pt}\mytexttt{\color{red} REAL8} & \textbf{NaN} \\
\hline
\end{tabularx}
}

\par
Return a non-signalling NaN (Not a Number)value.


\rule{\linewidth}{0.5pt}
\subsection*{\textsf{\colorbox{headtoc}{\color{white} FUNCTION}
isInfinite}}

\hypertarget{ecldoc:math.isinfinite}{}
\hspace{0pt} \hyperlink{ecldoc:Math}{Math} \textbackslash 

{\renewcommand{\arraystretch}{1.5}
\begin{tabularx}{\textwidth}{|>{\raggedright\arraybackslash}l|X|}
\hline
\hspace{0pt}\mytexttt{\color{red} BOOLEAN} & \textbf{isInfinite} \\
\hline
\multicolumn{2}{|>{\raggedright\arraybackslash}X|}{\hspace{0pt}\mytexttt{\color{param} (REAL8 val)}} \\
\hline
\end{tabularx}
}

\par
Return whether a real value is infinite (positive or negative).

\par
\begin{description}
\item [\colorbox{tagtype}{\color{white} \textbf{\textsf{PARAMETER}}}] \textbf{\underline{val}} The value to test.
\end{description}

\rule{\linewidth}{0.5pt}
\subsection*{\textsf{\colorbox{headtoc}{\color{white} FUNCTION}
isNaN}}

\hypertarget{ecldoc:math.isnan}{}
\hspace{0pt} \hyperlink{ecldoc:Math}{Math} \textbackslash 

{\renewcommand{\arraystretch}{1.5}
\begin{tabularx}{\textwidth}{|>{\raggedright\arraybackslash}l|X|}
\hline
\hspace{0pt}\mytexttt{\color{red} BOOLEAN} & \textbf{isNaN} \\
\hline
\multicolumn{2}{|>{\raggedright\arraybackslash}X|}{\hspace{0pt}\mytexttt{\color{param} (REAL8 val)}} \\
\hline
\end{tabularx}
}

\par
Return whether a real value is a NaN (not a number) value.

\par
\begin{description}
\item [\colorbox{tagtype}{\color{white} \textbf{\textsf{PARAMETER}}}] \textbf{\underline{val}} The value to test.
\end{description}

\rule{\linewidth}{0.5pt}
\subsection*{\textsf{\colorbox{headtoc}{\color{white} FUNCTION}
isFinite}}

\hypertarget{ecldoc:math.isfinite}{}
\hspace{0pt} \hyperlink{ecldoc:Math}{Math} \textbackslash 

{\renewcommand{\arraystretch}{1.5}
\begin{tabularx}{\textwidth}{|>{\raggedright\arraybackslash}l|X|}
\hline
\hspace{0pt}\mytexttt{\color{red} BOOLEAN} & \textbf{isFinite} \\
\hline
\multicolumn{2}{|>{\raggedright\arraybackslash}X|}{\hspace{0pt}\mytexttt{\color{param} (REAL8 val)}} \\
\hline
\end{tabularx}
}

\par
Return whether a real value is a valid value (neither infinite not NaN).

\par
\begin{description}
\item [\colorbox{tagtype}{\color{white} \textbf{\textsf{PARAMETER}}}] \textbf{\underline{val}} The value to test.
\end{description}

\rule{\linewidth}{0.5pt}
\subsection*{\textsf{\colorbox{headtoc}{\color{white} FUNCTION}
FMod}}

\hypertarget{ecldoc:math.fmod}{}
\hspace{0pt} \hyperlink{ecldoc:Math}{Math} \textbackslash 

{\renewcommand{\arraystretch}{1.5}
\begin{tabularx}{\textwidth}{|>{\raggedright\arraybackslash}l|X|}
\hline
\hspace{0pt}\mytexttt{\color{red} REAL8} & \textbf{FMod} \\
\hline
\multicolumn{2}{|>{\raggedright\arraybackslash}X|}{\hspace{0pt}\mytexttt{\color{param} (REAL8 numer, REAL8 denom)}} \\
\hline
\end{tabularx}
}

\par
Returns the floating-point remainder of numer/denom (rounded towards zero). If denom is zero, the result depends on the -fdivideByZero flag: 'zero' or unset: return zero. 'nan': return a non-signalling NaN value 'fail': throw an exception

\par
\begin{description}
\item [\colorbox{tagtype}{\color{white} \textbf{\textsf{PARAMETER}}}] \textbf{\underline{numer}} The numerator.
\item [\colorbox{tagtype}{\color{white} \textbf{\textsf{PARAMETER}}}] \textbf{\underline{denom}} The numerator.
\end{description}

\rule{\linewidth}{0.5pt}
\subsection*{\textsf{\colorbox{headtoc}{\color{white} FUNCTION}
FMatch}}

\hypertarget{ecldoc:math.fmatch}{}
\hspace{0pt} \hyperlink{ecldoc:Math}{Math} \textbackslash 

{\renewcommand{\arraystretch}{1.5}
\begin{tabularx}{\textwidth}{|>{\raggedright\arraybackslash}l|X|}
\hline
\hspace{0pt}\mytexttt{\color{red} BOOLEAN} & \textbf{FMatch} \\
\hline
\multicolumn{2}{|>{\raggedright\arraybackslash}X|}{\hspace{0pt}\mytexttt{\color{param} (REAL8 a, REAL8 b, REAL8 epsilon=0.0)}} \\
\hline
\end{tabularx}
}

\par
Returns whether two floating point values are the same, within margin of error epsilon.

\par
\begin{description}
\item [\colorbox{tagtype}{\color{white} \textbf{\textsf{PARAMETER}}}] \textbf{\underline{a}} The first value.
\item [\colorbox{tagtype}{\color{white} \textbf{\textsf{PARAMETER}}}] \textbf{\underline{b}} The second value.
\item [\colorbox{tagtype}{\color{white} \textbf{\textsf{PARAMETER}}}] \textbf{\underline{epsilon}} The allowable margin of error.
\end{description}

\rule{\linewidth}{0.5pt}


