\chapter*{\color{headtoc} intest}
\hypertarget{ecldoc:toc:root/intest}{}
\hyperlink{ecldoc:toc:root}{Go Up}


\section*{Table of Contents}
{\renewcommand{\arraystretch}{1.5}
\begin{longtable}{|p{\textwidth}|}
\hline
\hyperlink{ecldoc:toc:intest.example_11}{example\_11.ecl} \\
\hline
\hyperlink{ecldoc:toc:intest.example_2}{example\_2.ecl} \\
Basic Inheritance documentation : mod\_3 inherits both mod\_1 and mod\_2 \\
\hline
\hyperlink{ecldoc:toc:intest.example_3}{example\_3.ecl} \\
Example : Inheritance across files \\
\hline
\hyperlink{ecldoc:toc:intest.example_4}{example\_4.ecl} \\
Example : Inheritance across files \\
\hline
\hyperlink{ecldoc:toc:intest.example_5}{example\_5.ecl} \\
\hline
\hyperlink{ecldoc:toc:intest.example_7}{example\_7.ecl} \\
Basic Type Example \\
\hline
\hyperlink{ecldoc:toc:intest.example_9}{example\_9.ecl} \\
\hline
\hyperlink{ecldoc:toc:root/intest/in1intest}{in1intest} \\
\hline
\hyperlink{ecldoc:toc:root/intest/inintest}{inintest} \\
\hline
\end{longtable}
}

\chapter*{example\_11}
\hypertarget{ecldoc:toc:example_11}{}

\section*{\underline{IMPORTS}}
\begin{itemize}
\item Inintest
\item Example\_3
\item intest.Example\_3
\item intest.inintest.Example\_3
\item Inintest.Example\_3
\end{itemize}

\section*{\underline{DESCRIPTIONS}}
\subsection*{MODULE : example\_11}
\hypertarget{ecldoc:example_11}{}
\par
\begin{minipage}[t]{\textwidth}
\begin{flushleft}
  
\end{flushleft}
\end{minipage}
\hyperlink{ecldoc:toc:root}{Up} \\
\par
\par
\begin{enumerate}
\item \hyperlink{ecldoc:Inintest.Example_3}{Example\_3}
\end{enumerate}
\subsection*{MODULE : Example\_3}
\hypertarget{ecldoc:Inintest.Example_3}{}
\par
\begin{minipage}[t]{\textwidth}
\begin{flushleft}
  
\end{flushleft}
\end{minipage}
\hyperlink{ecldoc:example_11}{Up} \\
\par
\par
\textbf{OVERRIDE} : True \\
\begin{enumerate}
\item \hyperlink{ecldoc:Inintest.Example_3.mod_1}{mod\_1}
\end{enumerate}
\subsection*{MODULE : mod\_1}
\hypertarget{ecldoc:Inintest.Example_3.mod_1}{}
\par
\begin{minipage}[t]{\textwidth}
\begin{flushleft}
  
\end{flushleft}
\end{minipage}
\hyperlink{ecldoc:Inintest.Example_3}{Up} \\
\par
\par
\begin{enumerate}
\item \hyperlink{ecldoc:inintest.example_3.mod_1.v2_m1_ex3}{v2\_m1\_ex3}
\end{enumerate}
\subsection*{ATTRIBUTE : v2\_m1\_ex3}
\hypertarget{ecldoc:inintest.example_3.mod_1.v2_m1_ex3}{}
\par
\begin{minipage}[t]{\textwidth}
\begin{flushleft}
  
\end{flushleft}
\end{minipage}
\hyperlink{ecldoc:Inintest.Example_3.mod_1}{Up} \\
\par
\par




\chapter*{\color{headfile}
example_2
}
\hypertarget{ecldoc:toc:example_2}{}
\hyperlink{ecldoc:toc:root}{Go Up}


\section*{\underline{\textsf{DESCRIPTIONS}}}
\subsection*{\textsf{\colorbox{headtoc}{\color{white} MODULE}
example\_2}}

\hypertarget{ecldoc:example_2}{}

{\renewcommand{\arraystretch}{1.5}
\begin{tabularx}{\textwidth}{|>{\raggedright\arraybackslash}l|X|}
\hline
\hspace{0pt}\mytexttt{\color{red} } & \textbf{example\_2} \\
\hline
\end{tabularx}
}

\par
Basic Inheritance documentation : mod\_3 inherits both mod\_1 and mod\_2 . Inherits v2\_m1, v2\_m2, Overrides v1\_m1, new locals v2\_m3 . Interface Inheritance : mod\_4 inherits interface iface\_1, overrides v1\_i1


\textbf{Children}
\begin{enumerate}
\item \hyperlink{ecldoc:example_2.rec_1}{rec\_1}
\item \hyperlink{ecldoc:example_2.rec_2}{rec\_2}
\item \hyperlink{ecldoc:example_2.rec_3}{rec\_3}
\item \hyperlink{ecldoc:example_2.mod_1}{mod\_1}
\item \hyperlink{ecldoc:example_2.mod_2}{mod\_2}
\item \hyperlink{ecldoc:example_2.mod_3}{mod\_3}
\item \hyperlink{ecldoc:example_2.iface_1}{iface\_1}
\item \hyperlink{ecldoc:example_2.mod_4}{mod\_4}
\end{enumerate}

\rule{\linewidth}{0.5pt}

\subsection*{\textsf{\colorbox{headtoc}{\color{white} RECORD}
rec\_1}}

\hypertarget{ecldoc:example_2.rec_1}{}
\hspace{0pt} \hyperlink{ecldoc:example_2}{example_2} \textbackslash 

{\renewcommand{\arraystretch}{1.5}
\begin{tabularx}{\textwidth}{|>{\raggedright\arraybackslash}l|X|}
\hline
\hspace{0pt}\mytexttt{\color{red} } & \textbf{rec\_1} \\
\hline
\end{tabularx}
}

\par


\rule{\linewidth}{0.5pt}
\subsection*{\textsf{\colorbox{headtoc}{\color{white} RECORD}
rec\_2}}

\hypertarget{ecldoc:example_2.rec_2}{}
\hspace{0pt} \hyperlink{ecldoc:example_2}{example_2} \textbackslash 

{\renewcommand{\arraystretch}{1.5}
\begin{tabularx}{\textwidth}{|>{\raggedright\arraybackslash}l|X|}
\hline
\hspace{0pt}\mytexttt{\color{red} } & \textbf{rec\_2} \\
\hline
\end{tabularx}
}

\par


\rule{\linewidth}{0.5pt}
\subsection*{\textsf{\colorbox{headtoc}{\color{white} RECORD}
rec\_3}}

\hypertarget{ecldoc:example_2.rec_3}{}
\hspace{0pt} \hyperlink{ecldoc:example_2}{example_2} \textbackslash 

{\renewcommand{\arraystretch}{1.5}
\begin{tabularx}{\textwidth}{|>{\raggedright\arraybackslash}l|X|}
\hline
\hspace{0pt}\mytexttt{\color{red} } & \textbf{rec\_3} \\
\hline
\end{tabularx}
}

\par


\rule{\linewidth}{0.5pt}
\subsection*{\textsf{\colorbox{headtoc}{\color{white} MODULE}
mod\_1}}

\hypertarget{ecldoc:example_2.mod_1}{}
\hspace{0pt} \hyperlink{ecldoc:example_2}{example_2} \textbackslash 

{\renewcommand{\arraystretch}{1.5}
\begin{tabularx}{\textwidth}{|>{\raggedright\arraybackslash}l|X|}
\hline
\hspace{0pt}\mytexttt{\color{red} } & \textbf{mod\_1} \\
\hline
\end{tabularx}
}

\par


\textbf{Children}
\begin{enumerate}
\item \hyperlink{ecldoc:example_2.mod_1.v1_m1}{v1\_m1}
\item \hyperlink{ecldoc:example_2.mod_1.v2_m1}{v2\_m1}
\end{enumerate}

\rule{\linewidth}{0.5pt}

\subsection*{\textsf{\colorbox{headtoc}{\color{white} ATTRIBUTE}
v1\_m1}}

\hypertarget{ecldoc:example_2.mod_1.v1_m1}{}
\hspace{0pt} \hyperlink{ecldoc:example_2}{example_2} \textbackslash 
\hspace{0pt} \hyperlink{ecldoc:example_2.mod_1}{mod_1} \textbackslash 

{\renewcommand{\arraystretch}{1.5}
\begin{tabularx}{\textwidth}{|>{\raggedright\arraybackslash}l|X|}
\hline
\hspace{0pt}\mytexttt{\color{red} real8} & \textbf{v1\_m1} \\
\hline
\end{tabularx}
}

\par


\rule{\linewidth}{0.5pt}
\subsection*{\textsf{\colorbox{headtoc}{\color{white} ATTRIBUTE}
v2\_m1}}

\hypertarget{ecldoc:example_2.mod_1.v2_m1}{}
\hspace{0pt} \hyperlink{ecldoc:example_2}{example_2} \textbackslash 
\hspace{0pt} \hyperlink{ecldoc:example_2.mod_1}{mod_1} \textbackslash 

{\renewcommand{\arraystretch}{1.5}
\begin{tabularx}{\textwidth}{|>{\raggedright\arraybackslash}l|X|}
\hline
\hspace{0pt}\mytexttt{\color{red} } & \textbf{v2\_m1} \\
\hline
\end{tabularx}
}

\par


\rule{\linewidth}{0.5pt}


\subsection*{\textsf{\colorbox{headtoc}{\color{white} MODULE}
mod\_2}}

\hypertarget{ecldoc:example_2.mod_2}{}
\hspace{0pt} \hyperlink{ecldoc:example_2}{example_2} \textbackslash 

{\renewcommand{\arraystretch}{1.5}
\begin{tabularx}{\textwidth}{|>{\raggedright\arraybackslash}l|X|}
\hline
\hspace{0pt}\mytexttt{\color{red} } & \textbf{mod\_2} \\
\hline
\end{tabularx}
}

\par


\textbf{Children}
\begin{enumerate}
\item \hyperlink{ecldoc:example_2.mod_2.v1_m1}{v1\_m1}
\item \hyperlink{ecldoc:example_2.mod_2.v2_m2}{v2\_m2}
\end{enumerate}

\rule{\linewidth}{0.5pt}

\subsection*{\textsf{\colorbox{headtoc}{\color{white} ATTRIBUTE}
v1\_m1}}

\hypertarget{ecldoc:example_2.mod_2.v1_m1}{}
\hspace{0pt} \hyperlink{ecldoc:example_2}{example_2} \textbackslash 
\hspace{0pt} \hyperlink{ecldoc:example_2.mod_2}{mod_2} \textbackslash 

{\renewcommand{\arraystretch}{1.5}
\begin{tabularx}{\textwidth}{|>{\raggedright\arraybackslash}l|X|}
\hline
\hspace{0pt}\mytexttt{\color{red} } & \textbf{v1\_m1} \\
\hline
\end{tabularx}
}

\par


\rule{\linewidth}{0.5pt}
\subsection*{\textsf{\colorbox{headtoc}{\color{white} ATTRIBUTE}
v2\_m2}}

\hypertarget{ecldoc:example_2.mod_2.v2_m2}{}
\hspace{0pt} \hyperlink{ecldoc:example_2}{example_2} \textbackslash 
\hspace{0pt} \hyperlink{ecldoc:example_2.mod_2}{mod_2} \textbackslash 

{\renewcommand{\arraystretch}{1.5}
\begin{tabularx}{\textwidth}{|>{\raggedright\arraybackslash}l|X|}
\hline
\hspace{0pt}\mytexttt{\color{red} } & \textbf{v2\_m2} \\
\hline
\end{tabularx}
}

\par


\rule{\linewidth}{0.5pt}


\subsection*{\textsf{\colorbox{headtoc}{\color{white} MODULE}
mod\_3}}

\hypertarget{ecldoc:example_2.mod_3}{}
\hspace{0pt} \hyperlink{ecldoc:example_2}{example_2} \textbackslash 

{\renewcommand{\arraystretch}{1.5}
\begin{tabularx}{\textwidth}{|>{\raggedright\arraybackslash}l|X|}
\hline
\hspace{0pt}\mytexttt{\color{red} } & \textbf{mod\_3} \\
\hline
\end{tabularx}
}

\par


\textbf{Children}
\begin{enumerate}
\item \hyperlink{ecldoc:example_2.mod_1.v2_m1}{v2\_m1}
\item \hyperlink{ecldoc:example_2.mod_2.v2_m2}{v2\_m2}
\item \hyperlink{ecldoc:example_2.mod_3.v1_m1}{v1\_m1}
\item \hyperlink{ecldoc:example_2.mod_3.v2_m3}{v2\_m3}
\end{enumerate}

\rule{\linewidth}{0.5pt}

\subsection*{\textsf{\colorbox{headtoc}{\color{white} ATTRIBUTE}
v2\_m1}}

\hypertarget{ecldoc:example_2.mod_1.v2_m1}{}
\hspace{0pt} \hyperlink{ecldoc:example_2}{example_2} \textbackslash 
\hspace{0pt} \hyperlink{ecldoc:example_2.mod_3}{mod_3} \textbackslash 

{\renewcommand{\arraystretch}{1.5}
\begin{tabularx}{\textwidth}{|>{\raggedright\arraybackslash}l|X|}
\hline
\hspace{0pt}\mytexttt{\color{red} } & \textbf{v2\_m1} \\
\hline
\end{tabularx}
}

\par

\par
\begin{description}
\item [\colorbox{tagtype}{\color{white} \textbf{\textsf{INHERITED}}}] \textbf{\underline{}} True
\end{description}

\rule{\linewidth}{0.5pt}
\subsection*{\textsf{\colorbox{headtoc}{\color{white} ATTRIBUTE}
v2\_m2}}

\hypertarget{ecldoc:example_2.mod_2.v2_m2}{}
\hspace{0pt} \hyperlink{ecldoc:example_2}{example_2} \textbackslash 
\hspace{0pt} \hyperlink{ecldoc:example_2.mod_3}{mod_3} \textbackslash 

{\renewcommand{\arraystretch}{1.5}
\begin{tabularx}{\textwidth}{|>{\raggedright\arraybackslash}l|X|}
\hline
\hspace{0pt}\mytexttt{\color{red} } & \textbf{v2\_m2} \\
\hline
\end{tabularx}
}

\par

\par
\begin{description}
\item [\colorbox{tagtype}{\color{white} \textbf{\textsf{INHERITED}}}] \textbf{\underline{}} True
\end{description}

\rule{\linewidth}{0.5pt}
\subsection*{\textsf{\colorbox{headtoc}{\color{white} ATTRIBUTE}
v1\_m1}}

\hypertarget{ecldoc:example_2.mod_3.v1_m1}{}
\hspace{0pt} \hyperlink{ecldoc:example_2}{example_2} \textbackslash 
\hspace{0pt} \hyperlink{ecldoc:example_2.mod_3}{mod_3} \textbackslash 

{\renewcommand{\arraystretch}{1.5}
\begin{tabularx}{\textwidth}{|>{\raggedright\arraybackslash}l|X|}
\hline
\hspace{0pt}\mytexttt{\color{red} } & \textbf{v1\_m1} \\
\hline
\end{tabularx}
}

\par

\par
\begin{description}
\item [\colorbox{tagtype}{\color{white} \textbf{\textsf{OVERRIDE}}}] \textbf{\underline{}} True
\end{description}

\rule{\linewidth}{0.5pt}
\subsection*{\textsf{\colorbox{headtoc}{\color{white} ATTRIBUTE}
v2\_m3}}

\hypertarget{ecldoc:example_2.mod_3.v2_m3}{}
\hspace{0pt} \hyperlink{ecldoc:example_2}{example_2} \textbackslash 
\hspace{0pt} \hyperlink{ecldoc:example_2.mod_3}{mod_3} \textbackslash 

{\renewcommand{\arraystretch}{1.5}
\begin{tabularx}{\textwidth}{|>{\raggedright\arraybackslash}l|X|}
\hline
\hspace{0pt}\mytexttt{\color{red} } & \textbf{v2\_m3} \\
\hline
\end{tabularx}
}

\par


\rule{\linewidth}{0.5pt}


\subsection*{\textsf{\colorbox{headtoc}{\color{white} INTERFACE}
iface\_1}}

\hypertarget{ecldoc:example_2.iface_1}{}
\hspace{0pt} \hyperlink{ecldoc:example_2}{example_2} \textbackslash 

{\renewcommand{\arraystretch}{1.5}
\begin{tabularx}{\textwidth}{|>{\raggedright\arraybackslash}l|X|}
\hline
\hspace{0pt}\mytexttt{\color{red} } & \textbf{iface\_1} \\
\hline
\end{tabularx}
}

\par


\textbf{Children}
\begin{enumerate}
\item \hyperlink{ecldoc:example_2.iface_1.v1_i1}{v1\_i1}
\end{enumerate}

\rule{\linewidth}{0.5pt}

\subsection*{\textsf{\colorbox{headtoc}{\color{white} ATTRIBUTE}
v1\_i1}}

\hypertarget{ecldoc:example_2.iface_1.v1_i1}{}
\hspace{0pt} \hyperlink{ecldoc:example_2}{example_2} \textbackslash 
\hspace{0pt} \hyperlink{ecldoc:example_2.iface_1}{iface_1} \textbackslash 

{\renewcommand{\arraystretch}{1.5}
\begin{tabularx}{\textwidth}{|>{\raggedright\arraybackslash}l|X|}
\hline
\hspace{0pt}\mytexttt{\color{red} real8} & \textbf{v1\_i1} \\
\hline
\end{tabularx}
}

\par


\rule{\linewidth}{0.5pt}


\subsection*{\textsf{\colorbox{headtoc}{\color{white} MODULE}
mod\_4}}

\hypertarget{ecldoc:example_2.mod_4}{}
\hspace{0pt} \hyperlink{ecldoc:example_2}{example_2} \textbackslash 

{\renewcommand{\arraystretch}{1.5}
\begin{tabularx}{\textwidth}{|>{\raggedright\arraybackslash}l|X|}
\hline
\hspace{0pt}\mytexttt{\color{red} } & \textbf{mod\_4} \\
\hline
\end{tabularx}
}

\par


\textbf{Children}
\begin{enumerate}
\item \hyperlink{ecldoc:example_2.mod_4.v1_i1}{v1\_i1}
\item \hyperlink{ecldoc:example_2.mod_4.v2_m4}{v2\_m4}
\end{enumerate}

\rule{\linewidth}{0.5pt}

\subsection*{\textsf{\colorbox{headtoc}{\color{white} ATTRIBUTE}
v1\_i1}}

\hypertarget{ecldoc:example_2.mod_4.v1_i1}{}
\hspace{0pt} \hyperlink{ecldoc:example_2}{example_2} \textbackslash 
\hspace{0pt} \hyperlink{ecldoc:example_2.mod_4}{mod_4} \textbackslash 

{\renewcommand{\arraystretch}{1.5}
\begin{tabularx}{\textwidth}{|>{\raggedright\arraybackslash}l|X|}
\hline
\hspace{0pt}\mytexttt{\color{red} } & \textbf{v1\_i1} \\
\hline
\end{tabularx}
}

\par

\par
\begin{description}
\item [\colorbox{tagtype}{\color{white} \textbf{\textsf{OVERRIDE}}}] \textbf{\underline{}} True
\end{description}

\rule{\linewidth}{0.5pt}
\subsection*{\textsf{\colorbox{headtoc}{\color{white} ATTRIBUTE}
v2\_m4}}

\hypertarget{ecldoc:example_2.mod_4.v2_m4}{}
\hspace{0pt} \hyperlink{ecldoc:example_2}{example_2} \textbackslash 
\hspace{0pt} \hyperlink{ecldoc:example_2.mod_4}{mod_4} \textbackslash 

{\renewcommand{\arraystretch}{1.5}
\begin{tabularx}{\textwidth}{|>{\raggedright\arraybackslash}l|X|}
\hline
\hspace{0pt}\mytexttt{\color{red} STRING20} & \textbf{v2\_m4} \\
\hline
\end{tabularx}
}

\par


\rule{\linewidth}{0.5pt}





\chapter*{\color{headfile}
{\large intest\slash\hspace{0pt}}
{\large in1intest\slash\hspace{0pt}}
 \\
example_3
}
\hypertarget{ecldoc:toc:intest.in1intest.example_3}{}
\hyperlink{ecldoc:toc:root/intest/in1intest}{Go Up}


\section*{\underline{\textsf{DESCRIPTIONS}}}
\subsection*{\textsf{\colorbox{headtoc}{\color{white} MODULE}
Example\_3}}

\hypertarget{ecldoc:intest.in1intest.Example_3}{}

{\renewcommand{\arraystretch}{1.5}
\begin{tabularx}{\textwidth}{|>{\raggedright\arraybackslash}l|X|}
\hline
\hspace{0pt}\mytexttt{\color{red} } & \textbf{Example\_3} \\
\hline
\end{tabularx}
}

\par
Example : Inheritance across files mod\_1 in Example\_4 inherits mod\_1 in Example\_3


\textbf{Children}
\begin{enumerate}
\item \hyperlink{ecldoc:intest.in1intest.Example_3.mod_1}{mod\_1}
\end{enumerate}

\rule{\linewidth}{0.5pt}

\subsection*{\textsf{\colorbox{headtoc}{\color{white} MODULE}
mod\_1}}

\hypertarget{ecldoc:intest.in1intest.Example_3.mod_1}{}
\hspace{0pt} \hyperlink{ecldoc:intest.in1intest.Example_3}{Example_3} \textbackslash 

{\renewcommand{\arraystretch}{1.5}
\begin{tabularx}{\textwidth}{|>{\raggedright\arraybackslash}l|X|}
\hline
\hspace{0pt}\mytexttt{\color{red} } & \textbf{mod\_1} \\
\hline
\end{tabularx}
}

\par


\textbf{Children}
\begin{enumerate}
\item \hyperlink{ecldoc:intest.in1intest.example_3.mod_1.v1_m1}{v1\_m1}
\item \hyperlink{ecldoc:intest.in1intest.example_3.mod_1.v2_m1_ex3}{v2\_m1\_ex3}
\end{enumerate}

\rule{\linewidth}{0.5pt}

\subsection*{\textsf{\colorbox{headtoc}{\color{white} ATTRIBUTE}
v1\_m1}}

\hypertarget{ecldoc:intest.in1intest.example_3.mod_1.v1_m1}{}
\hspace{0pt} \hyperlink{ecldoc:intest.in1intest.Example_3}{Example_3} \textbackslash 
\hspace{0pt} \hyperlink{ecldoc:intest.in1intest.Example_3.mod_1}{mod_1} \textbackslash 

{\renewcommand{\arraystretch}{1.5}
\begin{tabularx}{\textwidth}{|>{\raggedright\arraybackslash}l|X|}
\hline
\hspace{0pt}\mytexttt{\color{red} } & \textbf{v1\_m1} \\
\hline
\end{tabularx}
}

\par


\rule{\linewidth}{0.5pt}
\subsection*{\textsf{\colorbox{headtoc}{\color{white} ATTRIBUTE}
v2\_m1\_ex3}}

\hypertarget{ecldoc:intest.in1intest.example_3.mod_1.v2_m1_ex3}{}
\hspace{0pt} \hyperlink{ecldoc:intest.in1intest.Example_3}{Example_3} \textbackslash 
\hspace{0pt} \hyperlink{ecldoc:intest.in1intest.Example_3.mod_1}{mod_1} \textbackslash 

{\renewcommand{\arraystretch}{1.5}
\begin{tabularx}{\textwidth}{|>{\raggedright\arraybackslash}l|X|}
\hline
\hspace{0pt}\mytexttt{\color{red} } & \textbf{v2\_m1\_ex3} \\
\hline
\end{tabularx}
}

\par


\rule{\linewidth}{0.5pt}





\chapter*{\color{headfile}
{\large intest\slash\hspace{0pt}}
 \\
example_4
}
\hypertarget{ecldoc:toc:intest.example_4}{}
\hyperlink{ecldoc:toc:root/intest}{Go Up}

\section*{\underline{\textsf{IMPORTS}}}
\begin{doublespace}
{\large
Example\_3.mod\_1 |
}
\end{doublespace}

\section*{\underline{\textsf{DESCRIPTIONS}}}
\subsection*{\textsf{\colorbox{headtoc}{\color{white} MODULE}
example\_4}}

\hypertarget{ecldoc:intest.example_4}{}

{\renewcommand{\arraystretch}{1.5}
\begin{tabularx}{\textwidth}{|>{\raggedright\arraybackslash}l|X|}
\hline
\hspace{0pt}\mytexttt{\color{red} } & \textbf{example\_4} \\
\hline
\end{tabularx}
}

\par





Example : Inheritance across files mod\_1 in Example\_4 inherits mod\_1 in Example\_3







\textbf{Children}
\begin{enumerate}
\item \hyperlink{ecldoc:intest.example_4.mod_1}{mod\_1}
: No Documentation Found
\end{enumerate}

\rule{\linewidth}{0.5pt}

\subsection*{\textsf{\colorbox{headtoc}{\color{white} MODULE}
mod\_1}}

\hypertarget{ecldoc:intest.example_4.mod_1}{}
\hspace{0pt} \hyperlink{ecldoc:intest.example_4}{example_4} \textbackslash 

{\renewcommand{\arraystretch}{1.5}
\begin{tabularx}{\textwidth}{|>{\raggedright\arraybackslash}l|X|}
\hline
\hspace{0pt}\mytexttt{\color{red} } & \textbf{mod\_1} \\
\hline
\end{tabularx}
}

\par





No Documentation Found










\par
\begin{description}
\item [\colorbox{tagtype}{\color{white} \textbf{\textsf{PARENT}}}] \textbf{Example\_3.mod\_1} <../example\_3.ecl.tex>
\end{description}


\textbf{Children}
\begin{enumerate}
\item \hyperlink{ecldoc:intest.example_4.mod_1.v2_m1_ex4}{v2\_m1\_ex4}
: No Documentation Found
\item \hyperlink{ecldoc:example_3.mod_1.v1_m1}{v1\_m1}
: Doc test 2
\item \hyperlink{ecldoc:example_3.mod_1.v2_m1_ex3}{v2\_m1\_ex3}
: DOC Test 3
\item \hyperlink{ecldoc:example_3.mod_1.abc}{abc}
: No Documentation Found
\item \hyperlink{ecldoc:example_3.mod_1.long_name}{long\_name}
: No Documentation Found
\end{enumerate}

\rule{\linewidth}{0.5pt}

\subsection*{\textsf{\colorbox{headtoc}{\color{white} ATTRIBUTE}
v2\_m1\_ex4}}

\hypertarget{ecldoc:intest.example_4.mod_1.v2_m1_ex4}{}
\hspace{0pt} \hyperlink{ecldoc:intest.example_4}{example_4} \textbackslash 
\hspace{0pt} \hyperlink{ecldoc:intest.example_4.mod_1}{mod_1} \textbackslash 

{\renewcommand{\arraystretch}{1.5}
\begin{tabularx}{\textwidth}{|>{\raggedright\arraybackslash}l|X|}
\hline
\hspace{0pt}\mytexttt{\color{red} } & \textbf{v2\_m1\_ex4} \\
\hline
\end{tabularx}
}

\par





No Documentation Found








\par
\begin{description}
\item [\colorbox{tagtype}{\color{white} \textbf{\textsf{RETURN}}}] \textbf{REAL8} --- 
\end{description}




\rule{\linewidth}{0.5pt}
\subsection*{\textsf{\colorbox{headtoc}{\color{white} ATTRIBUTE}
v1\_m1}}

\hypertarget{ecldoc:example_3.mod_1.v1_m1}{}
\hspace{0pt} \hyperlink{ecldoc:intest.example_4}{example_4} \textbackslash 
\hspace{0pt} \hyperlink{ecldoc:intest.example_4.mod_1}{mod_1} \textbackslash 

{\renewcommand{\arraystretch}{1.5}
\begin{tabularx}{\textwidth}{|>{\raggedright\arraybackslash}l|X|}
\hline
\hspace{0pt}\mytexttt{\color{red} } & \textbf{v1\_m1} \\
\hline
\end{tabularx}
}

\par





Doc test 2. Title end by period not newline 
\begin{verbatim}

 ABCD ||||
 CDEF ||||\end{verbatim}










\par
\begin{description}
\item [\colorbox{tagtype}{\color{white} \textbf{\textsf{RETURN}}}] \textbf{REAL8} --- 
\end{description}






\par
\begin{description}
\item [\colorbox{tagtype}{\color{white} \textbf{\textsf{INHERITED}}}] 
\end{description}



\rule{\linewidth}{0.5pt}
\subsection*{\textsf{\colorbox{headtoc}{\color{white} ATTRIBUTE}
v2\_m1\_ex3}}

\hypertarget{ecldoc:example_3.mod_1.v2_m1_ex3}{}
\hspace{0pt} \hyperlink{ecldoc:intest.example_4}{example_4} \textbackslash 
\hspace{0pt} \hyperlink{ecldoc:intest.example_4.mod_1}{mod_1} \textbackslash 

{\renewcommand{\arraystretch}{1.5}
\begin{tabularx}{\textwidth}{|>{\raggedright\arraybackslash}l|X|}
\hline
\hspace{0pt}\mytexttt{\color{red} } & \textbf{v2\_m1\_ex3} \\
\hline
\end{tabularx}
}

\par





DOC Test 3 No Period title








\par
\begin{description}
\item [\colorbox{tagtype}{\color{white} \textbf{\textsf{RETURN}}}] \textbf{REAL8} --- 
\end{description}






\par
\begin{description}
\item [\colorbox{tagtype}{\color{white} \textbf{\textsf{INHERITED}}}] 
\end{description}



\rule{\linewidth}{0.5pt}
\subsection*{\textsf{\colorbox{headtoc}{\color{white} FUNCTION}
abc}}

\hypertarget{ecldoc:example_3.mod_1.abc}{}
\hspace{0pt} \hyperlink{ecldoc:intest.example_4}{example_4} \textbackslash 
\hspace{0pt} \hyperlink{ecldoc:intest.example_4.mod_1}{mod_1} \textbackslash 

{\renewcommand{\arraystretch}{1.5}
\begin{tabularx}{\textwidth}{|>{\raggedright\arraybackslash}l|X|}
\hline
\hspace{0pt}\mytexttt{\color{red} REAL8} & \textbf{abc} \\
\hline
\multicolumn{2}{|>{\raggedright\arraybackslash}X|}{\hspace{0pt}\mytexttt{\color{param} (REAL8 x)}} \\
\hline
\end{tabularx}
}

\par





No Documentation Found






\par
\begin{description}
\item [\colorbox{tagtype}{\color{white} \textbf{\textsf{PARAMETER}}}] \textbf{\underline{x}} ||| REAL8 --- No Doc
\end{description}







\par
\begin{description}
\item [\colorbox{tagtype}{\color{white} \textbf{\textsf{RETURN}}}] \textbf{REAL8} --- 
\end{description}






\par
\begin{description}
\item [\colorbox{tagtype}{\color{white} \textbf{\textsf{INHERITED}}}] 
\end{description}



\rule{\linewidth}{0.5pt}
\subsection*{\textsf{\colorbox{headtoc}{\color{white} FUNCTION}
long\_name}}

\hypertarget{ecldoc:example_3.mod_1.long_name}{}
\hspace{0pt} \hyperlink{ecldoc:intest.example_4}{example_4} \textbackslash 
\hspace{0pt} \hyperlink{ecldoc:intest.example_4.mod_1}{mod_1} \textbackslash 

{\renewcommand{\arraystretch}{1.5}
\begin{tabularx}{\textwidth}{|>{\raggedright\arraybackslash}l|X|}
\hline
\hspace{0pt}\mytexttt{\color{red} } & \textbf{long\_name} \\
\hline
\multicolumn{2}{|>{\raggedright\arraybackslash}X|}{\hspace{0pt}\mytexttt{\color{param} (DATASET(\{REAL8 u\}) X, DATASET(\{REAL8 u\}) IntW, DATASET(\{REAL8 u\}) Intb, REAL8 BETA=0.1, REAL8 sparsityParam=0.1 , REAL8 LAMBDA=0.001, REAL8 ALPHA=0.1, UNSIGNED2 MaxIter=100)}} \\
\hline
\end{tabularx}
}

\par





No Documentation Found






\par
\begin{description}
\item [\colorbox{tagtype}{\color{white} \textbf{\textsf{PARAMETER}}}] \textbf{\underline{sparsityparam}} ||| REAL8 --- No Doc
\item [\colorbox{tagtype}{\color{white} \textbf{\textsf{PARAMETER}}}] \textbf{\underline{alpha}} ||| REAL8 --- No Doc
\item [\colorbox{tagtype}{\color{white} \textbf{\textsf{PARAMETER}}}] \textbf{\underline{lambda}} ||| REAL8 --- No Doc
\item [\colorbox{tagtype}{\color{white} \textbf{\textsf{PARAMETER}}}] \textbf{\underline{maxiter}} ||| UNSIGNED2 --- No Doc
\item [\colorbox{tagtype}{\color{white} \textbf{\textsf{PARAMETER}}}] \textbf{\underline{intb}} ||| TABLE ( \{ REAL8 u \} ) --- No Doc
\item [\colorbox{tagtype}{\color{white} \textbf{\textsf{PARAMETER}}}] \textbf{\underline{beta}} ||| REAL8 --- No Doc
\item [\colorbox{tagtype}{\color{white} \textbf{\textsf{PARAMETER}}}] \textbf{\underline{x}} ||| TABLE ( \{ REAL8 u \} ) --- No Doc
\item [\colorbox{tagtype}{\color{white} \textbf{\textsf{PARAMETER}}}] \textbf{\underline{intw}} ||| TABLE ( \{ REAL8 u \} ) --- No Doc
\end{description}







\par
\begin{description}
\item [\colorbox{tagtype}{\color{white} \textbf{\textsf{RETURN}}}] \textbf{REAL8} --- 
\end{description}






\par
\begin{description}
\item [\colorbox{tagtype}{\color{white} \textbf{\textsf{INHERITED}}}] 
\end{description}



\rule{\linewidth}{0.5pt}





\chapter*{example\_5}

\section*{\underline{IMPORTS}}

\section*{\underline{DESCRIPTIONS}}

\chapter*{\color{headfile}
example_7
}
\hypertarget{ecldoc:toc:example_7}{}
\hyperlink{ecldoc:toc:root}{Go Up}


\section*{\underline{\textsf{DESCRIPTIONS}}}
\subsection*{\textsf{\colorbox{headtoc}{\color{white} MODULE}
example\_7}}

\hypertarget{ecldoc:example_7}{}

{\renewcommand{\arraystretch}{1.5}
\begin{tabularx}{\textwidth}{|>{\raggedright\arraybackslash}l|X|}
\hline
\hspace{0pt}\mytexttt{\color{red} } & \textbf{example\_7} \\
\hline
\end{tabularx}
}

\par





Basic Type Example Source Code copied from ECL Documentation







\textbf{Children}
\begin{enumerate}
\item \hyperlink{ecldoc:example_7.r}{R}
: No Documentation Found
\end{enumerate}

\rule{\linewidth}{0.5pt}

\subsection*{\textsf{\colorbox{headtoc}{\color{white} RECORD}
R}}

\hypertarget{ecldoc:example_7.r}{}
\hspace{0pt} \hyperlink{ecldoc:example_7}{example_7} \textbackslash 

{\renewcommand{\arraystretch}{1.5}
\begin{tabularx}{\textwidth}{|>{\raggedright\arraybackslash}l|X|}
\hline
\hspace{0pt}\mytexttt{\color{red} } & \textbf{R} \\
\hline
\end{tabularx}
}

\par





No Documentation Found







\par
\begin{description}
\item [\colorbox{tagtype}{\color{white} \textbf{\textsf{FIELD}}}] \textbf{\underline{f3}} ||| SCALEINT --- No Doc
\item [\colorbox{tagtype}{\color{white} \textbf{\textsf{FIELD}}}] \textbf{\underline{f1}} ||| REVERSESTRING4 --- No Doc
\item [\colorbox{tagtype}{\color{white} \textbf{\textsf{FIELD}}}] \textbf{\underline{f2}} ||| NEEDC --- No Doc
\end{description}





\rule{\linewidth}{0.5pt}



\chapter*{\color{headfile}
example_9
}
\hypertarget{ecldoc:toc:example_9}{}
\hyperlink{ecldoc:toc:root}{Go Up}

\section*{\underline{\textsf{IMPORTS}}}
\begin{doublespace}
{\large
example\_8 |
example\_8.mod\_1 |
}
\end{doublespace}

\section*{\underline{\textsf{DESCRIPTIONS}}}

\chapter*{\color{headtoc} root}
\hypertarget{ecldoc:toc:root}{}
\hyperlink{ecldoc:toc:}{Go Up}


\section*{Table of Contents}
{\renewcommand{\arraystretch}{1.5}
\begin{longtable}{|p{\textwidth}|}
\hline
\hyperlink{ecldoc:toc:BLAS}{BLAS.ecl} \\
\hline
\hyperlink{ecldoc:toc:BundleBase}{BundleBase.ecl} \\
\hline
\hyperlink{ecldoc:toc:Date}{Date.ecl} \\
\hline
\hyperlink{ecldoc:toc:File}{File.ecl} \\
\hline
\hyperlink{ecldoc:toc:math}{math.ecl} \\
\hline
\hyperlink{ecldoc:toc:Metaphone}{Metaphone.ecl} \\
\hline
\hyperlink{ecldoc:toc:str}{str.ecl} \\
\hline
\hyperlink{ecldoc:toc:Uni}{Uni.ecl} \\
\hline
\hyperlink{ecldoc:toc:root/system}{system} \\
\hline
\end{longtable}
}

\chapter*{\color{headfile}
BLAS
}
\hypertarget{ecldoc:toc:BLAS}{}
\hyperlink{ecldoc:toc:root}{Go Up}

\section*{\underline{\textsf{IMPORTS}}}
\begin{doublespace}
{\large
lib\_eclblas |
}
\end{doublespace}

\section*{\underline{\textsf{DESCRIPTIONS}}}
\subsection*{\textsf{\colorbox{headtoc}{\color{white} MODULE}
BLAS}}

\hypertarget{ecldoc:blas}{}

{\renewcommand{\arraystretch}{1.5}
\begin{tabularx}{\textwidth}{|>{\raggedright\arraybackslash}l|X|}
\hline
\hspace{0pt}\mytexttt{\color{red} } & \textbf{BLAS} \\
\hline
\end{tabularx}
}

\par


\textbf{Children}
\begin{enumerate}
\item \hyperlink{ecldoc:BLAS.Types}{Types}
\item \hyperlink{ecldoc:blas.icellfunc}{ICellFunc}
: Function prototype for Apply2Cell
\item \hyperlink{ecldoc:blas.apply2cells}{Apply2Cells}
: Iterate matrix and apply function to each cell
\item \hyperlink{ecldoc:blas.dasum}{dasum}
: Absolute sum, the 1 norm of a vector
\item \hyperlink{ecldoc:blas.daxpy}{daxpy}
: alpha*X + Y
\item \hyperlink{ecldoc:blas.dgemm}{dgemm}
: alpha*op(A) op(B) + beta*C where op() is transpose
\item \hyperlink{ecldoc:blas.dgetf2}{dgetf2}
: Compute LU Factorization of matrix A
\item \hyperlink{ecldoc:blas.dpotf2}{dpotf2}
: DPOTF2 computes the Cholesky factorization of a real symmetric positive definite matrix A
\item \hyperlink{ecldoc:blas.dscal}{dscal}
: Scale a vector alpha
\item \hyperlink{ecldoc:blas.dsyrk}{dsyrk}
: Implements symmetric rank update C
\item \hyperlink{ecldoc:blas.dtrsm}{dtrsm}
: Triangular matrix solver
\item \hyperlink{ecldoc:blas.extract_diag}{extract\_diag}
: Extract the diagonal of he matrix
\item \hyperlink{ecldoc:blas.extract_tri}{extract\_tri}
: Extract the upper or lower triangle
\item \hyperlink{ecldoc:blas.make_diag}{make\_diag}
: Generate a diagonal matrix
\item \hyperlink{ecldoc:blas.make_vector}{make\_vector}
: Make a vector of dimension m
\item \hyperlink{ecldoc:blas.trace}{trace}
: The trace of the input matrix
\end{enumerate}

\rule{\linewidth}{0.5pt}

\subsection*{\textsf{\colorbox{headtoc}{\color{white} MODULE}
Types}}

\hypertarget{ecldoc:BLAS.Types}{}
\hspace{0pt} \hyperlink{ecldoc:blas}{BLAS} \textbackslash 

{\renewcommand{\arraystretch}{1.5}
\begin{tabularx}{\textwidth}{|>{\raggedright\arraybackslash}l|X|}
\hline
\hspace{0pt}\mytexttt{\color{red} } & \textbf{Types} \\
\hline
\end{tabularx}
}

\par


\textbf{Children}
\begin{enumerate}
\item \hyperlink{ecldoc:blas.types.value_t}{value\_t}
\item \hyperlink{ecldoc:blas.types.dimension_t}{dimension\_t}
\item \hyperlink{ecldoc:blas.types.matrix_t}{matrix\_t}
\item \hyperlink{ecldoc:ecldoc-Triangle}{Triangle}
\item \hyperlink{ecldoc:ecldoc-Diagonal}{Diagonal}
\item \hyperlink{ecldoc:ecldoc-Side}{Side}
\end{enumerate}

\rule{\linewidth}{0.5pt}

\subsection*{\textsf{\colorbox{headtoc}{\color{white} ATTRIBUTE}
value\_t}}

\hypertarget{ecldoc:blas.types.value_t}{}
\hspace{0pt} \hyperlink{ecldoc:blas}{BLAS} \textbackslash 
\hspace{0pt} \hyperlink{ecldoc:BLAS.Types}{Types} \textbackslash 

{\renewcommand{\arraystretch}{1.5}
\begin{tabularx}{\textwidth}{|>{\raggedright\arraybackslash}l|X|}
\hline
\hspace{0pt}\mytexttt{\color{red} } & \textbf{value\_t} \\
\hline
\end{tabularx}
}

\par


\rule{\linewidth}{0.5pt}
\subsection*{\textsf{\colorbox{headtoc}{\color{white} ATTRIBUTE}
dimension\_t}}

\hypertarget{ecldoc:blas.types.dimension_t}{}
\hspace{0pt} \hyperlink{ecldoc:blas}{BLAS} \textbackslash 
\hspace{0pt} \hyperlink{ecldoc:BLAS.Types}{Types} \textbackslash 

{\renewcommand{\arraystretch}{1.5}
\begin{tabularx}{\textwidth}{|>{\raggedright\arraybackslash}l|X|}
\hline
\hspace{0pt}\mytexttt{\color{red} } & \textbf{dimension\_t} \\
\hline
\end{tabularx}
}

\par


\rule{\linewidth}{0.5pt}
\subsection*{\textsf{\colorbox{headtoc}{\color{white} ATTRIBUTE}
matrix\_t}}

\hypertarget{ecldoc:blas.types.matrix_t}{}
\hspace{0pt} \hyperlink{ecldoc:blas}{BLAS} \textbackslash 
\hspace{0pt} \hyperlink{ecldoc:BLAS.Types}{Types} \textbackslash 

{\renewcommand{\arraystretch}{1.5}
\begin{tabularx}{\textwidth}{|>{\raggedright\arraybackslash}l|X|}
\hline
\hspace{0pt}\mytexttt{\color{red} } & \textbf{matrix\_t} \\
\hline
\end{tabularx}
}

\par


\rule{\linewidth}{0.5pt}
\subsection*{\textsf{\colorbox{headtoc}{\color{white} ATTRIBUTE}
Triangle}}

\hypertarget{ecldoc:ecldoc-Triangle}{}
\hspace{0pt} \hyperlink{ecldoc:blas}{BLAS} \textbackslash 
\hspace{0pt} \hyperlink{ecldoc:BLAS.Types}{Types} \textbackslash 

{\renewcommand{\arraystretch}{1.5}
\begin{tabularx}{\textwidth}{|>{\raggedright\arraybackslash}l|X|}
\hline
\hspace{0pt}\mytexttt{\color{red} } & \textbf{Triangle} \\
\hline
\end{tabularx}
}

\par


\rule{\linewidth}{0.5pt}
\subsection*{\textsf{\colorbox{headtoc}{\color{white} ATTRIBUTE}
Diagonal}}

\hypertarget{ecldoc:ecldoc-Diagonal}{}
\hspace{0pt} \hyperlink{ecldoc:blas}{BLAS} \textbackslash 
\hspace{0pt} \hyperlink{ecldoc:BLAS.Types}{Types} \textbackslash 

{\renewcommand{\arraystretch}{1.5}
\begin{tabularx}{\textwidth}{|>{\raggedright\arraybackslash}l|X|}
\hline
\hspace{0pt}\mytexttt{\color{red} } & \textbf{Diagonal} \\
\hline
\end{tabularx}
}

\par


\rule{\linewidth}{0.5pt}
\subsection*{\textsf{\colorbox{headtoc}{\color{white} ATTRIBUTE}
Side}}

\hypertarget{ecldoc:ecldoc-Side}{}
\hspace{0pt} \hyperlink{ecldoc:blas}{BLAS} \textbackslash 
\hspace{0pt} \hyperlink{ecldoc:BLAS.Types}{Types} \textbackslash 

{\renewcommand{\arraystretch}{1.5}
\begin{tabularx}{\textwidth}{|>{\raggedright\arraybackslash}l|X|}
\hline
\hspace{0pt}\mytexttt{\color{red} } & \textbf{Side} \\
\hline
\end{tabularx}
}

\par


\rule{\linewidth}{0.5pt}


\subsection*{\textsf{\colorbox{headtoc}{\color{white} FUNCTION}
ICellFunc}}

\hypertarget{ecldoc:blas.icellfunc}{}
\hspace{0pt} \hyperlink{ecldoc:blas}{BLAS} \textbackslash 

{\renewcommand{\arraystretch}{1.5}
\begin{tabularx}{\textwidth}{|>{\raggedright\arraybackslash}l|X|}
\hline
\hspace{0pt}\mytexttt{\color{red} Types.value\_t} & \textbf{ICellFunc} \\
\hline
\multicolumn{2}{|>{\raggedright\arraybackslash}X|}{\hspace{0pt}\mytexttt{\color{param} (Types.value\_t v, Types.dimension\_t r, Types.dimension\_t c)}} \\
\hline
\end{tabularx}
}

\par
Function prototype for Apply2Cell.

\par
\begin{description}
\item [\colorbox{tagtype}{\color{white} \textbf{\textsf{PARAMETER}}}] \textbf{\underline{v}} the value
\item [\colorbox{tagtype}{\color{white} \textbf{\textsf{PARAMETER}}}] \textbf{\underline{r}} the row ordinal
\item [\colorbox{tagtype}{\color{white} \textbf{\textsf{PARAMETER}}}] \textbf{\underline{c}} the column ordinal
\item [\colorbox{tagtype}{\color{white} \textbf{\textsf{RETURN}}}] \textbf{\underline{}} the updated value
\end{description}

\rule{\linewidth}{0.5pt}
\subsection*{\textsf{\colorbox{headtoc}{\color{white} FUNCTION}
Apply2Cells}}

\hypertarget{ecldoc:blas.apply2cells}{}
\hspace{0pt} \hyperlink{ecldoc:blas}{BLAS} \textbackslash 

{\renewcommand{\arraystretch}{1.5}
\begin{tabularx}{\textwidth}{|>{\raggedright\arraybackslash}l|X|}
\hline
\hspace{0pt}\mytexttt{\color{red} Types.matrix\_t} & \textbf{Apply2Cells} \\
\hline
\multicolumn{2}{|>{\raggedright\arraybackslash}X|}{\hspace{0pt}\mytexttt{\color{param} (Types.dimension\_t m, Types.dimension\_t n, Types.matrix\_t x, ICellFunc f)}} \\
\hline
\end{tabularx}
}

\par
Iterate matrix and apply function to each cell

\par
\begin{description}
\item [\colorbox{tagtype}{\color{white} \textbf{\textsf{PARAMETER}}}] \textbf{\underline{m}} number of rows
\item [\colorbox{tagtype}{\color{white} \textbf{\textsf{PARAMETER}}}] \textbf{\underline{n}} number of columns
\item [\colorbox{tagtype}{\color{white} \textbf{\textsf{PARAMETER}}}] \textbf{\underline{x}} matrix
\item [\colorbox{tagtype}{\color{white} \textbf{\textsf{PARAMETER}}}] \textbf{\underline{f}} function to apply
\item [\colorbox{tagtype}{\color{white} \textbf{\textsf{RETURN}}}] \textbf{\underline{}} updated matrix
\end{description}

\rule{\linewidth}{0.5pt}
\subsection*{\textsf{\colorbox{headtoc}{\color{white} FUNCTION}
dasum}}

\hypertarget{ecldoc:blas.dasum}{}
\hspace{0pt} \hyperlink{ecldoc:blas}{BLAS} \textbackslash 

{\renewcommand{\arraystretch}{1.5}
\begin{tabularx}{\textwidth}{|>{\raggedright\arraybackslash}l|X|}
\hline
\hspace{0pt}\mytexttt{\color{red} Types.value\_t} & \textbf{dasum} \\
\hline
\multicolumn{2}{|>{\raggedright\arraybackslash}X|}{\hspace{0pt}\mytexttt{\color{param} (Types.dimension\_t m, Types.matrix\_t x, Types.dimension\_t incx, Types.dimension\_t skipped=0)}} \\
\hline
\end{tabularx}
}

\par
Absolute sum, the 1 norm of a vector.

\par
\begin{description}
\item [\colorbox{tagtype}{\color{white} \textbf{\textsf{PARAMETER}}}] \textbf{\underline{m}} the number of entries
\item [\colorbox{tagtype}{\color{white} \textbf{\textsf{PARAMETER}}}] \textbf{\underline{x}} the column major matrix holding the vector
\item [\colorbox{tagtype}{\color{white} \textbf{\textsf{PARAMETER}}}] \textbf{\underline{incx}} the increment for x, 1 in the case of an actual vector
\item [\colorbox{tagtype}{\color{white} \textbf{\textsf{PARAMETER}}}] \textbf{\underline{skipped}} default is zero, the number of entries stepped over to get to the first entry
\item [\colorbox{tagtype}{\color{white} \textbf{\textsf{RETURN}}}] \textbf{\underline{}} the sum of the absolute values
\end{description}

\rule{\linewidth}{0.5pt}
\subsection*{\textsf{\colorbox{headtoc}{\color{white} FUNCTION}
daxpy}}

\hypertarget{ecldoc:blas.daxpy}{}
\hspace{0pt} \hyperlink{ecldoc:blas}{BLAS} \textbackslash 

{\renewcommand{\arraystretch}{1.5}
\begin{tabularx}{\textwidth}{|>{\raggedright\arraybackslash}l|X|}
\hline
\hspace{0pt}\mytexttt{\color{red} Types.matrix\_t} & \textbf{daxpy} \\
\hline
\multicolumn{2}{|>{\raggedright\arraybackslash}X|}{\hspace{0pt}\mytexttt{\color{param} (Types.dimension\_t N, Types.value\_t alpha, Types.matrix\_t X, Types.dimension\_t incX, Types.matrix\_t Y, Types.dimension\_t incY, Types.dimension\_t x\_skipped=0, Types.dimension\_t y\_skipped=0)}} \\
\hline
\end{tabularx}
}

\par
alpha*X + Y

\par
\begin{description}
\item [\colorbox{tagtype}{\color{white} \textbf{\textsf{PARAMETER}}}] \textbf{\underline{N}} number of elements in vector
\item [\colorbox{tagtype}{\color{white} \textbf{\textsf{PARAMETER}}}] \textbf{\underline{alpha}} the scalar multiplier
\item [\colorbox{tagtype}{\color{white} \textbf{\textsf{PARAMETER}}}] \textbf{\underline{X}} the column major matrix holding the vector X
\item [\colorbox{tagtype}{\color{white} \textbf{\textsf{PARAMETER}}}] \textbf{\underline{incX}} the increment or stride for the vector
\item [\colorbox{tagtype}{\color{white} \textbf{\textsf{PARAMETER}}}] \textbf{\underline{Y}} the column major matrix holding the vector Y
\item [\colorbox{tagtype}{\color{white} \textbf{\textsf{PARAMETER}}}] \textbf{\underline{incY}} the increment or stride of Y
\item [\colorbox{tagtype}{\color{white} \textbf{\textsf{PARAMETER}}}] \textbf{\underline{x\_skipped}} number of entries skipped to get to the first X
\item [\colorbox{tagtype}{\color{white} \textbf{\textsf{PARAMETER}}}] \textbf{\underline{y\_skipped}} number of entries skipped to get to the first Y
\item [\colorbox{tagtype}{\color{white} \textbf{\textsf{RETURN}}}] \textbf{\underline{}} the updated matrix
\end{description}

\rule{\linewidth}{0.5pt}
\subsection*{\textsf{\colorbox{headtoc}{\color{white} FUNCTION}
dgemm}}

\hypertarget{ecldoc:blas.dgemm}{}
\hspace{0pt} \hyperlink{ecldoc:blas}{BLAS} \textbackslash 

{\renewcommand{\arraystretch}{1.5}
\begin{tabularx}{\textwidth}{|>{\raggedright\arraybackslash}l|X|}
\hline
\hspace{0pt}\mytexttt{\color{red} Types.matrix\_t} & \textbf{dgemm} \\
\hline
\multicolumn{2}{|>{\raggedright\arraybackslash}X|}{\hspace{0pt}\mytexttt{\color{param} (BOOLEAN transposeA, BOOLEAN transposeB, Types.dimension\_t M, Types.dimension\_t N, Types.dimension\_t K, Types.value\_t alpha, Types.matrix\_t A, Types.matrix\_t B, Types.value\_t beta=0.0, Types.matrix\_t C=[])}} \\
\hline
\end{tabularx}
}

\par
alpha*op(A) op(B) + beta*C where op() is transpose

\par
\begin{description}
\item [\colorbox{tagtype}{\color{white} \textbf{\textsf{PARAMETER}}}] \textbf{\underline{transposeA}} true when transpose of A is used
\item [\colorbox{tagtype}{\color{white} \textbf{\textsf{PARAMETER}}}] \textbf{\underline{transposeB}} true when transpose of B is used
\item [\colorbox{tagtype}{\color{white} \textbf{\textsf{PARAMETER}}}] \textbf{\underline{M}} number of rows in product
\item [\colorbox{tagtype}{\color{white} \textbf{\textsf{PARAMETER}}}] \textbf{\underline{N}} number of columns in product
\item [\colorbox{tagtype}{\color{white} \textbf{\textsf{PARAMETER}}}] \textbf{\underline{K}} number of columns/rows for the multiplier/multiplicand
\item [\colorbox{tagtype}{\color{white} \textbf{\textsf{PARAMETER}}}] \textbf{\underline{alpha}} scalar used on A
\item [\colorbox{tagtype}{\color{white} \textbf{\textsf{PARAMETER}}}] \textbf{\underline{A}} matrix A
\item [\colorbox{tagtype}{\color{white} \textbf{\textsf{PARAMETER}}}] \textbf{\underline{B}} matrix B
\item [\colorbox{tagtype}{\color{white} \textbf{\textsf{PARAMETER}}}] \textbf{\underline{beta}} scalar for matrix C
\item [\colorbox{tagtype}{\color{white} \textbf{\textsf{PARAMETER}}}] \textbf{\underline{C}} matrix C or empty
\end{description}

\rule{\linewidth}{0.5pt}
\subsection*{\textsf{\colorbox{headtoc}{\color{white} FUNCTION}
dgetf2}}

\hypertarget{ecldoc:blas.dgetf2}{}
\hspace{0pt} \hyperlink{ecldoc:blas}{BLAS} \textbackslash 

{\renewcommand{\arraystretch}{1.5}
\begin{tabularx}{\textwidth}{|>{\raggedright\arraybackslash}l|X|}
\hline
\hspace{0pt}\mytexttt{\color{red} Types.matrix\_t} & \textbf{dgetf2} \\
\hline
\multicolumn{2}{|>{\raggedright\arraybackslash}X|}{\hspace{0pt}\mytexttt{\color{param} (Types.dimension\_t m, Types.dimension\_t n, Types.matrix\_t a)}} \\
\hline
\end{tabularx}
}

\par
Compute LU Factorization of matrix A.

\par
\begin{description}
\item [\colorbox{tagtype}{\color{white} \textbf{\textsf{PARAMETER}}}] \textbf{\underline{m}} number of rows of A
\item [\colorbox{tagtype}{\color{white} \textbf{\textsf{PARAMETER}}}] \textbf{\underline{n}} number of columns of A
\item [\colorbox{tagtype}{\color{white} \textbf{\textsf{RETURN}}}] \textbf{\underline{}} composite matrix of factors, lower triangle has an implied diagonal of ones. Upper triangle has the diagonal of the composite.
\end{description}

\rule{\linewidth}{0.5pt}
\subsection*{\textsf{\colorbox{headtoc}{\color{white} FUNCTION}
dpotf2}}

\hypertarget{ecldoc:blas.dpotf2}{}
\hspace{0pt} \hyperlink{ecldoc:blas}{BLAS} \textbackslash 

{\renewcommand{\arraystretch}{1.5}
\begin{tabularx}{\textwidth}{|>{\raggedright\arraybackslash}l|X|}
\hline
\hspace{0pt}\mytexttt{\color{red} Types.matrix\_t} & \textbf{dpotf2} \\
\hline
\multicolumn{2}{|>{\raggedright\arraybackslash}X|}{\hspace{0pt}\mytexttt{\color{param} (Types.Triangle tri, Types.dimension\_t r, Types.matrix\_t A, BOOLEAN clear=TRUE)}} \\
\hline
\end{tabularx}
}

\par
DPOTF2 computes the Cholesky factorization of a real symmetric positive definite matrix A. The factorization has the form A = U**T * U , if UPLO = 'U', or A = L * L**T, if UPLO = 'L', where U is an upper triangular matrix and L is lower triangular. This is the unblocked version of the algorithm, calling Level 2 BLAS.

\par
\begin{description}
\item [\colorbox{tagtype}{\color{white} \textbf{\textsf{PARAMETER}}}] \textbf{\underline{tri}} indicate whether upper or lower triangle is used
\item [\colorbox{tagtype}{\color{white} \textbf{\textsf{PARAMETER}}}] \textbf{\underline{r}} number of rows/columns in the square matrix
\item [\colorbox{tagtype}{\color{white} \textbf{\textsf{PARAMETER}}}] \textbf{\underline{A}} the square matrix
\item [\colorbox{tagtype}{\color{white} \textbf{\textsf{PARAMETER}}}] \textbf{\underline{clear}} clears the unused triangle
\item [\colorbox{tagtype}{\color{white} \textbf{\textsf{RETURN}}}] \textbf{\underline{}} the triangular matrix requested.
\end{description}

\rule{\linewidth}{0.5pt}
\subsection*{\textsf{\colorbox{headtoc}{\color{white} FUNCTION}
dscal}}

\hypertarget{ecldoc:blas.dscal}{}
\hspace{0pt} \hyperlink{ecldoc:blas}{BLAS} \textbackslash 

{\renewcommand{\arraystretch}{1.5}
\begin{tabularx}{\textwidth}{|>{\raggedright\arraybackslash}l|X|}
\hline
\hspace{0pt}\mytexttt{\color{red} Types.matrix\_t} & \textbf{dscal} \\
\hline
\multicolumn{2}{|>{\raggedright\arraybackslash}X|}{\hspace{0pt}\mytexttt{\color{param} (Types.dimension\_t N, Types.value\_t alpha, Types.matrix\_t X, Types.dimension\_t incX, Types.dimension\_t skipped=0)}} \\
\hline
\end{tabularx}
}

\par
Scale a vector alpha

\par
\begin{description}
\item [\colorbox{tagtype}{\color{white} \textbf{\textsf{PARAMETER}}}] \textbf{\underline{N}} number of elements in the vector
\item [\colorbox{tagtype}{\color{white} \textbf{\textsf{PARAMETER}}}] \textbf{\underline{alpha}} the scaling factor
\item [\colorbox{tagtype}{\color{white} \textbf{\textsf{PARAMETER}}}] \textbf{\underline{X}} the column major matrix holding the vector
\item [\colorbox{tagtype}{\color{white} \textbf{\textsf{PARAMETER}}}] \textbf{\underline{incX}} the stride to get to the next element in the vector
\item [\colorbox{tagtype}{\color{white} \textbf{\textsf{PARAMETER}}}] \textbf{\underline{skipped}} the number of elements skipped to get to the first element
\item [\colorbox{tagtype}{\color{white} \textbf{\textsf{RETURN}}}] \textbf{\underline{}} the updated matrix
\end{description}

\rule{\linewidth}{0.5pt}
\subsection*{\textsf{\colorbox{headtoc}{\color{white} FUNCTION}
dsyrk}}

\hypertarget{ecldoc:blas.dsyrk}{}
\hspace{0pt} \hyperlink{ecldoc:blas}{BLAS} \textbackslash 

{\renewcommand{\arraystretch}{1.5}
\begin{tabularx}{\textwidth}{|>{\raggedright\arraybackslash}l|X|}
\hline
\hspace{0pt}\mytexttt{\color{red} Types.matrix\_t} & \textbf{dsyrk} \\
\hline
\multicolumn{2}{|>{\raggedright\arraybackslash}X|}{\hspace{0pt}\mytexttt{\color{param} (Types.Triangle tri, BOOLEAN transposeA, Types.dimension\_t N, Types.dimension\_t K, Types.value\_t alpha, Types.matrix\_t A, Types.value\_t beta, Types.matrix\_t C, BOOLEAN clear=FALSE)}} \\
\hline
\end{tabularx}
}

\par
Implements symmetric rank update C 

\par
\begin{description}
\item [\colorbox{tagtype}{\color{white} \textbf{\textsf{PARAMETER}}}] \textbf{\underline{tri}} update upper or lower triangle
\item [\colorbox{tagtype}{\color{white} \textbf{\textsf{PARAMETER}}}] \textbf{\underline{transposeA}} Transpose the A matrix to be NxK
\item [\colorbox{tagtype}{\color{white} \textbf{\textsf{PARAMETER}}}] \textbf{\underline{N}} number of rows
\item [\colorbox{tagtype}{\color{white} \textbf{\textsf{PARAMETER}}}] \textbf{\underline{K}} number of columns in the update matrix or transpose
\item [\colorbox{tagtype}{\color{white} \textbf{\textsf{PARAMETER}}}] \textbf{\underline{alpha}} the alpha scalar
\item [\colorbox{tagtype}{\color{white} \textbf{\textsf{PARAMETER}}}] \textbf{\underline{A}} the update matrix, either NxK or KxN
\item [\colorbox{tagtype}{\color{white} \textbf{\textsf{PARAMETER}}}] \textbf{\underline{beta}} the beta scalar
\item [\colorbox{tagtype}{\color{white} \textbf{\textsf{PARAMETER}}}] \textbf{\underline{C}} the matrix to update
\item [\colorbox{tagtype}{\color{white} \textbf{\textsf{PARAMETER}}}] \textbf{\underline{clear}} clear the triangle that is not updated. BLAS assumes that symmetric matrices have only one of the triangles and this option lets you make that true.
\end{description}

\rule{\linewidth}{0.5pt}
\subsection*{\textsf{\colorbox{headtoc}{\color{white} FUNCTION}
dtrsm}}

\hypertarget{ecldoc:blas.dtrsm}{}
\hspace{0pt} \hyperlink{ecldoc:blas}{BLAS} \textbackslash 

{\renewcommand{\arraystretch}{1.5}
\begin{tabularx}{\textwidth}{|>{\raggedright\arraybackslash}l|X|}
\hline
\hspace{0pt}\mytexttt{\color{red} Types.matrix\_t} & \textbf{dtrsm} \\
\hline
\multicolumn{2}{|>{\raggedright\arraybackslash}X|}{\hspace{0pt}\mytexttt{\color{param} (Types.Side side, Types.Triangle tri, BOOLEAN transposeA, Types.Diagonal diag, Types.dimension\_t M, Types.dimension\_t N, Types.dimension\_t lda, Types.value\_t alpha, Types.matrix\_t A, Types.matrix\_t B)}} \\
\hline
\end{tabularx}
}

\par
Triangular matrix solver. op(A) X = alpha B or X op(A) = alpha B where op is Transpose, X and B is MxN

\par
\begin{description}
\item [\colorbox{tagtype}{\color{white} \textbf{\textsf{PARAMETER}}}] \textbf{\underline{side}} side for A, Side.Ax is op(A) X = alpha B
\item [\colorbox{tagtype}{\color{white} \textbf{\textsf{PARAMETER}}}] \textbf{\underline{tri}} Says whether A is Upper or Lower triangle
\item [\colorbox{tagtype}{\color{white} \textbf{\textsf{PARAMETER}}}] \textbf{\underline{transposeA}} is op(A) the transpose of A
\item [\colorbox{tagtype}{\color{white} \textbf{\textsf{PARAMETER}}}] \textbf{\underline{diag}} is the diagonal an implied unit diagonal or supplied
\item [\colorbox{tagtype}{\color{white} \textbf{\textsf{PARAMETER}}}] \textbf{\underline{M}} number of rows
\item [\colorbox{tagtype}{\color{white} \textbf{\textsf{PARAMETER}}}] \textbf{\underline{N}} number of columns
\item [\colorbox{tagtype}{\color{white} \textbf{\textsf{PARAMETER}}}] \textbf{\underline{lda}} the leading dimension of the A matrix, either M or N
\item [\colorbox{tagtype}{\color{white} \textbf{\textsf{PARAMETER}}}] \textbf{\underline{alpha}} the scalar multiplier for B
\item [\colorbox{tagtype}{\color{white} \textbf{\textsf{PARAMETER}}}] \textbf{\underline{A}} a triangular matrix
\item [\colorbox{tagtype}{\color{white} \textbf{\textsf{PARAMETER}}}] \textbf{\underline{B}} the matrix of values for the solve
\item [\colorbox{tagtype}{\color{white} \textbf{\textsf{RETURN}}}] \textbf{\underline{}} the matrix of coefficients to get B.
\end{description}

\rule{\linewidth}{0.5pt}
\subsection*{\textsf{\colorbox{headtoc}{\color{white} FUNCTION}
extract\_diag}}

\hypertarget{ecldoc:blas.extract_diag}{}
\hspace{0pt} \hyperlink{ecldoc:blas}{BLAS} \textbackslash 

{\renewcommand{\arraystretch}{1.5}
\begin{tabularx}{\textwidth}{|>{\raggedright\arraybackslash}l|X|}
\hline
\hspace{0pt}\mytexttt{\color{red} Types.matrix\_t} & \textbf{extract\_diag} \\
\hline
\multicolumn{2}{|>{\raggedright\arraybackslash}X|}{\hspace{0pt}\mytexttt{\color{param} (Types.dimension\_t m, Types.dimension\_t n, Types.matrix\_t x)}} \\
\hline
\end{tabularx}
}

\par
Extract the diagonal of he matrix

\par
\begin{description}
\item [\colorbox{tagtype}{\color{white} \textbf{\textsf{PARAMETER}}}] \textbf{\underline{m}} number of rows
\item [\colorbox{tagtype}{\color{white} \textbf{\textsf{PARAMETER}}}] \textbf{\underline{n}} number of columns
\item [\colorbox{tagtype}{\color{white} \textbf{\textsf{PARAMETER}}}] \textbf{\underline{x}} matrix from which to extract the diagonal
\item [\colorbox{tagtype}{\color{white} \textbf{\textsf{RETURN}}}] \textbf{\underline{}} diagonal matrix
\end{description}

\rule{\linewidth}{0.5pt}
\subsection*{\textsf{\colorbox{headtoc}{\color{white} FUNCTION}
extract\_tri}}

\hypertarget{ecldoc:blas.extract_tri}{}
\hspace{0pt} \hyperlink{ecldoc:blas}{BLAS} \textbackslash 

{\renewcommand{\arraystretch}{1.5}
\begin{tabularx}{\textwidth}{|>{\raggedright\arraybackslash}l|X|}
\hline
\hspace{0pt}\mytexttt{\color{red} Types.matrix\_t} & \textbf{extract\_tri} \\
\hline
\multicolumn{2}{|>{\raggedright\arraybackslash}X|}{\hspace{0pt}\mytexttt{\color{param} (Types.dimension\_t m, Types.dimension\_t n, Types.Triangle tri, Types.Diagonal dt, Types.matrix\_t a)}} \\
\hline
\end{tabularx}
}

\par
Extract the upper or lower triangle. Diagonal can be actual or implied unit diagonal.

\par
\begin{description}
\item [\colorbox{tagtype}{\color{white} \textbf{\textsf{PARAMETER}}}] \textbf{\underline{m}} number of rows
\item [\colorbox{tagtype}{\color{white} \textbf{\textsf{PARAMETER}}}] \textbf{\underline{n}} number of columns
\item [\colorbox{tagtype}{\color{white} \textbf{\textsf{PARAMETER}}}] \textbf{\underline{tri}} Upper or Lower specifier, Triangle.Lower or Triangle.Upper
\item [\colorbox{tagtype}{\color{white} \textbf{\textsf{PARAMETER}}}] \textbf{\underline{dt}} Use Diagonal.NotUnitTri or Diagonal.UnitTri
\item [\colorbox{tagtype}{\color{white} \textbf{\textsf{PARAMETER}}}] \textbf{\underline{a}} Matrix, usually a composite from factoring
\item [\colorbox{tagtype}{\color{white} \textbf{\textsf{RETURN}}}] \textbf{\underline{}} the triangle
\end{description}

\rule{\linewidth}{0.5pt}
\subsection*{\textsf{\colorbox{headtoc}{\color{white} FUNCTION}
make\_diag}}

\hypertarget{ecldoc:blas.make_diag}{}
\hspace{0pt} \hyperlink{ecldoc:blas}{BLAS} \textbackslash 

{\renewcommand{\arraystretch}{1.5}
\begin{tabularx}{\textwidth}{|>{\raggedright\arraybackslash}l|X|}
\hline
\hspace{0pt}\mytexttt{\color{red} Types.matrix\_t} & \textbf{make\_diag} \\
\hline
\multicolumn{2}{|>{\raggedright\arraybackslash}X|}{\hspace{0pt}\mytexttt{\color{param} (Types.dimension\_t m, Types.value\_t v=1.0, Types.matrix\_t X=[])}} \\
\hline
\end{tabularx}
}

\par
Generate a diagonal matrix.

\par
\begin{description}
\item [\colorbox{tagtype}{\color{white} \textbf{\textsf{PARAMETER}}}] \textbf{\underline{m}} number of diagonal entries
\item [\colorbox{tagtype}{\color{white} \textbf{\textsf{PARAMETER}}}] \textbf{\underline{v}} option value, defaults to 1
\item [\colorbox{tagtype}{\color{white} \textbf{\textsf{PARAMETER}}}] \textbf{\underline{X}} optional input of diagonal values, multiplied by v.
\item [\colorbox{tagtype}{\color{white} \textbf{\textsf{RETURN}}}] \textbf{\underline{}} a diagonal matrix
\end{description}

\rule{\linewidth}{0.5pt}
\subsection*{\textsf{\colorbox{headtoc}{\color{white} FUNCTION}
make\_vector}}

\hypertarget{ecldoc:blas.make_vector}{}
\hspace{0pt} \hyperlink{ecldoc:blas}{BLAS} \textbackslash 

{\renewcommand{\arraystretch}{1.5}
\begin{tabularx}{\textwidth}{|>{\raggedright\arraybackslash}l|X|}
\hline
\hspace{0pt}\mytexttt{\color{red} Types.matrix\_t} & \textbf{make\_vector} \\
\hline
\multicolumn{2}{|>{\raggedright\arraybackslash}X|}{\hspace{0pt}\mytexttt{\color{param} (Types.dimension\_t m, Types.value\_t v=1.0)}} \\
\hline
\end{tabularx}
}

\par
Make a vector of dimension m

\par
\begin{description}
\item [\colorbox{tagtype}{\color{white} \textbf{\textsf{PARAMETER}}}] \textbf{\underline{m}} number of elements
\item [\colorbox{tagtype}{\color{white} \textbf{\textsf{PARAMETER}}}] \textbf{\underline{v}} the values, defaults to 1
\item [\colorbox{tagtype}{\color{white} \textbf{\textsf{RETURN}}}] \textbf{\underline{}} the vector
\end{description}

\rule{\linewidth}{0.5pt}
\subsection*{\textsf{\colorbox{headtoc}{\color{white} FUNCTION}
trace}}

\hypertarget{ecldoc:blas.trace}{}
\hspace{0pt} \hyperlink{ecldoc:blas}{BLAS} \textbackslash 

{\renewcommand{\arraystretch}{1.5}
\begin{tabularx}{\textwidth}{|>{\raggedright\arraybackslash}l|X|}
\hline
\hspace{0pt}\mytexttt{\color{red} Types.value\_t} & \textbf{trace} \\
\hline
\multicolumn{2}{|>{\raggedright\arraybackslash}X|}{\hspace{0pt}\mytexttt{\color{param} (Types.dimension\_t m, Types.dimension\_t n, Types.matrix\_t x)}} \\
\hline
\end{tabularx}
}

\par
The trace of the input matrix

\par
\begin{description}
\item [\colorbox{tagtype}{\color{white} \textbf{\textsf{PARAMETER}}}] \textbf{\underline{m}} number of rows
\item [\colorbox{tagtype}{\color{white} \textbf{\textsf{PARAMETER}}}] \textbf{\underline{n}} number of columns
\item [\colorbox{tagtype}{\color{white} \textbf{\textsf{PARAMETER}}}] \textbf{\underline{x}} the matrix
\item [\colorbox{tagtype}{\color{white} \textbf{\textsf{RETURN}}}] \textbf{\underline{}} the trace (sum of the diagonal entries)
\end{description}

\rule{\linewidth}{0.5pt}



\chapter*{\color{headfile}
BundleBase
}
\hypertarget{ecldoc:toc:BundleBase}{}
\hyperlink{ecldoc:toc:root}{Go Up}


\section*{\underline{\textsf{DESCRIPTIONS}}}
\subsection*{\textsf{\colorbox{headtoc}{\color{white} MODULE}
BundleBase}}

\hypertarget{ecldoc:BundleBase}{}

{\renewcommand{\arraystretch}{1.5}
\begin{tabularx}{\textwidth}{|>{\raggedright\arraybackslash}l|X|}
\hline
\hspace{0pt}\mytexttt{\color{red} } & \textbf{BundleBase} \\
\hline
\end{tabularx}
}

\par





No Documentation Found







\textbf{Children}
\begin{enumerate}
\item \hyperlink{ecldoc:bundlebase.propertyrecord}{PropertyRecord}
: No Documentation Found
\item \hyperlink{ecldoc:bundlebase.name}{Name}
: No Documentation Found
\item \hyperlink{ecldoc:bundlebase.description}{Description}
: No Documentation Found
\item \hyperlink{ecldoc:bundlebase.authors}{Authors}
: No Documentation Found
\item \hyperlink{ecldoc:bundlebase.license}{License}
: No Documentation Found
\item \hyperlink{ecldoc:bundlebase.copyright}{Copyright}
: No Documentation Found
\item \hyperlink{ecldoc:bundlebase.dependson}{DependsOn}
: No Documentation Found
\item \hyperlink{ecldoc:bundlebase.version}{Version}
: No Documentation Found
\item \hyperlink{ecldoc:bundlebase.properties}{Properties}
: No Documentation Found
\item \hyperlink{ecldoc:bundlebase.platformversion}{PlatformVersion}
: No Documentation Found
\end{enumerate}

\rule{\linewidth}{0.5pt}

\subsection*{\textsf{\colorbox{headtoc}{\color{white} RECORD}
PropertyRecord}}

\hypertarget{ecldoc:bundlebase.propertyrecord}{}
\hspace{0pt} \hyperlink{ecldoc:BundleBase}{BundleBase} \textbackslash 

{\renewcommand{\arraystretch}{1.5}
\begin{tabularx}{\textwidth}{|>{\raggedright\arraybackslash}l|X|}
\hline
\hspace{0pt}\mytexttt{\color{red} } & \textbf{PropertyRecord} \\
\hline
\end{tabularx}
}

\par





No Documentation Found







\par
\begin{description}
\item [\colorbox{tagtype}{\color{white} \textbf{\textsf{FIELD}}}] \textbf{\underline{value}} ||| UTF8 --- No Doc
\item [\colorbox{tagtype}{\color{white} \textbf{\textsf{FIELD}}}] \textbf{\underline{key}} ||| UTF8 --- No Doc
\end{description}





\rule{\linewidth}{0.5pt}
\subsection*{\textsf{\colorbox{headtoc}{\color{white} ATTRIBUTE}
Name}}

\hypertarget{ecldoc:bundlebase.name}{}
\hspace{0pt} \hyperlink{ecldoc:BundleBase}{BundleBase} \textbackslash 

{\renewcommand{\arraystretch}{1.5}
\begin{tabularx}{\textwidth}{|>{\raggedright\arraybackslash}l|X|}
\hline
\hspace{0pt}\mytexttt{\color{red} STRING} & \textbf{Name} \\
\hline
\end{tabularx}
}

\par





No Documentation Found








\par
\begin{description}
\item [\colorbox{tagtype}{\color{white} \textbf{\textsf{RETURN}}}] \textbf{STRING} --- 
\end{description}




\rule{\linewidth}{0.5pt}
\subsection*{\textsf{\colorbox{headtoc}{\color{white} ATTRIBUTE}
Description}}

\hypertarget{ecldoc:bundlebase.description}{}
\hspace{0pt} \hyperlink{ecldoc:BundleBase}{BundleBase} \textbackslash 

{\renewcommand{\arraystretch}{1.5}
\begin{tabularx}{\textwidth}{|>{\raggedright\arraybackslash}l|X|}
\hline
\hspace{0pt}\mytexttt{\color{red} UTF8} & \textbf{Description} \\
\hline
\end{tabularx}
}

\par





No Documentation Found








\par
\begin{description}
\item [\colorbox{tagtype}{\color{white} \textbf{\textsf{RETURN}}}] \textbf{UTF8} --- 
\end{description}




\rule{\linewidth}{0.5pt}
\subsection*{\textsf{\colorbox{headtoc}{\color{white} ATTRIBUTE}
Authors}}

\hypertarget{ecldoc:bundlebase.authors}{}
\hspace{0pt} \hyperlink{ecldoc:BundleBase}{BundleBase} \textbackslash 

{\renewcommand{\arraystretch}{1.5}
\begin{tabularx}{\textwidth}{|>{\raggedright\arraybackslash}l|X|}
\hline
\hspace{0pt}\mytexttt{\color{red} SET OF UTF8} & \textbf{Authors} \\
\hline
\end{tabularx}
}

\par





No Documentation Found








\par
\begin{description}
\item [\colorbox{tagtype}{\color{white} \textbf{\textsf{RETURN}}}] \textbf{SET ( UTF8 )} --- 
\end{description}




\rule{\linewidth}{0.5pt}
\subsection*{\textsf{\colorbox{headtoc}{\color{white} ATTRIBUTE}
License}}

\hypertarget{ecldoc:bundlebase.license}{}
\hspace{0pt} \hyperlink{ecldoc:BundleBase}{BundleBase} \textbackslash 

{\renewcommand{\arraystretch}{1.5}
\begin{tabularx}{\textwidth}{|>{\raggedright\arraybackslash}l|X|}
\hline
\hspace{0pt}\mytexttt{\color{red} UTF8} & \textbf{License} \\
\hline
\end{tabularx}
}

\par





No Documentation Found








\par
\begin{description}
\item [\colorbox{tagtype}{\color{white} \textbf{\textsf{RETURN}}}] \textbf{UTF8} --- 
\end{description}




\rule{\linewidth}{0.5pt}
\subsection*{\textsf{\colorbox{headtoc}{\color{white} ATTRIBUTE}
Copyright}}

\hypertarget{ecldoc:bundlebase.copyright}{}
\hspace{0pt} \hyperlink{ecldoc:BundleBase}{BundleBase} \textbackslash 

{\renewcommand{\arraystretch}{1.5}
\begin{tabularx}{\textwidth}{|>{\raggedright\arraybackslash}l|X|}
\hline
\hspace{0pt}\mytexttt{\color{red} UTF8} & \textbf{Copyright} \\
\hline
\end{tabularx}
}

\par





No Documentation Found








\par
\begin{description}
\item [\colorbox{tagtype}{\color{white} \textbf{\textsf{RETURN}}}] \textbf{UTF8} --- 
\end{description}




\rule{\linewidth}{0.5pt}
\subsection*{\textsf{\colorbox{headtoc}{\color{white} ATTRIBUTE}
DependsOn}}

\hypertarget{ecldoc:bundlebase.dependson}{}
\hspace{0pt} \hyperlink{ecldoc:BundleBase}{BundleBase} \textbackslash 

{\renewcommand{\arraystretch}{1.5}
\begin{tabularx}{\textwidth}{|>{\raggedright\arraybackslash}l|X|}
\hline
\hspace{0pt}\mytexttt{\color{red} SET OF STRING} & \textbf{DependsOn} \\
\hline
\end{tabularx}
}

\par





No Documentation Found








\par
\begin{description}
\item [\colorbox{tagtype}{\color{white} \textbf{\textsf{RETURN}}}] \textbf{SET ( STRING )} --- 
\end{description}




\rule{\linewidth}{0.5pt}
\subsection*{\textsf{\colorbox{headtoc}{\color{white} ATTRIBUTE}
Version}}

\hypertarget{ecldoc:bundlebase.version}{}
\hspace{0pt} \hyperlink{ecldoc:BundleBase}{BundleBase} \textbackslash 

{\renewcommand{\arraystretch}{1.5}
\begin{tabularx}{\textwidth}{|>{\raggedright\arraybackslash}l|X|}
\hline
\hspace{0pt}\mytexttt{\color{red} STRING} & \textbf{Version} \\
\hline
\end{tabularx}
}

\par





No Documentation Found








\par
\begin{description}
\item [\colorbox{tagtype}{\color{white} \textbf{\textsf{RETURN}}}] \textbf{STRING} --- 
\end{description}




\rule{\linewidth}{0.5pt}
\subsection*{\textsf{\colorbox{headtoc}{\color{white} ATTRIBUTE}
Properties}}

\hypertarget{ecldoc:bundlebase.properties}{}
\hspace{0pt} \hyperlink{ecldoc:BundleBase}{BundleBase} \textbackslash 

{\renewcommand{\arraystretch}{1.5}
\begin{tabularx}{\textwidth}{|>{\raggedright\arraybackslash}l|X|}
\hline
\hspace{0pt}\mytexttt{\color{red} } & \textbf{Properties} \\
\hline
\end{tabularx}
}

\par





No Documentation Found








\par
\begin{description}
\item [\colorbox{tagtype}{\color{white} \textbf{\textsf{RETURN}}}] \textbf{DICTIONARY ( PropertyRecord )} --- 
\end{description}




\rule{\linewidth}{0.5pt}
\subsection*{\textsf{\colorbox{headtoc}{\color{white} ATTRIBUTE}
PlatformVersion}}

\hypertarget{ecldoc:bundlebase.platformversion}{}
\hspace{0pt} \hyperlink{ecldoc:BundleBase}{BundleBase} \textbackslash 

{\renewcommand{\arraystretch}{1.5}
\begin{tabularx}{\textwidth}{|>{\raggedright\arraybackslash}l|X|}
\hline
\hspace{0pt}\mytexttt{\color{red} STRING} & \textbf{PlatformVersion} \\
\hline
\end{tabularx}
}

\par





No Documentation Found








\par
\begin{description}
\item [\colorbox{tagtype}{\color{white} \textbf{\textsf{RETURN}}}] \textbf{STRING} --- 
\end{description}




\rule{\linewidth}{0.5pt}



\chapter*{\color{headfile}
Date
}
\hypertarget{ecldoc:toc:Date}{}
\hyperlink{ecldoc:toc:root}{Go Up}

\section*{\underline{\textsf{IMPORTS}}}
\begin{doublespace}
{\large
}
\end{doublespace}

\section*{\underline{\textsf{DESCRIPTIONS}}}
\subsection*{\textsf{\colorbox{headtoc}{\color{white} MODULE}
Date}}

\hypertarget{ecldoc:Date}{}

{\renewcommand{\arraystretch}{1.5}
\begin{tabularx}{\textwidth}{|>{\raggedright\arraybackslash}l|X|}
\hline
\hspace{0pt}\mytexttt{\color{red} } & \textbf{Date} \\
\hline
\end{tabularx}
}

\par





No Documentation Found







\textbf{Children}
\begin{enumerate}
\item \hyperlink{ecldoc:date.date_rec}{Date\_rec}
: No Documentation Found
\item \hyperlink{ecldoc:date.date_t}{Date\_t}
: No Documentation Found
\item \hyperlink{ecldoc:date.days_t}{Days\_t}
: No Documentation Found
\item \hyperlink{ecldoc:date.time_rec}{Time\_rec}
: No Documentation Found
\item \hyperlink{ecldoc:date.time_t}{Time\_t}
: No Documentation Found
\item \hyperlink{ecldoc:date.seconds_t}{Seconds\_t}
: No Documentation Found
\item \hyperlink{ecldoc:date.datetime_rec}{DateTime\_rec}
: No Documentation Found
\item \hyperlink{ecldoc:date.timestamp_t}{Timestamp\_t}
: No Documentation Found
\item \hyperlink{ecldoc:date.year}{Year}
: Extracts the year from a date type
\item \hyperlink{ecldoc:date.month}{Month}
: Extracts the month from a date type
\item \hyperlink{ecldoc:date.day}{Day}
: Extracts the day of the month from a date type
\item \hyperlink{ecldoc:date.hour}{Hour}
: Extracts the hour from a time type
\item \hyperlink{ecldoc:date.minute}{Minute}
: Extracts the minutes from a time type
\item \hyperlink{ecldoc:date.second}{Second}
: Extracts the seconds from a time type
\item \hyperlink{ecldoc:date.datefromparts}{DateFromParts}
: Combines year, month day to create a date type
\item \hyperlink{ecldoc:date.timefromparts}{TimeFromParts}
: Combines hour, minute second to create a time type
\item \hyperlink{ecldoc:date.secondsfromparts}{SecondsFromParts}
: Combines date and time components to create a seconds type
\item \hyperlink{ecldoc:date.secondstoparts}{SecondsToParts}
: Converts the number of seconds since epoch to a structure containing date and time parts
\item \hyperlink{ecldoc:date.timestamptoseconds}{TimestampToSeconds}
: Converts the number of microseconds since epoch to the number of seconds since epoch
\item \hyperlink{ecldoc:date.isleapyear}{IsLeapYear}
: Tests whether the year is a leap year in the Gregorian calendar
\item \hyperlink{ecldoc:date.isdateleapyear}{IsDateLeapYear}
: Tests whether a date is a leap year in the Gregorian calendar
\item \hyperlink{ecldoc:date.fromgregorianymd}{FromGregorianYMD}
: Combines year, month, day in the Gregorian calendar to create the number days since 31st December 1BC
\item \hyperlink{ecldoc:date.togregorianymd}{ToGregorianYMD}
: Converts the number days since 31st December 1BC to a date in the Gregorian calendar
\item \hyperlink{ecldoc:date.fromgregoriandate}{FromGregorianDate}
: Converts a date in the Gregorian calendar to the number days since 31st December 1BC
\item \hyperlink{ecldoc:date.togregoriandate}{ToGregorianDate}
: Converts the number days since 31st December 1BC to a date in the Gregorian calendar
\item \hyperlink{ecldoc:date.dayofyear}{DayOfYear}
: Returns a number representing the day of the year indicated by the given date
\item \hyperlink{ecldoc:date.dayofweek}{DayOfWeek}
: Returns a number representing the day of the week indicated by the given date
\item \hyperlink{ecldoc:date.isjulianleapyear}{IsJulianLeapYear}
: Tests whether the year is a leap year in the Julian calendar
\item \hyperlink{ecldoc:date.fromjulianymd}{FromJulianYMD}
: Combines year, month, day in the Julian calendar to create the number days since 31st December 1BC
\item \hyperlink{ecldoc:date.tojulianymd}{ToJulianYMD}
: Converts the number days since 31st December 1BC to a date in the Julian calendar
\item \hyperlink{ecldoc:date.fromjuliandate}{FromJulianDate}
: Converts a date in the Julian calendar to the number days since 31st December 1BC
\item \hyperlink{ecldoc:date.tojuliandate}{ToJulianDate}
: Converts the number days since 31st December 1BC to a date in the Julian calendar
\item \hyperlink{ecldoc:date.dayssince1900}{DaysSince1900}
: Returns the number of days since 1st January 1900 (using the Gregorian Calendar)
\item \hyperlink{ecldoc:date.todayssince1900}{ToDaysSince1900}
: Returns the number of days since 1st January 1900 (using the Gregorian Calendar)
\item \hyperlink{ecldoc:date.fromdayssince1900}{FromDaysSince1900}
: Converts the number days since 1st January 1900 to a date in the Julian calendar
\item \hyperlink{ecldoc:date.yearsbetween}{YearsBetween}
: Calculate the number of whole years between two dates
\item \hyperlink{ecldoc:date.monthsbetween}{MonthsBetween}
: Calculate the number of whole months between two dates
\item \hyperlink{ecldoc:date.daysbetween}{DaysBetween}
: Calculate the number of days between two dates
\item \hyperlink{ecldoc:date.datefromdaterec}{DateFromDateRec}
: Combines the fields from a Date\_rec to create a Date\_t
\item \hyperlink{ecldoc:date.datefromrec}{DateFromRec}
: Combines the fields from a Date\_rec to create a Date\_t
\item \hyperlink{ecldoc:date.timefromtimerec}{TimeFromTimeRec}
: Combines the fields from a Time\_rec to create a Time\_t
\item \hyperlink{ecldoc:date.datefromdatetimerec}{DateFromDateTimeRec}
: Combines the date fields from a DateTime\_rec to create a Date\_t
\item \hyperlink{ecldoc:date.timefromdatetimerec}{TimeFromDateTimeRec}
: Combines the time fields from a DateTime\_rec to create a Time\_t
\item \hyperlink{ecldoc:date.secondsfromdatetimerec}{SecondsFromDateTimeRec}
: Combines the date and time fields from a DateTime\_rec to create a Seconds\_t
\item \hyperlink{ecldoc:date.fromstringtodate}{FromStringToDate}
: Converts a string to a Date\_t using the relevant string format
\item \hyperlink{ecldoc:date.fromstring}{FromString}
: Converts a string to a date using the relevant string format
\item \hyperlink{ecldoc:date.fromstringtotime}{FromStringToTime}
: Converts a string to a Time\_t using the relevant string format
\item \hyperlink{ecldoc:date.matchdatestring}{MatchDateString}
: Matches a string against a set of date string formats and returns a valid Date\_t object from the first format that successfully parses the string
\item \hyperlink{ecldoc:date.matchtimestring}{MatchTimeString}
: Matches a string against a set of time string formats and returns a valid Time\_t object from the first format that successfully parses the string
\item \hyperlink{ecldoc:date.datetostring}{DateToString}
: Formats a date as a string
\item \hyperlink{ecldoc:date.timetostring}{TimeToString}
: Formats a time as a string
\item \hyperlink{ecldoc:date.secondstostring}{SecondsToString}
: Converts a Seconds\_t value into a human-readable string using a format template
\item \hyperlink{ecldoc:date.tostring}{ToString}
: Formats a date as a string
\item \hyperlink{ecldoc:date.convertdateformat}{ConvertDateFormat}
: Converts a date from one format to another
\item \hyperlink{ecldoc:date.convertformat}{ConvertFormat}
: Converts a date from one format to another
\item \hyperlink{ecldoc:date.converttimeformat}{ConvertTimeFormat}
: Converts a time from one format to another
\item \hyperlink{ecldoc:date.convertdateformatmultiple}{ConvertDateFormatMultiple}
: Converts a date that matches one of a set of formats to another
\item \hyperlink{ecldoc:date.convertformatmultiple}{ConvertFormatMultiple}
: Converts a date that matches one of a set of formats to another
\item \hyperlink{ecldoc:date.converttimeformatmultiple}{ConvertTimeFormatMultiple}
: Converts a time that matches one of a set of formats to another
\item \hyperlink{ecldoc:date.adjustdate}{AdjustDate}
: Adjusts a date by incrementing or decrementing year, month and/or day values
\item \hyperlink{ecldoc:date.adjustdatebyseconds}{AdjustDateBySeconds}
: Adjusts a date by adding or subtracting seconds
\item \hyperlink{ecldoc:date.adjusttime}{AdjustTime}
: Adjusts a time by incrementing or decrementing hour, minute and/or second values
\item \hyperlink{ecldoc:date.adjusttimebyseconds}{AdjustTimeBySeconds}
: Adjusts a time by adding or subtracting seconds
\item \hyperlink{ecldoc:date.adjustseconds}{AdjustSeconds}
: Adjusts a Seconds\_t value by adding or subtracting years, months, days, hours, minutes and/or seconds
\item \hyperlink{ecldoc:date.adjustcalendar}{AdjustCalendar}
: Adjusts a date by incrementing or decrementing months and/or years
\item \hyperlink{ecldoc:date.islocaldaylightsavingsineffect}{IsLocalDaylightSavingsInEffect}
: Returns a boolean indicating whether daylight savings time is currently in effect locally
\item \hyperlink{ecldoc:date.localtimezoneoffset}{LocalTimeZoneOffset}
: Returns the offset (in seconds) of the time represented from UTC, with positive values indicating locations east of the Prime Meridian
\item \hyperlink{ecldoc:date.currentdate}{CurrentDate}
: Returns the current date
\item \hyperlink{ecldoc:date.today}{Today}
: Returns the current date in the local time zone
\item \hyperlink{ecldoc:date.currenttime}{CurrentTime}
: Returns the current time of day
\item \hyperlink{ecldoc:date.currentseconds}{CurrentSeconds}
: Returns the current date and time as the number of seconds since epoch
\item \hyperlink{ecldoc:date.currenttimestamp}{CurrentTimestamp}
: Returns the current date and time as the number of microseconds since epoch
\item \hyperlink{ecldoc:date.datesformonth}{DatesForMonth}
: Returns the beginning and ending dates for the month surrounding the given date
\item \hyperlink{ecldoc:date.datesforweek}{DatesForWeek}
: Returns the beginning and ending dates for the week surrounding the given date (Sunday marks the beginning of a week)
\item \hyperlink{ecldoc:date.isvaliddate}{IsValidDate}
: Tests whether a date is valid, both by range-checking the year and by validating each of the other individual components
\item \hyperlink{ecldoc:date.isvalidgregoriandate}{IsValidGregorianDate}
: Tests whether a date is valid in the Gregorian calendar
\item \hyperlink{ecldoc:date.isvalidtime}{IsValidTime}
: Tests whether a time is valid
\item \hyperlink{ecldoc:date.createdate}{CreateDate}
: A transform to create a Date\_rec from the individual elements
\item \hyperlink{ecldoc:date.createdatefromseconds}{CreateDateFromSeconds}
: A transform to create a Date\_rec from a Seconds\_t value
\item \hyperlink{ecldoc:date.createtime}{CreateTime}
: A transform to create a Time\_rec from the individual elements
\item \hyperlink{ecldoc:date.createtimefromseconds}{CreateTimeFromSeconds}
: A transform to create a Time\_rec from a Seconds\_t value
\item \hyperlink{ecldoc:date.createdatetime}{CreateDateTime}
: A transform to create a DateTime\_rec from the individual elements
\item \hyperlink{ecldoc:date.createdatetimefromseconds}{CreateDateTimeFromSeconds}
: A transform to create a DateTime\_rec from a Seconds\_t value
\end{enumerate}

\rule{\linewidth}{0.5pt}

\subsection*{\textsf{\colorbox{headtoc}{\color{white} RECORD}
Date\_rec}}

\hypertarget{ecldoc:date.date_rec}{}
\hspace{0pt} \hyperlink{ecldoc:Date}{Date} \textbackslash 

{\renewcommand{\arraystretch}{1.5}
\begin{tabularx}{\textwidth}{|>{\raggedright\arraybackslash}l|X|}
\hline
\hspace{0pt}\mytexttt{\color{red} } & \textbf{Date\_rec} \\
\hline
\end{tabularx}
}

\par





No Documentation Found







\par
\begin{description}
\item [\colorbox{tagtype}{\color{white} \textbf{\textsf{FIELD}}}] \textbf{\underline{year}} ||| INTEGER2 --- No Doc
\item [\colorbox{tagtype}{\color{white} \textbf{\textsf{FIELD}}}] \textbf{\underline{month}} ||| UNSIGNED1 --- No Doc
\item [\colorbox{tagtype}{\color{white} \textbf{\textsf{FIELD}}}] \textbf{\underline{day}} ||| UNSIGNED1 --- No Doc
\end{description}





\rule{\linewidth}{0.5pt}
\subsection*{\textsf{\colorbox{headtoc}{\color{white} ATTRIBUTE}
Date\_t}}

\hypertarget{ecldoc:date.date_t}{}
\hspace{0pt} \hyperlink{ecldoc:Date}{Date} \textbackslash 

{\renewcommand{\arraystretch}{1.5}
\begin{tabularx}{\textwidth}{|>{\raggedright\arraybackslash}l|X|}
\hline
\hspace{0pt}\mytexttt{\color{red} } & \textbf{Date\_t} \\
\hline
\end{tabularx}
}

\par





No Documentation Found








\par
\begin{description}
\item [\colorbox{tagtype}{\color{white} \textbf{\textsf{RETURN}}}] \textbf{UNSIGNED4} --- 
\end{description}




\rule{\linewidth}{0.5pt}
\subsection*{\textsf{\colorbox{headtoc}{\color{white} ATTRIBUTE}
Days\_t}}

\hypertarget{ecldoc:date.days_t}{}
\hspace{0pt} \hyperlink{ecldoc:Date}{Date} \textbackslash 

{\renewcommand{\arraystretch}{1.5}
\begin{tabularx}{\textwidth}{|>{\raggedright\arraybackslash}l|X|}
\hline
\hspace{0pt}\mytexttt{\color{red} } & \textbf{Days\_t} \\
\hline
\end{tabularx}
}

\par





No Documentation Found








\par
\begin{description}
\item [\colorbox{tagtype}{\color{white} \textbf{\textsf{RETURN}}}] \textbf{INTEGER4} --- 
\end{description}




\rule{\linewidth}{0.5pt}
\subsection*{\textsf{\colorbox{headtoc}{\color{white} RECORD}
Time\_rec}}

\hypertarget{ecldoc:date.time_rec}{}
\hspace{0pt} \hyperlink{ecldoc:Date}{Date} \textbackslash 

{\renewcommand{\arraystretch}{1.5}
\begin{tabularx}{\textwidth}{|>{\raggedright\arraybackslash}l|X|}
\hline
\hspace{0pt}\mytexttt{\color{red} } & \textbf{Time\_rec} \\
\hline
\end{tabularx}
}

\par





No Documentation Found







\par
\begin{description}
\item [\colorbox{tagtype}{\color{white} \textbf{\textsf{FIELD}}}] \textbf{\underline{minute}} ||| UNSIGNED1 --- No Doc
\item [\colorbox{tagtype}{\color{white} \textbf{\textsf{FIELD}}}] \textbf{\underline{second}} ||| UNSIGNED1 --- No Doc
\item [\colorbox{tagtype}{\color{white} \textbf{\textsf{FIELD}}}] \textbf{\underline{hour}} ||| UNSIGNED1 --- No Doc
\end{description}





\rule{\linewidth}{0.5pt}
\subsection*{\textsf{\colorbox{headtoc}{\color{white} ATTRIBUTE}
Time\_t}}

\hypertarget{ecldoc:date.time_t}{}
\hspace{0pt} \hyperlink{ecldoc:Date}{Date} \textbackslash 

{\renewcommand{\arraystretch}{1.5}
\begin{tabularx}{\textwidth}{|>{\raggedright\arraybackslash}l|X|}
\hline
\hspace{0pt}\mytexttt{\color{red} } & \textbf{Time\_t} \\
\hline
\end{tabularx}
}

\par





No Documentation Found








\par
\begin{description}
\item [\colorbox{tagtype}{\color{white} \textbf{\textsf{RETURN}}}] \textbf{UNSIGNED3} --- 
\end{description}




\rule{\linewidth}{0.5pt}
\subsection*{\textsf{\colorbox{headtoc}{\color{white} ATTRIBUTE}
Seconds\_t}}

\hypertarget{ecldoc:date.seconds_t}{}
\hspace{0pt} \hyperlink{ecldoc:Date}{Date} \textbackslash 

{\renewcommand{\arraystretch}{1.5}
\begin{tabularx}{\textwidth}{|>{\raggedright\arraybackslash}l|X|}
\hline
\hspace{0pt}\mytexttt{\color{red} } & \textbf{Seconds\_t} \\
\hline
\end{tabularx}
}

\par





No Documentation Found








\par
\begin{description}
\item [\colorbox{tagtype}{\color{white} \textbf{\textsf{RETURN}}}] \textbf{INTEGER8} --- 
\end{description}




\rule{\linewidth}{0.5pt}
\subsection*{\textsf{\colorbox{headtoc}{\color{white} RECORD}
DateTime\_rec}}

\hypertarget{ecldoc:date.datetime_rec}{}
\hspace{0pt} \hyperlink{ecldoc:Date}{Date} \textbackslash 

{\renewcommand{\arraystretch}{1.5}
\begin{tabularx}{\textwidth}{|>{\raggedright\arraybackslash}l|X|}
\hline
\hspace{0pt}\mytexttt{\color{red} } & \textbf{DateTime\_rec} \\
\hline
\end{tabularx}
}

\par





No Documentation Found







\par
\begin{description}
\item [\colorbox{tagtype}{\color{white} \textbf{\textsf{FIELD}}}] \textbf{\underline{year}} ||| INTEGER2 --- No Doc
\item [\colorbox{tagtype}{\color{white} \textbf{\textsf{FIELD}}}] \textbf{\underline{second}} ||| UNSIGNED1 --- No Doc
\item [\colorbox{tagtype}{\color{white} \textbf{\textsf{FIELD}}}] \textbf{\underline{hour}} ||| UNSIGNED1 --- No Doc
\item [\colorbox{tagtype}{\color{white} \textbf{\textsf{FIELD}}}] \textbf{\underline{minute}} ||| UNSIGNED1 --- No Doc
\item [\colorbox{tagtype}{\color{white} \textbf{\textsf{FIELD}}}] \textbf{\underline{month}} ||| UNSIGNED1 --- No Doc
\item [\colorbox{tagtype}{\color{white} \textbf{\textsf{FIELD}}}] \textbf{\underline{day}} ||| UNSIGNED1 --- No Doc
\end{description}





\rule{\linewidth}{0.5pt}
\subsection*{\textsf{\colorbox{headtoc}{\color{white} ATTRIBUTE}
Timestamp\_t}}

\hypertarget{ecldoc:date.timestamp_t}{}
\hspace{0pt} \hyperlink{ecldoc:Date}{Date} \textbackslash 

{\renewcommand{\arraystretch}{1.5}
\begin{tabularx}{\textwidth}{|>{\raggedright\arraybackslash}l|X|}
\hline
\hspace{0pt}\mytexttt{\color{red} } & \textbf{Timestamp\_t} \\
\hline
\end{tabularx}
}

\par





No Documentation Found








\par
\begin{description}
\item [\colorbox{tagtype}{\color{white} \textbf{\textsf{RETURN}}}] \textbf{INTEGER8} --- 
\end{description}




\rule{\linewidth}{0.5pt}
\subsection*{\textsf{\colorbox{headtoc}{\color{white} FUNCTION}
Year}}

\hypertarget{ecldoc:date.year}{}
\hspace{0pt} \hyperlink{ecldoc:Date}{Date} \textbackslash 

{\renewcommand{\arraystretch}{1.5}
\begin{tabularx}{\textwidth}{|>{\raggedright\arraybackslash}l|X|}
\hline
\hspace{0pt}\mytexttt{\color{red} INTEGER2} & \textbf{Year} \\
\hline
\multicolumn{2}{|>{\raggedright\arraybackslash}X|}{\hspace{0pt}\mytexttt{\color{param} (Date\_t date)}} \\
\hline
\end{tabularx}
}

\par





Extracts the year from a date type.






\par
\begin{description}
\item [\colorbox{tagtype}{\color{white} \textbf{\textsf{PARAMETER}}}] \textbf{\underline{date}} ||| UNSIGNED4 --- The date.
\end{description}







\par
\begin{description}
\item [\colorbox{tagtype}{\color{white} \textbf{\textsf{RETURN}}}] \textbf{INTEGER2} --- An integer representing the year.
\end{description}




\rule{\linewidth}{0.5pt}
\subsection*{\textsf{\colorbox{headtoc}{\color{white} FUNCTION}
Month}}

\hypertarget{ecldoc:date.month}{}
\hspace{0pt} \hyperlink{ecldoc:Date}{Date} \textbackslash 

{\renewcommand{\arraystretch}{1.5}
\begin{tabularx}{\textwidth}{|>{\raggedright\arraybackslash}l|X|}
\hline
\hspace{0pt}\mytexttt{\color{red} UNSIGNED1} & \textbf{Month} \\
\hline
\multicolumn{2}{|>{\raggedright\arraybackslash}X|}{\hspace{0pt}\mytexttt{\color{param} (Date\_t date)}} \\
\hline
\end{tabularx}
}

\par





Extracts the month from a date type.






\par
\begin{description}
\item [\colorbox{tagtype}{\color{white} \textbf{\textsf{PARAMETER}}}] \textbf{\underline{date}} ||| UNSIGNED4 --- The date.
\end{description}







\par
\begin{description}
\item [\colorbox{tagtype}{\color{white} \textbf{\textsf{RETURN}}}] \textbf{UNSIGNED1} --- An integer representing the year.
\end{description}




\rule{\linewidth}{0.5pt}
\subsection*{\textsf{\colorbox{headtoc}{\color{white} FUNCTION}
Day}}

\hypertarget{ecldoc:date.day}{}
\hspace{0pt} \hyperlink{ecldoc:Date}{Date} \textbackslash 

{\renewcommand{\arraystretch}{1.5}
\begin{tabularx}{\textwidth}{|>{\raggedright\arraybackslash}l|X|}
\hline
\hspace{0pt}\mytexttt{\color{red} UNSIGNED1} & \textbf{Day} \\
\hline
\multicolumn{2}{|>{\raggedright\arraybackslash}X|}{\hspace{0pt}\mytexttt{\color{param} (Date\_t date)}} \\
\hline
\end{tabularx}
}

\par





Extracts the day of the month from a date type.






\par
\begin{description}
\item [\colorbox{tagtype}{\color{white} \textbf{\textsf{PARAMETER}}}] \textbf{\underline{date}} ||| UNSIGNED4 --- The date.
\end{description}







\par
\begin{description}
\item [\colorbox{tagtype}{\color{white} \textbf{\textsf{RETURN}}}] \textbf{UNSIGNED1} --- An integer representing the year.
\end{description}




\rule{\linewidth}{0.5pt}
\subsection*{\textsf{\colorbox{headtoc}{\color{white} FUNCTION}
Hour}}

\hypertarget{ecldoc:date.hour}{}
\hspace{0pt} \hyperlink{ecldoc:Date}{Date} \textbackslash 

{\renewcommand{\arraystretch}{1.5}
\begin{tabularx}{\textwidth}{|>{\raggedright\arraybackslash}l|X|}
\hline
\hspace{0pt}\mytexttt{\color{red} UNSIGNED1} & \textbf{Hour} \\
\hline
\multicolumn{2}{|>{\raggedright\arraybackslash}X|}{\hspace{0pt}\mytexttt{\color{param} (Time\_t time)}} \\
\hline
\end{tabularx}
}

\par





Extracts the hour from a time type.






\par
\begin{description}
\item [\colorbox{tagtype}{\color{white} \textbf{\textsf{PARAMETER}}}] \textbf{\underline{time}} ||| UNSIGNED3 --- The time.
\end{description}







\par
\begin{description}
\item [\colorbox{tagtype}{\color{white} \textbf{\textsf{RETURN}}}] \textbf{UNSIGNED1} --- An integer representing the hour.
\end{description}




\rule{\linewidth}{0.5pt}
\subsection*{\textsf{\colorbox{headtoc}{\color{white} FUNCTION}
Minute}}

\hypertarget{ecldoc:date.minute}{}
\hspace{0pt} \hyperlink{ecldoc:Date}{Date} \textbackslash 

{\renewcommand{\arraystretch}{1.5}
\begin{tabularx}{\textwidth}{|>{\raggedright\arraybackslash}l|X|}
\hline
\hspace{0pt}\mytexttt{\color{red} UNSIGNED1} & \textbf{Minute} \\
\hline
\multicolumn{2}{|>{\raggedright\arraybackslash}X|}{\hspace{0pt}\mytexttt{\color{param} (Time\_t time)}} \\
\hline
\end{tabularx}
}

\par





Extracts the minutes from a time type.






\par
\begin{description}
\item [\colorbox{tagtype}{\color{white} \textbf{\textsf{PARAMETER}}}] \textbf{\underline{time}} ||| UNSIGNED3 --- The time.
\end{description}







\par
\begin{description}
\item [\colorbox{tagtype}{\color{white} \textbf{\textsf{RETURN}}}] \textbf{UNSIGNED1} --- An integer representing the minutes.
\end{description}




\rule{\linewidth}{0.5pt}
\subsection*{\textsf{\colorbox{headtoc}{\color{white} FUNCTION}
Second}}

\hypertarget{ecldoc:date.second}{}
\hspace{0pt} \hyperlink{ecldoc:Date}{Date} \textbackslash 

{\renewcommand{\arraystretch}{1.5}
\begin{tabularx}{\textwidth}{|>{\raggedright\arraybackslash}l|X|}
\hline
\hspace{0pt}\mytexttt{\color{red} UNSIGNED1} & \textbf{Second} \\
\hline
\multicolumn{2}{|>{\raggedright\arraybackslash}X|}{\hspace{0pt}\mytexttt{\color{param} (Time\_t time)}} \\
\hline
\end{tabularx}
}

\par





Extracts the seconds from a time type.






\par
\begin{description}
\item [\colorbox{tagtype}{\color{white} \textbf{\textsf{PARAMETER}}}] \textbf{\underline{time}} ||| UNSIGNED3 --- The time.
\end{description}







\par
\begin{description}
\item [\colorbox{tagtype}{\color{white} \textbf{\textsf{RETURN}}}] \textbf{UNSIGNED1} --- An integer representing the seconds.
\end{description}




\rule{\linewidth}{0.5pt}
\subsection*{\textsf{\colorbox{headtoc}{\color{white} FUNCTION}
DateFromParts}}

\hypertarget{ecldoc:date.datefromparts}{}
\hspace{0pt} \hyperlink{ecldoc:Date}{Date} \textbackslash 

{\renewcommand{\arraystretch}{1.5}
\begin{tabularx}{\textwidth}{|>{\raggedright\arraybackslash}l|X|}
\hline
\hspace{0pt}\mytexttt{\color{red} Date\_t} & \textbf{DateFromParts} \\
\hline
\multicolumn{2}{|>{\raggedright\arraybackslash}X|}{\hspace{0pt}\mytexttt{\color{param} (INTEGER2 year, UNSIGNED1 month, UNSIGNED1 day)}} \\
\hline
\end{tabularx}
}

\par





Combines year, month day to create a date type.






\par
\begin{description}
\item [\colorbox{tagtype}{\color{white} \textbf{\textsf{PARAMETER}}}] \textbf{\underline{year}} ||| INTEGER2 --- The year (0-9999).
\item [\colorbox{tagtype}{\color{white} \textbf{\textsf{PARAMETER}}}] \textbf{\underline{month}} ||| UNSIGNED1 --- The month (1-12).
\item [\colorbox{tagtype}{\color{white} \textbf{\textsf{PARAMETER}}}] \textbf{\underline{day}} ||| UNSIGNED1 --- The day (1..daysInMonth).
\end{description}







\par
\begin{description}
\item [\colorbox{tagtype}{\color{white} \textbf{\textsf{RETURN}}}] \textbf{UNSIGNED4} --- A date created by combining the fields.
\end{description}




\rule{\linewidth}{0.5pt}
\subsection*{\textsf{\colorbox{headtoc}{\color{white} FUNCTION}
TimeFromParts}}

\hypertarget{ecldoc:date.timefromparts}{}
\hspace{0pt} \hyperlink{ecldoc:Date}{Date} \textbackslash 

{\renewcommand{\arraystretch}{1.5}
\begin{tabularx}{\textwidth}{|>{\raggedright\arraybackslash}l|X|}
\hline
\hspace{0pt}\mytexttt{\color{red} Time\_t} & \textbf{TimeFromParts} \\
\hline
\multicolumn{2}{|>{\raggedright\arraybackslash}X|}{\hspace{0pt}\mytexttt{\color{param} (UNSIGNED1 hour, UNSIGNED1 minute, UNSIGNED1 second)}} \\
\hline
\end{tabularx}
}

\par





Combines hour, minute second to create a time type.






\par
\begin{description}
\item [\colorbox{tagtype}{\color{white} \textbf{\textsf{PARAMETER}}}] \textbf{\underline{minute}} ||| UNSIGNED1 --- The minute (0-59).
\item [\colorbox{tagtype}{\color{white} \textbf{\textsf{PARAMETER}}}] \textbf{\underline{second}} ||| UNSIGNED1 --- The second (0-59).
\item [\colorbox{tagtype}{\color{white} \textbf{\textsf{PARAMETER}}}] \textbf{\underline{hour}} ||| UNSIGNED1 --- The hour (0-23).
\end{description}







\par
\begin{description}
\item [\colorbox{tagtype}{\color{white} \textbf{\textsf{RETURN}}}] \textbf{UNSIGNED3} --- A time created by combining the fields.
\end{description}




\rule{\linewidth}{0.5pt}
\subsection*{\textsf{\colorbox{headtoc}{\color{white} FUNCTION}
SecondsFromParts}}

\hypertarget{ecldoc:date.secondsfromparts}{}
\hspace{0pt} \hyperlink{ecldoc:Date}{Date} \textbackslash 

{\renewcommand{\arraystretch}{1.5}
\begin{tabularx}{\textwidth}{|>{\raggedright\arraybackslash}l|X|}
\hline
\hspace{0pt}\mytexttt{\color{red} Seconds\_t} & \textbf{SecondsFromParts} \\
\hline
\multicolumn{2}{|>{\raggedright\arraybackslash}X|}{\hspace{0pt}\mytexttt{\color{param} (INTEGER2 year, UNSIGNED1 month, UNSIGNED1 day, UNSIGNED1 hour, UNSIGNED1 minute, UNSIGNED1 second, BOOLEAN is\_local\_time = FALSE)}} \\
\hline
\end{tabularx}
}

\par





Combines date and time components to create a seconds type. The date must be represented within the Gregorian calendar after the year 1600.






\par
\begin{description}
\item [\colorbox{tagtype}{\color{white} \textbf{\textsf{PARAMETER}}}] \textbf{\underline{year}} ||| INTEGER2 --- The year (1601-30827).
\item [\colorbox{tagtype}{\color{white} \textbf{\textsf{PARAMETER}}}] \textbf{\underline{second}} ||| UNSIGNED1 --- The second (0-59).
\item [\colorbox{tagtype}{\color{white} \textbf{\textsf{PARAMETER}}}] \textbf{\underline{hour}} ||| UNSIGNED1 --- The hour (0-23).
\item [\colorbox{tagtype}{\color{white} \textbf{\textsf{PARAMETER}}}] \textbf{\underline{minute}} ||| UNSIGNED1 --- The minute (0-59).
\item [\colorbox{tagtype}{\color{white} \textbf{\textsf{PARAMETER}}}] \textbf{\underline{month}} ||| UNSIGNED1 --- The month (1-12).
\item [\colorbox{tagtype}{\color{white} \textbf{\textsf{PARAMETER}}}] \textbf{\underline{day}} ||| UNSIGNED1 --- The day (1..daysInMonth).
\item [\colorbox{tagtype}{\color{white} \textbf{\textsf{PARAMETER}}}] \textbf{\underline{is\_local\_time}} ||| BOOLEAN --- TRUE if the datetime components are expressed in local time rather than UTC, FALSE if the components are expressed in UTC. Optional, defaults to FALSE.
\end{description}







\par
\begin{description}
\item [\colorbox{tagtype}{\color{white} \textbf{\textsf{RETURN}}}] \textbf{INTEGER8} --- A Seconds\_t value created by combining the fields.
\end{description}




\rule{\linewidth}{0.5pt}
\subsection*{\textsf{\colorbox{headtoc}{\color{white} MODULE}
SecondsToParts}}

\hypertarget{ecldoc:date.secondstoparts}{}
\hspace{0pt} \hyperlink{ecldoc:Date}{Date} \textbackslash 

{\renewcommand{\arraystretch}{1.5}
\begin{tabularx}{\textwidth}{|>{\raggedright\arraybackslash}l|X|}
\hline
\hspace{0pt}\mytexttt{\color{red} } & \textbf{SecondsToParts} \\
\hline
\multicolumn{2}{|>{\raggedright\arraybackslash}X|}{\hspace{0pt}\mytexttt{\color{param} (Seconds\_t seconds)}} \\
\hline
\end{tabularx}
}

\par





Converts the number of seconds since epoch to a structure containing date and time parts. The result must be representable within the Gregorian calendar after the year 1600.






\par
\begin{description}
\item [\colorbox{tagtype}{\color{white} \textbf{\textsf{PARAMETER}}}] \textbf{\underline{seconds}} ||| INTEGER8 --- The number of seconds since epoch.
\end{description}







\par
\begin{description}
\item [\colorbox{tagtype}{\color{white} \textbf{\textsf{RETURN}}}] \textbf{} --- Module with exported attributes for year, month, day, hour, minute, second, day\_of\_week, date and time.
\end{description}




\textbf{Children}
\begin{enumerate}
\item \hyperlink{ecldoc:date.secondstoparts.result.year}{Year}
: No Documentation Found
\item \hyperlink{ecldoc:date.secondstoparts.result.month}{Month}
: No Documentation Found
\item \hyperlink{ecldoc:date.secondstoparts.result.day}{Day}
: No Documentation Found
\item \hyperlink{ecldoc:date.secondstoparts.result.hour}{Hour}
: No Documentation Found
\item \hyperlink{ecldoc:date.secondstoparts.result.minute}{Minute}
: No Documentation Found
\item \hyperlink{ecldoc:date.secondstoparts.result.second}{Second}
: No Documentation Found
\item \hyperlink{ecldoc:date.secondstoparts.result.day_of_week}{day\_of\_week}
: No Documentation Found
\item \hyperlink{ecldoc:date.secondstoparts.result.date}{date}
: Combines year, month day to create a date type
\item \hyperlink{ecldoc:date.secondstoparts.result.time}{time}
: Combines hour, minute second to create a time type
\end{enumerate}

\rule{\linewidth}{0.5pt}

\subsection*{\textsf{\colorbox{headtoc}{\color{white} ATTRIBUTE}
Year}}

\hypertarget{ecldoc:date.secondstoparts.result.year}{}
\hspace{0pt} \hyperlink{ecldoc:Date}{Date} \textbackslash 
\hspace{0pt} \hyperlink{ecldoc:date.secondstoparts}{SecondsToParts} \textbackslash 

{\renewcommand{\arraystretch}{1.5}
\begin{tabularx}{\textwidth}{|>{\raggedright\arraybackslash}l|X|}
\hline
\hspace{0pt}\mytexttt{\color{red} INTEGER2} & \textbf{Year} \\
\hline
\end{tabularx}
}

\par





No Documentation Found








\par
\begin{description}
\item [\colorbox{tagtype}{\color{white} \textbf{\textsf{RETURN}}}] \textbf{INTEGER2} --- 
\end{description}




\rule{\linewidth}{0.5pt}
\subsection*{\textsf{\colorbox{headtoc}{\color{white} ATTRIBUTE}
Month}}

\hypertarget{ecldoc:date.secondstoparts.result.month}{}
\hspace{0pt} \hyperlink{ecldoc:Date}{Date} \textbackslash 
\hspace{0pt} \hyperlink{ecldoc:date.secondstoparts}{SecondsToParts} \textbackslash 

{\renewcommand{\arraystretch}{1.5}
\begin{tabularx}{\textwidth}{|>{\raggedright\arraybackslash}l|X|}
\hline
\hspace{0pt}\mytexttt{\color{red} UNSIGNED1} & \textbf{Month} \\
\hline
\end{tabularx}
}

\par





No Documentation Found








\par
\begin{description}
\item [\colorbox{tagtype}{\color{white} \textbf{\textsf{RETURN}}}] \textbf{UNSIGNED1} --- 
\end{description}




\rule{\linewidth}{0.5pt}
\subsection*{\textsf{\colorbox{headtoc}{\color{white} ATTRIBUTE}
Day}}

\hypertarget{ecldoc:date.secondstoparts.result.day}{}
\hspace{0pt} \hyperlink{ecldoc:Date}{Date} \textbackslash 
\hspace{0pt} \hyperlink{ecldoc:date.secondstoparts}{SecondsToParts} \textbackslash 

{\renewcommand{\arraystretch}{1.5}
\begin{tabularx}{\textwidth}{|>{\raggedright\arraybackslash}l|X|}
\hline
\hspace{0pt}\mytexttt{\color{red} UNSIGNED1} & \textbf{Day} \\
\hline
\end{tabularx}
}

\par





No Documentation Found








\par
\begin{description}
\item [\colorbox{tagtype}{\color{white} \textbf{\textsf{RETURN}}}] \textbf{UNSIGNED1} --- 
\end{description}




\rule{\linewidth}{0.5pt}
\subsection*{\textsf{\colorbox{headtoc}{\color{white} ATTRIBUTE}
Hour}}

\hypertarget{ecldoc:date.secondstoparts.result.hour}{}
\hspace{0pt} \hyperlink{ecldoc:Date}{Date} \textbackslash 
\hspace{0pt} \hyperlink{ecldoc:date.secondstoparts}{SecondsToParts} \textbackslash 

{\renewcommand{\arraystretch}{1.5}
\begin{tabularx}{\textwidth}{|>{\raggedright\arraybackslash}l|X|}
\hline
\hspace{0pt}\mytexttt{\color{red} UNSIGNED1} & \textbf{Hour} \\
\hline
\end{tabularx}
}

\par





No Documentation Found








\par
\begin{description}
\item [\colorbox{tagtype}{\color{white} \textbf{\textsf{RETURN}}}] \textbf{UNSIGNED1} --- 
\end{description}




\rule{\linewidth}{0.5pt}
\subsection*{\textsf{\colorbox{headtoc}{\color{white} ATTRIBUTE}
Minute}}

\hypertarget{ecldoc:date.secondstoparts.result.minute}{}
\hspace{0pt} \hyperlink{ecldoc:Date}{Date} \textbackslash 
\hspace{0pt} \hyperlink{ecldoc:date.secondstoparts}{SecondsToParts} \textbackslash 

{\renewcommand{\arraystretch}{1.5}
\begin{tabularx}{\textwidth}{|>{\raggedright\arraybackslash}l|X|}
\hline
\hspace{0pt}\mytexttt{\color{red} UNSIGNED1} & \textbf{Minute} \\
\hline
\end{tabularx}
}

\par





No Documentation Found








\par
\begin{description}
\item [\colorbox{tagtype}{\color{white} \textbf{\textsf{RETURN}}}] \textbf{UNSIGNED1} --- 
\end{description}




\rule{\linewidth}{0.5pt}
\subsection*{\textsf{\colorbox{headtoc}{\color{white} ATTRIBUTE}
Second}}

\hypertarget{ecldoc:date.secondstoparts.result.second}{}
\hspace{0pt} \hyperlink{ecldoc:Date}{Date} \textbackslash 
\hspace{0pt} \hyperlink{ecldoc:date.secondstoparts}{SecondsToParts} \textbackslash 

{\renewcommand{\arraystretch}{1.5}
\begin{tabularx}{\textwidth}{|>{\raggedright\arraybackslash}l|X|}
\hline
\hspace{0pt}\mytexttt{\color{red} UNSIGNED1} & \textbf{Second} \\
\hline
\end{tabularx}
}

\par





No Documentation Found








\par
\begin{description}
\item [\colorbox{tagtype}{\color{white} \textbf{\textsf{RETURN}}}] \textbf{UNSIGNED1} --- 
\end{description}




\rule{\linewidth}{0.5pt}
\subsection*{\textsf{\colorbox{headtoc}{\color{white} ATTRIBUTE}
day\_of\_week}}

\hypertarget{ecldoc:date.secondstoparts.result.day_of_week}{}
\hspace{0pt} \hyperlink{ecldoc:Date}{Date} \textbackslash 
\hspace{0pt} \hyperlink{ecldoc:date.secondstoparts}{SecondsToParts} \textbackslash 

{\renewcommand{\arraystretch}{1.5}
\begin{tabularx}{\textwidth}{|>{\raggedright\arraybackslash}l|X|}
\hline
\hspace{0pt}\mytexttt{\color{red} UNSIGNED1} & \textbf{day\_of\_week} \\
\hline
\end{tabularx}
}

\par





No Documentation Found








\par
\begin{description}
\item [\colorbox{tagtype}{\color{white} \textbf{\textsf{RETURN}}}] \textbf{UNSIGNED1} --- 
\end{description}




\rule{\linewidth}{0.5pt}
\subsection*{\textsf{\colorbox{headtoc}{\color{white} ATTRIBUTE}
date}}

\hypertarget{ecldoc:date.secondstoparts.result.date}{}
\hspace{0pt} \hyperlink{ecldoc:Date}{Date} \textbackslash 
\hspace{0pt} \hyperlink{ecldoc:date.secondstoparts}{SecondsToParts} \textbackslash 

{\renewcommand{\arraystretch}{1.5}
\begin{tabularx}{\textwidth}{|>{\raggedright\arraybackslash}l|X|}
\hline
\hspace{0pt}\mytexttt{\color{red} Date\_t} & \textbf{date} \\
\hline
\end{tabularx}
}

\par





Combines year, month day to create a date type.






\par
\begin{description}
\item [\colorbox{tagtype}{\color{white} \textbf{\textsf{PARAMETER}}}] \textbf{\underline{year}} |||  --- The year (0-9999).
\item [\colorbox{tagtype}{\color{white} \textbf{\textsf{PARAMETER}}}] \textbf{\underline{month}} |||  --- The month (1-12).
\item [\colorbox{tagtype}{\color{white} \textbf{\textsf{PARAMETER}}}] \textbf{\underline{day}} |||  --- The day (1..daysInMonth).
\end{description}







\par
\begin{description}
\item [\colorbox{tagtype}{\color{white} \textbf{\textsf{RETURN}}}] \textbf{UNSIGNED4} --- A date created by combining the fields.
\end{description}




\rule{\linewidth}{0.5pt}
\subsection*{\textsf{\colorbox{headtoc}{\color{white} ATTRIBUTE}
time}}

\hypertarget{ecldoc:date.secondstoparts.result.time}{}
\hspace{0pt} \hyperlink{ecldoc:Date}{Date} \textbackslash 
\hspace{0pt} \hyperlink{ecldoc:date.secondstoparts}{SecondsToParts} \textbackslash 

{\renewcommand{\arraystretch}{1.5}
\begin{tabularx}{\textwidth}{|>{\raggedright\arraybackslash}l|X|}
\hline
\hspace{0pt}\mytexttt{\color{red} Time\_t} & \textbf{time} \\
\hline
\end{tabularx}
}

\par





Combines hour, minute second to create a time type.






\par
\begin{description}
\item [\colorbox{tagtype}{\color{white} \textbf{\textsf{PARAMETER}}}] \textbf{\underline{minute}} |||  --- The minute (0-59).
\item [\colorbox{tagtype}{\color{white} \textbf{\textsf{PARAMETER}}}] \textbf{\underline{second}} |||  --- The second (0-59).
\item [\colorbox{tagtype}{\color{white} \textbf{\textsf{PARAMETER}}}] \textbf{\underline{hour}} |||  --- The hour (0-23).
\end{description}







\par
\begin{description}
\item [\colorbox{tagtype}{\color{white} \textbf{\textsf{RETURN}}}] \textbf{UNSIGNED3} --- A time created by combining the fields.
\end{description}




\rule{\linewidth}{0.5pt}


\subsection*{\textsf{\colorbox{headtoc}{\color{white} FUNCTION}
TimestampToSeconds}}

\hypertarget{ecldoc:date.timestamptoseconds}{}
\hspace{0pt} \hyperlink{ecldoc:Date}{Date} \textbackslash 

{\renewcommand{\arraystretch}{1.5}
\begin{tabularx}{\textwidth}{|>{\raggedright\arraybackslash}l|X|}
\hline
\hspace{0pt}\mytexttt{\color{red} Seconds\_t} & \textbf{TimestampToSeconds} \\
\hline
\multicolumn{2}{|>{\raggedright\arraybackslash}X|}{\hspace{0pt}\mytexttt{\color{param} (Timestamp\_t timestamp)}} \\
\hline
\end{tabularx}
}

\par





Converts the number of microseconds since epoch to the number of seconds since epoch.






\par
\begin{description}
\item [\colorbox{tagtype}{\color{white} \textbf{\textsf{PARAMETER}}}] \textbf{\underline{timestamp}} ||| INTEGER8 --- The number of microseconds since epoch.
\end{description}







\par
\begin{description}
\item [\colorbox{tagtype}{\color{white} \textbf{\textsf{RETURN}}}] \textbf{INTEGER8} --- The number of seconds since epoch.
\end{description}




\rule{\linewidth}{0.5pt}
\subsection*{\textsf{\colorbox{headtoc}{\color{white} FUNCTION}
IsLeapYear}}

\hypertarget{ecldoc:date.isleapyear}{}
\hspace{0pt} \hyperlink{ecldoc:Date}{Date} \textbackslash 

{\renewcommand{\arraystretch}{1.5}
\begin{tabularx}{\textwidth}{|>{\raggedright\arraybackslash}l|X|}
\hline
\hspace{0pt}\mytexttt{\color{red} BOOLEAN} & \textbf{IsLeapYear} \\
\hline
\multicolumn{2}{|>{\raggedright\arraybackslash}X|}{\hspace{0pt}\mytexttt{\color{param} (INTEGER2 year)}} \\
\hline
\end{tabularx}
}

\par





Tests whether the year is a leap year in the Gregorian calendar.






\par
\begin{description}
\item [\colorbox{tagtype}{\color{white} \textbf{\textsf{PARAMETER}}}] \textbf{\underline{year}} ||| INTEGER2 --- The year (0-9999).
\end{description}







\par
\begin{description}
\item [\colorbox{tagtype}{\color{white} \textbf{\textsf{RETURN}}}] \textbf{BOOLEAN} --- True if the year is a leap year.
\end{description}




\rule{\linewidth}{0.5pt}
\subsection*{\textsf{\colorbox{headtoc}{\color{white} FUNCTION}
IsDateLeapYear}}

\hypertarget{ecldoc:date.isdateleapyear}{}
\hspace{0pt} \hyperlink{ecldoc:Date}{Date} \textbackslash 

{\renewcommand{\arraystretch}{1.5}
\begin{tabularx}{\textwidth}{|>{\raggedright\arraybackslash}l|X|}
\hline
\hspace{0pt}\mytexttt{\color{red} BOOLEAN} & \textbf{IsDateLeapYear} \\
\hline
\multicolumn{2}{|>{\raggedright\arraybackslash}X|}{\hspace{0pt}\mytexttt{\color{param} (Date\_t date)}} \\
\hline
\end{tabularx}
}

\par





Tests whether a date is a leap year in the Gregorian calendar.






\par
\begin{description}
\item [\colorbox{tagtype}{\color{white} \textbf{\textsf{PARAMETER}}}] \textbf{\underline{date}} ||| UNSIGNED4 --- The date.
\end{description}







\par
\begin{description}
\item [\colorbox{tagtype}{\color{white} \textbf{\textsf{RETURN}}}] \textbf{BOOLEAN} --- True if the year is a leap year.
\end{description}




\rule{\linewidth}{0.5pt}
\subsection*{\textsf{\colorbox{headtoc}{\color{white} FUNCTION}
FromGregorianYMD}}

\hypertarget{ecldoc:date.fromgregorianymd}{}
\hspace{0pt} \hyperlink{ecldoc:Date}{Date} \textbackslash 

{\renewcommand{\arraystretch}{1.5}
\begin{tabularx}{\textwidth}{|>{\raggedright\arraybackslash}l|X|}
\hline
\hspace{0pt}\mytexttt{\color{red} Days\_t} & \textbf{FromGregorianYMD} \\
\hline
\multicolumn{2}{|>{\raggedright\arraybackslash}X|}{\hspace{0pt}\mytexttt{\color{param} (INTEGER2 year, UNSIGNED1 month, UNSIGNED1 day)}} \\
\hline
\end{tabularx}
}

\par





Combines year, month, day in the Gregorian calendar to create the number days since 31st December 1BC.






\par
\begin{description}
\item [\colorbox{tagtype}{\color{white} \textbf{\textsf{PARAMETER}}}] \textbf{\underline{year}} ||| INTEGER2 --- The year (-4713..9999).
\item [\colorbox{tagtype}{\color{white} \textbf{\textsf{PARAMETER}}}] \textbf{\underline{month}} ||| UNSIGNED1 --- The month (1-12). A missing value (0) is treated as 1.
\item [\colorbox{tagtype}{\color{white} \textbf{\textsf{PARAMETER}}}] \textbf{\underline{day}} ||| UNSIGNED1 --- The day (1..daysInMonth). A missing value (0) is treated as 1.
\end{description}







\par
\begin{description}
\item [\colorbox{tagtype}{\color{white} \textbf{\textsf{RETURN}}}] \textbf{INTEGER4} --- The number of elapsed days (1 Jan 1AD = 1)
\end{description}




\rule{\linewidth}{0.5pt}
\subsection*{\textsf{\colorbox{headtoc}{\color{white} MODULE}
ToGregorianYMD}}

\hypertarget{ecldoc:date.togregorianymd}{}
\hspace{0pt} \hyperlink{ecldoc:Date}{Date} \textbackslash 

{\renewcommand{\arraystretch}{1.5}
\begin{tabularx}{\textwidth}{|>{\raggedright\arraybackslash}l|X|}
\hline
\hspace{0pt}\mytexttt{\color{red} } & \textbf{ToGregorianYMD} \\
\hline
\multicolumn{2}{|>{\raggedright\arraybackslash}X|}{\hspace{0pt}\mytexttt{\color{param} (Days\_t days)}} \\
\hline
\end{tabularx}
}

\par





Converts the number days since 31st December 1BC to a date in the Gregorian calendar.






\par
\begin{description}
\item [\colorbox{tagtype}{\color{white} \textbf{\textsf{PARAMETER}}}] \textbf{\underline{days}} ||| INTEGER4 --- The number of elapsed days (1 Jan 1AD = 1)
\end{description}







\par
\begin{description}
\item [\colorbox{tagtype}{\color{white} \textbf{\textsf{RETURN}}}] \textbf{} --- Module containing Year, Month, Day in the Gregorian calendar
\end{description}




\textbf{Children}
\begin{enumerate}
\item \hyperlink{ecldoc:date.togregorianymd.result.year}{year}
: No Documentation Found
\item \hyperlink{ecldoc:date.togregorianymd.result.month}{month}
: No Documentation Found
\item \hyperlink{ecldoc:date.togregorianymd.result.day}{day}
: No Documentation Found
\end{enumerate}

\rule{\linewidth}{0.5pt}

\subsection*{\textsf{\colorbox{headtoc}{\color{white} ATTRIBUTE}
year}}

\hypertarget{ecldoc:date.togregorianymd.result.year}{}
\hspace{0pt} \hyperlink{ecldoc:Date}{Date} \textbackslash 
\hspace{0pt} \hyperlink{ecldoc:date.togregorianymd}{ToGregorianYMD} \textbackslash 

{\renewcommand{\arraystretch}{1.5}
\begin{tabularx}{\textwidth}{|>{\raggedright\arraybackslash}l|X|}
\hline
\hspace{0pt}\mytexttt{\color{red} } & \textbf{year} \\
\hline
\end{tabularx}
}

\par





No Documentation Found








\par
\begin{description}
\item [\colorbox{tagtype}{\color{white} \textbf{\textsf{RETURN}}}] \textbf{INTEGER8} --- 
\end{description}




\rule{\linewidth}{0.5pt}
\subsection*{\textsf{\colorbox{headtoc}{\color{white} ATTRIBUTE}
month}}

\hypertarget{ecldoc:date.togregorianymd.result.month}{}
\hspace{0pt} \hyperlink{ecldoc:Date}{Date} \textbackslash 
\hspace{0pt} \hyperlink{ecldoc:date.togregorianymd}{ToGregorianYMD} \textbackslash 

{\renewcommand{\arraystretch}{1.5}
\begin{tabularx}{\textwidth}{|>{\raggedright\arraybackslash}l|X|}
\hline
\hspace{0pt}\mytexttt{\color{red} } & \textbf{month} \\
\hline
\end{tabularx}
}

\par





No Documentation Found








\par
\begin{description}
\item [\colorbox{tagtype}{\color{white} \textbf{\textsf{RETURN}}}] \textbf{INTEGER8} --- 
\end{description}




\rule{\linewidth}{0.5pt}
\subsection*{\textsf{\colorbox{headtoc}{\color{white} ATTRIBUTE}
day}}

\hypertarget{ecldoc:date.togregorianymd.result.day}{}
\hspace{0pt} \hyperlink{ecldoc:Date}{Date} \textbackslash 
\hspace{0pt} \hyperlink{ecldoc:date.togregorianymd}{ToGregorianYMD} \textbackslash 

{\renewcommand{\arraystretch}{1.5}
\begin{tabularx}{\textwidth}{|>{\raggedright\arraybackslash}l|X|}
\hline
\hspace{0pt}\mytexttt{\color{red} } & \textbf{day} \\
\hline
\end{tabularx}
}

\par





No Documentation Found








\par
\begin{description}
\item [\colorbox{tagtype}{\color{white} \textbf{\textsf{RETURN}}}] \textbf{INTEGER8} --- 
\end{description}




\rule{\linewidth}{0.5pt}


\subsection*{\textsf{\colorbox{headtoc}{\color{white} FUNCTION}
FromGregorianDate}}

\hypertarget{ecldoc:date.fromgregoriandate}{}
\hspace{0pt} \hyperlink{ecldoc:Date}{Date} \textbackslash 

{\renewcommand{\arraystretch}{1.5}
\begin{tabularx}{\textwidth}{|>{\raggedright\arraybackslash}l|X|}
\hline
\hspace{0pt}\mytexttt{\color{red} Days\_t} & \textbf{FromGregorianDate} \\
\hline
\multicolumn{2}{|>{\raggedright\arraybackslash}X|}{\hspace{0pt}\mytexttt{\color{param} (Date\_t date)}} \\
\hline
\end{tabularx}
}

\par





Converts a date in the Gregorian calendar to the number days since 31st December 1BC.






\par
\begin{description}
\item [\colorbox{tagtype}{\color{white} \textbf{\textsf{PARAMETER}}}] \textbf{\underline{date}} ||| UNSIGNED4 --- The date (using the Gregorian calendar)
\end{description}







\par
\begin{description}
\item [\colorbox{tagtype}{\color{white} \textbf{\textsf{RETURN}}}] \textbf{INTEGER4} --- The number of elapsed days (1 Jan 1AD = 1)
\end{description}




\rule{\linewidth}{0.5pt}
\subsection*{\textsf{\colorbox{headtoc}{\color{white} FUNCTION}
ToGregorianDate}}

\hypertarget{ecldoc:date.togregoriandate}{}
\hspace{0pt} \hyperlink{ecldoc:Date}{Date} \textbackslash 

{\renewcommand{\arraystretch}{1.5}
\begin{tabularx}{\textwidth}{|>{\raggedright\arraybackslash}l|X|}
\hline
\hspace{0pt}\mytexttt{\color{red} Date\_t} & \textbf{ToGregorianDate} \\
\hline
\multicolumn{2}{|>{\raggedright\arraybackslash}X|}{\hspace{0pt}\mytexttt{\color{param} (Days\_t days)}} \\
\hline
\end{tabularx}
}

\par





Converts the number days since 31st December 1BC to a date in the Gregorian calendar.






\par
\begin{description}
\item [\colorbox{tagtype}{\color{white} \textbf{\textsf{PARAMETER}}}] \textbf{\underline{days}} ||| INTEGER4 --- The number of elapsed days (1 Jan 1AD = 1)
\end{description}







\par
\begin{description}
\item [\colorbox{tagtype}{\color{white} \textbf{\textsf{RETURN}}}] \textbf{UNSIGNED4} --- A Date\_t in the Gregorian calendar
\end{description}




\rule{\linewidth}{0.5pt}
\subsection*{\textsf{\colorbox{headtoc}{\color{white} FUNCTION}
DayOfYear}}

\hypertarget{ecldoc:date.dayofyear}{}
\hspace{0pt} \hyperlink{ecldoc:Date}{Date} \textbackslash 

{\renewcommand{\arraystretch}{1.5}
\begin{tabularx}{\textwidth}{|>{\raggedright\arraybackslash}l|X|}
\hline
\hspace{0pt}\mytexttt{\color{red} UNSIGNED2} & \textbf{DayOfYear} \\
\hline
\multicolumn{2}{|>{\raggedright\arraybackslash}X|}{\hspace{0pt}\mytexttt{\color{param} (Date\_t date)}} \\
\hline
\end{tabularx}
}

\par





Returns a number representing the day of the year indicated by the given date. The date must be in the Gregorian calendar after the year 1600.






\par
\begin{description}
\item [\colorbox{tagtype}{\color{white} \textbf{\textsf{PARAMETER}}}] \textbf{\underline{date}} ||| UNSIGNED4 --- A Date\_t value.
\end{description}







\par
\begin{description}
\item [\colorbox{tagtype}{\color{white} \textbf{\textsf{RETURN}}}] \textbf{UNSIGNED2} --- A number (1-366) representing the number of days since the beginning of the year.
\end{description}




\rule{\linewidth}{0.5pt}
\subsection*{\textsf{\colorbox{headtoc}{\color{white} FUNCTION}
DayOfWeek}}

\hypertarget{ecldoc:date.dayofweek}{}
\hspace{0pt} \hyperlink{ecldoc:Date}{Date} \textbackslash 

{\renewcommand{\arraystretch}{1.5}
\begin{tabularx}{\textwidth}{|>{\raggedright\arraybackslash}l|X|}
\hline
\hspace{0pt}\mytexttt{\color{red} UNSIGNED1} & \textbf{DayOfWeek} \\
\hline
\multicolumn{2}{|>{\raggedright\arraybackslash}X|}{\hspace{0pt}\mytexttt{\color{param} (Date\_t date)}} \\
\hline
\end{tabularx}
}

\par





Returns a number representing the day of the week indicated by the given date. The date must be in the Gregorian calendar after the year 1600.






\par
\begin{description}
\item [\colorbox{tagtype}{\color{white} \textbf{\textsf{PARAMETER}}}] \textbf{\underline{date}} ||| UNSIGNED4 --- A Date\_t value.
\end{description}







\par
\begin{description}
\item [\colorbox{tagtype}{\color{white} \textbf{\textsf{RETURN}}}] \textbf{UNSIGNED1} --- A number 1-7 representing the day of the week, where 1 = Sunday.
\end{description}




\rule{\linewidth}{0.5pt}
\subsection*{\textsf{\colorbox{headtoc}{\color{white} FUNCTION}
IsJulianLeapYear}}

\hypertarget{ecldoc:date.isjulianleapyear}{}
\hspace{0pt} \hyperlink{ecldoc:Date}{Date} \textbackslash 

{\renewcommand{\arraystretch}{1.5}
\begin{tabularx}{\textwidth}{|>{\raggedright\arraybackslash}l|X|}
\hline
\hspace{0pt}\mytexttt{\color{red} BOOLEAN} & \textbf{IsJulianLeapYear} \\
\hline
\multicolumn{2}{|>{\raggedright\arraybackslash}X|}{\hspace{0pt}\mytexttt{\color{param} (INTEGER2 year)}} \\
\hline
\end{tabularx}
}

\par





Tests whether the year is a leap year in the Julian calendar.






\par
\begin{description}
\item [\colorbox{tagtype}{\color{white} \textbf{\textsf{PARAMETER}}}] \textbf{\underline{year}} ||| INTEGER2 --- The year (0-9999).
\end{description}







\par
\begin{description}
\item [\colorbox{tagtype}{\color{white} \textbf{\textsf{RETURN}}}] \textbf{BOOLEAN} --- True if the year is a leap year.
\end{description}




\rule{\linewidth}{0.5pt}
\subsection*{\textsf{\colorbox{headtoc}{\color{white} FUNCTION}
FromJulianYMD}}

\hypertarget{ecldoc:date.fromjulianymd}{}
\hspace{0pt} \hyperlink{ecldoc:Date}{Date} \textbackslash 

{\renewcommand{\arraystretch}{1.5}
\begin{tabularx}{\textwidth}{|>{\raggedright\arraybackslash}l|X|}
\hline
\hspace{0pt}\mytexttt{\color{red} Days\_t} & \textbf{FromJulianYMD} \\
\hline
\multicolumn{2}{|>{\raggedright\arraybackslash}X|}{\hspace{0pt}\mytexttt{\color{param} (INTEGER2 year, UNSIGNED1 month, UNSIGNED1 day)}} \\
\hline
\end{tabularx}
}

\par





Combines year, month, day in the Julian calendar to create the number days since 31st December 1BC.






\par
\begin{description}
\item [\colorbox{tagtype}{\color{white} \textbf{\textsf{PARAMETER}}}] \textbf{\underline{year}} ||| INTEGER2 --- The year (-4800..9999).
\item [\colorbox{tagtype}{\color{white} \textbf{\textsf{PARAMETER}}}] \textbf{\underline{month}} ||| UNSIGNED1 --- The month (1-12).
\item [\colorbox{tagtype}{\color{white} \textbf{\textsf{PARAMETER}}}] \textbf{\underline{day}} ||| UNSIGNED1 --- The day (1..daysInMonth).
\end{description}







\par
\begin{description}
\item [\colorbox{tagtype}{\color{white} \textbf{\textsf{RETURN}}}] \textbf{INTEGER4} --- The number of elapsed days (1 Jan 1AD = 1)
\end{description}




\rule{\linewidth}{0.5pt}
\subsection*{\textsf{\colorbox{headtoc}{\color{white} MODULE}
ToJulianYMD}}

\hypertarget{ecldoc:date.tojulianymd}{}
\hspace{0pt} \hyperlink{ecldoc:Date}{Date} \textbackslash 

{\renewcommand{\arraystretch}{1.5}
\begin{tabularx}{\textwidth}{|>{\raggedright\arraybackslash}l|X|}
\hline
\hspace{0pt}\mytexttt{\color{red} } & \textbf{ToJulianYMD} \\
\hline
\multicolumn{2}{|>{\raggedright\arraybackslash}X|}{\hspace{0pt}\mytexttt{\color{param} (Days\_t days)}} \\
\hline
\end{tabularx}
}

\par





Converts the number days since 31st December 1BC to a date in the Julian calendar.






\par
\begin{description}
\item [\colorbox{tagtype}{\color{white} \textbf{\textsf{PARAMETER}}}] \textbf{\underline{days}} ||| INTEGER4 --- The number of elapsed days (1 Jan 1AD = 1)
\end{description}







\par
\begin{description}
\item [\colorbox{tagtype}{\color{white} \textbf{\textsf{RETURN}}}] \textbf{} --- Module containing Year, Month, Day in the Julian calendar
\end{description}




\textbf{Children}
\begin{enumerate}
\item \hyperlink{ecldoc:date.tojulianymd.result.day}{Day}
: No Documentation Found
\item \hyperlink{ecldoc:date.tojulianymd.result.month}{Month}
: No Documentation Found
\item \hyperlink{ecldoc:date.tojulianymd.result.year}{Year}
: No Documentation Found
\end{enumerate}

\rule{\linewidth}{0.5pt}

\subsection*{\textsf{\colorbox{headtoc}{\color{white} ATTRIBUTE}
Day}}

\hypertarget{ecldoc:date.tojulianymd.result.day}{}
\hspace{0pt} \hyperlink{ecldoc:Date}{Date} \textbackslash 
\hspace{0pt} \hyperlink{ecldoc:date.tojulianymd}{ToJulianYMD} \textbackslash 

{\renewcommand{\arraystretch}{1.5}
\begin{tabularx}{\textwidth}{|>{\raggedright\arraybackslash}l|X|}
\hline
\hspace{0pt}\mytexttt{\color{red} UNSIGNED1} & \textbf{Day} \\
\hline
\end{tabularx}
}

\par





No Documentation Found








\par
\begin{description}
\item [\colorbox{tagtype}{\color{white} \textbf{\textsf{RETURN}}}] \textbf{UNSIGNED1} --- 
\end{description}




\rule{\linewidth}{0.5pt}
\subsection*{\textsf{\colorbox{headtoc}{\color{white} ATTRIBUTE}
Month}}

\hypertarget{ecldoc:date.tojulianymd.result.month}{}
\hspace{0pt} \hyperlink{ecldoc:Date}{Date} \textbackslash 
\hspace{0pt} \hyperlink{ecldoc:date.tojulianymd}{ToJulianYMD} \textbackslash 

{\renewcommand{\arraystretch}{1.5}
\begin{tabularx}{\textwidth}{|>{\raggedright\arraybackslash}l|X|}
\hline
\hspace{0pt}\mytexttt{\color{red} UNSIGNED1} & \textbf{Month} \\
\hline
\end{tabularx}
}

\par





No Documentation Found








\par
\begin{description}
\item [\colorbox{tagtype}{\color{white} \textbf{\textsf{RETURN}}}] \textbf{UNSIGNED1} --- 
\end{description}




\rule{\linewidth}{0.5pt}
\subsection*{\textsf{\colorbox{headtoc}{\color{white} ATTRIBUTE}
Year}}

\hypertarget{ecldoc:date.tojulianymd.result.year}{}
\hspace{0pt} \hyperlink{ecldoc:Date}{Date} \textbackslash 
\hspace{0pt} \hyperlink{ecldoc:date.tojulianymd}{ToJulianYMD} \textbackslash 

{\renewcommand{\arraystretch}{1.5}
\begin{tabularx}{\textwidth}{|>{\raggedright\arraybackslash}l|X|}
\hline
\hspace{0pt}\mytexttt{\color{red} INTEGER2} & \textbf{Year} \\
\hline
\end{tabularx}
}

\par





No Documentation Found








\par
\begin{description}
\item [\colorbox{tagtype}{\color{white} \textbf{\textsf{RETURN}}}] \textbf{INTEGER2} --- 
\end{description}




\rule{\linewidth}{0.5pt}


\subsection*{\textsf{\colorbox{headtoc}{\color{white} FUNCTION}
FromJulianDate}}

\hypertarget{ecldoc:date.fromjuliandate}{}
\hspace{0pt} \hyperlink{ecldoc:Date}{Date} \textbackslash 

{\renewcommand{\arraystretch}{1.5}
\begin{tabularx}{\textwidth}{|>{\raggedright\arraybackslash}l|X|}
\hline
\hspace{0pt}\mytexttt{\color{red} Days\_t} & \textbf{FromJulianDate} \\
\hline
\multicolumn{2}{|>{\raggedright\arraybackslash}X|}{\hspace{0pt}\mytexttt{\color{param} (Date\_t date)}} \\
\hline
\end{tabularx}
}

\par





Converts a date in the Julian calendar to the number days since 31st December 1BC.






\par
\begin{description}
\item [\colorbox{tagtype}{\color{white} \textbf{\textsf{PARAMETER}}}] \textbf{\underline{date}} ||| UNSIGNED4 --- The date (using the Julian calendar)
\end{description}







\par
\begin{description}
\item [\colorbox{tagtype}{\color{white} \textbf{\textsf{RETURN}}}] \textbf{INTEGER4} --- The number of elapsed days (1 Jan 1AD = 1)
\end{description}




\rule{\linewidth}{0.5pt}
\subsection*{\textsf{\colorbox{headtoc}{\color{white} FUNCTION}
ToJulianDate}}

\hypertarget{ecldoc:date.tojuliandate}{}
\hspace{0pt} \hyperlink{ecldoc:Date}{Date} \textbackslash 

{\renewcommand{\arraystretch}{1.5}
\begin{tabularx}{\textwidth}{|>{\raggedright\arraybackslash}l|X|}
\hline
\hspace{0pt}\mytexttt{\color{red} Date\_t} & \textbf{ToJulianDate} \\
\hline
\multicolumn{2}{|>{\raggedright\arraybackslash}X|}{\hspace{0pt}\mytexttt{\color{param} (Days\_t days)}} \\
\hline
\end{tabularx}
}

\par





Converts the number days since 31st December 1BC to a date in the Julian calendar.






\par
\begin{description}
\item [\colorbox{tagtype}{\color{white} \textbf{\textsf{PARAMETER}}}] \textbf{\underline{days}} ||| INTEGER4 --- The number of elapsed days (1 Jan 1AD = 1)
\end{description}







\par
\begin{description}
\item [\colorbox{tagtype}{\color{white} \textbf{\textsf{RETURN}}}] \textbf{UNSIGNED4} --- A Date\_t in the Julian calendar
\end{description}




\rule{\linewidth}{0.5pt}
\subsection*{\textsf{\colorbox{headtoc}{\color{white} FUNCTION}
DaysSince1900}}

\hypertarget{ecldoc:date.dayssince1900}{}
\hspace{0pt} \hyperlink{ecldoc:Date}{Date} \textbackslash 

{\renewcommand{\arraystretch}{1.5}
\begin{tabularx}{\textwidth}{|>{\raggedright\arraybackslash}l|X|}
\hline
\hspace{0pt}\mytexttt{\color{red} Days\_t} & \textbf{DaysSince1900} \\
\hline
\multicolumn{2}{|>{\raggedright\arraybackslash}X|}{\hspace{0pt}\mytexttt{\color{param} (INTEGER2 year, UNSIGNED1 month, UNSIGNED1 day)}} \\
\hline
\end{tabularx}
}

\par





Returns the number of days since 1st January 1900 (using the Gregorian Calendar)






\par
\begin{description}
\item [\colorbox{tagtype}{\color{white} \textbf{\textsf{PARAMETER}}}] \textbf{\underline{year}} ||| INTEGER2 --- The year (-4713..9999).
\item [\colorbox{tagtype}{\color{white} \textbf{\textsf{PARAMETER}}}] \textbf{\underline{month}} ||| UNSIGNED1 --- The month (1-12). A missing value (0) is treated as 1.
\item [\colorbox{tagtype}{\color{white} \textbf{\textsf{PARAMETER}}}] \textbf{\underline{day}} ||| UNSIGNED1 --- The day (1..daysInMonth). A missing value (0) is treated as 1.
\end{description}







\par
\begin{description}
\item [\colorbox{tagtype}{\color{white} \textbf{\textsf{RETURN}}}] \textbf{INTEGER4} --- The number of elapsed days since 1st January 1900
\end{description}




\rule{\linewidth}{0.5pt}
\subsection*{\textsf{\colorbox{headtoc}{\color{white} FUNCTION}
ToDaysSince1900}}

\hypertarget{ecldoc:date.todayssince1900}{}
\hspace{0pt} \hyperlink{ecldoc:Date}{Date} \textbackslash 

{\renewcommand{\arraystretch}{1.5}
\begin{tabularx}{\textwidth}{|>{\raggedright\arraybackslash}l|X|}
\hline
\hspace{0pt}\mytexttt{\color{red} Days\_t} & \textbf{ToDaysSince1900} \\
\hline
\multicolumn{2}{|>{\raggedright\arraybackslash}X|}{\hspace{0pt}\mytexttt{\color{param} (Date\_t date)}} \\
\hline
\end{tabularx}
}

\par





Returns the number of days since 1st January 1900 (using the Gregorian Calendar)






\par
\begin{description}
\item [\colorbox{tagtype}{\color{white} \textbf{\textsf{PARAMETER}}}] \textbf{\underline{date}} ||| UNSIGNED4 --- The date
\end{description}







\par
\begin{description}
\item [\colorbox{tagtype}{\color{white} \textbf{\textsf{RETURN}}}] \textbf{INTEGER4} --- The number of elapsed days since 1st January 1900
\end{description}




\rule{\linewidth}{0.5pt}
\subsection*{\textsf{\colorbox{headtoc}{\color{white} FUNCTION}
FromDaysSince1900}}

\hypertarget{ecldoc:date.fromdayssince1900}{}
\hspace{0pt} \hyperlink{ecldoc:Date}{Date} \textbackslash 

{\renewcommand{\arraystretch}{1.5}
\begin{tabularx}{\textwidth}{|>{\raggedright\arraybackslash}l|X|}
\hline
\hspace{0pt}\mytexttt{\color{red} Date\_t} & \textbf{FromDaysSince1900} \\
\hline
\multicolumn{2}{|>{\raggedright\arraybackslash}X|}{\hspace{0pt}\mytexttt{\color{param} (Days\_t days)}} \\
\hline
\end{tabularx}
}

\par





Converts the number days since 1st January 1900 to a date in the Julian calendar.






\par
\begin{description}
\item [\colorbox{tagtype}{\color{white} \textbf{\textsf{PARAMETER}}}] \textbf{\underline{days}} ||| INTEGER4 --- The number of elapsed days since 1st Jan 1900
\end{description}







\par
\begin{description}
\item [\colorbox{tagtype}{\color{white} \textbf{\textsf{RETURN}}}] \textbf{UNSIGNED4} --- A Date\_t in the Julian calendar
\end{description}




\rule{\linewidth}{0.5pt}
\subsection*{\textsf{\colorbox{headtoc}{\color{white} FUNCTION}
YearsBetween}}

\hypertarget{ecldoc:date.yearsbetween}{}
\hspace{0pt} \hyperlink{ecldoc:Date}{Date} \textbackslash 

{\renewcommand{\arraystretch}{1.5}
\begin{tabularx}{\textwidth}{|>{\raggedright\arraybackslash}l|X|}
\hline
\hspace{0pt}\mytexttt{\color{red} INTEGER} & \textbf{YearsBetween} \\
\hline
\multicolumn{2}{|>{\raggedright\arraybackslash}X|}{\hspace{0pt}\mytexttt{\color{param} (Date\_t from, Date\_t to)}} \\
\hline
\end{tabularx}
}

\par





Calculate the number of whole years between two dates.






\par
\begin{description}
\item [\colorbox{tagtype}{\color{white} \textbf{\textsf{PARAMETER}}}] \textbf{\underline{from}} ||| UNSIGNED4 --- The first date
\item [\colorbox{tagtype}{\color{white} \textbf{\textsf{PARAMETER}}}] \textbf{\underline{to}} ||| UNSIGNED4 --- The last date
\end{description}







\par
\begin{description}
\item [\colorbox{tagtype}{\color{white} \textbf{\textsf{RETURN}}}] \textbf{INTEGER8} --- The number of years between them.
\end{description}




\rule{\linewidth}{0.5pt}
\subsection*{\textsf{\colorbox{headtoc}{\color{white} FUNCTION}
MonthsBetween}}

\hypertarget{ecldoc:date.monthsbetween}{}
\hspace{0pt} \hyperlink{ecldoc:Date}{Date} \textbackslash 

{\renewcommand{\arraystretch}{1.5}
\begin{tabularx}{\textwidth}{|>{\raggedright\arraybackslash}l|X|}
\hline
\hspace{0pt}\mytexttt{\color{red} INTEGER} & \textbf{MonthsBetween} \\
\hline
\multicolumn{2}{|>{\raggedright\arraybackslash}X|}{\hspace{0pt}\mytexttt{\color{param} (Date\_t from, Date\_t to)}} \\
\hline
\end{tabularx}
}

\par





Calculate the number of whole months between two dates.






\par
\begin{description}
\item [\colorbox{tagtype}{\color{white} \textbf{\textsf{PARAMETER}}}] \textbf{\underline{from}} ||| UNSIGNED4 --- The first date
\item [\colorbox{tagtype}{\color{white} \textbf{\textsf{PARAMETER}}}] \textbf{\underline{to}} ||| UNSIGNED4 --- The last date
\end{description}







\par
\begin{description}
\item [\colorbox{tagtype}{\color{white} \textbf{\textsf{RETURN}}}] \textbf{INTEGER8} --- The number of months between them.
\end{description}




\rule{\linewidth}{0.5pt}
\subsection*{\textsf{\colorbox{headtoc}{\color{white} FUNCTION}
DaysBetween}}

\hypertarget{ecldoc:date.daysbetween}{}
\hspace{0pt} \hyperlink{ecldoc:Date}{Date} \textbackslash 

{\renewcommand{\arraystretch}{1.5}
\begin{tabularx}{\textwidth}{|>{\raggedright\arraybackslash}l|X|}
\hline
\hspace{0pt}\mytexttt{\color{red} INTEGER} & \textbf{DaysBetween} \\
\hline
\multicolumn{2}{|>{\raggedright\arraybackslash}X|}{\hspace{0pt}\mytexttt{\color{param} (Date\_t from, Date\_t to)}} \\
\hline
\end{tabularx}
}

\par





Calculate the number of days between two dates.






\par
\begin{description}
\item [\colorbox{tagtype}{\color{white} \textbf{\textsf{PARAMETER}}}] \textbf{\underline{from}} ||| UNSIGNED4 --- The first date
\item [\colorbox{tagtype}{\color{white} \textbf{\textsf{PARAMETER}}}] \textbf{\underline{to}} ||| UNSIGNED4 --- The last date
\end{description}







\par
\begin{description}
\item [\colorbox{tagtype}{\color{white} \textbf{\textsf{RETURN}}}] \textbf{INTEGER8} --- The number of days between them.
\end{description}




\rule{\linewidth}{0.5pt}
\subsection*{\textsf{\colorbox{headtoc}{\color{white} FUNCTION}
DateFromDateRec}}

\hypertarget{ecldoc:date.datefromdaterec}{}
\hspace{0pt} \hyperlink{ecldoc:Date}{Date} \textbackslash 

{\renewcommand{\arraystretch}{1.5}
\begin{tabularx}{\textwidth}{|>{\raggedright\arraybackslash}l|X|}
\hline
\hspace{0pt}\mytexttt{\color{red} Date\_t} & \textbf{DateFromDateRec} \\
\hline
\multicolumn{2}{|>{\raggedright\arraybackslash}X|}{\hspace{0pt}\mytexttt{\color{param} (Date\_rec date)}} \\
\hline
\end{tabularx}
}

\par





Combines the fields from a Date\_rec to create a Date\_t






\par
\begin{description}
\item [\colorbox{tagtype}{\color{white} \textbf{\textsf{PARAMETER}}}] \textbf{\underline{date}} ||| ROW ( Date\_rec ) --- The row containing the date.
\end{description}







\par
\begin{description}
\item [\colorbox{tagtype}{\color{white} \textbf{\textsf{RETURN}}}] \textbf{UNSIGNED4} --- A Date\_t representing the combined values.
\end{description}




\rule{\linewidth}{0.5pt}
\subsection*{\textsf{\colorbox{headtoc}{\color{white} FUNCTION}
DateFromRec}}

\hypertarget{ecldoc:date.datefromrec}{}
\hspace{0pt} \hyperlink{ecldoc:Date}{Date} \textbackslash 

{\renewcommand{\arraystretch}{1.5}
\begin{tabularx}{\textwidth}{|>{\raggedright\arraybackslash}l|X|}
\hline
\hspace{0pt}\mytexttt{\color{red} Date\_t} & \textbf{DateFromRec} \\
\hline
\multicolumn{2}{|>{\raggedright\arraybackslash}X|}{\hspace{0pt}\mytexttt{\color{param} (Date\_rec date)}} \\
\hline
\end{tabularx}
}

\par





Combines the fields from a Date\_rec to create a Date\_t






\par
\begin{description}
\item [\colorbox{tagtype}{\color{white} \textbf{\textsf{PARAMETER}}}] \textbf{\underline{date}} ||| ROW ( Date\_rec ) --- The row containing the date.
\end{description}







\par
\begin{description}
\item [\colorbox{tagtype}{\color{white} \textbf{\textsf{RETURN}}}] \textbf{UNSIGNED4} --- A Date\_t representing the combined values.
\end{description}




\rule{\linewidth}{0.5pt}
\subsection*{\textsf{\colorbox{headtoc}{\color{white} FUNCTION}
TimeFromTimeRec}}

\hypertarget{ecldoc:date.timefromtimerec}{}
\hspace{0pt} \hyperlink{ecldoc:Date}{Date} \textbackslash 

{\renewcommand{\arraystretch}{1.5}
\begin{tabularx}{\textwidth}{|>{\raggedright\arraybackslash}l|X|}
\hline
\hspace{0pt}\mytexttt{\color{red} Time\_t} & \textbf{TimeFromTimeRec} \\
\hline
\multicolumn{2}{|>{\raggedright\arraybackslash}X|}{\hspace{0pt}\mytexttt{\color{param} (Time\_rec time)}} \\
\hline
\end{tabularx}
}

\par





Combines the fields from a Time\_rec to create a Time\_t






\par
\begin{description}
\item [\colorbox{tagtype}{\color{white} \textbf{\textsf{PARAMETER}}}] \textbf{\underline{time}} ||| ROW ( Time\_rec ) --- The row containing the time.
\end{description}







\par
\begin{description}
\item [\colorbox{tagtype}{\color{white} \textbf{\textsf{RETURN}}}] \textbf{UNSIGNED3} --- A Time\_t representing the combined values.
\end{description}




\rule{\linewidth}{0.5pt}
\subsection*{\textsf{\colorbox{headtoc}{\color{white} FUNCTION}
DateFromDateTimeRec}}

\hypertarget{ecldoc:date.datefromdatetimerec}{}
\hspace{0pt} \hyperlink{ecldoc:Date}{Date} \textbackslash 

{\renewcommand{\arraystretch}{1.5}
\begin{tabularx}{\textwidth}{|>{\raggedright\arraybackslash}l|X|}
\hline
\hspace{0pt}\mytexttt{\color{red} Date\_t} & \textbf{DateFromDateTimeRec} \\
\hline
\multicolumn{2}{|>{\raggedright\arraybackslash}X|}{\hspace{0pt}\mytexttt{\color{param} (DateTime\_rec datetime)}} \\
\hline
\end{tabularx}
}

\par





Combines the date fields from a DateTime\_rec to create a Date\_t






\par
\begin{description}
\item [\colorbox{tagtype}{\color{white} \textbf{\textsf{PARAMETER}}}] \textbf{\underline{datetime}} ||| ROW ( DateTime\_rec ) --- The row containing the datetime.
\end{description}







\par
\begin{description}
\item [\colorbox{tagtype}{\color{white} \textbf{\textsf{RETURN}}}] \textbf{UNSIGNED4} --- A Date\_t representing the combined values.
\end{description}




\rule{\linewidth}{0.5pt}
\subsection*{\textsf{\colorbox{headtoc}{\color{white} FUNCTION}
TimeFromDateTimeRec}}

\hypertarget{ecldoc:date.timefromdatetimerec}{}
\hspace{0pt} \hyperlink{ecldoc:Date}{Date} \textbackslash 

{\renewcommand{\arraystretch}{1.5}
\begin{tabularx}{\textwidth}{|>{\raggedright\arraybackslash}l|X|}
\hline
\hspace{0pt}\mytexttt{\color{red} Time\_t} & \textbf{TimeFromDateTimeRec} \\
\hline
\multicolumn{2}{|>{\raggedright\arraybackslash}X|}{\hspace{0pt}\mytexttt{\color{param} (DateTime\_rec datetime)}} \\
\hline
\end{tabularx}
}

\par





Combines the time fields from a DateTime\_rec to create a Time\_t






\par
\begin{description}
\item [\colorbox{tagtype}{\color{white} \textbf{\textsf{PARAMETER}}}] \textbf{\underline{datetime}} ||| ROW ( DateTime\_rec ) --- The row containing the datetime.
\end{description}







\par
\begin{description}
\item [\colorbox{tagtype}{\color{white} \textbf{\textsf{RETURN}}}] \textbf{UNSIGNED3} --- A Time\_t representing the combined values.
\end{description}




\rule{\linewidth}{0.5pt}
\subsection*{\textsf{\colorbox{headtoc}{\color{white} FUNCTION}
SecondsFromDateTimeRec}}

\hypertarget{ecldoc:date.secondsfromdatetimerec}{}
\hspace{0pt} \hyperlink{ecldoc:Date}{Date} \textbackslash 

{\renewcommand{\arraystretch}{1.5}
\begin{tabularx}{\textwidth}{|>{\raggedright\arraybackslash}l|X|}
\hline
\hspace{0pt}\mytexttt{\color{red} Seconds\_t} & \textbf{SecondsFromDateTimeRec} \\
\hline
\multicolumn{2}{|>{\raggedright\arraybackslash}X|}{\hspace{0pt}\mytexttt{\color{param} (DateTime\_rec datetime, BOOLEAN is\_local\_time = FALSE)}} \\
\hline
\end{tabularx}
}

\par





Combines the date and time fields from a DateTime\_rec to create a Seconds\_t






\par
\begin{description}
\item [\colorbox{tagtype}{\color{white} \textbf{\textsf{PARAMETER}}}] \textbf{\underline{datetime}} ||| ROW ( DateTime\_rec ) --- The row containing the datetime.
\item [\colorbox{tagtype}{\color{white} \textbf{\textsf{PARAMETER}}}] \textbf{\underline{is\_local\_time}} ||| BOOLEAN --- TRUE if the datetime components are expressed in local time rather than UTC, FALSE if the components are expressed in UTC. Optional, defaults to FALSE.
\end{description}







\par
\begin{description}
\item [\colorbox{tagtype}{\color{white} \textbf{\textsf{RETURN}}}] \textbf{INTEGER8} --- A Seconds\_t representing the combined values.
\end{description}




\rule{\linewidth}{0.5pt}
\subsection*{\textsf{\colorbox{headtoc}{\color{white} FUNCTION}
FromStringToDate}}

\hypertarget{ecldoc:date.fromstringtodate}{}
\hspace{0pt} \hyperlink{ecldoc:Date}{Date} \textbackslash 

{\renewcommand{\arraystretch}{1.5}
\begin{tabularx}{\textwidth}{|>{\raggedright\arraybackslash}l|X|}
\hline
\hspace{0pt}\mytexttt{\color{red} Date\_t} & \textbf{FromStringToDate} \\
\hline
\multicolumn{2}{|>{\raggedright\arraybackslash}X|}{\hspace{0pt}\mytexttt{\color{param} (STRING date\_text, VARSTRING format)}} \\
\hline
\end{tabularx}
}

\par





Converts a string to a Date\_t using the relevant string format. The resulting date must be representable within the Gregorian calendar after the year 1600.






\par
\begin{description}
\item [\colorbox{tagtype}{\color{white} \textbf{\textsf{PARAMETER}}}] \textbf{\underline{date\_text}} ||| STRING --- The string to be converted.
\item [\colorbox{tagtype}{\color{white} \textbf{\textsf{PARAMETER}}}] \textbf{\underline{format}} ||| VARSTRING --- The format of the input string. (See documentation for strftime)
\end{description}







\par
\begin{description}
\item [\colorbox{tagtype}{\color{white} \textbf{\textsf{RETURN}}}] \textbf{UNSIGNED4} --- The date that was matched in the string. Returns 0 if failed to match or if the date components match but the result is an invalid date. Supported characters: \%B Full month name \%b or \%h Abbreviated month name \%d Day of month (two digits) \%e Day of month (two digits, or a space followed by a single digit) \%m Month (two digits) \%t Whitespace \%y year within century (00-99) \%Y Full year (yyyy) \%j Julian day (1-366) Common date formats American '\%m/\%d/\%Y' mm/dd/yyyy Euro '\%d/\%m/\%Y' dd/mm/yyyy Iso format '\%Y-\%m-\%d' yyyy-mm-dd Iso basic 'Y\%m\%d' yyyymmdd '\%d-\%b-\%Y' dd-mon-yyyy e.g., '21-Mar-1954'
\end{description}




\rule{\linewidth}{0.5pt}
\subsection*{\textsf{\colorbox{headtoc}{\color{white} FUNCTION}
FromString}}

\hypertarget{ecldoc:date.fromstring}{}
\hspace{0pt} \hyperlink{ecldoc:Date}{Date} \textbackslash 

{\renewcommand{\arraystretch}{1.5}
\begin{tabularx}{\textwidth}{|>{\raggedright\arraybackslash}l|X|}
\hline
\hspace{0pt}\mytexttt{\color{red} Date\_t} & \textbf{FromString} \\
\hline
\multicolumn{2}{|>{\raggedright\arraybackslash}X|}{\hspace{0pt}\mytexttt{\color{param} (STRING date\_text, VARSTRING format)}} \\
\hline
\end{tabularx}
}

\par





Converts a string to a date using the relevant string format.






\par
\begin{description}
\item [\colorbox{tagtype}{\color{white} \textbf{\textsf{PARAMETER}}}] \textbf{\underline{date\_text}} ||| STRING --- The string to be converted.
\item [\colorbox{tagtype}{\color{white} \textbf{\textsf{PARAMETER}}}] \textbf{\underline{format}} ||| VARSTRING --- The format of the input string. (See documentation for strftime)
\end{description}







\par
\begin{description}
\item [\colorbox{tagtype}{\color{white} \textbf{\textsf{RETURN}}}] \textbf{UNSIGNED4} --- The date that was matched in the string. Returns 0 if failed to match.
\end{description}




\rule{\linewidth}{0.5pt}
\subsection*{\textsf{\colorbox{headtoc}{\color{white} FUNCTION}
FromStringToTime}}

\hypertarget{ecldoc:date.fromstringtotime}{}
\hspace{0pt} \hyperlink{ecldoc:Date}{Date} \textbackslash 

{\renewcommand{\arraystretch}{1.5}
\begin{tabularx}{\textwidth}{|>{\raggedright\arraybackslash}l|X|}
\hline
\hspace{0pt}\mytexttt{\color{red} Time\_t} & \textbf{FromStringToTime} \\
\hline
\multicolumn{2}{|>{\raggedright\arraybackslash}X|}{\hspace{0pt}\mytexttt{\color{param} (STRING time\_text, VARSTRING format)}} \\
\hline
\end{tabularx}
}

\par





Converts a string to a Time\_t using the relevant string format.






\par
\begin{description}
\item [\colorbox{tagtype}{\color{white} \textbf{\textsf{PARAMETER}}}] \textbf{\underline{date\_text}} |||  --- The string to be converted.
\item [\colorbox{tagtype}{\color{white} \textbf{\textsf{PARAMETER}}}] \textbf{\underline{format}} ||| VARSTRING --- The format of the input string. (See documentation for strftime)
\item [\colorbox{tagtype}{\color{white} \textbf{\textsf{PARAMETER}}}] \textbf{\underline{time\_text}} ||| STRING --- No Doc
\end{description}







\par
\begin{description}
\item [\colorbox{tagtype}{\color{white} \textbf{\textsf{RETURN}}}] \textbf{UNSIGNED3} --- The time that was matched in the string. Returns 0 if failed to match. Supported characters: \%H Hour (two digits) \%k (two digits, or a space followed by a single digit) \%M Minute (two digits) \%S Second (two digits) \%t Whitespace
\end{description}




\rule{\linewidth}{0.5pt}
\subsection*{\textsf{\colorbox{headtoc}{\color{white} FUNCTION}
MatchDateString}}

\hypertarget{ecldoc:date.matchdatestring}{}
\hspace{0pt} \hyperlink{ecldoc:Date}{Date} \textbackslash 

{\renewcommand{\arraystretch}{1.5}
\begin{tabularx}{\textwidth}{|>{\raggedright\arraybackslash}l|X|}
\hline
\hspace{0pt}\mytexttt{\color{red} Date\_t} & \textbf{MatchDateString} \\
\hline
\multicolumn{2}{|>{\raggedright\arraybackslash}X|}{\hspace{0pt}\mytexttt{\color{param} (STRING date\_text, SET OF VARSTRING formats)}} \\
\hline
\end{tabularx}
}

\par





Matches a string against a set of date string formats and returns a valid Date\_t object from the first format that successfully parses the string.






\par
\begin{description}
\item [\colorbox{tagtype}{\color{white} \textbf{\textsf{PARAMETER}}}] \textbf{\underline{formats}} ||| SET ( VARSTRING ) --- A set of formats to check against the string. (See documentation for strftime)
\item [\colorbox{tagtype}{\color{white} \textbf{\textsf{PARAMETER}}}] \textbf{\underline{date\_text}} ||| STRING --- The string to be converted.
\end{description}







\par
\begin{description}
\item [\colorbox{tagtype}{\color{white} \textbf{\textsf{RETURN}}}] \textbf{UNSIGNED4} --- The date that was matched in the string. Returns 0 if failed to match.
\end{description}




\rule{\linewidth}{0.5pt}
\subsection*{\textsf{\colorbox{headtoc}{\color{white} FUNCTION}
MatchTimeString}}

\hypertarget{ecldoc:date.matchtimestring}{}
\hspace{0pt} \hyperlink{ecldoc:Date}{Date} \textbackslash 

{\renewcommand{\arraystretch}{1.5}
\begin{tabularx}{\textwidth}{|>{\raggedright\arraybackslash}l|X|}
\hline
\hspace{0pt}\mytexttt{\color{red} Time\_t} & \textbf{MatchTimeString} \\
\hline
\multicolumn{2}{|>{\raggedright\arraybackslash}X|}{\hspace{0pt}\mytexttt{\color{param} (STRING time\_text, SET OF VARSTRING formats)}} \\
\hline
\end{tabularx}
}

\par





Matches a string against a set of time string formats and returns a valid Time\_t object from the first format that successfully parses the string.






\par
\begin{description}
\item [\colorbox{tagtype}{\color{white} \textbf{\textsf{PARAMETER}}}] \textbf{\underline{formats}} ||| SET ( VARSTRING ) --- A set of formats to check against the string. (See documentation for strftime)
\item [\colorbox{tagtype}{\color{white} \textbf{\textsf{PARAMETER}}}] \textbf{\underline{time\_text}} ||| STRING --- The string to be converted.
\end{description}







\par
\begin{description}
\item [\colorbox{tagtype}{\color{white} \textbf{\textsf{RETURN}}}] \textbf{UNSIGNED3} --- The time that was matched in the string. Returns 0 if failed to match.
\end{description}




\rule{\linewidth}{0.5pt}
\subsection*{\textsf{\colorbox{headtoc}{\color{white} FUNCTION}
DateToString}}

\hypertarget{ecldoc:date.datetostring}{}
\hspace{0pt} \hyperlink{ecldoc:Date}{Date} \textbackslash 

{\renewcommand{\arraystretch}{1.5}
\begin{tabularx}{\textwidth}{|>{\raggedright\arraybackslash}l|X|}
\hline
\hspace{0pt}\mytexttt{\color{red} STRING} & \textbf{DateToString} \\
\hline
\multicolumn{2}{|>{\raggedright\arraybackslash}X|}{\hspace{0pt}\mytexttt{\color{param} (Date\_t date, VARSTRING format = '\%Y-\%m-\%d')}} \\
\hline
\end{tabularx}
}

\par





Formats a date as a string.






\par
\begin{description}
\item [\colorbox{tagtype}{\color{white} \textbf{\textsf{PARAMETER}}}] \textbf{\underline{date}} ||| UNSIGNED4 --- The date to be converted.
\item [\colorbox{tagtype}{\color{white} \textbf{\textsf{PARAMETER}}}] \textbf{\underline{format}} ||| VARSTRING --- The format template to use for the conversion; see strftime() for appropriate values. The maximum length of the resulting string is 255 characters. Optional; defaults to '\%Y-\%m-\%d' which is YYYY-MM-DD.
\end{description}







\par
\begin{description}
\item [\colorbox{tagtype}{\color{white} \textbf{\textsf{RETURN}}}] \textbf{STRING} --- Blank if date cannot be formatted, or the date in the requested format.
\end{description}




\rule{\linewidth}{0.5pt}
\subsection*{\textsf{\colorbox{headtoc}{\color{white} FUNCTION}
TimeToString}}

\hypertarget{ecldoc:date.timetostring}{}
\hspace{0pt} \hyperlink{ecldoc:Date}{Date} \textbackslash 

{\renewcommand{\arraystretch}{1.5}
\begin{tabularx}{\textwidth}{|>{\raggedright\arraybackslash}l|X|}
\hline
\hspace{0pt}\mytexttt{\color{red} STRING} & \textbf{TimeToString} \\
\hline
\multicolumn{2}{|>{\raggedright\arraybackslash}X|}{\hspace{0pt}\mytexttt{\color{param} (Time\_t time, VARSTRING format = '\%H:\%M:\%S')}} \\
\hline
\end{tabularx}
}

\par





Formats a time as a string.






\par
\begin{description}
\item [\colorbox{tagtype}{\color{white} \textbf{\textsf{PARAMETER}}}] \textbf{\underline{time}} ||| UNSIGNED3 --- The time to be converted.
\item [\colorbox{tagtype}{\color{white} \textbf{\textsf{PARAMETER}}}] \textbf{\underline{format}} ||| VARSTRING --- The format template to use for the conversion; see strftime() for appropriate values. The maximum length of the resulting string is 255 characters. Optional; defaults to '\%H:\%M:\%S' which is HH:MM:SS.
\end{description}







\par
\begin{description}
\item [\colorbox{tagtype}{\color{white} \textbf{\textsf{RETURN}}}] \textbf{STRING} --- Blank if the time cannot be formatted, or the time in the requested format.
\end{description}




\rule{\linewidth}{0.5pt}
\subsection*{\textsf{\colorbox{headtoc}{\color{white} FUNCTION}
SecondsToString}}

\hypertarget{ecldoc:date.secondstostring}{}
\hspace{0pt} \hyperlink{ecldoc:Date}{Date} \textbackslash 

{\renewcommand{\arraystretch}{1.5}
\begin{tabularx}{\textwidth}{|>{\raggedright\arraybackslash}l|X|}
\hline
\hspace{0pt}\mytexttt{\color{red} STRING} & \textbf{SecondsToString} \\
\hline
\multicolumn{2}{|>{\raggedright\arraybackslash}X|}{\hspace{0pt}\mytexttt{\color{param} (Seconds\_t seconds, VARSTRING format = '\%Y-\%m-\%dT\%H:\%M:\%S')}} \\
\hline
\end{tabularx}
}

\par





Converts a Seconds\_t value into a human-readable string using a format template.






\par
\begin{description}
\item [\colorbox{tagtype}{\color{white} \textbf{\textsf{PARAMETER}}}] \textbf{\underline{format}} ||| VARSTRING --- The format template to use for the conversion; see strftime() for appropriate values. The maximum length of the resulting string is 255 characters. Optional; defaults to '\%Y-\%m-\%dT\%H:\%M:\%S' which is YYYY-MM-DDTHH:MM:SS.
\item [\colorbox{tagtype}{\color{white} \textbf{\textsf{PARAMETER}}}] \textbf{\underline{seconds}} ||| INTEGER8 --- The seconds since epoch.
\end{description}







\par
\begin{description}
\item [\colorbox{tagtype}{\color{white} \textbf{\textsf{RETURN}}}] \textbf{STRING} --- The converted seconds as a string.
\end{description}




\rule{\linewidth}{0.5pt}
\subsection*{\textsf{\colorbox{headtoc}{\color{white} FUNCTION}
ToString}}

\hypertarget{ecldoc:date.tostring}{}
\hspace{0pt} \hyperlink{ecldoc:Date}{Date} \textbackslash 

{\renewcommand{\arraystretch}{1.5}
\begin{tabularx}{\textwidth}{|>{\raggedright\arraybackslash}l|X|}
\hline
\hspace{0pt}\mytexttt{\color{red} STRING} & \textbf{ToString} \\
\hline
\multicolumn{2}{|>{\raggedright\arraybackslash}X|}{\hspace{0pt}\mytexttt{\color{param} (Date\_t date, VARSTRING format)}} \\
\hline
\end{tabularx}
}

\par





Formats a date as a string.






\par
\begin{description}
\item [\colorbox{tagtype}{\color{white} \textbf{\textsf{PARAMETER}}}] \textbf{\underline{date}} ||| UNSIGNED4 --- The date to be converted.
\item [\colorbox{tagtype}{\color{white} \textbf{\textsf{PARAMETER}}}] \textbf{\underline{format}} ||| VARSTRING --- The format the date is output in. (See documentation for strftime)
\end{description}







\par
\begin{description}
\item [\colorbox{tagtype}{\color{white} \textbf{\textsf{RETURN}}}] \textbf{STRING} --- Blank if date cannot be formatted, or the date in the requested format.
\end{description}




\rule{\linewidth}{0.5pt}
\subsection*{\textsf{\colorbox{headtoc}{\color{white} FUNCTION}
ConvertDateFormat}}

\hypertarget{ecldoc:date.convertdateformat}{}
\hspace{0pt} \hyperlink{ecldoc:Date}{Date} \textbackslash 

{\renewcommand{\arraystretch}{1.5}
\begin{tabularx}{\textwidth}{|>{\raggedright\arraybackslash}l|X|}
\hline
\hspace{0pt}\mytexttt{\color{red} STRING} & \textbf{ConvertDateFormat} \\
\hline
\multicolumn{2}{|>{\raggedright\arraybackslash}X|}{\hspace{0pt}\mytexttt{\color{param} (STRING date\_text, VARSTRING from\_format='\%m/\%d/\%Y', VARSTRING to\_format='\%Y\%m\%d')}} \\
\hline
\end{tabularx}
}

\par





Converts a date from one format to another






\par
\begin{description}
\item [\colorbox{tagtype}{\color{white} \textbf{\textsf{PARAMETER}}}] \textbf{\underline{from\_format}} ||| VARSTRING --- The format the date is to be converted from.
\item [\colorbox{tagtype}{\color{white} \textbf{\textsf{PARAMETER}}}] \textbf{\underline{date\_text}} ||| STRING --- The string containing the date to be converted.
\item [\colorbox{tagtype}{\color{white} \textbf{\textsf{PARAMETER}}}] \textbf{\underline{to\_format}} ||| VARSTRING --- The format the date is to be converted to.
\end{description}







\par
\begin{description}
\item [\colorbox{tagtype}{\color{white} \textbf{\textsf{RETURN}}}] \textbf{STRING} --- The converted string, or blank if it failed to match the format.
\end{description}




\rule{\linewidth}{0.5pt}
\subsection*{\textsf{\colorbox{headtoc}{\color{white} FUNCTION}
ConvertFormat}}

\hypertarget{ecldoc:date.convertformat}{}
\hspace{0pt} \hyperlink{ecldoc:Date}{Date} \textbackslash 

{\renewcommand{\arraystretch}{1.5}
\begin{tabularx}{\textwidth}{|>{\raggedright\arraybackslash}l|X|}
\hline
\hspace{0pt}\mytexttt{\color{red} STRING} & \textbf{ConvertFormat} \\
\hline
\multicolumn{2}{|>{\raggedright\arraybackslash}X|}{\hspace{0pt}\mytexttt{\color{param} (STRING date\_text, VARSTRING from\_format='\%m/\%d/\%Y', VARSTRING to\_format='\%Y\%m\%d')}} \\
\hline
\end{tabularx}
}

\par





Converts a date from one format to another






\par
\begin{description}
\item [\colorbox{tagtype}{\color{white} \textbf{\textsf{PARAMETER}}}] \textbf{\underline{from\_format}} ||| VARSTRING --- The format the date is to be converted from.
\item [\colorbox{tagtype}{\color{white} \textbf{\textsf{PARAMETER}}}] \textbf{\underline{date\_text}} ||| STRING --- The string containing the date to be converted.
\item [\colorbox{tagtype}{\color{white} \textbf{\textsf{PARAMETER}}}] \textbf{\underline{to\_format}} ||| VARSTRING --- The format the date is to be converted to.
\end{description}







\par
\begin{description}
\item [\colorbox{tagtype}{\color{white} \textbf{\textsf{RETURN}}}] \textbf{STRING} --- The converted string, or blank if it failed to match the format.
\end{description}




\rule{\linewidth}{0.5pt}
\subsection*{\textsf{\colorbox{headtoc}{\color{white} FUNCTION}
ConvertTimeFormat}}

\hypertarget{ecldoc:date.converttimeformat}{}
\hspace{0pt} \hyperlink{ecldoc:Date}{Date} \textbackslash 

{\renewcommand{\arraystretch}{1.5}
\begin{tabularx}{\textwidth}{|>{\raggedright\arraybackslash}l|X|}
\hline
\hspace{0pt}\mytexttt{\color{red} STRING} & \textbf{ConvertTimeFormat} \\
\hline
\multicolumn{2}{|>{\raggedright\arraybackslash}X|}{\hspace{0pt}\mytexttt{\color{param} (STRING time\_text, VARSTRING from\_format='\%H\%M\%S', VARSTRING to\_format='\%H:\%M:\%S')}} \\
\hline
\end{tabularx}
}

\par





Converts a time from one format to another






\par
\begin{description}
\item [\colorbox{tagtype}{\color{white} \textbf{\textsf{PARAMETER}}}] \textbf{\underline{from\_format}} ||| VARSTRING --- The format the time is to be converted from.
\item [\colorbox{tagtype}{\color{white} \textbf{\textsf{PARAMETER}}}] \textbf{\underline{to\_format}} ||| VARSTRING --- The format the time is to be converted to.
\item [\colorbox{tagtype}{\color{white} \textbf{\textsf{PARAMETER}}}] \textbf{\underline{time\_text}} ||| STRING --- The string containing the time to be converted.
\end{description}







\par
\begin{description}
\item [\colorbox{tagtype}{\color{white} \textbf{\textsf{RETURN}}}] \textbf{STRING} --- The converted string, or blank if it failed to match the format.
\end{description}




\rule{\linewidth}{0.5pt}
\subsection*{\textsf{\colorbox{headtoc}{\color{white} FUNCTION}
ConvertDateFormatMultiple}}

\hypertarget{ecldoc:date.convertdateformatmultiple}{}
\hspace{0pt} \hyperlink{ecldoc:Date}{Date} \textbackslash 

{\renewcommand{\arraystretch}{1.5}
\begin{tabularx}{\textwidth}{|>{\raggedright\arraybackslash}l|X|}
\hline
\hspace{0pt}\mytexttt{\color{red} STRING} & \textbf{ConvertDateFormatMultiple} \\
\hline
\multicolumn{2}{|>{\raggedright\arraybackslash}X|}{\hspace{0pt}\mytexttt{\color{param} (STRING date\_text, SET OF VARSTRING from\_formats, VARSTRING to\_format='\%Y\%m\%d')}} \\
\hline
\end{tabularx}
}

\par





Converts a date that matches one of a set of formats to another.






\par
\begin{description}
\item [\colorbox{tagtype}{\color{white} \textbf{\textsf{PARAMETER}}}] \textbf{\underline{from\_formats}} ||| SET ( VARSTRING ) --- The list of formats the date is to be converted from.
\item [\colorbox{tagtype}{\color{white} \textbf{\textsf{PARAMETER}}}] \textbf{\underline{date\_text}} ||| STRING --- The string containing the date to be converted.
\item [\colorbox{tagtype}{\color{white} \textbf{\textsf{PARAMETER}}}] \textbf{\underline{to\_format}} ||| VARSTRING --- The format the date is to be converted to.
\end{description}







\par
\begin{description}
\item [\colorbox{tagtype}{\color{white} \textbf{\textsf{RETURN}}}] \textbf{STRING} --- The converted string, or blank if it failed to match the format.
\end{description}




\rule{\linewidth}{0.5pt}
\subsection*{\textsf{\colorbox{headtoc}{\color{white} FUNCTION}
ConvertFormatMultiple}}

\hypertarget{ecldoc:date.convertformatmultiple}{}
\hspace{0pt} \hyperlink{ecldoc:Date}{Date} \textbackslash 

{\renewcommand{\arraystretch}{1.5}
\begin{tabularx}{\textwidth}{|>{\raggedright\arraybackslash}l|X|}
\hline
\hspace{0pt}\mytexttt{\color{red} STRING} & \textbf{ConvertFormatMultiple} \\
\hline
\multicolumn{2}{|>{\raggedright\arraybackslash}X|}{\hspace{0pt}\mytexttt{\color{param} (STRING date\_text, SET OF VARSTRING from\_formats, VARSTRING to\_format='\%Y\%m\%d')}} \\
\hline
\end{tabularx}
}

\par





Converts a date that matches one of a set of formats to another.






\par
\begin{description}
\item [\colorbox{tagtype}{\color{white} \textbf{\textsf{PARAMETER}}}] \textbf{\underline{from\_formats}} ||| SET ( VARSTRING ) --- The list of formats the date is to be converted from.
\item [\colorbox{tagtype}{\color{white} \textbf{\textsf{PARAMETER}}}] \textbf{\underline{date\_text}} ||| STRING --- The string containing the date to be converted.
\item [\colorbox{tagtype}{\color{white} \textbf{\textsf{PARAMETER}}}] \textbf{\underline{to\_format}} ||| VARSTRING --- The format the date is to be converted to.
\end{description}







\par
\begin{description}
\item [\colorbox{tagtype}{\color{white} \textbf{\textsf{RETURN}}}] \textbf{STRING} --- The converted string, or blank if it failed to match the format.
\end{description}




\rule{\linewidth}{0.5pt}
\subsection*{\textsf{\colorbox{headtoc}{\color{white} FUNCTION}
ConvertTimeFormatMultiple}}

\hypertarget{ecldoc:date.converttimeformatmultiple}{}
\hspace{0pt} \hyperlink{ecldoc:Date}{Date} \textbackslash 

{\renewcommand{\arraystretch}{1.5}
\begin{tabularx}{\textwidth}{|>{\raggedright\arraybackslash}l|X|}
\hline
\hspace{0pt}\mytexttt{\color{red} STRING} & \textbf{ConvertTimeFormatMultiple} \\
\hline
\multicolumn{2}{|>{\raggedright\arraybackslash}X|}{\hspace{0pt}\mytexttt{\color{param} (STRING time\_text, SET OF VARSTRING from\_formats, VARSTRING to\_format='\%H:\%m:\%s')}} \\
\hline
\end{tabularx}
}

\par





Converts a time that matches one of a set of formats to another.






\par
\begin{description}
\item [\colorbox{tagtype}{\color{white} \textbf{\textsf{PARAMETER}}}] \textbf{\underline{from\_formats}} ||| SET ( VARSTRING ) --- The list of formats the time is to be converted from.
\item [\colorbox{tagtype}{\color{white} \textbf{\textsf{PARAMETER}}}] \textbf{\underline{to\_format}} ||| VARSTRING --- The format the time is to be converted to.
\item [\colorbox{tagtype}{\color{white} \textbf{\textsf{PARAMETER}}}] \textbf{\underline{time\_text}} ||| STRING --- The string containing the time to be converted.
\end{description}







\par
\begin{description}
\item [\colorbox{tagtype}{\color{white} \textbf{\textsf{RETURN}}}] \textbf{STRING} --- The converted string, or blank if it failed to match the format.
\end{description}




\rule{\linewidth}{0.5pt}
\subsection*{\textsf{\colorbox{headtoc}{\color{white} FUNCTION}
AdjustDate}}

\hypertarget{ecldoc:date.adjustdate}{}
\hspace{0pt} \hyperlink{ecldoc:Date}{Date} \textbackslash 

{\renewcommand{\arraystretch}{1.5}
\begin{tabularx}{\textwidth}{|>{\raggedright\arraybackslash}l|X|}
\hline
\hspace{0pt}\mytexttt{\color{red} Date\_t} & \textbf{AdjustDate} \\
\hline
\multicolumn{2}{|>{\raggedright\arraybackslash}X|}{\hspace{0pt}\mytexttt{\color{param} (Date\_t date, INTEGER2 year\_delta = 0, INTEGER4 month\_delta = 0, INTEGER4 day\_delta = 0)}} \\
\hline
\end{tabularx}
}

\par





Adjusts a date by incrementing or decrementing year, month and/or day values. The date must be in the Gregorian calendar after the year 1600. If the new calculated date is invalid then it will be normalized according to mktime() rules. Example: 20140130 + 1 month = 20140302.






\par
\begin{description}
\item [\colorbox{tagtype}{\color{white} \textbf{\textsf{PARAMETER}}}] \textbf{\underline{date}} ||| UNSIGNED4 --- The date to adjust.
\item [\colorbox{tagtype}{\color{white} \textbf{\textsf{PARAMETER}}}] \textbf{\underline{month\_delta}} ||| INTEGER4 --- The requested change to the month value; optional, defaults to zero.
\item [\colorbox{tagtype}{\color{white} \textbf{\textsf{PARAMETER}}}] \textbf{\underline{year\_delta}} ||| INTEGER2 --- The requested change to the year value; optional, defaults to zero.
\item [\colorbox{tagtype}{\color{white} \textbf{\textsf{PARAMETER}}}] \textbf{\underline{day\_delta}} ||| INTEGER4 --- The requested change to the day of month value; optional, defaults to zero.
\end{description}







\par
\begin{description}
\item [\colorbox{tagtype}{\color{white} \textbf{\textsf{RETURN}}}] \textbf{UNSIGNED4} --- The adjusted Date\_t value.
\end{description}




\rule{\linewidth}{0.5pt}
\subsection*{\textsf{\colorbox{headtoc}{\color{white} FUNCTION}
AdjustDateBySeconds}}

\hypertarget{ecldoc:date.adjustdatebyseconds}{}
\hspace{0pt} \hyperlink{ecldoc:Date}{Date} \textbackslash 

{\renewcommand{\arraystretch}{1.5}
\begin{tabularx}{\textwidth}{|>{\raggedright\arraybackslash}l|X|}
\hline
\hspace{0pt}\mytexttt{\color{red} Date\_t} & \textbf{AdjustDateBySeconds} \\
\hline
\multicolumn{2}{|>{\raggedright\arraybackslash}X|}{\hspace{0pt}\mytexttt{\color{param} (Date\_t date, INTEGER4 seconds\_delta)}} \\
\hline
\end{tabularx}
}

\par





Adjusts a date by adding or subtracting seconds. The date must be in the Gregorian calendar after the year 1600. If the new calculated date is invalid then it will be normalized according to mktime() rules. Example: 20140130 + 172800 seconds = 20140201.






\par
\begin{description}
\item [\colorbox{tagtype}{\color{white} \textbf{\textsf{PARAMETER}}}] \textbf{\underline{date}} ||| UNSIGNED4 --- The date to adjust.
\item [\colorbox{tagtype}{\color{white} \textbf{\textsf{PARAMETER}}}] \textbf{\underline{seconds\_delta}} ||| INTEGER4 --- The requested change to the date, in seconds.
\end{description}







\par
\begin{description}
\item [\colorbox{tagtype}{\color{white} \textbf{\textsf{RETURN}}}] \textbf{UNSIGNED4} --- The adjusted Date\_t value.
\end{description}




\rule{\linewidth}{0.5pt}
\subsection*{\textsf{\colorbox{headtoc}{\color{white} FUNCTION}
AdjustTime}}

\hypertarget{ecldoc:date.adjusttime}{}
\hspace{0pt} \hyperlink{ecldoc:Date}{Date} \textbackslash 

{\renewcommand{\arraystretch}{1.5}
\begin{tabularx}{\textwidth}{|>{\raggedright\arraybackslash}l|X|}
\hline
\hspace{0pt}\mytexttt{\color{red} Time\_t} & \textbf{AdjustTime} \\
\hline
\multicolumn{2}{|>{\raggedright\arraybackslash}X|}{\hspace{0pt}\mytexttt{\color{param} (Time\_t time, INTEGER2 hour\_delta = 0, INTEGER4 minute\_delta = 0, INTEGER4 second\_delta = 0)}} \\
\hline
\end{tabularx}
}

\par





Adjusts a time by incrementing or decrementing hour, minute and/or second values. If the new calculated time is invalid then it will be normalized according to mktime() rules.






\par
\begin{description}
\item [\colorbox{tagtype}{\color{white} \textbf{\textsf{PARAMETER}}}] \textbf{\underline{time}} ||| UNSIGNED3 --- The time to adjust.
\item [\colorbox{tagtype}{\color{white} \textbf{\textsf{PARAMETER}}}] \textbf{\underline{second\_delta}} ||| INTEGER4 --- The requested change to the second of month value; optional, defaults to zero.
\item [\colorbox{tagtype}{\color{white} \textbf{\textsf{PARAMETER}}}] \textbf{\underline{hour\_delta}} ||| INTEGER2 --- The requested change to the hour value; optional, defaults to zero.
\item [\colorbox{tagtype}{\color{white} \textbf{\textsf{PARAMETER}}}] \textbf{\underline{minute\_delta}} ||| INTEGER4 --- The requested change to the minute value; optional, defaults to zero.
\end{description}







\par
\begin{description}
\item [\colorbox{tagtype}{\color{white} \textbf{\textsf{RETURN}}}] \textbf{UNSIGNED3} --- The adjusted Time\_t value.
\end{description}




\rule{\linewidth}{0.5pt}
\subsection*{\textsf{\colorbox{headtoc}{\color{white} FUNCTION}
AdjustTimeBySeconds}}

\hypertarget{ecldoc:date.adjusttimebyseconds}{}
\hspace{0pt} \hyperlink{ecldoc:Date}{Date} \textbackslash 

{\renewcommand{\arraystretch}{1.5}
\begin{tabularx}{\textwidth}{|>{\raggedright\arraybackslash}l|X|}
\hline
\hspace{0pt}\mytexttt{\color{red} Time\_t} & \textbf{AdjustTimeBySeconds} \\
\hline
\multicolumn{2}{|>{\raggedright\arraybackslash}X|}{\hspace{0pt}\mytexttt{\color{param} (Time\_t time, INTEGER4 seconds\_delta)}} \\
\hline
\end{tabularx}
}

\par





Adjusts a time by adding or subtracting seconds. If the new calculated time is invalid then it will be normalized according to mktime() rules.






\par
\begin{description}
\item [\colorbox{tagtype}{\color{white} \textbf{\textsf{PARAMETER}}}] \textbf{\underline{time}} ||| UNSIGNED3 --- The time to adjust.
\item [\colorbox{tagtype}{\color{white} \textbf{\textsf{PARAMETER}}}] \textbf{\underline{seconds\_delta}} ||| INTEGER4 --- The requested change to the time, in seconds.
\end{description}







\par
\begin{description}
\item [\colorbox{tagtype}{\color{white} \textbf{\textsf{RETURN}}}] \textbf{UNSIGNED3} --- The adjusted Time\_t value.
\end{description}




\rule{\linewidth}{0.5pt}
\subsection*{\textsf{\colorbox{headtoc}{\color{white} FUNCTION}
AdjustSeconds}}

\hypertarget{ecldoc:date.adjustseconds}{}
\hspace{0pt} \hyperlink{ecldoc:Date}{Date} \textbackslash 

{\renewcommand{\arraystretch}{1.5}
\begin{tabularx}{\textwidth}{|>{\raggedright\arraybackslash}l|X|}
\hline
\hspace{0pt}\mytexttt{\color{red} Seconds\_t} & \textbf{AdjustSeconds} \\
\hline
\multicolumn{2}{|>{\raggedright\arraybackslash}X|}{\hspace{0pt}\mytexttt{\color{param} (Seconds\_t seconds, INTEGER2 year\_delta = 0, INTEGER4 month\_delta = 0, INTEGER4 day\_delta = 0, INTEGER4 hour\_delta = 0, INTEGER4 minute\_delta = 0, INTEGER4 second\_delta = 0)}} \\
\hline
\end{tabularx}
}

\par





Adjusts a Seconds\_t value by adding or subtracting years, months, days, hours, minutes and/or seconds. This is performed by first converting the seconds into a full date/time structure, applying any delta values to individual date/time components, then converting the structure back to the number of seconds. This interim date must lie within Gregorian calendar after the year 1600. If the interim structure is found to have an invalid date/time then it will be normalized according to mktime() rules. Therefore, some delta values (such as ''1 month'') are actually relative to the value of the seconds argument.






\par
\begin{description}
\item [\colorbox{tagtype}{\color{white} \textbf{\textsf{PARAMETER}}}] \textbf{\underline{month\_delta}} ||| INTEGER4 --- The requested change to the month value; optional, defaults to zero.
\item [\colorbox{tagtype}{\color{white} \textbf{\textsf{PARAMETER}}}] \textbf{\underline{second\_delta}} ||| INTEGER4 --- The requested change to the second of month value; optional, defaults to zero.
\item [\colorbox{tagtype}{\color{white} \textbf{\textsf{PARAMETER}}}] \textbf{\underline{seconds}} ||| INTEGER8 --- The number of seconds to adjust.
\item [\colorbox{tagtype}{\color{white} \textbf{\textsf{PARAMETER}}}] \textbf{\underline{day\_delta}} ||| INTEGER4 --- The requested change to the day of month value; optional, defaults to zero.
\item [\colorbox{tagtype}{\color{white} \textbf{\textsf{PARAMETER}}}] \textbf{\underline{minute\_delta}} ||| INTEGER4 --- The requested change to the minute value; optional, defaults to zero.
\item [\colorbox{tagtype}{\color{white} \textbf{\textsf{PARAMETER}}}] \textbf{\underline{year\_delta}} ||| INTEGER2 --- The requested change to the year value; optional, defaults to zero.
\item [\colorbox{tagtype}{\color{white} \textbf{\textsf{PARAMETER}}}] \textbf{\underline{hour\_delta}} ||| INTEGER4 --- The requested change to the hour value; optional, defaults to zero.
\end{description}







\par
\begin{description}
\item [\colorbox{tagtype}{\color{white} \textbf{\textsf{RETURN}}}] \textbf{INTEGER8} --- The adjusted Seconds\_t value.
\end{description}




\rule{\linewidth}{0.5pt}
\subsection*{\textsf{\colorbox{headtoc}{\color{white} FUNCTION}
AdjustCalendar}}

\hypertarget{ecldoc:date.adjustcalendar}{}
\hspace{0pt} \hyperlink{ecldoc:Date}{Date} \textbackslash 

{\renewcommand{\arraystretch}{1.5}
\begin{tabularx}{\textwidth}{|>{\raggedright\arraybackslash}l|X|}
\hline
\hspace{0pt}\mytexttt{\color{red} Date\_t} & \textbf{AdjustCalendar} \\
\hline
\multicolumn{2}{|>{\raggedright\arraybackslash}X|}{\hspace{0pt}\mytexttt{\color{param} (Date\_t date, INTEGER2 year\_delta = 0, INTEGER4 month\_delta = 0, INTEGER4 day\_delta = 0)}} \\
\hline
\end{tabularx}
}

\par





Adjusts a date by incrementing or decrementing months and/or years. This routine uses the rule outlined in McGinn v. State, 46 Neb. 427, 65 N.W. 46 (1895): ''The term calendar month, whether employed in statutes or contracts, and not appearing to have been used in a different sense, denotes a period terminating with the day of the succeeding month numerically corresponding to the day of its beginning, less one. If there be no corresponding day of the succeeding month, it terminates with the last day thereof.'' The internet suggests similar legal positions exist in the Commonwealth and Germany. Note that day adjustments are performed after year and month adjustments using the preceding rules. As an example, Jan. 31, 2014 + 1 month will result in Feb. 28, 2014; Jan. 31, 2014 + 1 month + 1 day will result in Mar. 1, 2014.






\par
\begin{description}
\item [\colorbox{tagtype}{\color{white} \textbf{\textsf{PARAMETER}}}] \textbf{\underline{date}} ||| UNSIGNED4 --- The date to adjust, in the Gregorian calendar after 1600.
\item [\colorbox{tagtype}{\color{white} \textbf{\textsf{PARAMETER}}}] \textbf{\underline{month\_delta}} ||| INTEGER4 --- The requested change to the month value; optional, defaults to zero.
\item [\colorbox{tagtype}{\color{white} \textbf{\textsf{PARAMETER}}}] \textbf{\underline{year\_delta}} ||| INTEGER2 --- The requested change to the year value; optional, defaults to zero.
\item [\colorbox{tagtype}{\color{white} \textbf{\textsf{PARAMETER}}}] \textbf{\underline{day\_delta}} ||| INTEGER4 --- The requested change to the day value; optional, defaults to zero.
\end{description}







\par
\begin{description}
\item [\colorbox{tagtype}{\color{white} \textbf{\textsf{RETURN}}}] \textbf{UNSIGNED4} --- The adjusted Date\_t value.
\end{description}




\rule{\linewidth}{0.5pt}
\subsection*{\textsf{\colorbox{headtoc}{\color{white} FUNCTION}
IsLocalDaylightSavingsInEffect}}

\hypertarget{ecldoc:date.islocaldaylightsavingsineffect}{}
\hspace{0pt} \hyperlink{ecldoc:Date}{Date} \textbackslash 

{\renewcommand{\arraystretch}{1.5}
\begin{tabularx}{\textwidth}{|>{\raggedright\arraybackslash}l|X|}
\hline
\hspace{0pt}\mytexttt{\color{red} BOOLEAN} & \textbf{IsLocalDaylightSavingsInEffect} \\
\hline
\multicolumn{2}{|>{\raggedright\arraybackslash}X|}{\hspace{0pt}\mytexttt{\color{param} ()}} \\
\hline
\end{tabularx}
}

\par





Returns a boolean indicating whether daylight savings time is currently in effect locally.








\par
\begin{description}
\item [\colorbox{tagtype}{\color{white} \textbf{\textsf{RETURN}}}] \textbf{BOOLEAN} --- TRUE if daylight savings time is currently in effect, FALSE otherwise.
\end{description}




\rule{\linewidth}{0.5pt}
\subsection*{\textsf{\colorbox{headtoc}{\color{white} FUNCTION}
LocalTimeZoneOffset}}

\hypertarget{ecldoc:date.localtimezoneoffset}{}
\hspace{0pt} \hyperlink{ecldoc:Date}{Date} \textbackslash 

{\renewcommand{\arraystretch}{1.5}
\begin{tabularx}{\textwidth}{|>{\raggedright\arraybackslash}l|X|}
\hline
\hspace{0pt}\mytexttt{\color{red} INTEGER4} & \textbf{LocalTimeZoneOffset} \\
\hline
\multicolumn{2}{|>{\raggedright\arraybackslash}X|}{\hspace{0pt}\mytexttt{\color{param} ()}} \\
\hline
\end{tabularx}
}

\par





Returns the offset (in seconds) of the time represented from UTC, with positive values indicating locations east of the Prime Meridian. Given a UTC time in seconds since epoch, you can find the local time by adding the result of this function to the seconds.








\par
\begin{description}
\item [\colorbox{tagtype}{\color{white} \textbf{\textsf{RETURN}}}] \textbf{INTEGER4} --- The number of seconds offset from UTC.
\end{description}




\rule{\linewidth}{0.5pt}
\subsection*{\textsf{\colorbox{headtoc}{\color{white} FUNCTION}
CurrentDate}}

\hypertarget{ecldoc:date.currentdate}{}
\hspace{0pt} \hyperlink{ecldoc:Date}{Date} \textbackslash 

{\renewcommand{\arraystretch}{1.5}
\begin{tabularx}{\textwidth}{|>{\raggedright\arraybackslash}l|X|}
\hline
\hspace{0pt}\mytexttt{\color{red} Date\_t} & \textbf{CurrentDate} \\
\hline
\multicolumn{2}{|>{\raggedright\arraybackslash}X|}{\hspace{0pt}\mytexttt{\color{param} (BOOLEAN in\_local\_time = FALSE)}} \\
\hline
\end{tabularx}
}

\par





Returns the current date.






\par
\begin{description}
\item [\colorbox{tagtype}{\color{white} \textbf{\textsf{PARAMETER}}}] \textbf{\underline{in\_local\_time}} ||| BOOLEAN --- TRUE if the returned value should be local to the cluster computing the date, FALSE for UTC. Optional, defaults to FALSE.
\end{description}







\par
\begin{description}
\item [\colorbox{tagtype}{\color{white} \textbf{\textsf{RETURN}}}] \textbf{UNSIGNED4} --- A Date\_t representing the current date.
\end{description}




\rule{\linewidth}{0.5pt}
\subsection*{\textsf{\colorbox{headtoc}{\color{white} FUNCTION}
Today}}

\hypertarget{ecldoc:date.today}{}
\hspace{0pt} \hyperlink{ecldoc:Date}{Date} \textbackslash 

{\renewcommand{\arraystretch}{1.5}
\begin{tabularx}{\textwidth}{|>{\raggedright\arraybackslash}l|X|}
\hline
\hspace{0pt}\mytexttt{\color{red} Date\_t} & \textbf{Today} \\
\hline
\multicolumn{2}{|>{\raggedright\arraybackslash}X|}{\hspace{0pt}\mytexttt{\color{param} ()}} \\
\hline
\end{tabularx}
}

\par





Returns the current date in the local time zone.








\par
\begin{description}
\item [\colorbox{tagtype}{\color{white} \textbf{\textsf{RETURN}}}] \textbf{UNSIGNED4} --- A Date\_t representing the current date.
\end{description}




\rule{\linewidth}{0.5pt}
\subsection*{\textsf{\colorbox{headtoc}{\color{white} FUNCTION}
CurrentTime}}

\hypertarget{ecldoc:date.currenttime}{}
\hspace{0pt} \hyperlink{ecldoc:Date}{Date} \textbackslash 

{\renewcommand{\arraystretch}{1.5}
\begin{tabularx}{\textwidth}{|>{\raggedright\arraybackslash}l|X|}
\hline
\hspace{0pt}\mytexttt{\color{red} Time\_t} & \textbf{CurrentTime} \\
\hline
\multicolumn{2}{|>{\raggedright\arraybackslash}X|}{\hspace{0pt}\mytexttt{\color{param} (BOOLEAN in\_local\_time = FALSE)}} \\
\hline
\end{tabularx}
}

\par





Returns the current time of day






\par
\begin{description}
\item [\colorbox{tagtype}{\color{white} \textbf{\textsf{PARAMETER}}}] \textbf{\underline{in\_local\_time}} ||| BOOLEAN --- TRUE if the returned value should be local to the cluster computing the time, FALSE for UTC. Optional, defaults to FALSE.
\end{description}







\par
\begin{description}
\item [\colorbox{tagtype}{\color{white} \textbf{\textsf{RETURN}}}] \textbf{UNSIGNED3} --- A Time\_t representing the current time of day.
\end{description}




\rule{\linewidth}{0.5pt}
\subsection*{\textsf{\colorbox{headtoc}{\color{white} FUNCTION}
CurrentSeconds}}

\hypertarget{ecldoc:date.currentseconds}{}
\hspace{0pt} \hyperlink{ecldoc:Date}{Date} \textbackslash 

{\renewcommand{\arraystretch}{1.5}
\begin{tabularx}{\textwidth}{|>{\raggedright\arraybackslash}l|X|}
\hline
\hspace{0pt}\mytexttt{\color{red} Seconds\_t} & \textbf{CurrentSeconds} \\
\hline
\multicolumn{2}{|>{\raggedright\arraybackslash}X|}{\hspace{0pt}\mytexttt{\color{param} (BOOLEAN in\_local\_time = FALSE)}} \\
\hline
\end{tabularx}
}

\par





Returns the current date and time as the number of seconds since epoch.






\par
\begin{description}
\item [\colorbox{tagtype}{\color{white} \textbf{\textsf{PARAMETER}}}] \textbf{\underline{in\_local\_time}} ||| BOOLEAN --- TRUE if the returned value should be local to the cluster computing the time, FALSE for UTC. Optional, defaults to FALSE.
\end{description}







\par
\begin{description}
\item [\colorbox{tagtype}{\color{white} \textbf{\textsf{RETURN}}}] \textbf{INTEGER8} --- A Seconds\_t representing the current time in UTC or local time, depending on the argument.
\end{description}




\rule{\linewidth}{0.5pt}
\subsection*{\textsf{\colorbox{headtoc}{\color{white} FUNCTION}
CurrentTimestamp}}

\hypertarget{ecldoc:date.currenttimestamp}{}
\hspace{0pt} \hyperlink{ecldoc:Date}{Date} \textbackslash 

{\renewcommand{\arraystretch}{1.5}
\begin{tabularx}{\textwidth}{|>{\raggedright\arraybackslash}l|X|}
\hline
\hspace{0pt}\mytexttt{\color{red} Timestamp\_t} & \textbf{CurrentTimestamp} \\
\hline
\multicolumn{2}{|>{\raggedright\arraybackslash}X|}{\hspace{0pt}\mytexttt{\color{param} (BOOLEAN in\_local\_time = FALSE)}} \\
\hline
\end{tabularx}
}

\par





Returns the current date and time as the number of microseconds since epoch.






\par
\begin{description}
\item [\colorbox{tagtype}{\color{white} \textbf{\textsf{PARAMETER}}}] \textbf{\underline{in\_local\_time}} ||| BOOLEAN --- TRUE if the returned value should be local to the cluster computing the time, FALSE for UTC. Optional, defaults to FALSE.
\end{description}







\par
\begin{description}
\item [\colorbox{tagtype}{\color{white} \textbf{\textsf{RETURN}}}] \textbf{INTEGER8} --- A Timestamp\_t representing the current time in microseconds in UTC or local time, depending on the argument.
\end{description}




\rule{\linewidth}{0.5pt}
\subsection*{\textsf{\colorbox{headtoc}{\color{white} MODULE}
DatesForMonth}}

\hypertarget{ecldoc:date.datesformonth}{}
\hspace{0pt} \hyperlink{ecldoc:Date}{Date} \textbackslash 

{\renewcommand{\arraystretch}{1.5}
\begin{tabularx}{\textwidth}{|>{\raggedright\arraybackslash}l|X|}
\hline
\hspace{0pt}\mytexttt{\color{red} } & \textbf{DatesForMonth} \\
\hline
\multicolumn{2}{|>{\raggedright\arraybackslash}X|}{\hspace{0pt}\mytexttt{\color{param} (Date\_t as\_of\_date = CurrentDate(FALSE))}} \\
\hline
\end{tabularx}
}

\par





Returns the beginning and ending dates for the month surrounding the given date.






\par
\begin{description}
\item [\colorbox{tagtype}{\color{white} \textbf{\textsf{PARAMETER}}}] \textbf{\underline{as\_of\_date}} ||| UNSIGNED4 --- The reference date from which the month will be calculated. This date must be a date within the Gregorian calendar. Optional, defaults to the current date in UTC.
\end{description}







\par
\begin{description}
\item [\colorbox{tagtype}{\color{white} \textbf{\textsf{RETURN}}}] \textbf{} --- Module with exported attributes for startDate and endDate.
\end{description}




\textbf{Children}
\begin{enumerate}
\item \hyperlink{ecldoc:date.datesformonth.result.startdate}{startDate}
: No Documentation Found
\item \hyperlink{ecldoc:date.datesformonth.result.enddate}{endDate}
: No Documentation Found
\end{enumerate}

\rule{\linewidth}{0.5pt}

\subsection*{\textsf{\colorbox{headtoc}{\color{white} ATTRIBUTE}
startDate}}

\hypertarget{ecldoc:date.datesformonth.result.startdate}{}
\hspace{0pt} \hyperlink{ecldoc:Date}{Date} \textbackslash 
\hspace{0pt} \hyperlink{ecldoc:date.datesformonth}{DatesForMonth} \textbackslash 

{\renewcommand{\arraystretch}{1.5}
\begin{tabularx}{\textwidth}{|>{\raggedright\arraybackslash}l|X|}
\hline
\hspace{0pt}\mytexttt{\color{red} Date\_t} & \textbf{startDate} \\
\hline
\end{tabularx}
}

\par





No Documentation Found








\par
\begin{description}
\item [\colorbox{tagtype}{\color{white} \textbf{\textsf{RETURN}}}] \textbf{UNSIGNED4} --- 
\end{description}




\rule{\linewidth}{0.5pt}
\subsection*{\textsf{\colorbox{headtoc}{\color{white} ATTRIBUTE}
endDate}}

\hypertarget{ecldoc:date.datesformonth.result.enddate}{}
\hspace{0pt} \hyperlink{ecldoc:Date}{Date} \textbackslash 
\hspace{0pt} \hyperlink{ecldoc:date.datesformonth}{DatesForMonth} \textbackslash 

{\renewcommand{\arraystretch}{1.5}
\begin{tabularx}{\textwidth}{|>{\raggedright\arraybackslash}l|X|}
\hline
\hspace{0pt}\mytexttt{\color{red} Date\_t} & \textbf{endDate} \\
\hline
\end{tabularx}
}

\par





No Documentation Found








\par
\begin{description}
\item [\colorbox{tagtype}{\color{white} \textbf{\textsf{RETURN}}}] \textbf{UNSIGNED4} --- 
\end{description}




\rule{\linewidth}{0.5pt}


\subsection*{\textsf{\colorbox{headtoc}{\color{white} MODULE}
DatesForWeek}}

\hypertarget{ecldoc:date.datesforweek}{}
\hspace{0pt} \hyperlink{ecldoc:Date}{Date} \textbackslash 

{\renewcommand{\arraystretch}{1.5}
\begin{tabularx}{\textwidth}{|>{\raggedright\arraybackslash}l|X|}
\hline
\hspace{0pt}\mytexttt{\color{red} } & \textbf{DatesForWeek} \\
\hline
\multicolumn{2}{|>{\raggedright\arraybackslash}X|}{\hspace{0pt}\mytexttt{\color{param} (Date\_t as\_of\_date = CurrentDate(FALSE))}} \\
\hline
\end{tabularx}
}

\par





Returns the beginning and ending dates for the week surrounding the given date (Sunday marks the beginning of a week).






\par
\begin{description}
\item [\colorbox{tagtype}{\color{white} \textbf{\textsf{PARAMETER}}}] \textbf{\underline{as\_of\_date}} ||| UNSIGNED4 --- The reference date from which the week will be calculated. This date must be a date within the Gregorian calendar. Optional, defaults to the current date in UTC.
\end{description}







\par
\begin{description}
\item [\colorbox{tagtype}{\color{white} \textbf{\textsf{RETURN}}}] \textbf{} --- Module with exported attributes for startDate and endDate.
\end{description}




\textbf{Children}
\begin{enumerate}
\item \hyperlink{ecldoc:date.datesforweek.result.startdate}{startDate}
: No Documentation Found
\item \hyperlink{ecldoc:date.datesforweek.result.enddate}{endDate}
: No Documentation Found
\end{enumerate}

\rule{\linewidth}{0.5pt}

\subsection*{\textsf{\colorbox{headtoc}{\color{white} ATTRIBUTE}
startDate}}

\hypertarget{ecldoc:date.datesforweek.result.startdate}{}
\hspace{0pt} \hyperlink{ecldoc:Date}{Date} \textbackslash 
\hspace{0pt} \hyperlink{ecldoc:date.datesforweek}{DatesForWeek} \textbackslash 

{\renewcommand{\arraystretch}{1.5}
\begin{tabularx}{\textwidth}{|>{\raggedright\arraybackslash}l|X|}
\hline
\hspace{0pt}\mytexttt{\color{red} Date\_t} & \textbf{startDate} \\
\hline
\end{tabularx}
}

\par





No Documentation Found








\par
\begin{description}
\item [\colorbox{tagtype}{\color{white} \textbf{\textsf{RETURN}}}] \textbf{UNSIGNED4} --- 
\end{description}




\rule{\linewidth}{0.5pt}
\subsection*{\textsf{\colorbox{headtoc}{\color{white} ATTRIBUTE}
endDate}}

\hypertarget{ecldoc:date.datesforweek.result.enddate}{}
\hspace{0pt} \hyperlink{ecldoc:Date}{Date} \textbackslash 
\hspace{0pt} \hyperlink{ecldoc:date.datesforweek}{DatesForWeek} \textbackslash 

{\renewcommand{\arraystretch}{1.5}
\begin{tabularx}{\textwidth}{|>{\raggedright\arraybackslash}l|X|}
\hline
\hspace{0pt}\mytexttt{\color{red} Date\_t} & \textbf{endDate} \\
\hline
\end{tabularx}
}

\par





No Documentation Found








\par
\begin{description}
\item [\colorbox{tagtype}{\color{white} \textbf{\textsf{RETURN}}}] \textbf{UNSIGNED4} --- 
\end{description}




\rule{\linewidth}{0.5pt}


\subsection*{\textsf{\colorbox{headtoc}{\color{white} FUNCTION}
IsValidDate}}

\hypertarget{ecldoc:date.isvaliddate}{}
\hspace{0pt} \hyperlink{ecldoc:Date}{Date} \textbackslash 

{\renewcommand{\arraystretch}{1.5}
\begin{tabularx}{\textwidth}{|>{\raggedright\arraybackslash}l|X|}
\hline
\hspace{0pt}\mytexttt{\color{red} BOOLEAN} & \textbf{IsValidDate} \\
\hline
\multicolumn{2}{|>{\raggedright\arraybackslash}X|}{\hspace{0pt}\mytexttt{\color{param} (Date\_t date, INTEGER2 yearLowerBound = 1800, INTEGER2 yearUpperBound = 2100)}} \\
\hline
\end{tabularx}
}

\par





Tests whether a date is valid, both by range-checking the year and by validating each of the other individual components.






\par
\begin{description}
\item [\colorbox{tagtype}{\color{white} \textbf{\textsf{PARAMETER}}}] \textbf{\underline{date}} ||| UNSIGNED4 --- The date to validate.
\item [\colorbox{tagtype}{\color{white} \textbf{\textsf{PARAMETER}}}] \textbf{\underline{yearUpperBound}} ||| INTEGER2 --- The maximum acceptable year. Optional; defaults to 2100.
\item [\colorbox{tagtype}{\color{white} \textbf{\textsf{PARAMETER}}}] \textbf{\underline{yearLowerBound}} ||| INTEGER2 --- The minimum acceptable year. Optional; defaults to 1800.
\end{description}







\par
\begin{description}
\item [\colorbox{tagtype}{\color{white} \textbf{\textsf{RETURN}}}] \textbf{BOOLEAN} --- TRUE if the date is valid, FALSE otherwise.
\end{description}




\rule{\linewidth}{0.5pt}
\subsection*{\textsf{\colorbox{headtoc}{\color{white} FUNCTION}
IsValidGregorianDate}}

\hypertarget{ecldoc:date.isvalidgregoriandate}{}
\hspace{0pt} \hyperlink{ecldoc:Date}{Date} \textbackslash 

{\renewcommand{\arraystretch}{1.5}
\begin{tabularx}{\textwidth}{|>{\raggedright\arraybackslash}l|X|}
\hline
\hspace{0pt}\mytexttt{\color{red} BOOLEAN} & \textbf{IsValidGregorianDate} \\
\hline
\multicolumn{2}{|>{\raggedright\arraybackslash}X|}{\hspace{0pt}\mytexttt{\color{param} (Date\_t date)}} \\
\hline
\end{tabularx}
}

\par





Tests whether a date is valid in the Gregorian calendar. The year must be between 1601 and 30827.






\par
\begin{description}
\item [\colorbox{tagtype}{\color{white} \textbf{\textsf{PARAMETER}}}] \textbf{\underline{date}} ||| UNSIGNED4 --- The Date\_t to validate.
\end{description}







\par
\begin{description}
\item [\colorbox{tagtype}{\color{white} \textbf{\textsf{RETURN}}}] \textbf{BOOLEAN} --- TRUE if the date is valid, FALSE otherwise.
\end{description}




\rule{\linewidth}{0.5pt}
\subsection*{\textsf{\colorbox{headtoc}{\color{white} FUNCTION}
IsValidTime}}

\hypertarget{ecldoc:date.isvalidtime}{}
\hspace{0pt} \hyperlink{ecldoc:Date}{Date} \textbackslash 

{\renewcommand{\arraystretch}{1.5}
\begin{tabularx}{\textwidth}{|>{\raggedright\arraybackslash}l|X|}
\hline
\hspace{0pt}\mytexttt{\color{red} BOOLEAN} & \textbf{IsValidTime} \\
\hline
\multicolumn{2}{|>{\raggedright\arraybackslash}X|}{\hspace{0pt}\mytexttt{\color{param} (Time\_t time)}} \\
\hline
\end{tabularx}
}

\par





Tests whether a time is valid.






\par
\begin{description}
\item [\colorbox{tagtype}{\color{white} \textbf{\textsf{PARAMETER}}}] \textbf{\underline{time}} ||| UNSIGNED3 --- The time to validate.
\end{description}







\par
\begin{description}
\item [\colorbox{tagtype}{\color{white} \textbf{\textsf{RETURN}}}] \textbf{BOOLEAN} --- TRUE if the time is valid, FALSE otherwise.
\end{description}




\rule{\linewidth}{0.5pt}
\subsection*{\textsf{\colorbox{headtoc}{\color{white} TRANSFORM}
CreateDate}}

\hypertarget{ecldoc:date.createdate}{}
\hspace{0pt} \hyperlink{ecldoc:Date}{Date} \textbackslash 

{\renewcommand{\arraystretch}{1.5}
\begin{tabularx}{\textwidth}{|>{\raggedright\arraybackslash}l|X|}
\hline
\hspace{0pt}\mytexttt{\color{red} Date\_rec} & \textbf{CreateDate} \\
\hline
\multicolumn{2}{|>{\raggedright\arraybackslash}X|}{\hspace{0pt}\mytexttt{\color{param} (INTEGER2 year, UNSIGNED1 month, UNSIGNED1 day)}} \\
\hline
\end{tabularx}
}

\par





A transform to create a Date\_rec from the individual elements






\par
\begin{description}
\item [\colorbox{tagtype}{\color{white} \textbf{\textsf{PARAMETER}}}] \textbf{\underline{year}} ||| INTEGER2 --- The year
\item [\colorbox{tagtype}{\color{white} \textbf{\textsf{PARAMETER}}}] \textbf{\underline{month}} ||| UNSIGNED1 --- The month (1-12).
\item [\colorbox{tagtype}{\color{white} \textbf{\textsf{PARAMETER}}}] \textbf{\underline{day}} ||| UNSIGNED1 --- The day (1..daysInMonth).
\end{description}







\par
\begin{description}
\item [\colorbox{tagtype}{\color{white} \textbf{\textsf{RETURN}}}] \textbf{Date\_rec} --- A transform that creates a Date\_rec containing the date.
\end{description}




\rule{\linewidth}{0.5pt}
\subsection*{\textsf{\colorbox{headtoc}{\color{white} TRANSFORM}
CreateDateFromSeconds}}

\hypertarget{ecldoc:date.createdatefromseconds}{}
\hspace{0pt} \hyperlink{ecldoc:Date}{Date} \textbackslash 

{\renewcommand{\arraystretch}{1.5}
\begin{tabularx}{\textwidth}{|>{\raggedright\arraybackslash}l|X|}
\hline
\hspace{0pt}\mytexttt{\color{red} Date\_rec} & \textbf{CreateDateFromSeconds} \\
\hline
\multicolumn{2}{|>{\raggedright\arraybackslash}X|}{\hspace{0pt}\mytexttt{\color{param} (Seconds\_t seconds)}} \\
\hline
\end{tabularx}
}

\par





A transform to create a Date\_rec from a Seconds\_t value.






\par
\begin{description}
\item [\colorbox{tagtype}{\color{white} \textbf{\textsf{PARAMETER}}}] \textbf{\underline{seconds}} ||| INTEGER8 --- The number seconds since epoch.
\end{description}







\par
\begin{description}
\item [\colorbox{tagtype}{\color{white} \textbf{\textsf{RETURN}}}] \textbf{Date\_rec} --- A transform that creates a Date\_rec containing the date.
\end{description}




\rule{\linewidth}{0.5pt}
\subsection*{\textsf{\colorbox{headtoc}{\color{white} TRANSFORM}
CreateTime}}

\hypertarget{ecldoc:date.createtime}{}
\hspace{0pt} \hyperlink{ecldoc:Date}{Date} \textbackslash 

{\renewcommand{\arraystretch}{1.5}
\begin{tabularx}{\textwidth}{|>{\raggedright\arraybackslash}l|X|}
\hline
\hspace{0pt}\mytexttt{\color{red} Time\_rec} & \textbf{CreateTime} \\
\hline
\multicolumn{2}{|>{\raggedright\arraybackslash}X|}{\hspace{0pt}\mytexttt{\color{param} (UNSIGNED1 hour, UNSIGNED1 minute, UNSIGNED1 second)}} \\
\hline
\end{tabularx}
}

\par





A transform to create a Time\_rec from the individual elements






\par
\begin{description}
\item [\colorbox{tagtype}{\color{white} \textbf{\textsf{PARAMETER}}}] \textbf{\underline{minute}} ||| UNSIGNED1 --- The minute (0-59).
\item [\colorbox{tagtype}{\color{white} \textbf{\textsf{PARAMETER}}}] \textbf{\underline{second}} ||| UNSIGNED1 --- The second (0-59).
\item [\colorbox{tagtype}{\color{white} \textbf{\textsf{PARAMETER}}}] \textbf{\underline{hour}} ||| UNSIGNED1 --- The hour (0-23).
\end{description}







\par
\begin{description}
\item [\colorbox{tagtype}{\color{white} \textbf{\textsf{RETURN}}}] \textbf{Time\_rec} --- A transform that creates a Time\_rec containing the time of day.
\end{description}




\rule{\linewidth}{0.5pt}
\subsection*{\textsf{\colorbox{headtoc}{\color{white} TRANSFORM}
CreateTimeFromSeconds}}

\hypertarget{ecldoc:date.createtimefromseconds}{}
\hspace{0pt} \hyperlink{ecldoc:Date}{Date} \textbackslash 

{\renewcommand{\arraystretch}{1.5}
\begin{tabularx}{\textwidth}{|>{\raggedright\arraybackslash}l|X|}
\hline
\hspace{0pt}\mytexttt{\color{red} Time\_rec} & \textbf{CreateTimeFromSeconds} \\
\hline
\multicolumn{2}{|>{\raggedright\arraybackslash}X|}{\hspace{0pt}\mytexttt{\color{param} (Seconds\_t seconds)}} \\
\hline
\end{tabularx}
}

\par





A transform to create a Time\_rec from a Seconds\_t value.






\par
\begin{description}
\item [\colorbox{tagtype}{\color{white} \textbf{\textsf{PARAMETER}}}] \textbf{\underline{seconds}} ||| INTEGER8 --- The number seconds since epoch.
\end{description}







\par
\begin{description}
\item [\colorbox{tagtype}{\color{white} \textbf{\textsf{RETURN}}}] \textbf{Time\_rec} --- A transform that creates a Time\_rec containing the time of day.
\end{description}




\rule{\linewidth}{0.5pt}
\subsection*{\textsf{\colorbox{headtoc}{\color{white} TRANSFORM}
CreateDateTime}}

\hypertarget{ecldoc:date.createdatetime}{}
\hspace{0pt} \hyperlink{ecldoc:Date}{Date} \textbackslash 

{\renewcommand{\arraystretch}{1.5}
\begin{tabularx}{\textwidth}{|>{\raggedright\arraybackslash}l|X|}
\hline
\hspace{0pt}\mytexttt{\color{red} DateTime\_rec} & \textbf{CreateDateTime} \\
\hline
\multicolumn{2}{|>{\raggedright\arraybackslash}X|}{\hspace{0pt}\mytexttt{\color{param} (INTEGER2 year, UNSIGNED1 month, UNSIGNED1 day, UNSIGNED1 hour, UNSIGNED1 minute, UNSIGNED1 second)}} \\
\hline
\end{tabularx}
}

\par





A transform to create a DateTime\_rec from the individual elements






\par
\begin{description}
\item [\colorbox{tagtype}{\color{white} \textbf{\textsf{PARAMETER}}}] \textbf{\underline{year}} ||| INTEGER2 --- The year
\item [\colorbox{tagtype}{\color{white} \textbf{\textsf{PARAMETER}}}] \textbf{\underline{second}} ||| UNSIGNED1 --- The second (0-59).
\item [\colorbox{tagtype}{\color{white} \textbf{\textsf{PARAMETER}}}] \textbf{\underline{hour}} ||| UNSIGNED1 --- The hour (0-23).
\item [\colorbox{tagtype}{\color{white} \textbf{\textsf{PARAMETER}}}] \textbf{\underline{minute}} ||| UNSIGNED1 --- The minute (0-59).
\item [\colorbox{tagtype}{\color{white} \textbf{\textsf{PARAMETER}}}] \textbf{\underline{month}} ||| UNSIGNED1 --- The month (1-12).
\item [\colorbox{tagtype}{\color{white} \textbf{\textsf{PARAMETER}}}] \textbf{\underline{day}} ||| UNSIGNED1 --- The day (1..daysInMonth).
\end{description}







\par
\begin{description}
\item [\colorbox{tagtype}{\color{white} \textbf{\textsf{RETURN}}}] \textbf{DateTime\_rec} --- A transform that creates a DateTime\_rec containing date and time components.
\end{description}




\rule{\linewidth}{0.5pt}
\subsection*{\textsf{\colorbox{headtoc}{\color{white} TRANSFORM}
CreateDateTimeFromSeconds}}

\hypertarget{ecldoc:date.createdatetimefromseconds}{}
\hspace{0pt} \hyperlink{ecldoc:Date}{Date} \textbackslash 

{\renewcommand{\arraystretch}{1.5}
\begin{tabularx}{\textwidth}{|>{\raggedright\arraybackslash}l|X|}
\hline
\hspace{0pt}\mytexttt{\color{red} DateTime\_rec} & \textbf{CreateDateTimeFromSeconds} \\
\hline
\multicolumn{2}{|>{\raggedright\arraybackslash}X|}{\hspace{0pt}\mytexttt{\color{param} (Seconds\_t seconds)}} \\
\hline
\end{tabularx}
}

\par





A transform to create a DateTime\_rec from a Seconds\_t value.






\par
\begin{description}
\item [\colorbox{tagtype}{\color{white} \textbf{\textsf{PARAMETER}}}] \textbf{\underline{seconds}} ||| INTEGER8 --- The number seconds since epoch.
\end{description}







\par
\begin{description}
\item [\colorbox{tagtype}{\color{white} \textbf{\textsf{RETURN}}}] \textbf{DateTime\_rec} --- A transform that creates a DateTime\_rec containing date and time components.
\end{description}




\rule{\linewidth}{0.5pt}



\chapter*{\color{headfile}
File
}
\hypertarget{ecldoc:toc:File}{}
\hyperlink{ecldoc:toc:root}{Go Up}

\section*{\underline{\textsf{IMPORTS}}}
\begin{doublespace}
{\large
lib\_fileservices |
}
\end{doublespace}

\section*{\underline{\textsf{DESCRIPTIONS}}}
\subsection*{\textsf{\colorbox{headtoc}{\color{white} MODULE}
File}}

\hypertarget{ecldoc:File}{}

{\renewcommand{\arraystretch}{1.5}
\begin{tabularx}{\textwidth}{|>{\raggedright\arraybackslash}l|X|}
\hline
\hspace{0pt}\mytexttt{\color{red} } & \textbf{File} \\
\hline
\end{tabularx}
}

\par


\textbf{Children}
\begin{enumerate}
\item \hyperlink{ecldoc:file.fsfilenamerecord}{FsFilenameRecord}
: A record containing information about filename
\item \hyperlink{ecldoc:file.fslogicalfilename}{FsLogicalFileName}
: An alias for a logical filename that is stored in a row
\item \hyperlink{ecldoc:file.fslogicalfilenamerecord}{FsLogicalFileNameRecord}
: A record containing a logical filename
\item \hyperlink{ecldoc:file.fslogicalfileinforecord}{FsLogicalFileInfoRecord}
: A record containing information about a logical file
\item \hyperlink{ecldoc:file.fslogicalsupersubrecord}{FsLogicalSuperSubRecord}
: A record containing information about a superfile and its contents
\item \hyperlink{ecldoc:file.fsfilerelationshiprecord}{FsFileRelationshipRecord}
: A record containing information about the relationship between two files
\item \hyperlink{ecldoc:file.recfmv_recsize}{RECFMV\_RECSIZE}
: Constant that indicates IBM RECFM V format file
\item \hyperlink{ecldoc:file.recfmvb_recsize}{RECFMVB\_RECSIZE}
: Constant that indicates IBM RECFM VB format file
\item \hyperlink{ecldoc:file.prefix_variable_recsize}{PREFIX\_VARIABLE\_RECSIZE}
: Constant that indicates a variable little endian 4 byte length prefixed file
\item \hyperlink{ecldoc:file.prefix_variable_bigendian_recsize}{PREFIX\_VARIABLE\_BIGENDIAN\_RECSIZE}
: Constant that indicates a variable big endian 4 byte length prefixed file
\item \hyperlink{ecldoc:file.fileexists}{FileExists}
: Returns whether the file exists
\item \hyperlink{ecldoc:file.deletelogicalfile}{DeleteLogicalFile}
: Removes the logical file from the system, and deletes from the disk
\item \hyperlink{ecldoc:file.setreadonly}{SetReadOnly}
: Changes whether access to a file is read only or not
\item \hyperlink{ecldoc:file.renamelogicalfile}{RenameLogicalFile}
: Changes the name of a logical file
\item \hyperlink{ecldoc:file.foreignlogicalfilename}{ForeignLogicalFileName}
: Returns a logical filename that can be used to refer to a logical file in a local or remote dali
\item \hyperlink{ecldoc:file.externallogicalfilename}{ExternalLogicalFileName}
: Returns an encoded logical filename that can be used to refer to a external file
\item \hyperlink{ecldoc:file.getfiledescription}{GetFileDescription}
: Returns a string containing the description information associated with the specified filename
\item \hyperlink{ecldoc:file.setfiledescription}{SetFileDescription}
: Sets the description associated with the specified filename
\item \hyperlink{ecldoc:file.remotedirectory}{RemoteDirectory}
: Returns a dataset containing a list of files from the specified machineIP and directory
\item \hyperlink{ecldoc:file.logicalfilelist}{LogicalFileList}
: Returns a dataset of information about the logical files known to the system
\item \hyperlink{ecldoc:file.comparefiles}{CompareFiles}
: Compares two files, and returns a result indicating how well they match
\item \hyperlink{ecldoc:file.verifyfile}{VerifyFile}
: Checks the system datastore (Dali) information for the file against the physical parts on disk
\item \hyperlink{ecldoc:file.addfilerelationship}{AddFileRelationship}
: Defines the relationship between two files
\item \hyperlink{ecldoc:file.filerelationshiplist}{FileRelationshipList}
: Returns a dataset of relationships
\item \hyperlink{ecldoc:file.removefilerelationship}{RemoveFileRelationship}
: Removes a relationship between two files
\item \hyperlink{ecldoc:file.getcolumnmapping}{GetColumnMapping}
: Returns the field mappings for the file, in the same format specified for the SetColumnMapping function
\item \hyperlink{ecldoc:file.setcolumnmapping}{SetColumnMapping}
: Defines how the data in the fields of the file mist be transformed between the actual data storage format and the input format used to query that data
\item \hyperlink{ecldoc:file.encoderfsquery}{EncodeRfsQuery}
: Returns a string that can be used in a DATASET declaration to read data from an RFS (Remote File Server) instance (e.g
\item \hyperlink{ecldoc:file.rfsaction}{RfsAction}
: Sends the query to the rfs server
\item \hyperlink{ecldoc:file.moveexternalfile}{MoveExternalFile}
: Moves the single physical file between two locations on the same remote machine
\item \hyperlink{ecldoc:file.deleteexternalfile}{DeleteExternalFile}
: Removes a single physical file from a remote machine
\item \hyperlink{ecldoc:file.createexternaldirectory}{CreateExternalDirectory}
: Creates the path on the location (if it does not already exist)
\item \hyperlink{ecldoc:file.getlogicalfileattribute}{GetLogicalFileAttribute}
: Returns the value of the given attribute for the specified logicalfilename
\item \hyperlink{ecldoc:file.protectlogicalfile}{ProtectLogicalFile}
: Toggles protection on and off for the specified logicalfilename
\item \hyperlink{ecldoc:file.dfuplusexec}{DfuPlusExec}
: The DfuPlusExec action executes the specified command line just as the DfuPLus.exe program would do
\item \hyperlink{ecldoc:file.fsprayfixed}{fSprayFixed}
: Sprays a file of fixed length records from a single machine and distributes it across the nodes of the destination group
\item \hyperlink{ecldoc:file.sprayfixed}{SprayFixed}
: Same as fSprayFixed, but does not return the DFU Workunit ID
\item \hyperlink{ecldoc:file.fsprayvariable}{fSprayVariable}
\item \hyperlink{ecldoc:file.sprayvariable}{SprayVariable}
\item \hyperlink{ecldoc:file.fspraydelimited}{fSprayDelimited}
: Sprays a file of fixed delimited records from a single machine and distributes it across the nodes of the destination group
\item \hyperlink{ecldoc:file.spraydelimited}{SprayDelimited}
: Same as fSprayDelimited, but does not return the DFU Workunit ID
\item \hyperlink{ecldoc:file.fsprayxml}{fSprayXml}
: Sprays an xml file from a single machine and distributes it across the nodes of the destination group
\item \hyperlink{ecldoc:file.sprayxml}{SprayXml}
: Same as fSprayXml, but does not return the DFU Workunit ID
\item \hyperlink{ecldoc:file.fdespray}{fDespray}
: Copies a distributed file from multiple machines, and desprays it to a single file on a single machine
\item \hyperlink{ecldoc:file.despray}{Despray}
: Same as fDespray, but does not return the DFU Workunit ID
\item \hyperlink{ecldoc:file.fcopy}{fCopy}
: Copies a distributed file to another distributed file
\item \hyperlink{ecldoc:file.copy}{Copy}
: Same as fCopy, but does not return the DFU Workunit ID
\item \hyperlink{ecldoc:file.freplicate}{fReplicate}
: Ensures the specified file is replicated to its mirror copies
\item \hyperlink{ecldoc:file.replicate}{Replicate}
: Same as fReplicated, but does not return the DFU Workunit ID
\item \hyperlink{ecldoc:file.fremotepull}{fRemotePull}
: Copies a distributed file to a distributed file on remote system
\item \hyperlink{ecldoc:file.remotepull}{RemotePull}
: Same as fRemotePull, but does not return the DFU Workunit ID
\item \hyperlink{ecldoc:file.fmonitorlogicalfilename}{fMonitorLogicalFileName}
: Creates a file monitor job in the DFU Server
\item \hyperlink{ecldoc:file.monitorlogicalfilename}{MonitorLogicalFileName}
: Same as fMonitorLogicalFileName, but does not return the DFU Workunit ID
\item \hyperlink{ecldoc:file.fmonitorfile}{fMonitorFile}
: Creates a file monitor job in the DFU Server
\item \hyperlink{ecldoc:file.monitorfile}{MonitorFile}
: Same as fMonitorFile, but does not return the DFU Workunit ID
\item \hyperlink{ecldoc:file.waitdfuworkunit}{WaitDfuWorkunit}
: Waits for the specified DFU workunit to finish
\item \hyperlink{ecldoc:file.abortdfuworkunit}{AbortDfuWorkunit}
: Aborts the specified DFU workunit
\item \hyperlink{ecldoc:file.createsuperfile}{CreateSuperFile}
: Creates an empty superfile
\item \hyperlink{ecldoc:file.superfileexists}{SuperFileExists}
: Checks if the specified filename is present in the Distributed File Utility (DFU) and is a SuperFile
\item \hyperlink{ecldoc:file.deletesuperfile}{DeleteSuperFile}
: Deletes the superfile
\item \hyperlink{ecldoc:file.getsuperfilesubcount}{GetSuperFileSubCount}
: Returns the number of sub-files contained within a superfile
\item \hyperlink{ecldoc:file.getsuperfilesubname}{GetSuperFileSubName}
: Returns the name of the Nth sub-file within a superfile
\item \hyperlink{ecldoc:file.findsuperfilesubname}{FindSuperFileSubName}
: Returns the position of a file within a superfile
\item \hyperlink{ecldoc:file.startsuperfiletransaction}{StartSuperFileTransaction}
: Starts a superfile transaction
\item \hyperlink{ecldoc:file.addsuperfile}{AddSuperFile}
: Adds a file to a superfile
\item \hyperlink{ecldoc:file.removesuperfile}{RemoveSuperFile}
: Removes a sub-file from a superfile
\item \hyperlink{ecldoc:file.clearsuperfile}{ClearSuperFile}
: Removes all sub-files from a superfile
\item \hyperlink{ecldoc:file.removeownedsubfiles}{RemoveOwnedSubFiles}
: Removes all soley-owned sub-files from a superfile
\item \hyperlink{ecldoc:file.deleteownedsubfiles}{DeleteOwnedSubFiles}
: Legacy version of RemoveOwnedSubFiles which was incorrectly named in a previous version
\item \hyperlink{ecldoc:file.swapsuperfile}{SwapSuperFile}
: Swap the contents of two superfiles
\item \hyperlink{ecldoc:file.replacesuperfile}{ReplaceSuperFile}
: Removes a sub-file from a superfile and replaces it with another
\item \hyperlink{ecldoc:file.finishsuperfiletransaction}{FinishSuperFileTransaction}
: Finishes a superfile transaction
\item \hyperlink{ecldoc:file.superfilecontents}{SuperFileContents}
: Returns the list of sub-files contained within a superfile
\item \hyperlink{ecldoc:file.logicalfilesuperowners}{LogicalFileSuperOwners}
: Returns the list of superfiles that a logical file is contained within
\item \hyperlink{ecldoc:file.logicalfilesupersublist}{LogicalFileSuperSubList}
: Returns the list of all the superfiles in the system and their component sub-files
\item \hyperlink{ecldoc:file.fpromotesuperfilelist}{fPromoteSuperFileList}
: Moves the sub-files from the first entry in the list of superfiles to the next in the list, repeating the process through the list of superfiles
\item \hyperlink{ecldoc:file.promotesuperfilelist}{PromoteSuperFileList}
: Same as fPromoteSuperFileList, but does not return the DFU Workunit ID
\end{enumerate}

\rule{\linewidth}{0.5pt}

\subsection*{\textsf{\colorbox{headtoc}{\color{white} RECORD}
FsFilenameRecord}}

\hypertarget{ecldoc:file.fsfilenamerecord}{}
\hspace{0pt} \hyperlink{ecldoc:File}{File} \textbackslash 

{\renewcommand{\arraystretch}{1.5}
\begin{tabularx}{\textwidth}{|>{\raggedright\arraybackslash}l|X|}
\hline
\hspace{0pt}\mytexttt{\color{red} } & \textbf{FsFilenameRecord} \\
\hline
\end{tabularx}
}

\par
A record containing information about filename. Includes name, size and when last modified. export FsFilenameRecord := RECORD string name; integer8 size; string19 modified; END;


\rule{\linewidth}{0.5pt}
\subsection*{\textsf{\colorbox{headtoc}{\color{white} ATTRIBUTE}
FsLogicalFileName}}

\hypertarget{ecldoc:file.fslogicalfilename}{}
\hspace{0pt} \hyperlink{ecldoc:File}{File} \textbackslash 

{\renewcommand{\arraystretch}{1.5}
\begin{tabularx}{\textwidth}{|>{\raggedright\arraybackslash}l|X|}
\hline
\hspace{0pt}\mytexttt{\color{red} } & \textbf{FsLogicalFileName} \\
\hline
\end{tabularx}
}

\par
An alias for a logical filename that is stored in a row.


\rule{\linewidth}{0.5pt}
\subsection*{\textsf{\colorbox{headtoc}{\color{white} RECORD}
FsLogicalFileNameRecord}}

\hypertarget{ecldoc:file.fslogicalfilenamerecord}{}
\hspace{0pt} \hyperlink{ecldoc:File}{File} \textbackslash 

{\renewcommand{\arraystretch}{1.5}
\begin{tabularx}{\textwidth}{|>{\raggedright\arraybackslash}l|X|}
\hline
\hspace{0pt}\mytexttt{\color{red} } & \textbf{FsLogicalFileNameRecord} \\
\hline
\end{tabularx}
}

\par
A record containing a logical filename. It contains the following fields:

\par
\begin{description}
\item [\colorbox{tagtype}{\color{white} \textbf{\textsf{FIELD}}}] \textbf{\underline{name}} The logical name of the file;
\end{description}

\rule{\linewidth}{0.5pt}
\subsection*{\textsf{\colorbox{headtoc}{\color{white} RECORD}
FsLogicalFileInfoRecord}}

\hypertarget{ecldoc:file.fslogicalfileinforecord}{}
\hspace{0pt} \hyperlink{ecldoc:File}{File} \textbackslash 

{\renewcommand{\arraystretch}{1.5}
\begin{tabularx}{\textwidth}{|>{\raggedright\arraybackslash}l|X|}
\hline
\hspace{0pt}\mytexttt{\color{red} } & \textbf{FsLogicalFileInfoRecord} \\
\hline
\end{tabularx}
}

\par
A record containing information about a logical file.

\par
\begin{description}
\item [\colorbox{tagtype}{\color{white} \textbf{\textsf{FIELD}}}] \textbf{\underline{superfile}} Is this a superfile?
\item [\colorbox{tagtype}{\color{white} \textbf{\textsf{FIELD}}}] \textbf{\underline{size}} Number of bytes in the file (before compression)
\item [\colorbox{tagtype}{\color{white} \textbf{\textsf{FIELD}}}] \textbf{\underline{rowcount}} Number of rows in the file.
\end{description}

\rule{\linewidth}{0.5pt}
\subsection*{\textsf{\colorbox{headtoc}{\color{white} RECORD}
FsLogicalSuperSubRecord}}

\hypertarget{ecldoc:file.fslogicalsupersubrecord}{}
\hspace{0pt} \hyperlink{ecldoc:File}{File} \textbackslash 

{\renewcommand{\arraystretch}{1.5}
\begin{tabularx}{\textwidth}{|>{\raggedright\arraybackslash}l|X|}
\hline
\hspace{0pt}\mytexttt{\color{red} } & \textbf{FsLogicalSuperSubRecord} \\
\hline
\end{tabularx}
}

\par
A record containing information about a superfile and its contents.

\par
\begin{description}
\item [\colorbox{tagtype}{\color{white} \textbf{\textsf{FIELD}}}] \textbf{\underline{supername}} The name of the superfile
\item [\colorbox{tagtype}{\color{white} \textbf{\textsf{FIELD}}}] \textbf{\underline{subname}} The name of the sub-file
\end{description}

\rule{\linewidth}{0.5pt}
\subsection*{\textsf{\colorbox{headtoc}{\color{white} RECORD}
FsFileRelationshipRecord}}

\hypertarget{ecldoc:file.fsfilerelationshiprecord}{}
\hspace{0pt} \hyperlink{ecldoc:File}{File} \textbackslash 

{\renewcommand{\arraystretch}{1.5}
\begin{tabularx}{\textwidth}{|>{\raggedright\arraybackslash}l|X|}
\hline
\hspace{0pt}\mytexttt{\color{red} } & \textbf{FsFileRelationshipRecord} \\
\hline
\end{tabularx}
}

\par
A record containing information about the relationship between two files.

\par
\begin{description}
\item [\colorbox{tagtype}{\color{white} \textbf{\textsf{FIELD}}}] \textbf{\underline{primaryfile}} The logical filename of the primary file
\item [\colorbox{tagtype}{\color{white} \textbf{\textsf{FIELD}}}] \textbf{\underline{secondaryfile}} The logical filename of the secondary file.
\item [\colorbox{tagtype}{\color{white} \textbf{\textsf{FIELD}}}] \textbf{\underline{primaryflds}} The name of the primary key field for the primary file. The value ''\_\_fileposition\_\_'' indicates the secondary is an INDEX that must use FETCH to access non-keyed fields.
\item [\colorbox{tagtype}{\color{white} \textbf{\textsf{FIELD}}}] \textbf{\underline{secondaryflds}} The name of the foreign key field relating to the primary file.
\item [\colorbox{tagtype}{\color{white} \textbf{\textsf{FIELD}}}] \textbf{\underline{kind}} The type of relationship between the primary and secondary files. Containing either 'link' or 'view'.
\item [\colorbox{tagtype}{\color{white} \textbf{\textsf{FIELD}}}] \textbf{\underline{cardinality}} The cardinality of the relationship. The format is <primary>:<secondary>. Valid values are ''1'' or ''M''.</secondary></primary>
\item [\colorbox{tagtype}{\color{white} \textbf{\textsf{FIELD}}}] \textbf{\underline{payload}} Indicates whether the primary or secondary are payload INDEXes.
\item [\colorbox{tagtype}{\color{white} \textbf{\textsf{FIELD}}}] \textbf{\underline{description}} The description of the relationship.
\end{description}

\rule{\linewidth}{0.5pt}
\subsection*{\textsf{\colorbox{headtoc}{\color{white} ATTRIBUTE}
RECFMV\_RECSIZE}}

\hypertarget{ecldoc:file.recfmv_recsize}{}
\hspace{0pt} \hyperlink{ecldoc:File}{File} \textbackslash 

{\renewcommand{\arraystretch}{1.5}
\begin{tabularx}{\textwidth}{|>{\raggedright\arraybackslash}l|X|}
\hline
\hspace{0pt}\mytexttt{\color{red} } & \textbf{RECFMV\_RECSIZE} \\
\hline
\end{tabularx}
}

\par
Constant that indicates IBM RECFM V format file. Can be passed to SprayFixed for the record size.


\rule{\linewidth}{0.5pt}
\subsection*{\textsf{\colorbox{headtoc}{\color{white} ATTRIBUTE}
RECFMVB\_RECSIZE}}

\hypertarget{ecldoc:file.recfmvb_recsize}{}
\hspace{0pt} \hyperlink{ecldoc:File}{File} \textbackslash 

{\renewcommand{\arraystretch}{1.5}
\begin{tabularx}{\textwidth}{|>{\raggedright\arraybackslash}l|X|}
\hline
\hspace{0pt}\mytexttt{\color{red} } & \textbf{RECFMVB\_RECSIZE} \\
\hline
\end{tabularx}
}

\par
Constant that indicates IBM RECFM VB format file. Can be passed to SprayFixed for the record size.


\rule{\linewidth}{0.5pt}
\subsection*{\textsf{\colorbox{headtoc}{\color{white} ATTRIBUTE}
PREFIX\_VARIABLE\_RECSIZE}}

\hypertarget{ecldoc:file.prefix_variable_recsize}{}
\hspace{0pt} \hyperlink{ecldoc:File}{File} \textbackslash 

{\renewcommand{\arraystretch}{1.5}
\begin{tabularx}{\textwidth}{|>{\raggedright\arraybackslash}l|X|}
\hline
\hspace{0pt}\mytexttt{\color{red} INTEGER4} & \textbf{PREFIX\_VARIABLE\_RECSIZE} \\
\hline
\end{tabularx}
}

\par
Constant that indicates a variable little endian 4 byte length prefixed file. Can be passed to SprayFixed for the record size.


\rule{\linewidth}{0.5pt}
\subsection*{\textsf{\colorbox{headtoc}{\color{white} ATTRIBUTE}
PREFIX\_VARIABLE\_BIGENDIAN\_RECSIZE}}

\hypertarget{ecldoc:file.prefix_variable_bigendian_recsize}{}
\hspace{0pt} \hyperlink{ecldoc:File}{File} \textbackslash 

{\renewcommand{\arraystretch}{1.5}
\begin{tabularx}{\textwidth}{|>{\raggedright\arraybackslash}l|X|}
\hline
\hspace{0pt}\mytexttt{\color{red} INTEGER4} & \textbf{PREFIX\_VARIABLE\_BIGENDIAN\_RECSIZE} \\
\hline
\end{tabularx}
}

\par
Constant that indicates a variable big endian 4 byte length prefixed file. Can be passed to SprayFixed for the record size.


\rule{\linewidth}{0.5pt}
\subsection*{\textsf{\colorbox{headtoc}{\color{white} FUNCTION}
FileExists}}

\hypertarget{ecldoc:file.fileexists}{}
\hspace{0pt} \hyperlink{ecldoc:File}{File} \textbackslash 

{\renewcommand{\arraystretch}{1.5}
\begin{tabularx}{\textwidth}{|>{\raggedright\arraybackslash}l|X|}
\hline
\hspace{0pt}\mytexttt{\color{red} boolean} & \textbf{FileExists} \\
\hline
\multicolumn{2}{|>{\raggedright\arraybackslash}X|}{\hspace{0pt}\mytexttt{\color{param} (varstring lfn, boolean physical=FALSE)}} \\
\hline
\end{tabularx}
}

\par
Returns whether the file exists.

\par
\begin{description}
\item [\colorbox{tagtype}{\color{white} \textbf{\textsf{PARAMETER}}}] \textbf{\underline{lfn}} The logical name of the file.
\item [\colorbox{tagtype}{\color{white} \textbf{\textsf{PARAMETER}}}] \textbf{\underline{physical}} Whether to also check for the physical existence on disk. Defaults to FALSE.
\item [\colorbox{tagtype}{\color{white} \textbf{\textsf{RETURN}}}] \textbf{\underline{}} Whether the file exists.
\end{description}

\rule{\linewidth}{0.5pt}
\subsection*{\textsf{\colorbox{headtoc}{\color{white} FUNCTION}
DeleteLogicalFile}}

\hypertarget{ecldoc:file.deletelogicalfile}{}
\hspace{0pt} \hyperlink{ecldoc:File}{File} \textbackslash 

{\renewcommand{\arraystretch}{1.5}
\begin{tabularx}{\textwidth}{|>{\raggedright\arraybackslash}l|X|}
\hline
\hspace{0pt}\mytexttt{\color{red} } & \textbf{DeleteLogicalFile} \\
\hline
\multicolumn{2}{|>{\raggedright\arraybackslash}X|}{\hspace{0pt}\mytexttt{\color{param} (varstring lfn, boolean allowMissing=FALSE)}} \\
\hline
\end{tabularx}
}

\par
Removes the logical file from the system, and deletes from the disk.

\par
\begin{description}
\item [\colorbox{tagtype}{\color{white} \textbf{\textsf{PARAMETER}}}] \textbf{\underline{lfn}} The logical name of the file.
\item [\colorbox{tagtype}{\color{white} \textbf{\textsf{PARAMETER}}}] \textbf{\underline{allowMissing}} Whether to suppress an error if the filename does not exist. Defaults to FALSE.
\end{description}

\rule{\linewidth}{0.5pt}
\subsection*{\textsf{\colorbox{headtoc}{\color{white} FUNCTION}
SetReadOnly}}

\hypertarget{ecldoc:file.setreadonly}{}
\hspace{0pt} \hyperlink{ecldoc:File}{File} \textbackslash 

{\renewcommand{\arraystretch}{1.5}
\begin{tabularx}{\textwidth}{|>{\raggedright\arraybackslash}l|X|}
\hline
\hspace{0pt}\mytexttt{\color{red} } & \textbf{SetReadOnly} \\
\hline
\multicolumn{2}{|>{\raggedright\arraybackslash}X|}{\hspace{0pt}\mytexttt{\color{param} (varstring lfn, boolean ro=TRUE)}} \\
\hline
\end{tabularx}
}

\par
Changes whether access to a file is read only or not.

\par
\begin{description}
\item [\colorbox{tagtype}{\color{white} \textbf{\textsf{PARAMETER}}}] \textbf{\underline{lfn}} The logical name of the file.
\item [\colorbox{tagtype}{\color{white} \textbf{\textsf{PARAMETER}}}] \textbf{\underline{ro}} Whether updates to the file are disallowed. Defaults to TRUE.
\end{description}

\rule{\linewidth}{0.5pt}
\subsection*{\textsf{\colorbox{headtoc}{\color{white} FUNCTION}
RenameLogicalFile}}

\hypertarget{ecldoc:file.renamelogicalfile}{}
\hspace{0pt} \hyperlink{ecldoc:File}{File} \textbackslash 

{\renewcommand{\arraystretch}{1.5}
\begin{tabularx}{\textwidth}{|>{\raggedright\arraybackslash}l|X|}
\hline
\hspace{0pt}\mytexttt{\color{red} } & \textbf{RenameLogicalFile} \\
\hline
\multicolumn{2}{|>{\raggedright\arraybackslash}X|}{\hspace{0pt}\mytexttt{\color{param} (varstring oldname, varstring newname)}} \\
\hline
\end{tabularx}
}

\par
Changes the name of a logical file.

\par
\begin{description}
\item [\colorbox{tagtype}{\color{white} \textbf{\textsf{PARAMETER}}}] \textbf{\underline{oldname}} The current name of the file to be renamed.
\item [\colorbox{tagtype}{\color{white} \textbf{\textsf{PARAMETER}}}] \textbf{\underline{newname}} The new logical name of the file.
\end{description}

\rule{\linewidth}{0.5pt}
\subsection*{\textsf{\colorbox{headtoc}{\color{white} FUNCTION}
ForeignLogicalFileName}}

\hypertarget{ecldoc:file.foreignlogicalfilename}{}
\hspace{0pt} \hyperlink{ecldoc:File}{File} \textbackslash 

{\renewcommand{\arraystretch}{1.5}
\begin{tabularx}{\textwidth}{|>{\raggedright\arraybackslash}l|X|}
\hline
\hspace{0pt}\mytexttt{\color{red} varstring} & \textbf{ForeignLogicalFileName} \\
\hline
\multicolumn{2}{|>{\raggedright\arraybackslash}X|}{\hspace{0pt}\mytexttt{\color{param} (varstring name, varstring foreigndali='', boolean abspath=FALSE)}} \\
\hline
\end{tabularx}
}

\par
Returns a logical filename that can be used to refer to a logical file in a local or remote dali.

\par
\begin{description}
\item [\colorbox{tagtype}{\color{white} \textbf{\textsf{PARAMETER}}}] \textbf{\underline{name}} The logical name of the file.
\item [\colorbox{tagtype}{\color{white} \textbf{\textsf{PARAMETER}}}] \textbf{\underline{foreigndali}} The IP address of the foreign dali used to resolve the file. If blank then the file is resolved locally. Defaults to blank.
\item [\colorbox{tagtype}{\color{white} \textbf{\textsf{PARAMETER}}}] \textbf{\underline{abspath}} Should a tilde (\~{}) be prepended to the resulting logical file name. Defaults to FALSE.
\end{description}

\rule{\linewidth}{0.5pt}
\subsection*{\textsf{\colorbox{headtoc}{\color{white} FUNCTION}
ExternalLogicalFileName}}

\hypertarget{ecldoc:file.externallogicalfilename}{}
\hspace{0pt} \hyperlink{ecldoc:File}{File} \textbackslash 

{\renewcommand{\arraystretch}{1.5}
\begin{tabularx}{\textwidth}{|>{\raggedright\arraybackslash}l|X|}
\hline
\hspace{0pt}\mytexttt{\color{red} varstring} & \textbf{ExternalLogicalFileName} \\
\hline
\multicolumn{2}{|>{\raggedright\arraybackslash}X|}{\hspace{0pt}\mytexttt{\color{param} (varstring location, varstring path, boolean abspath=TRUE)}} \\
\hline
\end{tabularx}
}

\par
Returns an encoded logical filename that can be used to refer to a external file. Examples include directly reading from a landing zone. Upper case characters and other details are escaped.

\par
\begin{description}
\item [\colorbox{tagtype}{\color{white} \textbf{\textsf{PARAMETER}}}] \textbf{\underline{location}} The IP address of the remote machine. '.' can be used for the local machine.
\item [\colorbox{tagtype}{\color{white} \textbf{\textsf{PARAMETER}}}] \textbf{\underline{path}} The path/name of the file on the remote machine.
\item [\colorbox{tagtype}{\color{white} \textbf{\textsf{PARAMETER}}}] \textbf{\underline{abspath}} Should a tilde (\~{}) be prepended to the resulting logical file name. Defaults to TRUE.
\item [\colorbox{tagtype}{\color{white} \textbf{\textsf{RETURN}}}] \textbf{\underline{}} The encoded logical filename.
\end{description}

\rule{\linewidth}{0.5pt}
\subsection*{\textsf{\colorbox{headtoc}{\color{white} FUNCTION}
GetFileDescription}}

\hypertarget{ecldoc:file.getfiledescription}{}
\hspace{0pt} \hyperlink{ecldoc:File}{File} \textbackslash 

{\renewcommand{\arraystretch}{1.5}
\begin{tabularx}{\textwidth}{|>{\raggedright\arraybackslash}l|X|}
\hline
\hspace{0pt}\mytexttt{\color{red} varstring} & \textbf{GetFileDescription} \\
\hline
\multicolumn{2}{|>{\raggedright\arraybackslash}X|}{\hspace{0pt}\mytexttt{\color{param} (varstring lfn)}} \\
\hline
\end{tabularx}
}

\par
Returns a string containing the description information associated with the specified filename. This description is set either through ECL watch or by using the FileServices.SetFileDescription function.

\par
\begin{description}
\item [\colorbox{tagtype}{\color{white} \textbf{\textsf{PARAMETER}}}] \textbf{\underline{lfn}} The logical name of the file.
\end{description}

\rule{\linewidth}{0.5pt}
\subsection*{\textsf{\colorbox{headtoc}{\color{white} FUNCTION}
SetFileDescription}}

\hypertarget{ecldoc:file.setfiledescription}{}
\hspace{0pt} \hyperlink{ecldoc:File}{File} \textbackslash 

{\renewcommand{\arraystretch}{1.5}
\begin{tabularx}{\textwidth}{|>{\raggedright\arraybackslash}l|X|}
\hline
\hspace{0pt}\mytexttt{\color{red} } & \textbf{SetFileDescription} \\
\hline
\multicolumn{2}{|>{\raggedright\arraybackslash}X|}{\hspace{0pt}\mytexttt{\color{param} (varstring lfn, varstring val)}} \\
\hline
\end{tabularx}
}

\par
Sets the description associated with the specified filename.

\par
\begin{description}
\item [\colorbox{tagtype}{\color{white} \textbf{\textsf{PARAMETER}}}] \textbf{\underline{lfn}} The logical name of the file.
\item [\colorbox{tagtype}{\color{white} \textbf{\textsf{PARAMETER}}}] \textbf{\underline{val}} The description to be associated with the file.
\end{description}

\rule{\linewidth}{0.5pt}
\subsection*{\textsf{\colorbox{headtoc}{\color{white} FUNCTION}
RemoteDirectory}}

\hypertarget{ecldoc:file.remotedirectory}{}
\hspace{0pt} \hyperlink{ecldoc:File}{File} \textbackslash 

{\renewcommand{\arraystretch}{1.5}
\begin{tabularx}{\textwidth}{|>{\raggedright\arraybackslash}l|X|}
\hline
\hspace{0pt}\mytexttt{\color{red} dataset(FsFilenameRecord)} & \textbf{RemoteDirectory} \\
\hline
\multicolumn{2}{|>{\raggedright\arraybackslash}X|}{\hspace{0pt}\mytexttt{\color{param} (varstring machineIP, varstring dir, varstring mask='*', boolean recurse=FALSE)}} \\
\hline
\end{tabularx}
}

\par
Returns a dataset containing a list of files from the specified machineIP and directory.

\par
\begin{description}
\item [\colorbox{tagtype}{\color{white} \textbf{\textsf{PARAMETER}}}] \textbf{\underline{machineIP}} The IP address of the remote machine.
\item [\colorbox{tagtype}{\color{white} \textbf{\textsf{PARAMETER}}}] \textbf{\underline{directory}} The path to the directory to read. This must be in the appropriate format for the operating system running on the remote machine.
\item [\colorbox{tagtype}{\color{white} \textbf{\textsf{PARAMETER}}}] \textbf{\underline{mask}} The filemask specifying which files to include in the result. Defaults to '*' (all files).
\item [\colorbox{tagtype}{\color{white} \textbf{\textsf{PARAMETER}}}] \textbf{\underline{recurse}} Whether to include files from subdirectories under the directory. Defaults to FALSE.
\end{description}

\rule{\linewidth}{0.5pt}
\subsection*{\textsf{\colorbox{headtoc}{\color{white} FUNCTION}
LogicalFileList}}

\hypertarget{ecldoc:file.logicalfilelist}{}
\hspace{0pt} \hyperlink{ecldoc:File}{File} \textbackslash 

{\renewcommand{\arraystretch}{1.5}
\begin{tabularx}{\textwidth}{|>{\raggedright\arraybackslash}l|X|}
\hline
\hspace{0pt}\mytexttt{\color{red} dataset(FsLogicalFileInfoRecord)} & \textbf{LogicalFileList} \\
\hline
\multicolumn{2}{|>{\raggedright\arraybackslash}X|}{\hspace{0pt}\mytexttt{\color{param} (varstring namepattern='*', boolean includenormal=TRUE, boolean includesuper=FALSE, boolean unknownszero=FALSE, varstring foreigndali='')}} \\
\hline
\end{tabularx}
}

\par
Returns a dataset of information about the logical files known to the system.

\par
\begin{description}
\item [\colorbox{tagtype}{\color{white} \textbf{\textsf{PARAMETER}}}] \textbf{\underline{namepattern}} The mask of the files to list. Defaults to '*' (all files).
\item [\colorbox{tagtype}{\color{white} \textbf{\textsf{PARAMETER}}}] \textbf{\underline{includenormal}} Whether to include 'normal' files. Defaults to TRUE.
\item [\colorbox{tagtype}{\color{white} \textbf{\textsf{PARAMETER}}}] \textbf{\underline{includesuper}} Whether to include SuperFiles. Defaults to FALSE.
\item [\colorbox{tagtype}{\color{white} \textbf{\textsf{PARAMETER}}}] \textbf{\underline{unknownszero}} Whether to set file sizes that are unknown to zero(0) instead of minus-one (-1). Defaults to FALSE.
\item [\colorbox{tagtype}{\color{white} \textbf{\textsf{PARAMETER}}}] \textbf{\underline{foreigndali}} The IP address of the foreign dali used to resolve the file. If blank then the file is resolved locally. Defaults to blank.
\end{description}

\rule{\linewidth}{0.5pt}
\subsection*{\textsf{\colorbox{headtoc}{\color{white} FUNCTION}
CompareFiles}}

\hypertarget{ecldoc:file.comparefiles}{}
\hspace{0pt} \hyperlink{ecldoc:File}{File} \textbackslash 

{\renewcommand{\arraystretch}{1.5}
\begin{tabularx}{\textwidth}{|>{\raggedright\arraybackslash}l|X|}
\hline
\hspace{0pt}\mytexttt{\color{red} INTEGER4} & \textbf{CompareFiles} \\
\hline
\multicolumn{2}{|>{\raggedright\arraybackslash}X|}{\hspace{0pt}\mytexttt{\color{param} (varstring lfn1, varstring lfn2, boolean logical\_only=TRUE, boolean use\_crcs=FALSE)}} \\
\hline
\end{tabularx}
}

\par
Compares two files, and returns a result indicating how well they match.

\par
\begin{description}
\item [\colorbox{tagtype}{\color{white} \textbf{\textsf{PARAMETER}}}] \textbf{\underline{file1}} The logical name of the first file.
\item [\colorbox{tagtype}{\color{white} \textbf{\textsf{PARAMETER}}}] \textbf{\underline{file2}} The logical name of the second file.
\item [\colorbox{tagtype}{\color{white} \textbf{\textsf{PARAMETER}}}] \textbf{\underline{logical\_only}} Whether to only compare logical information in the system datastore (Dali), and ignore physical information on disk. [Default TRUE]
\item [\colorbox{tagtype}{\color{white} \textbf{\textsf{PARAMETER}}}] \textbf{\underline{use\_crcs}} Whether to compare physical CRCs of all the parts on disk. This may be slow on large files. Defaults to FALSE.
\item [\colorbox{tagtype}{\color{white} \textbf{\textsf{RETURN}}}] \textbf{\underline{}} 0 if file1 and file2 match exactly 1 if file1 and file2 contents match, but file1 is newer than file2 -1 if file1 and file2 contents match, but file2 is newer than file1 2 if file1 and file2 contents do not match and file1 is newer than file2 -2 if file1 and file2 contents do not match and file2 is newer than file1
\end{description}

\rule{\linewidth}{0.5pt}
\subsection*{\textsf{\colorbox{headtoc}{\color{white} FUNCTION}
VerifyFile}}

\hypertarget{ecldoc:file.verifyfile}{}
\hspace{0pt} \hyperlink{ecldoc:File}{File} \textbackslash 

{\renewcommand{\arraystretch}{1.5}
\begin{tabularx}{\textwidth}{|>{\raggedright\arraybackslash}l|X|}
\hline
\hspace{0pt}\mytexttt{\color{red} varstring} & \textbf{VerifyFile} \\
\hline
\multicolumn{2}{|>{\raggedright\arraybackslash}X|}{\hspace{0pt}\mytexttt{\color{param} (varstring lfn, boolean usecrcs)}} \\
\hline
\end{tabularx}
}

\par
Checks the system datastore (Dali) information for the file against the physical parts on disk.

\par
\begin{description}
\item [\colorbox{tagtype}{\color{white} \textbf{\textsf{PARAMETER}}}] \textbf{\underline{lfn}} The name of the file to check.
\item [\colorbox{tagtype}{\color{white} \textbf{\textsf{PARAMETER}}}] \textbf{\underline{use\_crcs}} Whether to compare physical CRCs of all the parts on disk. This may be slow on large files.
\item [\colorbox{tagtype}{\color{white} \textbf{\textsf{RETURN}}}] \textbf{\underline{}} 'OK' - The file parts match the datastore information 'Could not find file: <filename>' - The logical filename was not found 'Could not find part file: <partname>' - The partname was not found 'Modified time differs for: <partname>' - The partname has a different timestamp 'File size differs for: <partname>' - The partname has a file size 'File CRC differs for: <partname>' - The partname has a different CRC</partname></partname></partname></partname></filename>
\end{description}

\rule{\linewidth}{0.5pt}
\subsection*{\textsf{\colorbox{headtoc}{\color{white} FUNCTION}
AddFileRelationship}}

\hypertarget{ecldoc:file.addfilerelationship}{}
\hspace{0pt} \hyperlink{ecldoc:File}{File} \textbackslash 

{\renewcommand{\arraystretch}{1.5}
\begin{tabularx}{\textwidth}{|>{\raggedright\arraybackslash}l|X|}
\hline
\hspace{0pt}\mytexttt{\color{red} } & \textbf{AddFileRelationship} \\
\hline
\multicolumn{2}{|>{\raggedright\arraybackslash}X|}{\hspace{0pt}\mytexttt{\color{param} (varstring primary, varstring secondary, varstring primaryflds, varstring secondaryflds, varstring kind='link', varstring cardinality, boolean payload, varstring description='')}} \\
\hline
\end{tabularx}
}

\par
Defines the relationship between two files. These may be DATASETs or INDEXes. Each record in the primary file should be uniquely defined by the primaryfields (ideally), preferably efficiently. This information is used by the roxie browser to link files together.

\par
\begin{description}
\item [\colorbox{tagtype}{\color{white} \textbf{\textsf{PARAMETER}}}] \textbf{\underline{primary}} The logical filename of the primary file.
\item [\colorbox{tagtype}{\color{white} \textbf{\textsf{PARAMETER}}}] \textbf{\underline{secondary}} The logical filename of the secondary file.
\item [\colorbox{tagtype}{\color{white} \textbf{\textsf{PARAMETER}}}] \textbf{\underline{primaryfields}} The name of the primary key field for the primary file. The value ''\_\_fileposition\_\_'' indicates the secondary is an INDEX that must use FETCH to access non-keyed fields.
\item [\colorbox{tagtype}{\color{white} \textbf{\textsf{PARAMETER}}}] \textbf{\underline{secondaryfields}} The name of the foreign key field relating to the primary file.
\item [\colorbox{tagtype}{\color{white} \textbf{\textsf{PARAMETER}}}] \textbf{\underline{relationship}} The type of relationship between the primary and secondary files. Containing either 'link' or 'view'. Default is ''link''.
\item [\colorbox{tagtype}{\color{white} \textbf{\textsf{PARAMETER}}}] \textbf{\underline{cardinality}} The cardinality of the relationship. The format is <primary>:<secondary>. Valid values are ''1'' or ''M''.</secondary></primary>
\item [\colorbox{tagtype}{\color{white} \textbf{\textsf{PARAMETER}}}] \textbf{\underline{payload}} Indicates whether the primary or secondary are payload INDEXes.
\item [\colorbox{tagtype}{\color{white} \textbf{\textsf{PARAMETER}}}] \textbf{\underline{description}} The description of the relationship.
\end{description}

\rule{\linewidth}{0.5pt}
\subsection*{\textsf{\colorbox{headtoc}{\color{white} FUNCTION}
FileRelationshipList}}

\hypertarget{ecldoc:file.filerelationshiplist}{}
\hspace{0pt} \hyperlink{ecldoc:File}{File} \textbackslash 

{\renewcommand{\arraystretch}{1.5}
\begin{tabularx}{\textwidth}{|>{\raggedright\arraybackslash}l|X|}
\hline
\hspace{0pt}\mytexttt{\color{red} dataset(FsFileRelationshipRecord)} & \textbf{FileRelationshipList} \\
\hline
\multicolumn{2}{|>{\raggedright\arraybackslash}X|}{\hspace{0pt}\mytexttt{\color{param} (varstring primary, varstring secondary, varstring primflds='', varstring secondaryflds='', varstring kind='link')}} \\
\hline
\end{tabularx}
}

\par
Returns a dataset of relationships. The return records are structured in the FsFileRelationshipRecord format.

\par
\begin{description}
\item [\colorbox{tagtype}{\color{white} \textbf{\textsf{PARAMETER}}}] \textbf{\underline{primary}} The logical filename of the primary file.
\item [\colorbox{tagtype}{\color{white} \textbf{\textsf{PARAMETER}}}] \textbf{\underline{secondary}} The logical filename of the secondary file.
\item [\colorbox{tagtype}{\color{white} \textbf{\textsf{PARAMETER}}}] \textbf{\underline{primaryfields}} The name of the primary key field for the primary file.
\item [\colorbox{tagtype}{\color{white} \textbf{\textsf{PARAMETER}}}] \textbf{\underline{secondaryfields}} The name of the foreign key field relating to the primary file.
\item [\colorbox{tagtype}{\color{white} \textbf{\textsf{PARAMETER}}}] \textbf{\underline{relationship}} The type of relationship between the primary and secondary files. Containing either 'link' or 'view'. Default is ''link''.
\end{description}

\rule{\linewidth}{0.5pt}
\subsection*{\textsf{\colorbox{headtoc}{\color{white} FUNCTION}
RemoveFileRelationship}}

\hypertarget{ecldoc:file.removefilerelationship}{}
\hspace{0pt} \hyperlink{ecldoc:File}{File} \textbackslash 

{\renewcommand{\arraystretch}{1.5}
\begin{tabularx}{\textwidth}{|>{\raggedright\arraybackslash}l|X|}
\hline
\hspace{0pt}\mytexttt{\color{red} } & \textbf{RemoveFileRelationship} \\
\hline
\multicolumn{2}{|>{\raggedright\arraybackslash}X|}{\hspace{0pt}\mytexttt{\color{param} (varstring primary, varstring secondary, varstring primaryflds='', varstring secondaryflds='', varstring kind='link')}} \\
\hline
\end{tabularx}
}

\par
Removes a relationship between two files.

\par
\begin{description}
\item [\colorbox{tagtype}{\color{white} \textbf{\textsf{PARAMETER}}}] \textbf{\underline{primary}} The logical filename of the primary file.
\item [\colorbox{tagtype}{\color{white} \textbf{\textsf{PARAMETER}}}] \textbf{\underline{secondary}} The logical filename of the secondary file.
\item [\colorbox{tagtype}{\color{white} \textbf{\textsf{PARAMETER}}}] \textbf{\underline{primaryfields}} The name of the primary key field for the primary file.
\item [\colorbox{tagtype}{\color{white} \textbf{\textsf{PARAMETER}}}] \textbf{\underline{secondaryfields}} The name of the foreign key field relating to the primary file.
\item [\colorbox{tagtype}{\color{white} \textbf{\textsf{PARAMETER}}}] \textbf{\underline{relationship}} The type of relationship between the primary and secondary files. Containing either 'link' or 'view'. Default is ''link''.
\end{description}

\rule{\linewidth}{0.5pt}
\subsection*{\textsf{\colorbox{headtoc}{\color{white} FUNCTION}
GetColumnMapping}}

\hypertarget{ecldoc:file.getcolumnmapping}{}
\hspace{0pt} \hyperlink{ecldoc:File}{File} \textbackslash 

{\renewcommand{\arraystretch}{1.5}
\begin{tabularx}{\textwidth}{|>{\raggedright\arraybackslash}l|X|}
\hline
\hspace{0pt}\mytexttt{\color{red} varstring} & \textbf{GetColumnMapping} \\
\hline
\multicolumn{2}{|>{\raggedright\arraybackslash}X|}{\hspace{0pt}\mytexttt{\color{param} (varstring lfn)}} \\
\hline
\end{tabularx}
}

\par
Returns the field mappings for the file, in the same format specified for the SetColumnMapping function.

\par
\begin{description}
\item [\colorbox{tagtype}{\color{white} \textbf{\textsf{PARAMETER}}}] \textbf{\underline{lfn}} The logical filename of the primary file.
\end{description}

\rule{\linewidth}{0.5pt}
\subsection*{\textsf{\colorbox{headtoc}{\color{white} FUNCTION}
SetColumnMapping}}

\hypertarget{ecldoc:file.setcolumnmapping}{}
\hspace{0pt} \hyperlink{ecldoc:File}{File} \textbackslash 

{\renewcommand{\arraystretch}{1.5}
\begin{tabularx}{\textwidth}{|>{\raggedright\arraybackslash}l|X|}
\hline
\hspace{0pt}\mytexttt{\color{red} } & \textbf{SetColumnMapping} \\
\hline
\multicolumn{2}{|>{\raggedright\arraybackslash}X|}{\hspace{0pt}\mytexttt{\color{param} (varstring lfn, varstring mapping)}} \\
\hline
\end{tabularx}
}

\par
Defines how the data in the fields of the file mist be transformed between the actual data storage format and the input format used to query that data. This is used by the user interface of the roxie browser.

\par
\begin{description}
\item [\colorbox{tagtype}{\color{white} \textbf{\textsf{PARAMETER}}}] \textbf{\underline{lfn}} The logical filename of the primary file.
\item [\colorbox{tagtype}{\color{white} \textbf{\textsf{PARAMETER}}}] \textbf{\underline{mapping}} A string containing a comma separated list of field mappings.
\end{description}

\rule{\linewidth}{0.5pt}
\subsection*{\textsf{\colorbox{headtoc}{\color{white} FUNCTION}
EncodeRfsQuery}}

\hypertarget{ecldoc:file.encoderfsquery}{}
\hspace{0pt} \hyperlink{ecldoc:File}{File} \textbackslash 

{\renewcommand{\arraystretch}{1.5}
\begin{tabularx}{\textwidth}{|>{\raggedright\arraybackslash}l|X|}
\hline
\hspace{0pt}\mytexttt{\color{red} varstring} & \textbf{EncodeRfsQuery} \\
\hline
\multicolumn{2}{|>{\raggedright\arraybackslash}X|}{\hspace{0pt}\mytexttt{\color{param} (varstring server, varstring query)}} \\
\hline
\end{tabularx}
}

\par
Returns a string that can be used in a DATASET declaration to read data from an RFS (Remote File Server) instance (e.g. rfsmysql) on another node.

\par
\begin{description}
\item [\colorbox{tagtype}{\color{white} \textbf{\textsf{PARAMETER}}}] \textbf{\underline{server}} A string containing the ip:port address for the remote file server.
\item [\colorbox{tagtype}{\color{white} \textbf{\textsf{PARAMETER}}}] \textbf{\underline{query}} The text of the query to send to the server
\end{description}

\rule{\linewidth}{0.5pt}
\subsection*{\textsf{\colorbox{headtoc}{\color{white} FUNCTION}
RfsAction}}

\hypertarget{ecldoc:file.rfsaction}{}
\hspace{0pt} \hyperlink{ecldoc:File}{File} \textbackslash 

{\renewcommand{\arraystretch}{1.5}
\begin{tabularx}{\textwidth}{|>{\raggedright\arraybackslash}l|X|}
\hline
\hspace{0pt}\mytexttt{\color{red} } & \textbf{RfsAction} \\
\hline
\multicolumn{2}{|>{\raggedright\arraybackslash}X|}{\hspace{0pt}\mytexttt{\color{param} (varstring server, varstring query)}} \\
\hline
\end{tabularx}
}

\par
Sends the query to the rfs server.

\par
\begin{description}
\item [\colorbox{tagtype}{\color{white} \textbf{\textsf{PARAMETER}}}] \textbf{\underline{server}} A string containing the ip:port address for the remote file server.
\item [\colorbox{tagtype}{\color{white} \textbf{\textsf{PARAMETER}}}] \textbf{\underline{query}} The text of the query to send to the server
\end{description}

\rule{\linewidth}{0.5pt}
\subsection*{\textsf{\colorbox{headtoc}{\color{white} FUNCTION}
MoveExternalFile}}

\hypertarget{ecldoc:file.moveexternalfile}{}
\hspace{0pt} \hyperlink{ecldoc:File}{File} \textbackslash 

{\renewcommand{\arraystretch}{1.5}
\begin{tabularx}{\textwidth}{|>{\raggedright\arraybackslash}l|X|}
\hline
\hspace{0pt}\mytexttt{\color{red} } & \textbf{MoveExternalFile} \\
\hline
\multicolumn{2}{|>{\raggedright\arraybackslash}X|}{\hspace{0pt}\mytexttt{\color{param} (varstring location, varstring frompath, varstring topath)}} \\
\hline
\end{tabularx}
}

\par
Moves the single physical file between two locations on the same remote machine. The dafileserv utility program must be running on the location machine.

\par
\begin{description}
\item [\colorbox{tagtype}{\color{white} \textbf{\textsf{PARAMETER}}}] \textbf{\underline{location}} The IP address of the remote machine.
\item [\colorbox{tagtype}{\color{white} \textbf{\textsf{PARAMETER}}}] \textbf{\underline{frompath}} The path/name of the file to move.
\item [\colorbox{tagtype}{\color{white} \textbf{\textsf{PARAMETER}}}] \textbf{\underline{topath}} The path/name of the target file.
\end{description}

\rule{\linewidth}{0.5pt}
\subsection*{\textsf{\colorbox{headtoc}{\color{white} FUNCTION}
DeleteExternalFile}}

\hypertarget{ecldoc:file.deleteexternalfile}{}
\hspace{0pt} \hyperlink{ecldoc:File}{File} \textbackslash 

{\renewcommand{\arraystretch}{1.5}
\begin{tabularx}{\textwidth}{|>{\raggedright\arraybackslash}l|X|}
\hline
\hspace{0pt}\mytexttt{\color{red} } & \textbf{DeleteExternalFile} \\
\hline
\multicolumn{2}{|>{\raggedright\arraybackslash}X|}{\hspace{0pt}\mytexttt{\color{param} (varstring location, varstring path)}} \\
\hline
\end{tabularx}
}

\par
Removes a single physical file from a remote machine. The dafileserv utility program must be running on the location machine.

\par
\begin{description}
\item [\colorbox{tagtype}{\color{white} \textbf{\textsf{PARAMETER}}}] \textbf{\underline{location}} The IP address of the remote machine.
\item [\colorbox{tagtype}{\color{white} \textbf{\textsf{PARAMETER}}}] \textbf{\underline{path}} The path/name of the file to remove.
\end{description}

\rule{\linewidth}{0.5pt}
\subsection*{\textsf{\colorbox{headtoc}{\color{white} FUNCTION}
CreateExternalDirectory}}

\hypertarget{ecldoc:file.createexternaldirectory}{}
\hspace{0pt} \hyperlink{ecldoc:File}{File} \textbackslash 

{\renewcommand{\arraystretch}{1.5}
\begin{tabularx}{\textwidth}{|>{\raggedright\arraybackslash}l|X|}
\hline
\hspace{0pt}\mytexttt{\color{red} } & \textbf{CreateExternalDirectory} \\
\hline
\multicolumn{2}{|>{\raggedright\arraybackslash}X|}{\hspace{0pt}\mytexttt{\color{param} (varstring location, varstring path)}} \\
\hline
\end{tabularx}
}

\par
Creates the path on the location (if it does not already exist). The dafileserv utility program must be running on the location machine.

\par
\begin{description}
\item [\colorbox{tagtype}{\color{white} \textbf{\textsf{PARAMETER}}}] \textbf{\underline{location}} The IP address of the remote machine.
\item [\colorbox{tagtype}{\color{white} \textbf{\textsf{PARAMETER}}}] \textbf{\underline{path}} The path/name of the file to remove.
\end{description}

\rule{\linewidth}{0.5pt}
\subsection*{\textsf{\colorbox{headtoc}{\color{white} FUNCTION}
GetLogicalFileAttribute}}

\hypertarget{ecldoc:file.getlogicalfileattribute}{}
\hspace{0pt} \hyperlink{ecldoc:File}{File} \textbackslash 

{\renewcommand{\arraystretch}{1.5}
\begin{tabularx}{\textwidth}{|>{\raggedright\arraybackslash}l|X|}
\hline
\hspace{0pt}\mytexttt{\color{red} varstring} & \textbf{GetLogicalFileAttribute} \\
\hline
\multicolumn{2}{|>{\raggedright\arraybackslash}X|}{\hspace{0pt}\mytexttt{\color{param} (varstring lfn, varstring attrname)}} \\
\hline
\end{tabularx}
}

\par
Returns the value of the given attribute for the specified logicalfilename.

\par
\begin{description}
\item [\colorbox{tagtype}{\color{white} \textbf{\textsf{PARAMETER}}}] \textbf{\underline{lfn}} The name of the logical file.
\item [\colorbox{tagtype}{\color{white} \textbf{\textsf{PARAMETER}}}] \textbf{\underline{attrname}} The name of the file attribute to return.
\end{description}

\rule{\linewidth}{0.5pt}
\subsection*{\textsf{\colorbox{headtoc}{\color{white} FUNCTION}
ProtectLogicalFile}}

\hypertarget{ecldoc:file.protectlogicalfile}{}
\hspace{0pt} \hyperlink{ecldoc:File}{File} \textbackslash 

{\renewcommand{\arraystretch}{1.5}
\begin{tabularx}{\textwidth}{|>{\raggedright\arraybackslash}l|X|}
\hline
\hspace{0pt}\mytexttt{\color{red} } & \textbf{ProtectLogicalFile} \\
\hline
\multicolumn{2}{|>{\raggedright\arraybackslash}X|}{\hspace{0pt}\mytexttt{\color{param} (varstring lfn, boolean value=TRUE)}} \\
\hline
\end{tabularx}
}

\par
Toggles protection on and off for the specified logicalfilename.

\par
\begin{description}
\item [\colorbox{tagtype}{\color{white} \textbf{\textsf{PARAMETER}}}] \textbf{\underline{lfn}} The name of the logical file.
\item [\colorbox{tagtype}{\color{white} \textbf{\textsf{PARAMETER}}}] \textbf{\underline{value}} TRUE to enable protection, FALSE to disable.
\end{description}

\rule{\linewidth}{0.5pt}
\subsection*{\textsf{\colorbox{headtoc}{\color{white} FUNCTION}
DfuPlusExec}}

\hypertarget{ecldoc:file.dfuplusexec}{}
\hspace{0pt} \hyperlink{ecldoc:File}{File} \textbackslash 

{\renewcommand{\arraystretch}{1.5}
\begin{tabularx}{\textwidth}{|>{\raggedright\arraybackslash}l|X|}
\hline
\hspace{0pt}\mytexttt{\color{red} } & \textbf{DfuPlusExec} \\
\hline
\multicolumn{2}{|>{\raggedright\arraybackslash}X|}{\hspace{0pt}\mytexttt{\color{param} (varstring cmdline)}} \\
\hline
\end{tabularx}
}

\par
The DfuPlusExec action executes the specified command line just as the DfuPLus.exe program would do. This allows you to have all the functionality of the DfuPLus.exe program available within your ECL code. param cmdline The DFUPlus.exe command line to execute. The valid arguments are documented in the Client Tools manual, in the section describing the DfuPlus.exe program.


\rule{\linewidth}{0.5pt}
\subsection*{\textsf{\colorbox{headtoc}{\color{white} FUNCTION}
fSprayFixed}}

\hypertarget{ecldoc:file.fsprayfixed}{}
\hspace{0pt} \hyperlink{ecldoc:File}{File} \textbackslash 

{\renewcommand{\arraystretch}{1.5}
\begin{tabularx}{\textwidth}{|>{\raggedright\arraybackslash}l|X|}
\hline
\hspace{0pt}\mytexttt{\color{red} varstring} & \textbf{fSprayFixed} \\
\hline
\multicolumn{2}{|>{\raggedright\arraybackslash}X|}{\hspace{0pt}\mytexttt{\color{param} (varstring sourceIP, varstring sourcePath, integer4 recordSize, varstring destinationGroup, varstring destinationLogicalName, integer4 timeOut=-1, varstring espServerIpPort=GETENV('ws\_fs\_server'), integer4 maxConnections=-1, boolean allowOverwrite=FALSE, boolean replicate=FALSE, boolean compress=FALSE, boolean failIfNoSourceFile=FALSE, integer4 expireDays=-1)}} \\
\hline
\end{tabularx}
}

\par
Sprays a file of fixed length records from a single machine and distributes it across the nodes of the destination group.

\par
\begin{description}
\item [\colorbox{tagtype}{\color{white} \textbf{\textsf{PARAMETER}}}] \textbf{\underline{sourceIP}} The IP address of the file.
\item [\colorbox{tagtype}{\color{white} \textbf{\textsf{PARAMETER}}}] \textbf{\underline{sourcePath}} The path and name of the file.
\item [\colorbox{tagtype}{\color{white} \textbf{\textsf{PARAMETER}}}] \textbf{\underline{recordsize}} The size (in bytes) of the records in the file.
\item [\colorbox{tagtype}{\color{white} \textbf{\textsf{PARAMETER}}}] \textbf{\underline{destinationGroup}} The name of the group to distribute the file across.
\item [\colorbox{tagtype}{\color{white} \textbf{\textsf{PARAMETER}}}] \textbf{\underline{destinationLogicalName}} The logical name of the file to create.
\item [\colorbox{tagtype}{\color{white} \textbf{\textsf{PARAMETER}}}] \textbf{\underline{timeOut}} The time in ms to wait for the operation to complete. A value of 0 causes the call to return immediately. Defaults to no timeout (-1).
\item [\colorbox{tagtype}{\color{white} \textbf{\textsf{PARAMETER}}}] \textbf{\underline{espServerIpPort}} The url of the ESP file copying service. Defaults to the value of ws\_fs\_server in the environment.
\item [\colorbox{tagtype}{\color{white} \textbf{\textsf{PARAMETER}}}] \textbf{\underline{maxConnections}} The maximum number of target nodes to write to concurrently. Defaults to 1.
\item [\colorbox{tagtype}{\color{white} \textbf{\textsf{PARAMETER}}}] \textbf{\underline{allowOverwrite}} Is it valid to overwrite an existing file of the same name? Defaults to FALSE
\item [\colorbox{tagtype}{\color{white} \textbf{\textsf{PARAMETER}}}] \textbf{\underline{replicate}} Whether to replicate the new file. Defaults to FALSE.
\item [\colorbox{tagtype}{\color{white} \textbf{\textsf{PARAMETER}}}] \textbf{\underline{compress}} Whether to compress the new file. Defaults to FALSE.
\item [\colorbox{tagtype}{\color{white} \textbf{\textsf{PARAMETER}}}] \textbf{\underline{failIfNoSourceFile}} If TRUE it causes a missing source file to trigger a failure. Defaults to FALSE.
\item [\colorbox{tagtype}{\color{white} \textbf{\textsf{PARAMETER}}}] \textbf{\underline{expireDays}} Number of days to auto-remove file. Default is -1, not expire.
\item [\colorbox{tagtype}{\color{white} \textbf{\textsf{RETURN}}}] \textbf{\underline{}} The DFU workunit id for the job.
\end{description}

\rule{\linewidth}{0.5pt}
\subsection*{\textsf{\colorbox{headtoc}{\color{white} FUNCTION}
SprayFixed}}

\hypertarget{ecldoc:file.sprayfixed}{}
\hspace{0pt} \hyperlink{ecldoc:File}{File} \textbackslash 

{\renewcommand{\arraystretch}{1.5}
\begin{tabularx}{\textwidth}{|>{\raggedright\arraybackslash}l|X|}
\hline
\hspace{0pt}\mytexttt{\color{red} } & \textbf{SprayFixed} \\
\hline
\multicolumn{2}{|>{\raggedright\arraybackslash}X|}{\hspace{0pt}\mytexttt{\color{param} (varstring sourceIP, varstring sourcePath, integer4 recordSize, varstring destinationGroup, varstring destinationLogicalName, integer4 timeOut=-1, varstring espServerIpPort=GETENV('ws\_fs\_server'), integer4 maxConnections=-1, boolean allowOverwrite=FALSE, boolean replicate=FALSE, boolean compress=FALSE, boolean failIfNoSourceFile=FALSE, integer4 expireDays=-1)}} \\
\hline
\end{tabularx}
}

\par
Same as fSprayFixed, but does not return the DFU Workunit ID.

\par
\begin{description}
\item [\colorbox{tagtype}{\color{white} \textbf{\textsf{SEE}}}] \textbf{\underline{}} fSprayFixed
\end{description}

\rule{\linewidth}{0.5pt}
\subsection*{\textsf{\colorbox{headtoc}{\color{white} FUNCTION}
fSprayVariable}}

\hypertarget{ecldoc:file.fsprayvariable}{}
\hspace{0pt} \hyperlink{ecldoc:File}{File} \textbackslash 

{\renewcommand{\arraystretch}{1.5}
\begin{tabularx}{\textwidth}{|>{\raggedright\arraybackslash}l|X|}
\hline
\hspace{0pt}\mytexttt{\color{red} varstring} & \textbf{fSprayVariable} \\
\hline
\multicolumn{2}{|>{\raggedright\arraybackslash}X|}{\hspace{0pt}\mytexttt{\color{param} (varstring sourceIP, varstring sourcePath, integer4 sourceMaxRecordSize=8192, varstring sourceCsvSeparate='\textbackslash \textbackslash ,', varstring sourceCsvTerminate='\textbackslash \textbackslash n,\textbackslash \textbackslash r\textbackslash \textbackslash n', varstring sourceCsvQuote='\textbackslash ''', varstring destinationGroup, varstring destinationLogicalName, integer4 timeOut=-1, varstring espServerIpPort=GETENV('ws\_fs\_server'), integer4 maxConnections=-1, boolean allowOverwrite=FALSE, boolean replicate=FALSE, boolean compress=FALSE, varstring sourceCsvEscape='', boolean failIfNoSourceFile=FALSE, boolean recordStructurePresent=FALSE, boolean quotedTerminator=TRUE, varstring encoding='ascii', integer4 expireDays=-1)}} \\
\hline
\end{tabularx}
}

\par


\rule{\linewidth}{0.5pt}
\subsection*{\textsf{\colorbox{headtoc}{\color{white} FUNCTION}
SprayVariable}}

\hypertarget{ecldoc:file.sprayvariable}{}
\hspace{0pt} \hyperlink{ecldoc:File}{File} \textbackslash 

{\renewcommand{\arraystretch}{1.5}
\begin{tabularx}{\textwidth}{|>{\raggedright\arraybackslash}l|X|}
\hline
\hspace{0pt}\mytexttt{\color{red} } & \textbf{SprayVariable} \\
\hline
\multicolumn{2}{|>{\raggedright\arraybackslash}X|}{\hspace{0pt}\mytexttt{\color{param} (varstring sourceIP, varstring sourcePath, integer4 sourceMaxRecordSize=8192, varstring sourceCsvSeparate='\textbackslash \textbackslash ,', varstring sourceCsvTerminate='\textbackslash \textbackslash n,\textbackslash \textbackslash r\textbackslash \textbackslash n', varstring sourceCsvQuote='\textbackslash ''', varstring destinationGroup, varstring destinationLogicalName, integer4 timeOut=-1, varstring espServerIpPort=GETENV('ws\_fs\_server'), integer4 maxConnections=-1, boolean allowOverwrite=FALSE, boolean replicate=FALSE, boolean compress=FALSE, varstring sourceCsvEscape='', boolean failIfNoSourceFile=FALSE, boolean recordStructurePresent=FALSE, boolean quotedTerminator=TRUE, varstring encoding='ascii', integer4 expireDays=-1)}} \\
\hline
\end{tabularx}
}

\par


\rule{\linewidth}{0.5pt}
\subsection*{\textsf{\colorbox{headtoc}{\color{white} FUNCTION}
fSprayDelimited}}

\hypertarget{ecldoc:file.fspraydelimited}{}
\hspace{0pt} \hyperlink{ecldoc:File}{File} \textbackslash 

{\renewcommand{\arraystretch}{1.5}
\begin{tabularx}{\textwidth}{|>{\raggedright\arraybackslash}l|X|}
\hline
\hspace{0pt}\mytexttt{\color{red} varstring} & \textbf{fSprayDelimited} \\
\hline
\multicolumn{2}{|>{\raggedright\arraybackslash}X|}{\hspace{0pt}\mytexttt{\color{param} (varstring sourceIP, varstring sourcePath, integer4 sourceMaxRecordSize=8192, varstring sourceCsvSeparate='\textbackslash \textbackslash ,', varstring sourceCsvTerminate='\textbackslash \textbackslash n,\textbackslash \textbackslash r\textbackslash \textbackslash n', varstring sourceCsvQuote='\textbackslash ''', varstring destinationGroup, varstring destinationLogicalName, integer4 timeOut=-1, varstring espServerIpPort=GETENV('ws\_fs\_server'), integer4 maxConnections=-1, boolean allowOverwrite=FALSE, boolean replicate=FALSE, boolean compress=FALSE, varstring sourceCsvEscape='', boolean failIfNoSourceFile=FALSE, boolean recordStructurePresent=FALSE, boolean quotedTerminator=TRUE, varstring encoding='ascii', integer4 expireDays=-1)}} \\
\hline
\end{tabularx}
}

\par
Sprays a file of fixed delimited records from a single machine and distributes it across the nodes of the destination group.

\par
\begin{description}
\item [\colorbox{tagtype}{\color{white} \textbf{\textsf{PARAMETER}}}] \textbf{\underline{sourceIP}} The IP address of the file.
\item [\colorbox{tagtype}{\color{white} \textbf{\textsf{PARAMETER}}}] \textbf{\underline{sourcePath}} The path and name of the file.
\item [\colorbox{tagtype}{\color{white} \textbf{\textsf{PARAMETER}}}] \textbf{\underline{sourceCsvSeparate}} The character sequence which separates fields in the file.
\item [\colorbox{tagtype}{\color{white} \textbf{\textsf{PARAMETER}}}] \textbf{\underline{sourceCsvTerminate}} The character sequence which separates records in the file.
\item [\colorbox{tagtype}{\color{white} \textbf{\textsf{PARAMETER}}}] \textbf{\underline{sourceCsvQuote}} A string which can be used to delimit fields in the file.
\item [\colorbox{tagtype}{\color{white} \textbf{\textsf{PARAMETER}}}] \textbf{\underline{sourceMaxRecordSize}} The maximum size (in bytes) of the records in the file.
\item [\colorbox{tagtype}{\color{white} \textbf{\textsf{PARAMETER}}}] \textbf{\underline{destinationGroup}} The name of the group to distribute the file across.
\item [\colorbox{tagtype}{\color{white} \textbf{\textsf{PARAMETER}}}] \textbf{\underline{destinationLogicalName}} The logical name of the file to create.
\item [\colorbox{tagtype}{\color{white} \textbf{\textsf{PARAMETER}}}] \textbf{\underline{timeOut}} The time in ms to wait for the operation to complete. A value of 0 causes the call to return immediately. Defaults to no timeout (-1).
\item [\colorbox{tagtype}{\color{white} \textbf{\textsf{PARAMETER}}}] \textbf{\underline{espServerIpPort}} The url of the ESP file copying service. Defaults to the value of ws\_fs\_server in the environment.
\item [\colorbox{tagtype}{\color{white} \textbf{\textsf{PARAMETER}}}] \textbf{\underline{maxConnections}} The maximum number of target nodes to write to concurrently. Defaults to 1.
\item [\colorbox{tagtype}{\color{white} \textbf{\textsf{PARAMETER}}}] \textbf{\underline{allowOverwrite}} Is it valid to overwrite an existing file of the same name? Defaults to FALSE
\item [\colorbox{tagtype}{\color{white} \textbf{\textsf{PARAMETER}}}] \textbf{\underline{replicate}} Whether to replicate the new file. Defaults to FALSE.
\item [\colorbox{tagtype}{\color{white} \textbf{\textsf{PARAMETER}}}] \textbf{\underline{compress}} Whether to compress the new file. Defaults to FALSE.
\item [\colorbox{tagtype}{\color{white} \textbf{\textsf{PARAMETER}}}] \textbf{\underline{sourceCsvEscape}} A character that is used to escape quote characters. Defaults to none.
\item [\colorbox{tagtype}{\color{white} \textbf{\textsf{PARAMETER}}}] \textbf{\underline{failIfNoSourceFile}} If TRUE it causes a missing source file to trigger a failure. Defaults to FALSE.
\item [\colorbox{tagtype}{\color{white} \textbf{\textsf{PARAMETER}}}] \textbf{\underline{recordStructurePresent}} If TRUE derives the record structure from the header of the file.
\item [\colorbox{tagtype}{\color{white} \textbf{\textsf{PARAMETER}}}] \textbf{\underline{quotedTerminator}} Can the terminator character be included in a quoted field. Defaults to TRUE. If FALSE it allows quicker partitioning of the file (avoiding a complete file scan).
\item [\colorbox{tagtype}{\color{white} \textbf{\textsf{PARAMETER}}}] \textbf{\underline{expireDays}} Number of days to auto-remove file. Default is -1, not expire.
\item [\colorbox{tagtype}{\color{white} \textbf{\textsf{RETURN}}}] \textbf{\underline{}} The DFU workunit id for the job.
\end{description}

\rule{\linewidth}{0.5pt}
\subsection*{\textsf{\colorbox{headtoc}{\color{white} FUNCTION}
SprayDelimited}}

\hypertarget{ecldoc:file.spraydelimited}{}
\hspace{0pt} \hyperlink{ecldoc:File}{File} \textbackslash 

{\renewcommand{\arraystretch}{1.5}
\begin{tabularx}{\textwidth}{|>{\raggedright\arraybackslash}l|X|}
\hline
\hspace{0pt}\mytexttt{\color{red} } & \textbf{SprayDelimited} \\
\hline
\multicolumn{2}{|>{\raggedright\arraybackslash}X|}{\hspace{0pt}\mytexttt{\color{param} (varstring sourceIP, varstring sourcePath, integer4 sourceMaxRecordSize=8192, varstring sourceCsvSeparate='\textbackslash \textbackslash ,', varstring sourceCsvTerminate='\textbackslash \textbackslash n,\textbackslash \textbackslash r\textbackslash \textbackslash n', varstring sourceCsvQuote='\textbackslash ''', varstring destinationGroup, varstring destinationLogicalName, integer4 timeOut=-1, varstring espServerIpPort=GETENV('ws\_fs\_server'), integer4 maxConnections=-1, boolean allowOverwrite=FALSE, boolean replicate=FALSE, boolean compress=FALSE, varstring sourceCsvEscape='', boolean failIfNoSourceFile=FALSE, boolean recordStructurePresent=FALSE, boolean quotedTerminator=TRUE, const varstring encoding='ascii', integer4 expireDays=-1)}} \\
\hline
\end{tabularx}
}

\par
Same as fSprayDelimited, but does not return the DFU Workunit ID.

\par
\begin{description}
\item [\colorbox{tagtype}{\color{white} \textbf{\textsf{SEE}}}] \textbf{\underline{}} fSprayDelimited
\end{description}

\rule{\linewidth}{0.5pt}
\subsection*{\textsf{\colorbox{headtoc}{\color{white} FUNCTION}
fSprayXml}}

\hypertarget{ecldoc:file.fsprayxml}{}
\hspace{0pt} \hyperlink{ecldoc:File}{File} \textbackslash 

{\renewcommand{\arraystretch}{1.5}
\begin{tabularx}{\textwidth}{|>{\raggedright\arraybackslash}l|X|}
\hline
\hspace{0pt}\mytexttt{\color{red} varstring} & \textbf{fSprayXml} \\
\hline
\multicolumn{2}{|>{\raggedright\arraybackslash}X|}{\hspace{0pt}\mytexttt{\color{param} (varstring sourceIP, varstring sourcePath, integer4 sourceMaxRecordSize=8192, varstring sourceRowTag, varstring sourceEncoding='utf8', varstring destinationGroup, varstring destinationLogicalName, integer4 timeOut=-1, varstring espServerIpPort=GETENV('ws\_fs\_server'), integer4 maxConnections=-1, boolean allowOverwrite=FALSE, boolean replicate=FALSE, boolean compress=FALSE, boolean failIfNoSourceFile=FALSE, integer4 expireDays=-1)}} \\
\hline
\end{tabularx}
}

\par
Sprays an xml file from a single machine and distributes it across the nodes of the destination group.

\par
\begin{description}
\item [\colorbox{tagtype}{\color{white} \textbf{\textsf{PARAMETER}}}] \textbf{\underline{sourceIP}} The IP address of the file.
\item [\colorbox{tagtype}{\color{white} \textbf{\textsf{PARAMETER}}}] \textbf{\underline{sourcePath}} The path and name of the file.
\item [\colorbox{tagtype}{\color{white} \textbf{\textsf{PARAMETER}}}] \textbf{\underline{sourceMaxRecordSize}} The maximum size (in bytes) of the records in the file.
\item [\colorbox{tagtype}{\color{white} \textbf{\textsf{PARAMETER}}}] \textbf{\underline{sourceRowTag}} The xml tag that is used to delimit records in the source file. (This tag cannot recursivly nest.)
\item [\colorbox{tagtype}{\color{white} \textbf{\textsf{PARAMETER}}}] \textbf{\underline{sourceEncoding}} The unicode encoding of the file. (utf8,utf8n,utf16be,utf16le,utf32be,utf32le)
\item [\colorbox{tagtype}{\color{white} \textbf{\textsf{PARAMETER}}}] \textbf{\underline{destinationGroup}} The name of the group to distribute the file across.
\item [\colorbox{tagtype}{\color{white} \textbf{\textsf{PARAMETER}}}] \textbf{\underline{destinationLogicalName}} The logical name of the file to create.
\item [\colorbox{tagtype}{\color{white} \textbf{\textsf{PARAMETER}}}] \textbf{\underline{timeOut}} The time in ms to wait for the operation to complete. A value of 0 causes the call to return immediately. Defaults to no timeout (-1).
\item [\colorbox{tagtype}{\color{white} \textbf{\textsf{PARAMETER}}}] \textbf{\underline{espServerIpPort}} The url of the ESP file copying service. Defaults to the value of ws\_fs\_server in the environment.
\item [\colorbox{tagtype}{\color{white} \textbf{\textsf{PARAMETER}}}] \textbf{\underline{maxConnections}} The maximum number of target nodes to write to concurrently. Defaults to 1.
\item [\colorbox{tagtype}{\color{white} \textbf{\textsf{PARAMETER}}}] \textbf{\underline{allowOverwrite}} Is it valid to overwrite an existing file of the same name? Defaults to FALSE
\item [\colorbox{tagtype}{\color{white} \textbf{\textsf{PARAMETER}}}] \textbf{\underline{replicate}} Whether to replicate the new file. Defaults to FALSE.
\item [\colorbox{tagtype}{\color{white} \textbf{\textsf{PARAMETER}}}] \textbf{\underline{compress}} Whether to compress the new file. Defaults to FALSE.
\item [\colorbox{tagtype}{\color{white} \textbf{\textsf{PARAMETER}}}] \textbf{\underline{failIfNoSourceFile}} If TRUE it causes a missing source file to trigger a failure. Defaults to FALSE.
\item [\colorbox{tagtype}{\color{white} \textbf{\textsf{PARAMETER}}}] \textbf{\underline{expireDays}} Number of days to auto-remove file. Default is -1, not expire.
\item [\colorbox{tagtype}{\color{white} \textbf{\textsf{RETURN}}}] \textbf{\underline{}} The DFU workunit id for the job.
\end{description}

\rule{\linewidth}{0.5pt}
\subsection*{\textsf{\colorbox{headtoc}{\color{white} FUNCTION}
SprayXml}}

\hypertarget{ecldoc:file.sprayxml}{}
\hspace{0pt} \hyperlink{ecldoc:File}{File} \textbackslash 

{\renewcommand{\arraystretch}{1.5}
\begin{tabularx}{\textwidth}{|>{\raggedright\arraybackslash}l|X|}
\hline
\hspace{0pt}\mytexttt{\color{red} } & \textbf{SprayXml} \\
\hline
\multicolumn{2}{|>{\raggedright\arraybackslash}X|}{\hspace{0pt}\mytexttt{\color{param} (varstring sourceIP, varstring sourcePath, integer4 sourceMaxRecordSize=8192, varstring sourceRowTag, varstring sourceEncoding='utf8', varstring destinationGroup, varstring destinationLogicalName, integer4 timeOut=-1, varstring espServerIpPort=GETENV('ws\_fs\_server'), integer4 maxConnections=-1, boolean allowOverwrite=FALSE, boolean replicate=FALSE, boolean compress=FALSE, boolean failIfNoSourceFile=FALSE, integer4 expireDays=-1)}} \\
\hline
\end{tabularx}
}

\par
Same as fSprayXml, but does not return the DFU Workunit ID.

\par
\begin{description}
\item [\colorbox{tagtype}{\color{white} \textbf{\textsf{SEE}}}] \textbf{\underline{}} fSprayXml
\end{description}

\rule{\linewidth}{0.5pt}
\subsection*{\textsf{\colorbox{headtoc}{\color{white} FUNCTION}
fDespray}}

\hypertarget{ecldoc:file.fdespray}{}
\hspace{0pt} \hyperlink{ecldoc:File}{File} \textbackslash 

{\renewcommand{\arraystretch}{1.5}
\begin{tabularx}{\textwidth}{|>{\raggedright\arraybackslash}l|X|}
\hline
\hspace{0pt}\mytexttt{\color{red} varstring} & \textbf{fDespray} \\
\hline
\multicolumn{2}{|>{\raggedright\arraybackslash}X|}{\hspace{0pt}\mytexttt{\color{param} (varstring logicalName, varstring destinationIP, varstring destinationPath, integer4 timeOut=-1, varstring espServerIpPort=GETENV('ws\_fs\_server'), integer4 maxConnections=-1, boolean allowOverwrite=FALSE)}} \\
\hline
\end{tabularx}
}

\par
Copies a distributed file from multiple machines, and desprays it to a single file on a single machine.

\par
\begin{description}
\item [\colorbox{tagtype}{\color{white} \textbf{\textsf{PARAMETER}}}] \textbf{\underline{logicalName}} The name of the file to despray.
\item [\colorbox{tagtype}{\color{white} \textbf{\textsf{PARAMETER}}}] \textbf{\underline{destinationIP}} The IP of the target machine.
\item [\colorbox{tagtype}{\color{white} \textbf{\textsf{PARAMETER}}}] \textbf{\underline{destinationPath}} The path of the file to create on the destination machine.
\item [\colorbox{tagtype}{\color{white} \textbf{\textsf{PARAMETER}}}] \textbf{\underline{timeOut}} The time in ms to wait for the operation to complete. A value of 0 causes the call to return immediately. Defaults to no timeout (-1).
\item [\colorbox{tagtype}{\color{white} \textbf{\textsf{PARAMETER}}}] \textbf{\underline{espServerIpPort}} The url of the ESP file copying service. Defaults to the value of ws\_fs\_server in the environment.
\item [\colorbox{tagtype}{\color{white} \textbf{\textsf{PARAMETER}}}] \textbf{\underline{maxConnections}} The maximum number of target nodes to write to concurrently. Defaults to 1.
\item [\colorbox{tagtype}{\color{white} \textbf{\textsf{PARAMETER}}}] \textbf{\underline{allowOverwrite}} Is it valid to overwrite an existing file of the same name? Defaults to FALSE
\item [\colorbox{tagtype}{\color{white} \textbf{\textsf{RETURN}}}] \textbf{\underline{}} The DFU workunit id for the job.
\end{description}

\rule{\linewidth}{0.5pt}
\subsection*{\textsf{\colorbox{headtoc}{\color{white} FUNCTION}
Despray}}

\hypertarget{ecldoc:file.despray}{}
\hspace{0pt} \hyperlink{ecldoc:File}{File} \textbackslash 

{\renewcommand{\arraystretch}{1.5}
\begin{tabularx}{\textwidth}{|>{\raggedright\arraybackslash}l|X|}
\hline
\hspace{0pt}\mytexttt{\color{red} } & \textbf{Despray} \\
\hline
\multicolumn{2}{|>{\raggedright\arraybackslash}X|}{\hspace{0pt}\mytexttt{\color{param} (varstring logicalName, varstring destinationIP, varstring destinationPath, integer4 timeOut=-1, varstring espServerIpPort=GETENV('ws\_fs\_server'), integer4 maxConnections=-1, boolean allowOverwrite=FALSE)}} \\
\hline
\end{tabularx}
}

\par
Same as fDespray, but does not return the DFU Workunit ID.

\par
\begin{description}
\item [\colorbox{tagtype}{\color{white} \textbf{\textsf{SEE}}}] \textbf{\underline{}} fDespray
\end{description}

\rule{\linewidth}{0.5pt}
\subsection*{\textsf{\colorbox{headtoc}{\color{white} FUNCTION}
fCopy}}

\hypertarget{ecldoc:file.fcopy}{}
\hspace{0pt} \hyperlink{ecldoc:File}{File} \textbackslash 

{\renewcommand{\arraystretch}{1.5}
\begin{tabularx}{\textwidth}{|>{\raggedright\arraybackslash}l|X|}
\hline
\hspace{0pt}\mytexttt{\color{red} varstring} & \textbf{fCopy} \\
\hline
\multicolumn{2}{|>{\raggedright\arraybackslash}X|}{\hspace{0pt}\mytexttt{\color{param} (varstring sourceLogicalName, varstring destinationGroup, varstring destinationLogicalName, varstring sourceDali='', integer4 timeOut=-1, varstring espServerIpPort=GETENV('ws\_fs\_server'), integer4 maxConnections=-1, boolean allowOverwrite=FALSE, boolean replicate=FALSE, boolean asSuperfile=FALSE, boolean compress=FALSE, boolean forcePush=FALSE, integer4 transferBufferSize=0, boolean preserveCompression=TRUE)}} \\
\hline
\end{tabularx}
}

\par
Copies a distributed file to another distributed file.

\par
\begin{description}
\item [\colorbox{tagtype}{\color{white} \textbf{\textsf{PARAMETER}}}] \textbf{\underline{sourceLogicalName}} The name of the file to despray.
\item [\colorbox{tagtype}{\color{white} \textbf{\textsf{PARAMETER}}}] \textbf{\underline{destinationGroup}} The name of the group to distribute the file across.
\item [\colorbox{tagtype}{\color{white} \textbf{\textsf{PARAMETER}}}] \textbf{\underline{destinationLogicalName}} The logical name of the file to create.
\item [\colorbox{tagtype}{\color{white} \textbf{\textsf{PARAMETER}}}] \textbf{\underline{sourceDali}} The dali that contains the source file (blank implies same dali). Defaults to same dali.
\item [\colorbox{tagtype}{\color{white} \textbf{\textsf{PARAMETER}}}] \textbf{\underline{timeOut}} The time in ms to wait for the operation to complete. A value of 0 causes the call to return immediately. Defaults to no timeout (-1).
\item [\colorbox{tagtype}{\color{white} \textbf{\textsf{PARAMETER}}}] \textbf{\underline{espServerIpPort}} The url of the ESP file copying service. Defaults to the value of ws\_fs\_server in the environment.
\item [\colorbox{tagtype}{\color{white} \textbf{\textsf{PARAMETER}}}] \textbf{\underline{maxConnections}} The maximum number of target nodes to write to concurrently. Defaults to 1.
\item [\colorbox{tagtype}{\color{white} \textbf{\textsf{PARAMETER}}}] \textbf{\underline{allowOverwrite}} Is it valid to overwrite an existing file of the same name? Defaults to FALSE
\item [\colorbox{tagtype}{\color{white} \textbf{\textsf{PARAMETER}}}] \textbf{\underline{replicate}} Should the copied file also be replicated on the destination? Defaults to FALSE
\item [\colorbox{tagtype}{\color{white} \textbf{\textsf{PARAMETER}}}] \textbf{\underline{asSuperfile}} Should the file be copied as a superfile? If TRUE and source is a superfile, then the operation creates a superfile on the target, creating sub-files as needed and only overwriting existing sub-files whose content has changed. If FALSE, a single file is created. Defaults to FALSE.
\item [\colorbox{tagtype}{\color{white} \textbf{\textsf{PARAMETER}}}] \textbf{\underline{compress}} Whether to compress the new file. Defaults to FALSE.
\item [\colorbox{tagtype}{\color{white} \textbf{\textsf{PARAMETER}}}] \textbf{\underline{forcePush}} Should the copy process be executed on the source nodes (push) or on the destination nodes (pull)? Default is to pull.
\item [\colorbox{tagtype}{\color{white} \textbf{\textsf{PARAMETER}}}] \textbf{\underline{transferBufferSize}} Overrides the size (in bytes) of the internal buffer used to copy the file. Default is 64k.
\item [\colorbox{tagtype}{\color{white} \textbf{\textsf{RETURN}}}] \textbf{\underline{}} The DFU workunit id for the job.
\end{description}

\rule{\linewidth}{0.5pt}
\subsection*{\textsf{\colorbox{headtoc}{\color{white} FUNCTION}
Copy}}

\hypertarget{ecldoc:file.copy}{}
\hspace{0pt} \hyperlink{ecldoc:File}{File} \textbackslash 

{\renewcommand{\arraystretch}{1.5}
\begin{tabularx}{\textwidth}{|>{\raggedright\arraybackslash}l|X|}
\hline
\hspace{0pt}\mytexttt{\color{red} } & \textbf{Copy} \\
\hline
\multicolumn{2}{|>{\raggedright\arraybackslash}X|}{\hspace{0pt}\mytexttt{\color{param} (varstring sourceLogicalName, varstring destinationGroup, varstring destinationLogicalName, varstring sourceDali='', integer4 timeOut=-1, varstring espServerIpPort=GETENV('ws\_fs\_server'), integer4 maxConnections=-1, boolean allowOverwrite=FALSE, boolean replicate=FALSE, boolean asSuperfile=FALSE, boolean compress=FALSE, boolean forcePush=FALSE, integer4 transferBufferSize=0, boolean preserveCompression=TRUE)}} \\
\hline
\end{tabularx}
}

\par
Same as fCopy, but does not return the DFU Workunit ID.

\par
\begin{description}
\item [\colorbox{tagtype}{\color{white} \textbf{\textsf{SEE}}}] \textbf{\underline{}} fCopy
\end{description}

\rule{\linewidth}{0.5pt}
\subsection*{\textsf{\colorbox{headtoc}{\color{white} FUNCTION}
fReplicate}}

\hypertarget{ecldoc:file.freplicate}{}
\hspace{0pt} \hyperlink{ecldoc:File}{File} \textbackslash 

{\renewcommand{\arraystretch}{1.5}
\begin{tabularx}{\textwidth}{|>{\raggedright\arraybackslash}l|X|}
\hline
\hspace{0pt}\mytexttt{\color{red} varstring} & \textbf{fReplicate} \\
\hline
\multicolumn{2}{|>{\raggedright\arraybackslash}X|}{\hspace{0pt}\mytexttt{\color{param} (varstring logicalName, integer4 timeOut=-1, varstring espServerIpPort=GETENV('ws\_fs\_server'))}} \\
\hline
\end{tabularx}
}

\par
Ensures the specified file is replicated to its mirror copies.

\par
\begin{description}
\item [\colorbox{tagtype}{\color{white} \textbf{\textsf{PARAMETER}}}] \textbf{\underline{logicalName}} The name of the file to replicate.
\item [\colorbox{tagtype}{\color{white} \textbf{\textsf{PARAMETER}}}] \textbf{\underline{timeOut}} The time in ms to wait for the operation to complete. A value of 0 causes the call to return immediately. Defaults to no timeout (-1).
\item [\colorbox{tagtype}{\color{white} \textbf{\textsf{PARAMETER}}}] \textbf{\underline{espServerIpPort}} The url of the ESP file copying service. Defaults to the value of ws\_fs\_server in the environment.
\item [\colorbox{tagtype}{\color{white} \textbf{\textsf{RETURN}}}] \textbf{\underline{}} The DFU workunit id for the job.
\end{description}

\rule{\linewidth}{0.5pt}
\subsection*{\textsf{\colorbox{headtoc}{\color{white} FUNCTION}
Replicate}}

\hypertarget{ecldoc:file.replicate}{}
\hspace{0pt} \hyperlink{ecldoc:File}{File} \textbackslash 

{\renewcommand{\arraystretch}{1.5}
\begin{tabularx}{\textwidth}{|>{\raggedright\arraybackslash}l|X|}
\hline
\hspace{0pt}\mytexttt{\color{red} } & \textbf{Replicate} \\
\hline
\multicolumn{2}{|>{\raggedright\arraybackslash}X|}{\hspace{0pt}\mytexttt{\color{param} (varstring logicalName, integer4 timeOut=-1, varstring espServerIpPort=GETENV('ws\_fs\_server'))}} \\
\hline
\end{tabularx}
}

\par
Same as fReplicated, but does not return the DFU Workunit ID.

\par
\begin{description}
\item [\colorbox{tagtype}{\color{white} \textbf{\textsf{SEE}}}] \textbf{\underline{}} fReplicate
\end{description}

\rule{\linewidth}{0.5pt}
\subsection*{\textsf{\colorbox{headtoc}{\color{white} FUNCTION}
fRemotePull}}

\hypertarget{ecldoc:file.fremotepull}{}
\hspace{0pt} \hyperlink{ecldoc:File}{File} \textbackslash 

{\renewcommand{\arraystretch}{1.5}
\begin{tabularx}{\textwidth}{|>{\raggedright\arraybackslash}l|X|}
\hline
\hspace{0pt}\mytexttt{\color{red} varstring} & \textbf{fRemotePull} \\
\hline
\multicolumn{2}{|>{\raggedright\arraybackslash}X|}{\hspace{0pt}\mytexttt{\color{param} (varstring remoteEspFsURL, varstring sourceLogicalName, varstring destinationGroup, varstring destinationLogicalName, integer4 timeOut=-1, integer4 maxConnections=-1, boolean allowOverwrite=FALSE, boolean replicate=FALSE, boolean asSuperfile=FALSE, boolean forcePush=FALSE, integer4 transferBufferSize=0, boolean wrap=FALSE, boolean compress=FALSE)}} \\
\hline
\end{tabularx}
}

\par
Copies a distributed file to a distributed file on remote system. Similar to fCopy, except the copy executes remotely. Since the DFU workunit executes on the remote DFU server, the user name authentication must be the same on both systems, and the user must have rights to copy files on both systems.

\par
\begin{description}
\item [\colorbox{tagtype}{\color{white} \textbf{\textsf{PARAMETER}}}] \textbf{\underline{remoteEspFsURL}} The url of the remote ESP file copying service.
\item [\colorbox{tagtype}{\color{white} \textbf{\textsf{PARAMETER}}}] \textbf{\underline{sourceLogicalName}} The name of the file to despray.
\item [\colorbox{tagtype}{\color{white} \textbf{\textsf{PARAMETER}}}] \textbf{\underline{destinationGroup}} The name of the group to distribute the file across.
\item [\colorbox{tagtype}{\color{white} \textbf{\textsf{PARAMETER}}}] \textbf{\underline{destinationLogicalName}} The logical name of the file to create.
\item [\colorbox{tagtype}{\color{white} \textbf{\textsf{PARAMETER}}}] \textbf{\underline{timeOut}} The time in ms to wait for the operation to complete. A value of 0 causes the call to return immediately. Defaults to no timeout (-1).
\item [\colorbox{tagtype}{\color{white} \textbf{\textsf{PARAMETER}}}] \textbf{\underline{maxConnections}} The maximum number of target nodes to write to concurrently. Defaults to 1.
\item [\colorbox{tagtype}{\color{white} \textbf{\textsf{PARAMETER}}}] \textbf{\underline{allowOverwrite}} Is it valid to overwrite an existing file of the same name? Defaults to FALSE
\item [\colorbox{tagtype}{\color{white} \textbf{\textsf{PARAMETER}}}] \textbf{\underline{replicate}} Should the copied file also be replicated on the destination? Defaults to FALSE
\item [\colorbox{tagtype}{\color{white} \textbf{\textsf{PARAMETER}}}] \textbf{\underline{asSuperfile}} Should the file be copied as a superfile? If TRUE and source is a superfile, then the operation creates a superfile on the target, creating sub-files as needed and only overwriting existing sub-files whose content has changed. If FALSE a single file is created. Defaults to FALSE.
\item [\colorbox{tagtype}{\color{white} \textbf{\textsf{PARAMETER}}}] \textbf{\underline{compress}} Whether to compress the new file. Defaults to FALSE.
\item [\colorbox{tagtype}{\color{white} \textbf{\textsf{PARAMETER}}}] \textbf{\underline{forcePush}} Should the copy process should be executed on the source nodes (push) or on the destination nodes (pull)? Default is to pull.
\item [\colorbox{tagtype}{\color{white} \textbf{\textsf{PARAMETER}}}] \textbf{\underline{transferBufferSize}} Overrides the size (in bytes) of the internal buffer used to copy the file. Default is 64k.
\item [\colorbox{tagtype}{\color{white} \textbf{\textsf{PARAMETER}}}] \textbf{\underline{wrap}} Should the fileparts be wrapped when copying to a smaller sized cluster? The default is FALSE.
\item [\colorbox{tagtype}{\color{white} \textbf{\textsf{RETURN}}}] \textbf{\underline{}} The DFU workunit id for the job.
\end{description}

\rule{\linewidth}{0.5pt}
\subsection*{\textsf{\colorbox{headtoc}{\color{white} FUNCTION}
RemotePull}}

\hypertarget{ecldoc:file.remotepull}{}
\hspace{0pt} \hyperlink{ecldoc:File}{File} \textbackslash 

{\renewcommand{\arraystretch}{1.5}
\begin{tabularx}{\textwidth}{|>{\raggedright\arraybackslash}l|X|}
\hline
\hspace{0pt}\mytexttt{\color{red} } & \textbf{RemotePull} \\
\hline
\multicolumn{2}{|>{\raggedright\arraybackslash}X|}{\hspace{0pt}\mytexttt{\color{param} (varstring remoteEspFsURL, varstring sourceLogicalName, varstring destinationGroup, varstring destinationLogicalName, integer4 timeOut=-1, integer4 maxConnections=-1, boolean allowOverwrite=FALSE, boolean replicate=FALSE, boolean asSuperfile=FALSE, boolean forcePush=FALSE, integer4 transferBufferSize=0, boolean wrap=FALSE, boolean compress=FALSE)}} \\
\hline
\end{tabularx}
}

\par
Same as fRemotePull, but does not return the DFU Workunit ID.

\par
\begin{description}
\item [\colorbox{tagtype}{\color{white} \textbf{\textsf{SEE}}}] \textbf{\underline{}} fRemotePull
\end{description}

\rule{\linewidth}{0.5pt}
\subsection*{\textsf{\colorbox{headtoc}{\color{white} FUNCTION}
fMonitorLogicalFileName}}

\hypertarget{ecldoc:file.fmonitorlogicalfilename}{}
\hspace{0pt} \hyperlink{ecldoc:File}{File} \textbackslash 

{\renewcommand{\arraystretch}{1.5}
\begin{tabularx}{\textwidth}{|>{\raggedright\arraybackslash}l|X|}
\hline
\hspace{0pt}\mytexttt{\color{red} varstring} & \textbf{fMonitorLogicalFileName} \\
\hline
\multicolumn{2}{|>{\raggedright\arraybackslash}X|}{\hspace{0pt}\mytexttt{\color{param} (varstring eventToFire, varstring name, integer4 shotCount=1, varstring espServerIpPort=GETENV('ws\_fs\_server'))}} \\
\hline
\end{tabularx}
}

\par
Creates a file monitor job in the DFU Server. If an appropriately named file arrives in this interval it will fire the event with the name of the triggering object as the event subtype (see the EVENT function).

\par
\begin{description}
\item [\colorbox{tagtype}{\color{white} \textbf{\textsf{PARAMETER}}}] \textbf{\underline{eventToFire}} The user-defined name of the event to fire when the filename appears. This value is used as the first parameter to the EVENT function.
\item [\colorbox{tagtype}{\color{white} \textbf{\textsf{PARAMETER}}}] \textbf{\underline{name}} The name of the logical file to monitor. This may contain wildcard characters ( * and ?)
\item [\colorbox{tagtype}{\color{white} \textbf{\textsf{PARAMETER}}}] \textbf{\underline{shotCount}} The number of times to generate the event before the monitoring job completes. A value of -1 indicates the monitoring job continues until manually aborted. The default is 1.
\item [\colorbox{tagtype}{\color{white} \textbf{\textsf{PARAMETER}}}] \textbf{\underline{espServerIpPort}} The url of the ESP file copying service. Defaults to the value of ws\_fs\_server in the environment.
\item [\colorbox{tagtype}{\color{white} \textbf{\textsf{RETURN}}}] \textbf{\underline{}} The DFU workunit id for the job.
\end{description}

\rule{\linewidth}{0.5pt}
\subsection*{\textsf{\colorbox{headtoc}{\color{white} FUNCTION}
MonitorLogicalFileName}}

\hypertarget{ecldoc:file.monitorlogicalfilename}{}
\hspace{0pt} \hyperlink{ecldoc:File}{File} \textbackslash 

{\renewcommand{\arraystretch}{1.5}
\begin{tabularx}{\textwidth}{|>{\raggedright\arraybackslash}l|X|}
\hline
\hspace{0pt}\mytexttt{\color{red} } & \textbf{MonitorLogicalFileName} \\
\hline
\multicolumn{2}{|>{\raggedright\arraybackslash}X|}{\hspace{0pt}\mytexttt{\color{param} (varstring eventToFire, varstring name, integer4 shotCount=1, varstring espServerIpPort=GETENV('ws\_fs\_server'))}} \\
\hline
\end{tabularx}
}

\par
Same as fMonitorLogicalFileName, but does not return the DFU Workunit ID.

\par
\begin{description}
\item [\colorbox{tagtype}{\color{white} \textbf{\textsf{SEE}}}] \textbf{\underline{}} fMonitorLogicalFileName
\end{description}

\rule{\linewidth}{0.5pt}
\subsection*{\textsf{\colorbox{headtoc}{\color{white} FUNCTION}
fMonitorFile}}

\hypertarget{ecldoc:file.fmonitorfile}{}
\hspace{0pt} \hyperlink{ecldoc:File}{File} \textbackslash 

{\renewcommand{\arraystretch}{1.5}
\begin{tabularx}{\textwidth}{|>{\raggedright\arraybackslash}l|X|}
\hline
\hspace{0pt}\mytexttt{\color{red} varstring} & \textbf{fMonitorFile} \\
\hline
\multicolumn{2}{|>{\raggedright\arraybackslash}X|}{\hspace{0pt}\mytexttt{\color{param} (varstring eventToFire, varstring ip, varstring filename, boolean subDirs=FALSE, integer4 shotCount=1, varstring espServerIpPort=GETENV('ws\_fs\_server'))}} \\
\hline
\end{tabularx}
}

\par
Creates a file monitor job in the DFU Server. If an appropriately named file arrives in this interval it will fire the event with the name of the triggering object as the event subtype (see the EVENT function).

\par
\begin{description}
\item [\colorbox{tagtype}{\color{white} \textbf{\textsf{PARAMETER}}}] \textbf{\underline{eventToFire}} The user-defined name of the event to fire when the filename appears. This value is used as the first parameter to the EVENT function.
\item [\colorbox{tagtype}{\color{white} \textbf{\textsf{PARAMETER}}}] \textbf{\underline{ip}} The the IP address for the file to monitor. This may be omitted if the filename parameter contains a complete URL.
\item [\colorbox{tagtype}{\color{white} \textbf{\textsf{PARAMETER}}}] \textbf{\underline{filename}} The full path of the file(s) to monitor. This may contain wildcard characters ( * and ?)
\item [\colorbox{tagtype}{\color{white} \textbf{\textsf{PARAMETER}}}] \textbf{\underline{subDirs}} Whether to include files in sub-directories (when the filename contains wildcards). Defaults to FALSE.
\item [\colorbox{tagtype}{\color{white} \textbf{\textsf{PARAMETER}}}] \textbf{\underline{shotCount}} The number of times to generate the event before the monitoring job completes. A value of -1 indicates the monitoring job continues until manually aborted. The default is 1.
\item [\colorbox{tagtype}{\color{white} \textbf{\textsf{PARAMETER}}}] \textbf{\underline{espServerIpPort}} The url of the ESP file copying service. Defaults to the value of ws\_fs\_server in the environment.
\item [\colorbox{tagtype}{\color{white} \textbf{\textsf{RETURN}}}] \textbf{\underline{}} The DFU workunit id for the job.
\end{description}

\rule{\linewidth}{0.5pt}
\subsection*{\textsf{\colorbox{headtoc}{\color{white} FUNCTION}
MonitorFile}}

\hypertarget{ecldoc:file.monitorfile}{}
\hspace{0pt} \hyperlink{ecldoc:File}{File} \textbackslash 

{\renewcommand{\arraystretch}{1.5}
\begin{tabularx}{\textwidth}{|>{\raggedright\arraybackslash}l|X|}
\hline
\hspace{0pt}\mytexttt{\color{red} } & \textbf{MonitorFile} \\
\hline
\multicolumn{2}{|>{\raggedright\arraybackslash}X|}{\hspace{0pt}\mytexttt{\color{param} (varstring eventToFire, varstring ip, varstring filename, boolean subdirs=FALSE, integer4 shotCount=1, varstring espServerIpPort=GETENV('ws\_fs\_server'))}} \\
\hline
\end{tabularx}
}

\par
Same as fMonitorFile, but does not return the DFU Workunit ID.

\par
\begin{description}
\item [\colorbox{tagtype}{\color{white} \textbf{\textsf{SEE}}}] \textbf{\underline{}} fMonitorFile
\end{description}

\rule{\linewidth}{0.5pt}
\subsection*{\textsf{\colorbox{headtoc}{\color{white} FUNCTION}
WaitDfuWorkunit}}

\hypertarget{ecldoc:file.waitdfuworkunit}{}
\hspace{0pt} \hyperlink{ecldoc:File}{File} \textbackslash 

{\renewcommand{\arraystretch}{1.5}
\begin{tabularx}{\textwidth}{|>{\raggedright\arraybackslash}l|X|}
\hline
\hspace{0pt}\mytexttt{\color{red} varstring} & \textbf{WaitDfuWorkunit} \\
\hline
\multicolumn{2}{|>{\raggedright\arraybackslash}X|}{\hspace{0pt}\mytexttt{\color{param} (varstring wuid, integer4 timeOut=-1, varstring espServerIpPort=GETENV('ws\_fs\_server'))}} \\
\hline
\end{tabularx}
}

\par
Waits for the specified DFU workunit to finish.

\par
\begin{description}
\item [\colorbox{tagtype}{\color{white} \textbf{\textsf{PARAMETER}}}] \textbf{\underline{wuid}} The dfu wfid to wait for.
\item [\colorbox{tagtype}{\color{white} \textbf{\textsf{PARAMETER}}}] \textbf{\underline{timeOut}} The time in ms to wait for the operation to complete. A value of 0 causes the call to return immediately. Defaults to no timeout (-1).
\item [\colorbox{tagtype}{\color{white} \textbf{\textsf{PARAMETER}}}] \textbf{\underline{espServerIpPort}} The url of the ESP file copying service. Defaults to the value of ws\_fs\_server in the environment.
\item [\colorbox{tagtype}{\color{white} \textbf{\textsf{RETURN}}}] \textbf{\underline{}} A string containing the final status string of the DFU workunit.
\end{description}

\rule{\linewidth}{0.5pt}
\subsection*{\textsf{\colorbox{headtoc}{\color{white} FUNCTION}
AbortDfuWorkunit}}

\hypertarget{ecldoc:file.abortdfuworkunit}{}
\hspace{0pt} \hyperlink{ecldoc:File}{File} \textbackslash 

{\renewcommand{\arraystretch}{1.5}
\begin{tabularx}{\textwidth}{|>{\raggedright\arraybackslash}l|X|}
\hline
\hspace{0pt}\mytexttt{\color{red} } & \textbf{AbortDfuWorkunit} \\
\hline
\multicolumn{2}{|>{\raggedright\arraybackslash}X|}{\hspace{0pt}\mytexttt{\color{param} (varstring wuid, varstring espServerIpPort=GETENV('ws\_fs\_server'))}} \\
\hline
\end{tabularx}
}

\par
Aborts the specified DFU workunit.

\par
\begin{description}
\item [\colorbox{tagtype}{\color{white} \textbf{\textsf{PARAMETER}}}] \textbf{\underline{wuid}} The dfu wfid to abort.
\item [\colorbox{tagtype}{\color{white} \textbf{\textsf{PARAMETER}}}] \textbf{\underline{espServerIpPort}} The url of the ESP file copying service. Defaults to the value of ws\_fs\_server in the environment.
\end{description}

\rule{\linewidth}{0.5pt}
\subsection*{\textsf{\colorbox{headtoc}{\color{white} FUNCTION}
CreateSuperFile}}

\hypertarget{ecldoc:file.createsuperfile}{}
\hspace{0pt} \hyperlink{ecldoc:File}{File} \textbackslash 

{\renewcommand{\arraystretch}{1.5}
\begin{tabularx}{\textwidth}{|>{\raggedright\arraybackslash}l|X|}
\hline
\hspace{0pt}\mytexttt{\color{red} } & \textbf{CreateSuperFile} \\
\hline
\multicolumn{2}{|>{\raggedright\arraybackslash}X|}{\hspace{0pt}\mytexttt{\color{param} (varstring superName, boolean sequentialParts=FALSE, boolean allowExist=FALSE)}} \\
\hline
\end{tabularx}
}

\par
Creates an empty superfile. This function is not included in a superfile transaction.

\par
\begin{description}
\item [\colorbox{tagtype}{\color{white} \textbf{\textsf{PARAMETER}}}] \textbf{\underline{superName}} The logical name of the superfile.
\item [\colorbox{tagtype}{\color{white} \textbf{\textsf{PARAMETER}}}] \textbf{\underline{sequentialParts}} Whether the sub-files must be sequentially ordered. Default to FALSE.
\item [\colorbox{tagtype}{\color{white} \textbf{\textsf{PARAMETER}}}] \textbf{\underline{allowExist}} Indicating whether to post an error if the superfile already exists. If TRUE, no error is posted. Defaults to FALSE.
\end{description}

\rule{\linewidth}{0.5pt}
\subsection*{\textsf{\colorbox{headtoc}{\color{white} FUNCTION}
SuperFileExists}}

\hypertarget{ecldoc:file.superfileexists}{}
\hspace{0pt} \hyperlink{ecldoc:File}{File} \textbackslash 

{\renewcommand{\arraystretch}{1.5}
\begin{tabularx}{\textwidth}{|>{\raggedright\arraybackslash}l|X|}
\hline
\hspace{0pt}\mytexttt{\color{red} boolean} & \textbf{SuperFileExists} \\
\hline
\multicolumn{2}{|>{\raggedright\arraybackslash}X|}{\hspace{0pt}\mytexttt{\color{param} (varstring superName)}} \\
\hline
\end{tabularx}
}

\par
Checks if the specified filename is present in the Distributed File Utility (DFU) and is a SuperFile.

\par
\begin{description}
\item [\colorbox{tagtype}{\color{white} \textbf{\textsf{PARAMETER}}}] \textbf{\underline{superName}} The logical name of the superfile.
\item [\colorbox{tagtype}{\color{white} \textbf{\textsf{RETURN}}}] \textbf{\underline{}} Whether the file exists.
\item [\colorbox{tagtype}{\color{white} \textbf{\textsf{SEE}}}] \textbf{\underline{}} FileExists
\end{description}

\rule{\linewidth}{0.5pt}
\subsection*{\textsf{\colorbox{headtoc}{\color{white} FUNCTION}
DeleteSuperFile}}

\hypertarget{ecldoc:file.deletesuperfile}{}
\hspace{0pt} \hyperlink{ecldoc:File}{File} \textbackslash 

{\renewcommand{\arraystretch}{1.5}
\begin{tabularx}{\textwidth}{|>{\raggedright\arraybackslash}l|X|}
\hline
\hspace{0pt}\mytexttt{\color{red} } & \textbf{DeleteSuperFile} \\
\hline
\multicolumn{2}{|>{\raggedright\arraybackslash}X|}{\hspace{0pt}\mytexttt{\color{param} (varstring superName, boolean deletesub=FALSE)}} \\
\hline
\end{tabularx}
}

\par
Deletes the superfile.

\par
\begin{description}
\item [\colorbox{tagtype}{\color{white} \textbf{\textsf{PARAMETER}}}] \textbf{\underline{superName}} The logical name of the superfile.
\item [\colorbox{tagtype}{\color{white} \textbf{\textsf{SEE}}}] \textbf{\underline{}} FileExists
\end{description}

\rule{\linewidth}{0.5pt}
\subsection*{\textsf{\colorbox{headtoc}{\color{white} FUNCTION}
GetSuperFileSubCount}}

\hypertarget{ecldoc:file.getsuperfilesubcount}{}
\hspace{0pt} \hyperlink{ecldoc:File}{File} \textbackslash 

{\renewcommand{\arraystretch}{1.5}
\begin{tabularx}{\textwidth}{|>{\raggedright\arraybackslash}l|X|}
\hline
\hspace{0pt}\mytexttt{\color{red} unsigned4} & \textbf{GetSuperFileSubCount} \\
\hline
\multicolumn{2}{|>{\raggedright\arraybackslash}X|}{\hspace{0pt}\mytexttt{\color{param} (varstring superName)}} \\
\hline
\end{tabularx}
}

\par
Returns the number of sub-files contained within a superfile.

\par
\begin{description}
\item [\colorbox{tagtype}{\color{white} \textbf{\textsf{PARAMETER}}}] \textbf{\underline{superName}} The logical name of the superfile.
\item [\colorbox{tagtype}{\color{white} \textbf{\textsf{RETURN}}}] \textbf{\underline{}} The number of sub-files within the superfile.
\end{description}

\rule{\linewidth}{0.5pt}
\subsection*{\textsf{\colorbox{headtoc}{\color{white} FUNCTION}
GetSuperFileSubName}}

\hypertarget{ecldoc:file.getsuperfilesubname}{}
\hspace{0pt} \hyperlink{ecldoc:File}{File} \textbackslash 

{\renewcommand{\arraystretch}{1.5}
\begin{tabularx}{\textwidth}{|>{\raggedright\arraybackslash}l|X|}
\hline
\hspace{0pt}\mytexttt{\color{red} varstring} & \textbf{GetSuperFileSubName} \\
\hline
\multicolumn{2}{|>{\raggedright\arraybackslash}X|}{\hspace{0pt}\mytexttt{\color{param} (varstring superName, unsigned4 fileNum, boolean absPath=FALSE)}} \\
\hline
\end{tabularx}
}

\par
Returns the name of the Nth sub-file within a superfile.

\par
\begin{description}
\item [\colorbox{tagtype}{\color{white} \textbf{\textsf{PARAMETER}}}] \textbf{\underline{superName}} The logical name of the superfile.
\item [\colorbox{tagtype}{\color{white} \textbf{\textsf{PARAMETER}}}] \textbf{\underline{fileNum}} The 1-based position of the sub-file to return the name of.
\item [\colorbox{tagtype}{\color{white} \textbf{\textsf{PARAMETER}}}] \textbf{\underline{absPath}} Whether to prepend '\~{}' to the name of the resulting logical file name.
\item [\colorbox{tagtype}{\color{white} \textbf{\textsf{RETURN}}}] \textbf{\underline{}} The logical name of the selected sub-file.
\end{description}

\rule{\linewidth}{0.5pt}
\subsection*{\textsf{\colorbox{headtoc}{\color{white} FUNCTION}
FindSuperFileSubName}}

\hypertarget{ecldoc:file.findsuperfilesubname}{}
\hspace{0pt} \hyperlink{ecldoc:File}{File} \textbackslash 

{\renewcommand{\arraystretch}{1.5}
\begin{tabularx}{\textwidth}{|>{\raggedright\arraybackslash}l|X|}
\hline
\hspace{0pt}\mytexttt{\color{red} unsigned4} & \textbf{FindSuperFileSubName} \\
\hline
\multicolumn{2}{|>{\raggedright\arraybackslash}X|}{\hspace{0pt}\mytexttt{\color{param} (varstring superName, varstring subName)}} \\
\hline
\end{tabularx}
}

\par
Returns the position of a file within a superfile.

\par
\begin{description}
\item [\colorbox{tagtype}{\color{white} \textbf{\textsf{PARAMETER}}}] \textbf{\underline{superName}} The logical name of the superfile.
\item [\colorbox{tagtype}{\color{white} \textbf{\textsf{PARAMETER}}}] \textbf{\underline{subName}} The logical name of the sub-file.
\item [\colorbox{tagtype}{\color{white} \textbf{\textsf{RETURN}}}] \textbf{\underline{}} The 1-based position of the sub-file within the superfile.
\end{description}

\rule{\linewidth}{0.5pt}
\subsection*{\textsf{\colorbox{headtoc}{\color{white} FUNCTION}
StartSuperFileTransaction}}

\hypertarget{ecldoc:file.startsuperfiletransaction}{}
\hspace{0pt} \hyperlink{ecldoc:File}{File} \textbackslash 

{\renewcommand{\arraystretch}{1.5}
\begin{tabularx}{\textwidth}{|>{\raggedright\arraybackslash}l|X|}
\hline
\hspace{0pt}\mytexttt{\color{red} } & \textbf{StartSuperFileTransaction} \\
\hline
\multicolumn{2}{|>{\raggedright\arraybackslash}X|}{\hspace{0pt}\mytexttt{\color{param} ()}} \\
\hline
\end{tabularx}
}

\par
Starts a superfile transaction. All superfile operations within the transaction will either be executed atomically or rolled back when the transaction is finished.


\rule{\linewidth}{0.5pt}
\subsection*{\textsf{\colorbox{headtoc}{\color{white} FUNCTION}
AddSuperFile}}

\hypertarget{ecldoc:file.addsuperfile}{}
\hspace{0pt} \hyperlink{ecldoc:File}{File} \textbackslash 

{\renewcommand{\arraystretch}{1.5}
\begin{tabularx}{\textwidth}{|>{\raggedright\arraybackslash}l|X|}
\hline
\hspace{0pt}\mytexttt{\color{red} } & \textbf{AddSuperFile} \\
\hline
\multicolumn{2}{|>{\raggedright\arraybackslash}X|}{\hspace{0pt}\mytexttt{\color{param} (varstring superName, varstring subName, unsigned4 atPos=0, boolean addContents=FALSE, boolean strict=FALSE)}} \\
\hline
\end{tabularx}
}

\par
Adds a file to a superfile.

\par
\begin{description}
\item [\colorbox{tagtype}{\color{white} \textbf{\textsf{PARAMETER}}}] \textbf{\underline{superName}} The logical name of the superfile.
\item [\colorbox{tagtype}{\color{white} \textbf{\textsf{PARAMETER}}}] \textbf{\underline{subName}} The name of the logical file to add.
\item [\colorbox{tagtype}{\color{white} \textbf{\textsf{PARAMETER}}}] \textbf{\underline{atPos}} The position to add the sub-file, or 0 to append. Defaults to 0.
\item [\colorbox{tagtype}{\color{white} \textbf{\textsf{PARAMETER}}}] \textbf{\underline{addContents}} Controls whether adding a superfile adds the superfile, or its contents. Defaults to FALSE (do not expand).
\item [\colorbox{tagtype}{\color{white} \textbf{\textsf{PARAMETER}}}] \textbf{\underline{strict}} Check addContents only if subName is a superfile, and ensure superfiles exist.
\end{description}

\rule{\linewidth}{0.5pt}
\subsection*{\textsf{\colorbox{headtoc}{\color{white} FUNCTION}
RemoveSuperFile}}

\hypertarget{ecldoc:file.removesuperfile}{}
\hspace{0pt} \hyperlink{ecldoc:File}{File} \textbackslash 

{\renewcommand{\arraystretch}{1.5}
\begin{tabularx}{\textwidth}{|>{\raggedright\arraybackslash}l|X|}
\hline
\hspace{0pt}\mytexttt{\color{red} } & \textbf{RemoveSuperFile} \\
\hline
\multicolumn{2}{|>{\raggedright\arraybackslash}X|}{\hspace{0pt}\mytexttt{\color{param} (varstring superName, varstring subName, boolean del=FALSE, boolean removeContents=FALSE)}} \\
\hline
\end{tabularx}
}

\par
Removes a sub-file from a superfile.

\par
\begin{description}
\item [\colorbox{tagtype}{\color{white} \textbf{\textsf{PARAMETER}}}] \textbf{\underline{superName}} The logical name of the superfile.
\item [\colorbox{tagtype}{\color{white} \textbf{\textsf{PARAMETER}}}] \textbf{\underline{subName}} The name of the sub-file to remove.
\item [\colorbox{tagtype}{\color{white} \textbf{\textsf{PARAMETER}}}] \textbf{\underline{del}} Indicates whether the sub-file should also be removed from the disk. Defaults to FALSE.
\item [\colorbox{tagtype}{\color{white} \textbf{\textsf{PARAMETER}}}] \textbf{\underline{removeContents}} Controls whether the contents of a sub-file which is a superfile should be recursively removed. Defaults to FALSE.
\end{description}

\rule{\linewidth}{0.5pt}
\subsection*{\textsf{\colorbox{headtoc}{\color{white} FUNCTION}
ClearSuperFile}}

\hypertarget{ecldoc:file.clearsuperfile}{}
\hspace{0pt} \hyperlink{ecldoc:File}{File} \textbackslash 

{\renewcommand{\arraystretch}{1.5}
\begin{tabularx}{\textwidth}{|>{\raggedright\arraybackslash}l|X|}
\hline
\hspace{0pt}\mytexttt{\color{red} } & \textbf{ClearSuperFile} \\
\hline
\multicolumn{2}{|>{\raggedright\arraybackslash}X|}{\hspace{0pt}\mytexttt{\color{param} (varstring superName, boolean del=FALSE)}} \\
\hline
\end{tabularx}
}

\par
Removes all sub-files from a superfile.

\par
\begin{description}
\item [\colorbox{tagtype}{\color{white} \textbf{\textsf{PARAMETER}}}] \textbf{\underline{superName}} The logical name of the superfile.
\item [\colorbox{tagtype}{\color{white} \textbf{\textsf{PARAMETER}}}] \textbf{\underline{del}} Indicates whether the sub-files should also be removed from the disk. Defaults to FALSE.
\end{description}

\rule{\linewidth}{0.5pt}
\subsection*{\textsf{\colorbox{headtoc}{\color{white} FUNCTION}
RemoveOwnedSubFiles}}

\hypertarget{ecldoc:file.removeownedsubfiles}{}
\hspace{0pt} \hyperlink{ecldoc:File}{File} \textbackslash 

{\renewcommand{\arraystretch}{1.5}
\begin{tabularx}{\textwidth}{|>{\raggedright\arraybackslash}l|X|}
\hline
\hspace{0pt}\mytexttt{\color{red} } & \textbf{RemoveOwnedSubFiles} \\
\hline
\multicolumn{2}{|>{\raggedright\arraybackslash}X|}{\hspace{0pt}\mytexttt{\color{param} (varstring superName, boolean del=FALSE)}} \\
\hline
\end{tabularx}
}

\par
Removes all soley-owned sub-files from a superfile. If a sub-file is also contained within another superfile then it is retained.

\par
\begin{description}
\item [\colorbox{tagtype}{\color{white} \textbf{\textsf{PARAMETER}}}] \textbf{\underline{superName}} The logical name of the superfile.
\end{description}

\rule{\linewidth}{0.5pt}
\subsection*{\textsf{\colorbox{headtoc}{\color{white} FUNCTION}
DeleteOwnedSubFiles}}

\hypertarget{ecldoc:file.deleteownedsubfiles}{}
\hspace{0pt} \hyperlink{ecldoc:File}{File} \textbackslash 

{\renewcommand{\arraystretch}{1.5}
\begin{tabularx}{\textwidth}{|>{\raggedright\arraybackslash}l|X|}
\hline
\hspace{0pt}\mytexttt{\color{red} } & \textbf{DeleteOwnedSubFiles} \\
\hline
\multicolumn{2}{|>{\raggedright\arraybackslash}X|}{\hspace{0pt}\mytexttt{\color{param} (varstring superName)}} \\
\hline
\end{tabularx}
}

\par
Legacy version of RemoveOwnedSubFiles which was incorrectly named in a previous version.

\par
\begin{description}
\item [\colorbox{tagtype}{\color{white} \textbf{\textsf{SEE}}}] \textbf{\underline{}} RemoveOwnedSubFIles
\end{description}

\rule{\linewidth}{0.5pt}
\subsection*{\textsf{\colorbox{headtoc}{\color{white} FUNCTION}
SwapSuperFile}}

\hypertarget{ecldoc:file.swapsuperfile}{}
\hspace{0pt} \hyperlink{ecldoc:File}{File} \textbackslash 

{\renewcommand{\arraystretch}{1.5}
\begin{tabularx}{\textwidth}{|>{\raggedright\arraybackslash}l|X|}
\hline
\hspace{0pt}\mytexttt{\color{red} } & \textbf{SwapSuperFile} \\
\hline
\multicolumn{2}{|>{\raggedright\arraybackslash}X|}{\hspace{0pt}\mytexttt{\color{param} (varstring superName1, varstring superName2)}} \\
\hline
\end{tabularx}
}

\par
Swap the contents of two superfiles.

\par
\begin{description}
\item [\colorbox{tagtype}{\color{white} \textbf{\textsf{PARAMETER}}}] \textbf{\underline{superName1}} The logical name of the first superfile.
\item [\colorbox{tagtype}{\color{white} \textbf{\textsf{PARAMETER}}}] \textbf{\underline{superName2}} The logical name of the second superfile.
\end{description}

\rule{\linewidth}{0.5pt}
\subsection*{\textsf{\colorbox{headtoc}{\color{white} FUNCTION}
ReplaceSuperFile}}

\hypertarget{ecldoc:file.replacesuperfile}{}
\hspace{0pt} \hyperlink{ecldoc:File}{File} \textbackslash 

{\renewcommand{\arraystretch}{1.5}
\begin{tabularx}{\textwidth}{|>{\raggedright\arraybackslash}l|X|}
\hline
\hspace{0pt}\mytexttt{\color{red} } & \textbf{ReplaceSuperFile} \\
\hline
\multicolumn{2}{|>{\raggedright\arraybackslash}X|}{\hspace{0pt}\mytexttt{\color{param} (varstring superName, varstring oldSubFile, varstring newSubFile)}} \\
\hline
\end{tabularx}
}

\par
Removes a sub-file from a superfile and replaces it with another.

\par
\begin{description}
\item [\colorbox{tagtype}{\color{white} \textbf{\textsf{PARAMETER}}}] \textbf{\underline{superName}} The logical name of the superfile.
\item [\colorbox{tagtype}{\color{white} \textbf{\textsf{PARAMETER}}}] \textbf{\underline{oldSubFile}} The logical name of the sub-file to remove.
\item [\colorbox{tagtype}{\color{white} \textbf{\textsf{PARAMETER}}}] \textbf{\underline{newSubFile}} The logical name of the sub-file to replace within the superfile.
\end{description}

\rule{\linewidth}{0.5pt}
\subsection*{\textsf{\colorbox{headtoc}{\color{white} FUNCTION}
FinishSuperFileTransaction}}

\hypertarget{ecldoc:file.finishsuperfiletransaction}{}
\hspace{0pt} \hyperlink{ecldoc:File}{File} \textbackslash 

{\renewcommand{\arraystretch}{1.5}
\begin{tabularx}{\textwidth}{|>{\raggedright\arraybackslash}l|X|}
\hline
\hspace{0pt}\mytexttt{\color{red} } & \textbf{FinishSuperFileTransaction} \\
\hline
\multicolumn{2}{|>{\raggedright\arraybackslash}X|}{\hspace{0pt}\mytexttt{\color{param} (boolean rollback=FALSE)}} \\
\hline
\end{tabularx}
}

\par
Finishes a superfile transaction. This executes all the operations since the matching StartSuperFileTransaction(). If there are any errors, then all of the operations are rolled back.


\rule{\linewidth}{0.5pt}
\subsection*{\textsf{\colorbox{headtoc}{\color{white} FUNCTION}
SuperFileContents}}

\hypertarget{ecldoc:file.superfilecontents}{}
\hspace{0pt} \hyperlink{ecldoc:File}{File} \textbackslash 

{\renewcommand{\arraystretch}{1.5}
\begin{tabularx}{\textwidth}{|>{\raggedright\arraybackslash}l|X|}
\hline
\hspace{0pt}\mytexttt{\color{red} dataset(FsLogicalFileNameRecord)} & \textbf{SuperFileContents} \\
\hline
\multicolumn{2}{|>{\raggedright\arraybackslash}X|}{\hspace{0pt}\mytexttt{\color{param} (varstring superName, boolean recurse=FALSE)}} \\
\hline
\end{tabularx}
}

\par
Returns the list of sub-files contained within a superfile.

\par
\begin{description}
\item [\colorbox{tagtype}{\color{white} \textbf{\textsf{PARAMETER}}}] \textbf{\underline{superName}} The logical name of the superfile.
\item [\colorbox{tagtype}{\color{white} \textbf{\textsf{PARAMETER}}}] \textbf{\underline{recurse}} Should the contents of child-superfiles be expanded. Default is FALSE.
\item [\colorbox{tagtype}{\color{white} \textbf{\textsf{RETURN}}}] \textbf{\underline{}} A dataset containing the names of the sub-files.
\end{description}

\rule{\linewidth}{0.5pt}
\subsection*{\textsf{\colorbox{headtoc}{\color{white} FUNCTION}
LogicalFileSuperOwners}}

\hypertarget{ecldoc:file.logicalfilesuperowners}{}
\hspace{0pt} \hyperlink{ecldoc:File}{File} \textbackslash 

{\renewcommand{\arraystretch}{1.5}
\begin{tabularx}{\textwidth}{|>{\raggedright\arraybackslash}l|X|}
\hline
\hspace{0pt}\mytexttt{\color{red} dataset(FsLogicalFileNameRecord)} & \textbf{LogicalFileSuperOwners} \\
\hline
\multicolumn{2}{|>{\raggedright\arraybackslash}X|}{\hspace{0pt}\mytexttt{\color{param} (varstring name)}} \\
\hline
\end{tabularx}
}

\par
Returns the list of superfiles that a logical file is contained within.

\par
\begin{description}
\item [\colorbox{tagtype}{\color{white} \textbf{\textsf{PARAMETER}}}] \textbf{\underline{name}} The name of the logical file.
\item [\colorbox{tagtype}{\color{white} \textbf{\textsf{RETURN}}}] \textbf{\underline{}} A dataset containing the names of the superfiles.
\end{description}

\rule{\linewidth}{0.5pt}
\subsection*{\textsf{\colorbox{headtoc}{\color{white} FUNCTION}
LogicalFileSuperSubList}}

\hypertarget{ecldoc:file.logicalfilesupersublist}{}
\hspace{0pt} \hyperlink{ecldoc:File}{File} \textbackslash 

{\renewcommand{\arraystretch}{1.5}
\begin{tabularx}{\textwidth}{|>{\raggedright\arraybackslash}l|X|}
\hline
\hspace{0pt}\mytexttt{\color{red} dataset(FsLogicalSuperSubRecord)} & \textbf{LogicalFileSuperSubList} \\
\hline
\multicolumn{2}{|>{\raggedright\arraybackslash}X|}{\hspace{0pt}\mytexttt{\color{param} ()}} \\
\hline
\end{tabularx}
}

\par
Returns the list of all the superfiles in the system and their component sub-files.

\par
\begin{description}
\item [\colorbox{tagtype}{\color{white} \textbf{\textsf{RETURN}}}] \textbf{\underline{}} A dataset containing pairs of superName,subName for each component file.
\end{description}

\rule{\linewidth}{0.5pt}
\subsection*{\textsf{\colorbox{headtoc}{\color{white} FUNCTION}
fPromoteSuperFileList}}

\hypertarget{ecldoc:file.fpromotesuperfilelist}{}
\hspace{0pt} \hyperlink{ecldoc:File}{File} \textbackslash 

{\renewcommand{\arraystretch}{1.5}
\begin{tabularx}{\textwidth}{|>{\raggedright\arraybackslash}l|X|}
\hline
\hspace{0pt}\mytexttt{\color{red} varstring} & \textbf{fPromoteSuperFileList} \\
\hline
\multicolumn{2}{|>{\raggedright\arraybackslash}X|}{\hspace{0pt}\mytexttt{\color{param} (set of varstring superNames, varstring addHead='', boolean delTail=FALSE, boolean createOnlyOne=FALSE, boolean reverse=FALSE)}} \\
\hline
\end{tabularx}
}

\par
Moves the sub-files from the first entry in the list of superfiles to the next in the list, repeating the process through the list of superfiles.

\par
\begin{description}
\item [\colorbox{tagtype}{\color{white} \textbf{\textsf{PARAMETER}}}] \textbf{\underline{superNames}} A set of the names of the superfiles to act on. Any that do not exist will be created. The contents of each superfile will be moved to the next in the list.
\item [\colorbox{tagtype}{\color{white} \textbf{\textsf{PARAMETER}}}] \textbf{\underline{addHead}} A string containing a comma-delimited list of logical file names to add to the first superfile after the promotion process is complete. Defaults to ''.
\item [\colorbox{tagtype}{\color{white} \textbf{\textsf{PARAMETER}}}] \textbf{\underline{delTail}} Indicates whether to physically delete the contents moved out of the last superfile. The default is FALSE.
\item [\colorbox{tagtype}{\color{white} \textbf{\textsf{PARAMETER}}}] \textbf{\underline{createOnlyOne}} Specifies whether to only create a single superfile (truncate the list at the first non-existent superfile). The default is FALSE.
\item [\colorbox{tagtype}{\color{white} \textbf{\textsf{PARAMETER}}}] \textbf{\underline{reverse}} Reverse the order of processing the superfiles list, effectively 'demoting' instead of 'promoting' the sub-files. The default is FALSE.
\item [\colorbox{tagtype}{\color{white} \textbf{\textsf{RETURN}}}] \textbf{\underline{}} A string containing a comma separated list of the previous sub-file contents of the emptied superfile.
\end{description}

\rule{\linewidth}{0.5pt}
\subsection*{\textsf{\colorbox{headtoc}{\color{white} FUNCTION}
PromoteSuperFileList}}

\hypertarget{ecldoc:file.promotesuperfilelist}{}
\hspace{0pt} \hyperlink{ecldoc:File}{File} \textbackslash 

{\renewcommand{\arraystretch}{1.5}
\begin{tabularx}{\textwidth}{|>{\raggedright\arraybackslash}l|X|}
\hline
\hspace{0pt}\mytexttt{\color{red} } & \textbf{PromoteSuperFileList} \\
\hline
\multicolumn{2}{|>{\raggedright\arraybackslash}X|}{\hspace{0pt}\mytexttt{\color{param} (set of varstring superNames, varstring addHead='', boolean delTail=FALSE, boolean createOnlyOne=FALSE, boolean reverse=FALSE)}} \\
\hline
\end{tabularx}
}

\par
Same as fPromoteSuperFileList, but does not return the DFU Workunit ID.

\par
\begin{description}
\item [\colorbox{tagtype}{\color{white} \textbf{\textsf{SEE}}}] \textbf{\underline{}} fPromoteSuperFileList
\end{description}

\rule{\linewidth}{0.5pt}



\chapter*{math}
\hypertarget{ecldoc:toc:math}{}

\section*{\underline{IMPORTS}}

\section*{\underline{DESCRIPTIONS}}
\subsection*{MODULE : Math}
\hypertarget{ecldoc:Math}{}
\hyperlink{ecldoc:toc:root}{Up} :

{\renewcommand{\arraystretch}{1.5}
\begin{tabularx}{\textwidth}{|>{\raggedright\arraybackslash}l|X|}
\hline
\hspace{0pt} & Math \\
\hline
\end{tabularx}
}

\par


\hyperlink{ecldoc:math.infinity}{Infinity}  |
\hyperlink{ecldoc:math.nan}{NaN}  |
\hyperlink{ecldoc:math.isinfinite}{isInfinite}  |
\hyperlink{ecldoc:math.isnan}{isNaN}  |
\hyperlink{ecldoc:math.isfinite}{isFinite}  |
\hyperlink{ecldoc:math.fmod}{FMod}  |
\hyperlink{ecldoc:math.fmatch}{FMatch}  |

\rule{\linewidth}{0.5pt}

\subsection*{ATTRIBUTE : Infinity}
\hypertarget{ecldoc:math.infinity}{}
\hyperlink{ecldoc:Math}{Up} :
\hspace{0pt} \hyperlink{ecldoc:Math}{Math} \textbackslash 

{\renewcommand{\arraystretch}{1.5}
\begin{tabularx}{\textwidth}{|>{\raggedright\arraybackslash}l|X|}
\hline
\hspace{0pt}REAL8 & Infinity \\
\hline
\end{tabularx}
}

\par
Return a real ''infinity'' value.


\rule{\linewidth}{0.5pt}
\subsection*{ATTRIBUTE : NaN}
\hypertarget{ecldoc:math.nan}{}
\hyperlink{ecldoc:Math}{Up} :
\hspace{0pt} \hyperlink{ecldoc:Math}{Math} \textbackslash 

{\renewcommand{\arraystretch}{1.5}
\begin{tabularx}{\textwidth}{|>{\raggedright\arraybackslash}l|X|}
\hline
\hspace{0pt}REAL8 & NaN \\
\hline
\end{tabularx}
}

\par
Return a non-signalling NaN (Not a Number)value.


\rule{\linewidth}{0.5pt}
\subsection*{FUNCTION : isInfinite}
\hypertarget{ecldoc:math.isinfinite}{}
\hyperlink{ecldoc:Math}{Up} :
\hspace{0pt} \hyperlink{ecldoc:Math}{Math} \textbackslash 

{\renewcommand{\arraystretch}{1.5}
\begin{tabularx}{\textwidth}{|>{\raggedright\arraybackslash}l|X|}
\hline
\hspace{0pt}BOOLEAN & isInfinite \\
\hline
\multicolumn{2}{|>{\raggedright\arraybackslash}X|}{\hspace{0pt}(REAL8 val)} \\
\hline
\end{tabularx}
}

\par
Return whether a real value is infinite (positive or negative).

\par
\begin{description}
\item [\textbf{Parameter}] val ||| The value to test.
\end{description}

\rule{\linewidth}{0.5pt}
\subsection*{FUNCTION : isNaN}
\hypertarget{ecldoc:math.isnan}{}
\hyperlink{ecldoc:Math}{Up} :
\hspace{0pt} \hyperlink{ecldoc:Math}{Math} \textbackslash 

{\renewcommand{\arraystretch}{1.5}
\begin{tabularx}{\textwidth}{|>{\raggedright\arraybackslash}l|X|}
\hline
\hspace{0pt}BOOLEAN & isNaN \\
\hline
\multicolumn{2}{|>{\raggedright\arraybackslash}X|}{\hspace{0pt}(REAL8 val)} \\
\hline
\end{tabularx}
}

\par
Return whether a real value is a NaN (not a number) value.

\par
\begin{description}
\item [\textbf{Parameter}] val ||| The value to test.
\end{description}

\rule{\linewidth}{0.5pt}
\subsection*{FUNCTION : isFinite}
\hypertarget{ecldoc:math.isfinite}{}
\hyperlink{ecldoc:Math}{Up} :
\hspace{0pt} \hyperlink{ecldoc:Math}{Math} \textbackslash 

{\renewcommand{\arraystretch}{1.5}
\begin{tabularx}{\textwidth}{|>{\raggedright\arraybackslash}l|X|}
\hline
\hspace{0pt}BOOLEAN & isFinite \\
\hline
\multicolumn{2}{|>{\raggedright\arraybackslash}X|}{\hspace{0pt}(REAL8 val)} \\
\hline
\end{tabularx}
}

\par
Return whether a real value is a valid value (neither infinite not NaN).

\par
\begin{description}
\item [\textbf{Parameter}] val ||| The value to test.
\end{description}

\rule{\linewidth}{0.5pt}
\subsection*{FUNCTION : FMod}
\hypertarget{ecldoc:math.fmod}{}
\hyperlink{ecldoc:Math}{Up} :
\hspace{0pt} \hyperlink{ecldoc:Math}{Math} \textbackslash 

{\renewcommand{\arraystretch}{1.5}
\begin{tabularx}{\textwidth}{|>{\raggedright\arraybackslash}l|X|}
\hline
\hspace{0pt}REAL8 & FMod \\
\hline
\multicolumn{2}{|>{\raggedright\arraybackslash}X|}{\hspace{0pt}(REAL8 numer, REAL8 denom)} \\
\hline
\end{tabularx}
}

\par
Returns the floating-point remainder of numer/denom (rounded towards zero). If denom is zero, the result depends on the -fdivideByZero flag: 'zero' or unset: return zero. 'nan': return a non-signalling NaN value 'fail': throw an exception

\par
\begin{description}
\item [\textbf{Parameter}] numer ||| The numerator.
\item [\textbf{Parameter}] denom ||| The numerator.
\end{description}

\rule{\linewidth}{0.5pt}
\subsection*{FUNCTION : FMatch}
\hypertarget{ecldoc:math.fmatch}{}
\hyperlink{ecldoc:Math}{Up} :
\hspace{0pt} \hyperlink{ecldoc:Math}{Math} \textbackslash 

{\renewcommand{\arraystretch}{1.5}
\begin{tabularx}{\textwidth}{|>{\raggedright\arraybackslash}l|X|}
\hline
\hspace{0pt}BOOLEAN & FMatch \\
\hline
\multicolumn{2}{|>{\raggedright\arraybackslash}X|}{\hspace{0pt}(REAL8 a, REAL8 b, REAL8 epsilon=0.0)} \\
\hline
\end{tabularx}
}

\par
Returns whether two floating point values are the same, within margin of error epsilon.

\par
\begin{description}
\item [\textbf{Parameter}] a ||| The first value.
\item [\textbf{Parameter}] b ||| The second value.
\item [\textbf{Parameter}] epsilon ||| The allowable margin of error.
\end{description}

\rule{\linewidth}{0.5pt}



\chapter*{Metaphone}
\hypertarget{ecldoc:toc:Metaphone}{}

\section*{\underline{IMPORTS}}
\begin{itemize}
\item lib\_metaphone
\end{itemize}

\section*{\underline{DESCRIPTIONS}}
\subsection*{MODULE : Metaphone}
\hypertarget{ecldoc:Metaphone}{}
\hyperlink{ecldoc:toc:root}{Up} :

{\renewcommand{\arraystretch}{1.5}
\begin{tabularx}{\textwidth}{|>{\raggedright\arraybackslash}l|X|}
\hline
\hspace{0pt} & Metaphone \\
\hline
\end{tabularx}
}

\par


\hyperlink{ecldoc:metaphone.primary}{primary}  |
\hyperlink{ecldoc:metaphone.secondary}{secondary}  |
\hyperlink{ecldoc:metaphone.double}{double}  |

\rule{\linewidth}{0.5pt}

\subsection*{FUNCTION : primary}
\hypertarget{ecldoc:metaphone.primary}{}
\hyperlink{ecldoc:Metaphone}{Up} :
\hspace{0pt} \hyperlink{ecldoc:Metaphone}{Metaphone} \textbackslash 

{\renewcommand{\arraystretch}{1.5}
\begin{tabularx}{\textwidth}{|>{\raggedright\arraybackslash}l|X|}
\hline
\hspace{0pt}String & primary \\
\hline
\multicolumn{2}{|>{\raggedright\arraybackslash}X|}{\hspace{0pt}(STRING src)} \\
\hline
\end{tabularx}
}

\par
Returns the primary metaphone value

\par
\begin{description}
\item [\textbf{Parameter}] src ||| The string whose metphone is to be calculated.
\item [\textbf{See}] http://en.wikipedia.org/wiki/Metaphone\#Double\_Metaphone
\end{description}

\rule{\linewidth}{0.5pt}
\subsection*{FUNCTION : secondary}
\hypertarget{ecldoc:metaphone.secondary}{}
\hyperlink{ecldoc:Metaphone}{Up} :
\hspace{0pt} \hyperlink{ecldoc:Metaphone}{Metaphone} \textbackslash 

{\renewcommand{\arraystretch}{1.5}
\begin{tabularx}{\textwidth}{|>{\raggedright\arraybackslash}l|X|}
\hline
\hspace{0pt}String & secondary \\
\hline
\multicolumn{2}{|>{\raggedright\arraybackslash}X|}{\hspace{0pt}(STRING src)} \\
\hline
\end{tabularx}
}

\par
Returns the secondary metaphone value

\par
\begin{description}
\item [\textbf{Parameter}] src ||| The string whose metphone is to be calculated.
\item [\textbf{See}] http://en.wikipedia.org/wiki/Metaphone\#Double\_Metaphone
\end{description}

\rule{\linewidth}{0.5pt}
\subsection*{FUNCTION : double}
\hypertarget{ecldoc:metaphone.double}{}
\hyperlink{ecldoc:Metaphone}{Up} :
\hspace{0pt} \hyperlink{ecldoc:Metaphone}{Metaphone} \textbackslash 

{\renewcommand{\arraystretch}{1.5}
\begin{tabularx}{\textwidth}{|>{\raggedright\arraybackslash}l|X|}
\hline
\hspace{0pt}String & double \\
\hline
\multicolumn{2}{|>{\raggedright\arraybackslash}X|}{\hspace{0pt}(STRING src)} \\
\hline
\end{tabularx}
}

\par
Returns the double metaphone value (primary and secondary concatenated

\par
\begin{description}
\item [\textbf{Parameter}] src ||| The string whose metphone is to be calculated.
\item [\textbf{See}] http://en.wikipedia.org/wiki/Metaphone\#Double\_Metaphone
\end{description}

\rule{\linewidth}{0.5pt}



\chapter*{str}
\hypertarget{ecldoc:toc:str}{}

\section*{\underline{IMPORTS}}
\begin{itemize}
\item lib\_stringlib
\end{itemize}

\section*{\underline{DESCRIPTIONS}}
\subsection*{MODULE : Str}
\hypertarget{ecldoc:Str}{}
\hyperlink{ecldoc:toc:root}{Up} :

{\renewcommand{\arraystretch}{1.5}
\begin{tabularx}{\textwidth}{|>{\raggedright\arraybackslash}l|X|}
\hline
\hspace{0pt} & Str \\
\hline
\end{tabularx}
}

\par


\hyperlink{ecldoc:str.compareignorecase}{CompareIgnoreCase}  |
\hyperlink{ecldoc:str.equalignorecase}{EqualIgnoreCase}  |
\hyperlink{ecldoc:str.find}{Find}  |
\hyperlink{ecldoc:str.findcount}{FindCount}  |
\hyperlink{ecldoc:str.wildmatch}{WildMatch}  |
\hyperlink{ecldoc:str.contains}{Contains}  |
\hyperlink{ecldoc:str.filterout}{FilterOut}  |
\hyperlink{ecldoc:str.filter}{Filter}  |
\hyperlink{ecldoc:str.substituteincluded}{SubstituteIncluded}  |
\hyperlink{ecldoc:str.substituteexcluded}{SubstituteExcluded}  |
\hyperlink{ecldoc:str.translate}{Translate}  |
\hyperlink{ecldoc:str.tolowercase}{ToLowerCase}  |
\hyperlink{ecldoc:str.touppercase}{ToUpperCase}  |
\hyperlink{ecldoc:str.tocapitalcase}{ToCapitalCase}  |
\hyperlink{ecldoc:str.totitlecase}{ToTitleCase}  |
\hyperlink{ecldoc:str.reverse}{Reverse}  |
\hyperlink{ecldoc:str.findreplace}{FindReplace}  |
\hyperlink{ecldoc:str.extract}{Extract}  |
\hyperlink{ecldoc:str.cleanspaces}{CleanSpaces}  |
\hyperlink{ecldoc:str.startswith}{StartsWith}  |
\hyperlink{ecldoc:str.endswith}{EndsWith}  |
\hyperlink{ecldoc:str.removesuffix}{RemoveSuffix}  |
\hyperlink{ecldoc:str.extractmultiple}{ExtractMultiple}  |
\hyperlink{ecldoc:str.countwords}{CountWords}  |
\hyperlink{ecldoc:str.splitwords}{SplitWords}  |
\hyperlink{ecldoc:str.combinewords}{CombineWords}  |
\hyperlink{ecldoc:str.editdistance}{EditDistance}  |
\hyperlink{ecldoc:str.editdistancewithinradius}{EditDistanceWithinRadius}  |
\hyperlink{ecldoc:str.wordcount}{WordCount}  |
\hyperlink{ecldoc:str.getnthword}{GetNthWord}  |
\hyperlink{ecldoc:str.excludefirstword}{ExcludeFirstWord}  |
\hyperlink{ecldoc:str.excludelastword}{ExcludeLastWord}  |
\hyperlink{ecldoc:str.excludenthword}{ExcludeNthWord}  |
\hyperlink{ecldoc:str.findword}{FindWord}  |
\hyperlink{ecldoc:str.repeat}{Repeat}  |
\hyperlink{ecldoc:str.tohexpairs}{ToHexPairs}  |
\hyperlink{ecldoc:str.fromhexpairs}{FromHexPairs}  |
\hyperlink{ecldoc:str.encodebase64}{EncodeBase64}  |
\hyperlink{ecldoc:str.decodebase64}{DecodeBase64}  |

\rule{\linewidth}{0.5pt}

\subsection*{FUNCTION : CompareIgnoreCase}
\hypertarget{ecldoc:str.compareignorecase}{}
\hyperlink{ecldoc:Str}{Up} :
\hspace{0pt} \hyperlink{ecldoc:Str}{Str} \textbackslash 

{\renewcommand{\arraystretch}{1.5}
\begin{tabularx}{\textwidth}{|>{\raggedright\arraybackslash}l|X|}
\hline
\hspace{0pt}INTEGER4 & CompareIgnoreCase \\
\hline
\multicolumn{2}{|>{\raggedright\arraybackslash}X|}{\hspace{0pt}(STRING src1, STRING src2)} \\
\hline
\end{tabularx}
}

\par
Compares the two strings case insensitively. Returns a negative integer, zero, or a positive integer according to whether the first string is less than, equal to, or greater than the second.

\par
\begin{description}
\item [\textbf{Parameter}] src1 ||| The first string to be compared.
\item [\textbf{Parameter}] src2 ||| The second string to be compared.
\item [\textbf{See}] Str.EqualIgnoreCase
\end{description}

\rule{\linewidth}{0.5pt}
\subsection*{FUNCTION : EqualIgnoreCase}
\hypertarget{ecldoc:str.equalignorecase}{}
\hyperlink{ecldoc:Str}{Up} :
\hspace{0pt} \hyperlink{ecldoc:Str}{Str} \textbackslash 

{\renewcommand{\arraystretch}{1.5}
\begin{tabularx}{\textwidth}{|>{\raggedright\arraybackslash}l|X|}
\hline
\hspace{0pt}BOOLEAN & EqualIgnoreCase \\
\hline
\multicolumn{2}{|>{\raggedright\arraybackslash}X|}{\hspace{0pt}(STRING src1, STRING src2)} \\
\hline
\end{tabularx}
}

\par
Tests whether the two strings are identical ignoring differences in case.

\par
\begin{description}
\item [\textbf{Parameter}] src1 ||| The first string to be compared.
\item [\textbf{Parameter}] src2 ||| The second string to be compared.
\item [\textbf{See}] Str.CompareIgnoreCase
\end{description}

\rule{\linewidth}{0.5pt}
\subsection*{FUNCTION : Find}
\hypertarget{ecldoc:str.find}{}
\hyperlink{ecldoc:Str}{Up} :
\hspace{0pt} \hyperlink{ecldoc:Str}{Str} \textbackslash 

{\renewcommand{\arraystretch}{1.5}
\begin{tabularx}{\textwidth}{|>{\raggedright\arraybackslash}l|X|}
\hline
\hspace{0pt}UNSIGNED4 & Find \\
\hline
\multicolumn{2}{|>{\raggedright\arraybackslash}X|}{\hspace{0pt}(STRING src, STRING sought, UNSIGNED4 instance = 1)} \\
\hline
\end{tabularx}
}

\par
Returns the character position of the nth match of the search string with the first string. If no match is found the attribute returns 0. If an instance is omitted the position of the first instance is returned.

\par
\begin{description}
\item [\textbf{Parameter}] src ||| The string that is searched
\item [\textbf{Parameter}] sought ||| The string being sought.
\item [\textbf{Parameter}] instance ||| Which match instance are we interested in?
\end{description}

\rule{\linewidth}{0.5pt}
\subsection*{FUNCTION : FindCount}
\hypertarget{ecldoc:str.findcount}{}
\hyperlink{ecldoc:Str}{Up} :
\hspace{0pt} \hyperlink{ecldoc:Str}{Str} \textbackslash 

{\renewcommand{\arraystretch}{1.5}
\begin{tabularx}{\textwidth}{|>{\raggedright\arraybackslash}l|X|}
\hline
\hspace{0pt}UNSIGNED4 & FindCount \\
\hline
\multicolumn{2}{|>{\raggedright\arraybackslash}X|}{\hspace{0pt}(STRING src, STRING sought)} \\
\hline
\end{tabularx}
}

\par
Returns the number of occurences of the second string within the first string.

\par
\begin{description}
\item [\textbf{Parameter}] src ||| The string that is searched
\item [\textbf{Parameter}] sought ||| The string being sought.
\end{description}

\rule{\linewidth}{0.5pt}
\subsection*{FUNCTION : WildMatch}
\hypertarget{ecldoc:str.wildmatch}{}
\hyperlink{ecldoc:Str}{Up} :
\hspace{0pt} \hyperlink{ecldoc:Str}{Str} \textbackslash 

{\renewcommand{\arraystretch}{1.5}
\begin{tabularx}{\textwidth}{|>{\raggedright\arraybackslash}l|X|}
\hline
\hspace{0pt}BOOLEAN & WildMatch \\
\hline
\multicolumn{2}{|>{\raggedright\arraybackslash}X|}{\hspace{0pt}(STRING src, STRING \_pattern, BOOLEAN ignore\_case)} \\
\hline
\end{tabularx}
}

\par
Tests if the search string matches the pattern. The pattern can contain wildcards '?' (single character) and '*' (multiple character).

\par
\begin{description}
\item [\textbf{Parameter}] src ||| The string that is being tested.
\item [\textbf{Parameter}] pattern ||| The pattern to match against.
\item [\textbf{Parameter}] ignore\_case ||| Whether to ignore differences in case between characters
\end{description}

\rule{\linewidth}{0.5pt}
\subsection*{FUNCTION : Contains}
\hypertarget{ecldoc:str.contains}{}
\hyperlink{ecldoc:Str}{Up} :
\hspace{0pt} \hyperlink{ecldoc:Str}{Str} \textbackslash 

{\renewcommand{\arraystretch}{1.5}
\begin{tabularx}{\textwidth}{|>{\raggedright\arraybackslash}l|X|}
\hline
\hspace{0pt}BOOLEAN & Contains \\
\hline
\multicolumn{2}{|>{\raggedright\arraybackslash}X|}{\hspace{0pt}(STRING src, STRING \_pattern, BOOLEAN ignore\_case)} \\
\hline
\end{tabularx}
}

\par
Tests if the search string contains each of the characters in the pattern. If the pattern contains duplicate characters those characters will match once for each occurence in the pattern.

\par
\begin{description}
\item [\textbf{Parameter}] src ||| The string that is being tested.
\item [\textbf{Parameter}] pattern ||| The pattern to match against.
\item [\textbf{Parameter}] ignore\_case ||| Whether to ignore differences in case between characters
\end{description}

\rule{\linewidth}{0.5pt}
\subsection*{FUNCTION : FilterOut}
\hypertarget{ecldoc:str.filterout}{}
\hyperlink{ecldoc:Str}{Up} :
\hspace{0pt} \hyperlink{ecldoc:Str}{Str} \textbackslash 

{\renewcommand{\arraystretch}{1.5}
\begin{tabularx}{\textwidth}{|>{\raggedright\arraybackslash}l|X|}
\hline
\hspace{0pt}STRING & FilterOut \\
\hline
\multicolumn{2}{|>{\raggedright\arraybackslash}X|}{\hspace{0pt}(STRING src, STRING filter)} \\
\hline
\end{tabularx}
}

\par
Returns the first string with all characters within the second string removed.

\par
\begin{description}
\item [\textbf{Parameter}] src ||| The string that is being tested.
\item [\textbf{Parameter}] filter ||| The string containing the set of characters to be excluded.
\item [\textbf{See}] Str.Filter
\end{description}

\rule{\linewidth}{0.5pt}
\subsection*{FUNCTION : Filter}
\hypertarget{ecldoc:str.filter}{}
\hyperlink{ecldoc:Str}{Up} :
\hspace{0pt} \hyperlink{ecldoc:Str}{Str} \textbackslash 

{\renewcommand{\arraystretch}{1.5}
\begin{tabularx}{\textwidth}{|>{\raggedright\arraybackslash}l|X|}
\hline
\hspace{0pt}STRING & Filter \\
\hline
\multicolumn{2}{|>{\raggedright\arraybackslash}X|}{\hspace{0pt}(STRING src, STRING filter)} \\
\hline
\end{tabularx}
}

\par
Returns the first string with all characters not within the second string removed.

\par
\begin{description}
\item [\textbf{Parameter}] src ||| The string that is being tested.
\item [\textbf{Parameter}] filter ||| The string containing the set of characters to be included.
\item [\textbf{See}] Str.FilterOut
\end{description}

\rule{\linewidth}{0.5pt}
\subsection*{FUNCTION : SubstituteIncluded}
\hypertarget{ecldoc:str.substituteincluded}{}
\hyperlink{ecldoc:Str}{Up} :
\hspace{0pt} \hyperlink{ecldoc:Str}{Str} \textbackslash 

{\renewcommand{\arraystretch}{1.5}
\begin{tabularx}{\textwidth}{|>{\raggedright\arraybackslash}l|X|}
\hline
\hspace{0pt}STRING & SubstituteIncluded \\
\hline
\multicolumn{2}{|>{\raggedright\arraybackslash}X|}{\hspace{0pt}(STRING src, STRING filter, STRING1 replace\_char)} \\
\hline
\end{tabularx}
}

\par
Returns the source string with the replacement character substituted for all characters included in the filter string. MORE: Should this be a general string substitution?

\par
\begin{description}
\item [\textbf{Parameter}] src ||| The string that is being tested.
\item [\textbf{Parameter}] filter ||| The string containing the set of characters to be included.
\item [\textbf{Parameter}] replace\_char ||| The character to be substituted into the result.
\item [\textbf{See}] Std.Str.Translate, Std.Str.SubstituteExcluded
\end{description}

\rule{\linewidth}{0.5pt}
\subsection*{FUNCTION : SubstituteExcluded}
\hypertarget{ecldoc:str.substituteexcluded}{}
\hyperlink{ecldoc:Str}{Up} :
\hspace{0pt} \hyperlink{ecldoc:Str}{Str} \textbackslash 

{\renewcommand{\arraystretch}{1.5}
\begin{tabularx}{\textwidth}{|>{\raggedright\arraybackslash}l|X|}
\hline
\hspace{0pt}STRING & SubstituteExcluded \\
\hline
\multicolumn{2}{|>{\raggedright\arraybackslash}X|}{\hspace{0pt}(STRING src, STRING filter, STRING1 replace\_char)} \\
\hline
\end{tabularx}
}

\par
Returns the source string with the replacement character substituted for all characters not included in the filter string. MORE: Should this be a general string substitution?

\par
\begin{description}
\item [\textbf{Parameter}] src ||| The string that is being tested.
\item [\textbf{Parameter}] filter ||| The string containing the set of characters to be included.
\item [\textbf{Parameter}] replace\_char ||| The character to be substituted into the result.
\item [\textbf{See}] Std.Str.SubstituteIncluded
\end{description}

\rule{\linewidth}{0.5pt}
\subsection*{FUNCTION : Translate}
\hypertarget{ecldoc:str.translate}{}
\hyperlink{ecldoc:Str}{Up} :
\hspace{0pt} \hyperlink{ecldoc:Str}{Str} \textbackslash 

{\renewcommand{\arraystretch}{1.5}
\begin{tabularx}{\textwidth}{|>{\raggedright\arraybackslash}l|X|}
\hline
\hspace{0pt}STRING & Translate \\
\hline
\multicolumn{2}{|>{\raggedright\arraybackslash}X|}{\hspace{0pt}(STRING src, STRING search, STRING replacement)} \\
\hline
\end{tabularx}
}

\par
Returns the source string with the all characters that match characters in the search string replaced with the character at the corresponding position in the replacement string.

\par
\begin{description}
\item [\textbf{Parameter}] src ||| The string that is being tested.
\item [\textbf{Parameter}] search ||| The string containing the set of characters to be included.
\item [\textbf{Parameter}] replacement ||| The string containing the characters to act as replacements.
\item [\textbf{See}] Std.Str.SubstituteIncluded
\end{description}

\rule{\linewidth}{0.5pt}
\subsection*{FUNCTION : ToLowerCase}
\hypertarget{ecldoc:str.tolowercase}{}
\hyperlink{ecldoc:Str}{Up} :
\hspace{0pt} \hyperlink{ecldoc:Str}{Str} \textbackslash 

{\renewcommand{\arraystretch}{1.5}
\begin{tabularx}{\textwidth}{|>{\raggedright\arraybackslash}l|X|}
\hline
\hspace{0pt}STRING & ToLowerCase \\
\hline
\multicolumn{2}{|>{\raggedright\arraybackslash}X|}{\hspace{0pt}(STRING src)} \\
\hline
\end{tabularx}
}

\par
Returns the argument string with all upper case characters converted to lower case.

\par
\begin{description}
\item [\textbf{Parameter}] src ||| The string that is being converted.
\end{description}

\rule{\linewidth}{0.5pt}
\subsection*{FUNCTION : ToUpperCase}
\hypertarget{ecldoc:str.touppercase}{}
\hyperlink{ecldoc:Str}{Up} :
\hspace{0pt} \hyperlink{ecldoc:Str}{Str} \textbackslash 

{\renewcommand{\arraystretch}{1.5}
\begin{tabularx}{\textwidth}{|>{\raggedright\arraybackslash}l|X|}
\hline
\hspace{0pt}STRING & ToUpperCase \\
\hline
\multicolumn{2}{|>{\raggedright\arraybackslash}X|}{\hspace{0pt}(STRING src)} \\
\hline
\end{tabularx}
}

\par
Return the argument string with all lower case characters converted to upper case.

\par
\begin{description}
\item [\textbf{Parameter}] src ||| The string that is being converted.
\end{description}

\rule{\linewidth}{0.5pt}
\subsection*{FUNCTION : ToCapitalCase}
\hypertarget{ecldoc:str.tocapitalcase}{}
\hyperlink{ecldoc:Str}{Up} :
\hspace{0pt} \hyperlink{ecldoc:Str}{Str} \textbackslash 

{\renewcommand{\arraystretch}{1.5}
\begin{tabularx}{\textwidth}{|>{\raggedright\arraybackslash}l|X|}
\hline
\hspace{0pt}STRING & ToCapitalCase \\
\hline
\multicolumn{2}{|>{\raggedright\arraybackslash}X|}{\hspace{0pt}(STRING src)} \\
\hline
\end{tabularx}
}

\par
Returns the argument string with the first letter of each word in upper case and all other letters left as-is. A contiguous sequence of alphanumeric characters is treated as a word.

\par
\begin{description}
\item [\textbf{Parameter}] src ||| The string that is being converted.
\end{description}

\rule{\linewidth}{0.5pt}
\subsection*{FUNCTION : ToTitleCase}
\hypertarget{ecldoc:str.totitlecase}{}
\hyperlink{ecldoc:Str}{Up} :
\hspace{0pt} \hyperlink{ecldoc:Str}{Str} \textbackslash 

{\renewcommand{\arraystretch}{1.5}
\begin{tabularx}{\textwidth}{|>{\raggedright\arraybackslash}l|X|}
\hline
\hspace{0pt}STRING & ToTitleCase \\
\hline
\multicolumn{2}{|>{\raggedright\arraybackslash}X|}{\hspace{0pt}(STRING src)} \\
\hline
\end{tabularx}
}

\par
Returns the argument string with the first letter of each word in upper case and all other letters lower case. A contiguous sequence of alphanumeric characters is treated as a word.

\par
\begin{description}
\item [\textbf{Parameter}] src ||| The string that is being converted.
\end{description}

\rule{\linewidth}{0.5pt}
\subsection*{FUNCTION : Reverse}
\hypertarget{ecldoc:str.reverse}{}
\hyperlink{ecldoc:Str}{Up} :
\hspace{0pt} \hyperlink{ecldoc:Str}{Str} \textbackslash 

{\renewcommand{\arraystretch}{1.5}
\begin{tabularx}{\textwidth}{|>{\raggedright\arraybackslash}l|X|}
\hline
\hspace{0pt}STRING & Reverse \\
\hline
\multicolumn{2}{|>{\raggedright\arraybackslash}X|}{\hspace{0pt}(STRING src)} \\
\hline
\end{tabularx}
}

\par
Returns the argument string with all characters in reverse order. Note the argument is not TRIMMED before it is reversed.

\par
\begin{description}
\item [\textbf{Parameter}] src ||| The string that is being reversed.
\end{description}

\rule{\linewidth}{0.5pt}
\subsection*{FUNCTION : FindReplace}
\hypertarget{ecldoc:str.findreplace}{}
\hyperlink{ecldoc:Str}{Up} :
\hspace{0pt} \hyperlink{ecldoc:Str}{Str} \textbackslash 

{\renewcommand{\arraystretch}{1.5}
\begin{tabularx}{\textwidth}{|>{\raggedright\arraybackslash}l|X|}
\hline
\hspace{0pt}STRING & FindReplace \\
\hline
\multicolumn{2}{|>{\raggedright\arraybackslash}X|}{\hspace{0pt}(STRING src, STRING sought, STRING replacement)} \\
\hline
\end{tabularx}
}

\par
Returns the source string with the replacement string substituted for all instances of the search string.

\par
\begin{description}
\item [\textbf{Parameter}] src ||| The string that is being transformed.
\item [\textbf{Parameter}] sought ||| The string to be replaced.
\item [\textbf{Parameter}] replacement ||| The string to be substituted into the result.
\end{description}

\rule{\linewidth}{0.5pt}
\subsection*{FUNCTION : Extract}
\hypertarget{ecldoc:str.extract}{}
\hyperlink{ecldoc:Str}{Up} :
\hspace{0pt} \hyperlink{ecldoc:Str}{Str} \textbackslash 

{\renewcommand{\arraystretch}{1.5}
\begin{tabularx}{\textwidth}{|>{\raggedright\arraybackslash}l|X|}
\hline
\hspace{0pt}STRING & Extract \\
\hline
\multicolumn{2}{|>{\raggedright\arraybackslash}X|}{\hspace{0pt}(STRING src, UNSIGNED4 instance)} \\
\hline
\end{tabularx}
}

\par
Returns the nth element from a comma separated string.

\par
\begin{description}
\item [\textbf{Parameter}] src ||| The string containing the comma separated list.
\item [\textbf{Parameter}] instance ||| Which item to select from the list.
\end{description}

\rule{\linewidth}{0.5pt}
\subsection*{FUNCTION : CleanSpaces}
\hypertarget{ecldoc:str.cleanspaces}{}
\hyperlink{ecldoc:Str}{Up} :
\hspace{0pt} \hyperlink{ecldoc:Str}{Str} \textbackslash 

{\renewcommand{\arraystretch}{1.5}
\begin{tabularx}{\textwidth}{|>{\raggedright\arraybackslash}l|X|}
\hline
\hspace{0pt}STRING & CleanSpaces \\
\hline
\multicolumn{2}{|>{\raggedright\arraybackslash}X|}{\hspace{0pt}(STRING src)} \\
\hline
\end{tabularx}
}

\par
Returns the source string with all instances of multiple adjacent space characters (2 or more spaces together) reduced to a single space character. Leading and trailing spaces are removed, and tab characters are converted to spaces.

\par
\begin{description}
\item [\textbf{Parameter}] src ||| The string to be cleaned.
\end{description}

\rule{\linewidth}{0.5pt}
\subsection*{FUNCTION : StartsWith}
\hypertarget{ecldoc:str.startswith}{}
\hyperlink{ecldoc:Str}{Up} :
\hspace{0pt} \hyperlink{ecldoc:Str}{Str} \textbackslash 

{\renewcommand{\arraystretch}{1.5}
\begin{tabularx}{\textwidth}{|>{\raggedright\arraybackslash}l|X|}
\hline
\hspace{0pt}BOOLEAN & StartsWith \\
\hline
\multicolumn{2}{|>{\raggedright\arraybackslash}X|}{\hspace{0pt}(STRING src, STRING prefix)} \\
\hline
\end{tabularx}
}

\par
Returns true if the prefix string matches the leading characters in the source string. Trailing spaces are stripped from the prefix before matching. // x.myString.StartsWith('x') as an alternative syntax would be even better

\par
\begin{description}
\item [\textbf{Parameter}] src ||| The string being searched in.
\item [\textbf{Parameter}] prefix ||| The prefix to search for.
\end{description}

\rule{\linewidth}{0.5pt}
\subsection*{FUNCTION : EndsWith}
\hypertarget{ecldoc:str.endswith}{}
\hyperlink{ecldoc:Str}{Up} :
\hspace{0pt} \hyperlink{ecldoc:Str}{Str} \textbackslash 

{\renewcommand{\arraystretch}{1.5}
\begin{tabularx}{\textwidth}{|>{\raggedright\arraybackslash}l|X|}
\hline
\hspace{0pt}BOOLEAN & EndsWith \\
\hline
\multicolumn{2}{|>{\raggedright\arraybackslash}X|}{\hspace{0pt}(STRING src, STRING suffix)} \\
\hline
\end{tabularx}
}

\par
Returns true if the suffix string matches the trailing characters in the source string. Trailing spaces are stripped from both strings before matching.

\par
\begin{description}
\item [\textbf{Parameter}] src ||| The string being searched in.
\item [\textbf{Parameter}] suffix ||| The prefix to search for.
\end{description}

\rule{\linewidth}{0.5pt}
\subsection*{FUNCTION : RemoveSuffix}
\hypertarget{ecldoc:str.removesuffix}{}
\hyperlink{ecldoc:Str}{Up} :
\hspace{0pt} \hyperlink{ecldoc:Str}{Str} \textbackslash 

{\renewcommand{\arraystretch}{1.5}
\begin{tabularx}{\textwidth}{|>{\raggedright\arraybackslash}l|X|}
\hline
\hspace{0pt}STRING & RemoveSuffix \\
\hline
\multicolumn{2}{|>{\raggedright\arraybackslash}X|}{\hspace{0pt}(STRING src, STRING suffix)} \\
\hline
\end{tabularx}
}

\par
Removes the suffix from the search string, if present, and returns the result. Trailing spaces are stripped from both strings before matching.

\par
\begin{description}
\item [\textbf{Parameter}] src ||| The string being searched in.
\item [\textbf{Parameter}] suffix ||| The prefix to search for.
\end{description}

\rule{\linewidth}{0.5pt}
\subsection*{FUNCTION : ExtractMultiple}
\hypertarget{ecldoc:str.extractmultiple}{}
\hyperlink{ecldoc:Str}{Up} :
\hspace{0pt} \hyperlink{ecldoc:Str}{Str} \textbackslash 

{\renewcommand{\arraystretch}{1.5}
\begin{tabularx}{\textwidth}{|>{\raggedright\arraybackslash}l|X|}
\hline
\hspace{0pt}STRING & ExtractMultiple \\
\hline
\multicolumn{2}{|>{\raggedright\arraybackslash}X|}{\hspace{0pt}(STRING src, UNSIGNED8 mask)} \\
\hline
\end{tabularx}
}

\par
Returns a string containing a list of elements from a comma separated string.

\par
\begin{description}
\item [\textbf{Parameter}] src ||| The string containing the comma separated list.
\item [\textbf{Parameter}] mask ||| A bitmask of which elements should be included. Bit 0 is item1, bit1 item 2 etc.
\end{description}

\rule{\linewidth}{0.5pt}
\subsection*{FUNCTION : CountWords}
\hypertarget{ecldoc:str.countwords}{}
\hyperlink{ecldoc:Str}{Up} :
\hspace{0pt} \hyperlink{ecldoc:Str}{Str} \textbackslash 

{\renewcommand{\arraystretch}{1.5}
\begin{tabularx}{\textwidth}{|>{\raggedright\arraybackslash}l|X|}
\hline
\hspace{0pt}UNSIGNED4 & CountWords \\
\hline
\multicolumn{2}{|>{\raggedright\arraybackslash}X|}{\hspace{0pt}(STRING src, STRING separator, BOOLEAN allow\_blank = FALSE)} \\
\hline
\end{tabularx}
}

\par
Returns the number of words that the string contains. Words are separated by one or more separator strings. No spaces are stripped from either string before matching.

\par
\begin{description}
\item [\textbf{Parameter}] src ||| The string being searched in.
\item [\textbf{Parameter}] separator ||| The string used to separate words
\item [\textbf{Parameter}] allow\_blank ||| Indicates if empty/blank string items are included in the results.
\end{description}

\rule{\linewidth}{0.5pt}
\subsection*{FUNCTION : SplitWords}
\hypertarget{ecldoc:str.splitwords}{}
\hyperlink{ecldoc:Str}{Up} :
\hspace{0pt} \hyperlink{ecldoc:Str}{Str} \textbackslash 

{\renewcommand{\arraystretch}{1.5}
\begin{tabularx}{\textwidth}{|>{\raggedright\arraybackslash}l|X|}
\hline
\hspace{0pt}SET OF STRING & SplitWords \\
\hline
\multicolumn{2}{|>{\raggedright\arraybackslash}X|}{\hspace{0pt}(STRING src, STRING separator, BOOLEAN allow\_blank = FALSE)} \\
\hline
\end{tabularx}
}

\par
Returns the list of words extracted from the string. Words are separated by one or more separator strings. No spaces are stripped from either string before matching.

\par
\begin{description}
\item [\textbf{Parameter}] src ||| The string being searched in.
\item [\textbf{Parameter}] separator ||| The string used to separate words
\item [\textbf{Parameter}] allow\_blank ||| Indicates if empty/blank string items are included in the results.
\end{description}

\rule{\linewidth}{0.5pt}
\subsection*{FUNCTION : CombineWords}
\hypertarget{ecldoc:str.combinewords}{}
\hyperlink{ecldoc:Str}{Up} :
\hspace{0pt} \hyperlink{ecldoc:Str}{Str} \textbackslash 

{\renewcommand{\arraystretch}{1.5}
\begin{tabularx}{\textwidth}{|>{\raggedright\arraybackslash}l|X|}
\hline
\hspace{0pt}STRING & CombineWords \\
\hline
\multicolumn{2}{|>{\raggedright\arraybackslash}X|}{\hspace{0pt}(SET OF STRING words, STRING separator)} \\
\hline
\end{tabularx}
}

\par
Returns the list of words extracted from the string. Words are separated by one or more separator strings. No spaces are stripped from either string before matching.

\par
\begin{description}
\item [\textbf{Parameter}] words ||| The set of strings to be combined.
\item [\textbf{Parameter}] separator ||| The string used to separate words.
\end{description}

\rule{\linewidth}{0.5pt}
\subsection*{FUNCTION : EditDistance}
\hypertarget{ecldoc:str.editdistance}{}
\hyperlink{ecldoc:Str}{Up} :
\hspace{0pt} \hyperlink{ecldoc:Str}{Str} \textbackslash 

{\renewcommand{\arraystretch}{1.5}
\begin{tabularx}{\textwidth}{|>{\raggedright\arraybackslash}l|X|}
\hline
\hspace{0pt}UNSIGNED4 & EditDistance \\
\hline
\multicolumn{2}{|>{\raggedright\arraybackslash}X|}{\hspace{0pt}(STRING \_left, STRING \_right)} \\
\hline
\end{tabularx}
}

\par
Returns the minimum edit distance between the two strings. An insert change or delete counts as a single edit. The two strings are trimmed before comparing.

\par
\begin{description}
\item [\textbf{Parameter}] \_left ||| The first string to be compared.
\item [\textbf{Parameter}] \_right ||| The second string to be compared.
\item [\textbf{Return}] The minimum edit distance between the two strings.
\end{description}

\rule{\linewidth}{0.5pt}
\subsection*{FUNCTION : EditDistanceWithinRadius}
\hypertarget{ecldoc:str.editdistancewithinradius}{}
\hyperlink{ecldoc:Str}{Up} :
\hspace{0pt} \hyperlink{ecldoc:Str}{Str} \textbackslash 

{\renewcommand{\arraystretch}{1.5}
\begin{tabularx}{\textwidth}{|>{\raggedright\arraybackslash}l|X|}
\hline
\hspace{0pt}BOOLEAN & EditDistanceWithinRadius \\
\hline
\multicolumn{2}{|>{\raggedright\arraybackslash}X|}{\hspace{0pt}(STRING \_left, STRING \_right, UNSIGNED4 radius)} \\
\hline
\end{tabularx}
}

\par
Returns true if the minimum edit distance between the two strings is with a specific range. The two strings are trimmed before comparing.

\par
\begin{description}
\item [\textbf{Parameter}] \_left ||| The first string to be compared.
\item [\textbf{Parameter}] \_right ||| The second string to be compared.
\item [\textbf{Parameter}] radius ||| The maximum edit distance that is accepable.
\item [\textbf{Return}] Whether or not the two strings are within the given specified edit distance.
\end{description}

\rule{\linewidth}{0.5pt}
\subsection*{FUNCTION : WordCount}
\hypertarget{ecldoc:str.wordcount}{}
\hyperlink{ecldoc:Str}{Up} :
\hspace{0pt} \hyperlink{ecldoc:Str}{Str} \textbackslash 

{\renewcommand{\arraystretch}{1.5}
\begin{tabularx}{\textwidth}{|>{\raggedright\arraybackslash}l|X|}
\hline
\hspace{0pt}UNSIGNED4 & WordCount \\
\hline
\multicolumn{2}{|>{\raggedright\arraybackslash}X|}{\hspace{0pt}(STRING text)} \\
\hline
\end{tabularx}
}

\par
Returns the number of words in the string. Words are separated by one or more spaces.

\par
\begin{description}
\item [\textbf{Parameter}] text ||| The string to be broken into words.
\item [\textbf{Return}] The number of words in the string.
\end{description}

\rule{\linewidth}{0.5pt}
\subsection*{FUNCTION : GetNthWord}
\hypertarget{ecldoc:str.getnthword}{}
\hyperlink{ecldoc:Str}{Up} :
\hspace{0pt} \hyperlink{ecldoc:Str}{Str} \textbackslash 

{\renewcommand{\arraystretch}{1.5}
\begin{tabularx}{\textwidth}{|>{\raggedright\arraybackslash}l|X|}
\hline
\hspace{0pt}STRING & GetNthWord \\
\hline
\multicolumn{2}{|>{\raggedright\arraybackslash}X|}{\hspace{0pt}(STRING text, UNSIGNED4 n)} \\
\hline
\end{tabularx}
}

\par
Returns the n-th word from the string. Words are separated by one or more spaces.

\par
\begin{description}
\item [\textbf{Parameter}] text ||| The string to be broken into words.
\item [\textbf{Parameter}] n ||| Which word should be returned from the function.
\item [\textbf{Return}] The number of words in the string.
\end{description}

\rule{\linewidth}{0.5pt}
\subsection*{FUNCTION : ExcludeFirstWord}
\hypertarget{ecldoc:str.excludefirstword}{}
\hyperlink{ecldoc:Str}{Up} :
\hspace{0pt} \hyperlink{ecldoc:Str}{Str} \textbackslash 

{\renewcommand{\arraystretch}{1.5}
\begin{tabularx}{\textwidth}{|>{\raggedright\arraybackslash}l|X|}
\hline
\hspace{0pt} & ExcludeFirstWord \\
\hline
\multicolumn{2}{|>{\raggedright\arraybackslash}X|}{\hspace{0pt}(STRING text)} \\
\hline
\end{tabularx}
}

\par
Returns everything except the first word from the string. Words are separated by one or more whitespace characters. Whitespace before and after the first word is also removed.

\par
\begin{description}
\item [\textbf{Parameter}] text ||| The string to be broken into words.
\item [\textbf{Return}] The string excluding the first word.
\end{description}

\rule{\linewidth}{0.5pt}
\subsection*{FUNCTION : ExcludeLastWord}
\hypertarget{ecldoc:str.excludelastword}{}
\hyperlink{ecldoc:Str}{Up} :
\hspace{0pt} \hyperlink{ecldoc:Str}{Str} \textbackslash 

{\renewcommand{\arraystretch}{1.5}
\begin{tabularx}{\textwidth}{|>{\raggedright\arraybackslash}l|X|}
\hline
\hspace{0pt} & ExcludeLastWord \\
\hline
\multicolumn{2}{|>{\raggedright\arraybackslash}X|}{\hspace{0pt}(STRING text)} \\
\hline
\end{tabularx}
}

\par
Returns everything except the last word from the string. Words are separated by one or more whitespace characters. Whitespace after a word is removed with the word and leading whitespace is removed with the first word.

\par
\begin{description}
\item [\textbf{Parameter}] text ||| The string to be broken into words.
\item [\textbf{Return}] The string excluding the last word.
\end{description}

\rule{\linewidth}{0.5pt}
\subsection*{FUNCTION : ExcludeNthWord}
\hypertarget{ecldoc:str.excludenthword}{}
\hyperlink{ecldoc:Str}{Up} :
\hspace{0pt} \hyperlink{ecldoc:Str}{Str} \textbackslash 

{\renewcommand{\arraystretch}{1.5}
\begin{tabularx}{\textwidth}{|>{\raggedright\arraybackslash}l|X|}
\hline
\hspace{0pt} & ExcludeNthWord \\
\hline
\multicolumn{2}{|>{\raggedright\arraybackslash}X|}{\hspace{0pt}(STRING text, UNSIGNED2 n)} \\
\hline
\end{tabularx}
}

\par
Returns everything except the nth word from the string. Words are separated by one or more whitespace characters. Whitespace after a word is removed with the word and leading whitespace is removed with the first word.

\par
\begin{description}
\item [\textbf{Parameter}] text ||| The string to be broken into words.
\item [\textbf{Parameter}] n ||| Which word should be returned from the function.
\item [\textbf{Return}] The string excluding the nth word.
\end{description}

\rule{\linewidth}{0.5pt}
\subsection*{FUNCTION : FindWord}
\hypertarget{ecldoc:str.findword}{}
\hyperlink{ecldoc:Str}{Up} :
\hspace{0pt} \hyperlink{ecldoc:Str}{Str} \textbackslash 

{\renewcommand{\arraystretch}{1.5}
\begin{tabularx}{\textwidth}{|>{\raggedright\arraybackslash}l|X|}
\hline
\hspace{0pt}BOOLEAN & FindWord \\
\hline
\multicolumn{2}{|>{\raggedright\arraybackslash}X|}{\hspace{0pt}(STRING src, STRING word, BOOLEAN ignore\_case=FALSE)} \\
\hline
\end{tabularx}
}

\par
Tests if the search string contains the supplied word as a whole word.

\par
\begin{description}
\item [\textbf{Parameter}] src ||| The string that is being tested.
\item [\textbf{Parameter}] word ||| The word to be searched for.
\item [\textbf{Parameter}] ignore\_case ||| Whether to ignore differences in case between characters.
\end{description}

\rule{\linewidth}{0.5pt}
\subsection*{FUNCTION : Repeat}
\hypertarget{ecldoc:str.repeat}{}
\hyperlink{ecldoc:Str}{Up} :
\hspace{0pt} \hyperlink{ecldoc:Str}{Str} \textbackslash 

{\renewcommand{\arraystretch}{1.5}
\begin{tabularx}{\textwidth}{|>{\raggedright\arraybackslash}l|X|}
\hline
\hspace{0pt}STRING & Repeat \\
\hline
\multicolumn{2}{|>{\raggedright\arraybackslash}X|}{\hspace{0pt}(STRING text, UNSIGNED4 n)} \\
\hline
\end{tabularx}
}

\par


\rule{\linewidth}{0.5pt}
\subsection*{FUNCTION : ToHexPairs}
\hypertarget{ecldoc:str.tohexpairs}{}
\hyperlink{ecldoc:Str}{Up} :
\hspace{0pt} \hyperlink{ecldoc:Str}{Str} \textbackslash 

{\renewcommand{\arraystretch}{1.5}
\begin{tabularx}{\textwidth}{|>{\raggedright\arraybackslash}l|X|}
\hline
\hspace{0pt}STRING & ToHexPairs \\
\hline
\multicolumn{2}{|>{\raggedright\arraybackslash}X|}{\hspace{0pt}(DATA value)} \\
\hline
\end{tabularx}
}

\par


\rule{\linewidth}{0.5pt}
\subsection*{FUNCTION : FromHexPairs}
\hypertarget{ecldoc:str.fromhexpairs}{}
\hyperlink{ecldoc:Str}{Up} :
\hspace{0pt} \hyperlink{ecldoc:Str}{Str} \textbackslash 

{\renewcommand{\arraystretch}{1.5}
\begin{tabularx}{\textwidth}{|>{\raggedright\arraybackslash}l|X|}
\hline
\hspace{0pt}DATA & FromHexPairs \\
\hline
\multicolumn{2}{|>{\raggedright\arraybackslash}X|}{\hspace{0pt}(STRING hex\_pairs)} \\
\hline
\end{tabularx}
}

\par


\rule{\linewidth}{0.5pt}
\subsection*{FUNCTION : EncodeBase64}
\hypertarget{ecldoc:str.encodebase64}{}
\hyperlink{ecldoc:Str}{Up} :
\hspace{0pt} \hyperlink{ecldoc:Str}{Str} \textbackslash 

{\renewcommand{\arraystretch}{1.5}
\begin{tabularx}{\textwidth}{|>{\raggedright\arraybackslash}l|X|}
\hline
\hspace{0pt}STRING & EncodeBase64 \\
\hline
\multicolumn{2}{|>{\raggedright\arraybackslash}X|}{\hspace{0pt}(DATA value)} \\
\hline
\end{tabularx}
}

\par


\rule{\linewidth}{0.5pt}
\subsection*{FUNCTION : DecodeBase64}
\hypertarget{ecldoc:str.decodebase64}{}
\hyperlink{ecldoc:Str}{Up} :
\hspace{0pt} \hyperlink{ecldoc:Str}{Str} \textbackslash 

{\renewcommand{\arraystretch}{1.5}
\begin{tabularx}{\textwidth}{|>{\raggedright\arraybackslash}l|X|}
\hline
\hspace{0pt}DATA & DecodeBase64 \\
\hline
\multicolumn{2}{|>{\raggedright\arraybackslash}X|}{\hspace{0pt}(STRING value)} \\
\hline
\end{tabularx}
}

\par


\rule{\linewidth}{0.5pt}



\chapter*{\color{headfile}
Uni
}
\hypertarget{ecldoc:toc:Uni}{}
\hyperlink{ecldoc:toc:root}{Go Up}

\section*{\underline{\textsf{IMPORTS}}}
\begin{doublespace}
{\large
lib\_unicodelib |
}
\end{doublespace}

\section*{\underline{\textsf{DESCRIPTIONS}}}
\subsection*{\textsf{\colorbox{headtoc}{\color{white} MODULE}
Uni}}

\hypertarget{ecldoc:Uni}{}

{\renewcommand{\arraystretch}{1.5}
\begin{tabularx}{\textwidth}{|>{\raggedright\arraybackslash}l|X|}
\hline
\hspace{0pt}\mytexttt{\color{red} } & \textbf{Uni} \\
\hline
\end{tabularx}
}

\par





No Documentation Found







\textbf{Children}
\begin{enumerate}
\item \hyperlink{ecldoc:uni.filterout}{FilterOut}
: Returns the first string with all characters within the second string removed
\item \hyperlink{ecldoc:uni.filter}{Filter}
: Returns the first string with all characters not within the second string removed
\item \hyperlink{ecldoc:uni.substituteincluded}{SubstituteIncluded}
: Returns the source string with the replacement character substituted for all characters included in the filter string
\item \hyperlink{ecldoc:uni.substituteexcluded}{SubstituteExcluded}
: Returns the source string with the replacement character substituted for all characters not included in the filter string
\item \hyperlink{ecldoc:uni.find}{Find}
: Returns the character position of the nth match of the search string with the first string
\item \hyperlink{ecldoc:uni.findword}{FindWord}
: Tests if the search string contains the supplied word as a whole word
\item \hyperlink{ecldoc:uni.localefind}{LocaleFind}
: Returns the character position of the nth match of the search string with the first string
\item \hyperlink{ecldoc:uni.localefindatstrength}{LocaleFindAtStrength}
: Returns the character position of the nth match of the search string with the first string
\item \hyperlink{ecldoc:uni.extract}{Extract}
: Returns the nth element from a comma separated string
\item \hyperlink{ecldoc:uni.tolowercase}{ToLowerCase}
: Returns the argument string with all upper case characters converted to lower case
\item \hyperlink{ecldoc:uni.touppercase}{ToUpperCase}
: Return the argument string with all lower case characters converted to upper case
\item \hyperlink{ecldoc:uni.totitlecase}{ToTitleCase}
: Returns the upper case variant of the string using the rules for a particular locale
\item \hyperlink{ecldoc:uni.localetolowercase}{LocaleToLowerCase}
: Returns the lower case variant of the string using the rules for a particular locale
\item \hyperlink{ecldoc:uni.localetouppercase}{LocaleToUpperCase}
: Returns the upper case variant of the string using the rules for a particular locale
\item \hyperlink{ecldoc:uni.localetotitlecase}{LocaleToTitleCase}
: Returns the upper case variant of the string using the rules for a particular locale
\item \hyperlink{ecldoc:uni.compareignorecase}{CompareIgnoreCase}
: Compares the two strings case insensitively
\item \hyperlink{ecldoc:uni.compareatstrength}{CompareAtStrength}
: Compares the two strings case insensitively
\item \hyperlink{ecldoc:uni.localecompareignorecase}{LocaleCompareIgnoreCase}
: Compares the two strings case insensitively
\item \hyperlink{ecldoc:uni.localecompareatstrength}{LocaleCompareAtStrength}
: Compares the two strings case insensitively
\item \hyperlink{ecldoc:uni.reverse}{Reverse}
: Returns the argument string with all characters in reverse order
\item \hyperlink{ecldoc:uni.findreplace}{FindReplace}
: Returns the source string with the replacement string substituted for all instances of the search string
\item \hyperlink{ecldoc:uni.localefindreplace}{LocaleFindReplace}
: Returns the source string with the replacement string substituted for all instances of the search string
\item \hyperlink{ecldoc:uni.localefindatstrengthreplace}{LocaleFindAtStrengthReplace}
: Returns the source string with the replacement string substituted for all instances of the search string
\item \hyperlink{ecldoc:uni.cleanaccents}{CleanAccents}
: Returns the source string with all accented characters replaced with unaccented
\item \hyperlink{ecldoc:uni.cleanspaces}{CleanSpaces}
: Returns the source string with all instances of multiple adjacent space characters (2 or more spaces together) reduced to a single space character
\item \hyperlink{ecldoc:uni.wildmatch}{WildMatch}
: Tests if the search string matches the pattern
\item \hyperlink{ecldoc:uni.contains}{Contains}
: Tests if the search string contains each of the characters in the pattern
\item \hyperlink{ecldoc:uni.editdistance}{EditDistance}
: Returns the minimum edit distance between the two strings
\item \hyperlink{ecldoc:uni.editdistancewithinradius}{EditDistanceWithinRadius}
: Returns true if the minimum edit distance between the two strings is with a specific range
\item \hyperlink{ecldoc:uni.wordcount}{WordCount}
: Returns the number of words in the string
\item \hyperlink{ecldoc:uni.getnthword}{GetNthWord}
: Returns the n-th word from the string
\end{enumerate}

\rule{\linewidth}{0.5pt}

\subsection*{\textsf{\colorbox{headtoc}{\color{white} FUNCTION}
FilterOut}}

\hypertarget{ecldoc:uni.filterout}{}
\hspace{0pt} \hyperlink{ecldoc:Uni}{Uni} \textbackslash 

{\renewcommand{\arraystretch}{1.5}
\begin{tabularx}{\textwidth}{|>{\raggedright\arraybackslash}l|X|}
\hline
\hspace{0pt}\mytexttt{\color{red} unicode} & \textbf{FilterOut} \\
\hline
\multicolumn{2}{|>{\raggedright\arraybackslash}X|}{\hspace{0pt}\mytexttt{\color{param} (unicode src, unicode filter)}} \\
\hline
\end{tabularx}
}

\par





Returns the first string with all characters within the second string removed.






\par
\begin{description}
\item [\colorbox{tagtype}{\color{white} \textbf{\textsf{PARAMETER}}}] \textbf{\underline{src}} ||| UNICODE --- The string that is being tested.
\item [\colorbox{tagtype}{\color{white} \textbf{\textsf{PARAMETER}}}] \textbf{\underline{filter}} ||| UNICODE --- The string containing the set of characters to be excluded.
\end{description}







\par
\begin{description}
\item [\colorbox{tagtype}{\color{white} \textbf{\textsf{RETURN}}}] \textbf{UNICODE} --- 
\end{description}






\par
\begin{description}
\item [\colorbox{tagtype}{\color{white} \textbf{\textsf{SEE}}}] Std.Uni.Filter
\end{description}




\rule{\linewidth}{0.5pt}
\subsection*{\textsf{\colorbox{headtoc}{\color{white} FUNCTION}
Filter}}

\hypertarget{ecldoc:uni.filter}{}
\hspace{0pt} \hyperlink{ecldoc:Uni}{Uni} \textbackslash 

{\renewcommand{\arraystretch}{1.5}
\begin{tabularx}{\textwidth}{|>{\raggedright\arraybackslash}l|X|}
\hline
\hspace{0pt}\mytexttt{\color{red} unicode} & \textbf{Filter} \\
\hline
\multicolumn{2}{|>{\raggedright\arraybackslash}X|}{\hspace{0pt}\mytexttt{\color{param} (unicode src, unicode filter)}} \\
\hline
\end{tabularx}
}

\par





Returns the first string with all characters not within the second string removed.






\par
\begin{description}
\item [\colorbox{tagtype}{\color{white} \textbf{\textsf{PARAMETER}}}] \textbf{\underline{src}} ||| UNICODE --- The string that is being tested.
\item [\colorbox{tagtype}{\color{white} \textbf{\textsf{PARAMETER}}}] \textbf{\underline{filter}} ||| UNICODE --- The string containing the set of characters to be included.
\end{description}







\par
\begin{description}
\item [\colorbox{tagtype}{\color{white} \textbf{\textsf{RETURN}}}] \textbf{UNICODE} --- 
\end{description}






\par
\begin{description}
\item [\colorbox{tagtype}{\color{white} \textbf{\textsf{SEE}}}] Std.Uni.FilterOut
\end{description}




\rule{\linewidth}{0.5pt}
\subsection*{\textsf{\colorbox{headtoc}{\color{white} FUNCTION}
SubstituteIncluded}}

\hypertarget{ecldoc:uni.substituteincluded}{}
\hspace{0pt} \hyperlink{ecldoc:Uni}{Uni} \textbackslash 

{\renewcommand{\arraystretch}{1.5}
\begin{tabularx}{\textwidth}{|>{\raggedright\arraybackslash}l|X|}
\hline
\hspace{0pt}\mytexttt{\color{red} unicode} & \textbf{SubstituteIncluded} \\
\hline
\multicolumn{2}{|>{\raggedright\arraybackslash}X|}{\hspace{0pt}\mytexttt{\color{param} (unicode src, unicode filter, unicode replace\_char)}} \\
\hline
\end{tabularx}
}

\par





Returns the source string with the replacement character substituted for all characters included in the filter string. MORE: Should this be a general string substitution?






\par
\begin{description}
\item [\colorbox{tagtype}{\color{white} \textbf{\textsf{PARAMETER}}}] \textbf{\underline{replace\_char}} ||| UNICODE --- The character to be substituted into the result.
\item [\colorbox{tagtype}{\color{white} \textbf{\textsf{PARAMETER}}}] \textbf{\underline{src}} ||| UNICODE --- The string that is being tested.
\item [\colorbox{tagtype}{\color{white} \textbf{\textsf{PARAMETER}}}] \textbf{\underline{filter}} ||| UNICODE --- The string containing the set of characters to be included.
\end{description}







\par
\begin{description}
\item [\colorbox{tagtype}{\color{white} \textbf{\textsf{RETURN}}}] \textbf{UNICODE} --- 
\end{description}






\par
\begin{description}
\item [\colorbox{tagtype}{\color{white} \textbf{\textsf{SEE}}}] Std.Uni.SubstituteOut
\end{description}




\rule{\linewidth}{0.5pt}
\subsection*{\textsf{\colorbox{headtoc}{\color{white} FUNCTION}
SubstituteExcluded}}

\hypertarget{ecldoc:uni.substituteexcluded}{}
\hspace{0pt} \hyperlink{ecldoc:Uni}{Uni} \textbackslash 

{\renewcommand{\arraystretch}{1.5}
\begin{tabularx}{\textwidth}{|>{\raggedright\arraybackslash}l|X|}
\hline
\hspace{0pt}\mytexttt{\color{red} unicode} & \textbf{SubstituteExcluded} \\
\hline
\multicolumn{2}{|>{\raggedright\arraybackslash}X|}{\hspace{0pt}\mytexttt{\color{param} (unicode src, unicode filter, unicode replace\_char)}} \\
\hline
\end{tabularx}
}

\par





Returns the source string with the replacement character substituted for all characters not included in the filter string. MORE: Should this be a general string substitution?






\par
\begin{description}
\item [\colorbox{tagtype}{\color{white} \textbf{\textsf{PARAMETER}}}] \textbf{\underline{replace\_char}} ||| UNICODE --- The character to be substituted into the result.
\item [\colorbox{tagtype}{\color{white} \textbf{\textsf{PARAMETER}}}] \textbf{\underline{src}} ||| UNICODE --- The string that is being tested.
\item [\colorbox{tagtype}{\color{white} \textbf{\textsf{PARAMETER}}}] \textbf{\underline{filter}} ||| UNICODE --- The string containing the set of characters to be included.
\end{description}







\par
\begin{description}
\item [\colorbox{tagtype}{\color{white} \textbf{\textsf{RETURN}}}] \textbf{UNICODE} --- 
\end{description}






\par
\begin{description}
\item [\colorbox{tagtype}{\color{white} \textbf{\textsf{SEE}}}] Std.Uni.SubstituteIncluded
\end{description}




\rule{\linewidth}{0.5pt}
\subsection*{\textsf{\colorbox{headtoc}{\color{white} FUNCTION}
Find}}

\hypertarget{ecldoc:uni.find}{}
\hspace{0pt} \hyperlink{ecldoc:Uni}{Uni} \textbackslash 

{\renewcommand{\arraystretch}{1.5}
\begin{tabularx}{\textwidth}{|>{\raggedright\arraybackslash}l|X|}
\hline
\hspace{0pt}\mytexttt{\color{red} UNSIGNED4} & \textbf{Find} \\
\hline
\multicolumn{2}{|>{\raggedright\arraybackslash}X|}{\hspace{0pt}\mytexttt{\color{param} (unicode src, unicode sought, unsigned4 instance)}} \\
\hline
\end{tabularx}
}

\par





Returns the character position of the nth match of the search string with the first string. If no match is found the attribute returns 0. If an instance is omitted the position of the first instance is returned.






\par
\begin{description}
\item [\colorbox{tagtype}{\color{white} \textbf{\textsf{PARAMETER}}}] \textbf{\underline{instance}} ||| UNSIGNED4 --- Which match instance are we interested in?
\item [\colorbox{tagtype}{\color{white} \textbf{\textsf{PARAMETER}}}] \textbf{\underline{src}} ||| UNICODE --- The string that is searched
\item [\colorbox{tagtype}{\color{white} \textbf{\textsf{PARAMETER}}}] \textbf{\underline{sought}} ||| UNICODE --- The string being sought.
\end{description}







\par
\begin{description}
\item [\colorbox{tagtype}{\color{white} \textbf{\textsf{RETURN}}}] \textbf{UNSIGNED4} --- 
\end{description}




\rule{\linewidth}{0.5pt}
\subsection*{\textsf{\colorbox{headtoc}{\color{white} FUNCTION}
FindWord}}

\hypertarget{ecldoc:uni.findword}{}
\hspace{0pt} \hyperlink{ecldoc:Uni}{Uni} \textbackslash 

{\renewcommand{\arraystretch}{1.5}
\begin{tabularx}{\textwidth}{|>{\raggedright\arraybackslash}l|X|}
\hline
\hspace{0pt}\mytexttt{\color{red} BOOLEAN} & \textbf{FindWord} \\
\hline
\multicolumn{2}{|>{\raggedright\arraybackslash}X|}{\hspace{0pt}\mytexttt{\color{param} (UNICODE src, UNICODE word, BOOLEAN ignore\_case=FALSE)}} \\
\hline
\end{tabularx}
}

\par





Tests if the search string contains the supplied word as a whole word.






\par
\begin{description}
\item [\colorbox{tagtype}{\color{white} \textbf{\textsf{PARAMETER}}}] \textbf{\underline{word}} ||| UNICODE --- The word to be searched for.
\item [\colorbox{tagtype}{\color{white} \textbf{\textsf{PARAMETER}}}] \textbf{\underline{src}} ||| UNICODE --- The string that is being tested.
\item [\colorbox{tagtype}{\color{white} \textbf{\textsf{PARAMETER}}}] \textbf{\underline{ignore\_case}} ||| BOOLEAN --- Whether to ignore differences in case between characters.
\end{description}







\par
\begin{description}
\item [\colorbox{tagtype}{\color{white} \textbf{\textsf{RETURN}}}] \textbf{BOOLEAN} --- 
\end{description}




\rule{\linewidth}{0.5pt}
\subsection*{\textsf{\colorbox{headtoc}{\color{white} FUNCTION}
LocaleFind}}

\hypertarget{ecldoc:uni.localefind}{}
\hspace{0pt} \hyperlink{ecldoc:Uni}{Uni} \textbackslash 

{\renewcommand{\arraystretch}{1.5}
\begin{tabularx}{\textwidth}{|>{\raggedright\arraybackslash}l|X|}
\hline
\hspace{0pt}\mytexttt{\color{red} UNSIGNED4} & \textbf{LocaleFind} \\
\hline
\multicolumn{2}{|>{\raggedright\arraybackslash}X|}{\hspace{0pt}\mytexttt{\color{param} (unicode src, unicode sought, unsigned4 instance, varstring locale\_name)}} \\
\hline
\end{tabularx}
}

\par





Returns the character position of the nth match of the search string with the first string. If no match is found the attribute returns 0. If an instance is omitted the position of the first instance is returned.






\par
\begin{description}
\item [\colorbox{tagtype}{\color{white} \textbf{\textsf{PARAMETER}}}] \textbf{\underline{instance}} ||| UNSIGNED4 --- Which match instance are we interested in?
\item [\colorbox{tagtype}{\color{white} \textbf{\textsf{PARAMETER}}}] \textbf{\underline{src}} ||| UNICODE --- The string that is searched
\item [\colorbox{tagtype}{\color{white} \textbf{\textsf{PARAMETER}}}] \textbf{\underline{sought}} ||| UNICODE --- The string being sought.
\item [\colorbox{tagtype}{\color{white} \textbf{\textsf{PARAMETER}}}] \textbf{\underline{locale\_name}} ||| VARSTRING --- The locale to use for the comparison
\end{description}







\par
\begin{description}
\item [\colorbox{tagtype}{\color{white} \textbf{\textsf{RETURN}}}] \textbf{UNSIGNED4} --- 
\end{description}




\rule{\linewidth}{0.5pt}
\subsection*{\textsf{\colorbox{headtoc}{\color{white} FUNCTION}
LocaleFindAtStrength}}

\hypertarget{ecldoc:uni.localefindatstrength}{}
\hspace{0pt} \hyperlink{ecldoc:Uni}{Uni} \textbackslash 

{\renewcommand{\arraystretch}{1.5}
\begin{tabularx}{\textwidth}{|>{\raggedright\arraybackslash}l|X|}
\hline
\hspace{0pt}\mytexttt{\color{red} UNSIGNED4} & \textbf{LocaleFindAtStrength} \\
\hline
\multicolumn{2}{|>{\raggedright\arraybackslash}X|}{\hspace{0pt}\mytexttt{\color{param} (unicode src, unicode tofind, unsigned4 instance, varstring locale\_name, integer1 strength)}} \\
\hline
\end{tabularx}
}

\par





Returns the character position of the nth match of the search string with the first string. If no match is found the attribute returns 0. If an instance is omitted the position of the first instance is returned.






\par
\begin{description}
\item [\colorbox{tagtype}{\color{white} \textbf{\textsf{PARAMETER}}}] \textbf{\underline{instance}} ||| UNSIGNED4 --- Which match instance are we interested in?
\item [\colorbox{tagtype}{\color{white} \textbf{\textsf{PARAMETER}}}] \textbf{\underline{strength}} ||| INTEGER1 --- The strength of the comparison 1 ignores accents and case, differentiating only between letters 2 ignores case but differentiates between accents. 3 differentiates between accents and case but ignores e.g. differences between Hiragana and Katakana 4 differentiates between accents and case and e.g. Hiragana/Katakana, but ignores e.g. Hebrew cantellation marks 5 differentiates between all strings whose canonically decomposed forms (NFDNormalization Form D) are non-identical
\item [\colorbox{tagtype}{\color{white} \textbf{\textsf{PARAMETER}}}] \textbf{\underline{src}} ||| UNICODE --- The string that is searched
\item [\colorbox{tagtype}{\color{white} \textbf{\textsf{PARAMETER}}}] \textbf{\underline{sought}} |||  --- The string being sought.
\item [\colorbox{tagtype}{\color{white} \textbf{\textsf{PARAMETER}}}] \textbf{\underline{locale\_name}} ||| VARSTRING --- The locale to use for the comparison
\item [\colorbox{tagtype}{\color{white} \textbf{\textsf{PARAMETER}}}] \textbf{\underline{tofind}} ||| UNICODE --- No Doc
\end{description}







\par
\begin{description}
\item [\colorbox{tagtype}{\color{white} \textbf{\textsf{RETURN}}}] \textbf{UNSIGNED4} --- 
\end{description}




\rule{\linewidth}{0.5pt}
\subsection*{\textsf{\colorbox{headtoc}{\color{white} FUNCTION}
Extract}}

\hypertarget{ecldoc:uni.extract}{}
\hspace{0pt} \hyperlink{ecldoc:Uni}{Uni} \textbackslash 

{\renewcommand{\arraystretch}{1.5}
\begin{tabularx}{\textwidth}{|>{\raggedright\arraybackslash}l|X|}
\hline
\hspace{0pt}\mytexttt{\color{red} unicode} & \textbf{Extract} \\
\hline
\multicolumn{2}{|>{\raggedright\arraybackslash}X|}{\hspace{0pt}\mytexttt{\color{param} (unicode src, unsigned4 instance)}} \\
\hline
\end{tabularx}
}

\par





Returns the nth element from a comma separated string.






\par
\begin{description}
\item [\colorbox{tagtype}{\color{white} \textbf{\textsf{PARAMETER}}}] \textbf{\underline{instance}} ||| UNSIGNED4 --- Which item to select from the list.
\item [\colorbox{tagtype}{\color{white} \textbf{\textsf{PARAMETER}}}] \textbf{\underline{src}} ||| UNICODE --- The string containing the comma separated list.
\end{description}







\par
\begin{description}
\item [\colorbox{tagtype}{\color{white} \textbf{\textsf{RETURN}}}] \textbf{UNICODE} --- 
\end{description}




\rule{\linewidth}{0.5pt}
\subsection*{\textsf{\colorbox{headtoc}{\color{white} FUNCTION}
ToLowerCase}}

\hypertarget{ecldoc:uni.tolowercase}{}
\hspace{0pt} \hyperlink{ecldoc:Uni}{Uni} \textbackslash 

{\renewcommand{\arraystretch}{1.5}
\begin{tabularx}{\textwidth}{|>{\raggedright\arraybackslash}l|X|}
\hline
\hspace{0pt}\mytexttt{\color{red} unicode} & \textbf{ToLowerCase} \\
\hline
\multicolumn{2}{|>{\raggedright\arraybackslash}X|}{\hspace{0pt}\mytexttt{\color{param} (unicode src)}} \\
\hline
\end{tabularx}
}

\par





Returns the argument string with all upper case characters converted to lower case.






\par
\begin{description}
\item [\colorbox{tagtype}{\color{white} \textbf{\textsf{PARAMETER}}}] \textbf{\underline{src}} ||| UNICODE --- The string that is being converted.
\end{description}







\par
\begin{description}
\item [\colorbox{tagtype}{\color{white} \textbf{\textsf{RETURN}}}] \textbf{UNICODE} --- 
\end{description}




\rule{\linewidth}{0.5pt}
\subsection*{\textsf{\colorbox{headtoc}{\color{white} FUNCTION}
ToUpperCase}}

\hypertarget{ecldoc:uni.touppercase}{}
\hspace{0pt} \hyperlink{ecldoc:Uni}{Uni} \textbackslash 

{\renewcommand{\arraystretch}{1.5}
\begin{tabularx}{\textwidth}{|>{\raggedright\arraybackslash}l|X|}
\hline
\hspace{0pt}\mytexttt{\color{red} unicode} & \textbf{ToUpperCase} \\
\hline
\multicolumn{2}{|>{\raggedright\arraybackslash}X|}{\hspace{0pt}\mytexttt{\color{param} (unicode src)}} \\
\hline
\end{tabularx}
}

\par





Return the argument string with all lower case characters converted to upper case.






\par
\begin{description}
\item [\colorbox{tagtype}{\color{white} \textbf{\textsf{PARAMETER}}}] \textbf{\underline{src}} ||| UNICODE --- The string that is being converted.
\end{description}







\par
\begin{description}
\item [\colorbox{tagtype}{\color{white} \textbf{\textsf{RETURN}}}] \textbf{UNICODE} --- 
\end{description}




\rule{\linewidth}{0.5pt}
\subsection*{\textsf{\colorbox{headtoc}{\color{white} FUNCTION}
ToTitleCase}}

\hypertarget{ecldoc:uni.totitlecase}{}
\hspace{0pt} \hyperlink{ecldoc:Uni}{Uni} \textbackslash 

{\renewcommand{\arraystretch}{1.5}
\begin{tabularx}{\textwidth}{|>{\raggedright\arraybackslash}l|X|}
\hline
\hspace{0pt}\mytexttt{\color{red} unicode} & \textbf{ToTitleCase} \\
\hline
\multicolumn{2}{|>{\raggedright\arraybackslash}X|}{\hspace{0pt}\mytexttt{\color{param} (unicode src)}} \\
\hline
\end{tabularx}
}

\par





Returns the upper case variant of the string using the rules for a particular locale.






\par
\begin{description}
\item [\colorbox{tagtype}{\color{white} \textbf{\textsf{PARAMETER}}}] \textbf{\underline{src}} ||| UNICODE --- The string that is being converted.
\item [\colorbox{tagtype}{\color{white} \textbf{\textsf{PARAMETER}}}] \textbf{\underline{locale\_name}} |||  --- The locale to use for the comparison
\end{description}







\par
\begin{description}
\item [\colorbox{tagtype}{\color{white} \textbf{\textsf{RETURN}}}] \textbf{UNICODE} --- 
\end{description}




\rule{\linewidth}{0.5pt}
\subsection*{\textsf{\colorbox{headtoc}{\color{white} FUNCTION}
LocaleToLowerCase}}

\hypertarget{ecldoc:uni.localetolowercase}{}
\hspace{0pt} \hyperlink{ecldoc:Uni}{Uni} \textbackslash 

{\renewcommand{\arraystretch}{1.5}
\begin{tabularx}{\textwidth}{|>{\raggedright\arraybackslash}l|X|}
\hline
\hspace{0pt}\mytexttt{\color{red} unicode} & \textbf{LocaleToLowerCase} \\
\hline
\multicolumn{2}{|>{\raggedright\arraybackslash}X|}{\hspace{0pt}\mytexttt{\color{param} (unicode src, varstring locale\_name)}} \\
\hline
\end{tabularx}
}

\par





Returns the lower case variant of the string using the rules for a particular locale.






\par
\begin{description}
\item [\colorbox{tagtype}{\color{white} \textbf{\textsf{PARAMETER}}}] \textbf{\underline{src}} ||| UNICODE --- The string that is being converted.
\item [\colorbox{tagtype}{\color{white} \textbf{\textsf{PARAMETER}}}] \textbf{\underline{locale\_name}} ||| VARSTRING --- The locale to use for the comparison
\end{description}







\par
\begin{description}
\item [\colorbox{tagtype}{\color{white} \textbf{\textsf{RETURN}}}] \textbf{UNICODE} --- 
\end{description}




\rule{\linewidth}{0.5pt}
\subsection*{\textsf{\colorbox{headtoc}{\color{white} FUNCTION}
LocaleToUpperCase}}

\hypertarget{ecldoc:uni.localetouppercase}{}
\hspace{0pt} \hyperlink{ecldoc:Uni}{Uni} \textbackslash 

{\renewcommand{\arraystretch}{1.5}
\begin{tabularx}{\textwidth}{|>{\raggedright\arraybackslash}l|X|}
\hline
\hspace{0pt}\mytexttt{\color{red} unicode} & \textbf{LocaleToUpperCase} \\
\hline
\multicolumn{2}{|>{\raggedright\arraybackslash}X|}{\hspace{0pt}\mytexttt{\color{param} (unicode src, varstring locale\_name)}} \\
\hline
\end{tabularx}
}

\par





Returns the upper case variant of the string using the rules for a particular locale.






\par
\begin{description}
\item [\colorbox{tagtype}{\color{white} \textbf{\textsf{PARAMETER}}}] \textbf{\underline{src}} ||| UNICODE --- The string that is being converted.
\item [\colorbox{tagtype}{\color{white} \textbf{\textsf{PARAMETER}}}] \textbf{\underline{locale\_name}} ||| VARSTRING --- The locale to use for the comparison
\end{description}







\par
\begin{description}
\item [\colorbox{tagtype}{\color{white} \textbf{\textsf{RETURN}}}] \textbf{UNICODE} --- 
\end{description}




\rule{\linewidth}{0.5pt}
\subsection*{\textsf{\colorbox{headtoc}{\color{white} FUNCTION}
LocaleToTitleCase}}

\hypertarget{ecldoc:uni.localetotitlecase}{}
\hspace{0pt} \hyperlink{ecldoc:Uni}{Uni} \textbackslash 

{\renewcommand{\arraystretch}{1.5}
\begin{tabularx}{\textwidth}{|>{\raggedright\arraybackslash}l|X|}
\hline
\hspace{0pt}\mytexttt{\color{red} unicode} & \textbf{LocaleToTitleCase} \\
\hline
\multicolumn{2}{|>{\raggedright\arraybackslash}X|}{\hspace{0pt}\mytexttt{\color{param} (unicode src, varstring locale\_name)}} \\
\hline
\end{tabularx}
}

\par





Returns the upper case variant of the string using the rules for a particular locale.






\par
\begin{description}
\item [\colorbox{tagtype}{\color{white} \textbf{\textsf{PARAMETER}}}] \textbf{\underline{src}} ||| UNICODE --- The string that is being converted.
\item [\colorbox{tagtype}{\color{white} \textbf{\textsf{PARAMETER}}}] \textbf{\underline{locale\_name}} ||| VARSTRING --- The locale to use for the comparison
\end{description}







\par
\begin{description}
\item [\colorbox{tagtype}{\color{white} \textbf{\textsf{RETURN}}}] \textbf{UNICODE} --- 
\end{description}




\rule{\linewidth}{0.5pt}
\subsection*{\textsf{\colorbox{headtoc}{\color{white} FUNCTION}
CompareIgnoreCase}}

\hypertarget{ecldoc:uni.compareignorecase}{}
\hspace{0pt} \hyperlink{ecldoc:Uni}{Uni} \textbackslash 

{\renewcommand{\arraystretch}{1.5}
\begin{tabularx}{\textwidth}{|>{\raggedright\arraybackslash}l|X|}
\hline
\hspace{0pt}\mytexttt{\color{red} integer4} & \textbf{CompareIgnoreCase} \\
\hline
\multicolumn{2}{|>{\raggedright\arraybackslash}X|}{\hspace{0pt}\mytexttt{\color{param} (unicode src1, unicode src2)}} \\
\hline
\end{tabularx}
}

\par





Compares the two strings case insensitively. Equivalent to comparing at strength 2.






\par
\begin{description}
\item [\colorbox{tagtype}{\color{white} \textbf{\textsf{PARAMETER}}}] \textbf{\underline{src2}} ||| UNICODE --- The second string to be compared.
\item [\colorbox{tagtype}{\color{white} \textbf{\textsf{PARAMETER}}}] \textbf{\underline{src1}} ||| UNICODE --- The first string to be compared.
\end{description}







\par
\begin{description}
\item [\colorbox{tagtype}{\color{white} \textbf{\textsf{RETURN}}}] \textbf{INTEGER4} --- 
\end{description}






\par
\begin{description}
\item [\colorbox{tagtype}{\color{white} \textbf{\textsf{SEE}}}] Std.Uni.CompareAtStrength
\end{description}




\rule{\linewidth}{0.5pt}
\subsection*{\textsf{\colorbox{headtoc}{\color{white} FUNCTION}
CompareAtStrength}}

\hypertarget{ecldoc:uni.compareatstrength}{}
\hspace{0pt} \hyperlink{ecldoc:Uni}{Uni} \textbackslash 

{\renewcommand{\arraystretch}{1.5}
\begin{tabularx}{\textwidth}{|>{\raggedright\arraybackslash}l|X|}
\hline
\hspace{0pt}\mytexttt{\color{red} integer4} & \textbf{CompareAtStrength} \\
\hline
\multicolumn{2}{|>{\raggedright\arraybackslash}X|}{\hspace{0pt}\mytexttt{\color{param} (unicode src1, unicode src2, integer1 strength)}} \\
\hline
\end{tabularx}
}

\par





Compares the two strings case insensitively. Equivalent to comparing at strength 2.






\par
\begin{description}
\item [\colorbox{tagtype}{\color{white} \textbf{\textsf{PARAMETER}}}] \textbf{\underline{src2}} ||| UNICODE --- The second string to be compared.
\item [\colorbox{tagtype}{\color{white} \textbf{\textsf{PARAMETER}}}] \textbf{\underline{strength}} ||| INTEGER1 --- The strength of the comparison 1 ignores accents and case, differentiating only between letters 2 ignores case but differentiates between accents. 3 differentiates between accents and case but ignores e.g. differences between Hiragana and Katakana 4 differentiates between accents and case and e.g. Hiragana/Katakana, but ignores e.g. Hebrew cantellation marks 5 differentiates between all strings whose canonically decomposed forms (NFDNormalization Form D) are non-identical
\item [\colorbox{tagtype}{\color{white} \textbf{\textsf{PARAMETER}}}] \textbf{\underline{src1}} ||| UNICODE --- The first string to be compared.
\end{description}







\par
\begin{description}
\item [\colorbox{tagtype}{\color{white} \textbf{\textsf{RETURN}}}] \textbf{INTEGER4} --- 
\end{description}






\par
\begin{description}
\item [\colorbox{tagtype}{\color{white} \textbf{\textsf{SEE}}}] Std.Uni.CompareAtStrength
\end{description}




\rule{\linewidth}{0.5pt}
\subsection*{\textsf{\colorbox{headtoc}{\color{white} FUNCTION}
LocaleCompareIgnoreCase}}

\hypertarget{ecldoc:uni.localecompareignorecase}{}
\hspace{0pt} \hyperlink{ecldoc:Uni}{Uni} \textbackslash 

{\renewcommand{\arraystretch}{1.5}
\begin{tabularx}{\textwidth}{|>{\raggedright\arraybackslash}l|X|}
\hline
\hspace{0pt}\mytexttt{\color{red} integer4} & \textbf{LocaleCompareIgnoreCase} \\
\hline
\multicolumn{2}{|>{\raggedright\arraybackslash}X|}{\hspace{0pt}\mytexttt{\color{param} (unicode src1, unicode src2, varstring locale\_name)}} \\
\hline
\end{tabularx}
}

\par





Compares the two strings case insensitively. Equivalent to comparing at strength 2.






\par
\begin{description}
\item [\colorbox{tagtype}{\color{white} \textbf{\textsf{PARAMETER}}}] \textbf{\underline{src2}} ||| UNICODE --- The second string to be compared.
\item [\colorbox{tagtype}{\color{white} \textbf{\textsf{PARAMETER}}}] \textbf{\underline{src1}} ||| UNICODE --- The first string to be compared.
\item [\colorbox{tagtype}{\color{white} \textbf{\textsf{PARAMETER}}}] \textbf{\underline{locale\_name}} ||| VARSTRING --- The locale to use for the comparison
\end{description}







\par
\begin{description}
\item [\colorbox{tagtype}{\color{white} \textbf{\textsf{RETURN}}}] \textbf{INTEGER4} --- 
\end{description}






\par
\begin{description}
\item [\colorbox{tagtype}{\color{white} \textbf{\textsf{SEE}}}] Std.Uni.CompareAtStrength
\end{description}




\rule{\linewidth}{0.5pt}
\subsection*{\textsf{\colorbox{headtoc}{\color{white} FUNCTION}
LocaleCompareAtStrength}}

\hypertarget{ecldoc:uni.localecompareatstrength}{}
\hspace{0pt} \hyperlink{ecldoc:Uni}{Uni} \textbackslash 

{\renewcommand{\arraystretch}{1.5}
\begin{tabularx}{\textwidth}{|>{\raggedright\arraybackslash}l|X|}
\hline
\hspace{0pt}\mytexttt{\color{red} integer4} & \textbf{LocaleCompareAtStrength} \\
\hline
\multicolumn{2}{|>{\raggedright\arraybackslash}X|}{\hspace{0pt}\mytexttt{\color{param} (unicode src1, unicode src2, varstring locale\_name, integer1 strength)}} \\
\hline
\end{tabularx}
}

\par





Compares the two strings case insensitively. Equivalent to comparing at strength 2.






\par
\begin{description}
\item [\colorbox{tagtype}{\color{white} \textbf{\textsf{PARAMETER}}}] \textbf{\underline{src2}} ||| UNICODE --- The second string to be compared.
\item [\colorbox{tagtype}{\color{white} \textbf{\textsf{PARAMETER}}}] \textbf{\underline{strength}} ||| INTEGER1 --- The strength of the comparison 1 ignores accents and case, differentiating only between letters 2 ignores case but differentiates between accents. 3 differentiates between accents and case but ignores e.g. differences between Hiragana and Katakana 4 differentiates between accents and case and e.g. Hiragana/Katakana, but ignores e.g. Hebrew cantellation marks 5 differentiates between all strings whose canonically decomposed forms (NFDNormalization Form D) are non-identical
\item [\colorbox{tagtype}{\color{white} \textbf{\textsf{PARAMETER}}}] \textbf{\underline{src1}} ||| UNICODE --- The first string to be compared.
\item [\colorbox{tagtype}{\color{white} \textbf{\textsf{PARAMETER}}}] \textbf{\underline{locale\_name}} ||| VARSTRING --- The locale to use for the comparison
\end{description}







\par
\begin{description}
\item [\colorbox{tagtype}{\color{white} \textbf{\textsf{RETURN}}}] \textbf{INTEGER4} --- 
\end{description}




\rule{\linewidth}{0.5pt}
\subsection*{\textsf{\colorbox{headtoc}{\color{white} FUNCTION}
Reverse}}

\hypertarget{ecldoc:uni.reverse}{}
\hspace{0pt} \hyperlink{ecldoc:Uni}{Uni} \textbackslash 

{\renewcommand{\arraystretch}{1.5}
\begin{tabularx}{\textwidth}{|>{\raggedright\arraybackslash}l|X|}
\hline
\hspace{0pt}\mytexttt{\color{red} unicode} & \textbf{Reverse} \\
\hline
\multicolumn{2}{|>{\raggedright\arraybackslash}X|}{\hspace{0pt}\mytexttt{\color{param} (unicode src)}} \\
\hline
\end{tabularx}
}

\par





Returns the argument string with all characters in reverse order. Note the argument is not TRIMMED before it is reversed.






\par
\begin{description}
\item [\colorbox{tagtype}{\color{white} \textbf{\textsf{PARAMETER}}}] \textbf{\underline{src}} ||| UNICODE --- The string that is being reversed.
\end{description}







\par
\begin{description}
\item [\colorbox{tagtype}{\color{white} \textbf{\textsf{RETURN}}}] \textbf{UNICODE} --- 
\end{description}




\rule{\linewidth}{0.5pt}
\subsection*{\textsf{\colorbox{headtoc}{\color{white} FUNCTION}
FindReplace}}

\hypertarget{ecldoc:uni.findreplace}{}
\hspace{0pt} \hyperlink{ecldoc:Uni}{Uni} \textbackslash 

{\renewcommand{\arraystretch}{1.5}
\begin{tabularx}{\textwidth}{|>{\raggedright\arraybackslash}l|X|}
\hline
\hspace{0pt}\mytexttt{\color{red} unicode} & \textbf{FindReplace} \\
\hline
\multicolumn{2}{|>{\raggedright\arraybackslash}X|}{\hspace{0pt}\mytexttt{\color{param} (unicode src, unicode sought, unicode replacement)}} \\
\hline
\end{tabularx}
}

\par





Returns the source string with the replacement string substituted for all instances of the search string.






\par
\begin{description}
\item [\colorbox{tagtype}{\color{white} \textbf{\textsf{PARAMETER}}}] \textbf{\underline{src}} ||| UNICODE --- The string that is being transformed.
\item [\colorbox{tagtype}{\color{white} \textbf{\textsf{PARAMETER}}}] \textbf{\underline{replacement}} ||| UNICODE --- The string to be substituted into the result.
\item [\colorbox{tagtype}{\color{white} \textbf{\textsf{PARAMETER}}}] \textbf{\underline{sought}} ||| UNICODE --- The string to be replaced.
\end{description}







\par
\begin{description}
\item [\colorbox{tagtype}{\color{white} \textbf{\textsf{RETURN}}}] \textbf{UNICODE} --- 
\end{description}




\rule{\linewidth}{0.5pt}
\subsection*{\textsf{\colorbox{headtoc}{\color{white} FUNCTION}
LocaleFindReplace}}

\hypertarget{ecldoc:uni.localefindreplace}{}
\hspace{0pt} \hyperlink{ecldoc:Uni}{Uni} \textbackslash 

{\renewcommand{\arraystretch}{1.5}
\begin{tabularx}{\textwidth}{|>{\raggedright\arraybackslash}l|X|}
\hline
\hspace{0pt}\mytexttt{\color{red} unicode} & \textbf{LocaleFindReplace} \\
\hline
\multicolumn{2}{|>{\raggedright\arraybackslash}X|}{\hspace{0pt}\mytexttt{\color{param} (unicode src, unicode sought, unicode replacement, varstring locale\_name)}} \\
\hline
\end{tabularx}
}

\par





Returns the source string with the replacement string substituted for all instances of the search string.






\par
\begin{description}
\item [\colorbox{tagtype}{\color{white} \textbf{\textsf{PARAMETER}}}] \textbf{\underline{src}} ||| UNICODE --- The string that is being transformed.
\item [\colorbox{tagtype}{\color{white} \textbf{\textsf{PARAMETER}}}] \textbf{\underline{replacement}} ||| UNICODE --- The string to be substituted into the result.
\item [\colorbox{tagtype}{\color{white} \textbf{\textsf{PARAMETER}}}] \textbf{\underline{sought}} ||| UNICODE --- The string to be replaced.
\item [\colorbox{tagtype}{\color{white} \textbf{\textsf{PARAMETER}}}] \textbf{\underline{locale\_name}} ||| VARSTRING --- The locale to use for the comparison
\end{description}







\par
\begin{description}
\item [\colorbox{tagtype}{\color{white} \textbf{\textsf{RETURN}}}] \textbf{UNICODE} --- 
\end{description}




\rule{\linewidth}{0.5pt}
\subsection*{\textsf{\colorbox{headtoc}{\color{white} FUNCTION}
LocaleFindAtStrengthReplace}}

\hypertarget{ecldoc:uni.localefindatstrengthreplace}{}
\hspace{0pt} \hyperlink{ecldoc:Uni}{Uni} \textbackslash 

{\renewcommand{\arraystretch}{1.5}
\begin{tabularx}{\textwidth}{|>{\raggedright\arraybackslash}l|X|}
\hline
\hspace{0pt}\mytexttt{\color{red} unicode} & \textbf{LocaleFindAtStrengthReplace} \\
\hline
\multicolumn{2}{|>{\raggedright\arraybackslash}X|}{\hspace{0pt}\mytexttt{\color{param} (unicode src, unicode sought, unicode replacement, varstring locale\_name, integer1 strength)}} \\
\hline
\end{tabularx}
}

\par





Returns the source string with the replacement string substituted for all instances of the search string.






\par
\begin{description}
\item [\colorbox{tagtype}{\color{white} \textbf{\textsf{PARAMETER}}}] \textbf{\underline{strength}} ||| INTEGER1 --- The strength of the comparison
\item [\colorbox{tagtype}{\color{white} \textbf{\textsf{PARAMETER}}}] \textbf{\underline{src}} ||| UNICODE --- The string that is being transformed.
\item [\colorbox{tagtype}{\color{white} \textbf{\textsf{PARAMETER}}}] \textbf{\underline{replacement}} ||| UNICODE --- The string to be substituted into the result.
\item [\colorbox{tagtype}{\color{white} \textbf{\textsf{PARAMETER}}}] \textbf{\underline{sought}} ||| UNICODE --- The string to be replaced.
\item [\colorbox{tagtype}{\color{white} \textbf{\textsf{PARAMETER}}}] \textbf{\underline{locale\_name}} ||| VARSTRING --- The locale to use for the comparison
\end{description}







\par
\begin{description}
\item [\colorbox{tagtype}{\color{white} \textbf{\textsf{RETURN}}}] \textbf{UNICODE} --- 
\end{description}




\rule{\linewidth}{0.5pt}
\subsection*{\textsf{\colorbox{headtoc}{\color{white} FUNCTION}
CleanAccents}}

\hypertarget{ecldoc:uni.cleanaccents}{}
\hspace{0pt} \hyperlink{ecldoc:Uni}{Uni} \textbackslash 

{\renewcommand{\arraystretch}{1.5}
\begin{tabularx}{\textwidth}{|>{\raggedright\arraybackslash}l|X|}
\hline
\hspace{0pt}\mytexttt{\color{red} unicode} & \textbf{CleanAccents} \\
\hline
\multicolumn{2}{|>{\raggedright\arraybackslash}X|}{\hspace{0pt}\mytexttt{\color{param} (unicode src)}} \\
\hline
\end{tabularx}
}

\par





Returns the source string with all accented characters replaced with unaccented.






\par
\begin{description}
\item [\colorbox{tagtype}{\color{white} \textbf{\textsf{PARAMETER}}}] \textbf{\underline{src}} ||| UNICODE --- The string that is being transformed.
\end{description}







\par
\begin{description}
\item [\colorbox{tagtype}{\color{white} \textbf{\textsf{RETURN}}}] \textbf{UNICODE} --- 
\end{description}




\rule{\linewidth}{0.5pt}
\subsection*{\textsf{\colorbox{headtoc}{\color{white} FUNCTION}
CleanSpaces}}

\hypertarget{ecldoc:uni.cleanspaces}{}
\hspace{0pt} \hyperlink{ecldoc:Uni}{Uni} \textbackslash 

{\renewcommand{\arraystretch}{1.5}
\begin{tabularx}{\textwidth}{|>{\raggedright\arraybackslash}l|X|}
\hline
\hspace{0pt}\mytexttt{\color{red} unicode} & \textbf{CleanSpaces} \\
\hline
\multicolumn{2}{|>{\raggedright\arraybackslash}X|}{\hspace{0pt}\mytexttt{\color{param} (unicode src)}} \\
\hline
\end{tabularx}
}

\par





Returns the source string with all instances of multiple adjacent space characters (2 or more spaces together) reduced to a single space character. Leading and trailing spaces are removed, and tab characters are converted to spaces.






\par
\begin{description}
\item [\colorbox{tagtype}{\color{white} \textbf{\textsf{PARAMETER}}}] \textbf{\underline{src}} ||| UNICODE --- The string to be cleaned.
\end{description}







\par
\begin{description}
\item [\colorbox{tagtype}{\color{white} \textbf{\textsf{RETURN}}}] \textbf{UNICODE} --- 
\end{description}




\rule{\linewidth}{0.5pt}
\subsection*{\textsf{\colorbox{headtoc}{\color{white} FUNCTION}
WildMatch}}

\hypertarget{ecldoc:uni.wildmatch}{}
\hspace{0pt} \hyperlink{ecldoc:Uni}{Uni} \textbackslash 

{\renewcommand{\arraystretch}{1.5}
\begin{tabularx}{\textwidth}{|>{\raggedright\arraybackslash}l|X|}
\hline
\hspace{0pt}\mytexttt{\color{red} boolean} & \textbf{WildMatch} \\
\hline
\multicolumn{2}{|>{\raggedright\arraybackslash}X|}{\hspace{0pt}\mytexttt{\color{param} (unicode src, unicode \_pattern, boolean \_noCase)}} \\
\hline
\end{tabularx}
}

\par





Tests if the search string matches the pattern. The pattern can contain wildcards '?' (single character) and '*' (multiple character).






\par
\begin{description}
\item [\colorbox{tagtype}{\color{white} \textbf{\textsf{PARAMETER}}}] \textbf{\underline{pattern}} |||  --- The pattern to match against.
\item [\colorbox{tagtype}{\color{white} \textbf{\textsf{PARAMETER}}}] \textbf{\underline{src}} ||| UNICODE --- The string that is being tested.
\item [\colorbox{tagtype}{\color{white} \textbf{\textsf{PARAMETER}}}] \textbf{\underline{ignore\_case}} |||  --- Whether to ignore differences in case between characters
\item [\colorbox{tagtype}{\color{white} \textbf{\textsf{PARAMETER}}}] \textbf{\underline{\_nocase}} ||| BOOLEAN --- No Doc
\item [\colorbox{tagtype}{\color{white} \textbf{\textsf{PARAMETER}}}] \textbf{\underline{\_pattern}} ||| UNICODE --- No Doc
\end{description}







\par
\begin{description}
\item [\colorbox{tagtype}{\color{white} \textbf{\textsf{RETURN}}}] \textbf{BOOLEAN} --- 
\end{description}




\rule{\linewidth}{0.5pt}
\subsection*{\textsf{\colorbox{headtoc}{\color{white} FUNCTION}
Contains}}

\hypertarget{ecldoc:uni.contains}{}
\hspace{0pt} \hyperlink{ecldoc:Uni}{Uni} \textbackslash 

{\renewcommand{\arraystretch}{1.5}
\begin{tabularx}{\textwidth}{|>{\raggedright\arraybackslash}l|X|}
\hline
\hspace{0pt}\mytexttt{\color{red} BOOLEAN} & \textbf{Contains} \\
\hline
\multicolumn{2}{|>{\raggedright\arraybackslash}X|}{\hspace{0pt}\mytexttt{\color{param} (unicode src, unicode \_pattern, boolean \_noCase)}} \\
\hline
\end{tabularx}
}

\par





Tests if the search string contains each of the characters in the pattern. If the pattern contains duplicate characters those characters will match once for each occurence in the pattern.






\par
\begin{description}
\item [\colorbox{tagtype}{\color{white} \textbf{\textsf{PARAMETER}}}] \textbf{\underline{pattern}} |||  --- The pattern to match against.
\item [\colorbox{tagtype}{\color{white} \textbf{\textsf{PARAMETER}}}] \textbf{\underline{src}} ||| UNICODE --- The string that is being tested.
\item [\colorbox{tagtype}{\color{white} \textbf{\textsf{PARAMETER}}}] \textbf{\underline{ignore\_case}} |||  --- Whether to ignore differences in case between characters
\item [\colorbox{tagtype}{\color{white} \textbf{\textsf{PARAMETER}}}] \textbf{\underline{\_nocase}} ||| BOOLEAN --- No Doc
\item [\colorbox{tagtype}{\color{white} \textbf{\textsf{PARAMETER}}}] \textbf{\underline{\_pattern}} ||| UNICODE --- No Doc
\end{description}







\par
\begin{description}
\item [\colorbox{tagtype}{\color{white} \textbf{\textsf{RETURN}}}] \textbf{BOOLEAN} --- 
\end{description}




\rule{\linewidth}{0.5pt}
\subsection*{\textsf{\colorbox{headtoc}{\color{white} FUNCTION}
EditDistance}}

\hypertarget{ecldoc:uni.editdistance}{}
\hspace{0pt} \hyperlink{ecldoc:Uni}{Uni} \textbackslash 

{\renewcommand{\arraystretch}{1.5}
\begin{tabularx}{\textwidth}{|>{\raggedright\arraybackslash}l|X|}
\hline
\hspace{0pt}\mytexttt{\color{red} UNSIGNED4} & \textbf{EditDistance} \\
\hline
\multicolumn{2}{|>{\raggedright\arraybackslash}X|}{\hspace{0pt}\mytexttt{\color{param} (unicode \_left, unicode \_right, varstring localename = '')}} \\
\hline
\end{tabularx}
}

\par





Returns the minimum edit distance between the two strings. An insert change or delete counts as a single edit. The two strings are trimmed before comparing.






\par
\begin{description}
\item [\colorbox{tagtype}{\color{white} \textbf{\textsf{PARAMETER}}}] \textbf{\underline{\_left}} ||| UNICODE --- The first string to be compared.
\item [\colorbox{tagtype}{\color{white} \textbf{\textsf{PARAMETER}}}] \textbf{\underline{localname}} |||  --- The locale to use for the comparison. Defaults to ''.
\item [\colorbox{tagtype}{\color{white} \textbf{\textsf{PARAMETER}}}] \textbf{\underline{\_right}} ||| UNICODE --- The second string to be compared.
\item [\colorbox{tagtype}{\color{white} \textbf{\textsf{PARAMETER}}}] \textbf{\underline{localename}} ||| VARSTRING --- No Doc
\end{description}







\par
\begin{description}
\item [\colorbox{tagtype}{\color{white} \textbf{\textsf{RETURN}}}] \textbf{UNSIGNED4} --- The minimum edit distance between the two strings.
\end{description}




\rule{\linewidth}{0.5pt}
\subsection*{\textsf{\colorbox{headtoc}{\color{white} FUNCTION}
EditDistanceWithinRadius}}

\hypertarget{ecldoc:uni.editdistancewithinradius}{}
\hspace{0pt} \hyperlink{ecldoc:Uni}{Uni} \textbackslash 

{\renewcommand{\arraystretch}{1.5}
\begin{tabularx}{\textwidth}{|>{\raggedright\arraybackslash}l|X|}
\hline
\hspace{0pt}\mytexttt{\color{red} BOOLEAN} & \textbf{EditDistanceWithinRadius} \\
\hline
\multicolumn{2}{|>{\raggedright\arraybackslash}X|}{\hspace{0pt}\mytexttt{\color{param} (unicode \_left, unicode \_right, unsigned4 radius, varstring localename = '')}} \\
\hline
\end{tabularx}
}

\par





Returns true if the minimum edit distance between the two strings is with a specific range. The two strings are trimmed before comparing.






\par
\begin{description}
\item [\colorbox{tagtype}{\color{white} \textbf{\textsf{PARAMETER}}}] \textbf{\underline{\_left}} ||| UNICODE --- The first string to be compared.
\item [\colorbox{tagtype}{\color{white} \textbf{\textsf{PARAMETER}}}] \textbf{\underline{localname}} |||  --- The locale to use for the comparison. Defaults to ''.
\item [\colorbox{tagtype}{\color{white} \textbf{\textsf{PARAMETER}}}] \textbf{\underline{\_right}} ||| UNICODE --- The second string to be compared.
\item [\colorbox{tagtype}{\color{white} \textbf{\textsf{PARAMETER}}}] \textbf{\underline{radius}} ||| UNSIGNED4 --- The maximum edit distance that is accepable.
\item [\colorbox{tagtype}{\color{white} \textbf{\textsf{PARAMETER}}}] \textbf{\underline{localename}} ||| VARSTRING --- No Doc
\end{description}







\par
\begin{description}
\item [\colorbox{tagtype}{\color{white} \textbf{\textsf{RETURN}}}] \textbf{BOOLEAN} --- Whether or not the two strings are within the given specified edit distance.
\end{description}




\rule{\linewidth}{0.5pt}
\subsection*{\textsf{\colorbox{headtoc}{\color{white} FUNCTION}
WordCount}}

\hypertarget{ecldoc:uni.wordcount}{}
\hspace{0pt} \hyperlink{ecldoc:Uni}{Uni} \textbackslash 

{\renewcommand{\arraystretch}{1.5}
\begin{tabularx}{\textwidth}{|>{\raggedright\arraybackslash}l|X|}
\hline
\hspace{0pt}\mytexttt{\color{red} unsigned4} & \textbf{WordCount} \\
\hline
\multicolumn{2}{|>{\raggedright\arraybackslash}X|}{\hspace{0pt}\mytexttt{\color{param} (unicode text, varstring localename = '')}} \\
\hline
\end{tabularx}
}

\par





Returns the number of words in the string. Word boundaries are marked by the unicode break semantics.






\par
\begin{description}
\item [\colorbox{tagtype}{\color{white} \textbf{\textsf{PARAMETER}}}] \textbf{\underline{localname}} |||  --- The locale to use for the break semantics. Defaults to ''.
\item [\colorbox{tagtype}{\color{white} \textbf{\textsf{PARAMETER}}}] \textbf{\underline{text}} ||| UNICODE --- The string to be broken into words.
\item [\colorbox{tagtype}{\color{white} \textbf{\textsf{PARAMETER}}}] \textbf{\underline{localename}} ||| VARSTRING --- No Doc
\end{description}







\par
\begin{description}
\item [\colorbox{tagtype}{\color{white} \textbf{\textsf{RETURN}}}] \textbf{UNSIGNED4} --- The number of words in the string.
\end{description}




\rule{\linewidth}{0.5pt}
\subsection*{\textsf{\colorbox{headtoc}{\color{white} FUNCTION}
GetNthWord}}

\hypertarget{ecldoc:uni.getnthword}{}
\hspace{0pt} \hyperlink{ecldoc:Uni}{Uni} \textbackslash 

{\renewcommand{\arraystretch}{1.5}
\begin{tabularx}{\textwidth}{|>{\raggedright\arraybackslash}l|X|}
\hline
\hspace{0pt}\mytexttt{\color{red} unicode} & \textbf{GetNthWord} \\
\hline
\multicolumn{2}{|>{\raggedright\arraybackslash}X|}{\hspace{0pt}\mytexttt{\color{param} (unicode text, unsigned4 n, varstring localename = '')}} \\
\hline
\end{tabularx}
}

\par





Returns the n-th word from the string. Word boundaries are marked by the unicode break semantics.






\par
\begin{description}
\item [\colorbox{tagtype}{\color{white} \textbf{\textsf{PARAMETER}}}] \textbf{\underline{localname}} |||  --- The locale to use for the break semantics. Defaults to ''.
\item [\colorbox{tagtype}{\color{white} \textbf{\textsf{PARAMETER}}}] \textbf{\underline{n}} ||| UNSIGNED4 --- Which word should be returned from the function.
\item [\colorbox{tagtype}{\color{white} \textbf{\textsf{PARAMETER}}}] \textbf{\underline{text}} ||| UNICODE --- The string to be broken into words.
\item [\colorbox{tagtype}{\color{white} \textbf{\textsf{PARAMETER}}}] \textbf{\underline{localename}} ||| VARSTRING --- No Doc
\end{description}







\par
\begin{description}
\item [\colorbox{tagtype}{\color{white} \textbf{\textsf{RETURN}}}] \textbf{UNICODE} --- The number of words in the string.
\end{description}




\rule{\linewidth}{0.5pt}



\chapter*{\color{headtoc} root}
\hypertarget{ecldoc:toc:root}{}
\hyperlink{ecldoc:toc:}{Go Up}


\section*{Table of Contents}
{\renewcommand{\arraystretch}{1.5}
\begin{longtable}{|p{\textwidth}|}
\hline
\hyperlink{ecldoc:toc:BLAS}{BLAS.ecl} \\
\hline
\hyperlink{ecldoc:toc:BundleBase}{BundleBase.ecl} \\
\hline
\hyperlink{ecldoc:toc:Date}{Date.ecl} \\
\hline
\hyperlink{ecldoc:toc:File}{File.ecl} \\
\hline
\hyperlink{ecldoc:toc:math}{math.ecl} \\
\hline
\hyperlink{ecldoc:toc:Metaphone}{Metaphone.ecl} \\
\hline
\hyperlink{ecldoc:toc:str}{str.ecl} \\
\hline
\hyperlink{ecldoc:toc:Uni}{Uni.ecl} \\
\hline
\hyperlink{ecldoc:toc:root/system}{system} \\
\hline
\end{longtable}
}

\chapter*{\color{headfile}
BLAS
}
\hypertarget{ecldoc:toc:BLAS}{}
\hyperlink{ecldoc:toc:root}{Go Up}

\section*{\underline{\textsf{IMPORTS}}}
\begin{doublespace}
{\large
lib\_eclblas |
}
\end{doublespace}

\section*{\underline{\textsf{DESCRIPTIONS}}}
\subsection*{\textsf{\colorbox{headtoc}{\color{white} MODULE}
BLAS}}

\hypertarget{ecldoc:blas}{}

{\renewcommand{\arraystretch}{1.5}
\begin{tabularx}{\textwidth}{|>{\raggedright\arraybackslash}l|X|}
\hline
\hspace{0pt}\mytexttt{\color{red} } & \textbf{BLAS} \\
\hline
\end{tabularx}
}

\par


\textbf{Children}
\begin{enumerate}
\item \hyperlink{ecldoc:BLAS.Types}{Types}
\item \hyperlink{ecldoc:blas.icellfunc}{ICellFunc}
: Function prototype for Apply2Cell
\item \hyperlink{ecldoc:blas.apply2cells}{Apply2Cells}
: Iterate matrix and apply function to each cell
\item \hyperlink{ecldoc:blas.dasum}{dasum}
: Absolute sum, the 1 norm of a vector
\item \hyperlink{ecldoc:blas.daxpy}{daxpy}
: alpha*X + Y
\item \hyperlink{ecldoc:blas.dgemm}{dgemm}
: alpha*op(A) op(B) + beta*C where op() is transpose
\item \hyperlink{ecldoc:blas.dgetf2}{dgetf2}
: Compute LU Factorization of matrix A
\item \hyperlink{ecldoc:blas.dpotf2}{dpotf2}
: DPOTF2 computes the Cholesky factorization of a real symmetric positive definite matrix A
\item \hyperlink{ecldoc:blas.dscal}{dscal}
: Scale a vector alpha
\item \hyperlink{ecldoc:blas.dsyrk}{dsyrk}
: Implements symmetric rank update C
\item \hyperlink{ecldoc:blas.dtrsm}{dtrsm}
: Triangular matrix solver
\item \hyperlink{ecldoc:blas.extract_diag}{extract\_diag}
: Extract the diagonal of he matrix
\item \hyperlink{ecldoc:blas.extract_tri}{extract\_tri}
: Extract the upper or lower triangle
\item \hyperlink{ecldoc:blas.make_diag}{make\_diag}
: Generate a diagonal matrix
\item \hyperlink{ecldoc:blas.make_vector}{make\_vector}
: Make a vector of dimension m
\item \hyperlink{ecldoc:blas.trace}{trace}
: The trace of the input matrix
\end{enumerate}

\rule{\linewidth}{0.5pt}

\subsection*{\textsf{\colorbox{headtoc}{\color{white} MODULE}
Types}}

\hypertarget{ecldoc:BLAS.Types}{}
\hspace{0pt} \hyperlink{ecldoc:blas}{BLAS} \textbackslash 

{\renewcommand{\arraystretch}{1.5}
\begin{tabularx}{\textwidth}{|>{\raggedright\arraybackslash}l|X|}
\hline
\hspace{0pt}\mytexttt{\color{red} } & \textbf{Types} \\
\hline
\end{tabularx}
}

\par


\textbf{Children}
\begin{enumerate}
\item \hyperlink{ecldoc:blas.types.value_t}{value\_t}
\item \hyperlink{ecldoc:blas.types.dimension_t}{dimension\_t}
\item \hyperlink{ecldoc:blas.types.matrix_t}{matrix\_t}
\item \hyperlink{ecldoc:ecldoc-Triangle}{Triangle}
\item \hyperlink{ecldoc:ecldoc-Diagonal}{Diagonal}
\item \hyperlink{ecldoc:ecldoc-Side}{Side}
\end{enumerate}

\rule{\linewidth}{0.5pt}

\subsection*{\textsf{\colorbox{headtoc}{\color{white} ATTRIBUTE}
value\_t}}

\hypertarget{ecldoc:blas.types.value_t}{}
\hspace{0pt} \hyperlink{ecldoc:blas}{BLAS} \textbackslash 
\hspace{0pt} \hyperlink{ecldoc:BLAS.Types}{Types} \textbackslash 

{\renewcommand{\arraystretch}{1.5}
\begin{tabularx}{\textwidth}{|>{\raggedright\arraybackslash}l|X|}
\hline
\hspace{0pt}\mytexttt{\color{red} } & \textbf{value\_t} \\
\hline
\end{tabularx}
}

\par


\rule{\linewidth}{0.5pt}
\subsection*{\textsf{\colorbox{headtoc}{\color{white} ATTRIBUTE}
dimension\_t}}

\hypertarget{ecldoc:blas.types.dimension_t}{}
\hspace{0pt} \hyperlink{ecldoc:blas}{BLAS} \textbackslash 
\hspace{0pt} \hyperlink{ecldoc:BLAS.Types}{Types} \textbackslash 

{\renewcommand{\arraystretch}{1.5}
\begin{tabularx}{\textwidth}{|>{\raggedright\arraybackslash}l|X|}
\hline
\hspace{0pt}\mytexttt{\color{red} } & \textbf{dimension\_t} \\
\hline
\end{tabularx}
}

\par


\rule{\linewidth}{0.5pt}
\subsection*{\textsf{\colorbox{headtoc}{\color{white} ATTRIBUTE}
matrix\_t}}

\hypertarget{ecldoc:blas.types.matrix_t}{}
\hspace{0pt} \hyperlink{ecldoc:blas}{BLAS} \textbackslash 
\hspace{0pt} \hyperlink{ecldoc:BLAS.Types}{Types} \textbackslash 

{\renewcommand{\arraystretch}{1.5}
\begin{tabularx}{\textwidth}{|>{\raggedright\arraybackslash}l|X|}
\hline
\hspace{0pt}\mytexttt{\color{red} } & \textbf{matrix\_t} \\
\hline
\end{tabularx}
}

\par


\rule{\linewidth}{0.5pt}
\subsection*{\textsf{\colorbox{headtoc}{\color{white} ATTRIBUTE}
Triangle}}

\hypertarget{ecldoc:ecldoc-Triangle}{}
\hspace{0pt} \hyperlink{ecldoc:blas}{BLAS} \textbackslash 
\hspace{0pt} \hyperlink{ecldoc:BLAS.Types}{Types} \textbackslash 

{\renewcommand{\arraystretch}{1.5}
\begin{tabularx}{\textwidth}{|>{\raggedright\arraybackslash}l|X|}
\hline
\hspace{0pt}\mytexttt{\color{red} } & \textbf{Triangle} \\
\hline
\end{tabularx}
}

\par


\rule{\linewidth}{0.5pt}
\subsection*{\textsf{\colorbox{headtoc}{\color{white} ATTRIBUTE}
Diagonal}}

\hypertarget{ecldoc:ecldoc-Diagonal}{}
\hspace{0pt} \hyperlink{ecldoc:blas}{BLAS} \textbackslash 
\hspace{0pt} \hyperlink{ecldoc:BLAS.Types}{Types} \textbackslash 

{\renewcommand{\arraystretch}{1.5}
\begin{tabularx}{\textwidth}{|>{\raggedright\arraybackslash}l|X|}
\hline
\hspace{0pt}\mytexttt{\color{red} } & \textbf{Diagonal} \\
\hline
\end{tabularx}
}

\par


\rule{\linewidth}{0.5pt}
\subsection*{\textsf{\colorbox{headtoc}{\color{white} ATTRIBUTE}
Side}}

\hypertarget{ecldoc:ecldoc-Side}{}
\hspace{0pt} \hyperlink{ecldoc:blas}{BLAS} \textbackslash 
\hspace{0pt} \hyperlink{ecldoc:BLAS.Types}{Types} \textbackslash 

{\renewcommand{\arraystretch}{1.5}
\begin{tabularx}{\textwidth}{|>{\raggedright\arraybackslash}l|X|}
\hline
\hspace{0pt}\mytexttt{\color{red} } & \textbf{Side} \\
\hline
\end{tabularx}
}

\par


\rule{\linewidth}{0.5pt}


\subsection*{\textsf{\colorbox{headtoc}{\color{white} FUNCTION}
ICellFunc}}

\hypertarget{ecldoc:blas.icellfunc}{}
\hspace{0pt} \hyperlink{ecldoc:blas}{BLAS} \textbackslash 

{\renewcommand{\arraystretch}{1.5}
\begin{tabularx}{\textwidth}{|>{\raggedright\arraybackslash}l|X|}
\hline
\hspace{0pt}\mytexttt{\color{red} Types.value\_t} & \textbf{ICellFunc} \\
\hline
\multicolumn{2}{|>{\raggedright\arraybackslash}X|}{\hspace{0pt}\mytexttt{\color{param} (Types.value\_t v, Types.dimension\_t r, Types.dimension\_t c)}} \\
\hline
\end{tabularx}
}

\par
Function prototype for Apply2Cell.

\par
\begin{description}
\item [\colorbox{tagtype}{\color{white} \textbf{\textsf{PARAMETER}}}] \textbf{\underline{v}} the value
\item [\colorbox{tagtype}{\color{white} \textbf{\textsf{PARAMETER}}}] \textbf{\underline{r}} the row ordinal
\item [\colorbox{tagtype}{\color{white} \textbf{\textsf{PARAMETER}}}] \textbf{\underline{c}} the column ordinal
\item [\colorbox{tagtype}{\color{white} \textbf{\textsf{RETURN}}}] \textbf{\underline{}} the updated value
\end{description}

\rule{\linewidth}{0.5pt}
\subsection*{\textsf{\colorbox{headtoc}{\color{white} FUNCTION}
Apply2Cells}}

\hypertarget{ecldoc:blas.apply2cells}{}
\hspace{0pt} \hyperlink{ecldoc:blas}{BLAS} \textbackslash 

{\renewcommand{\arraystretch}{1.5}
\begin{tabularx}{\textwidth}{|>{\raggedright\arraybackslash}l|X|}
\hline
\hspace{0pt}\mytexttt{\color{red} Types.matrix\_t} & \textbf{Apply2Cells} \\
\hline
\multicolumn{2}{|>{\raggedright\arraybackslash}X|}{\hspace{0pt}\mytexttt{\color{param} (Types.dimension\_t m, Types.dimension\_t n, Types.matrix\_t x, ICellFunc f)}} \\
\hline
\end{tabularx}
}

\par
Iterate matrix and apply function to each cell

\par
\begin{description}
\item [\colorbox{tagtype}{\color{white} \textbf{\textsf{PARAMETER}}}] \textbf{\underline{m}} number of rows
\item [\colorbox{tagtype}{\color{white} \textbf{\textsf{PARAMETER}}}] \textbf{\underline{n}} number of columns
\item [\colorbox{tagtype}{\color{white} \textbf{\textsf{PARAMETER}}}] \textbf{\underline{x}} matrix
\item [\colorbox{tagtype}{\color{white} \textbf{\textsf{PARAMETER}}}] \textbf{\underline{f}} function to apply
\item [\colorbox{tagtype}{\color{white} \textbf{\textsf{RETURN}}}] \textbf{\underline{}} updated matrix
\end{description}

\rule{\linewidth}{0.5pt}
\subsection*{\textsf{\colorbox{headtoc}{\color{white} FUNCTION}
dasum}}

\hypertarget{ecldoc:blas.dasum}{}
\hspace{0pt} \hyperlink{ecldoc:blas}{BLAS} \textbackslash 

{\renewcommand{\arraystretch}{1.5}
\begin{tabularx}{\textwidth}{|>{\raggedright\arraybackslash}l|X|}
\hline
\hspace{0pt}\mytexttt{\color{red} Types.value\_t} & \textbf{dasum} \\
\hline
\multicolumn{2}{|>{\raggedright\arraybackslash}X|}{\hspace{0pt}\mytexttt{\color{param} (Types.dimension\_t m, Types.matrix\_t x, Types.dimension\_t incx, Types.dimension\_t skipped=0)}} \\
\hline
\end{tabularx}
}

\par
Absolute sum, the 1 norm of a vector.

\par
\begin{description}
\item [\colorbox{tagtype}{\color{white} \textbf{\textsf{PARAMETER}}}] \textbf{\underline{m}} the number of entries
\item [\colorbox{tagtype}{\color{white} \textbf{\textsf{PARAMETER}}}] \textbf{\underline{x}} the column major matrix holding the vector
\item [\colorbox{tagtype}{\color{white} \textbf{\textsf{PARAMETER}}}] \textbf{\underline{incx}} the increment for x, 1 in the case of an actual vector
\item [\colorbox{tagtype}{\color{white} \textbf{\textsf{PARAMETER}}}] \textbf{\underline{skipped}} default is zero, the number of entries stepped over to get to the first entry
\item [\colorbox{tagtype}{\color{white} \textbf{\textsf{RETURN}}}] \textbf{\underline{}} the sum of the absolute values
\end{description}

\rule{\linewidth}{0.5pt}
\subsection*{\textsf{\colorbox{headtoc}{\color{white} FUNCTION}
daxpy}}

\hypertarget{ecldoc:blas.daxpy}{}
\hspace{0pt} \hyperlink{ecldoc:blas}{BLAS} \textbackslash 

{\renewcommand{\arraystretch}{1.5}
\begin{tabularx}{\textwidth}{|>{\raggedright\arraybackslash}l|X|}
\hline
\hspace{0pt}\mytexttt{\color{red} Types.matrix\_t} & \textbf{daxpy} \\
\hline
\multicolumn{2}{|>{\raggedright\arraybackslash}X|}{\hspace{0pt}\mytexttt{\color{param} (Types.dimension\_t N, Types.value\_t alpha, Types.matrix\_t X, Types.dimension\_t incX, Types.matrix\_t Y, Types.dimension\_t incY, Types.dimension\_t x\_skipped=0, Types.dimension\_t y\_skipped=0)}} \\
\hline
\end{tabularx}
}

\par
alpha*X + Y

\par
\begin{description}
\item [\colorbox{tagtype}{\color{white} \textbf{\textsf{PARAMETER}}}] \textbf{\underline{N}} number of elements in vector
\item [\colorbox{tagtype}{\color{white} \textbf{\textsf{PARAMETER}}}] \textbf{\underline{alpha}} the scalar multiplier
\item [\colorbox{tagtype}{\color{white} \textbf{\textsf{PARAMETER}}}] \textbf{\underline{X}} the column major matrix holding the vector X
\item [\colorbox{tagtype}{\color{white} \textbf{\textsf{PARAMETER}}}] \textbf{\underline{incX}} the increment or stride for the vector
\item [\colorbox{tagtype}{\color{white} \textbf{\textsf{PARAMETER}}}] \textbf{\underline{Y}} the column major matrix holding the vector Y
\item [\colorbox{tagtype}{\color{white} \textbf{\textsf{PARAMETER}}}] \textbf{\underline{incY}} the increment or stride of Y
\item [\colorbox{tagtype}{\color{white} \textbf{\textsf{PARAMETER}}}] \textbf{\underline{x\_skipped}} number of entries skipped to get to the first X
\item [\colorbox{tagtype}{\color{white} \textbf{\textsf{PARAMETER}}}] \textbf{\underline{y\_skipped}} number of entries skipped to get to the first Y
\item [\colorbox{tagtype}{\color{white} \textbf{\textsf{RETURN}}}] \textbf{\underline{}} the updated matrix
\end{description}

\rule{\linewidth}{0.5pt}
\subsection*{\textsf{\colorbox{headtoc}{\color{white} FUNCTION}
dgemm}}

\hypertarget{ecldoc:blas.dgemm}{}
\hspace{0pt} \hyperlink{ecldoc:blas}{BLAS} \textbackslash 

{\renewcommand{\arraystretch}{1.5}
\begin{tabularx}{\textwidth}{|>{\raggedright\arraybackslash}l|X|}
\hline
\hspace{0pt}\mytexttt{\color{red} Types.matrix\_t} & \textbf{dgemm} \\
\hline
\multicolumn{2}{|>{\raggedright\arraybackslash}X|}{\hspace{0pt}\mytexttt{\color{param} (BOOLEAN transposeA, BOOLEAN transposeB, Types.dimension\_t M, Types.dimension\_t N, Types.dimension\_t K, Types.value\_t alpha, Types.matrix\_t A, Types.matrix\_t B, Types.value\_t beta=0.0, Types.matrix\_t C=[])}} \\
\hline
\end{tabularx}
}

\par
alpha*op(A) op(B) + beta*C where op() is transpose

\par
\begin{description}
\item [\colorbox{tagtype}{\color{white} \textbf{\textsf{PARAMETER}}}] \textbf{\underline{transposeA}} true when transpose of A is used
\item [\colorbox{tagtype}{\color{white} \textbf{\textsf{PARAMETER}}}] \textbf{\underline{transposeB}} true when transpose of B is used
\item [\colorbox{tagtype}{\color{white} \textbf{\textsf{PARAMETER}}}] \textbf{\underline{M}} number of rows in product
\item [\colorbox{tagtype}{\color{white} \textbf{\textsf{PARAMETER}}}] \textbf{\underline{N}} number of columns in product
\item [\colorbox{tagtype}{\color{white} \textbf{\textsf{PARAMETER}}}] \textbf{\underline{K}} number of columns/rows for the multiplier/multiplicand
\item [\colorbox{tagtype}{\color{white} \textbf{\textsf{PARAMETER}}}] \textbf{\underline{alpha}} scalar used on A
\item [\colorbox{tagtype}{\color{white} \textbf{\textsf{PARAMETER}}}] \textbf{\underline{A}} matrix A
\item [\colorbox{tagtype}{\color{white} \textbf{\textsf{PARAMETER}}}] \textbf{\underline{B}} matrix B
\item [\colorbox{tagtype}{\color{white} \textbf{\textsf{PARAMETER}}}] \textbf{\underline{beta}} scalar for matrix C
\item [\colorbox{tagtype}{\color{white} \textbf{\textsf{PARAMETER}}}] \textbf{\underline{C}} matrix C or empty
\end{description}

\rule{\linewidth}{0.5pt}
\subsection*{\textsf{\colorbox{headtoc}{\color{white} FUNCTION}
dgetf2}}

\hypertarget{ecldoc:blas.dgetf2}{}
\hspace{0pt} \hyperlink{ecldoc:blas}{BLAS} \textbackslash 

{\renewcommand{\arraystretch}{1.5}
\begin{tabularx}{\textwidth}{|>{\raggedright\arraybackslash}l|X|}
\hline
\hspace{0pt}\mytexttt{\color{red} Types.matrix\_t} & \textbf{dgetf2} \\
\hline
\multicolumn{2}{|>{\raggedright\arraybackslash}X|}{\hspace{0pt}\mytexttt{\color{param} (Types.dimension\_t m, Types.dimension\_t n, Types.matrix\_t a)}} \\
\hline
\end{tabularx}
}

\par
Compute LU Factorization of matrix A.

\par
\begin{description}
\item [\colorbox{tagtype}{\color{white} \textbf{\textsf{PARAMETER}}}] \textbf{\underline{m}} number of rows of A
\item [\colorbox{tagtype}{\color{white} \textbf{\textsf{PARAMETER}}}] \textbf{\underline{n}} number of columns of A
\item [\colorbox{tagtype}{\color{white} \textbf{\textsf{RETURN}}}] \textbf{\underline{}} composite matrix of factors, lower triangle has an implied diagonal of ones. Upper triangle has the diagonal of the composite.
\end{description}

\rule{\linewidth}{0.5pt}
\subsection*{\textsf{\colorbox{headtoc}{\color{white} FUNCTION}
dpotf2}}

\hypertarget{ecldoc:blas.dpotf2}{}
\hspace{0pt} \hyperlink{ecldoc:blas}{BLAS} \textbackslash 

{\renewcommand{\arraystretch}{1.5}
\begin{tabularx}{\textwidth}{|>{\raggedright\arraybackslash}l|X|}
\hline
\hspace{0pt}\mytexttt{\color{red} Types.matrix\_t} & \textbf{dpotf2} \\
\hline
\multicolumn{2}{|>{\raggedright\arraybackslash}X|}{\hspace{0pt}\mytexttt{\color{param} (Types.Triangle tri, Types.dimension\_t r, Types.matrix\_t A, BOOLEAN clear=TRUE)}} \\
\hline
\end{tabularx}
}

\par
DPOTF2 computes the Cholesky factorization of a real symmetric positive definite matrix A. The factorization has the form A = U**T * U , if UPLO = 'U', or A = L * L**T, if UPLO = 'L', where U is an upper triangular matrix and L is lower triangular. This is the unblocked version of the algorithm, calling Level 2 BLAS.

\par
\begin{description}
\item [\colorbox{tagtype}{\color{white} \textbf{\textsf{PARAMETER}}}] \textbf{\underline{tri}} indicate whether upper or lower triangle is used
\item [\colorbox{tagtype}{\color{white} \textbf{\textsf{PARAMETER}}}] \textbf{\underline{r}} number of rows/columns in the square matrix
\item [\colorbox{tagtype}{\color{white} \textbf{\textsf{PARAMETER}}}] \textbf{\underline{A}} the square matrix
\item [\colorbox{tagtype}{\color{white} \textbf{\textsf{PARAMETER}}}] \textbf{\underline{clear}} clears the unused triangle
\item [\colorbox{tagtype}{\color{white} \textbf{\textsf{RETURN}}}] \textbf{\underline{}} the triangular matrix requested.
\end{description}

\rule{\linewidth}{0.5pt}
\subsection*{\textsf{\colorbox{headtoc}{\color{white} FUNCTION}
dscal}}

\hypertarget{ecldoc:blas.dscal}{}
\hspace{0pt} \hyperlink{ecldoc:blas}{BLAS} \textbackslash 

{\renewcommand{\arraystretch}{1.5}
\begin{tabularx}{\textwidth}{|>{\raggedright\arraybackslash}l|X|}
\hline
\hspace{0pt}\mytexttt{\color{red} Types.matrix\_t} & \textbf{dscal} \\
\hline
\multicolumn{2}{|>{\raggedright\arraybackslash}X|}{\hspace{0pt}\mytexttt{\color{param} (Types.dimension\_t N, Types.value\_t alpha, Types.matrix\_t X, Types.dimension\_t incX, Types.dimension\_t skipped=0)}} \\
\hline
\end{tabularx}
}

\par
Scale a vector alpha

\par
\begin{description}
\item [\colorbox{tagtype}{\color{white} \textbf{\textsf{PARAMETER}}}] \textbf{\underline{N}} number of elements in the vector
\item [\colorbox{tagtype}{\color{white} \textbf{\textsf{PARAMETER}}}] \textbf{\underline{alpha}} the scaling factor
\item [\colorbox{tagtype}{\color{white} \textbf{\textsf{PARAMETER}}}] \textbf{\underline{X}} the column major matrix holding the vector
\item [\colorbox{tagtype}{\color{white} \textbf{\textsf{PARAMETER}}}] \textbf{\underline{incX}} the stride to get to the next element in the vector
\item [\colorbox{tagtype}{\color{white} \textbf{\textsf{PARAMETER}}}] \textbf{\underline{skipped}} the number of elements skipped to get to the first element
\item [\colorbox{tagtype}{\color{white} \textbf{\textsf{RETURN}}}] \textbf{\underline{}} the updated matrix
\end{description}

\rule{\linewidth}{0.5pt}
\subsection*{\textsf{\colorbox{headtoc}{\color{white} FUNCTION}
dsyrk}}

\hypertarget{ecldoc:blas.dsyrk}{}
\hspace{0pt} \hyperlink{ecldoc:blas}{BLAS} \textbackslash 

{\renewcommand{\arraystretch}{1.5}
\begin{tabularx}{\textwidth}{|>{\raggedright\arraybackslash}l|X|}
\hline
\hspace{0pt}\mytexttt{\color{red} Types.matrix\_t} & \textbf{dsyrk} \\
\hline
\multicolumn{2}{|>{\raggedright\arraybackslash}X|}{\hspace{0pt}\mytexttt{\color{param} (Types.Triangle tri, BOOLEAN transposeA, Types.dimension\_t N, Types.dimension\_t K, Types.value\_t alpha, Types.matrix\_t A, Types.value\_t beta, Types.matrix\_t C, BOOLEAN clear=FALSE)}} \\
\hline
\end{tabularx}
}

\par
Implements symmetric rank update C 

\par
\begin{description}
\item [\colorbox{tagtype}{\color{white} \textbf{\textsf{PARAMETER}}}] \textbf{\underline{tri}} update upper or lower triangle
\item [\colorbox{tagtype}{\color{white} \textbf{\textsf{PARAMETER}}}] \textbf{\underline{transposeA}} Transpose the A matrix to be NxK
\item [\colorbox{tagtype}{\color{white} \textbf{\textsf{PARAMETER}}}] \textbf{\underline{N}} number of rows
\item [\colorbox{tagtype}{\color{white} \textbf{\textsf{PARAMETER}}}] \textbf{\underline{K}} number of columns in the update matrix or transpose
\item [\colorbox{tagtype}{\color{white} \textbf{\textsf{PARAMETER}}}] \textbf{\underline{alpha}} the alpha scalar
\item [\colorbox{tagtype}{\color{white} \textbf{\textsf{PARAMETER}}}] \textbf{\underline{A}} the update matrix, either NxK or KxN
\item [\colorbox{tagtype}{\color{white} \textbf{\textsf{PARAMETER}}}] \textbf{\underline{beta}} the beta scalar
\item [\colorbox{tagtype}{\color{white} \textbf{\textsf{PARAMETER}}}] \textbf{\underline{C}} the matrix to update
\item [\colorbox{tagtype}{\color{white} \textbf{\textsf{PARAMETER}}}] \textbf{\underline{clear}} clear the triangle that is not updated. BLAS assumes that symmetric matrices have only one of the triangles and this option lets you make that true.
\end{description}

\rule{\linewidth}{0.5pt}
\subsection*{\textsf{\colorbox{headtoc}{\color{white} FUNCTION}
dtrsm}}

\hypertarget{ecldoc:blas.dtrsm}{}
\hspace{0pt} \hyperlink{ecldoc:blas}{BLAS} \textbackslash 

{\renewcommand{\arraystretch}{1.5}
\begin{tabularx}{\textwidth}{|>{\raggedright\arraybackslash}l|X|}
\hline
\hspace{0pt}\mytexttt{\color{red} Types.matrix\_t} & \textbf{dtrsm} \\
\hline
\multicolumn{2}{|>{\raggedright\arraybackslash}X|}{\hspace{0pt}\mytexttt{\color{param} (Types.Side side, Types.Triangle tri, BOOLEAN transposeA, Types.Diagonal diag, Types.dimension\_t M, Types.dimension\_t N, Types.dimension\_t lda, Types.value\_t alpha, Types.matrix\_t A, Types.matrix\_t B)}} \\
\hline
\end{tabularx}
}

\par
Triangular matrix solver. op(A) X = alpha B or X op(A) = alpha B where op is Transpose, X and B is MxN

\par
\begin{description}
\item [\colorbox{tagtype}{\color{white} \textbf{\textsf{PARAMETER}}}] \textbf{\underline{side}} side for A, Side.Ax is op(A) X = alpha B
\item [\colorbox{tagtype}{\color{white} \textbf{\textsf{PARAMETER}}}] \textbf{\underline{tri}} Says whether A is Upper or Lower triangle
\item [\colorbox{tagtype}{\color{white} \textbf{\textsf{PARAMETER}}}] \textbf{\underline{transposeA}} is op(A) the transpose of A
\item [\colorbox{tagtype}{\color{white} \textbf{\textsf{PARAMETER}}}] \textbf{\underline{diag}} is the diagonal an implied unit diagonal or supplied
\item [\colorbox{tagtype}{\color{white} \textbf{\textsf{PARAMETER}}}] \textbf{\underline{M}} number of rows
\item [\colorbox{tagtype}{\color{white} \textbf{\textsf{PARAMETER}}}] \textbf{\underline{N}} number of columns
\item [\colorbox{tagtype}{\color{white} \textbf{\textsf{PARAMETER}}}] \textbf{\underline{lda}} the leading dimension of the A matrix, either M or N
\item [\colorbox{tagtype}{\color{white} \textbf{\textsf{PARAMETER}}}] \textbf{\underline{alpha}} the scalar multiplier for B
\item [\colorbox{tagtype}{\color{white} \textbf{\textsf{PARAMETER}}}] \textbf{\underline{A}} a triangular matrix
\item [\colorbox{tagtype}{\color{white} \textbf{\textsf{PARAMETER}}}] \textbf{\underline{B}} the matrix of values for the solve
\item [\colorbox{tagtype}{\color{white} \textbf{\textsf{RETURN}}}] \textbf{\underline{}} the matrix of coefficients to get B.
\end{description}

\rule{\linewidth}{0.5pt}
\subsection*{\textsf{\colorbox{headtoc}{\color{white} FUNCTION}
extract\_diag}}

\hypertarget{ecldoc:blas.extract_diag}{}
\hspace{0pt} \hyperlink{ecldoc:blas}{BLAS} \textbackslash 

{\renewcommand{\arraystretch}{1.5}
\begin{tabularx}{\textwidth}{|>{\raggedright\arraybackslash}l|X|}
\hline
\hspace{0pt}\mytexttt{\color{red} Types.matrix\_t} & \textbf{extract\_diag} \\
\hline
\multicolumn{2}{|>{\raggedright\arraybackslash}X|}{\hspace{0pt}\mytexttt{\color{param} (Types.dimension\_t m, Types.dimension\_t n, Types.matrix\_t x)}} \\
\hline
\end{tabularx}
}

\par
Extract the diagonal of he matrix

\par
\begin{description}
\item [\colorbox{tagtype}{\color{white} \textbf{\textsf{PARAMETER}}}] \textbf{\underline{m}} number of rows
\item [\colorbox{tagtype}{\color{white} \textbf{\textsf{PARAMETER}}}] \textbf{\underline{n}} number of columns
\item [\colorbox{tagtype}{\color{white} \textbf{\textsf{PARAMETER}}}] \textbf{\underline{x}} matrix from which to extract the diagonal
\item [\colorbox{tagtype}{\color{white} \textbf{\textsf{RETURN}}}] \textbf{\underline{}} diagonal matrix
\end{description}

\rule{\linewidth}{0.5pt}
\subsection*{\textsf{\colorbox{headtoc}{\color{white} FUNCTION}
extract\_tri}}

\hypertarget{ecldoc:blas.extract_tri}{}
\hspace{0pt} \hyperlink{ecldoc:blas}{BLAS} \textbackslash 

{\renewcommand{\arraystretch}{1.5}
\begin{tabularx}{\textwidth}{|>{\raggedright\arraybackslash}l|X|}
\hline
\hspace{0pt}\mytexttt{\color{red} Types.matrix\_t} & \textbf{extract\_tri} \\
\hline
\multicolumn{2}{|>{\raggedright\arraybackslash}X|}{\hspace{0pt}\mytexttt{\color{param} (Types.dimension\_t m, Types.dimension\_t n, Types.Triangle tri, Types.Diagonal dt, Types.matrix\_t a)}} \\
\hline
\end{tabularx}
}

\par
Extract the upper or lower triangle. Diagonal can be actual or implied unit diagonal.

\par
\begin{description}
\item [\colorbox{tagtype}{\color{white} \textbf{\textsf{PARAMETER}}}] \textbf{\underline{m}} number of rows
\item [\colorbox{tagtype}{\color{white} \textbf{\textsf{PARAMETER}}}] \textbf{\underline{n}} number of columns
\item [\colorbox{tagtype}{\color{white} \textbf{\textsf{PARAMETER}}}] \textbf{\underline{tri}} Upper or Lower specifier, Triangle.Lower or Triangle.Upper
\item [\colorbox{tagtype}{\color{white} \textbf{\textsf{PARAMETER}}}] \textbf{\underline{dt}} Use Diagonal.NotUnitTri or Diagonal.UnitTri
\item [\colorbox{tagtype}{\color{white} \textbf{\textsf{PARAMETER}}}] \textbf{\underline{a}} Matrix, usually a composite from factoring
\item [\colorbox{tagtype}{\color{white} \textbf{\textsf{RETURN}}}] \textbf{\underline{}} the triangle
\end{description}

\rule{\linewidth}{0.5pt}
\subsection*{\textsf{\colorbox{headtoc}{\color{white} FUNCTION}
make\_diag}}

\hypertarget{ecldoc:blas.make_diag}{}
\hspace{0pt} \hyperlink{ecldoc:blas}{BLAS} \textbackslash 

{\renewcommand{\arraystretch}{1.5}
\begin{tabularx}{\textwidth}{|>{\raggedright\arraybackslash}l|X|}
\hline
\hspace{0pt}\mytexttt{\color{red} Types.matrix\_t} & \textbf{make\_diag} \\
\hline
\multicolumn{2}{|>{\raggedright\arraybackslash}X|}{\hspace{0pt}\mytexttt{\color{param} (Types.dimension\_t m, Types.value\_t v=1.0, Types.matrix\_t X=[])}} \\
\hline
\end{tabularx}
}

\par
Generate a diagonal matrix.

\par
\begin{description}
\item [\colorbox{tagtype}{\color{white} \textbf{\textsf{PARAMETER}}}] \textbf{\underline{m}} number of diagonal entries
\item [\colorbox{tagtype}{\color{white} \textbf{\textsf{PARAMETER}}}] \textbf{\underline{v}} option value, defaults to 1
\item [\colorbox{tagtype}{\color{white} \textbf{\textsf{PARAMETER}}}] \textbf{\underline{X}} optional input of diagonal values, multiplied by v.
\item [\colorbox{tagtype}{\color{white} \textbf{\textsf{RETURN}}}] \textbf{\underline{}} a diagonal matrix
\end{description}

\rule{\linewidth}{0.5pt}
\subsection*{\textsf{\colorbox{headtoc}{\color{white} FUNCTION}
make\_vector}}

\hypertarget{ecldoc:blas.make_vector}{}
\hspace{0pt} \hyperlink{ecldoc:blas}{BLAS} \textbackslash 

{\renewcommand{\arraystretch}{1.5}
\begin{tabularx}{\textwidth}{|>{\raggedright\arraybackslash}l|X|}
\hline
\hspace{0pt}\mytexttt{\color{red} Types.matrix\_t} & \textbf{make\_vector} \\
\hline
\multicolumn{2}{|>{\raggedright\arraybackslash}X|}{\hspace{0pt}\mytexttt{\color{param} (Types.dimension\_t m, Types.value\_t v=1.0)}} \\
\hline
\end{tabularx}
}

\par
Make a vector of dimension m

\par
\begin{description}
\item [\colorbox{tagtype}{\color{white} \textbf{\textsf{PARAMETER}}}] \textbf{\underline{m}} number of elements
\item [\colorbox{tagtype}{\color{white} \textbf{\textsf{PARAMETER}}}] \textbf{\underline{v}} the values, defaults to 1
\item [\colorbox{tagtype}{\color{white} \textbf{\textsf{RETURN}}}] \textbf{\underline{}} the vector
\end{description}

\rule{\linewidth}{0.5pt}
\subsection*{\textsf{\colorbox{headtoc}{\color{white} FUNCTION}
trace}}

\hypertarget{ecldoc:blas.trace}{}
\hspace{0pt} \hyperlink{ecldoc:blas}{BLAS} \textbackslash 

{\renewcommand{\arraystretch}{1.5}
\begin{tabularx}{\textwidth}{|>{\raggedright\arraybackslash}l|X|}
\hline
\hspace{0pt}\mytexttt{\color{red} Types.value\_t} & \textbf{trace} \\
\hline
\multicolumn{2}{|>{\raggedright\arraybackslash}X|}{\hspace{0pt}\mytexttt{\color{param} (Types.dimension\_t m, Types.dimension\_t n, Types.matrix\_t x)}} \\
\hline
\end{tabularx}
}

\par
The trace of the input matrix

\par
\begin{description}
\item [\colorbox{tagtype}{\color{white} \textbf{\textsf{PARAMETER}}}] \textbf{\underline{m}} number of rows
\item [\colorbox{tagtype}{\color{white} \textbf{\textsf{PARAMETER}}}] \textbf{\underline{n}} number of columns
\item [\colorbox{tagtype}{\color{white} \textbf{\textsf{PARAMETER}}}] \textbf{\underline{x}} the matrix
\item [\colorbox{tagtype}{\color{white} \textbf{\textsf{RETURN}}}] \textbf{\underline{}} the trace (sum of the diagonal entries)
\end{description}

\rule{\linewidth}{0.5pt}



\chapter*{\color{headfile}
BundleBase
}
\hypertarget{ecldoc:toc:BundleBase}{}
\hyperlink{ecldoc:toc:root}{Go Up}


\section*{\underline{\textsf{DESCRIPTIONS}}}
\subsection*{\textsf{\colorbox{headtoc}{\color{white} MODULE}
BundleBase}}

\hypertarget{ecldoc:BundleBase}{}

{\renewcommand{\arraystretch}{1.5}
\begin{tabularx}{\textwidth}{|>{\raggedright\arraybackslash}l|X|}
\hline
\hspace{0pt}\mytexttt{\color{red} } & \textbf{BundleBase} \\
\hline
\end{tabularx}
}

\par





No Documentation Found







\textbf{Children}
\begin{enumerate}
\item \hyperlink{ecldoc:bundlebase.propertyrecord}{PropertyRecord}
: No Documentation Found
\item \hyperlink{ecldoc:bundlebase.name}{Name}
: No Documentation Found
\item \hyperlink{ecldoc:bundlebase.description}{Description}
: No Documentation Found
\item \hyperlink{ecldoc:bundlebase.authors}{Authors}
: No Documentation Found
\item \hyperlink{ecldoc:bundlebase.license}{License}
: No Documentation Found
\item \hyperlink{ecldoc:bundlebase.copyright}{Copyright}
: No Documentation Found
\item \hyperlink{ecldoc:bundlebase.dependson}{DependsOn}
: No Documentation Found
\item \hyperlink{ecldoc:bundlebase.version}{Version}
: No Documentation Found
\item \hyperlink{ecldoc:bundlebase.properties}{Properties}
: No Documentation Found
\item \hyperlink{ecldoc:bundlebase.platformversion}{PlatformVersion}
: No Documentation Found
\end{enumerate}

\rule{\linewidth}{0.5pt}

\subsection*{\textsf{\colorbox{headtoc}{\color{white} RECORD}
PropertyRecord}}

\hypertarget{ecldoc:bundlebase.propertyrecord}{}
\hspace{0pt} \hyperlink{ecldoc:BundleBase}{BundleBase} \textbackslash 

{\renewcommand{\arraystretch}{1.5}
\begin{tabularx}{\textwidth}{|>{\raggedright\arraybackslash}l|X|}
\hline
\hspace{0pt}\mytexttt{\color{red} } & \textbf{PropertyRecord} \\
\hline
\end{tabularx}
}

\par





No Documentation Found







\par
\begin{description}
\item [\colorbox{tagtype}{\color{white} \textbf{\textsf{FIELD}}}] \textbf{\underline{value}} ||| UTF8 --- No Doc
\item [\colorbox{tagtype}{\color{white} \textbf{\textsf{FIELD}}}] \textbf{\underline{key}} ||| UTF8 --- No Doc
\end{description}





\rule{\linewidth}{0.5pt}
\subsection*{\textsf{\colorbox{headtoc}{\color{white} ATTRIBUTE}
Name}}

\hypertarget{ecldoc:bundlebase.name}{}
\hspace{0pt} \hyperlink{ecldoc:BundleBase}{BundleBase} \textbackslash 

{\renewcommand{\arraystretch}{1.5}
\begin{tabularx}{\textwidth}{|>{\raggedright\arraybackslash}l|X|}
\hline
\hspace{0pt}\mytexttt{\color{red} STRING} & \textbf{Name} \\
\hline
\end{tabularx}
}

\par





No Documentation Found








\par
\begin{description}
\item [\colorbox{tagtype}{\color{white} \textbf{\textsf{RETURN}}}] \textbf{STRING} --- 
\end{description}




\rule{\linewidth}{0.5pt}
\subsection*{\textsf{\colorbox{headtoc}{\color{white} ATTRIBUTE}
Description}}

\hypertarget{ecldoc:bundlebase.description}{}
\hspace{0pt} \hyperlink{ecldoc:BundleBase}{BundleBase} \textbackslash 

{\renewcommand{\arraystretch}{1.5}
\begin{tabularx}{\textwidth}{|>{\raggedright\arraybackslash}l|X|}
\hline
\hspace{0pt}\mytexttt{\color{red} UTF8} & \textbf{Description} \\
\hline
\end{tabularx}
}

\par





No Documentation Found








\par
\begin{description}
\item [\colorbox{tagtype}{\color{white} \textbf{\textsf{RETURN}}}] \textbf{UTF8} --- 
\end{description}




\rule{\linewidth}{0.5pt}
\subsection*{\textsf{\colorbox{headtoc}{\color{white} ATTRIBUTE}
Authors}}

\hypertarget{ecldoc:bundlebase.authors}{}
\hspace{0pt} \hyperlink{ecldoc:BundleBase}{BundleBase} \textbackslash 

{\renewcommand{\arraystretch}{1.5}
\begin{tabularx}{\textwidth}{|>{\raggedright\arraybackslash}l|X|}
\hline
\hspace{0pt}\mytexttt{\color{red} SET OF UTF8} & \textbf{Authors} \\
\hline
\end{tabularx}
}

\par





No Documentation Found








\par
\begin{description}
\item [\colorbox{tagtype}{\color{white} \textbf{\textsf{RETURN}}}] \textbf{SET ( UTF8 )} --- 
\end{description}




\rule{\linewidth}{0.5pt}
\subsection*{\textsf{\colorbox{headtoc}{\color{white} ATTRIBUTE}
License}}

\hypertarget{ecldoc:bundlebase.license}{}
\hspace{0pt} \hyperlink{ecldoc:BundleBase}{BundleBase} \textbackslash 

{\renewcommand{\arraystretch}{1.5}
\begin{tabularx}{\textwidth}{|>{\raggedright\arraybackslash}l|X|}
\hline
\hspace{0pt}\mytexttt{\color{red} UTF8} & \textbf{License} \\
\hline
\end{tabularx}
}

\par





No Documentation Found








\par
\begin{description}
\item [\colorbox{tagtype}{\color{white} \textbf{\textsf{RETURN}}}] \textbf{UTF8} --- 
\end{description}




\rule{\linewidth}{0.5pt}
\subsection*{\textsf{\colorbox{headtoc}{\color{white} ATTRIBUTE}
Copyright}}

\hypertarget{ecldoc:bundlebase.copyright}{}
\hspace{0pt} \hyperlink{ecldoc:BundleBase}{BundleBase} \textbackslash 

{\renewcommand{\arraystretch}{1.5}
\begin{tabularx}{\textwidth}{|>{\raggedright\arraybackslash}l|X|}
\hline
\hspace{0pt}\mytexttt{\color{red} UTF8} & \textbf{Copyright} \\
\hline
\end{tabularx}
}

\par





No Documentation Found








\par
\begin{description}
\item [\colorbox{tagtype}{\color{white} \textbf{\textsf{RETURN}}}] \textbf{UTF8} --- 
\end{description}




\rule{\linewidth}{0.5pt}
\subsection*{\textsf{\colorbox{headtoc}{\color{white} ATTRIBUTE}
DependsOn}}

\hypertarget{ecldoc:bundlebase.dependson}{}
\hspace{0pt} \hyperlink{ecldoc:BundleBase}{BundleBase} \textbackslash 

{\renewcommand{\arraystretch}{1.5}
\begin{tabularx}{\textwidth}{|>{\raggedright\arraybackslash}l|X|}
\hline
\hspace{0pt}\mytexttt{\color{red} SET OF STRING} & \textbf{DependsOn} \\
\hline
\end{tabularx}
}

\par





No Documentation Found








\par
\begin{description}
\item [\colorbox{tagtype}{\color{white} \textbf{\textsf{RETURN}}}] \textbf{SET ( STRING )} --- 
\end{description}




\rule{\linewidth}{0.5pt}
\subsection*{\textsf{\colorbox{headtoc}{\color{white} ATTRIBUTE}
Version}}

\hypertarget{ecldoc:bundlebase.version}{}
\hspace{0pt} \hyperlink{ecldoc:BundleBase}{BundleBase} \textbackslash 

{\renewcommand{\arraystretch}{1.5}
\begin{tabularx}{\textwidth}{|>{\raggedright\arraybackslash}l|X|}
\hline
\hspace{0pt}\mytexttt{\color{red} STRING} & \textbf{Version} \\
\hline
\end{tabularx}
}

\par





No Documentation Found








\par
\begin{description}
\item [\colorbox{tagtype}{\color{white} \textbf{\textsf{RETURN}}}] \textbf{STRING} --- 
\end{description}




\rule{\linewidth}{0.5pt}
\subsection*{\textsf{\colorbox{headtoc}{\color{white} ATTRIBUTE}
Properties}}

\hypertarget{ecldoc:bundlebase.properties}{}
\hspace{0pt} \hyperlink{ecldoc:BundleBase}{BundleBase} \textbackslash 

{\renewcommand{\arraystretch}{1.5}
\begin{tabularx}{\textwidth}{|>{\raggedright\arraybackslash}l|X|}
\hline
\hspace{0pt}\mytexttt{\color{red} } & \textbf{Properties} \\
\hline
\end{tabularx}
}

\par





No Documentation Found








\par
\begin{description}
\item [\colorbox{tagtype}{\color{white} \textbf{\textsf{RETURN}}}] \textbf{DICTIONARY ( PropertyRecord )} --- 
\end{description}




\rule{\linewidth}{0.5pt}
\subsection*{\textsf{\colorbox{headtoc}{\color{white} ATTRIBUTE}
PlatformVersion}}

\hypertarget{ecldoc:bundlebase.platformversion}{}
\hspace{0pt} \hyperlink{ecldoc:BundleBase}{BundleBase} \textbackslash 

{\renewcommand{\arraystretch}{1.5}
\begin{tabularx}{\textwidth}{|>{\raggedright\arraybackslash}l|X|}
\hline
\hspace{0pt}\mytexttt{\color{red} STRING} & \textbf{PlatformVersion} \\
\hline
\end{tabularx}
}

\par





No Documentation Found








\par
\begin{description}
\item [\colorbox{tagtype}{\color{white} \textbf{\textsf{RETURN}}}] \textbf{STRING} --- 
\end{description}




\rule{\linewidth}{0.5pt}



\chapter*{\color{headfile}
Date
}
\hypertarget{ecldoc:toc:Date}{}
\hyperlink{ecldoc:toc:root}{Go Up}

\section*{\underline{\textsf{IMPORTS}}}
\begin{doublespace}
{\large
}
\end{doublespace}

\section*{\underline{\textsf{DESCRIPTIONS}}}
\subsection*{\textsf{\colorbox{headtoc}{\color{white} MODULE}
Date}}

\hypertarget{ecldoc:Date}{}

{\renewcommand{\arraystretch}{1.5}
\begin{tabularx}{\textwidth}{|>{\raggedright\arraybackslash}l|X|}
\hline
\hspace{0pt}\mytexttt{\color{red} } & \textbf{Date} \\
\hline
\end{tabularx}
}

\par





No Documentation Found







\textbf{Children}
\begin{enumerate}
\item \hyperlink{ecldoc:date.date_rec}{Date\_rec}
: No Documentation Found
\item \hyperlink{ecldoc:date.date_t}{Date\_t}
: No Documentation Found
\item \hyperlink{ecldoc:date.days_t}{Days\_t}
: No Documentation Found
\item \hyperlink{ecldoc:date.time_rec}{Time\_rec}
: No Documentation Found
\item \hyperlink{ecldoc:date.time_t}{Time\_t}
: No Documentation Found
\item \hyperlink{ecldoc:date.seconds_t}{Seconds\_t}
: No Documentation Found
\item \hyperlink{ecldoc:date.datetime_rec}{DateTime\_rec}
: No Documentation Found
\item \hyperlink{ecldoc:date.timestamp_t}{Timestamp\_t}
: No Documentation Found
\item \hyperlink{ecldoc:date.year}{Year}
: Extracts the year from a date type
\item \hyperlink{ecldoc:date.month}{Month}
: Extracts the month from a date type
\item \hyperlink{ecldoc:date.day}{Day}
: Extracts the day of the month from a date type
\item \hyperlink{ecldoc:date.hour}{Hour}
: Extracts the hour from a time type
\item \hyperlink{ecldoc:date.minute}{Minute}
: Extracts the minutes from a time type
\item \hyperlink{ecldoc:date.second}{Second}
: Extracts the seconds from a time type
\item \hyperlink{ecldoc:date.datefromparts}{DateFromParts}
: Combines year, month day to create a date type
\item \hyperlink{ecldoc:date.timefromparts}{TimeFromParts}
: Combines hour, minute second to create a time type
\item \hyperlink{ecldoc:date.secondsfromparts}{SecondsFromParts}
: Combines date and time components to create a seconds type
\item \hyperlink{ecldoc:date.secondstoparts}{SecondsToParts}
: Converts the number of seconds since epoch to a structure containing date and time parts
\item \hyperlink{ecldoc:date.timestamptoseconds}{TimestampToSeconds}
: Converts the number of microseconds since epoch to the number of seconds since epoch
\item \hyperlink{ecldoc:date.isleapyear}{IsLeapYear}
: Tests whether the year is a leap year in the Gregorian calendar
\item \hyperlink{ecldoc:date.isdateleapyear}{IsDateLeapYear}
: Tests whether a date is a leap year in the Gregorian calendar
\item \hyperlink{ecldoc:date.fromgregorianymd}{FromGregorianYMD}
: Combines year, month, day in the Gregorian calendar to create the number days since 31st December 1BC
\item \hyperlink{ecldoc:date.togregorianymd}{ToGregorianYMD}
: Converts the number days since 31st December 1BC to a date in the Gregorian calendar
\item \hyperlink{ecldoc:date.fromgregoriandate}{FromGregorianDate}
: Converts a date in the Gregorian calendar to the number days since 31st December 1BC
\item \hyperlink{ecldoc:date.togregoriandate}{ToGregorianDate}
: Converts the number days since 31st December 1BC to a date in the Gregorian calendar
\item \hyperlink{ecldoc:date.dayofyear}{DayOfYear}
: Returns a number representing the day of the year indicated by the given date
\item \hyperlink{ecldoc:date.dayofweek}{DayOfWeek}
: Returns a number representing the day of the week indicated by the given date
\item \hyperlink{ecldoc:date.isjulianleapyear}{IsJulianLeapYear}
: Tests whether the year is a leap year in the Julian calendar
\item \hyperlink{ecldoc:date.fromjulianymd}{FromJulianYMD}
: Combines year, month, day in the Julian calendar to create the number days since 31st December 1BC
\item \hyperlink{ecldoc:date.tojulianymd}{ToJulianYMD}
: Converts the number days since 31st December 1BC to a date in the Julian calendar
\item \hyperlink{ecldoc:date.fromjuliandate}{FromJulianDate}
: Converts a date in the Julian calendar to the number days since 31st December 1BC
\item \hyperlink{ecldoc:date.tojuliandate}{ToJulianDate}
: Converts the number days since 31st December 1BC to a date in the Julian calendar
\item \hyperlink{ecldoc:date.dayssince1900}{DaysSince1900}
: Returns the number of days since 1st January 1900 (using the Gregorian Calendar)
\item \hyperlink{ecldoc:date.todayssince1900}{ToDaysSince1900}
: Returns the number of days since 1st January 1900 (using the Gregorian Calendar)
\item \hyperlink{ecldoc:date.fromdayssince1900}{FromDaysSince1900}
: Converts the number days since 1st January 1900 to a date in the Julian calendar
\item \hyperlink{ecldoc:date.yearsbetween}{YearsBetween}
: Calculate the number of whole years between two dates
\item \hyperlink{ecldoc:date.monthsbetween}{MonthsBetween}
: Calculate the number of whole months between two dates
\item \hyperlink{ecldoc:date.daysbetween}{DaysBetween}
: Calculate the number of days between two dates
\item \hyperlink{ecldoc:date.datefromdaterec}{DateFromDateRec}
: Combines the fields from a Date\_rec to create a Date\_t
\item \hyperlink{ecldoc:date.datefromrec}{DateFromRec}
: Combines the fields from a Date\_rec to create a Date\_t
\item \hyperlink{ecldoc:date.timefromtimerec}{TimeFromTimeRec}
: Combines the fields from a Time\_rec to create a Time\_t
\item \hyperlink{ecldoc:date.datefromdatetimerec}{DateFromDateTimeRec}
: Combines the date fields from a DateTime\_rec to create a Date\_t
\item \hyperlink{ecldoc:date.timefromdatetimerec}{TimeFromDateTimeRec}
: Combines the time fields from a DateTime\_rec to create a Time\_t
\item \hyperlink{ecldoc:date.secondsfromdatetimerec}{SecondsFromDateTimeRec}
: Combines the date and time fields from a DateTime\_rec to create a Seconds\_t
\item \hyperlink{ecldoc:date.fromstringtodate}{FromStringToDate}
: Converts a string to a Date\_t using the relevant string format
\item \hyperlink{ecldoc:date.fromstring}{FromString}
: Converts a string to a date using the relevant string format
\item \hyperlink{ecldoc:date.fromstringtotime}{FromStringToTime}
: Converts a string to a Time\_t using the relevant string format
\item \hyperlink{ecldoc:date.matchdatestring}{MatchDateString}
: Matches a string against a set of date string formats and returns a valid Date\_t object from the first format that successfully parses the string
\item \hyperlink{ecldoc:date.matchtimestring}{MatchTimeString}
: Matches a string against a set of time string formats and returns a valid Time\_t object from the first format that successfully parses the string
\item \hyperlink{ecldoc:date.datetostring}{DateToString}
: Formats a date as a string
\item \hyperlink{ecldoc:date.timetostring}{TimeToString}
: Formats a time as a string
\item \hyperlink{ecldoc:date.secondstostring}{SecondsToString}
: Converts a Seconds\_t value into a human-readable string using a format template
\item \hyperlink{ecldoc:date.tostring}{ToString}
: Formats a date as a string
\item \hyperlink{ecldoc:date.convertdateformat}{ConvertDateFormat}
: Converts a date from one format to another
\item \hyperlink{ecldoc:date.convertformat}{ConvertFormat}
: Converts a date from one format to another
\item \hyperlink{ecldoc:date.converttimeformat}{ConvertTimeFormat}
: Converts a time from one format to another
\item \hyperlink{ecldoc:date.convertdateformatmultiple}{ConvertDateFormatMultiple}
: Converts a date that matches one of a set of formats to another
\item \hyperlink{ecldoc:date.convertformatmultiple}{ConvertFormatMultiple}
: Converts a date that matches one of a set of formats to another
\item \hyperlink{ecldoc:date.converttimeformatmultiple}{ConvertTimeFormatMultiple}
: Converts a time that matches one of a set of formats to another
\item \hyperlink{ecldoc:date.adjustdate}{AdjustDate}
: Adjusts a date by incrementing or decrementing year, month and/or day values
\item \hyperlink{ecldoc:date.adjustdatebyseconds}{AdjustDateBySeconds}
: Adjusts a date by adding or subtracting seconds
\item \hyperlink{ecldoc:date.adjusttime}{AdjustTime}
: Adjusts a time by incrementing or decrementing hour, minute and/or second values
\item \hyperlink{ecldoc:date.adjusttimebyseconds}{AdjustTimeBySeconds}
: Adjusts a time by adding or subtracting seconds
\item \hyperlink{ecldoc:date.adjustseconds}{AdjustSeconds}
: Adjusts a Seconds\_t value by adding or subtracting years, months, days, hours, minutes and/or seconds
\item \hyperlink{ecldoc:date.adjustcalendar}{AdjustCalendar}
: Adjusts a date by incrementing or decrementing months and/or years
\item \hyperlink{ecldoc:date.islocaldaylightsavingsineffect}{IsLocalDaylightSavingsInEffect}
: Returns a boolean indicating whether daylight savings time is currently in effect locally
\item \hyperlink{ecldoc:date.localtimezoneoffset}{LocalTimeZoneOffset}
: Returns the offset (in seconds) of the time represented from UTC, with positive values indicating locations east of the Prime Meridian
\item \hyperlink{ecldoc:date.currentdate}{CurrentDate}
: Returns the current date
\item \hyperlink{ecldoc:date.today}{Today}
: Returns the current date in the local time zone
\item \hyperlink{ecldoc:date.currenttime}{CurrentTime}
: Returns the current time of day
\item \hyperlink{ecldoc:date.currentseconds}{CurrentSeconds}
: Returns the current date and time as the number of seconds since epoch
\item \hyperlink{ecldoc:date.currenttimestamp}{CurrentTimestamp}
: Returns the current date and time as the number of microseconds since epoch
\item \hyperlink{ecldoc:date.datesformonth}{DatesForMonth}
: Returns the beginning and ending dates for the month surrounding the given date
\item \hyperlink{ecldoc:date.datesforweek}{DatesForWeek}
: Returns the beginning and ending dates for the week surrounding the given date (Sunday marks the beginning of a week)
\item \hyperlink{ecldoc:date.isvaliddate}{IsValidDate}
: Tests whether a date is valid, both by range-checking the year and by validating each of the other individual components
\item \hyperlink{ecldoc:date.isvalidgregoriandate}{IsValidGregorianDate}
: Tests whether a date is valid in the Gregorian calendar
\item \hyperlink{ecldoc:date.isvalidtime}{IsValidTime}
: Tests whether a time is valid
\item \hyperlink{ecldoc:date.createdate}{CreateDate}
: A transform to create a Date\_rec from the individual elements
\item \hyperlink{ecldoc:date.createdatefromseconds}{CreateDateFromSeconds}
: A transform to create a Date\_rec from a Seconds\_t value
\item \hyperlink{ecldoc:date.createtime}{CreateTime}
: A transform to create a Time\_rec from the individual elements
\item \hyperlink{ecldoc:date.createtimefromseconds}{CreateTimeFromSeconds}
: A transform to create a Time\_rec from a Seconds\_t value
\item \hyperlink{ecldoc:date.createdatetime}{CreateDateTime}
: A transform to create a DateTime\_rec from the individual elements
\item \hyperlink{ecldoc:date.createdatetimefromseconds}{CreateDateTimeFromSeconds}
: A transform to create a DateTime\_rec from a Seconds\_t value
\end{enumerate}

\rule{\linewidth}{0.5pt}

\subsection*{\textsf{\colorbox{headtoc}{\color{white} RECORD}
Date\_rec}}

\hypertarget{ecldoc:date.date_rec}{}
\hspace{0pt} \hyperlink{ecldoc:Date}{Date} \textbackslash 

{\renewcommand{\arraystretch}{1.5}
\begin{tabularx}{\textwidth}{|>{\raggedright\arraybackslash}l|X|}
\hline
\hspace{0pt}\mytexttt{\color{red} } & \textbf{Date\_rec} \\
\hline
\end{tabularx}
}

\par





No Documentation Found







\par
\begin{description}
\item [\colorbox{tagtype}{\color{white} \textbf{\textsf{FIELD}}}] \textbf{\underline{year}} ||| INTEGER2 --- No Doc
\item [\colorbox{tagtype}{\color{white} \textbf{\textsf{FIELD}}}] \textbf{\underline{month}} ||| UNSIGNED1 --- No Doc
\item [\colorbox{tagtype}{\color{white} \textbf{\textsf{FIELD}}}] \textbf{\underline{day}} ||| UNSIGNED1 --- No Doc
\end{description}





\rule{\linewidth}{0.5pt}
\subsection*{\textsf{\colorbox{headtoc}{\color{white} ATTRIBUTE}
Date\_t}}

\hypertarget{ecldoc:date.date_t}{}
\hspace{0pt} \hyperlink{ecldoc:Date}{Date} \textbackslash 

{\renewcommand{\arraystretch}{1.5}
\begin{tabularx}{\textwidth}{|>{\raggedright\arraybackslash}l|X|}
\hline
\hspace{0pt}\mytexttt{\color{red} } & \textbf{Date\_t} \\
\hline
\end{tabularx}
}

\par





No Documentation Found








\par
\begin{description}
\item [\colorbox{tagtype}{\color{white} \textbf{\textsf{RETURN}}}] \textbf{UNSIGNED4} --- 
\end{description}




\rule{\linewidth}{0.5pt}
\subsection*{\textsf{\colorbox{headtoc}{\color{white} ATTRIBUTE}
Days\_t}}

\hypertarget{ecldoc:date.days_t}{}
\hspace{0pt} \hyperlink{ecldoc:Date}{Date} \textbackslash 

{\renewcommand{\arraystretch}{1.5}
\begin{tabularx}{\textwidth}{|>{\raggedright\arraybackslash}l|X|}
\hline
\hspace{0pt}\mytexttt{\color{red} } & \textbf{Days\_t} \\
\hline
\end{tabularx}
}

\par





No Documentation Found








\par
\begin{description}
\item [\colorbox{tagtype}{\color{white} \textbf{\textsf{RETURN}}}] \textbf{INTEGER4} --- 
\end{description}




\rule{\linewidth}{0.5pt}
\subsection*{\textsf{\colorbox{headtoc}{\color{white} RECORD}
Time\_rec}}

\hypertarget{ecldoc:date.time_rec}{}
\hspace{0pt} \hyperlink{ecldoc:Date}{Date} \textbackslash 

{\renewcommand{\arraystretch}{1.5}
\begin{tabularx}{\textwidth}{|>{\raggedright\arraybackslash}l|X|}
\hline
\hspace{0pt}\mytexttt{\color{red} } & \textbf{Time\_rec} \\
\hline
\end{tabularx}
}

\par





No Documentation Found







\par
\begin{description}
\item [\colorbox{tagtype}{\color{white} \textbf{\textsf{FIELD}}}] \textbf{\underline{minute}} ||| UNSIGNED1 --- No Doc
\item [\colorbox{tagtype}{\color{white} \textbf{\textsf{FIELD}}}] \textbf{\underline{second}} ||| UNSIGNED1 --- No Doc
\item [\colorbox{tagtype}{\color{white} \textbf{\textsf{FIELD}}}] \textbf{\underline{hour}} ||| UNSIGNED1 --- No Doc
\end{description}





\rule{\linewidth}{0.5pt}
\subsection*{\textsf{\colorbox{headtoc}{\color{white} ATTRIBUTE}
Time\_t}}

\hypertarget{ecldoc:date.time_t}{}
\hspace{0pt} \hyperlink{ecldoc:Date}{Date} \textbackslash 

{\renewcommand{\arraystretch}{1.5}
\begin{tabularx}{\textwidth}{|>{\raggedright\arraybackslash}l|X|}
\hline
\hspace{0pt}\mytexttt{\color{red} } & \textbf{Time\_t} \\
\hline
\end{tabularx}
}

\par





No Documentation Found








\par
\begin{description}
\item [\colorbox{tagtype}{\color{white} \textbf{\textsf{RETURN}}}] \textbf{UNSIGNED3} --- 
\end{description}




\rule{\linewidth}{0.5pt}
\subsection*{\textsf{\colorbox{headtoc}{\color{white} ATTRIBUTE}
Seconds\_t}}

\hypertarget{ecldoc:date.seconds_t}{}
\hspace{0pt} \hyperlink{ecldoc:Date}{Date} \textbackslash 

{\renewcommand{\arraystretch}{1.5}
\begin{tabularx}{\textwidth}{|>{\raggedright\arraybackslash}l|X|}
\hline
\hspace{0pt}\mytexttt{\color{red} } & \textbf{Seconds\_t} \\
\hline
\end{tabularx}
}

\par





No Documentation Found








\par
\begin{description}
\item [\colorbox{tagtype}{\color{white} \textbf{\textsf{RETURN}}}] \textbf{INTEGER8} --- 
\end{description}




\rule{\linewidth}{0.5pt}
\subsection*{\textsf{\colorbox{headtoc}{\color{white} RECORD}
DateTime\_rec}}

\hypertarget{ecldoc:date.datetime_rec}{}
\hspace{0pt} \hyperlink{ecldoc:Date}{Date} \textbackslash 

{\renewcommand{\arraystretch}{1.5}
\begin{tabularx}{\textwidth}{|>{\raggedright\arraybackslash}l|X|}
\hline
\hspace{0pt}\mytexttt{\color{red} } & \textbf{DateTime\_rec} \\
\hline
\end{tabularx}
}

\par





No Documentation Found







\par
\begin{description}
\item [\colorbox{tagtype}{\color{white} \textbf{\textsf{FIELD}}}] \textbf{\underline{year}} ||| INTEGER2 --- No Doc
\item [\colorbox{tagtype}{\color{white} \textbf{\textsf{FIELD}}}] \textbf{\underline{second}} ||| UNSIGNED1 --- No Doc
\item [\colorbox{tagtype}{\color{white} \textbf{\textsf{FIELD}}}] \textbf{\underline{hour}} ||| UNSIGNED1 --- No Doc
\item [\colorbox{tagtype}{\color{white} \textbf{\textsf{FIELD}}}] \textbf{\underline{minute}} ||| UNSIGNED1 --- No Doc
\item [\colorbox{tagtype}{\color{white} \textbf{\textsf{FIELD}}}] \textbf{\underline{month}} ||| UNSIGNED1 --- No Doc
\item [\colorbox{tagtype}{\color{white} \textbf{\textsf{FIELD}}}] \textbf{\underline{day}} ||| UNSIGNED1 --- No Doc
\end{description}





\rule{\linewidth}{0.5pt}
\subsection*{\textsf{\colorbox{headtoc}{\color{white} ATTRIBUTE}
Timestamp\_t}}

\hypertarget{ecldoc:date.timestamp_t}{}
\hspace{0pt} \hyperlink{ecldoc:Date}{Date} \textbackslash 

{\renewcommand{\arraystretch}{1.5}
\begin{tabularx}{\textwidth}{|>{\raggedright\arraybackslash}l|X|}
\hline
\hspace{0pt}\mytexttt{\color{red} } & \textbf{Timestamp\_t} \\
\hline
\end{tabularx}
}

\par





No Documentation Found








\par
\begin{description}
\item [\colorbox{tagtype}{\color{white} \textbf{\textsf{RETURN}}}] \textbf{INTEGER8} --- 
\end{description}




\rule{\linewidth}{0.5pt}
\subsection*{\textsf{\colorbox{headtoc}{\color{white} FUNCTION}
Year}}

\hypertarget{ecldoc:date.year}{}
\hspace{0pt} \hyperlink{ecldoc:Date}{Date} \textbackslash 

{\renewcommand{\arraystretch}{1.5}
\begin{tabularx}{\textwidth}{|>{\raggedright\arraybackslash}l|X|}
\hline
\hspace{0pt}\mytexttt{\color{red} INTEGER2} & \textbf{Year} \\
\hline
\multicolumn{2}{|>{\raggedright\arraybackslash}X|}{\hspace{0pt}\mytexttt{\color{param} (Date\_t date)}} \\
\hline
\end{tabularx}
}

\par





Extracts the year from a date type.






\par
\begin{description}
\item [\colorbox{tagtype}{\color{white} \textbf{\textsf{PARAMETER}}}] \textbf{\underline{date}} ||| UNSIGNED4 --- The date.
\end{description}







\par
\begin{description}
\item [\colorbox{tagtype}{\color{white} \textbf{\textsf{RETURN}}}] \textbf{INTEGER2} --- An integer representing the year.
\end{description}




\rule{\linewidth}{0.5pt}
\subsection*{\textsf{\colorbox{headtoc}{\color{white} FUNCTION}
Month}}

\hypertarget{ecldoc:date.month}{}
\hspace{0pt} \hyperlink{ecldoc:Date}{Date} \textbackslash 

{\renewcommand{\arraystretch}{1.5}
\begin{tabularx}{\textwidth}{|>{\raggedright\arraybackslash}l|X|}
\hline
\hspace{0pt}\mytexttt{\color{red} UNSIGNED1} & \textbf{Month} \\
\hline
\multicolumn{2}{|>{\raggedright\arraybackslash}X|}{\hspace{0pt}\mytexttt{\color{param} (Date\_t date)}} \\
\hline
\end{tabularx}
}

\par





Extracts the month from a date type.






\par
\begin{description}
\item [\colorbox{tagtype}{\color{white} \textbf{\textsf{PARAMETER}}}] \textbf{\underline{date}} ||| UNSIGNED4 --- The date.
\end{description}







\par
\begin{description}
\item [\colorbox{tagtype}{\color{white} \textbf{\textsf{RETURN}}}] \textbf{UNSIGNED1} --- An integer representing the year.
\end{description}




\rule{\linewidth}{0.5pt}
\subsection*{\textsf{\colorbox{headtoc}{\color{white} FUNCTION}
Day}}

\hypertarget{ecldoc:date.day}{}
\hspace{0pt} \hyperlink{ecldoc:Date}{Date} \textbackslash 

{\renewcommand{\arraystretch}{1.5}
\begin{tabularx}{\textwidth}{|>{\raggedright\arraybackslash}l|X|}
\hline
\hspace{0pt}\mytexttt{\color{red} UNSIGNED1} & \textbf{Day} \\
\hline
\multicolumn{2}{|>{\raggedright\arraybackslash}X|}{\hspace{0pt}\mytexttt{\color{param} (Date\_t date)}} \\
\hline
\end{tabularx}
}

\par





Extracts the day of the month from a date type.






\par
\begin{description}
\item [\colorbox{tagtype}{\color{white} \textbf{\textsf{PARAMETER}}}] \textbf{\underline{date}} ||| UNSIGNED4 --- The date.
\end{description}







\par
\begin{description}
\item [\colorbox{tagtype}{\color{white} \textbf{\textsf{RETURN}}}] \textbf{UNSIGNED1} --- An integer representing the year.
\end{description}




\rule{\linewidth}{0.5pt}
\subsection*{\textsf{\colorbox{headtoc}{\color{white} FUNCTION}
Hour}}

\hypertarget{ecldoc:date.hour}{}
\hspace{0pt} \hyperlink{ecldoc:Date}{Date} \textbackslash 

{\renewcommand{\arraystretch}{1.5}
\begin{tabularx}{\textwidth}{|>{\raggedright\arraybackslash}l|X|}
\hline
\hspace{0pt}\mytexttt{\color{red} UNSIGNED1} & \textbf{Hour} \\
\hline
\multicolumn{2}{|>{\raggedright\arraybackslash}X|}{\hspace{0pt}\mytexttt{\color{param} (Time\_t time)}} \\
\hline
\end{tabularx}
}

\par





Extracts the hour from a time type.






\par
\begin{description}
\item [\colorbox{tagtype}{\color{white} \textbf{\textsf{PARAMETER}}}] \textbf{\underline{time}} ||| UNSIGNED3 --- The time.
\end{description}







\par
\begin{description}
\item [\colorbox{tagtype}{\color{white} \textbf{\textsf{RETURN}}}] \textbf{UNSIGNED1} --- An integer representing the hour.
\end{description}




\rule{\linewidth}{0.5pt}
\subsection*{\textsf{\colorbox{headtoc}{\color{white} FUNCTION}
Minute}}

\hypertarget{ecldoc:date.minute}{}
\hspace{0pt} \hyperlink{ecldoc:Date}{Date} \textbackslash 

{\renewcommand{\arraystretch}{1.5}
\begin{tabularx}{\textwidth}{|>{\raggedright\arraybackslash}l|X|}
\hline
\hspace{0pt}\mytexttt{\color{red} UNSIGNED1} & \textbf{Minute} \\
\hline
\multicolumn{2}{|>{\raggedright\arraybackslash}X|}{\hspace{0pt}\mytexttt{\color{param} (Time\_t time)}} \\
\hline
\end{tabularx}
}

\par





Extracts the minutes from a time type.






\par
\begin{description}
\item [\colorbox{tagtype}{\color{white} \textbf{\textsf{PARAMETER}}}] \textbf{\underline{time}} ||| UNSIGNED3 --- The time.
\end{description}







\par
\begin{description}
\item [\colorbox{tagtype}{\color{white} \textbf{\textsf{RETURN}}}] \textbf{UNSIGNED1} --- An integer representing the minutes.
\end{description}




\rule{\linewidth}{0.5pt}
\subsection*{\textsf{\colorbox{headtoc}{\color{white} FUNCTION}
Second}}

\hypertarget{ecldoc:date.second}{}
\hspace{0pt} \hyperlink{ecldoc:Date}{Date} \textbackslash 

{\renewcommand{\arraystretch}{1.5}
\begin{tabularx}{\textwidth}{|>{\raggedright\arraybackslash}l|X|}
\hline
\hspace{0pt}\mytexttt{\color{red} UNSIGNED1} & \textbf{Second} \\
\hline
\multicolumn{2}{|>{\raggedright\arraybackslash}X|}{\hspace{0pt}\mytexttt{\color{param} (Time\_t time)}} \\
\hline
\end{tabularx}
}

\par





Extracts the seconds from a time type.






\par
\begin{description}
\item [\colorbox{tagtype}{\color{white} \textbf{\textsf{PARAMETER}}}] \textbf{\underline{time}} ||| UNSIGNED3 --- The time.
\end{description}







\par
\begin{description}
\item [\colorbox{tagtype}{\color{white} \textbf{\textsf{RETURN}}}] \textbf{UNSIGNED1} --- An integer representing the seconds.
\end{description}




\rule{\linewidth}{0.5pt}
\subsection*{\textsf{\colorbox{headtoc}{\color{white} FUNCTION}
DateFromParts}}

\hypertarget{ecldoc:date.datefromparts}{}
\hspace{0pt} \hyperlink{ecldoc:Date}{Date} \textbackslash 

{\renewcommand{\arraystretch}{1.5}
\begin{tabularx}{\textwidth}{|>{\raggedright\arraybackslash}l|X|}
\hline
\hspace{0pt}\mytexttt{\color{red} Date\_t} & \textbf{DateFromParts} \\
\hline
\multicolumn{2}{|>{\raggedright\arraybackslash}X|}{\hspace{0pt}\mytexttt{\color{param} (INTEGER2 year, UNSIGNED1 month, UNSIGNED1 day)}} \\
\hline
\end{tabularx}
}

\par





Combines year, month day to create a date type.






\par
\begin{description}
\item [\colorbox{tagtype}{\color{white} \textbf{\textsf{PARAMETER}}}] \textbf{\underline{year}} ||| INTEGER2 --- The year (0-9999).
\item [\colorbox{tagtype}{\color{white} \textbf{\textsf{PARAMETER}}}] \textbf{\underline{month}} ||| UNSIGNED1 --- The month (1-12).
\item [\colorbox{tagtype}{\color{white} \textbf{\textsf{PARAMETER}}}] \textbf{\underline{day}} ||| UNSIGNED1 --- The day (1..daysInMonth).
\end{description}







\par
\begin{description}
\item [\colorbox{tagtype}{\color{white} \textbf{\textsf{RETURN}}}] \textbf{UNSIGNED4} --- A date created by combining the fields.
\end{description}




\rule{\linewidth}{0.5pt}
\subsection*{\textsf{\colorbox{headtoc}{\color{white} FUNCTION}
TimeFromParts}}

\hypertarget{ecldoc:date.timefromparts}{}
\hspace{0pt} \hyperlink{ecldoc:Date}{Date} \textbackslash 

{\renewcommand{\arraystretch}{1.5}
\begin{tabularx}{\textwidth}{|>{\raggedright\arraybackslash}l|X|}
\hline
\hspace{0pt}\mytexttt{\color{red} Time\_t} & \textbf{TimeFromParts} \\
\hline
\multicolumn{2}{|>{\raggedright\arraybackslash}X|}{\hspace{0pt}\mytexttt{\color{param} (UNSIGNED1 hour, UNSIGNED1 minute, UNSIGNED1 second)}} \\
\hline
\end{tabularx}
}

\par





Combines hour, minute second to create a time type.






\par
\begin{description}
\item [\colorbox{tagtype}{\color{white} \textbf{\textsf{PARAMETER}}}] \textbf{\underline{minute}} ||| UNSIGNED1 --- The minute (0-59).
\item [\colorbox{tagtype}{\color{white} \textbf{\textsf{PARAMETER}}}] \textbf{\underline{second}} ||| UNSIGNED1 --- The second (0-59).
\item [\colorbox{tagtype}{\color{white} \textbf{\textsf{PARAMETER}}}] \textbf{\underline{hour}} ||| UNSIGNED1 --- The hour (0-23).
\end{description}







\par
\begin{description}
\item [\colorbox{tagtype}{\color{white} \textbf{\textsf{RETURN}}}] \textbf{UNSIGNED3} --- A time created by combining the fields.
\end{description}




\rule{\linewidth}{0.5pt}
\subsection*{\textsf{\colorbox{headtoc}{\color{white} FUNCTION}
SecondsFromParts}}

\hypertarget{ecldoc:date.secondsfromparts}{}
\hspace{0pt} \hyperlink{ecldoc:Date}{Date} \textbackslash 

{\renewcommand{\arraystretch}{1.5}
\begin{tabularx}{\textwidth}{|>{\raggedright\arraybackslash}l|X|}
\hline
\hspace{0pt}\mytexttt{\color{red} Seconds\_t} & \textbf{SecondsFromParts} \\
\hline
\multicolumn{2}{|>{\raggedright\arraybackslash}X|}{\hspace{0pt}\mytexttt{\color{param} (INTEGER2 year, UNSIGNED1 month, UNSIGNED1 day, UNSIGNED1 hour, UNSIGNED1 minute, UNSIGNED1 second, BOOLEAN is\_local\_time = FALSE)}} \\
\hline
\end{tabularx}
}

\par





Combines date and time components to create a seconds type. The date must be represented within the Gregorian calendar after the year 1600.






\par
\begin{description}
\item [\colorbox{tagtype}{\color{white} \textbf{\textsf{PARAMETER}}}] \textbf{\underline{year}} ||| INTEGER2 --- The year (1601-30827).
\item [\colorbox{tagtype}{\color{white} \textbf{\textsf{PARAMETER}}}] \textbf{\underline{second}} ||| UNSIGNED1 --- The second (0-59).
\item [\colorbox{tagtype}{\color{white} \textbf{\textsf{PARAMETER}}}] \textbf{\underline{hour}} ||| UNSIGNED1 --- The hour (0-23).
\item [\colorbox{tagtype}{\color{white} \textbf{\textsf{PARAMETER}}}] \textbf{\underline{minute}} ||| UNSIGNED1 --- The minute (0-59).
\item [\colorbox{tagtype}{\color{white} \textbf{\textsf{PARAMETER}}}] \textbf{\underline{month}} ||| UNSIGNED1 --- The month (1-12).
\item [\colorbox{tagtype}{\color{white} \textbf{\textsf{PARAMETER}}}] \textbf{\underline{day}} ||| UNSIGNED1 --- The day (1..daysInMonth).
\item [\colorbox{tagtype}{\color{white} \textbf{\textsf{PARAMETER}}}] \textbf{\underline{is\_local\_time}} ||| BOOLEAN --- TRUE if the datetime components are expressed in local time rather than UTC, FALSE if the components are expressed in UTC. Optional, defaults to FALSE.
\end{description}







\par
\begin{description}
\item [\colorbox{tagtype}{\color{white} \textbf{\textsf{RETURN}}}] \textbf{INTEGER8} --- A Seconds\_t value created by combining the fields.
\end{description}




\rule{\linewidth}{0.5pt}
\subsection*{\textsf{\colorbox{headtoc}{\color{white} MODULE}
SecondsToParts}}

\hypertarget{ecldoc:date.secondstoparts}{}
\hspace{0pt} \hyperlink{ecldoc:Date}{Date} \textbackslash 

{\renewcommand{\arraystretch}{1.5}
\begin{tabularx}{\textwidth}{|>{\raggedright\arraybackslash}l|X|}
\hline
\hspace{0pt}\mytexttt{\color{red} } & \textbf{SecondsToParts} \\
\hline
\multicolumn{2}{|>{\raggedright\arraybackslash}X|}{\hspace{0pt}\mytexttt{\color{param} (Seconds\_t seconds)}} \\
\hline
\end{tabularx}
}

\par





Converts the number of seconds since epoch to a structure containing date and time parts. The result must be representable within the Gregorian calendar after the year 1600.






\par
\begin{description}
\item [\colorbox{tagtype}{\color{white} \textbf{\textsf{PARAMETER}}}] \textbf{\underline{seconds}} ||| INTEGER8 --- The number of seconds since epoch.
\end{description}







\par
\begin{description}
\item [\colorbox{tagtype}{\color{white} \textbf{\textsf{RETURN}}}] \textbf{} --- Module with exported attributes for year, month, day, hour, minute, second, day\_of\_week, date and time.
\end{description}




\textbf{Children}
\begin{enumerate}
\item \hyperlink{ecldoc:date.secondstoparts.result.year}{Year}
: No Documentation Found
\item \hyperlink{ecldoc:date.secondstoparts.result.month}{Month}
: No Documentation Found
\item \hyperlink{ecldoc:date.secondstoparts.result.day}{Day}
: No Documentation Found
\item \hyperlink{ecldoc:date.secondstoparts.result.hour}{Hour}
: No Documentation Found
\item \hyperlink{ecldoc:date.secondstoparts.result.minute}{Minute}
: No Documentation Found
\item \hyperlink{ecldoc:date.secondstoparts.result.second}{Second}
: No Documentation Found
\item \hyperlink{ecldoc:date.secondstoparts.result.day_of_week}{day\_of\_week}
: No Documentation Found
\item \hyperlink{ecldoc:date.secondstoparts.result.date}{date}
: Combines year, month day to create a date type
\item \hyperlink{ecldoc:date.secondstoparts.result.time}{time}
: Combines hour, minute second to create a time type
\end{enumerate}

\rule{\linewidth}{0.5pt}

\subsection*{\textsf{\colorbox{headtoc}{\color{white} ATTRIBUTE}
Year}}

\hypertarget{ecldoc:date.secondstoparts.result.year}{}
\hspace{0pt} \hyperlink{ecldoc:Date}{Date} \textbackslash 
\hspace{0pt} \hyperlink{ecldoc:date.secondstoparts}{SecondsToParts} \textbackslash 

{\renewcommand{\arraystretch}{1.5}
\begin{tabularx}{\textwidth}{|>{\raggedright\arraybackslash}l|X|}
\hline
\hspace{0pt}\mytexttt{\color{red} INTEGER2} & \textbf{Year} \\
\hline
\end{tabularx}
}

\par





No Documentation Found








\par
\begin{description}
\item [\colorbox{tagtype}{\color{white} \textbf{\textsf{RETURN}}}] \textbf{INTEGER2} --- 
\end{description}




\rule{\linewidth}{0.5pt}
\subsection*{\textsf{\colorbox{headtoc}{\color{white} ATTRIBUTE}
Month}}

\hypertarget{ecldoc:date.secondstoparts.result.month}{}
\hspace{0pt} \hyperlink{ecldoc:Date}{Date} \textbackslash 
\hspace{0pt} \hyperlink{ecldoc:date.secondstoparts}{SecondsToParts} \textbackslash 

{\renewcommand{\arraystretch}{1.5}
\begin{tabularx}{\textwidth}{|>{\raggedright\arraybackslash}l|X|}
\hline
\hspace{0pt}\mytexttt{\color{red} UNSIGNED1} & \textbf{Month} \\
\hline
\end{tabularx}
}

\par





No Documentation Found








\par
\begin{description}
\item [\colorbox{tagtype}{\color{white} \textbf{\textsf{RETURN}}}] \textbf{UNSIGNED1} --- 
\end{description}




\rule{\linewidth}{0.5pt}
\subsection*{\textsf{\colorbox{headtoc}{\color{white} ATTRIBUTE}
Day}}

\hypertarget{ecldoc:date.secondstoparts.result.day}{}
\hspace{0pt} \hyperlink{ecldoc:Date}{Date} \textbackslash 
\hspace{0pt} \hyperlink{ecldoc:date.secondstoparts}{SecondsToParts} \textbackslash 

{\renewcommand{\arraystretch}{1.5}
\begin{tabularx}{\textwidth}{|>{\raggedright\arraybackslash}l|X|}
\hline
\hspace{0pt}\mytexttt{\color{red} UNSIGNED1} & \textbf{Day} \\
\hline
\end{tabularx}
}

\par





No Documentation Found








\par
\begin{description}
\item [\colorbox{tagtype}{\color{white} \textbf{\textsf{RETURN}}}] \textbf{UNSIGNED1} --- 
\end{description}




\rule{\linewidth}{0.5pt}
\subsection*{\textsf{\colorbox{headtoc}{\color{white} ATTRIBUTE}
Hour}}

\hypertarget{ecldoc:date.secondstoparts.result.hour}{}
\hspace{0pt} \hyperlink{ecldoc:Date}{Date} \textbackslash 
\hspace{0pt} \hyperlink{ecldoc:date.secondstoparts}{SecondsToParts} \textbackslash 

{\renewcommand{\arraystretch}{1.5}
\begin{tabularx}{\textwidth}{|>{\raggedright\arraybackslash}l|X|}
\hline
\hspace{0pt}\mytexttt{\color{red} UNSIGNED1} & \textbf{Hour} \\
\hline
\end{tabularx}
}

\par





No Documentation Found








\par
\begin{description}
\item [\colorbox{tagtype}{\color{white} \textbf{\textsf{RETURN}}}] \textbf{UNSIGNED1} --- 
\end{description}




\rule{\linewidth}{0.5pt}
\subsection*{\textsf{\colorbox{headtoc}{\color{white} ATTRIBUTE}
Minute}}

\hypertarget{ecldoc:date.secondstoparts.result.minute}{}
\hspace{0pt} \hyperlink{ecldoc:Date}{Date} \textbackslash 
\hspace{0pt} \hyperlink{ecldoc:date.secondstoparts}{SecondsToParts} \textbackslash 

{\renewcommand{\arraystretch}{1.5}
\begin{tabularx}{\textwidth}{|>{\raggedright\arraybackslash}l|X|}
\hline
\hspace{0pt}\mytexttt{\color{red} UNSIGNED1} & \textbf{Minute} \\
\hline
\end{tabularx}
}

\par





No Documentation Found








\par
\begin{description}
\item [\colorbox{tagtype}{\color{white} \textbf{\textsf{RETURN}}}] \textbf{UNSIGNED1} --- 
\end{description}




\rule{\linewidth}{0.5pt}
\subsection*{\textsf{\colorbox{headtoc}{\color{white} ATTRIBUTE}
Second}}

\hypertarget{ecldoc:date.secondstoparts.result.second}{}
\hspace{0pt} \hyperlink{ecldoc:Date}{Date} \textbackslash 
\hspace{0pt} \hyperlink{ecldoc:date.secondstoparts}{SecondsToParts} \textbackslash 

{\renewcommand{\arraystretch}{1.5}
\begin{tabularx}{\textwidth}{|>{\raggedright\arraybackslash}l|X|}
\hline
\hspace{0pt}\mytexttt{\color{red} UNSIGNED1} & \textbf{Second} \\
\hline
\end{tabularx}
}

\par





No Documentation Found








\par
\begin{description}
\item [\colorbox{tagtype}{\color{white} \textbf{\textsf{RETURN}}}] \textbf{UNSIGNED1} --- 
\end{description}




\rule{\linewidth}{0.5pt}
\subsection*{\textsf{\colorbox{headtoc}{\color{white} ATTRIBUTE}
day\_of\_week}}

\hypertarget{ecldoc:date.secondstoparts.result.day_of_week}{}
\hspace{0pt} \hyperlink{ecldoc:Date}{Date} \textbackslash 
\hspace{0pt} \hyperlink{ecldoc:date.secondstoparts}{SecondsToParts} \textbackslash 

{\renewcommand{\arraystretch}{1.5}
\begin{tabularx}{\textwidth}{|>{\raggedright\arraybackslash}l|X|}
\hline
\hspace{0pt}\mytexttt{\color{red} UNSIGNED1} & \textbf{day\_of\_week} \\
\hline
\end{tabularx}
}

\par





No Documentation Found








\par
\begin{description}
\item [\colorbox{tagtype}{\color{white} \textbf{\textsf{RETURN}}}] \textbf{UNSIGNED1} --- 
\end{description}




\rule{\linewidth}{0.5pt}
\subsection*{\textsf{\colorbox{headtoc}{\color{white} ATTRIBUTE}
date}}

\hypertarget{ecldoc:date.secondstoparts.result.date}{}
\hspace{0pt} \hyperlink{ecldoc:Date}{Date} \textbackslash 
\hspace{0pt} \hyperlink{ecldoc:date.secondstoparts}{SecondsToParts} \textbackslash 

{\renewcommand{\arraystretch}{1.5}
\begin{tabularx}{\textwidth}{|>{\raggedright\arraybackslash}l|X|}
\hline
\hspace{0pt}\mytexttt{\color{red} Date\_t} & \textbf{date} \\
\hline
\end{tabularx}
}

\par





Combines year, month day to create a date type.






\par
\begin{description}
\item [\colorbox{tagtype}{\color{white} \textbf{\textsf{PARAMETER}}}] \textbf{\underline{year}} |||  --- The year (0-9999).
\item [\colorbox{tagtype}{\color{white} \textbf{\textsf{PARAMETER}}}] \textbf{\underline{month}} |||  --- The month (1-12).
\item [\colorbox{tagtype}{\color{white} \textbf{\textsf{PARAMETER}}}] \textbf{\underline{day}} |||  --- The day (1..daysInMonth).
\end{description}







\par
\begin{description}
\item [\colorbox{tagtype}{\color{white} \textbf{\textsf{RETURN}}}] \textbf{UNSIGNED4} --- A date created by combining the fields.
\end{description}




\rule{\linewidth}{0.5pt}
\subsection*{\textsf{\colorbox{headtoc}{\color{white} ATTRIBUTE}
time}}

\hypertarget{ecldoc:date.secondstoparts.result.time}{}
\hspace{0pt} \hyperlink{ecldoc:Date}{Date} \textbackslash 
\hspace{0pt} \hyperlink{ecldoc:date.secondstoparts}{SecondsToParts} \textbackslash 

{\renewcommand{\arraystretch}{1.5}
\begin{tabularx}{\textwidth}{|>{\raggedright\arraybackslash}l|X|}
\hline
\hspace{0pt}\mytexttt{\color{red} Time\_t} & \textbf{time} \\
\hline
\end{tabularx}
}

\par





Combines hour, minute second to create a time type.






\par
\begin{description}
\item [\colorbox{tagtype}{\color{white} \textbf{\textsf{PARAMETER}}}] \textbf{\underline{minute}} |||  --- The minute (0-59).
\item [\colorbox{tagtype}{\color{white} \textbf{\textsf{PARAMETER}}}] \textbf{\underline{second}} |||  --- The second (0-59).
\item [\colorbox{tagtype}{\color{white} \textbf{\textsf{PARAMETER}}}] \textbf{\underline{hour}} |||  --- The hour (0-23).
\end{description}







\par
\begin{description}
\item [\colorbox{tagtype}{\color{white} \textbf{\textsf{RETURN}}}] \textbf{UNSIGNED3} --- A time created by combining the fields.
\end{description}




\rule{\linewidth}{0.5pt}


\subsection*{\textsf{\colorbox{headtoc}{\color{white} FUNCTION}
TimestampToSeconds}}

\hypertarget{ecldoc:date.timestamptoseconds}{}
\hspace{0pt} \hyperlink{ecldoc:Date}{Date} \textbackslash 

{\renewcommand{\arraystretch}{1.5}
\begin{tabularx}{\textwidth}{|>{\raggedright\arraybackslash}l|X|}
\hline
\hspace{0pt}\mytexttt{\color{red} Seconds\_t} & \textbf{TimestampToSeconds} \\
\hline
\multicolumn{2}{|>{\raggedright\arraybackslash}X|}{\hspace{0pt}\mytexttt{\color{param} (Timestamp\_t timestamp)}} \\
\hline
\end{tabularx}
}

\par





Converts the number of microseconds since epoch to the number of seconds since epoch.






\par
\begin{description}
\item [\colorbox{tagtype}{\color{white} \textbf{\textsf{PARAMETER}}}] \textbf{\underline{timestamp}} ||| INTEGER8 --- The number of microseconds since epoch.
\end{description}







\par
\begin{description}
\item [\colorbox{tagtype}{\color{white} \textbf{\textsf{RETURN}}}] \textbf{INTEGER8} --- The number of seconds since epoch.
\end{description}




\rule{\linewidth}{0.5pt}
\subsection*{\textsf{\colorbox{headtoc}{\color{white} FUNCTION}
IsLeapYear}}

\hypertarget{ecldoc:date.isleapyear}{}
\hspace{0pt} \hyperlink{ecldoc:Date}{Date} \textbackslash 

{\renewcommand{\arraystretch}{1.5}
\begin{tabularx}{\textwidth}{|>{\raggedright\arraybackslash}l|X|}
\hline
\hspace{0pt}\mytexttt{\color{red} BOOLEAN} & \textbf{IsLeapYear} \\
\hline
\multicolumn{2}{|>{\raggedright\arraybackslash}X|}{\hspace{0pt}\mytexttt{\color{param} (INTEGER2 year)}} \\
\hline
\end{tabularx}
}

\par





Tests whether the year is a leap year in the Gregorian calendar.






\par
\begin{description}
\item [\colorbox{tagtype}{\color{white} \textbf{\textsf{PARAMETER}}}] \textbf{\underline{year}} ||| INTEGER2 --- The year (0-9999).
\end{description}







\par
\begin{description}
\item [\colorbox{tagtype}{\color{white} \textbf{\textsf{RETURN}}}] \textbf{BOOLEAN} --- True if the year is a leap year.
\end{description}




\rule{\linewidth}{0.5pt}
\subsection*{\textsf{\colorbox{headtoc}{\color{white} FUNCTION}
IsDateLeapYear}}

\hypertarget{ecldoc:date.isdateleapyear}{}
\hspace{0pt} \hyperlink{ecldoc:Date}{Date} \textbackslash 

{\renewcommand{\arraystretch}{1.5}
\begin{tabularx}{\textwidth}{|>{\raggedright\arraybackslash}l|X|}
\hline
\hspace{0pt}\mytexttt{\color{red} BOOLEAN} & \textbf{IsDateLeapYear} \\
\hline
\multicolumn{2}{|>{\raggedright\arraybackslash}X|}{\hspace{0pt}\mytexttt{\color{param} (Date\_t date)}} \\
\hline
\end{tabularx}
}

\par





Tests whether a date is a leap year in the Gregorian calendar.






\par
\begin{description}
\item [\colorbox{tagtype}{\color{white} \textbf{\textsf{PARAMETER}}}] \textbf{\underline{date}} ||| UNSIGNED4 --- The date.
\end{description}







\par
\begin{description}
\item [\colorbox{tagtype}{\color{white} \textbf{\textsf{RETURN}}}] \textbf{BOOLEAN} --- True if the year is a leap year.
\end{description}




\rule{\linewidth}{0.5pt}
\subsection*{\textsf{\colorbox{headtoc}{\color{white} FUNCTION}
FromGregorianYMD}}

\hypertarget{ecldoc:date.fromgregorianymd}{}
\hspace{0pt} \hyperlink{ecldoc:Date}{Date} \textbackslash 

{\renewcommand{\arraystretch}{1.5}
\begin{tabularx}{\textwidth}{|>{\raggedright\arraybackslash}l|X|}
\hline
\hspace{0pt}\mytexttt{\color{red} Days\_t} & \textbf{FromGregorianYMD} \\
\hline
\multicolumn{2}{|>{\raggedright\arraybackslash}X|}{\hspace{0pt}\mytexttt{\color{param} (INTEGER2 year, UNSIGNED1 month, UNSIGNED1 day)}} \\
\hline
\end{tabularx}
}

\par





Combines year, month, day in the Gregorian calendar to create the number days since 31st December 1BC.






\par
\begin{description}
\item [\colorbox{tagtype}{\color{white} \textbf{\textsf{PARAMETER}}}] \textbf{\underline{year}} ||| INTEGER2 --- The year (-4713..9999).
\item [\colorbox{tagtype}{\color{white} \textbf{\textsf{PARAMETER}}}] \textbf{\underline{month}} ||| UNSIGNED1 --- The month (1-12). A missing value (0) is treated as 1.
\item [\colorbox{tagtype}{\color{white} \textbf{\textsf{PARAMETER}}}] \textbf{\underline{day}} ||| UNSIGNED1 --- The day (1..daysInMonth). A missing value (0) is treated as 1.
\end{description}







\par
\begin{description}
\item [\colorbox{tagtype}{\color{white} \textbf{\textsf{RETURN}}}] \textbf{INTEGER4} --- The number of elapsed days (1 Jan 1AD = 1)
\end{description}




\rule{\linewidth}{0.5pt}
\subsection*{\textsf{\colorbox{headtoc}{\color{white} MODULE}
ToGregorianYMD}}

\hypertarget{ecldoc:date.togregorianymd}{}
\hspace{0pt} \hyperlink{ecldoc:Date}{Date} \textbackslash 

{\renewcommand{\arraystretch}{1.5}
\begin{tabularx}{\textwidth}{|>{\raggedright\arraybackslash}l|X|}
\hline
\hspace{0pt}\mytexttt{\color{red} } & \textbf{ToGregorianYMD} \\
\hline
\multicolumn{2}{|>{\raggedright\arraybackslash}X|}{\hspace{0pt}\mytexttt{\color{param} (Days\_t days)}} \\
\hline
\end{tabularx}
}

\par





Converts the number days since 31st December 1BC to a date in the Gregorian calendar.






\par
\begin{description}
\item [\colorbox{tagtype}{\color{white} \textbf{\textsf{PARAMETER}}}] \textbf{\underline{days}} ||| INTEGER4 --- The number of elapsed days (1 Jan 1AD = 1)
\end{description}







\par
\begin{description}
\item [\colorbox{tagtype}{\color{white} \textbf{\textsf{RETURN}}}] \textbf{} --- Module containing Year, Month, Day in the Gregorian calendar
\end{description}




\textbf{Children}
\begin{enumerate}
\item \hyperlink{ecldoc:date.togregorianymd.result.year}{year}
: No Documentation Found
\item \hyperlink{ecldoc:date.togregorianymd.result.month}{month}
: No Documentation Found
\item \hyperlink{ecldoc:date.togregorianymd.result.day}{day}
: No Documentation Found
\end{enumerate}

\rule{\linewidth}{0.5pt}

\subsection*{\textsf{\colorbox{headtoc}{\color{white} ATTRIBUTE}
year}}

\hypertarget{ecldoc:date.togregorianymd.result.year}{}
\hspace{0pt} \hyperlink{ecldoc:Date}{Date} \textbackslash 
\hspace{0pt} \hyperlink{ecldoc:date.togregorianymd}{ToGregorianYMD} \textbackslash 

{\renewcommand{\arraystretch}{1.5}
\begin{tabularx}{\textwidth}{|>{\raggedright\arraybackslash}l|X|}
\hline
\hspace{0pt}\mytexttt{\color{red} } & \textbf{year} \\
\hline
\end{tabularx}
}

\par





No Documentation Found








\par
\begin{description}
\item [\colorbox{tagtype}{\color{white} \textbf{\textsf{RETURN}}}] \textbf{INTEGER8} --- 
\end{description}




\rule{\linewidth}{0.5pt}
\subsection*{\textsf{\colorbox{headtoc}{\color{white} ATTRIBUTE}
month}}

\hypertarget{ecldoc:date.togregorianymd.result.month}{}
\hspace{0pt} \hyperlink{ecldoc:Date}{Date} \textbackslash 
\hspace{0pt} \hyperlink{ecldoc:date.togregorianymd}{ToGregorianYMD} \textbackslash 

{\renewcommand{\arraystretch}{1.5}
\begin{tabularx}{\textwidth}{|>{\raggedright\arraybackslash}l|X|}
\hline
\hspace{0pt}\mytexttt{\color{red} } & \textbf{month} \\
\hline
\end{tabularx}
}

\par





No Documentation Found








\par
\begin{description}
\item [\colorbox{tagtype}{\color{white} \textbf{\textsf{RETURN}}}] \textbf{INTEGER8} --- 
\end{description}




\rule{\linewidth}{0.5pt}
\subsection*{\textsf{\colorbox{headtoc}{\color{white} ATTRIBUTE}
day}}

\hypertarget{ecldoc:date.togregorianymd.result.day}{}
\hspace{0pt} \hyperlink{ecldoc:Date}{Date} \textbackslash 
\hspace{0pt} \hyperlink{ecldoc:date.togregorianymd}{ToGregorianYMD} \textbackslash 

{\renewcommand{\arraystretch}{1.5}
\begin{tabularx}{\textwidth}{|>{\raggedright\arraybackslash}l|X|}
\hline
\hspace{0pt}\mytexttt{\color{red} } & \textbf{day} \\
\hline
\end{tabularx}
}

\par





No Documentation Found








\par
\begin{description}
\item [\colorbox{tagtype}{\color{white} \textbf{\textsf{RETURN}}}] \textbf{INTEGER8} --- 
\end{description}




\rule{\linewidth}{0.5pt}


\subsection*{\textsf{\colorbox{headtoc}{\color{white} FUNCTION}
FromGregorianDate}}

\hypertarget{ecldoc:date.fromgregoriandate}{}
\hspace{0pt} \hyperlink{ecldoc:Date}{Date} \textbackslash 

{\renewcommand{\arraystretch}{1.5}
\begin{tabularx}{\textwidth}{|>{\raggedright\arraybackslash}l|X|}
\hline
\hspace{0pt}\mytexttt{\color{red} Days\_t} & \textbf{FromGregorianDate} \\
\hline
\multicolumn{2}{|>{\raggedright\arraybackslash}X|}{\hspace{0pt}\mytexttt{\color{param} (Date\_t date)}} \\
\hline
\end{tabularx}
}

\par





Converts a date in the Gregorian calendar to the number days since 31st December 1BC.






\par
\begin{description}
\item [\colorbox{tagtype}{\color{white} \textbf{\textsf{PARAMETER}}}] \textbf{\underline{date}} ||| UNSIGNED4 --- The date (using the Gregorian calendar)
\end{description}







\par
\begin{description}
\item [\colorbox{tagtype}{\color{white} \textbf{\textsf{RETURN}}}] \textbf{INTEGER4} --- The number of elapsed days (1 Jan 1AD = 1)
\end{description}




\rule{\linewidth}{0.5pt}
\subsection*{\textsf{\colorbox{headtoc}{\color{white} FUNCTION}
ToGregorianDate}}

\hypertarget{ecldoc:date.togregoriandate}{}
\hspace{0pt} \hyperlink{ecldoc:Date}{Date} \textbackslash 

{\renewcommand{\arraystretch}{1.5}
\begin{tabularx}{\textwidth}{|>{\raggedright\arraybackslash}l|X|}
\hline
\hspace{0pt}\mytexttt{\color{red} Date\_t} & \textbf{ToGregorianDate} \\
\hline
\multicolumn{2}{|>{\raggedright\arraybackslash}X|}{\hspace{0pt}\mytexttt{\color{param} (Days\_t days)}} \\
\hline
\end{tabularx}
}

\par





Converts the number days since 31st December 1BC to a date in the Gregorian calendar.






\par
\begin{description}
\item [\colorbox{tagtype}{\color{white} \textbf{\textsf{PARAMETER}}}] \textbf{\underline{days}} ||| INTEGER4 --- The number of elapsed days (1 Jan 1AD = 1)
\end{description}







\par
\begin{description}
\item [\colorbox{tagtype}{\color{white} \textbf{\textsf{RETURN}}}] \textbf{UNSIGNED4} --- A Date\_t in the Gregorian calendar
\end{description}




\rule{\linewidth}{0.5pt}
\subsection*{\textsf{\colorbox{headtoc}{\color{white} FUNCTION}
DayOfYear}}

\hypertarget{ecldoc:date.dayofyear}{}
\hspace{0pt} \hyperlink{ecldoc:Date}{Date} \textbackslash 

{\renewcommand{\arraystretch}{1.5}
\begin{tabularx}{\textwidth}{|>{\raggedright\arraybackslash}l|X|}
\hline
\hspace{0pt}\mytexttt{\color{red} UNSIGNED2} & \textbf{DayOfYear} \\
\hline
\multicolumn{2}{|>{\raggedright\arraybackslash}X|}{\hspace{0pt}\mytexttt{\color{param} (Date\_t date)}} \\
\hline
\end{tabularx}
}

\par





Returns a number representing the day of the year indicated by the given date. The date must be in the Gregorian calendar after the year 1600.






\par
\begin{description}
\item [\colorbox{tagtype}{\color{white} \textbf{\textsf{PARAMETER}}}] \textbf{\underline{date}} ||| UNSIGNED4 --- A Date\_t value.
\end{description}







\par
\begin{description}
\item [\colorbox{tagtype}{\color{white} \textbf{\textsf{RETURN}}}] \textbf{UNSIGNED2} --- A number (1-366) representing the number of days since the beginning of the year.
\end{description}




\rule{\linewidth}{0.5pt}
\subsection*{\textsf{\colorbox{headtoc}{\color{white} FUNCTION}
DayOfWeek}}

\hypertarget{ecldoc:date.dayofweek}{}
\hspace{0pt} \hyperlink{ecldoc:Date}{Date} \textbackslash 

{\renewcommand{\arraystretch}{1.5}
\begin{tabularx}{\textwidth}{|>{\raggedright\arraybackslash}l|X|}
\hline
\hspace{0pt}\mytexttt{\color{red} UNSIGNED1} & \textbf{DayOfWeek} \\
\hline
\multicolumn{2}{|>{\raggedright\arraybackslash}X|}{\hspace{0pt}\mytexttt{\color{param} (Date\_t date)}} \\
\hline
\end{tabularx}
}

\par





Returns a number representing the day of the week indicated by the given date. The date must be in the Gregorian calendar after the year 1600.






\par
\begin{description}
\item [\colorbox{tagtype}{\color{white} \textbf{\textsf{PARAMETER}}}] \textbf{\underline{date}} ||| UNSIGNED4 --- A Date\_t value.
\end{description}







\par
\begin{description}
\item [\colorbox{tagtype}{\color{white} \textbf{\textsf{RETURN}}}] \textbf{UNSIGNED1} --- A number 1-7 representing the day of the week, where 1 = Sunday.
\end{description}




\rule{\linewidth}{0.5pt}
\subsection*{\textsf{\colorbox{headtoc}{\color{white} FUNCTION}
IsJulianLeapYear}}

\hypertarget{ecldoc:date.isjulianleapyear}{}
\hspace{0pt} \hyperlink{ecldoc:Date}{Date} \textbackslash 

{\renewcommand{\arraystretch}{1.5}
\begin{tabularx}{\textwidth}{|>{\raggedright\arraybackslash}l|X|}
\hline
\hspace{0pt}\mytexttt{\color{red} BOOLEAN} & \textbf{IsJulianLeapYear} \\
\hline
\multicolumn{2}{|>{\raggedright\arraybackslash}X|}{\hspace{0pt}\mytexttt{\color{param} (INTEGER2 year)}} \\
\hline
\end{tabularx}
}

\par





Tests whether the year is a leap year in the Julian calendar.






\par
\begin{description}
\item [\colorbox{tagtype}{\color{white} \textbf{\textsf{PARAMETER}}}] \textbf{\underline{year}} ||| INTEGER2 --- The year (0-9999).
\end{description}







\par
\begin{description}
\item [\colorbox{tagtype}{\color{white} \textbf{\textsf{RETURN}}}] \textbf{BOOLEAN} --- True if the year is a leap year.
\end{description}




\rule{\linewidth}{0.5pt}
\subsection*{\textsf{\colorbox{headtoc}{\color{white} FUNCTION}
FromJulianYMD}}

\hypertarget{ecldoc:date.fromjulianymd}{}
\hspace{0pt} \hyperlink{ecldoc:Date}{Date} \textbackslash 

{\renewcommand{\arraystretch}{1.5}
\begin{tabularx}{\textwidth}{|>{\raggedright\arraybackslash}l|X|}
\hline
\hspace{0pt}\mytexttt{\color{red} Days\_t} & \textbf{FromJulianYMD} \\
\hline
\multicolumn{2}{|>{\raggedright\arraybackslash}X|}{\hspace{0pt}\mytexttt{\color{param} (INTEGER2 year, UNSIGNED1 month, UNSIGNED1 day)}} \\
\hline
\end{tabularx}
}

\par





Combines year, month, day in the Julian calendar to create the number days since 31st December 1BC.






\par
\begin{description}
\item [\colorbox{tagtype}{\color{white} \textbf{\textsf{PARAMETER}}}] \textbf{\underline{year}} ||| INTEGER2 --- The year (-4800..9999).
\item [\colorbox{tagtype}{\color{white} \textbf{\textsf{PARAMETER}}}] \textbf{\underline{month}} ||| UNSIGNED1 --- The month (1-12).
\item [\colorbox{tagtype}{\color{white} \textbf{\textsf{PARAMETER}}}] \textbf{\underline{day}} ||| UNSIGNED1 --- The day (1..daysInMonth).
\end{description}







\par
\begin{description}
\item [\colorbox{tagtype}{\color{white} \textbf{\textsf{RETURN}}}] \textbf{INTEGER4} --- The number of elapsed days (1 Jan 1AD = 1)
\end{description}




\rule{\linewidth}{0.5pt}
\subsection*{\textsf{\colorbox{headtoc}{\color{white} MODULE}
ToJulianYMD}}

\hypertarget{ecldoc:date.tojulianymd}{}
\hspace{0pt} \hyperlink{ecldoc:Date}{Date} \textbackslash 

{\renewcommand{\arraystretch}{1.5}
\begin{tabularx}{\textwidth}{|>{\raggedright\arraybackslash}l|X|}
\hline
\hspace{0pt}\mytexttt{\color{red} } & \textbf{ToJulianYMD} \\
\hline
\multicolumn{2}{|>{\raggedright\arraybackslash}X|}{\hspace{0pt}\mytexttt{\color{param} (Days\_t days)}} \\
\hline
\end{tabularx}
}

\par





Converts the number days since 31st December 1BC to a date in the Julian calendar.






\par
\begin{description}
\item [\colorbox{tagtype}{\color{white} \textbf{\textsf{PARAMETER}}}] \textbf{\underline{days}} ||| INTEGER4 --- The number of elapsed days (1 Jan 1AD = 1)
\end{description}







\par
\begin{description}
\item [\colorbox{tagtype}{\color{white} \textbf{\textsf{RETURN}}}] \textbf{} --- Module containing Year, Month, Day in the Julian calendar
\end{description}




\textbf{Children}
\begin{enumerate}
\item \hyperlink{ecldoc:date.tojulianymd.result.day}{Day}
: No Documentation Found
\item \hyperlink{ecldoc:date.tojulianymd.result.month}{Month}
: No Documentation Found
\item \hyperlink{ecldoc:date.tojulianymd.result.year}{Year}
: No Documentation Found
\end{enumerate}

\rule{\linewidth}{0.5pt}

\subsection*{\textsf{\colorbox{headtoc}{\color{white} ATTRIBUTE}
Day}}

\hypertarget{ecldoc:date.tojulianymd.result.day}{}
\hspace{0pt} \hyperlink{ecldoc:Date}{Date} \textbackslash 
\hspace{0pt} \hyperlink{ecldoc:date.tojulianymd}{ToJulianYMD} \textbackslash 

{\renewcommand{\arraystretch}{1.5}
\begin{tabularx}{\textwidth}{|>{\raggedright\arraybackslash}l|X|}
\hline
\hspace{0pt}\mytexttt{\color{red} UNSIGNED1} & \textbf{Day} \\
\hline
\end{tabularx}
}

\par





No Documentation Found








\par
\begin{description}
\item [\colorbox{tagtype}{\color{white} \textbf{\textsf{RETURN}}}] \textbf{UNSIGNED1} --- 
\end{description}




\rule{\linewidth}{0.5pt}
\subsection*{\textsf{\colorbox{headtoc}{\color{white} ATTRIBUTE}
Month}}

\hypertarget{ecldoc:date.tojulianymd.result.month}{}
\hspace{0pt} \hyperlink{ecldoc:Date}{Date} \textbackslash 
\hspace{0pt} \hyperlink{ecldoc:date.tojulianymd}{ToJulianYMD} \textbackslash 

{\renewcommand{\arraystretch}{1.5}
\begin{tabularx}{\textwidth}{|>{\raggedright\arraybackslash}l|X|}
\hline
\hspace{0pt}\mytexttt{\color{red} UNSIGNED1} & \textbf{Month} \\
\hline
\end{tabularx}
}

\par





No Documentation Found








\par
\begin{description}
\item [\colorbox{tagtype}{\color{white} \textbf{\textsf{RETURN}}}] \textbf{UNSIGNED1} --- 
\end{description}




\rule{\linewidth}{0.5pt}
\subsection*{\textsf{\colorbox{headtoc}{\color{white} ATTRIBUTE}
Year}}

\hypertarget{ecldoc:date.tojulianymd.result.year}{}
\hspace{0pt} \hyperlink{ecldoc:Date}{Date} \textbackslash 
\hspace{0pt} \hyperlink{ecldoc:date.tojulianymd}{ToJulianYMD} \textbackslash 

{\renewcommand{\arraystretch}{1.5}
\begin{tabularx}{\textwidth}{|>{\raggedright\arraybackslash}l|X|}
\hline
\hspace{0pt}\mytexttt{\color{red} INTEGER2} & \textbf{Year} \\
\hline
\end{tabularx}
}

\par





No Documentation Found








\par
\begin{description}
\item [\colorbox{tagtype}{\color{white} \textbf{\textsf{RETURN}}}] \textbf{INTEGER2} --- 
\end{description}




\rule{\linewidth}{0.5pt}


\subsection*{\textsf{\colorbox{headtoc}{\color{white} FUNCTION}
FromJulianDate}}

\hypertarget{ecldoc:date.fromjuliandate}{}
\hspace{0pt} \hyperlink{ecldoc:Date}{Date} \textbackslash 

{\renewcommand{\arraystretch}{1.5}
\begin{tabularx}{\textwidth}{|>{\raggedright\arraybackslash}l|X|}
\hline
\hspace{0pt}\mytexttt{\color{red} Days\_t} & \textbf{FromJulianDate} \\
\hline
\multicolumn{2}{|>{\raggedright\arraybackslash}X|}{\hspace{0pt}\mytexttt{\color{param} (Date\_t date)}} \\
\hline
\end{tabularx}
}

\par





Converts a date in the Julian calendar to the number days since 31st December 1BC.






\par
\begin{description}
\item [\colorbox{tagtype}{\color{white} \textbf{\textsf{PARAMETER}}}] \textbf{\underline{date}} ||| UNSIGNED4 --- The date (using the Julian calendar)
\end{description}







\par
\begin{description}
\item [\colorbox{tagtype}{\color{white} \textbf{\textsf{RETURN}}}] \textbf{INTEGER4} --- The number of elapsed days (1 Jan 1AD = 1)
\end{description}




\rule{\linewidth}{0.5pt}
\subsection*{\textsf{\colorbox{headtoc}{\color{white} FUNCTION}
ToJulianDate}}

\hypertarget{ecldoc:date.tojuliandate}{}
\hspace{0pt} \hyperlink{ecldoc:Date}{Date} \textbackslash 

{\renewcommand{\arraystretch}{1.5}
\begin{tabularx}{\textwidth}{|>{\raggedright\arraybackslash}l|X|}
\hline
\hspace{0pt}\mytexttt{\color{red} Date\_t} & \textbf{ToJulianDate} \\
\hline
\multicolumn{2}{|>{\raggedright\arraybackslash}X|}{\hspace{0pt}\mytexttt{\color{param} (Days\_t days)}} \\
\hline
\end{tabularx}
}

\par





Converts the number days since 31st December 1BC to a date in the Julian calendar.






\par
\begin{description}
\item [\colorbox{tagtype}{\color{white} \textbf{\textsf{PARAMETER}}}] \textbf{\underline{days}} ||| INTEGER4 --- The number of elapsed days (1 Jan 1AD = 1)
\end{description}







\par
\begin{description}
\item [\colorbox{tagtype}{\color{white} \textbf{\textsf{RETURN}}}] \textbf{UNSIGNED4} --- A Date\_t in the Julian calendar
\end{description}




\rule{\linewidth}{0.5pt}
\subsection*{\textsf{\colorbox{headtoc}{\color{white} FUNCTION}
DaysSince1900}}

\hypertarget{ecldoc:date.dayssince1900}{}
\hspace{0pt} \hyperlink{ecldoc:Date}{Date} \textbackslash 

{\renewcommand{\arraystretch}{1.5}
\begin{tabularx}{\textwidth}{|>{\raggedright\arraybackslash}l|X|}
\hline
\hspace{0pt}\mytexttt{\color{red} Days\_t} & \textbf{DaysSince1900} \\
\hline
\multicolumn{2}{|>{\raggedright\arraybackslash}X|}{\hspace{0pt}\mytexttt{\color{param} (INTEGER2 year, UNSIGNED1 month, UNSIGNED1 day)}} \\
\hline
\end{tabularx}
}

\par





Returns the number of days since 1st January 1900 (using the Gregorian Calendar)






\par
\begin{description}
\item [\colorbox{tagtype}{\color{white} \textbf{\textsf{PARAMETER}}}] \textbf{\underline{year}} ||| INTEGER2 --- The year (-4713..9999).
\item [\colorbox{tagtype}{\color{white} \textbf{\textsf{PARAMETER}}}] \textbf{\underline{month}} ||| UNSIGNED1 --- The month (1-12). A missing value (0) is treated as 1.
\item [\colorbox{tagtype}{\color{white} \textbf{\textsf{PARAMETER}}}] \textbf{\underline{day}} ||| UNSIGNED1 --- The day (1..daysInMonth). A missing value (0) is treated as 1.
\end{description}







\par
\begin{description}
\item [\colorbox{tagtype}{\color{white} \textbf{\textsf{RETURN}}}] \textbf{INTEGER4} --- The number of elapsed days since 1st January 1900
\end{description}




\rule{\linewidth}{0.5pt}
\subsection*{\textsf{\colorbox{headtoc}{\color{white} FUNCTION}
ToDaysSince1900}}

\hypertarget{ecldoc:date.todayssince1900}{}
\hspace{0pt} \hyperlink{ecldoc:Date}{Date} \textbackslash 

{\renewcommand{\arraystretch}{1.5}
\begin{tabularx}{\textwidth}{|>{\raggedright\arraybackslash}l|X|}
\hline
\hspace{0pt}\mytexttt{\color{red} Days\_t} & \textbf{ToDaysSince1900} \\
\hline
\multicolumn{2}{|>{\raggedright\arraybackslash}X|}{\hspace{0pt}\mytexttt{\color{param} (Date\_t date)}} \\
\hline
\end{tabularx}
}

\par





Returns the number of days since 1st January 1900 (using the Gregorian Calendar)






\par
\begin{description}
\item [\colorbox{tagtype}{\color{white} \textbf{\textsf{PARAMETER}}}] \textbf{\underline{date}} ||| UNSIGNED4 --- The date
\end{description}







\par
\begin{description}
\item [\colorbox{tagtype}{\color{white} \textbf{\textsf{RETURN}}}] \textbf{INTEGER4} --- The number of elapsed days since 1st January 1900
\end{description}




\rule{\linewidth}{0.5pt}
\subsection*{\textsf{\colorbox{headtoc}{\color{white} FUNCTION}
FromDaysSince1900}}

\hypertarget{ecldoc:date.fromdayssince1900}{}
\hspace{0pt} \hyperlink{ecldoc:Date}{Date} \textbackslash 

{\renewcommand{\arraystretch}{1.5}
\begin{tabularx}{\textwidth}{|>{\raggedright\arraybackslash}l|X|}
\hline
\hspace{0pt}\mytexttt{\color{red} Date\_t} & \textbf{FromDaysSince1900} \\
\hline
\multicolumn{2}{|>{\raggedright\arraybackslash}X|}{\hspace{0pt}\mytexttt{\color{param} (Days\_t days)}} \\
\hline
\end{tabularx}
}

\par





Converts the number days since 1st January 1900 to a date in the Julian calendar.






\par
\begin{description}
\item [\colorbox{tagtype}{\color{white} \textbf{\textsf{PARAMETER}}}] \textbf{\underline{days}} ||| INTEGER4 --- The number of elapsed days since 1st Jan 1900
\end{description}







\par
\begin{description}
\item [\colorbox{tagtype}{\color{white} \textbf{\textsf{RETURN}}}] \textbf{UNSIGNED4} --- A Date\_t in the Julian calendar
\end{description}




\rule{\linewidth}{0.5pt}
\subsection*{\textsf{\colorbox{headtoc}{\color{white} FUNCTION}
YearsBetween}}

\hypertarget{ecldoc:date.yearsbetween}{}
\hspace{0pt} \hyperlink{ecldoc:Date}{Date} \textbackslash 

{\renewcommand{\arraystretch}{1.5}
\begin{tabularx}{\textwidth}{|>{\raggedright\arraybackslash}l|X|}
\hline
\hspace{0pt}\mytexttt{\color{red} INTEGER} & \textbf{YearsBetween} \\
\hline
\multicolumn{2}{|>{\raggedright\arraybackslash}X|}{\hspace{0pt}\mytexttt{\color{param} (Date\_t from, Date\_t to)}} \\
\hline
\end{tabularx}
}

\par





Calculate the number of whole years between two dates.






\par
\begin{description}
\item [\colorbox{tagtype}{\color{white} \textbf{\textsf{PARAMETER}}}] \textbf{\underline{from}} ||| UNSIGNED4 --- The first date
\item [\colorbox{tagtype}{\color{white} \textbf{\textsf{PARAMETER}}}] \textbf{\underline{to}} ||| UNSIGNED4 --- The last date
\end{description}







\par
\begin{description}
\item [\colorbox{tagtype}{\color{white} \textbf{\textsf{RETURN}}}] \textbf{INTEGER8} --- The number of years between them.
\end{description}




\rule{\linewidth}{0.5pt}
\subsection*{\textsf{\colorbox{headtoc}{\color{white} FUNCTION}
MonthsBetween}}

\hypertarget{ecldoc:date.monthsbetween}{}
\hspace{0pt} \hyperlink{ecldoc:Date}{Date} \textbackslash 

{\renewcommand{\arraystretch}{1.5}
\begin{tabularx}{\textwidth}{|>{\raggedright\arraybackslash}l|X|}
\hline
\hspace{0pt}\mytexttt{\color{red} INTEGER} & \textbf{MonthsBetween} \\
\hline
\multicolumn{2}{|>{\raggedright\arraybackslash}X|}{\hspace{0pt}\mytexttt{\color{param} (Date\_t from, Date\_t to)}} \\
\hline
\end{tabularx}
}

\par





Calculate the number of whole months between two dates.






\par
\begin{description}
\item [\colorbox{tagtype}{\color{white} \textbf{\textsf{PARAMETER}}}] \textbf{\underline{from}} ||| UNSIGNED4 --- The first date
\item [\colorbox{tagtype}{\color{white} \textbf{\textsf{PARAMETER}}}] \textbf{\underline{to}} ||| UNSIGNED4 --- The last date
\end{description}







\par
\begin{description}
\item [\colorbox{tagtype}{\color{white} \textbf{\textsf{RETURN}}}] \textbf{INTEGER8} --- The number of months between them.
\end{description}




\rule{\linewidth}{0.5pt}
\subsection*{\textsf{\colorbox{headtoc}{\color{white} FUNCTION}
DaysBetween}}

\hypertarget{ecldoc:date.daysbetween}{}
\hspace{0pt} \hyperlink{ecldoc:Date}{Date} \textbackslash 

{\renewcommand{\arraystretch}{1.5}
\begin{tabularx}{\textwidth}{|>{\raggedright\arraybackslash}l|X|}
\hline
\hspace{0pt}\mytexttt{\color{red} INTEGER} & \textbf{DaysBetween} \\
\hline
\multicolumn{2}{|>{\raggedright\arraybackslash}X|}{\hspace{0pt}\mytexttt{\color{param} (Date\_t from, Date\_t to)}} \\
\hline
\end{tabularx}
}

\par





Calculate the number of days between two dates.






\par
\begin{description}
\item [\colorbox{tagtype}{\color{white} \textbf{\textsf{PARAMETER}}}] \textbf{\underline{from}} ||| UNSIGNED4 --- The first date
\item [\colorbox{tagtype}{\color{white} \textbf{\textsf{PARAMETER}}}] \textbf{\underline{to}} ||| UNSIGNED4 --- The last date
\end{description}







\par
\begin{description}
\item [\colorbox{tagtype}{\color{white} \textbf{\textsf{RETURN}}}] \textbf{INTEGER8} --- The number of days between them.
\end{description}




\rule{\linewidth}{0.5pt}
\subsection*{\textsf{\colorbox{headtoc}{\color{white} FUNCTION}
DateFromDateRec}}

\hypertarget{ecldoc:date.datefromdaterec}{}
\hspace{0pt} \hyperlink{ecldoc:Date}{Date} \textbackslash 

{\renewcommand{\arraystretch}{1.5}
\begin{tabularx}{\textwidth}{|>{\raggedright\arraybackslash}l|X|}
\hline
\hspace{0pt}\mytexttt{\color{red} Date\_t} & \textbf{DateFromDateRec} \\
\hline
\multicolumn{2}{|>{\raggedright\arraybackslash}X|}{\hspace{0pt}\mytexttt{\color{param} (Date\_rec date)}} \\
\hline
\end{tabularx}
}

\par





Combines the fields from a Date\_rec to create a Date\_t






\par
\begin{description}
\item [\colorbox{tagtype}{\color{white} \textbf{\textsf{PARAMETER}}}] \textbf{\underline{date}} ||| ROW ( Date\_rec ) --- The row containing the date.
\end{description}







\par
\begin{description}
\item [\colorbox{tagtype}{\color{white} \textbf{\textsf{RETURN}}}] \textbf{UNSIGNED4} --- A Date\_t representing the combined values.
\end{description}




\rule{\linewidth}{0.5pt}
\subsection*{\textsf{\colorbox{headtoc}{\color{white} FUNCTION}
DateFromRec}}

\hypertarget{ecldoc:date.datefromrec}{}
\hspace{0pt} \hyperlink{ecldoc:Date}{Date} \textbackslash 

{\renewcommand{\arraystretch}{1.5}
\begin{tabularx}{\textwidth}{|>{\raggedright\arraybackslash}l|X|}
\hline
\hspace{0pt}\mytexttt{\color{red} Date\_t} & \textbf{DateFromRec} \\
\hline
\multicolumn{2}{|>{\raggedright\arraybackslash}X|}{\hspace{0pt}\mytexttt{\color{param} (Date\_rec date)}} \\
\hline
\end{tabularx}
}

\par





Combines the fields from a Date\_rec to create a Date\_t






\par
\begin{description}
\item [\colorbox{tagtype}{\color{white} \textbf{\textsf{PARAMETER}}}] \textbf{\underline{date}} ||| ROW ( Date\_rec ) --- The row containing the date.
\end{description}







\par
\begin{description}
\item [\colorbox{tagtype}{\color{white} \textbf{\textsf{RETURN}}}] \textbf{UNSIGNED4} --- A Date\_t representing the combined values.
\end{description}




\rule{\linewidth}{0.5pt}
\subsection*{\textsf{\colorbox{headtoc}{\color{white} FUNCTION}
TimeFromTimeRec}}

\hypertarget{ecldoc:date.timefromtimerec}{}
\hspace{0pt} \hyperlink{ecldoc:Date}{Date} \textbackslash 

{\renewcommand{\arraystretch}{1.5}
\begin{tabularx}{\textwidth}{|>{\raggedright\arraybackslash}l|X|}
\hline
\hspace{0pt}\mytexttt{\color{red} Time\_t} & \textbf{TimeFromTimeRec} \\
\hline
\multicolumn{2}{|>{\raggedright\arraybackslash}X|}{\hspace{0pt}\mytexttt{\color{param} (Time\_rec time)}} \\
\hline
\end{tabularx}
}

\par





Combines the fields from a Time\_rec to create a Time\_t






\par
\begin{description}
\item [\colorbox{tagtype}{\color{white} \textbf{\textsf{PARAMETER}}}] \textbf{\underline{time}} ||| ROW ( Time\_rec ) --- The row containing the time.
\end{description}







\par
\begin{description}
\item [\colorbox{tagtype}{\color{white} \textbf{\textsf{RETURN}}}] \textbf{UNSIGNED3} --- A Time\_t representing the combined values.
\end{description}




\rule{\linewidth}{0.5pt}
\subsection*{\textsf{\colorbox{headtoc}{\color{white} FUNCTION}
DateFromDateTimeRec}}

\hypertarget{ecldoc:date.datefromdatetimerec}{}
\hspace{0pt} \hyperlink{ecldoc:Date}{Date} \textbackslash 

{\renewcommand{\arraystretch}{1.5}
\begin{tabularx}{\textwidth}{|>{\raggedright\arraybackslash}l|X|}
\hline
\hspace{0pt}\mytexttt{\color{red} Date\_t} & \textbf{DateFromDateTimeRec} \\
\hline
\multicolumn{2}{|>{\raggedright\arraybackslash}X|}{\hspace{0pt}\mytexttt{\color{param} (DateTime\_rec datetime)}} \\
\hline
\end{tabularx}
}

\par





Combines the date fields from a DateTime\_rec to create a Date\_t






\par
\begin{description}
\item [\colorbox{tagtype}{\color{white} \textbf{\textsf{PARAMETER}}}] \textbf{\underline{datetime}} ||| ROW ( DateTime\_rec ) --- The row containing the datetime.
\end{description}







\par
\begin{description}
\item [\colorbox{tagtype}{\color{white} \textbf{\textsf{RETURN}}}] \textbf{UNSIGNED4} --- A Date\_t representing the combined values.
\end{description}




\rule{\linewidth}{0.5pt}
\subsection*{\textsf{\colorbox{headtoc}{\color{white} FUNCTION}
TimeFromDateTimeRec}}

\hypertarget{ecldoc:date.timefromdatetimerec}{}
\hspace{0pt} \hyperlink{ecldoc:Date}{Date} \textbackslash 

{\renewcommand{\arraystretch}{1.5}
\begin{tabularx}{\textwidth}{|>{\raggedright\arraybackslash}l|X|}
\hline
\hspace{0pt}\mytexttt{\color{red} Time\_t} & \textbf{TimeFromDateTimeRec} \\
\hline
\multicolumn{2}{|>{\raggedright\arraybackslash}X|}{\hspace{0pt}\mytexttt{\color{param} (DateTime\_rec datetime)}} \\
\hline
\end{tabularx}
}

\par





Combines the time fields from a DateTime\_rec to create a Time\_t






\par
\begin{description}
\item [\colorbox{tagtype}{\color{white} \textbf{\textsf{PARAMETER}}}] \textbf{\underline{datetime}} ||| ROW ( DateTime\_rec ) --- The row containing the datetime.
\end{description}







\par
\begin{description}
\item [\colorbox{tagtype}{\color{white} \textbf{\textsf{RETURN}}}] \textbf{UNSIGNED3} --- A Time\_t representing the combined values.
\end{description}




\rule{\linewidth}{0.5pt}
\subsection*{\textsf{\colorbox{headtoc}{\color{white} FUNCTION}
SecondsFromDateTimeRec}}

\hypertarget{ecldoc:date.secondsfromdatetimerec}{}
\hspace{0pt} \hyperlink{ecldoc:Date}{Date} \textbackslash 

{\renewcommand{\arraystretch}{1.5}
\begin{tabularx}{\textwidth}{|>{\raggedright\arraybackslash}l|X|}
\hline
\hspace{0pt}\mytexttt{\color{red} Seconds\_t} & \textbf{SecondsFromDateTimeRec} \\
\hline
\multicolumn{2}{|>{\raggedright\arraybackslash}X|}{\hspace{0pt}\mytexttt{\color{param} (DateTime\_rec datetime, BOOLEAN is\_local\_time = FALSE)}} \\
\hline
\end{tabularx}
}

\par





Combines the date and time fields from a DateTime\_rec to create a Seconds\_t






\par
\begin{description}
\item [\colorbox{tagtype}{\color{white} \textbf{\textsf{PARAMETER}}}] \textbf{\underline{datetime}} ||| ROW ( DateTime\_rec ) --- The row containing the datetime.
\item [\colorbox{tagtype}{\color{white} \textbf{\textsf{PARAMETER}}}] \textbf{\underline{is\_local\_time}} ||| BOOLEAN --- TRUE if the datetime components are expressed in local time rather than UTC, FALSE if the components are expressed in UTC. Optional, defaults to FALSE.
\end{description}







\par
\begin{description}
\item [\colorbox{tagtype}{\color{white} \textbf{\textsf{RETURN}}}] \textbf{INTEGER8} --- A Seconds\_t representing the combined values.
\end{description}




\rule{\linewidth}{0.5pt}
\subsection*{\textsf{\colorbox{headtoc}{\color{white} FUNCTION}
FromStringToDate}}

\hypertarget{ecldoc:date.fromstringtodate}{}
\hspace{0pt} \hyperlink{ecldoc:Date}{Date} \textbackslash 

{\renewcommand{\arraystretch}{1.5}
\begin{tabularx}{\textwidth}{|>{\raggedright\arraybackslash}l|X|}
\hline
\hspace{0pt}\mytexttt{\color{red} Date\_t} & \textbf{FromStringToDate} \\
\hline
\multicolumn{2}{|>{\raggedright\arraybackslash}X|}{\hspace{0pt}\mytexttt{\color{param} (STRING date\_text, VARSTRING format)}} \\
\hline
\end{tabularx}
}

\par





Converts a string to a Date\_t using the relevant string format. The resulting date must be representable within the Gregorian calendar after the year 1600.






\par
\begin{description}
\item [\colorbox{tagtype}{\color{white} \textbf{\textsf{PARAMETER}}}] \textbf{\underline{date\_text}} ||| STRING --- The string to be converted.
\item [\colorbox{tagtype}{\color{white} \textbf{\textsf{PARAMETER}}}] \textbf{\underline{format}} ||| VARSTRING --- The format of the input string. (See documentation for strftime)
\end{description}







\par
\begin{description}
\item [\colorbox{tagtype}{\color{white} \textbf{\textsf{RETURN}}}] \textbf{UNSIGNED4} --- The date that was matched in the string. Returns 0 if failed to match or if the date components match but the result is an invalid date. Supported characters: \%B Full month name \%b or \%h Abbreviated month name \%d Day of month (two digits) \%e Day of month (two digits, or a space followed by a single digit) \%m Month (two digits) \%t Whitespace \%y year within century (00-99) \%Y Full year (yyyy) \%j Julian day (1-366) Common date formats American '\%m/\%d/\%Y' mm/dd/yyyy Euro '\%d/\%m/\%Y' dd/mm/yyyy Iso format '\%Y-\%m-\%d' yyyy-mm-dd Iso basic 'Y\%m\%d' yyyymmdd '\%d-\%b-\%Y' dd-mon-yyyy e.g., '21-Mar-1954'
\end{description}




\rule{\linewidth}{0.5pt}
\subsection*{\textsf{\colorbox{headtoc}{\color{white} FUNCTION}
FromString}}

\hypertarget{ecldoc:date.fromstring}{}
\hspace{0pt} \hyperlink{ecldoc:Date}{Date} \textbackslash 

{\renewcommand{\arraystretch}{1.5}
\begin{tabularx}{\textwidth}{|>{\raggedright\arraybackslash}l|X|}
\hline
\hspace{0pt}\mytexttt{\color{red} Date\_t} & \textbf{FromString} \\
\hline
\multicolumn{2}{|>{\raggedright\arraybackslash}X|}{\hspace{0pt}\mytexttt{\color{param} (STRING date\_text, VARSTRING format)}} \\
\hline
\end{tabularx}
}

\par





Converts a string to a date using the relevant string format.






\par
\begin{description}
\item [\colorbox{tagtype}{\color{white} \textbf{\textsf{PARAMETER}}}] \textbf{\underline{date\_text}} ||| STRING --- The string to be converted.
\item [\colorbox{tagtype}{\color{white} \textbf{\textsf{PARAMETER}}}] \textbf{\underline{format}} ||| VARSTRING --- The format of the input string. (See documentation for strftime)
\end{description}







\par
\begin{description}
\item [\colorbox{tagtype}{\color{white} \textbf{\textsf{RETURN}}}] \textbf{UNSIGNED4} --- The date that was matched in the string. Returns 0 if failed to match.
\end{description}




\rule{\linewidth}{0.5pt}
\subsection*{\textsf{\colorbox{headtoc}{\color{white} FUNCTION}
FromStringToTime}}

\hypertarget{ecldoc:date.fromstringtotime}{}
\hspace{0pt} \hyperlink{ecldoc:Date}{Date} \textbackslash 

{\renewcommand{\arraystretch}{1.5}
\begin{tabularx}{\textwidth}{|>{\raggedright\arraybackslash}l|X|}
\hline
\hspace{0pt}\mytexttt{\color{red} Time\_t} & \textbf{FromStringToTime} \\
\hline
\multicolumn{2}{|>{\raggedright\arraybackslash}X|}{\hspace{0pt}\mytexttt{\color{param} (STRING time\_text, VARSTRING format)}} \\
\hline
\end{tabularx}
}

\par





Converts a string to a Time\_t using the relevant string format.






\par
\begin{description}
\item [\colorbox{tagtype}{\color{white} \textbf{\textsf{PARAMETER}}}] \textbf{\underline{date\_text}} |||  --- The string to be converted.
\item [\colorbox{tagtype}{\color{white} \textbf{\textsf{PARAMETER}}}] \textbf{\underline{format}} ||| VARSTRING --- The format of the input string. (See documentation for strftime)
\item [\colorbox{tagtype}{\color{white} \textbf{\textsf{PARAMETER}}}] \textbf{\underline{time\_text}} ||| STRING --- No Doc
\end{description}







\par
\begin{description}
\item [\colorbox{tagtype}{\color{white} \textbf{\textsf{RETURN}}}] \textbf{UNSIGNED3} --- The time that was matched in the string. Returns 0 if failed to match. Supported characters: \%H Hour (two digits) \%k (two digits, or a space followed by a single digit) \%M Minute (two digits) \%S Second (two digits) \%t Whitespace
\end{description}




\rule{\linewidth}{0.5pt}
\subsection*{\textsf{\colorbox{headtoc}{\color{white} FUNCTION}
MatchDateString}}

\hypertarget{ecldoc:date.matchdatestring}{}
\hspace{0pt} \hyperlink{ecldoc:Date}{Date} \textbackslash 

{\renewcommand{\arraystretch}{1.5}
\begin{tabularx}{\textwidth}{|>{\raggedright\arraybackslash}l|X|}
\hline
\hspace{0pt}\mytexttt{\color{red} Date\_t} & \textbf{MatchDateString} \\
\hline
\multicolumn{2}{|>{\raggedright\arraybackslash}X|}{\hspace{0pt}\mytexttt{\color{param} (STRING date\_text, SET OF VARSTRING formats)}} \\
\hline
\end{tabularx}
}

\par





Matches a string against a set of date string formats and returns a valid Date\_t object from the first format that successfully parses the string.






\par
\begin{description}
\item [\colorbox{tagtype}{\color{white} \textbf{\textsf{PARAMETER}}}] \textbf{\underline{formats}} ||| SET ( VARSTRING ) --- A set of formats to check against the string. (See documentation for strftime)
\item [\colorbox{tagtype}{\color{white} \textbf{\textsf{PARAMETER}}}] \textbf{\underline{date\_text}} ||| STRING --- The string to be converted.
\end{description}







\par
\begin{description}
\item [\colorbox{tagtype}{\color{white} \textbf{\textsf{RETURN}}}] \textbf{UNSIGNED4} --- The date that was matched in the string. Returns 0 if failed to match.
\end{description}




\rule{\linewidth}{0.5pt}
\subsection*{\textsf{\colorbox{headtoc}{\color{white} FUNCTION}
MatchTimeString}}

\hypertarget{ecldoc:date.matchtimestring}{}
\hspace{0pt} \hyperlink{ecldoc:Date}{Date} \textbackslash 

{\renewcommand{\arraystretch}{1.5}
\begin{tabularx}{\textwidth}{|>{\raggedright\arraybackslash}l|X|}
\hline
\hspace{0pt}\mytexttt{\color{red} Time\_t} & \textbf{MatchTimeString} \\
\hline
\multicolumn{2}{|>{\raggedright\arraybackslash}X|}{\hspace{0pt}\mytexttt{\color{param} (STRING time\_text, SET OF VARSTRING formats)}} \\
\hline
\end{tabularx}
}

\par





Matches a string against a set of time string formats and returns a valid Time\_t object from the first format that successfully parses the string.






\par
\begin{description}
\item [\colorbox{tagtype}{\color{white} \textbf{\textsf{PARAMETER}}}] \textbf{\underline{formats}} ||| SET ( VARSTRING ) --- A set of formats to check against the string. (See documentation for strftime)
\item [\colorbox{tagtype}{\color{white} \textbf{\textsf{PARAMETER}}}] \textbf{\underline{time\_text}} ||| STRING --- The string to be converted.
\end{description}







\par
\begin{description}
\item [\colorbox{tagtype}{\color{white} \textbf{\textsf{RETURN}}}] \textbf{UNSIGNED3} --- The time that was matched in the string. Returns 0 if failed to match.
\end{description}




\rule{\linewidth}{0.5pt}
\subsection*{\textsf{\colorbox{headtoc}{\color{white} FUNCTION}
DateToString}}

\hypertarget{ecldoc:date.datetostring}{}
\hspace{0pt} \hyperlink{ecldoc:Date}{Date} \textbackslash 

{\renewcommand{\arraystretch}{1.5}
\begin{tabularx}{\textwidth}{|>{\raggedright\arraybackslash}l|X|}
\hline
\hspace{0pt}\mytexttt{\color{red} STRING} & \textbf{DateToString} \\
\hline
\multicolumn{2}{|>{\raggedright\arraybackslash}X|}{\hspace{0pt}\mytexttt{\color{param} (Date\_t date, VARSTRING format = '\%Y-\%m-\%d')}} \\
\hline
\end{tabularx}
}

\par





Formats a date as a string.






\par
\begin{description}
\item [\colorbox{tagtype}{\color{white} \textbf{\textsf{PARAMETER}}}] \textbf{\underline{date}} ||| UNSIGNED4 --- The date to be converted.
\item [\colorbox{tagtype}{\color{white} \textbf{\textsf{PARAMETER}}}] \textbf{\underline{format}} ||| VARSTRING --- The format template to use for the conversion; see strftime() for appropriate values. The maximum length of the resulting string is 255 characters. Optional; defaults to '\%Y-\%m-\%d' which is YYYY-MM-DD.
\end{description}







\par
\begin{description}
\item [\colorbox{tagtype}{\color{white} \textbf{\textsf{RETURN}}}] \textbf{STRING} --- Blank if date cannot be formatted, or the date in the requested format.
\end{description}




\rule{\linewidth}{0.5pt}
\subsection*{\textsf{\colorbox{headtoc}{\color{white} FUNCTION}
TimeToString}}

\hypertarget{ecldoc:date.timetostring}{}
\hspace{0pt} \hyperlink{ecldoc:Date}{Date} \textbackslash 

{\renewcommand{\arraystretch}{1.5}
\begin{tabularx}{\textwidth}{|>{\raggedright\arraybackslash}l|X|}
\hline
\hspace{0pt}\mytexttt{\color{red} STRING} & \textbf{TimeToString} \\
\hline
\multicolumn{2}{|>{\raggedright\arraybackslash}X|}{\hspace{0pt}\mytexttt{\color{param} (Time\_t time, VARSTRING format = '\%H:\%M:\%S')}} \\
\hline
\end{tabularx}
}

\par





Formats a time as a string.






\par
\begin{description}
\item [\colorbox{tagtype}{\color{white} \textbf{\textsf{PARAMETER}}}] \textbf{\underline{time}} ||| UNSIGNED3 --- The time to be converted.
\item [\colorbox{tagtype}{\color{white} \textbf{\textsf{PARAMETER}}}] \textbf{\underline{format}} ||| VARSTRING --- The format template to use for the conversion; see strftime() for appropriate values. The maximum length of the resulting string is 255 characters. Optional; defaults to '\%H:\%M:\%S' which is HH:MM:SS.
\end{description}







\par
\begin{description}
\item [\colorbox{tagtype}{\color{white} \textbf{\textsf{RETURN}}}] \textbf{STRING} --- Blank if the time cannot be formatted, or the time in the requested format.
\end{description}




\rule{\linewidth}{0.5pt}
\subsection*{\textsf{\colorbox{headtoc}{\color{white} FUNCTION}
SecondsToString}}

\hypertarget{ecldoc:date.secondstostring}{}
\hspace{0pt} \hyperlink{ecldoc:Date}{Date} \textbackslash 

{\renewcommand{\arraystretch}{1.5}
\begin{tabularx}{\textwidth}{|>{\raggedright\arraybackslash}l|X|}
\hline
\hspace{0pt}\mytexttt{\color{red} STRING} & \textbf{SecondsToString} \\
\hline
\multicolumn{2}{|>{\raggedright\arraybackslash}X|}{\hspace{0pt}\mytexttt{\color{param} (Seconds\_t seconds, VARSTRING format = '\%Y-\%m-\%dT\%H:\%M:\%S')}} \\
\hline
\end{tabularx}
}

\par





Converts a Seconds\_t value into a human-readable string using a format template.






\par
\begin{description}
\item [\colorbox{tagtype}{\color{white} \textbf{\textsf{PARAMETER}}}] \textbf{\underline{format}} ||| VARSTRING --- The format template to use for the conversion; see strftime() for appropriate values. The maximum length of the resulting string is 255 characters. Optional; defaults to '\%Y-\%m-\%dT\%H:\%M:\%S' which is YYYY-MM-DDTHH:MM:SS.
\item [\colorbox{tagtype}{\color{white} \textbf{\textsf{PARAMETER}}}] \textbf{\underline{seconds}} ||| INTEGER8 --- The seconds since epoch.
\end{description}







\par
\begin{description}
\item [\colorbox{tagtype}{\color{white} \textbf{\textsf{RETURN}}}] \textbf{STRING} --- The converted seconds as a string.
\end{description}




\rule{\linewidth}{0.5pt}
\subsection*{\textsf{\colorbox{headtoc}{\color{white} FUNCTION}
ToString}}

\hypertarget{ecldoc:date.tostring}{}
\hspace{0pt} \hyperlink{ecldoc:Date}{Date} \textbackslash 

{\renewcommand{\arraystretch}{1.5}
\begin{tabularx}{\textwidth}{|>{\raggedright\arraybackslash}l|X|}
\hline
\hspace{0pt}\mytexttt{\color{red} STRING} & \textbf{ToString} \\
\hline
\multicolumn{2}{|>{\raggedright\arraybackslash}X|}{\hspace{0pt}\mytexttt{\color{param} (Date\_t date, VARSTRING format)}} \\
\hline
\end{tabularx}
}

\par





Formats a date as a string.






\par
\begin{description}
\item [\colorbox{tagtype}{\color{white} \textbf{\textsf{PARAMETER}}}] \textbf{\underline{date}} ||| UNSIGNED4 --- The date to be converted.
\item [\colorbox{tagtype}{\color{white} \textbf{\textsf{PARAMETER}}}] \textbf{\underline{format}} ||| VARSTRING --- The format the date is output in. (See documentation for strftime)
\end{description}







\par
\begin{description}
\item [\colorbox{tagtype}{\color{white} \textbf{\textsf{RETURN}}}] \textbf{STRING} --- Blank if date cannot be formatted, or the date in the requested format.
\end{description}




\rule{\linewidth}{0.5pt}
\subsection*{\textsf{\colorbox{headtoc}{\color{white} FUNCTION}
ConvertDateFormat}}

\hypertarget{ecldoc:date.convertdateformat}{}
\hspace{0pt} \hyperlink{ecldoc:Date}{Date} \textbackslash 

{\renewcommand{\arraystretch}{1.5}
\begin{tabularx}{\textwidth}{|>{\raggedright\arraybackslash}l|X|}
\hline
\hspace{0pt}\mytexttt{\color{red} STRING} & \textbf{ConvertDateFormat} \\
\hline
\multicolumn{2}{|>{\raggedright\arraybackslash}X|}{\hspace{0pt}\mytexttt{\color{param} (STRING date\_text, VARSTRING from\_format='\%m/\%d/\%Y', VARSTRING to\_format='\%Y\%m\%d')}} \\
\hline
\end{tabularx}
}

\par





Converts a date from one format to another






\par
\begin{description}
\item [\colorbox{tagtype}{\color{white} \textbf{\textsf{PARAMETER}}}] \textbf{\underline{from\_format}} ||| VARSTRING --- The format the date is to be converted from.
\item [\colorbox{tagtype}{\color{white} \textbf{\textsf{PARAMETER}}}] \textbf{\underline{date\_text}} ||| STRING --- The string containing the date to be converted.
\item [\colorbox{tagtype}{\color{white} \textbf{\textsf{PARAMETER}}}] \textbf{\underline{to\_format}} ||| VARSTRING --- The format the date is to be converted to.
\end{description}







\par
\begin{description}
\item [\colorbox{tagtype}{\color{white} \textbf{\textsf{RETURN}}}] \textbf{STRING} --- The converted string, or blank if it failed to match the format.
\end{description}




\rule{\linewidth}{0.5pt}
\subsection*{\textsf{\colorbox{headtoc}{\color{white} FUNCTION}
ConvertFormat}}

\hypertarget{ecldoc:date.convertformat}{}
\hspace{0pt} \hyperlink{ecldoc:Date}{Date} \textbackslash 

{\renewcommand{\arraystretch}{1.5}
\begin{tabularx}{\textwidth}{|>{\raggedright\arraybackslash}l|X|}
\hline
\hspace{0pt}\mytexttt{\color{red} STRING} & \textbf{ConvertFormat} \\
\hline
\multicolumn{2}{|>{\raggedright\arraybackslash}X|}{\hspace{0pt}\mytexttt{\color{param} (STRING date\_text, VARSTRING from\_format='\%m/\%d/\%Y', VARSTRING to\_format='\%Y\%m\%d')}} \\
\hline
\end{tabularx}
}

\par





Converts a date from one format to another






\par
\begin{description}
\item [\colorbox{tagtype}{\color{white} \textbf{\textsf{PARAMETER}}}] \textbf{\underline{from\_format}} ||| VARSTRING --- The format the date is to be converted from.
\item [\colorbox{tagtype}{\color{white} \textbf{\textsf{PARAMETER}}}] \textbf{\underline{date\_text}} ||| STRING --- The string containing the date to be converted.
\item [\colorbox{tagtype}{\color{white} \textbf{\textsf{PARAMETER}}}] \textbf{\underline{to\_format}} ||| VARSTRING --- The format the date is to be converted to.
\end{description}







\par
\begin{description}
\item [\colorbox{tagtype}{\color{white} \textbf{\textsf{RETURN}}}] \textbf{STRING} --- The converted string, or blank if it failed to match the format.
\end{description}




\rule{\linewidth}{0.5pt}
\subsection*{\textsf{\colorbox{headtoc}{\color{white} FUNCTION}
ConvertTimeFormat}}

\hypertarget{ecldoc:date.converttimeformat}{}
\hspace{0pt} \hyperlink{ecldoc:Date}{Date} \textbackslash 

{\renewcommand{\arraystretch}{1.5}
\begin{tabularx}{\textwidth}{|>{\raggedright\arraybackslash}l|X|}
\hline
\hspace{0pt}\mytexttt{\color{red} STRING} & \textbf{ConvertTimeFormat} \\
\hline
\multicolumn{2}{|>{\raggedright\arraybackslash}X|}{\hspace{0pt}\mytexttt{\color{param} (STRING time\_text, VARSTRING from\_format='\%H\%M\%S', VARSTRING to\_format='\%H:\%M:\%S')}} \\
\hline
\end{tabularx}
}

\par





Converts a time from one format to another






\par
\begin{description}
\item [\colorbox{tagtype}{\color{white} \textbf{\textsf{PARAMETER}}}] \textbf{\underline{from\_format}} ||| VARSTRING --- The format the time is to be converted from.
\item [\colorbox{tagtype}{\color{white} \textbf{\textsf{PARAMETER}}}] \textbf{\underline{to\_format}} ||| VARSTRING --- The format the time is to be converted to.
\item [\colorbox{tagtype}{\color{white} \textbf{\textsf{PARAMETER}}}] \textbf{\underline{time\_text}} ||| STRING --- The string containing the time to be converted.
\end{description}







\par
\begin{description}
\item [\colorbox{tagtype}{\color{white} \textbf{\textsf{RETURN}}}] \textbf{STRING} --- The converted string, or blank if it failed to match the format.
\end{description}




\rule{\linewidth}{0.5pt}
\subsection*{\textsf{\colorbox{headtoc}{\color{white} FUNCTION}
ConvertDateFormatMultiple}}

\hypertarget{ecldoc:date.convertdateformatmultiple}{}
\hspace{0pt} \hyperlink{ecldoc:Date}{Date} \textbackslash 

{\renewcommand{\arraystretch}{1.5}
\begin{tabularx}{\textwidth}{|>{\raggedright\arraybackslash}l|X|}
\hline
\hspace{0pt}\mytexttt{\color{red} STRING} & \textbf{ConvertDateFormatMultiple} \\
\hline
\multicolumn{2}{|>{\raggedright\arraybackslash}X|}{\hspace{0pt}\mytexttt{\color{param} (STRING date\_text, SET OF VARSTRING from\_formats, VARSTRING to\_format='\%Y\%m\%d')}} \\
\hline
\end{tabularx}
}

\par





Converts a date that matches one of a set of formats to another.






\par
\begin{description}
\item [\colorbox{tagtype}{\color{white} \textbf{\textsf{PARAMETER}}}] \textbf{\underline{from\_formats}} ||| SET ( VARSTRING ) --- The list of formats the date is to be converted from.
\item [\colorbox{tagtype}{\color{white} \textbf{\textsf{PARAMETER}}}] \textbf{\underline{date\_text}} ||| STRING --- The string containing the date to be converted.
\item [\colorbox{tagtype}{\color{white} \textbf{\textsf{PARAMETER}}}] \textbf{\underline{to\_format}} ||| VARSTRING --- The format the date is to be converted to.
\end{description}







\par
\begin{description}
\item [\colorbox{tagtype}{\color{white} \textbf{\textsf{RETURN}}}] \textbf{STRING} --- The converted string, or blank if it failed to match the format.
\end{description}




\rule{\linewidth}{0.5pt}
\subsection*{\textsf{\colorbox{headtoc}{\color{white} FUNCTION}
ConvertFormatMultiple}}

\hypertarget{ecldoc:date.convertformatmultiple}{}
\hspace{0pt} \hyperlink{ecldoc:Date}{Date} \textbackslash 

{\renewcommand{\arraystretch}{1.5}
\begin{tabularx}{\textwidth}{|>{\raggedright\arraybackslash}l|X|}
\hline
\hspace{0pt}\mytexttt{\color{red} STRING} & \textbf{ConvertFormatMultiple} \\
\hline
\multicolumn{2}{|>{\raggedright\arraybackslash}X|}{\hspace{0pt}\mytexttt{\color{param} (STRING date\_text, SET OF VARSTRING from\_formats, VARSTRING to\_format='\%Y\%m\%d')}} \\
\hline
\end{tabularx}
}

\par





Converts a date that matches one of a set of formats to another.






\par
\begin{description}
\item [\colorbox{tagtype}{\color{white} \textbf{\textsf{PARAMETER}}}] \textbf{\underline{from\_formats}} ||| SET ( VARSTRING ) --- The list of formats the date is to be converted from.
\item [\colorbox{tagtype}{\color{white} \textbf{\textsf{PARAMETER}}}] \textbf{\underline{date\_text}} ||| STRING --- The string containing the date to be converted.
\item [\colorbox{tagtype}{\color{white} \textbf{\textsf{PARAMETER}}}] \textbf{\underline{to\_format}} ||| VARSTRING --- The format the date is to be converted to.
\end{description}







\par
\begin{description}
\item [\colorbox{tagtype}{\color{white} \textbf{\textsf{RETURN}}}] \textbf{STRING} --- The converted string, or blank if it failed to match the format.
\end{description}




\rule{\linewidth}{0.5pt}
\subsection*{\textsf{\colorbox{headtoc}{\color{white} FUNCTION}
ConvertTimeFormatMultiple}}

\hypertarget{ecldoc:date.converttimeformatmultiple}{}
\hspace{0pt} \hyperlink{ecldoc:Date}{Date} \textbackslash 

{\renewcommand{\arraystretch}{1.5}
\begin{tabularx}{\textwidth}{|>{\raggedright\arraybackslash}l|X|}
\hline
\hspace{0pt}\mytexttt{\color{red} STRING} & \textbf{ConvertTimeFormatMultiple} \\
\hline
\multicolumn{2}{|>{\raggedright\arraybackslash}X|}{\hspace{0pt}\mytexttt{\color{param} (STRING time\_text, SET OF VARSTRING from\_formats, VARSTRING to\_format='\%H:\%m:\%s')}} \\
\hline
\end{tabularx}
}

\par





Converts a time that matches one of a set of formats to another.






\par
\begin{description}
\item [\colorbox{tagtype}{\color{white} \textbf{\textsf{PARAMETER}}}] \textbf{\underline{from\_formats}} ||| SET ( VARSTRING ) --- The list of formats the time is to be converted from.
\item [\colorbox{tagtype}{\color{white} \textbf{\textsf{PARAMETER}}}] \textbf{\underline{to\_format}} ||| VARSTRING --- The format the time is to be converted to.
\item [\colorbox{tagtype}{\color{white} \textbf{\textsf{PARAMETER}}}] \textbf{\underline{time\_text}} ||| STRING --- The string containing the time to be converted.
\end{description}







\par
\begin{description}
\item [\colorbox{tagtype}{\color{white} \textbf{\textsf{RETURN}}}] \textbf{STRING} --- The converted string, or blank if it failed to match the format.
\end{description}




\rule{\linewidth}{0.5pt}
\subsection*{\textsf{\colorbox{headtoc}{\color{white} FUNCTION}
AdjustDate}}

\hypertarget{ecldoc:date.adjustdate}{}
\hspace{0pt} \hyperlink{ecldoc:Date}{Date} \textbackslash 

{\renewcommand{\arraystretch}{1.5}
\begin{tabularx}{\textwidth}{|>{\raggedright\arraybackslash}l|X|}
\hline
\hspace{0pt}\mytexttt{\color{red} Date\_t} & \textbf{AdjustDate} \\
\hline
\multicolumn{2}{|>{\raggedright\arraybackslash}X|}{\hspace{0pt}\mytexttt{\color{param} (Date\_t date, INTEGER2 year\_delta = 0, INTEGER4 month\_delta = 0, INTEGER4 day\_delta = 0)}} \\
\hline
\end{tabularx}
}

\par





Adjusts a date by incrementing or decrementing year, month and/or day values. The date must be in the Gregorian calendar after the year 1600. If the new calculated date is invalid then it will be normalized according to mktime() rules. Example: 20140130 + 1 month = 20140302.






\par
\begin{description}
\item [\colorbox{tagtype}{\color{white} \textbf{\textsf{PARAMETER}}}] \textbf{\underline{date}} ||| UNSIGNED4 --- The date to adjust.
\item [\colorbox{tagtype}{\color{white} \textbf{\textsf{PARAMETER}}}] \textbf{\underline{month\_delta}} ||| INTEGER4 --- The requested change to the month value; optional, defaults to zero.
\item [\colorbox{tagtype}{\color{white} \textbf{\textsf{PARAMETER}}}] \textbf{\underline{year\_delta}} ||| INTEGER2 --- The requested change to the year value; optional, defaults to zero.
\item [\colorbox{tagtype}{\color{white} \textbf{\textsf{PARAMETER}}}] \textbf{\underline{day\_delta}} ||| INTEGER4 --- The requested change to the day of month value; optional, defaults to zero.
\end{description}







\par
\begin{description}
\item [\colorbox{tagtype}{\color{white} \textbf{\textsf{RETURN}}}] \textbf{UNSIGNED4} --- The adjusted Date\_t value.
\end{description}




\rule{\linewidth}{0.5pt}
\subsection*{\textsf{\colorbox{headtoc}{\color{white} FUNCTION}
AdjustDateBySeconds}}

\hypertarget{ecldoc:date.adjustdatebyseconds}{}
\hspace{0pt} \hyperlink{ecldoc:Date}{Date} \textbackslash 

{\renewcommand{\arraystretch}{1.5}
\begin{tabularx}{\textwidth}{|>{\raggedright\arraybackslash}l|X|}
\hline
\hspace{0pt}\mytexttt{\color{red} Date\_t} & \textbf{AdjustDateBySeconds} \\
\hline
\multicolumn{2}{|>{\raggedright\arraybackslash}X|}{\hspace{0pt}\mytexttt{\color{param} (Date\_t date, INTEGER4 seconds\_delta)}} \\
\hline
\end{tabularx}
}

\par





Adjusts a date by adding or subtracting seconds. The date must be in the Gregorian calendar after the year 1600. If the new calculated date is invalid then it will be normalized according to mktime() rules. Example: 20140130 + 172800 seconds = 20140201.






\par
\begin{description}
\item [\colorbox{tagtype}{\color{white} \textbf{\textsf{PARAMETER}}}] \textbf{\underline{date}} ||| UNSIGNED4 --- The date to adjust.
\item [\colorbox{tagtype}{\color{white} \textbf{\textsf{PARAMETER}}}] \textbf{\underline{seconds\_delta}} ||| INTEGER4 --- The requested change to the date, in seconds.
\end{description}







\par
\begin{description}
\item [\colorbox{tagtype}{\color{white} \textbf{\textsf{RETURN}}}] \textbf{UNSIGNED4} --- The adjusted Date\_t value.
\end{description}




\rule{\linewidth}{0.5pt}
\subsection*{\textsf{\colorbox{headtoc}{\color{white} FUNCTION}
AdjustTime}}

\hypertarget{ecldoc:date.adjusttime}{}
\hspace{0pt} \hyperlink{ecldoc:Date}{Date} \textbackslash 

{\renewcommand{\arraystretch}{1.5}
\begin{tabularx}{\textwidth}{|>{\raggedright\arraybackslash}l|X|}
\hline
\hspace{0pt}\mytexttt{\color{red} Time\_t} & \textbf{AdjustTime} \\
\hline
\multicolumn{2}{|>{\raggedright\arraybackslash}X|}{\hspace{0pt}\mytexttt{\color{param} (Time\_t time, INTEGER2 hour\_delta = 0, INTEGER4 minute\_delta = 0, INTEGER4 second\_delta = 0)}} \\
\hline
\end{tabularx}
}

\par





Adjusts a time by incrementing or decrementing hour, minute and/or second values. If the new calculated time is invalid then it will be normalized according to mktime() rules.






\par
\begin{description}
\item [\colorbox{tagtype}{\color{white} \textbf{\textsf{PARAMETER}}}] \textbf{\underline{time}} ||| UNSIGNED3 --- The time to adjust.
\item [\colorbox{tagtype}{\color{white} \textbf{\textsf{PARAMETER}}}] \textbf{\underline{second\_delta}} ||| INTEGER4 --- The requested change to the second of month value; optional, defaults to zero.
\item [\colorbox{tagtype}{\color{white} \textbf{\textsf{PARAMETER}}}] \textbf{\underline{hour\_delta}} ||| INTEGER2 --- The requested change to the hour value; optional, defaults to zero.
\item [\colorbox{tagtype}{\color{white} \textbf{\textsf{PARAMETER}}}] \textbf{\underline{minute\_delta}} ||| INTEGER4 --- The requested change to the minute value; optional, defaults to zero.
\end{description}







\par
\begin{description}
\item [\colorbox{tagtype}{\color{white} \textbf{\textsf{RETURN}}}] \textbf{UNSIGNED3} --- The adjusted Time\_t value.
\end{description}




\rule{\linewidth}{0.5pt}
\subsection*{\textsf{\colorbox{headtoc}{\color{white} FUNCTION}
AdjustTimeBySeconds}}

\hypertarget{ecldoc:date.adjusttimebyseconds}{}
\hspace{0pt} \hyperlink{ecldoc:Date}{Date} \textbackslash 

{\renewcommand{\arraystretch}{1.5}
\begin{tabularx}{\textwidth}{|>{\raggedright\arraybackslash}l|X|}
\hline
\hspace{0pt}\mytexttt{\color{red} Time\_t} & \textbf{AdjustTimeBySeconds} \\
\hline
\multicolumn{2}{|>{\raggedright\arraybackslash}X|}{\hspace{0pt}\mytexttt{\color{param} (Time\_t time, INTEGER4 seconds\_delta)}} \\
\hline
\end{tabularx}
}

\par





Adjusts a time by adding or subtracting seconds. If the new calculated time is invalid then it will be normalized according to mktime() rules.






\par
\begin{description}
\item [\colorbox{tagtype}{\color{white} \textbf{\textsf{PARAMETER}}}] \textbf{\underline{time}} ||| UNSIGNED3 --- The time to adjust.
\item [\colorbox{tagtype}{\color{white} \textbf{\textsf{PARAMETER}}}] \textbf{\underline{seconds\_delta}} ||| INTEGER4 --- The requested change to the time, in seconds.
\end{description}







\par
\begin{description}
\item [\colorbox{tagtype}{\color{white} \textbf{\textsf{RETURN}}}] \textbf{UNSIGNED3} --- The adjusted Time\_t value.
\end{description}




\rule{\linewidth}{0.5pt}
\subsection*{\textsf{\colorbox{headtoc}{\color{white} FUNCTION}
AdjustSeconds}}

\hypertarget{ecldoc:date.adjustseconds}{}
\hspace{0pt} \hyperlink{ecldoc:Date}{Date} \textbackslash 

{\renewcommand{\arraystretch}{1.5}
\begin{tabularx}{\textwidth}{|>{\raggedright\arraybackslash}l|X|}
\hline
\hspace{0pt}\mytexttt{\color{red} Seconds\_t} & \textbf{AdjustSeconds} \\
\hline
\multicolumn{2}{|>{\raggedright\arraybackslash}X|}{\hspace{0pt}\mytexttt{\color{param} (Seconds\_t seconds, INTEGER2 year\_delta = 0, INTEGER4 month\_delta = 0, INTEGER4 day\_delta = 0, INTEGER4 hour\_delta = 0, INTEGER4 minute\_delta = 0, INTEGER4 second\_delta = 0)}} \\
\hline
\end{tabularx}
}

\par





Adjusts a Seconds\_t value by adding or subtracting years, months, days, hours, minutes and/or seconds. This is performed by first converting the seconds into a full date/time structure, applying any delta values to individual date/time components, then converting the structure back to the number of seconds. This interim date must lie within Gregorian calendar after the year 1600. If the interim structure is found to have an invalid date/time then it will be normalized according to mktime() rules. Therefore, some delta values (such as ''1 month'') are actually relative to the value of the seconds argument.






\par
\begin{description}
\item [\colorbox{tagtype}{\color{white} \textbf{\textsf{PARAMETER}}}] \textbf{\underline{month\_delta}} ||| INTEGER4 --- The requested change to the month value; optional, defaults to zero.
\item [\colorbox{tagtype}{\color{white} \textbf{\textsf{PARAMETER}}}] \textbf{\underline{second\_delta}} ||| INTEGER4 --- The requested change to the second of month value; optional, defaults to zero.
\item [\colorbox{tagtype}{\color{white} \textbf{\textsf{PARAMETER}}}] \textbf{\underline{seconds}} ||| INTEGER8 --- The number of seconds to adjust.
\item [\colorbox{tagtype}{\color{white} \textbf{\textsf{PARAMETER}}}] \textbf{\underline{day\_delta}} ||| INTEGER4 --- The requested change to the day of month value; optional, defaults to zero.
\item [\colorbox{tagtype}{\color{white} \textbf{\textsf{PARAMETER}}}] \textbf{\underline{minute\_delta}} ||| INTEGER4 --- The requested change to the minute value; optional, defaults to zero.
\item [\colorbox{tagtype}{\color{white} \textbf{\textsf{PARAMETER}}}] \textbf{\underline{year\_delta}} ||| INTEGER2 --- The requested change to the year value; optional, defaults to zero.
\item [\colorbox{tagtype}{\color{white} \textbf{\textsf{PARAMETER}}}] \textbf{\underline{hour\_delta}} ||| INTEGER4 --- The requested change to the hour value; optional, defaults to zero.
\end{description}







\par
\begin{description}
\item [\colorbox{tagtype}{\color{white} \textbf{\textsf{RETURN}}}] \textbf{INTEGER8} --- The adjusted Seconds\_t value.
\end{description}




\rule{\linewidth}{0.5pt}
\subsection*{\textsf{\colorbox{headtoc}{\color{white} FUNCTION}
AdjustCalendar}}

\hypertarget{ecldoc:date.adjustcalendar}{}
\hspace{0pt} \hyperlink{ecldoc:Date}{Date} \textbackslash 

{\renewcommand{\arraystretch}{1.5}
\begin{tabularx}{\textwidth}{|>{\raggedright\arraybackslash}l|X|}
\hline
\hspace{0pt}\mytexttt{\color{red} Date\_t} & \textbf{AdjustCalendar} \\
\hline
\multicolumn{2}{|>{\raggedright\arraybackslash}X|}{\hspace{0pt}\mytexttt{\color{param} (Date\_t date, INTEGER2 year\_delta = 0, INTEGER4 month\_delta = 0, INTEGER4 day\_delta = 0)}} \\
\hline
\end{tabularx}
}

\par





Adjusts a date by incrementing or decrementing months and/or years. This routine uses the rule outlined in McGinn v. State, 46 Neb. 427, 65 N.W. 46 (1895): ''The term calendar month, whether employed in statutes or contracts, and not appearing to have been used in a different sense, denotes a period terminating with the day of the succeeding month numerically corresponding to the day of its beginning, less one. If there be no corresponding day of the succeeding month, it terminates with the last day thereof.'' The internet suggests similar legal positions exist in the Commonwealth and Germany. Note that day adjustments are performed after year and month adjustments using the preceding rules. As an example, Jan. 31, 2014 + 1 month will result in Feb. 28, 2014; Jan. 31, 2014 + 1 month + 1 day will result in Mar. 1, 2014.






\par
\begin{description}
\item [\colorbox{tagtype}{\color{white} \textbf{\textsf{PARAMETER}}}] \textbf{\underline{date}} ||| UNSIGNED4 --- The date to adjust, in the Gregorian calendar after 1600.
\item [\colorbox{tagtype}{\color{white} \textbf{\textsf{PARAMETER}}}] \textbf{\underline{month\_delta}} ||| INTEGER4 --- The requested change to the month value; optional, defaults to zero.
\item [\colorbox{tagtype}{\color{white} \textbf{\textsf{PARAMETER}}}] \textbf{\underline{year\_delta}} ||| INTEGER2 --- The requested change to the year value; optional, defaults to zero.
\item [\colorbox{tagtype}{\color{white} \textbf{\textsf{PARAMETER}}}] \textbf{\underline{day\_delta}} ||| INTEGER4 --- The requested change to the day value; optional, defaults to zero.
\end{description}







\par
\begin{description}
\item [\colorbox{tagtype}{\color{white} \textbf{\textsf{RETURN}}}] \textbf{UNSIGNED4} --- The adjusted Date\_t value.
\end{description}




\rule{\linewidth}{0.5pt}
\subsection*{\textsf{\colorbox{headtoc}{\color{white} FUNCTION}
IsLocalDaylightSavingsInEffect}}

\hypertarget{ecldoc:date.islocaldaylightsavingsineffect}{}
\hspace{0pt} \hyperlink{ecldoc:Date}{Date} \textbackslash 

{\renewcommand{\arraystretch}{1.5}
\begin{tabularx}{\textwidth}{|>{\raggedright\arraybackslash}l|X|}
\hline
\hspace{0pt}\mytexttt{\color{red} BOOLEAN} & \textbf{IsLocalDaylightSavingsInEffect} \\
\hline
\multicolumn{2}{|>{\raggedright\arraybackslash}X|}{\hspace{0pt}\mytexttt{\color{param} ()}} \\
\hline
\end{tabularx}
}

\par





Returns a boolean indicating whether daylight savings time is currently in effect locally.








\par
\begin{description}
\item [\colorbox{tagtype}{\color{white} \textbf{\textsf{RETURN}}}] \textbf{BOOLEAN} --- TRUE if daylight savings time is currently in effect, FALSE otherwise.
\end{description}




\rule{\linewidth}{0.5pt}
\subsection*{\textsf{\colorbox{headtoc}{\color{white} FUNCTION}
LocalTimeZoneOffset}}

\hypertarget{ecldoc:date.localtimezoneoffset}{}
\hspace{0pt} \hyperlink{ecldoc:Date}{Date} \textbackslash 

{\renewcommand{\arraystretch}{1.5}
\begin{tabularx}{\textwidth}{|>{\raggedright\arraybackslash}l|X|}
\hline
\hspace{0pt}\mytexttt{\color{red} INTEGER4} & \textbf{LocalTimeZoneOffset} \\
\hline
\multicolumn{2}{|>{\raggedright\arraybackslash}X|}{\hspace{0pt}\mytexttt{\color{param} ()}} \\
\hline
\end{tabularx}
}

\par





Returns the offset (in seconds) of the time represented from UTC, with positive values indicating locations east of the Prime Meridian. Given a UTC time in seconds since epoch, you can find the local time by adding the result of this function to the seconds.








\par
\begin{description}
\item [\colorbox{tagtype}{\color{white} \textbf{\textsf{RETURN}}}] \textbf{INTEGER4} --- The number of seconds offset from UTC.
\end{description}




\rule{\linewidth}{0.5pt}
\subsection*{\textsf{\colorbox{headtoc}{\color{white} FUNCTION}
CurrentDate}}

\hypertarget{ecldoc:date.currentdate}{}
\hspace{0pt} \hyperlink{ecldoc:Date}{Date} \textbackslash 

{\renewcommand{\arraystretch}{1.5}
\begin{tabularx}{\textwidth}{|>{\raggedright\arraybackslash}l|X|}
\hline
\hspace{0pt}\mytexttt{\color{red} Date\_t} & \textbf{CurrentDate} \\
\hline
\multicolumn{2}{|>{\raggedright\arraybackslash}X|}{\hspace{0pt}\mytexttt{\color{param} (BOOLEAN in\_local\_time = FALSE)}} \\
\hline
\end{tabularx}
}

\par





Returns the current date.






\par
\begin{description}
\item [\colorbox{tagtype}{\color{white} \textbf{\textsf{PARAMETER}}}] \textbf{\underline{in\_local\_time}} ||| BOOLEAN --- TRUE if the returned value should be local to the cluster computing the date, FALSE for UTC. Optional, defaults to FALSE.
\end{description}







\par
\begin{description}
\item [\colorbox{tagtype}{\color{white} \textbf{\textsf{RETURN}}}] \textbf{UNSIGNED4} --- A Date\_t representing the current date.
\end{description}




\rule{\linewidth}{0.5pt}
\subsection*{\textsf{\colorbox{headtoc}{\color{white} FUNCTION}
Today}}

\hypertarget{ecldoc:date.today}{}
\hspace{0pt} \hyperlink{ecldoc:Date}{Date} \textbackslash 

{\renewcommand{\arraystretch}{1.5}
\begin{tabularx}{\textwidth}{|>{\raggedright\arraybackslash}l|X|}
\hline
\hspace{0pt}\mytexttt{\color{red} Date\_t} & \textbf{Today} \\
\hline
\multicolumn{2}{|>{\raggedright\arraybackslash}X|}{\hspace{0pt}\mytexttt{\color{param} ()}} \\
\hline
\end{tabularx}
}

\par





Returns the current date in the local time zone.








\par
\begin{description}
\item [\colorbox{tagtype}{\color{white} \textbf{\textsf{RETURN}}}] \textbf{UNSIGNED4} --- A Date\_t representing the current date.
\end{description}




\rule{\linewidth}{0.5pt}
\subsection*{\textsf{\colorbox{headtoc}{\color{white} FUNCTION}
CurrentTime}}

\hypertarget{ecldoc:date.currenttime}{}
\hspace{0pt} \hyperlink{ecldoc:Date}{Date} \textbackslash 

{\renewcommand{\arraystretch}{1.5}
\begin{tabularx}{\textwidth}{|>{\raggedright\arraybackslash}l|X|}
\hline
\hspace{0pt}\mytexttt{\color{red} Time\_t} & \textbf{CurrentTime} \\
\hline
\multicolumn{2}{|>{\raggedright\arraybackslash}X|}{\hspace{0pt}\mytexttt{\color{param} (BOOLEAN in\_local\_time = FALSE)}} \\
\hline
\end{tabularx}
}

\par





Returns the current time of day






\par
\begin{description}
\item [\colorbox{tagtype}{\color{white} \textbf{\textsf{PARAMETER}}}] \textbf{\underline{in\_local\_time}} ||| BOOLEAN --- TRUE if the returned value should be local to the cluster computing the time, FALSE for UTC. Optional, defaults to FALSE.
\end{description}







\par
\begin{description}
\item [\colorbox{tagtype}{\color{white} \textbf{\textsf{RETURN}}}] \textbf{UNSIGNED3} --- A Time\_t representing the current time of day.
\end{description}




\rule{\linewidth}{0.5pt}
\subsection*{\textsf{\colorbox{headtoc}{\color{white} FUNCTION}
CurrentSeconds}}

\hypertarget{ecldoc:date.currentseconds}{}
\hspace{0pt} \hyperlink{ecldoc:Date}{Date} \textbackslash 

{\renewcommand{\arraystretch}{1.5}
\begin{tabularx}{\textwidth}{|>{\raggedright\arraybackslash}l|X|}
\hline
\hspace{0pt}\mytexttt{\color{red} Seconds\_t} & \textbf{CurrentSeconds} \\
\hline
\multicolumn{2}{|>{\raggedright\arraybackslash}X|}{\hspace{0pt}\mytexttt{\color{param} (BOOLEAN in\_local\_time = FALSE)}} \\
\hline
\end{tabularx}
}

\par





Returns the current date and time as the number of seconds since epoch.






\par
\begin{description}
\item [\colorbox{tagtype}{\color{white} \textbf{\textsf{PARAMETER}}}] \textbf{\underline{in\_local\_time}} ||| BOOLEAN --- TRUE if the returned value should be local to the cluster computing the time, FALSE for UTC. Optional, defaults to FALSE.
\end{description}







\par
\begin{description}
\item [\colorbox{tagtype}{\color{white} \textbf{\textsf{RETURN}}}] \textbf{INTEGER8} --- A Seconds\_t representing the current time in UTC or local time, depending on the argument.
\end{description}




\rule{\linewidth}{0.5pt}
\subsection*{\textsf{\colorbox{headtoc}{\color{white} FUNCTION}
CurrentTimestamp}}

\hypertarget{ecldoc:date.currenttimestamp}{}
\hspace{0pt} \hyperlink{ecldoc:Date}{Date} \textbackslash 

{\renewcommand{\arraystretch}{1.5}
\begin{tabularx}{\textwidth}{|>{\raggedright\arraybackslash}l|X|}
\hline
\hspace{0pt}\mytexttt{\color{red} Timestamp\_t} & \textbf{CurrentTimestamp} \\
\hline
\multicolumn{2}{|>{\raggedright\arraybackslash}X|}{\hspace{0pt}\mytexttt{\color{param} (BOOLEAN in\_local\_time = FALSE)}} \\
\hline
\end{tabularx}
}

\par





Returns the current date and time as the number of microseconds since epoch.






\par
\begin{description}
\item [\colorbox{tagtype}{\color{white} \textbf{\textsf{PARAMETER}}}] \textbf{\underline{in\_local\_time}} ||| BOOLEAN --- TRUE if the returned value should be local to the cluster computing the time, FALSE for UTC. Optional, defaults to FALSE.
\end{description}







\par
\begin{description}
\item [\colorbox{tagtype}{\color{white} \textbf{\textsf{RETURN}}}] \textbf{INTEGER8} --- A Timestamp\_t representing the current time in microseconds in UTC or local time, depending on the argument.
\end{description}




\rule{\linewidth}{0.5pt}
\subsection*{\textsf{\colorbox{headtoc}{\color{white} MODULE}
DatesForMonth}}

\hypertarget{ecldoc:date.datesformonth}{}
\hspace{0pt} \hyperlink{ecldoc:Date}{Date} \textbackslash 

{\renewcommand{\arraystretch}{1.5}
\begin{tabularx}{\textwidth}{|>{\raggedright\arraybackslash}l|X|}
\hline
\hspace{0pt}\mytexttt{\color{red} } & \textbf{DatesForMonth} \\
\hline
\multicolumn{2}{|>{\raggedright\arraybackslash}X|}{\hspace{0pt}\mytexttt{\color{param} (Date\_t as\_of\_date = CurrentDate(FALSE))}} \\
\hline
\end{tabularx}
}

\par





Returns the beginning and ending dates for the month surrounding the given date.






\par
\begin{description}
\item [\colorbox{tagtype}{\color{white} \textbf{\textsf{PARAMETER}}}] \textbf{\underline{as\_of\_date}} ||| UNSIGNED4 --- The reference date from which the month will be calculated. This date must be a date within the Gregorian calendar. Optional, defaults to the current date in UTC.
\end{description}







\par
\begin{description}
\item [\colorbox{tagtype}{\color{white} \textbf{\textsf{RETURN}}}] \textbf{} --- Module with exported attributes for startDate and endDate.
\end{description}




\textbf{Children}
\begin{enumerate}
\item \hyperlink{ecldoc:date.datesformonth.result.startdate}{startDate}
: No Documentation Found
\item \hyperlink{ecldoc:date.datesformonth.result.enddate}{endDate}
: No Documentation Found
\end{enumerate}

\rule{\linewidth}{0.5pt}

\subsection*{\textsf{\colorbox{headtoc}{\color{white} ATTRIBUTE}
startDate}}

\hypertarget{ecldoc:date.datesformonth.result.startdate}{}
\hspace{0pt} \hyperlink{ecldoc:Date}{Date} \textbackslash 
\hspace{0pt} \hyperlink{ecldoc:date.datesformonth}{DatesForMonth} \textbackslash 

{\renewcommand{\arraystretch}{1.5}
\begin{tabularx}{\textwidth}{|>{\raggedright\arraybackslash}l|X|}
\hline
\hspace{0pt}\mytexttt{\color{red} Date\_t} & \textbf{startDate} \\
\hline
\end{tabularx}
}

\par





No Documentation Found








\par
\begin{description}
\item [\colorbox{tagtype}{\color{white} \textbf{\textsf{RETURN}}}] \textbf{UNSIGNED4} --- 
\end{description}




\rule{\linewidth}{0.5pt}
\subsection*{\textsf{\colorbox{headtoc}{\color{white} ATTRIBUTE}
endDate}}

\hypertarget{ecldoc:date.datesformonth.result.enddate}{}
\hspace{0pt} \hyperlink{ecldoc:Date}{Date} \textbackslash 
\hspace{0pt} \hyperlink{ecldoc:date.datesformonth}{DatesForMonth} \textbackslash 

{\renewcommand{\arraystretch}{1.5}
\begin{tabularx}{\textwidth}{|>{\raggedright\arraybackslash}l|X|}
\hline
\hspace{0pt}\mytexttt{\color{red} Date\_t} & \textbf{endDate} \\
\hline
\end{tabularx}
}

\par





No Documentation Found








\par
\begin{description}
\item [\colorbox{tagtype}{\color{white} \textbf{\textsf{RETURN}}}] \textbf{UNSIGNED4} --- 
\end{description}




\rule{\linewidth}{0.5pt}


\subsection*{\textsf{\colorbox{headtoc}{\color{white} MODULE}
DatesForWeek}}

\hypertarget{ecldoc:date.datesforweek}{}
\hspace{0pt} \hyperlink{ecldoc:Date}{Date} \textbackslash 

{\renewcommand{\arraystretch}{1.5}
\begin{tabularx}{\textwidth}{|>{\raggedright\arraybackslash}l|X|}
\hline
\hspace{0pt}\mytexttt{\color{red} } & \textbf{DatesForWeek} \\
\hline
\multicolumn{2}{|>{\raggedright\arraybackslash}X|}{\hspace{0pt}\mytexttt{\color{param} (Date\_t as\_of\_date = CurrentDate(FALSE))}} \\
\hline
\end{tabularx}
}

\par





Returns the beginning and ending dates for the week surrounding the given date (Sunday marks the beginning of a week).






\par
\begin{description}
\item [\colorbox{tagtype}{\color{white} \textbf{\textsf{PARAMETER}}}] \textbf{\underline{as\_of\_date}} ||| UNSIGNED4 --- The reference date from which the week will be calculated. This date must be a date within the Gregorian calendar. Optional, defaults to the current date in UTC.
\end{description}







\par
\begin{description}
\item [\colorbox{tagtype}{\color{white} \textbf{\textsf{RETURN}}}] \textbf{} --- Module with exported attributes for startDate and endDate.
\end{description}




\textbf{Children}
\begin{enumerate}
\item \hyperlink{ecldoc:date.datesforweek.result.startdate}{startDate}
: No Documentation Found
\item \hyperlink{ecldoc:date.datesforweek.result.enddate}{endDate}
: No Documentation Found
\end{enumerate}

\rule{\linewidth}{0.5pt}

\subsection*{\textsf{\colorbox{headtoc}{\color{white} ATTRIBUTE}
startDate}}

\hypertarget{ecldoc:date.datesforweek.result.startdate}{}
\hspace{0pt} \hyperlink{ecldoc:Date}{Date} \textbackslash 
\hspace{0pt} \hyperlink{ecldoc:date.datesforweek}{DatesForWeek} \textbackslash 

{\renewcommand{\arraystretch}{1.5}
\begin{tabularx}{\textwidth}{|>{\raggedright\arraybackslash}l|X|}
\hline
\hspace{0pt}\mytexttt{\color{red} Date\_t} & \textbf{startDate} \\
\hline
\end{tabularx}
}

\par





No Documentation Found








\par
\begin{description}
\item [\colorbox{tagtype}{\color{white} \textbf{\textsf{RETURN}}}] \textbf{UNSIGNED4} --- 
\end{description}




\rule{\linewidth}{0.5pt}
\subsection*{\textsf{\colorbox{headtoc}{\color{white} ATTRIBUTE}
endDate}}

\hypertarget{ecldoc:date.datesforweek.result.enddate}{}
\hspace{0pt} \hyperlink{ecldoc:Date}{Date} \textbackslash 
\hspace{0pt} \hyperlink{ecldoc:date.datesforweek}{DatesForWeek} \textbackslash 

{\renewcommand{\arraystretch}{1.5}
\begin{tabularx}{\textwidth}{|>{\raggedright\arraybackslash}l|X|}
\hline
\hspace{0pt}\mytexttt{\color{red} Date\_t} & \textbf{endDate} \\
\hline
\end{tabularx}
}

\par





No Documentation Found








\par
\begin{description}
\item [\colorbox{tagtype}{\color{white} \textbf{\textsf{RETURN}}}] \textbf{UNSIGNED4} --- 
\end{description}




\rule{\linewidth}{0.5pt}


\subsection*{\textsf{\colorbox{headtoc}{\color{white} FUNCTION}
IsValidDate}}

\hypertarget{ecldoc:date.isvaliddate}{}
\hspace{0pt} \hyperlink{ecldoc:Date}{Date} \textbackslash 

{\renewcommand{\arraystretch}{1.5}
\begin{tabularx}{\textwidth}{|>{\raggedright\arraybackslash}l|X|}
\hline
\hspace{0pt}\mytexttt{\color{red} BOOLEAN} & \textbf{IsValidDate} \\
\hline
\multicolumn{2}{|>{\raggedright\arraybackslash}X|}{\hspace{0pt}\mytexttt{\color{param} (Date\_t date, INTEGER2 yearLowerBound = 1800, INTEGER2 yearUpperBound = 2100)}} \\
\hline
\end{tabularx}
}

\par





Tests whether a date is valid, both by range-checking the year and by validating each of the other individual components.






\par
\begin{description}
\item [\colorbox{tagtype}{\color{white} \textbf{\textsf{PARAMETER}}}] \textbf{\underline{date}} ||| UNSIGNED4 --- The date to validate.
\item [\colorbox{tagtype}{\color{white} \textbf{\textsf{PARAMETER}}}] \textbf{\underline{yearUpperBound}} ||| INTEGER2 --- The maximum acceptable year. Optional; defaults to 2100.
\item [\colorbox{tagtype}{\color{white} \textbf{\textsf{PARAMETER}}}] \textbf{\underline{yearLowerBound}} ||| INTEGER2 --- The minimum acceptable year. Optional; defaults to 1800.
\end{description}







\par
\begin{description}
\item [\colorbox{tagtype}{\color{white} \textbf{\textsf{RETURN}}}] \textbf{BOOLEAN} --- TRUE if the date is valid, FALSE otherwise.
\end{description}




\rule{\linewidth}{0.5pt}
\subsection*{\textsf{\colorbox{headtoc}{\color{white} FUNCTION}
IsValidGregorianDate}}

\hypertarget{ecldoc:date.isvalidgregoriandate}{}
\hspace{0pt} \hyperlink{ecldoc:Date}{Date} \textbackslash 

{\renewcommand{\arraystretch}{1.5}
\begin{tabularx}{\textwidth}{|>{\raggedright\arraybackslash}l|X|}
\hline
\hspace{0pt}\mytexttt{\color{red} BOOLEAN} & \textbf{IsValidGregorianDate} \\
\hline
\multicolumn{2}{|>{\raggedright\arraybackslash}X|}{\hspace{0pt}\mytexttt{\color{param} (Date\_t date)}} \\
\hline
\end{tabularx}
}

\par





Tests whether a date is valid in the Gregorian calendar. The year must be between 1601 and 30827.






\par
\begin{description}
\item [\colorbox{tagtype}{\color{white} \textbf{\textsf{PARAMETER}}}] \textbf{\underline{date}} ||| UNSIGNED4 --- The Date\_t to validate.
\end{description}







\par
\begin{description}
\item [\colorbox{tagtype}{\color{white} \textbf{\textsf{RETURN}}}] \textbf{BOOLEAN} --- TRUE if the date is valid, FALSE otherwise.
\end{description}




\rule{\linewidth}{0.5pt}
\subsection*{\textsf{\colorbox{headtoc}{\color{white} FUNCTION}
IsValidTime}}

\hypertarget{ecldoc:date.isvalidtime}{}
\hspace{0pt} \hyperlink{ecldoc:Date}{Date} \textbackslash 

{\renewcommand{\arraystretch}{1.5}
\begin{tabularx}{\textwidth}{|>{\raggedright\arraybackslash}l|X|}
\hline
\hspace{0pt}\mytexttt{\color{red} BOOLEAN} & \textbf{IsValidTime} \\
\hline
\multicolumn{2}{|>{\raggedright\arraybackslash}X|}{\hspace{0pt}\mytexttt{\color{param} (Time\_t time)}} \\
\hline
\end{tabularx}
}

\par





Tests whether a time is valid.






\par
\begin{description}
\item [\colorbox{tagtype}{\color{white} \textbf{\textsf{PARAMETER}}}] \textbf{\underline{time}} ||| UNSIGNED3 --- The time to validate.
\end{description}







\par
\begin{description}
\item [\colorbox{tagtype}{\color{white} \textbf{\textsf{RETURN}}}] \textbf{BOOLEAN} --- TRUE if the time is valid, FALSE otherwise.
\end{description}




\rule{\linewidth}{0.5pt}
\subsection*{\textsf{\colorbox{headtoc}{\color{white} TRANSFORM}
CreateDate}}

\hypertarget{ecldoc:date.createdate}{}
\hspace{0pt} \hyperlink{ecldoc:Date}{Date} \textbackslash 

{\renewcommand{\arraystretch}{1.5}
\begin{tabularx}{\textwidth}{|>{\raggedright\arraybackslash}l|X|}
\hline
\hspace{0pt}\mytexttt{\color{red} Date\_rec} & \textbf{CreateDate} \\
\hline
\multicolumn{2}{|>{\raggedright\arraybackslash}X|}{\hspace{0pt}\mytexttt{\color{param} (INTEGER2 year, UNSIGNED1 month, UNSIGNED1 day)}} \\
\hline
\end{tabularx}
}

\par





A transform to create a Date\_rec from the individual elements






\par
\begin{description}
\item [\colorbox{tagtype}{\color{white} \textbf{\textsf{PARAMETER}}}] \textbf{\underline{year}} ||| INTEGER2 --- The year
\item [\colorbox{tagtype}{\color{white} \textbf{\textsf{PARAMETER}}}] \textbf{\underline{month}} ||| UNSIGNED1 --- The month (1-12).
\item [\colorbox{tagtype}{\color{white} \textbf{\textsf{PARAMETER}}}] \textbf{\underline{day}} ||| UNSIGNED1 --- The day (1..daysInMonth).
\end{description}







\par
\begin{description}
\item [\colorbox{tagtype}{\color{white} \textbf{\textsf{RETURN}}}] \textbf{Date\_rec} --- A transform that creates a Date\_rec containing the date.
\end{description}




\rule{\linewidth}{0.5pt}
\subsection*{\textsf{\colorbox{headtoc}{\color{white} TRANSFORM}
CreateDateFromSeconds}}

\hypertarget{ecldoc:date.createdatefromseconds}{}
\hspace{0pt} \hyperlink{ecldoc:Date}{Date} \textbackslash 

{\renewcommand{\arraystretch}{1.5}
\begin{tabularx}{\textwidth}{|>{\raggedright\arraybackslash}l|X|}
\hline
\hspace{0pt}\mytexttt{\color{red} Date\_rec} & \textbf{CreateDateFromSeconds} \\
\hline
\multicolumn{2}{|>{\raggedright\arraybackslash}X|}{\hspace{0pt}\mytexttt{\color{param} (Seconds\_t seconds)}} \\
\hline
\end{tabularx}
}

\par





A transform to create a Date\_rec from a Seconds\_t value.






\par
\begin{description}
\item [\colorbox{tagtype}{\color{white} \textbf{\textsf{PARAMETER}}}] \textbf{\underline{seconds}} ||| INTEGER8 --- The number seconds since epoch.
\end{description}







\par
\begin{description}
\item [\colorbox{tagtype}{\color{white} \textbf{\textsf{RETURN}}}] \textbf{Date\_rec} --- A transform that creates a Date\_rec containing the date.
\end{description}




\rule{\linewidth}{0.5pt}
\subsection*{\textsf{\colorbox{headtoc}{\color{white} TRANSFORM}
CreateTime}}

\hypertarget{ecldoc:date.createtime}{}
\hspace{0pt} \hyperlink{ecldoc:Date}{Date} \textbackslash 

{\renewcommand{\arraystretch}{1.5}
\begin{tabularx}{\textwidth}{|>{\raggedright\arraybackslash}l|X|}
\hline
\hspace{0pt}\mytexttt{\color{red} Time\_rec} & \textbf{CreateTime} \\
\hline
\multicolumn{2}{|>{\raggedright\arraybackslash}X|}{\hspace{0pt}\mytexttt{\color{param} (UNSIGNED1 hour, UNSIGNED1 minute, UNSIGNED1 second)}} \\
\hline
\end{tabularx}
}

\par





A transform to create a Time\_rec from the individual elements






\par
\begin{description}
\item [\colorbox{tagtype}{\color{white} \textbf{\textsf{PARAMETER}}}] \textbf{\underline{minute}} ||| UNSIGNED1 --- The minute (0-59).
\item [\colorbox{tagtype}{\color{white} \textbf{\textsf{PARAMETER}}}] \textbf{\underline{second}} ||| UNSIGNED1 --- The second (0-59).
\item [\colorbox{tagtype}{\color{white} \textbf{\textsf{PARAMETER}}}] \textbf{\underline{hour}} ||| UNSIGNED1 --- The hour (0-23).
\end{description}







\par
\begin{description}
\item [\colorbox{tagtype}{\color{white} \textbf{\textsf{RETURN}}}] \textbf{Time\_rec} --- A transform that creates a Time\_rec containing the time of day.
\end{description}




\rule{\linewidth}{0.5pt}
\subsection*{\textsf{\colorbox{headtoc}{\color{white} TRANSFORM}
CreateTimeFromSeconds}}

\hypertarget{ecldoc:date.createtimefromseconds}{}
\hspace{0pt} \hyperlink{ecldoc:Date}{Date} \textbackslash 

{\renewcommand{\arraystretch}{1.5}
\begin{tabularx}{\textwidth}{|>{\raggedright\arraybackslash}l|X|}
\hline
\hspace{0pt}\mytexttt{\color{red} Time\_rec} & \textbf{CreateTimeFromSeconds} \\
\hline
\multicolumn{2}{|>{\raggedright\arraybackslash}X|}{\hspace{0pt}\mytexttt{\color{param} (Seconds\_t seconds)}} \\
\hline
\end{tabularx}
}

\par





A transform to create a Time\_rec from a Seconds\_t value.






\par
\begin{description}
\item [\colorbox{tagtype}{\color{white} \textbf{\textsf{PARAMETER}}}] \textbf{\underline{seconds}} ||| INTEGER8 --- The number seconds since epoch.
\end{description}







\par
\begin{description}
\item [\colorbox{tagtype}{\color{white} \textbf{\textsf{RETURN}}}] \textbf{Time\_rec} --- A transform that creates a Time\_rec containing the time of day.
\end{description}




\rule{\linewidth}{0.5pt}
\subsection*{\textsf{\colorbox{headtoc}{\color{white} TRANSFORM}
CreateDateTime}}

\hypertarget{ecldoc:date.createdatetime}{}
\hspace{0pt} \hyperlink{ecldoc:Date}{Date} \textbackslash 

{\renewcommand{\arraystretch}{1.5}
\begin{tabularx}{\textwidth}{|>{\raggedright\arraybackslash}l|X|}
\hline
\hspace{0pt}\mytexttt{\color{red} DateTime\_rec} & \textbf{CreateDateTime} \\
\hline
\multicolumn{2}{|>{\raggedright\arraybackslash}X|}{\hspace{0pt}\mytexttt{\color{param} (INTEGER2 year, UNSIGNED1 month, UNSIGNED1 day, UNSIGNED1 hour, UNSIGNED1 minute, UNSIGNED1 second)}} \\
\hline
\end{tabularx}
}

\par





A transform to create a DateTime\_rec from the individual elements






\par
\begin{description}
\item [\colorbox{tagtype}{\color{white} \textbf{\textsf{PARAMETER}}}] \textbf{\underline{year}} ||| INTEGER2 --- The year
\item [\colorbox{tagtype}{\color{white} \textbf{\textsf{PARAMETER}}}] \textbf{\underline{second}} ||| UNSIGNED1 --- The second (0-59).
\item [\colorbox{tagtype}{\color{white} \textbf{\textsf{PARAMETER}}}] \textbf{\underline{hour}} ||| UNSIGNED1 --- The hour (0-23).
\item [\colorbox{tagtype}{\color{white} \textbf{\textsf{PARAMETER}}}] \textbf{\underline{minute}} ||| UNSIGNED1 --- The minute (0-59).
\item [\colorbox{tagtype}{\color{white} \textbf{\textsf{PARAMETER}}}] \textbf{\underline{month}} ||| UNSIGNED1 --- The month (1-12).
\item [\colorbox{tagtype}{\color{white} \textbf{\textsf{PARAMETER}}}] \textbf{\underline{day}} ||| UNSIGNED1 --- The day (1..daysInMonth).
\end{description}







\par
\begin{description}
\item [\colorbox{tagtype}{\color{white} \textbf{\textsf{RETURN}}}] \textbf{DateTime\_rec} --- A transform that creates a DateTime\_rec containing date and time components.
\end{description}




\rule{\linewidth}{0.5pt}
\subsection*{\textsf{\colorbox{headtoc}{\color{white} TRANSFORM}
CreateDateTimeFromSeconds}}

\hypertarget{ecldoc:date.createdatetimefromseconds}{}
\hspace{0pt} \hyperlink{ecldoc:Date}{Date} \textbackslash 

{\renewcommand{\arraystretch}{1.5}
\begin{tabularx}{\textwidth}{|>{\raggedright\arraybackslash}l|X|}
\hline
\hspace{0pt}\mytexttt{\color{red} DateTime\_rec} & \textbf{CreateDateTimeFromSeconds} \\
\hline
\multicolumn{2}{|>{\raggedright\arraybackslash}X|}{\hspace{0pt}\mytexttt{\color{param} (Seconds\_t seconds)}} \\
\hline
\end{tabularx}
}

\par





A transform to create a DateTime\_rec from a Seconds\_t value.






\par
\begin{description}
\item [\colorbox{tagtype}{\color{white} \textbf{\textsf{PARAMETER}}}] \textbf{\underline{seconds}} ||| INTEGER8 --- The number seconds since epoch.
\end{description}







\par
\begin{description}
\item [\colorbox{tagtype}{\color{white} \textbf{\textsf{RETURN}}}] \textbf{DateTime\_rec} --- A transform that creates a DateTime\_rec containing date and time components.
\end{description}




\rule{\linewidth}{0.5pt}



\chapter*{\color{headfile}
File
}
\hypertarget{ecldoc:toc:File}{}
\hyperlink{ecldoc:toc:root}{Go Up}

\section*{\underline{\textsf{IMPORTS}}}
\begin{doublespace}
{\large
lib\_fileservices |
}
\end{doublespace}

\section*{\underline{\textsf{DESCRIPTIONS}}}
\subsection*{\textsf{\colorbox{headtoc}{\color{white} MODULE}
File}}

\hypertarget{ecldoc:File}{}

{\renewcommand{\arraystretch}{1.5}
\begin{tabularx}{\textwidth}{|>{\raggedright\arraybackslash}l|X|}
\hline
\hspace{0pt}\mytexttt{\color{red} } & \textbf{File} \\
\hline
\end{tabularx}
}

\par


\textbf{Children}
\begin{enumerate}
\item \hyperlink{ecldoc:file.fsfilenamerecord}{FsFilenameRecord}
: A record containing information about filename
\item \hyperlink{ecldoc:file.fslogicalfilename}{FsLogicalFileName}
: An alias for a logical filename that is stored in a row
\item \hyperlink{ecldoc:file.fslogicalfilenamerecord}{FsLogicalFileNameRecord}
: A record containing a logical filename
\item \hyperlink{ecldoc:file.fslogicalfileinforecord}{FsLogicalFileInfoRecord}
: A record containing information about a logical file
\item \hyperlink{ecldoc:file.fslogicalsupersubrecord}{FsLogicalSuperSubRecord}
: A record containing information about a superfile and its contents
\item \hyperlink{ecldoc:file.fsfilerelationshiprecord}{FsFileRelationshipRecord}
: A record containing information about the relationship between two files
\item \hyperlink{ecldoc:file.recfmv_recsize}{RECFMV\_RECSIZE}
: Constant that indicates IBM RECFM V format file
\item \hyperlink{ecldoc:file.recfmvb_recsize}{RECFMVB\_RECSIZE}
: Constant that indicates IBM RECFM VB format file
\item \hyperlink{ecldoc:file.prefix_variable_recsize}{PREFIX\_VARIABLE\_RECSIZE}
: Constant that indicates a variable little endian 4 byte length prefixed file
\item \hyperlink{ecldoc:file.prefix_variable_bigendian_recsize}{PREFIX\_VARIABLE\_BIGENDIAN\_RECSIZE}
: Constant that indicates a variable big endian 4 byte length prefixed file
\item \hyperlink{ecldoc:file.fileexists}{FileExists}
: Returns whether the file exists
\item \hyperlink{ecldoc:file.deletelogicalfile}{DeleteLogicalFile}
: Removes the logical file from the system, and deletes from the disk
\item \hyperlink{ecldoc:file.setreadonly}{SetReadOnly}
: Changes whether access to a file is read only or not
\item \hyperlink{ecldoc:file.renamelogicalfile}{RenameLogicalFile}
: Changes the name of a logical file
\item \hyperlink{ecldoc:file.foreignlogicalfilename}{ForeignLogicalFileName}
: Returns a logical filename that can be used to refer to a logical file in a local or remote dali
\item \hyperlink{ecldoc:file.externallogicalfilename}{ExternalLogicalFileName}
: Returns an encoded logical filename that can be used to refer to a external file
\item \hyperlink{ecldoc:file.getfiledescription}{GetFileDescription}
: Returns a string containing the description information associated with the specified filename
\item \hyperlink{ecldoc:file.setfiledescription}{SetFileDescription}
: Sets the description associated with the specified filename
\item \hyperlink{ecldoc:file.remotedirectory}{RemoteDirectory}
: Returns a dataset containing a list of files from the specified machineIP and directory
\item \hyperlink{ecldoc:file.logicalfilelist}{LogicalFileList}
: Returns a dataset of information about the logical files known to the system
\item \hyperlink{ecldoc:file.comparefiles}{CompareFiles}
: Compares two files, and returns a result indicating how well they match
\item \hyperlink{ecldoc:file.verifyfile}{VerifyFile}
: Checks the system datastore (Dali) information for the file against the physical parts on disk
\item \hyperlink{ecldoc:file.addfilerelationship}{AddFileRelationship}
: Defines the relationship between two files
\item \hyperlink{ecldoc:file.filerelationshiplist}{FileRelationshipList}
: Returns a dataset of relationships
\item \hyperlink{ecldoc:file.removefilerelationship}{RemoveFileRelationship}
: Removes a relationship between two files
\item \hyperlink{ecldoc:file.getcolumnmapping}{GetColumnMapping}
: Returns the field mappings for the file, in the same format specified for the SetColumnMapping function
\item \hyperlink{ecldoc:file.setcolumnmapping}{SetColumnMapping}
: Defines how the data in the fields of the file mist be transformed between the actual data storage format and the input format used to query that data
\item \hyperlink{ecldoc:file.encoderfsquery}{EncodeRfsQuery}
: Returns a string that can be used in a DATASET declaration to read data from an RFS (Remote File Server) instance (e.g
\item \hyperlink{ecldoc:file.rfsaction}{RfsAction}
: Sends the query to the rfs server
\item \hyperlink{ecldoc:file.moveexternalfile}{MoveExternalFile}
: Moves the single physical file between two locations on the same remote machine
\item \hyperlink{ecldoc:file.deleteexternalfile}{DeleteExternalFile}
: Removes a single physical file from a remote machine
\item \hyperlink{ecldoc:file.createexternaldirectory}{CreateExternalDirectory}
: Creates the path on the location (if it does not already exist)
\item \hyperlink{ecldoc:file.getlogicalfileattribute}{GetLogicalFileAttribute}
: Returns the value of the given attribute for the specified logicalfilename
\item \hyperlink{ecldoc:file.protectlogicalfile}{ProtectLogicalFile}
: Toggles protection on and off for the specified logicalfilename
\item \hyperlink{ecldoc:file.dfuplusexec}{DfuPlusExec}
: The DfuPlusExec action executes the specified command line just as the DfuPLus.exe program would do
\item \hyperlink{ecldoc:file.fsprayfixed}{fSprayFixed}
: Sprays a file of fixed length records from a single machine and distributes it across the nodes of the destination group
\item \hyperlink{ecldoc:file.sprayfixed}{SprayFixed}
: Same as fSprayFixed, but does not return the DFU Workunit ID
\item \hyperlink{ecldoc:file.fsprayvariable}{fSprayVariable}
\item \hyperlink{ecldoc:file.sprayvariable}{SprayVariable}
\item \hyperlink{ecldoc:file.fspraydelimited}{fSprayDelimited}
: Sprays a file of fixed delimited records from a single machine and distributes it across the nodes of the destination group
\item \hyperlink{ecldoc:file.spraydelimited}{SprayDelimited}
: Same as fSprayDelimited, but does not return the DFU Workunit ID
\item \hyperlink{ecldoc:file.fsprayxml}{fSprayXml}
: Sprays an xml file from a single machine and distributes it across the nodes of the destination group
\item \hyperlink{ecldoc:file.sprayxml}{SprayXml}
: Same as fSprayXml, but does not return the DFU Workunit ID
\item \hyperlink{ecldoc:file.fdespray}{fDespray}
: Copies a distributed file from multiple machines, and desprays it to a single file on a single machine
\item \hyperlink{ecldoc:file.despray}{Despray}
: Same as fDespray, but does not return the DFU Workunit ID
\item \hyperlink{ecldoc:file.fcopy}{fCopy}
: Copies a distributed file to another distributed file
\item \hyperlink{ecldoc:file.copy}{Copy}
: Same as fCopy, but does not return the DFU Workunit ID
\item \hyperlink{ecldoc:file.freplicate}{fReplicate}
: Ensures the specified file is replicated to its mirror copies
\item \hyperlink{ecldoc:file.replicate}{Replicate}
: Same as fReplicated, but does not return the DFU Workunit ID
\item \hyperlink{ecldoc:file.fremotepull}{fRemotePull}
: Copies a distributed file to a distributed file on remote system
\item \hyperlink{ecldoc:file.remotepull}{RemotePull}
: Same as fRemotePull, but does not return the DFU Workunit ID
\item \hyperlink{ecldoc:file.fmonitorlogicalfilename}{fMonitorLogicalFileName}
: Creates a file monitor job in the DFU Server
\item \hyperlink{ecldoc:file.monitorlogicalfilename}{MonitorLogicalFileName}
: Same as fMonitorLogicalFileName, but does not return the DFU Workunit ID
\item \hyperlink{ecldoc:file.fmonitorfile}{fMonitorFile}
: Creates a file monitor job in the DFU Server
\item \hyperlink{ecldoc:file.monitorfile}{MonitorFile}
: Same as fMonitorFile, but does not return the DFU Workunit ID
\item \hyperlink{ecldoc:file.waitdfuworkunit}{WaitDfuWorkunit}
: Waits for the specified DFU workunit to finish
\item \hyperlink{ecldoc:file.abortdfuworkunit}{AbortDfuWorkunit}
: Aborts the specified DFU workunit
\item \hyperlink{ecldoc:file.createsuperfile}{CreateSuperFile}
: Creates an empty superfile
\item \hyperlink{ecldoc:file.superfileexists}{SuperFileExists}
: Checks if the specified filename is present in the Distributed File Utility (DFU) and is a SuperFile
\item \hyperlink{ecldoc:file.deletesuperfile}{DeleteSuperFile}
: Deletes the superfile
\item \hyperlink{ecldoc:file.getsuperfilesubcount}{GetSuperFileSubCount}
: Returns the number of sub-files contained within a superfile
\item \hyperlink{ecldoc:file.getsuperfilesubname}{GetSuperFileSubName}
: Returns the name of the Nth sub-file within a superfile
\item \hyperlink{ecldoc:file.findsuperfilesubname}{FindSuperFileSubName}
: Returns the position of a file within a superfile
\item \hyperlink{ecldoc:file.startsuperfiletransaction}{StartSuperFileTransaction}
: Starts a superfile transaction
\item \hyperlink{ecldoc:file.addsuperfile}{AddSuperFile}
: Adds a file to a superfile
\item \hyperlink{ecldoc:file.removesuperfile}{RemoveSuperFile}
: Removes a sub-file from a superfile
\item \hyperlink{ecldoc:file.clearsuperfile}{ClearSuperFile}
: Removes all sub-files from a superfile
\item \hyperlink{ecldoc:file.removeownedsubfiles}{RemoveOwnedSubFiles}
: Removes all soley-owned sub-files from a superfile
\item \hyperlink{ecldoc:file.deleteownedsubfiles}{DeleteOwnedSubFiles}
: Legacy version of RemoveOwnedSubFiles which was incorrectly named in a previous version
\item \hyperlink{ecldoc:file.swapsuperfile}{SwapSuperFile}
: Swap the contents of two superfiles
\item \hyperlink{ecldoc:file.replacesuperfile}{ReplaceSuperFile}
: Removes a sub-file from a superfile and replaces it with another
\item \hyperlink{ecldoc:file.finishsuperfiletransaction}{FinishSuperFileTransaction}
: Finishes a superfile transaction
\item \hyperlink{ecldoc:file.superfilecontents}{SuperFileContents}
: Returns the list of sub-files contained within a superfile
\item \hyperlink{ecldoc:file.logicalfilesuperowners}{LogicalFileSuperOwners}
: Returns the list of superfiles that a logical file is contained within
\item \hyperlink{ecldoc:file.logicalfilesupersublist}{LogicalFileSuperSubList}
: Returns the list of all the superfiles in the system and their component sub-files
\item \hyperlink{ecldoc:file.fpromotesuperfilelist}{fPromoteSuperFileList}
: Moves the sub-files from the first entry in the list of superfiles to the next in the list, repeating the process through the list of superfiles
\item \hyperlink{ecldoc:file.promotesuperfilelist}{PromoteSuperFileList}
: Same as fPromoteSuperFileList, but does not return the DFU Workunit ID
\end{enumerate}

\rule{\linewidth}{0.5pt}

\subsection*{\textsf{\colorbox{headtoc}{\color{white} RECORD}
FsFilenameRecord}}

\hypertarget{ecldoc:file.fsfilenamerecord}{}
\hspace{0pt} \hyperlink{ecldoc:File}{File} \textbackslash 

{\renewcommand{\arraystretch}{1.5}
\begin{tabularx}{\textwidth}{|>{\raggedright\arraybackslash}l|X|}
\hline
\hspace{0pt}\mytexttt{\color{red} } & \textbf{FsFilenameRecord} \\
\hline
\end{tabularx}
}

\par
A record containing information about filename. Includes name, size and when last modified. export FsFilenameRecord := RECORD string name; integer8 size; string19 modified; END;


\rule{\linewidth}{0.5pt}
\subsection*{\textsf{\colorbox{headtoc}{\color{white} ATTRIBUTE}
FsLogicalFileName}}

\hypertarget{ecldoc:file.fslogicalfilename}{}
\hspace{0pt} \hyperlink{ecldoc:File}{File} \textbackslash 

{\renewcommand{\arraystretch}{1.5}
\begin{tabularx}{\textwidth}{|>{\raggedright\arraybackslash}l|X|}
\hline
\hspace{0pt}\mytexttt{\color{red} } & \textbf{FsLogicalFileName} \\
\hline
\end{tabularx}
}

\par
An alias for a logical filename that is stored in a row.


\rule{\linewidth}{0.5pt}
\subsection*{\textsf{\colorbox{headtoc}{\color{white} RECORD}
FsLogicalFileNameRecord}}

\hypertarget{ecldoc:file.fslogicalfilenamerecord}{}
\hspace{0pt} \hyperlink{ecldoc:File}{File} \textbackslash 

{\renewcommand{\arraystretch}{1.5}
\begin{tabularx}{\textwidth}{|>{\raggedright\arraybackslash}l|X|}
\hline
\hspace{0pt}\mytexttt{\color{red} } & \textbf{FsLogicalFileNameRecord} \\
\hline
\end{tabularx}
}

\par
A record containing a logical filename. It contains the following fields:

\par
\begin{description}
\item [\colorbox{tagtype}{\color{white} \textbf{\textsf{FIELD}}}] \textbf{\underline{name}} The logical name of the file;
\end{description}

\rule{\linewidth}{0.5pt}
\subsection*{\textsf{\colorbox{headtoc}{\color{white} RECORD}
FsLogicalFileInfoRecord}}

\hypertarget{ecldoc:file.fslogicalfileinforecord}{}
\hspace{0pt} \hyperlink{ecldoc:File}{File} \textbackslash 

{\renewcommand{\arraystretch}{1.5}
\begin{tabularx}{\textwidth}{|>{\raggedright\arraybackslash}l|X|}
\hline
\hspace{0pt}\mytexttt{\color{red} } & \textbf{FsLogicalFileInfoRecord} \\
\hline
\end{tabularx}
}

\par
A record containing information about a logical file.

\par
\begin{description}
\item [\colorbox{tagtype}{\color{white} \textbf{\textsf{FIELD}}}] \textbf{\underline{superfile}} Is this a superfile?
\item [\colorbox{tagtype}{\color{white} \textbf{\textsf{FIELD}}}] \textbf{\underline{size}} Number of bytes in the file (before compression)
\item [\colorbox{tagtype}{\color{white} \textbf{\textsf{FIELD}}}] \textbf{\underline{rowcount}} Number of rows in the file.
\end{description}

\rule{\linewidth}{0.5pt}
\subsection*{\textsf{\colorbox{headtoc}{\color{white} RECORD}
FsLogicalSuperSubRecord}}

\hypertarget{ecldoc:file.fslogicalsupersubrecord}{}
\hspace{0pt} \hyperlink{ecldoc:File}{File} \textbackslash 

{\renewcommand{\arraystretch}{1.5}
\begin{tabularx}{\textwidth}{|>{\raggedright\arraybackslash}l|X|}
\hline
\hspace{0pt}\mytexttt{\color{red} } & \textbf{FsLogicalSuperSubRecord} \\
\hline
\end{tabularx}
}

\par
A record containing information about a superfile and its contents.

\par
\begin{description}
\item [\colorbox{tagtype}{\color{white} \textbf{\textsf{FIELD}}}] \textbf{\underline{supername}} The name of the superfile
\item [\colorbox{tagtype}{\color{white} \textbf{\textsf{FIELD}}}] \textbf{\underline{subname}} The name of the sub-file
\end{description}

\rule{\linewidth}{0.5pt}
\subsection*{\textsf{\colorbox{headtoc}{\color{white} RECORD}
FsFileRelationshipRecord}}

\hypertarget{ecldoc:file.fsfilerelationshiprecord}{}
\hspace{0pt} \hyperlink{ecldoc:File}{File} \textbackslash 

{\renewcommand{\arraystretch}{1.5}
\begin{tabularx}{\textwidth}{|>{\raggedright\arraybackslash}l|X|}
\hline
\hspace{0pt}\mytexttt{\color{red} } & \textbf{FsFileRelationshipRecord} \\
\hline
\end{tabularx}
}

\par
A record containing information about the relationship between two files.

\par
\begin{description}
\item [\colorbox{tagtype}{\color{white} \textbf{\textsf{FIELD}}}] \textbf{\underline{primaryfile}} The logical filename of the primary file
\item [\colorbox{tagtype}{\color{white} \textbf{\textsf{FIELD}}}] \textbf{\underline{secondaryfile}} The logical filename of the secondary file.
\item [\colorbox{tagtype}{\color{white} \textbf{\textsf{FIELD}}}] \textbf{\underline{primaryflds}} The name of the primary key field for the primary file. The value ''\_\_fileposition\_\_'' indicates the secondary is an INDEX that must use FETCH to access non-keyed fields.
\item [\colorbox{tagtype}{\color{white} \textbf{\textsf{FIELD}}}] \textbf{\underline{secondaryflds}} The name of the foreign key field relating to the primary file.
\item [\colorbox{tagtype}{\color{white} \textbf{\textsf{FIELD}}}] \textbf{\underline{kind}} The type of relationship between the primary and secondary files. Containing either 'link' or 'view'.
\item [\colorbox{tagtype}{\color{white} \textbf{\textsf{FIELD}}}] \textbf{\underline{cardinality}} The cardinality of the relationship. The format is <primary>:<secondary>. Valid values are ''1'' or ''M''.</secondary></primary>
\item [\colorbox{tagtype}{\color{white} \textbf{\textsf{FIELD}}}] \textbf{\underline{payload}} Indicates whether the primary or secondary are payload INDEXes.
\item [\colorbox{tagtype}{\color{white} \textbf{\textsf{FIELD}}}] \textbf{\underline{description}} The description of the relationship.
\end{description}

\rule{\linewidth}{0.5pt}
\subsection*{\textsf{\colorbox{headtoc}{\color{white} ATTRIBUTE}
RECFMV\_RECSIZE}}

\hypertarget{ecldoc:file.recfmv_recsize}{}
\hspace{0pt} \hyperlink{ecldoc:File}{File} \textbackslash 

{\renewcommand{\arraystretch}{1.5}
\begin{tabularx}{\textwidth}{|>{\raggedright\arraybackslash}l|X|}
\hline
\hspace{0pt}\mytexttt{\color{red} } & \textbf{RECFMV\_RECSIZE} \\
\hline
\end{tabularx}
}

\par
Constant that indicates IBM RECFM V format file. Can be passed to SprayFixed for the record size.


\rule{\linewidth}{0.5pt}
\subsection*{\textsf{\colorbox{headtoc}{\color{white} ATTRIBUTE}
RECFMVB\_RECSIZE}}

\hypertarget{ecldoc:file.recfmvb_recsize}{}
\hspace{0pt} \hyperlink{ecldoc:File}{File} \textbackslash 

{\renewcommand{\arraystretch}{1.5}
\begin{tabularx}{\textwidth}{|>{\raggedright\arraybackslash}l|X|}
\hline
\hspace{0pt}\mytexttt{\color{red} } & \textbf{RECFMVB\_RECSIZE} \\
\hline
\end{tabularx}
}

\par
Constant that indicates IBM RECFM VB format file. Can be passed to SprayFixed for the record size.


\rule{\linewidth}{0.5pt}
\subsection*{\textsf{\colorbox{headtoc}{\color{white} ATTRIBUTE}
PREFIX\_VARIABLE\_RECSIZE}}

\hypertarget{ecldoc:file.prefix_variable_recsize}{}
\hspace{0pt} \hyperlink{ecldoc:File}{File} \textbackslash 

{\renewcommand{\arraystretch}{1.5}
\begin{tabularx}{\textwidth}{|>{\raggedright\arraybackslash}l|X|}
\hline
\hspace{0pt}\mytexttt{\color{red} INTEGER4} & \textbf{PREFIX\_VARIABLE\_RECSIZE} \\
\hline
\end{tabularx}
}

\par
Constant that indicates a variable little endian 4 byte length prefixed file. Can be passed to SprayFixed for the record size.


\rule{\linewidth}{0.5pt}
\subsection*{\textsf{\colorbox{headtoc}{\color{white} ATTRIBUTE}
PREFIX\_VARIABLE\_BIGENDIAN\_RECSIZE}}

\hypertarget{ecldoc:file.prefix_variable_bigendian_recsize}{}
\hspace{0pt} \hyperlink{ecldoc:File}{File} \textbackslash 

{\renewcommand{\arraystretch}{1.5}
\begin{tabularx}{\textwidth}{|>{\raggedright\arraybackslash}l|X|}
\hline
\hspace{0pt}\mytexttt{\color{red} INTEGER4} & \textbf{PREFIX\_VARIABLE\_BIGENDIAN\_RECSIZE} \\
\hline
\end{tabularx}
}

\par
Constant that indicates a variable big endian 4 byte length prefixed file. Can be passed to SprayFixed for the record size.


\rule{\linewidth}{0.5pt}
\subsection*{\textsf{\colorbox{headtoc}{\color{white} FUNCTION}
FileExists}}

\hypertarget{ecldoc:file.fileexists}{}
\hspace{0pt} \hyperlink{ecldoc:File}{File} \textbackslash 

{\renewcommand{\arraystretch}{1.5}
\begin{tabularx}{\textwidth}{|>{\raggedright\arraybackslash}l|X|}
\hline
\hspace{0pt}\mytexttt{\color{red} boolean} & \textbf{FileExists} \\
\hline
\multicolumn{2}{|>{\raggedright\arraybackslash}X|}{\hspace{0pt}\mytexttt{\color{param} (varstring lfn, boolean physical=FALSE)}} \\
\hline
\end{tabularx}
}

\par
Returns whether the file exists.

\par
\begin{description}
\item [\colorbox{tagtype}{\color{white} \textbf{\textsf{PARAMETER}}}] \textbf{\underline{lfn}} The logical name of the file.
\item [\colorbox{tagtype}{\color{white} \textbf{\textsf{PARAMETER}}}] \textbf{\underline{physical}} Whether to also check for the physical existence on disk. Defaults to FALSE.
\item [\colorbox{tagtype}{\color{white} \textbf{\textsf{RETURN}}}] \textbf{\underline{}} Whether the file exists.
\end{description}

\rule{\linewidth}{0.5pt}
\subsection*{\textsf{\colorbox{headtoc}{\color{white} FUNCTION}
DeleteLogicalFile}}

\hypertarget{ecldoc:file.deletelogicalfile}{}
\hspace{0pt} \hyperlink{ecldoc:File}{File} \textbackslash 

{\renewcommand{\arraystretch}{1.5}
\begin{tabularx}{\textwidth}{|>{\raggedright\arraybackslash}l|X|}
\hline
\hspace{0pt}\mytexttt{\color{red} } & \textbf{DeleteLogicalFile} \\
\hline
\multicolumn{2}{|>{\raggedright\arraybackslash}X|}{\hspace{0pt}\mytexttt{\color{param} (varstring lfn, boolean allowMissing=FALSE)}} \\
\hline
\end{tabularx}
}

\par
Removes the logical file from the system, and deletes from the disk.

\par
\begin{description}
\item [\colorbox{tagtype}{\color{white} \textbf{\textsf{PARAMETER}}}] \textbf{\underline{lfn}} The logical name of the file.
\item [\colorbox{tagtype}{\color{white} \textbf{\textsf{PARAMETER}}}] \textbf{\underline{allowMissing}} Whether to suppress an error if the filename does not exist. Defaults to FALSE.
\end{description}

\rule{\linewidth}{0.5pt}
\subsection*{\textsf{\colorbox{headtoc}{\color{white} FUNCTION}
SetReadOnly}}

\hypertarget{ecldoc:file.setreadonly}{}
\hspace{0pt} \hyperlink{ecldoc:File}{File} \textbackslash 

{\renewcommand{\arraystretch}{1.5}
\begin{tabularx}{\textwidth}{|>{\raggedright\arraybackslash}l|X|}
\hline
\hspace{0pt}\mytexttt{\color{red} } & \textbf{SetReadOnly} \\
\hline
\multicolumn{2}{|>{\raggedright\arraybackslash}X|}{\hspace{0pt}\mytexttt{\color{param} (varstring lfn, boolean ro=TRUE)}} \\
\hline
\end{tabularx}
}

\par
Changes whether access to a file is read only or not.

\par
\begin{description}
\item [\colorbox{tagtype}{\color{white} \textbf{\textsf{PARAMETER}}}] \textbf{\underline{lfn}} The logical name of the file.
\item [\colorbox{tagtype}{\color{white} \textbf{\textsf{PARAMETER}}}] \textbf{\underline{ro}} Whether updates to the file are disallowed. Defaults to TRUE.
\end{description}

\rule{\linewidth}{0.5pt}
\subsection*{\textsf{\colorbox{headtoc}{\color{white} FUNCTION}
RenameLogicalFile}}

\hypertarget{ecldoc:file.renamelogicalfile}{}
\hspace{0pt} \hyperlink{ecldoc:File}{File} \textbackslash 

{\renewcommand{\arraystretch}{1.5}
\begin{tabularx}{\textwidth}{|>{\raggedright\arraybackslash}l|X|}
\hline
\hspace{0pt}\mytexttt{\color{red} } & \textbf{RenameLogicalFile} \\
\hline
\multicolumn{2}{|>{\raggedright\arraybackslash}X|}{\hspace{0pt}\mytexttt{\color{param} (varstring oldname, varstring newname)}} \\
\hline
\end{tabularx}
}

\par
Changes the name of a logical file.

\par
\begin{description}
\item [\colorbox{tagtype}{\color{white} \textbf{\textsf{PARAMETER}}}] \textbf{\underline{oldname}} The current name of the file to be renamed.
\item [\colorbox{tagtype}{\color{white} \textbf{\textsf{PARAMETER}}}] \textbf{\underline{newname}} The new logical name of the file.
\end{description}

\rule{\linewidth}{0.5pt}
\subsection*{\textsf{\colorbox{headtoc}{\color{white} FUNCTION}
ForeignLogicalFileName}}

\hypertarget{ecldoc:file.foreignlogicalfilename}{}
\hspace{0pt} \hyperlink{ecldoc:File}{File} \textbackslash 

{\renewcommand{\arraystretch}{1.5}
\begin{tabularx}{\textwidth}{|>{\raggedright\arraybackslash}l|X|}
\hline
\hspace{0pt}\mytexttt{\color{red} varstring} & \textbf{ForeignLogicalFileName} \\
\hline
\multicolumn{2}{|>{\raggedright\arraybackslash}X|}{\hspace{0pt}\mytexttt{\color{param} (varstring name, varstring foreigndali='', boolean abspath=FALSE)}} \\
\hline
\end{tabularx}
}

\par
Returns a logical filename that can be used to refer to a logical file in a local or remote dali.

\par
\begin{description}
\item [\colorbox{tagtype}{\color{white} \textbf{\textsf{PARAMETER}}}] \textbf{\underline{name}} The logical name of the file.
\item [\colorbox{tagtype}{\color{white} \textbf{\textsf{PARAMETER}}}] \textbf{\underline{foreigndali}} The IP address of the foreign dali used to resolve the file. If blank then the file is resolved locally. Defaults to blank.
\item [\colorbox{tagtype}{\color{white} \textbf{\textsf{PARAMETER}}}] \textbf{\underline{abspath}} Should a tilde (\~{}) be prepended to the resulting logical file name. Defaults to FALSE.
\end{description}

\rule{\linewidth}{0.5pt}
\subsection*{\textsf{\colorbox{headtoc}{\color{white} FUNCTION}
ExternalLogicalFileName}}

\hypertarget{ecldoc:file.externallogicalfilename}{}
\hspace{0pt} \hyperlink{ecldoc:File}{File} \textbackslash 

{\renewcommand{\arraystretch}{1.5}
\begin{tabularx}{\textwidth}{|>{\raggedright\arraybackslash}l|X|}
\hline
\hspace{0pt}\mytexttt{\color{red} varstring} & \textbf{ExternalLogicalFileName} \\
\hline
\multicolumn{2}{|>{\raggedright\arraybackslash}X|}{\hspace{0pt}\mytexttt{\color{param} (varstring location, varstring path, boolean abspath=TRUE)}} \\
\hline
\end{tabularx}
}

\par
Returns an encoded logical filename that can be used to refer to a external file. Examples include directly reading from a landing zone. Upper case characters and other details are escaped.

\par
\begin{description}
\item [\colorbox{tagtype}{\color{white} \textbf{\textsf{PARAMETER}}}] \textbf{\underline{location}} The IP address of the remote machine. '.' can be used for the local machine.
\item [\colorbox{tagtype}{\color{white} \textbf{\textsf{PARAMETER}}}] \textbf{\underline{path}} The path/name of the file on the remote machine.
\item [\colorbox{tagtype}{\color{white} \textbf{\textsf{PARAMETER}}}] \textbf{\underline{abspath}} Should a tilde (\~{}) be prepended to the resulting logical file name. Defaults to TRUE.
\item [\colorbox{tagtype}{\color{white} \textbf{\textsf{RETURN}}}] \textbf{\underline{}} The encoded logical filename.
\end{description}

\rule{\linewidth}{0.5pt}
\subsection*{\textsf{\colorbox{headtoc}{\color{white} FUNCTION}
GetFileDescription}}

\hypertarget{ecldoc:file.getfiledescription}{}
\hspace{0pt} \hyperlink{ecldoc:File}{File} \textbackslash 

{\renewcommand{\arraystretch}{1.5}
\begin{tabularx}{\textwidth}{|>{\raggedright\arraybackslash}l|X|}
\hline
\hspace{0pt}\mytexttt{\color{red} varstring} & \textbf{GetFileDescription} \\
\hline
\multicolumn{2}{|>{\raggedright\arraybackslash}X|}{\hspace{0pt}\mytexttt{\color{param} (varstring lfn)}} \\
\hline
\end{tabularx}
}

\par
Returns a string containing the description information associated with the specified filename. This description is set either through ECL watch or by using the FileServices.SetFileDescription function.

\par
\begin{description}
\item [\colorbox{tagtype}{\color{white} \textbf{\textsf{PARAMETER}}}] \textbf{\underline{lfn}} The logical name of the file.
\end{description}

\rule{\linewidth}{0.5pt}
\subsection*{\textsf{\colorbox{headtoc}{\color{white} FUNCTION}
SetFileDescription}}

\hypertarget{ecldoc:file.setfiledescription}{}
\hspace{0pt} \hyperlink{ecldoc:File}{File} \textbackslash 

{\renewcommand{\arraystretch}{1.5}
\begin{tabularx}{\textwidth}{|>{\raggedright\arraybackslash}l|X|}
\hline
\hspace{0pt}\mytexttt{\color{red} } & \textbf{SetFileDescription} \\
\hline
\multicolumn{2}{|>{\raggedright\arraybackslash}X|}{\hspace{0pt}\mytexttt{\color{param} (varstring lfn, varstring val)}} \\
\hline
\end{tabularx}
}

\par
Sets the description associated with the specified filename.

\par
\begin{description}
\item [\colorbox{tagtype}{\color{white} \textbf{\textsf{PARAMETER}}}] \textbf{\underline{lfn}} The logical name of the file.
\item [\colorbox{tagtype}{\color{white} \textbf{\textsf{PARAMETER}}}] \textbf{\underline{val}} The description to be associated with the file.
\end{description}

\rule{\linewidth}{0.5pt}
\subsection*{\textsf{\colorbox{headtoc}{\color{white} FUNCTION}
RemoteDirectory}}

\hypertarget{ecldoc:file.remotedirectory}{}
\hspace{0pt} \hyperlink{ecldoc:File}{File} \textbackslash 

{\renewcommand{\arraystretch}{1.5}
\begin{tabularx}{\textwidth}{|>{\raggedright\arraybackslash}l|X|}
\hline
\hspace{0pt}\mytexttt{\color{red} dataset(FsFilenameRecord)} & \textbf{RemoteDirectory} \\
\hline
\multicolumn{2}{|>{\raggedright\arraybackslash}X|}{\hspace{0pt}\mytexttt{\color{param} (varstring machineIP, varstring dir, varstring mask='*', boolean recurse=FALSE)}} \\
\hline
\end{tabularx}
}

\par
Returns a dataset containing a list of files from the specified machineIP and directory.

\par
\begin{description}
\item [\colorbox{tagtype}{\color{white} \textbf{\textsf{PARAMETER}}}] \textbf{\underline{machineIP}} The IP address of the remote machine.
\item [\colorbox{tagtype}{\color{white} \textbf{\textsf{PARAMETER}}}] \textbf{\underline{directory}} The path to the directory to read. This must be in the appropriate format for the operating system running on the remote machine.
\item [\colorbox{tagtype}{\color{white} \textbf{\textsf{PARAMETER}}}] \textbf{\underline{mask}} The filemask specifying which files to include in the result. Defaults to '*' (all files).
\item [\colorbox{tagtype}{\color{white} \textbf{\textsf{PARAMETER}}}] \textbf{\underline{recurse}} Whether to include files from subdirectories under the directory. Defaults to FALSE.
\end{description}

\rule{\linewidth}{0.5pt}
\subsection*{\textsf{\colorbox{headtoc}{\color{white} FUNCTION}
LogicalFileList}}

\hypertarget{ecldoc:file.logicalfilelist}{}
\hspace{0pt} \hyperlink{ecldoc:File}{File} \textbackslash 

{\renewcommand{\arraystretch}{1.5}
\begin{tabularx}{\textwidth}{|>{\raggedright\arraybackslash}l|X|}
\hline
\hspace{0pt}\mytexttt{\color{red} dataset(FsLogicalFileInfoRecord)} & \textbf{LogicalFileList} \\
\hline
\multicolumn{2}{|>{\raggedright\arraybackslash}X|}{\hspace{0pt}\mytexttt{\color{param} (varstring namepattern='*', boolean includenormal=TRUE, boolean includesuper=FALSE, boolean unknownszero=FALSE, varstring foreigndali='')}} \\
\hline
\end{tabularx}
}

\par
Returns a dataset of information about the logical files known to the system.

\par
\begin{description}
\item [\colorbox{tagtype}{\color{white} \textbf{\textsf{PARAMETER}}}] \textbf{\underline{namepattern}} The mask of the files to list. Defaults to '*' (all files).
\item [\colorbox{tagtype}{\color{white} \textbf{\textsf{PARAMETER}}}] \textbf{\underline{includenormal}} Whether to include 'normal' files. Defaults to TRUE.
\item [\colorbox{tagtype}{\color{white} \textbf{\textsf{PARAMETER}}}] \textbf{\underline{includesuper}} Whether to include SuperFiles. Defaults to FALSE.
\item [\colorbox{tagtype}{\color{white} \textbf{\textsf{PARAMETER}}}] \textbf{\underline{unknownszero}} Whether to set file sizes that are unknown to zero(0) instead of minus-one (-1). Defaults to FALSE.
\item [\colorbox{tagtype}{\color{white} \textbf{\textsf{PARAMETER}}}] \textbf{\underline{foreigndali}} The IP address of the foreign dali used to resolve the file. If blank then the file is resolved locally. Defaults to blank.
\end{description}

\rule{\linewidth}{0.5pt}
\subsection*{\textsf{\colorbox{headtoc}{\color{white} FUNCTION}
CompareFiles}}

\hypertarget{ecldoc:file.comparefiles}{}
\hspace{0pt} \hyperlink{ecldoc:File}{File} \textbackslash 

{\renewcommand{\arraystretch}{1.5}
\begin{tabularx}{\textwidth}{|>{\raggedright\arraybackslash}l|X|}
\hline
\hspace{0pt}\mytexttt{\color{red} INTEGER4} & \textbf{CompareFiles} \\
\hline
\multicolumn{2}{|>{\raggedright\arraybackslash}X|}{\hspace{0pt}\mytexttt{\color{param} (varstring lfn1, varstring lfn2, boolean logical\_only=TRUE, boolean use\_crcs=FALSE)}} \\
\hline
\end{tabularx}
}

\par
Compares two files, and returns a result indicating how well they match.

\par
\begin{description}
\item [\colorbox{tagtype}{\color{white} \textbf{\textsf{PARAMETER}}}] \textbf{\underline{file1}} The logical name of the first file.
\item [\colorbox{tagtype}{\color{white} \textbf{\textsf{PARAMETER}}}] \textbf{\underline{file2}} The logical name of the second file.
\item [\colorbox{tagtype}{\color{white} \textbf{\textsf{PARAMETER}}}] \textbf{\underline{logical\_only}} Whether to only compare logical information in the system datastore (Dali), and ignore physical information on disk. [Default TRUE]
\item [\colorbox{tagtype}{\color{white} \textbf{\textsf{PARAMETER}}}] \textbf{\underline{use\_crcs}} Whether to compare physical CRCs of all the parts on disk. This may be slow on large files. Defaults to FALSE.
\item [\colorbox{tagtype}{\color{white} \textbf{\textsf{RETURN}}}] \textbf{\underline{}} 0 if file1 and file2 match exactly 1 if file1 and file2 contents match, but file1 is newer than file2 -1 if file1 and file2 contents match, but file2 is newer than file1 2 if file1 and file2 contents do not match and file1 is newer than file2 -2 if file1 and file2 contents do not match and file2 is newer than file1
\end{description}

\rule{\linewidth}{0.5pt}
\subsection*{\textsf{\colorbox{headtoc}{\color{white} FUNCTION}
VerifyFile}}

\hypertarget{ecldoc:file.verifyfile}{}
\hspace{0pt} \hyperlink{ecldoc:File}{File} \textbackslash 

{\renewcommand{\arraystretch}{1.5}
\begin{tabularx}{\textwidth}{|>{\raggedright\arraybackslash}l|X|}
\hline
\hspace{0pt}\mytexttt{\color{red} varstring} & \textbf{VerifyFile} \\
\hline
\multicolumn{2}{|>{\raggedright\arraybackslash}X|}{\hspace{0pt}\mytexttt{\color{param} (varstring lfn, boolean usecrcs)}} \\
\hline
\end{tabularx}
}

\par
Checks the system datastore (Dali) information for the file against the physical parts on disk.

\par
\begin{description}
\item [\colorbox{tagtype}{\color{white} \textbf{\textsf{PARAMETER}}}] \textbf{\underline{lfn}} The name of the file to check.
\item [\colorbox{tagtype}{\color{white} \textbf{\textsf{PARAMETER}}}] \textbf{\underline{use\_crcs}} Whether to compare physical CRCs of all the parts on disk. This may be slow on large files.
\item [\colorbox{tagtype}{\color{white} \textbf{\textsf{RETURN}}}] \textbf{\underline{}} 'OK' - The file parts match the datastore information 'Could not find file: <filename>' - The logical filename was not found 'Could not find part file: <partname>' - The partname was not found 'Modified time differs for: <partname>' - The partname has a different timestamp 'File size differs for: <partname>' - The partname has a file size 'File CRC differs for: <partname>' - The partname has a different CRC</partname></partname></partname></partname></filename>
\end{description}

\rule{\linewidth}{0.5pt}
\subsection*{\textsf{\colorbox{headtoc}{\color{white} FUNCTION}
AddFileRelationship}}

\hypertarget{ecldoc:file.addfilerelationship}{}
\hspace{0pt} \hyperlink{ecldoc:File}{File} \textbackslash 

{\renewcommand{\arraystretch}{1.5}
\begin{tabularx}{\textwidth}{|>{\raggedright\arraybackslash}l|X|}
\hline
\hspace{0pt}\mytexttt{\color{red} } & \textbf{AddFileRelationship} \\
\hline
\multicolumn{2}{|>{\raggedright\arraybackslash}X|}{\hspace{0pt}\mytexttt{\color{param} (varstring primary, varstring secondary, varstring primaryflds, varstring secondaryflds, varstring kind='link', varstring cardinality, boolean payload, varstring description='')}} \\
\hline
\end{tabularx}
}

\par
Defines the relationship between two files. These may be DATASETs or INDEXes. Each record in the primary file should be uniquely defined by the primaryfields (ideally), preferably efficiently. This information is used by the roxie browser to link files together.

\par
\begin{description}
\item [\colorbox{tagtype}{\color{white} \textbf{\textsf{PARAMETER}}}] \textbf{\underline{primary}} The logical filename of the primary file.
\item [\colorbox{tagtype}{\color{white} \textbf{\textsf{PARAMETER}}}] \textbf{\underline{secondary}} The logical filename of the secondary file.
\item [\colorbox{tagtype}{\color{white} \textbf{\textsf{PARAMETER}}}] \textbf{\underline{primaryfields}} The name of the primary key field for the primary file. The value ''\_\_fileposition\_\_'' indicates the secondary is an INDEX that must use FETCH to access non-keyed fields.
\item [\colorbox{tagtype}{\color{white} \textbf{\textsf{PARAMETER}}}] \textbf{\underline{secondaryfields}} The name of the foreign key field relating to the primary file.
\item [\colorbox{tagtype}{\color{white} \textbf{\textsf{PARAMETER}}}] \textbf{\underline{relationship}} The type of relationship between the primary and secondary files. Containing either 'link' or 'view'. Default is ''link''.
\item [\colorbox{tagtype}{\color{white} \textbf{\textsf{PARAMETER}}}] \textbf{\underline{cardinality}} The cardinality of the relationship. The format is <primary>:<secondary>. Valid values are ''1'' or ''M''.</secondary></primary>
\item [\colorbox{tagtype}{\color{white} \textbf{\textsf{PARAMETER}}}] \textbf{\underline{payload}} Indicates whether the primary or secondary are payload INDEXes.
\item [\colorbox{tagtype}{\color{white} \textbf{\textsf{PARAMETER}}}] \textbf{\underline{description}} The description of the relationship.
\end{description}

\rule{\linewidth}{0.5pt}
\subsection*{\textsf{\colorbox{headtoc}{\color{white} FUNCTION}
FileRelationshipList}}

\hypertarget{ecldoc:file.filerelationshiplist}{}
\hspace{0pt} \hyperlink{ecldoc:File}{File} \textbackslash 

{\renewcommand{\arraystretch}{1.5}
\begin{tabularx}{\textwidth}{|>{\raggedright\arraybackslash}l|X|}
\hline
\hspace{0pt}\mytexttt{\color{red} dataset(FsFileRelationshipRecord)} & \textbf{FileRelationshipList} \\
\hline
\multicolumn{2}{|>{\raggedright\arraybackslash}X|}{\hspace{0pt}\mytexttt{\color{param} (varstring primary, varstring secondary, varstring primflds='', varstring secondaryflds='', varstring kind='link')}} \\
\hline
\end{tabularx}
}

\par
Returns a dataset of relationships. The return records are structured in the FsFileRelationshipRecord format.

\par
\begin{description}
\item [\colorbox{tagtype}{\color{white} \textbf{\textsf{PARAMETER}}}] \textbf{\underline{primary}} The logical filename of the primary file.
\item [\colorbox{tagtype}{\color{white} \textbf{\textsf{PARAMETER}}}] \textbf{\underline{secondary}} The logical filename of the secondary file.
\item [\colorbox{tagtype}{\color{white} \textbf{\textsf{PARAMETER}}}] \textbf{\underline{primaryfields}} The name of the primary key field for the primary file.
\item [\colorbox{tagtype}{\color{white} \textbf{\textsf{PARAMETER}}}] \textbf{\underline{secondaryfields}} The name of the foreign key field relating to the primary file.
\item [\colorbox{tagtype}{\color{white} \textbf{\textsf{PARAMETER}}}] \textbf{\underline{relationship}} The type of relationship between the primary and secondary files. Containing either 'link' or 'view'. Default is ''link''.
\end{description}

\rule{\linewidth}{0.5pt}
\subsection*{\textsf{\colorbox{headtoc}{\color{white} FUNCTION}
RemoveFileRelationship}}

\hypertarget{ecldoc:file.removefilerelationship}{}
\hspace{0pt} \hyperlink{ecldoc:File}{File} \textbackslash 

{\renewcommand{\arraystretch}{1.5}
\begin{tabularx}{\textwidth}{|>{\raggedright\arraybackslash}l|X|}
\hline
\hspace{0pt}\mytexttt{\color{red} } & \textbf{RemoveFileRelationship} \\
\hline
\multicolumn{2}{|>{\raggedright\arraybackslash}X|}{\hspace{0pt}\mytexttt{\color{param} (varstring primary, varstring secondary, varstring primaryflds='', varstring secondaryflds='', varstring kind='link')}} \\
\hline
\end{tabularx}
}

\par
Removes a relationship between two files.

\par
\begin{description}
\item [\colorbox{tagtype}{\color{white} \textbf{\textsf{PARAMETER}}}] \textbf{\underline{primary}} The logical filename of the primary file.
\item [\colorbox{tagtype}{\color{white} \textbf{\textsf{PARAMETER}}}] \textbf{\underline{secondary}} The logical filename of the secondary file.
\item [\colorbox{tagtype}{\color{white} \textbf{\textsf{PARAMETER}}}] \textbf{\underline{primaryfields}} The name of the primary key field for the primary file.
\item [\colorbox{tagtype}{\color{white} \textbf{\textsf{PARAMETER}}}] \textbf{\underline{secondaryfields}} The name of the foreign key field relating to the primary file.
\item [\colorbox{tagtype}{\color{white} \textbf{\textsf{PARAMETER}}}] \textbf{\underline{relationship}} The type of relationship between the primary and secondary files. Containing either 'link' or 'view'. Default is ''link''.
\end{description}

\rule{\linewidth}{0.5pt}
\subsection*{\textsf{\colorbox{headtoc}{\color{white} FUNCTION}
GetColumnMapping}}

\hypertarget{ecldoc:file.getcolumnmapping}{}
\hspace{0pt} \hyperlink{ecldoc:File}{File} \textbackslash 

{\renewcommand{\arraystretch}{1.5}
\begin{tabularx}{\textwidth}{|>{\raggedright\arraybackslash}l|X|}
\hline
\hspace{0pt}\mytexttt{\color{red} varstring} & \textbf{GetColumnMapping} \\
\hline
\multicolumn{2}{|>{\raggedright\arraybackslash}X|}{\hspace{0pt}\mytexttt{\color{param} (varstring lfn)}} \\
\hline
\end{tabularx}
}

\par
Returns the field mappings for the file, in the same format specified for the SetColumnMapping function.

\par
\begin{description}
\item [\colorbox{tagtype}{\color{white} \textbf{\textsf{PARAMETER}}}] \textbf{\underline{lfn}} The logical filename of the primary file.
\end{description}

\rule{\linewidth}{0.5pt}
\subsection*{\textsf{\colorbox{headtoc}{\color{white} FUNCTION}
SetColumnMapping}}

\hypertarget{ecldoc:file.setcolumnmapping}{}
\hspace{0pt} \hyperlink{ecldoc:File}{File} \textbackslash 

{\renewcommand{\arraystretch}{1.5}
\begin{tabularx}{\textwidth}{|>{\raggedright\arraybackslash}l|X|}
\hline
\hspace{0pt}\mytexttt{\color{red} } & \textbf{SetColumnMapping} \\
\hline
\multicolumn{2}{|>{\raggedright\arraybackslash}X|}{\hspace{0pt}\mytexttt{\color{param} (varstring lfn, varstring mapping)}} \\
\hline
\end{tabularx}
}

\par
Defines how the data in the fields of the file mist be transformed between the actual data storage format and the input format used to query that data. This is used by the user interface of the roxie browser.

\par
\begin{description}
\item [\colorbox{tagtype}{\color{white} \textbf{\textsf{PARAMETER}}}] \textbf{\underline{lfn}} The logical filename of the primary file.
\item [\colorbox{tagtype}{\color{white} \textbf{\textsf{PARAMETER}}}] \textbf{\underline{mapping}} A string containing a comma separated list of field mappings.
\end{description}

\rule{\linewidth}{0.5pt}
\subsection*{\textsf{\colorbox{headtoc}{\color{white} FUNCTION}
EncodeRfsQuery}}

\hypertarget{ecldoc:file.encoderfsquery}{}
\hspace{0pt} \hyperlink{ecldoc:File}{File} \textbackslash 

{\renewcommand{\arraystretch}{1.5}
\begin{tabularx}{\textwidth}{|>{\raggedright\arraybackslash}l|X|}
\hline
\hspace{0pt}\mytexttt{\color{red} varstring} & \textbf{EncodeRfsQuery} \\
\hline
\multicolumn{2}{|>{\raggedright\arraybackslash}X|}{\hspace{0pt}\mytexttt{\color{param} (varstring server, varstring query)}} \\
\hline
\end{tabularx}
}

\par
Returns a string that can be used in a DATASET declaration to read data from an RFS (Remote File Server) instance (e.g. rfsmysql) on another node.

\par
\begin{description}
\item [\colorbox{tagtype}{\color{white} \textbf{\textsf{PARAMETER}}}] \textbf{\underline{server}} A string containing the ip:port address for the remote file server.
\item [\colorbox{tagtype}{\color{white} \textbf{\textsf{PARAMETER}}}] \textbf{\underline{query}} The text of the query to send to the server
\end{description}

\rule{\linewidth}{0.5pt}
\subsection*{\textsf{\colorbox{headtoc}{\color{white} FUNCTION}
RfsAction}}

\hypertarget{ecldoc:file.rfsaction}{}
\hspace{0pt} \hyperlink{ecldoc:File}{File} \textbackslash 

{\renewcommand{\arraystretch}{1.5}
\begin{tabularx}{\textwidth}{|>{\raggedright\arraybackslash}l|X|}
\hline
\hspace{0pt}\mytexttt{\color{red} } & \textbf{RfsAction} \\
\hline
\multicolumn{2}{|>{\raggedright\arraybackslash}X|}{\hspace{0pt}\mytexttt{\color{param} (varstring server, varstring query)}} \\
\hline
\end{tabularx}
}

\par
Sends the query to the rfs server.

\par
\begin{description}
\item [\colorbox{tagtype}{\color{white} \textbf{\textsf{PARAMETER}}}] \textbf{\underline{server}} A string containing the ip:port address for the remote file server.
\item [\colorbox{tagtype}{\color{white} \textbf{\textsf{PARAMETER}}}] \textbf{\underline{query}} The text of the query to send to the server
\end{description}

\rule{\linewidth}{0.5pt}
\subsection*{\textsf{\colorbox{headtoc}{\color{white} FUNCTION}
MoveExternalFile}}

\hypertarget{ecldoc:file.moveexternalfile}{}
\hspace{0pt} \hyperlink{ecldoc:File}{File} \textbackslash 

{\renewcommand{\arraystretch}{1.5}
\begin{tabularx}{\textwidth}{|>{\raggedright\arraybackslash}l|X|}
\hline
\hspace{0pt}\mytexttt{\color{red} } & \textbf{MoveExternalFile} \\
\hline
\multicolumn{2}{|>{\raggedright\arraybackslash}X|}{\hspace{0pt}\mytexttt{\color{param} (varstring location, varstring frompath, varstring topath)}} \\
\hline
\end{tabularx}
}

\par
Moves the single physical file between two locations on the same remote machine. The dafileserv utility program must be running on the location machine.

\par
\begin{description}
\item [\colorbox{tagtype}{\color{white} \textbf{\textsf{PARAMETER}}}] \textbf{\underline{location}} The IP address of the remote machine.
\item [\colorbox{tagtype}{\color{white} \textbf{\textsf{PARAMETER}}}] \textbf{\underline{frompath}} The path/name of the file to move.
\item [\colorbox{tagtype}{\color{white} \textbf{\textsf{PARAMETER}}}] \textbf{\underline{topath}} The path/name of the target file.
\end{description}

\rule{\linewidth}{0.5pt}
\subsection*{\textsf{\colorbox{headtoc}{\color{white} FUNCTION}
DeleteExternalFile}}

\hypertarget{ecldoc:file.deleteexternalfile}{}
\hspace{0pt} \hyperlink{ecldoc:File}{File} \textbackslash 

{\renewcommand{\arraystretch}{1.5}
\begin{tabularx}{\textwidth}{|>{\raggedright\arraybackslash}l|X|}
\hline
\hspace{0pt}\mytexttt{\color{red} } & \textbf{DeleteExternalFile} \\
\hline
\multicolumn{2}{|>{\raggedright\arraybackslash}X|}{\hspace{0pt}\mytexttt{\color{param} (varstring location, varstring path)}} \\
\hline
\end{tabularx}
}

\par
Removes a single physical file from a remote machine. The dafileserv utility program must be running on the location machine.

\par
\begin{description}
\item [\colorbox{tagtype}{\color{white} \textbf{\textsf{PARAMETER}}}] \textbf{\underline{location}} The IP address of the remote machine.
\item [\colorbox{tagtype}{\color{white} \textbf{\textsf{PARAMETER}}}] \textbf{\underline{path}} The path/name of the file to remove.
\end{description}

\rule{\linewidth}{0.5pt}
\subsection*{\textsf{\colorbox{headtoc}{\color{white} FUNCTION}
CreateExternalDirectory}}

\hypertarget{ecldoc:file.createexternaldirectory}{}
\hspace{0pt} \hyperlink{ecldoc:File}{File} \textbackslash 

{\renewcommand{\arraystretch}{1.5}
\begin{tabularx}{\textwidth}{|>{\raggedright\arraybackslash}l|X|}
\hline
\hspace{0pt}\mytexttt{\color{red} } & \textbf{CreateExternalDirectory} \\
\hline
\multicolumn{2}{|>{\raggedright\arraybackslash}X|}{\hspace{0pt}\mytexttt{\color{param} (varstring location, varstring path)}} \\
\hline
\end{tabularx}
}

\par
Creates the path on the location (if it does not already exist). The dafileserv utility program must be running on the location machine.

\par
\begin{description}
\item [\colorbox{tagtype}{\color{white} \textbf{\textsf{PARAMETER}}}] \textbf{\underline{location}} The IP address of the remote machine.
\item [\colorbox{tagtype}{\color{white} \textbf{\textsf{PARAMETER}}}] \textbf{\underline{path}} The path/name of the file to remove.
\end{description}

\rule{\linewidth}{0.5pt}
\subsection*{\textsf{\colorbox{headtoc}{\color{white} FUNCTION}
GetLogicalFileAttribute}}

\hypertarget{ecldoc:file.getlogicalfileattribute}{}
\hspace{0pt} \hyperlink{ecldoc:File}{File} \textbackslash 

{\renewcommand{\arraystretch}{1.5}
\begin{tabularx}{\textwidth}{|>{\raggedright\arraybackslash}l|X|}
\hline
\hspace{0pt}\mytexttt{\color{red} varstring} & \textbf{GetLogicalFileAttribute} \\
\hline
\multicolumn{2}{|>{\raggedright\arraybackslash}X|}{\hspace{0pt}\mytexttt{\color{param} (varstring lfn, varstring attrname)}} \\
\hline
\end{tabularx}
}

\par
Returns the value of the given attribute for the specified logicalfilename.

\par
\begin{description}
\item [\colorbox{tagtype}{\color{white} \textbf{\textsf{PARAMETER}}}] \textbf{\underline{lfn}} The name of the logical file.
\item [\colorbox{tagtype}{\color{white} \textbf{\textsf{PARAMETER}}}] \textbf{\underline{attrname}} The name of the file attribute to return.
\end{description}

\rule{\linewidth}{0.5pt}
\subsection*{\textsf{\colorbox{headtoc}{\color{white} FUNCTION}
ProtectLogicalFile}}

\hypertarget{ecldoc:file.protectlogicalfile}{}
\hspace{0pt} \hyperlink{ecldoc:File}{File} \textbackslash 

{\renewcommand{\arraystretch}{1.5}
\begin{tabularx}{\textwidth}{|>{\raggedright\arraybackslash}l|X|}
\hline
\hspace{0pt}\mytexttt{\color{red} } & \textbf{ProtectLogicalFile} \\
\hline
\multicolumn{2}{|>{\raggedright\arraybackslash}X|}{\hspace{0pt}\mytexttt{\color{param} (varstring lfn, boolean value=TRUE)}} \\
\hline
\end{tabularx}
}

\par
Toggles protection on and off for the specified logicalfilename.

\par
\begin{description}
\item [\colorbox{tagtype}{\color{white} \textbf{\textsf{PARAMETER}}}] \textbf{\underline{lfn}} The name of the logical file.
\item [\colorbox{tagtype}{\color{white} \textbf{\textsf{PARAMETER}}}] \textbf{\underline{value}} TRUE to enable protection, FALSE to disable.
\end{description}

\rule{\linewidth}{0.5pt}
\subsection*{\textsf{\colorbox{headtoc}{\color{white} FUNCTION}
DfuPlusExec}}

\hypertarget{ecldoc:file.dfuplusexec}{}
\hspace{0pt} \hyperlink{ecldoc:File}{File} \textbackslash 

{\renewcommand{\arraystretch}{1.5}
\begin{tabularx}{\textwidth}{|>{\raggedright\arraybackslash}l|X|}
\hline
\hspace{0pt}\mytexttt{\color{red} } & \textbf{DfuPlusExec} \\
\hline
\multicolumn{2}{|>{\raggedright\arraybackslash}X|}{\hspace{0pt}\mytexttt{\color{param} (varstring cmdline)}} \\
\hline
\end{tabularx}
}

\par
The DfuPlusExec action executes the specified command line just as the DfuPLus.exe program would do. This allows you to have all the functionality of the DfuPLus.exe program available within your ECL code. param cmdline The DFUPlus.exe command line to execute. The valid arguments are documented in the Client Tools manual, in the section describing the DfuPlus.exe program.


\rule{\linewidth}{0.5pt}
\subsection*{\textsf{\colorbox{headtoc}{\color{white} FUNCTION}
fSprayFixed}}

\hypertarget{ecldoc:file.fsprayfixed}{}
\hspace{0pt} \hyperlink{ecldoc:File}{File} \textbackslash 

{\renewcommand{\arraystretch}{1.5}
\begin{tabularx}{\textwidth}{|>{\raggedright\arraybackslash}l|X|}
\hline
\hspace{0pt}\mytexttt{\color{red} varstring} & \textbf{fSprayFixed} \\
\hline
\multicolumn{2}{|>{\raggedright\arraybackslash}X|}{\hspace{0pt}\mytexttt{\color{param} (varstring sourceIP, varstring sourcePath, integer4 recordSize, varstring destinationGroup, varstring destinationLogicalName, integer4 timeOut=-1, varstring espServerIpPort=GETENV('ws\_fs\_server'), integer4 maxConnections=-1, boolean allowOverwrite=FALSE, boolean replicate=FALSE, boolean compress=FALSE, boolean failIfNoSourceFile=FALSE, integer4 expireDays=-1)}} \\
\hline
\end{tabularx}
}

\par
Sprays a file of fixed length records from a single machine and distributes it across the nodes of the destination group.

\par
\begin{description}
\item [\colorbox{tagtype}{\color{white} \textbf{\textsf{PARAMETER}}}] \textbf{\underline{sourceIP}} The IP address of the file.
\item [\colorbox{tagtype}{\color{white} \textbf{\textsf{PARAMETER}}}] \textbf{\underline{sourcePath}} The path and name of the file.
\item [\colorbox{tagtype}{\color{white} \textbf{\textsf{PARAMETER}}}] \textbf{\underline{recordsize}} The size (in bytes) of the records in the file.
\item [\colorbox{tagtype}{\color{white} \textbf{\textsf{PARAMETER}}}] \textbf{\underline{destinationGroup}} The name of the group to distribute the file across.
\item [\colorbox{tagtype}{\color{white} \textbf{\textsf{PARAMETER}}}] \textbf{\underline{destinationLogicalName}} The logical name of the file to create.
\item [\colorbox{tagtype}{\color{white} \textbf{\textsf{PARAMETER}}}] \textbf{\underline{timeOut}} The time in ms to wait for the operation to complete. A value of 0 causes the call to return immediately. Defaults to no timeout (-1).
\item [\colorbox{tagtype}{\color{white} \textbf{\textsf{PARAMETER}}}] \textbf{\underline{espServerIpPort}} The url of the ESP file copying service. Defaults to the value of ws\_fs\_server in the environment.
\item [\colorbox{tagtype}{\color{white} \textbf{\textsf{PARAMETER}}}] \textbf{\underline{maxConnections}} The maximum number of target nodes to write to concurrently. Defaults to 1.
\item [\colorbox{tagtype}{\color{white} \textbf{\textsf{PARAMETER}}}] \textbf{\underline{allowOverwrite}} Is it valid to overwrite an existing file of the same name? Defaults to FALSE
\item [\colorbox{tagtype}{\color{white} \textbf{\textsf{PARAMETER}}}] \textbf{\underline{replicate}} Whether to replicate the new file. Defaults to FALSE.
\item [\colorbox{tagtype}{\color{white} \textbf{\textsf{PARAMETER}}}] \textbf{\underline{compress}} Whether to compress the new file. Defaults to FALSE.
\item [\colorbox{tagtype}{\color{white} \textbf{\textsf{PARAMETER}}}] \textbf{\underline{failIfNoSourceFile}} If TRUE it causes a missing source file to trigger a failure. Defaults to FALSE.
\item [\colorbox{tagtype}{\color{white} \textbf{\textsf{PARAMETER}}}] \textbf{\underline{expireDays}} Number of days to auto-remove file. Default is -1, not expire.
\item [\colorbox{tagtype}{\color{white} \textbf{\textsf{RETURN}}}] \textbf{\underline{}} The DFU workunit id for the job.
\end{description}

\rule{\linewidth}{0.5pt}
\subsection*{\textsf{\colorbox{headtoc}{\color{white} FUNCTION}
SprayFixed}}

\hypertarget{ecldoc:file.sprayfixed}{}
\hspace{0pt} \hyperlink{ecldoc:File}{File} \textbackslash 

{\renewcommand{\arraystretch}{1.5}
\begin{tabularx}{\textwidth}{|>{\raggedright\arraybackslash}l|X|}
\hline
\hspace{0pt}\mytexttt{\color{red} } & \textbf{SprayFixed} \\
\hline
\multicolumn{2}{|>{\raggedright\arraybackslash}X|}{\hspace{0pt}\mytexttt{\color{param} (varstring sourceIP, varstring sourcePath, integer4 recordSize, varstring destinationGroup, varstring destinationLogicalName, integer4 timeOut=-1, varstring espServerIpPort=GETENV('ws\_fs\_server'), integer4 maxConnections=-1, boolean allowOverwrite=FALSE, boolean replicate=FALSE, boolean compress=FALSE, boolean failIfNoSourceFile=FALSE, integer4 expireDays=-1)}} \\
\hline
\end{tabularx}
}

\par
Same as fSprayFixed, but does not return the DFU Workunit ID.

\par
\begin{description}
\item [\colorbox{tagtype}{\color{white} \textbf{\textsf{SEE}}}] \textbf{\underline{}} fSprayFixed
\end{description}

\rule{\linewidth}{0.5pt}
\subsection*{\textsf{\colorbox{headtoc}{\color{white} FUNCTION}
fSprayVariable}}

\hypertarget{ecldoc:file.fsprayvariable}{}
\hspace{0pt} \hyperlink{ecldoc:File}{File} \textbackslash 

{\renewcommand{\arraystretch}{1.5}
\begin{tabularx}{\textwidth}{|>{\raggedright\arraybackslash}l|X|}
\hline
\hspace{0pt}\mytexttt{\color{red} varstring} & \textbf{fSprayVariable} \\
\hline
\multicolumn{2}{|>{\raggedright\arraybackslash}X|}{\hspace{0pt}\mytexttt{\color{param} (varstring sourceIP, varstring sourcePath, integer4 sourceMaxRecordSize=8192, varstring sourceCsvSeparate='\textbackslash \textbackslash ,', varstring sourceCsvTerminate='\textbackslash \textbackslash n,\textbackslash \textbackslash r\textbackslash \textbackslash n', varstring sourceCsvQuote='\textbackslash ''', varstring destinationGroup, varstring destinationLogicalName, integer4 timeOut=-1, varstring espServerIpPort=GETENV('ws\_fs\_server'), integer4 maxConnections=-1, boolean allowOverwrite=FALSE, boolean replicate=FALSE, boolean compress=FALSE, varstring sourceCsvEscape='', boolean failIfNoSourceFile=FALSE, boolean recordStructurePresent=FALSE, boolean quotedTerminator=TRUE, varstring encoding='ascii', integer4 expireDays=-1)}} \\
\hline
\end{tabularx}
}

\par


\rule{\linewidth}{0.5pt}
\subsection*{\textsf{\colorbox{headtoc}{\color{white} FUNCTION}
SprayVariable}}

\hypertarget{ecldoc:file.sprayvariable}{}
\hspace{0pt} \hyperlink{ecldoc:File}{File} \textbackslash 

{\renewcommand{\arraystretch}{1.5}
\begin{tabularx}{\textwidth}{|>{\raggedright\arraybackslash}l|X|}
\hline
\hspace{0pt}\mytexttt{\color{red} } & \textbf{SprayVariable} \\
\hline
\multicolumn{2}{|>{\raggedright\arraybackslash}X|}{\hspace{0pt}\mytexttt{\color{param} (varstring sourceIP, varstring sourcePath, integer4 sourceMaxRecordSize=8192, varstring sourceCsvSeparate='\textbackslash \textbackslash ,', varstring sourceCsvTerminate='\textbackslash \textbackslash n,\textbackslash \textbackslash r\textbackslash \textbackslash n', varstring sourceCsvQuote='\textbackslash ''', varstring destinationGroup, varstring destinationLogicalName, integer4 timeOut=-1, varstring espServerIpPort=GETENV('ws\_fs\_server'), integer4 maxConnections=-1, boolean allowOverwrite=FALSE, boolean replicate=FALSE, boolean compress=FALSE, varstring sourceCsvEscape='', boolean failIfNoSourceFile=FALSE, boolean recordStructurePresent=FALSE, boolean quotedTerminator=TRUE, varstring encoding='ascii', integer4 expireDays=-1)}} \\
\hline
\end{tabularx}
}

\par


\rule{\linewidth}{0.5pt}
\subsection*{\textsf{\colorbox{headtoc}{\color{white} FUNCTION}
fSprayDelimited}}

\hypertarget{ecldoc:file.fspraydelimited}{}
\hspace{0pt} \hyperlink{ecldoc:File}{File} \textbackslash 

{\renewcommand{\arraystretch}{1.5}
\begin{tabularx}{\textwidth}{|>{\raggedright\arraybackslash}l|X|}
\hline
\hspace{0pt}\mytexttt{\color{red} varstring} & \textbf{fSprayDelimited} \\
\hline
\multicolumn{2}{|>{\raggedright\arraybackslash}X|}{\hspace{0pt}\mytexttt{\color{param} (varstring sourceIP, varstring sourcePath, integer4 sourceMaxRecordSize=8192, varstring sourceCsvSeparate='\textbackslash \textbackslash ,', varstring sourceCsvTerminate='\textbackslash \textbackslash n,\textbackslash \textbackslash r\textbackslash \textbackslash n', varstring sourceCsvQuote='\textbackslash ''', varstring destinationGroup, varstring destinationLogicalName, integer4 timeOut=-1, varstring espServerIpPort=GETENV('ws\_fs\_server'), integer4 maxConnections=-1, boolean allowOverwrite=FALSE, boolean replicate=FALSE, boolean compress=FALSE, varstring sourceCsvEscape='', boolean failIfNoSourceFile=FALSE, boolean recordStructurePresent=FALSE, boolean quotedTerminator=TRUE, varstring encoding='ascii', integer4 expireDays=-1)}} \\
\hline
\end{tabularx}
}

\par
Sprays a file of fixed delimited records from a single machine and distributes it across the nodes of the destination group.

\par
\begin{description}
\item [\colorbox{tagtype}{\color{white} \textbf{\textsf{PARAMETER}}}] \textbf{\underline{sourceIP}} The IP address of the file.
\item [\colorbox{tagtype}{\color{white} \textbf{\textsf{PARAMETER}}}] \textbf{\underline{sourcePath}} The path and name of the file.
\item [\colorbox{tagtype}{\color{white} \textbf{\textsf{PARAMETER}}}] \textbf{\underline{sourceCsvSeparate}} The character sequence which separates fields in the file.
\item [\colorbox{tagtype}{\color{white} \textbf{\textsf{PARAMETER}}}] \textbf{\underline{sourceCsvTerminate}} The character sequence which separates records in the file.
\item [\colorbox{tagtype}{\color{white} \textbf{\textsf{PARAMETER}}}] \textbf{\underline{sourceCsvQuote}} A string which can be used to delimit fields in the file.
\item [\colorbox{tagtype}{\color{white} \textbf{\textsf{PARAMETER}}}] \textbf{\underline{sourceMaxRecordSize}} The maximum size (in bytes) of the records in the file.
\item [\colorbox{tagtype}{\color{white} \textbf{\textsf{PARAMETER}}}] \textbf{\underline{destinationGroup}} The name of the group to distribute the file across.
\item [\colorbox{tagtype}{\color{white} \textbf{\textsf{PARAMETER}}}] \textbf{\underline{destinationLogicalName}} The logical name of the file to create.
\item [\colorbox{tagtype}{\color{white} \textbf{\textsf{PARAMETER}}}] \textbf{\underline{timeOut}} The time in ms to wait for the operation to complete. A value of 0 causes the call to return immediately. Defaults to no timeout (-1).
\item [\colorbox{tagtype}{\color{white} \textbf{\textsf{PARAMETER}}}] \textbf{\underline{espServerIpPort}} The url of the ESP file copying service. Defaults to the value of ws\_fs\_server in the environment.
\item [\colorbox{tagtype}{\color{white} \textbf{\textsf{PARAMETER}}}] \textbf{\underline{maxConnections}} The maximum number of target nodes to write to concurrently. Defaults to 1.
\item [\colorbox{tagtype}{\color{white} \textbf{\textsf{PARAMETER}}}] \textbf{\underline{allowOverwrite}} Is it valid to overwrite an existing file of the same name? Defaults to FALSE
\item [\colorbox{tagtype}{\color{white} \textbf{\textsf{PARAMETER}}}] \textbf{\underline{replicate}} Whether to replicate the new file. Defaults to FALSE.
\item [\colorbox{tagtype}{\color{white} \textbf{\textsf{PARAMETER}}}] \textbf{\underline{compress}} Whether to compress the new file. Defaults to FALSE.
\item [\colorbox{tagtype}{\color{white} \textbf{\textsf{PARAMETER}}}] \textbf{\underline{sourceCsvEscape}} A character that is used to escape quote characters. Defaults to none.
\item [\colorbox{tagtype}{\color{white} \textbf{\textsf{PARAMETER}}}] \textbf{\underline{failIfNoSourceFile}} If TRUE it causes a missing source file to trigger a failure. Defaults to FALSE.
\item [\colorbox{tagtype}{\color{white} \textbf{\textsf{PARAMETER}}}] \textbf{\underline{recordStructurePresent}} If TRUE derives the record structure from the header of the file.
\item [\colorbox{tagtype}{\color{white} \textbf{\textsf{PARAMETER}}}] \textbf{\underline{quotedTerminator}} Can the terminator character be included in a quoted field. Defaults to TRUE. If FALSE it allows quicker partitioning of the file (avoiding a complete file scan).
\item [\colorbox{tagtype}{\color{white} \textbf{\textsf{PARAMETER}}}] \textbf{\underline{expireDays}} Number of days to auto-remove file. Default is -1, not expire.
\item [\colorbox{tagtype}{\color{white} \textbf{\textsf{RETURN}}}] \textbf{\underline{}} The DFU workunit id for the job.
\end{description}

\rule{\linewidth}{0.5pt}
\subsection*{\textsf{\colorbox{headtoc}{\color{white} FUNCTION}
SprayDelimited}}

\hypertarget{ecldoc:file.spraydelimited}{}
\hspace{0pt} \hyperlink{ecldoc:File}{File} \textbackslash 

{\renewcommand{\arraystretch}{1.5}
\begin{tabularx}{\textwidth}{|>{\raggedright\arraybackslash}l|X|}
\hline
\hspace{0pt}\mytexttt{\color{red} } & \textbf{SprayDelimited} \\
\hline
\multicolumn{2}{|>{\raggedright\arraybackslash}X|}{\hspace{0pt}\mytexttt{\color{param} (varstring sourceIP, varstring sourcePath, integer4 sourceMaxRecordSize=8192, varstring sourceCsvSeparate='\textbackslash \textbackslash ,', varstring sourceCsvTerminate='\textbackslash \textbackslash n,\textbackslash \textbackslash r\textbackslash \textbackslash n', varstring sourceCsvQuote='\textbackslash ''', varstring destinationGroup, varstring destinationLogicalName, integer4 timeOut=-1, varstring espServerIpPort=GETENV('ws\_fs\_server'), integer4 maxConnections=-1, boolean allowOverwrite=FALSE, boolean replicate=FALSE, boolean compress=FALSE, varstring sourceCsvEscape='', boolean failIfNoSourceFile=FALSE, boolean recordStructurePresent=FALSE, boolean quotedTerminator=TRUE, const varstring encoding='ascii', integer4 expireDays=-1)}} \\
\hline
\end{tabularx}
}

\par
Same as fSprayDelimited, but does not return the DFU Workunit ID.

\par
\begin{description}
\item [\colorbox{tagtype}{\color{white} \textbf{\textsf{SEE}}}] \textbf{\underline{}} fSprayDelimited
\end{description}

\rule{\linewidth}{0.5pt}
\subsection*{\textsf{\colorbox{headtoc}{\color{white} FUNCTION}
fSprayXml}}

\hypertarget{ecldoc:file.fsprayxml}{}
\hspace{0pt} \hyperlink{ecldoc:File}{File} \textbackslash 

{\renewcommand{\arraystretch}{1.5}
\begin{tabularx}{\textwidth}{|>{\raggedright\arraybackslash}l|X|}
\hline
\hspace{0pt}\mytexttt{\color{red} varstring} & \textbf{fSprayXml} \\
\hline
\multicolumn{2}{|>{\raggedright\arraybackslash}X|}{\hspace{0pt}\mytexttt{\color{param} (varstring sourceIP, varstring sourcePath, integer4 sourceMaxRecordSize=8192, varstring sourceRowTag, varstring sourceEncoding='utf8', varstring destinationGroup, varstring destinationLogicalName, integer4 timeOut=-1, varstring espServerIpPort=GETENV('ws\_fs\_server'), integer4 maxConnections=-1, boolean allowOverwrite=FALSE, boolean replicate=FALSE, boolean compress=FALSE, boolean failIfNoSourceFile=FALSE, integer4 expireDays=-1)}} \\
\hline
\end{tabularx}
}

\par
Sprays an xml file from a single machine and distributes it across the nodes of the destination group.

\par
\begin{description}
\item [\colorbox{tagtype}{\color{white} \textbf{\textsf{PARAMETER}}}] \textbf{\underline{sourceIP}} The IP address of the file.
\item [\colorbox{tagtype}{\color{white} \textbf{\textsf{PARAMETER}}}] \textbf{\underline{sourcePath}} The path and name of the file.
\item [\colorbox{tagtype}{\color{white} \textbf{\textsf{PARAMETER}}}] \textbf{\underline{sourceMaxRecordSize}} The maximum size (in bytes) of the records in the file.
\item [\colorbox{tagtype}{\color{white} \textbf{\textsf{PARAMETER}}}] \textbf{\underline{sourceRowTag}} The xml tag that is used to delimit records in the source file. (This tag cannot recursivly nest.)
\item [\colorbox{tagtype}{\color{white} \textbf{\textsf{PARAMETER}}}] \textbf{\underline{sourceEncoding}} The unicode encoding of the file. (utf8,utf8n,utf16be,utf16le,utf32be,utf32le)
\item [\colorbox{tagtype}{\color{white} \textbf{\textsf{PARAMETER}}}] \textbf{\underline{destinationGroup}} The name of the group to distribute the file across.
\item [\colorbox{tagtype}{\color{white} \textbf{\textsf{PARAMETER}}}] \textbf{\underline{destinationLogicalName}} The logical name of the file to create.
\item [\colorbox{tagtype}{\color{white} \textbf{\textsf{PARAMETER}}}] \textbf{\underline{timeOut}} The time in ms to wait for the operation to complete. A value of 0 causes the call to return immediately. Defaults to no timeout (-1).
\item [\colorbox{tagtype}{\color{white} \textbf{\textsf{PARAMETER}}}] \textbf{\underline{espServerIpPort}} The url of the ESP file copying service. Defaults to the value of ws\_fs\_server in the environment.
\item [\colorbox{tagtype}{\color{white} \textbf{\textsf{PARAMETER}}}] \textbf{\underline{maxConnections}} The maximum number of target nodes to write to concurrently. Defaults to 1.
\item [\colorbox{tagtype}{\color{white} \textbf{\textsf{PARAMETER}}}] \textbf{\underline{allowOverwrite}} Is it valid to overwrite an existing file of the same name? Defaults to FALSE
\item [\colorbox{tagtype}{\color{white} \textbf{\textsf{PARAMETER}}}] \textbf{\underline{replicate}} Whether to replicate the new file. Defaults to FALSE.
\item [\colorbox{tagtype}{\color{white} \textbf{\textsf{PARAMETER}}}] \textbf{\underline{compress}} Whether to compress the new file. Defaults to FALSE.
\item [\colorbox{tagtype}{\color{white} \textbf{\textsf{PARAMETER}}}] \textbf{\underline{failIfNoSourceFile}} If TRUE it causes a missing source file to trigger a failure. Defaults to FALSE.
\item [\colorbox{tagtype}{\color{white} \textbf{\textsf{PARAMETER}}}] \textbf{\underline{expireDays}} Number of days to auto-remove file. Default is -1, not expire.
\item [\colorbox{tagtype}{\color{white} \textbf{\textsf{RETURN}}}] \textbf{\underline{}} The DFU workunit id for the job.
\end{description}

\rule{\linewidth}{0.5pt}
\subsection*{\textsf{\colorbox{headtoc}{\color{white} FUNCTION}
SprayXml}}

\hypertarget{ecldoc:file.sprayxml}{}
\hspace{0pt} \hyperlink{ecldoc:File}{File} \textbackslash 

{\renewcommand{\arraystretch}{1.5}
\begin{tabularx}{\textwidth}{|>{\raggedright\arraybackslash}l|X|}
\hline
\hspace{0pt}\mytexttt{\color{red} } & \textbf{SprayXml} \\
\hline
\multicolumn{2}{|>{\raggedright\arraybackslash}X|}{\hspace{0pt}\mytexttt{\color{param} (varstring sourceIP, varstring sourcePath, integer4 sourceMaxRecordSize=8192, varstring sourceRowTag, varstring sourceEncoding='utf8', varstring destinationGroup, varstring destinationLogicalName, integer4 timeOut=-1, varstring espServerIpPort=GETENV('ws\_fs\_server'), integer4 maxConnections=-1, boolean allowOverwrite=FALSE, boolean replicate=FALSE, boolean compress=FALSE, boolean failIfNoSourceFile=FALSE, integer4 expireDays=-1)}} \\
\hline
\end{tabularx}
}

\par
Same as fSprayXml, but does not return the DFU Workunit ID.

\par
\begin{description}
\item [\colorbox{tagtype}{\color{white} \textbf{\textsf{SEE}}}] \textbf{\underline{}} fSprayXml
\end{description}

\rule{\linewidth}{0.5pt}
\subsection*{\textsf{\colorbox{headtoc}{\color{white} FUNCTION}
fDespray}}

\hypertarget{ecldoc:file.fdespray}{}
\hspace{0pt} \hyperlink{ecldoc:File}{File} \textbackslash 

{\renewcommand{\arraystretch}{1.5}
\begin{tabularx}{\textwidth}{|>{\raggedright\arraybackslash}l|X|}
\hline
\hspace{0pt}\mytexttt{\color{red} varstring} & \textbf{fDespray} \\
\hline
\multicolumn{2}{|>{\raggedright\arraybackslash}X|}{\hspace{0pt}\mytexttt{\color{param} (varstring logicalName, varstring destinationIP, varstring destinationPath, integer4 timeOut=-1, varstring espServerIpPort=GETENV('ws\_fs\_server'), integer4 maxConnections=-1, boolean allowOverwrite=FALSE)}} \\
\hline
\end{tabularx}
}

\par
Copies a distributed file from multiple machines, and desprays it to a single file on a single machine.

\par
\begin{description}
\item [\colorbox{tagtype}{\color{white} \textbf{\textsf{PARAMETER}}}] \textbf{\underline{logicalName}} The name of the file to despray.
\item [\colorbox{tagtype}{\color{white} \textbf{\textsf{PARAMETER}}}] \textbf{\underline{destinationIP}} The IP of the target machine.
\item [\colorbox{tagtype}{\color{white} \textbf{\textsf{PARAMETER}}}] \textbf{\underline{destinationPath}} The path of the file to create on the destination machine.
\item [\colorbox{tagtype}{\color{white} \textbf{\textsf{PARAMETER}}}] \textbf{\underline{timeOut}} The time in ms to wait for the operation to complete. A value of 0 causes the call to return immediately. Defaults to no timeout (-1).
\item [\colorbox{tagtype}{\color{white} \textbf{\textsf{PARAMETER}}}] \textbf{\underline{espServerIpPort}} The url of the ESP file copying service. Defaults to the value of ws\_fs\_server in the environment.
\item [\colorbox{tagtype}{\color{white} \textbf{\textsf{PARAMETER}}}] \textbf{\underline{maxConnections}} The maximum number of target nodes to write to concurrently. Defaults to 1.
\item [\colorbox{tagtype}{\color{white} \textbf{\textsf{PARAMETER}}}] \textbf{\underline{allowOverwrite}} Is it valid to overwrite an existing file of the same name? Defaults to FALSE
\item [\colorbox{tagtype}{\color{white} \textbf{\textsf{RETURN}}}] \textbf{\underline{}} The DFU workunit id for the job.
\end{description}

\rule{\linewidth}{0.5pt}
\subsection*{\textsf{\colorbox{headtoc}{\color{white} FUNCTION}
Despray}}

\hypertarget{ecldoc:file.despray}{}
\hspace{0pt} \hyperlink{ecldoc:File}{File} \textbackslash 

{\renewcommand{\arraystretch}{1.5}
\begin{tabularx}{\textwidth}{|>{\raggedright\arraybackslash}l|X|}
\hline
\hspace{0pt}\mytexttt{\color{red} } & \textbf{Despray} \\
\hline
\multicolumn{2}{|>{\raggedright\arraybackslash}X|}{\hspace{0pt}\mytexttt{\color{param} (varstring logicalName, varstring destinationIP, varstring destinationPath, integer4 timeOut=-1, varstring espServerIpPort=GETENV('ws\_fs\_server'), integer4 maxConnections=-1, boolean allowOverwrite=FALSE)}} \\
\hline
\end{tabularx}
}

\par
Same as fDespray, but does not return the DFU Workunit ID.

\par
\begin{description}
\item [\colorbox{tagtype}{\color{white} \textbf{\textsf{SEE}}}] \textbf{\underline{}} fDespray
\end{description}

\rule{\linewidth}{0.5pt}
\subsection*{\textsf{\colorbox{headtoc}{\color{white} FUNCTION}
fCopy}}

\hypertarget{ecldoc:file.fcopy}{}
\hspace{0pt} \hyperlink{ecldoc:File}{File} \textbackslash 

{\renewcommand{\arraystretch}{1.5}
\begin{tabularx}{\textwidth}{|>{\raggedright\arraybackslash}l|X|}
\hline
\hspace{0pt}\mytexttt{\color{red} varstring} & \textbf{fCopy} \\
\hline
\multicolumn{2}{|>{\raggedright\arraybackslash}X|}{\hspace{0pt}\mytexttt{\color{param} (varstring sourceLogicalName, varstring destinationGroup, varstring destinationLogicalName, varstring sourceDali='', integer4 timeOut=-1, varstring espServerIpPort=GETENV('ws\_fs\_server'), integer4 maxConnections=-1, boolean allowOverwrite=FALSE, boolean replicate=FALSE, boolean asSuperfile=FALSE, boolean compress=FALSE, boolean forcePush=FALSE, integer4 transferBufferSize=0, boolean preserveCompression=TRUE)}} \\
\hline
\end{tabularx}
}

\par
Copies a distributed file to another distributed file.

\par
\begin{description}
\item [\colorbox{tagtype}{\color{white} \textbf{\textsf{PARAMETER}}}] \textbf{\underline{sourceLogicalName}} The name of the file to despray.
\item [\colorbox{tagtype}{\color{white} \textbf{\textsf{PARAMETER}}}] \textbf{\underline{destinationGroup}} The name of the group to distribute the file across.
\item [\colorbox{tagtype}{\color{white} \textbf{\textsf{PARAMETER}}}] \textbf{\underline{destinationLogicalName}} The logical name of the file to create.
\item [\colorbox{tagtype}{\color{white} \textbf{\textsf{PARAMETER}}}] \textbf{\underline{sourceDali}} The dali that contains the source file (blank implies same dali). Defaults to same dali.
\item [\colorbox{tagtype}{\color{white} \textbf{\textsf{PARAMETER}}}] \textbf{\underline{timeOut}} The time in ms to wait for the operation to complete. A value of 0 causes the call to return immediately. Defaults to no timeout (-1).
\item [\colorbox{tagtype}{\color{white} \textbf{\textsf{PARAMETER}}}] \textbf{\underline{espServerIpPort}} The url of the ESP file copying service. Defaults to the value of ws\_fs\_server in the environment.
\item [\colorbox{tagtype}{\color{white} \textbf{\textsf{PARAMETER}}}] \textbf{\underline{maxConnections}} The maximum number of target nodes to write to concurrently. Defaults to 1.
\item [\colorbox{tagtype}{\color{white} \textbf{\textsf{PARAMETER}}}] \textbf{\underline{allowOverwrite}} Is it valid to overwrite an existing file of the same name? Defaults to FALSE
\item [\colorbox{tagtype}{\color{white} \textbf{\textsf{PARAMETER}}}] \textbf{\underline{replicate}} Should the copied file also be replicated on the destination? Defaults to FALSE
\item [\colorbox{tagtype}{\color{white} \textbf{\textsf{PARAMETER}}}] \textbf{\underline{asSuperfile}} Should the file be copied as a superfile? If TRUE and source is a superfile, then the operation creates a superfile on the target, creating sub-files as needed and only overwriting existing sub-files whose content has changed. If FALSE, a single file is created. Defaults to FALSE.
\item [\colorbox{tagtype}{\color{white} \textbf{\textsf{PARAMETER}}}] \textbf{\underline{compress}} Whether to compress the new file. Defaults to FALSE.
\item [\colorbox{tagtype}{\color{white} \textbf{\textsf{PARAMETER}}}] \textbf{\underline{forcePush}} Should the copy process be executed on the source nodes (push) or on the destination nodes (pull)? Default is to pull.
\item [\colorbox{tagtype}{\color{white} \textbf{\textsf{PARAMETER}}}] \textbf{\underline{transferBufferSize}} Overrides the size (in bytes) of the internal buffer used to copy the file. Default is 64k.
\item [\colorbox{tagtype}{\color{white} \textbf{\textsf{RETURN}}}] \textbf{\underline{}} The DFU workunit id for the job.
\end{description}

\rule{\linewidth}{0.5pt}
\subsection*{\textsf{\colorbox{headtoc}{\color{white} FUNCTION}
Copy}}

\hypertarget{ecldoc:file.copy}{}
\hspace{0pt} \hyperlink{ecldoc:File}{File} \textbackslash 

{\renewcommand{\arraystretch}{1.5}
\begin{tabularx}{\textwidth}{|>{\raggedright\arraybackslash}l|X|}
\hline
\hspace{0pt}\mytexttt{\color{red} } & \textbf{Copy} \\
\hline
\multicolumn{2}{|>{\raggedright\arraybackslash}X|}{\hspace{0pt}\mytexttt{\color{param} (varstring sourceLogicalName, varstring destinationGroup, varstring destinationLogicalName, varstring sourceDali='', integer4 timeOut=-1, varstring espServerIpPort=GETENV('ws\_fs\_server'), integer4 maxConnections=-1, boolean allowOverwrite=FALSE, boolean replicate=FALSE, boolean asSuperfile=FALSE, boolean compress=FALSE, boolean forcePush=FALSE, integer4 transferBufferSize=0, boolean preserveCompression=TRUE)}} \\
\hline
\end{tabularx}
}

\par
Same as fCopy, but does not return the DFU Workunit ID.

\par
\begin{description}
\item [\colorbox{tagtype}{\color{white} \textbf{\textsf{SEE}}}] \textbf{\underline{}} fCopy
\end{description}

\rule{\linewidth}{0.5pt}
\subsection*{\textsf{\colorbox{headtoc}{\color{white} FUNCTION}
fReplicate}}

\hypertarget{ecldoc:file.freplicate}{}
\hspace{0pt} \hyperlink{ecldoc:File}{File} \textbackslash 

{\renewcommand{\arraystretch}{1.5}
\begin{tabularx}{\textwidth}{|>{\raggedright\arraybackslash}l|X|}
\hline
\hspace{0pt}\mytexttt{\color{red} varstring} & \textbf{fReplicate} \\
\hline
\multicolumn{2}{|>{\raggedright\arraybackslash}X|}{\hspace{0pt}\mytexttt{\color{param} (varstring logicalName, integer4 timeOut=-1, varstring espServerIpPort=GETENV('ws\_fs\_server'))}} \\
\hline
\end{tabularx}
}

\par
Ensures the specified file is replicated to its mirror copies.

\par
\begin{description}
\item [\colorbox{tagtype}{\color{white} \textbf{\textsf{PARAMETER}}}] \textbf{\underline{logicalName}} The name of the file to replicate.
\item [\colorbox{tagtype}{\color{white} \textbf{\textsf{PARAMETER}}}] \textbf{\underline{timeOut}} The time in ms to wait for the operation to complete. A value of 0 causes the call to return immediately. Defaults to no timeout (-1).
\item [\colorbox{tagtype}{\color{white} \textbf{\textsf{PARAMETER}}}] \textbf{\underline{espServerIpPort}} The url of the ESP file copying service. Defaults to the value of ws\_fs\_server in the environment.
\item [\colorbox{tagtype}{\color{white} \textbf{\textsf{RETURN}}}] \textbf{\underline{}} The DFU workunit id for the job.
\end{description}

\rule{\linewidth}{0.5pt}
\subsection*{\textsf{\colorbox{headtoc}{\color{white} FUNCTION}
Replicate}}

\hypertarget{ecldoc:file.replicate}{}
\hspace{0pt} \hyperlink{ecldoc:File}{File} \textbackslash 

{\renewcommand{\arraystretch}{1.5}
\begin{tabularx}{\textwidth}{|>{\raggedright\arraybackslash}l|X|}
\hline
\hspace{0pt}\mytexttt{\color{red} } & \textbf{Replicate} \\
\hline
\multicolumn{2}{|>{\raggedright\arraybackslash}X|}{\hspace{0pt}\mytexttt{\color{param} (varstring logicalName, integer4 timeOut=-1, varstring espServerIpPort=GETENV('ws\_fs\_server'))}} \\
\hline
\end{tabularx}
}

\par
Same as fReplicated, but does not return the DFU Workunit ID.

\par
\begin{description}
\item [\colorbox{tagtype}{\color{white} \textbf{\textsf{SEE}}}] \textbf{\underline{}} fReplicate
\end{description}

\rule{\linewidth}{0.5pt}
\subsection*{\textsf{\colorbox{headtoc}{\color{white} FUNCTION}
fRemotePull}}

\hypertarget{ecldoc:file.fremotepull}{}
\hspace{0pt} \hyperlink{ecldoc:File}{File} \textbackslash 

{\renewcommand{\arraystretch}{1.5}
\begin{tabularx}{\textwidth}{|>{\raggedright\arraybackslash}l|X|}
\hline
\hspace{0pt}\mytexttt{\color{red} varstring} & \textbf{fRemotePull} \\
\hline
\multicolumn{2}{|>{\raggedright\arraybackslash}X|}{\hspace{0pt}\mytexttt{\color{param} (varstring remoteEspFsURL, varstring sourceLogicalName, varstring destinationGroup, varstring destinationLogicalName, integer4 timeOut=-1, integer4 maxConnections=-1, boolean allowOverwrite=FALSE, boolean replicate=FALSE, boolean asSuperfile=FALSE, boolean forcePush=FALSE, integer4 transferBufferSize=0, boolean wrap=FALSE, boolean compress=FALSE)}} \\
\hline
\end{tabularx}
}

\par
Copies a distributed file to a distributed file on remote system. Similar to fCopy, except the copy executes remotely. Since the DFU workunit executes on the remote DFU server, the user name authentication must be the same on both systems, and the user must have rights to copy files on both systems.

\par
\begin{description}
\item [\colorbox{tagtype}{\color{white} \textbf{\textsf{PARAMETER}}}] \textbf{\underline{remoteEspFsURL}} The url of the remote ESP file copying service.
\item [\colorbox{tagtype}{\color{white} \textbf{\textsf{PARAMETER}}}] \textbf{\underline{sourceLogicalName}} The name of the file to despray.
\item [\colorbox{tagtype}{\color{white} \textbf{\textsf{PARAMETER}}}] \textbf{\underline{destinationGroup}} The name of the group to distribute the file across.
\item [\colorbox{tagtype}{\color{white} \textbf{\textsf{PARAMETER}}}] \textbf{\underline{destinationLogicalName}} The logical name of the file to create.
\item [\colorbox{tagtype}{\color{white} \textbf{\textsf{PARAMETER}}}] \textbf{\underline{timeOut}} The time in ms to wait for the operation to complete. A value of 0 causes the call to return immediately. Defaults to no timeout (-1).
\item [\colorbox{tagtype}{\color{white} \textbf{\textsf{PARAMETER}}}] \textbf{\underline{maxConnections}} The maximum number of target nodes to write to concurrently. Defaults to 1.
\item [\colorbox{tagtype}{\color{white} \textbf{\textsf{PARAMETER}}}] \textbf{\underline{allowOverwrite}} Is it valid to overwrite an existing file of the same name? Defaults to FALSE
\item [\colorbox{tagtype}{\color{white} \textbf{\textsf{PARAMETER}}}] \textbf{\underline{replicate}} Should the copied file also be replicated on the destination? Defaults to FALSE
\item [\colorbox{tagtype}{\color{white} \textbf{\textsf{PARAMETER}}}] \textbf{\underline{asSuperfile}} Should the file be copied as a superfile? If TRUE and source is a superfile, then the operation creates a superfile on the target, creating sub-files as needed and only overwriting existing sub-files whose content has changed. If FALSE a single file is created. Defaults to FALSE.
\item [\colorbox{tagtype}{\color{white} \textbf{\textsf{PARAMETER}}}] \textbf{\underline{compress}} Whether to compress the new file. Defaults to FALSE.
\item [\colorbox{tagtype}{\color{white} \textbf{\textsf{PARAMETER}}}] \textbf{\underline{forcePush}} Should the copy process should be executed on the source nodes (push) or on the destination nodes (pull)? Default is to pull.
\item [\colorbox{tagtype}{\color{white} \textbf{\textsf{PARAMETER}}}] \textbf{\underline{transferBufferSize}} Overrides the size (in bytes) of the internal buffer used to copy the file. Default is 64k.
\item [\colorbox{tagtype}{\color{white} \textbf{\textsf{PARAMETER}}}] \textbf{\underline{wrap}} Should the fileparts be wrapped when copying to a smaller sized cluster? The default is FALSE.
\item [\colorbox{tagtype}{\color{white} \textbf{\textsf{RETURN}}}] \textbf{\underline{}} The DFU workunit id for the job.
\end{description}

\rule{\linewidth}{0.5pt}
\subsection*{\textsf{\colorbox{headtoc}{\color{white} FUNCTION}
RemotePull}}

\hypertarget{ecldoc:file.remotepull}{}
\hspace{0pt} \hyperlink{ecldoc:File}{File} \textbackslash 

{\renewcommand{\arraystretch}{1.5}
\begin{tabularx}{\textwidth}{|>{\raggedright\arraybackslash}l|X|}
\hline
\hspace{0pt}\mytexttt{\color{red} } & \textbf{RemotePull} \\
\hline
\multicolumn{2}{|>{\raggedright\arraybackslash}X|}{\hspace{0pt}\mytexttt{\color{param} (varstring remoteEspFsURL, varstring sourceLogicalName, varstring destinationGroup, varstring destinationLogicalName, integer4 timeOut=-1, integer4 maxConnections=-1, boolean allowOverwrite=FALSE, boolean replicate=FALSE, boolean asSuperfile=FALSE, boolean forcePush=FALSE, integer4 transferBufferSize=0, boolean wrap=FALSE, boolean compress=FALSE)}} \\
\hline
\end{tabularx}
}

\par
Same as fRemotePull, but does not return the DFU Workunit ID.

\par
\begin{description}
\item [\colorbox{tagtype}{\color{white} \textbf{\textsf{SEE}}}] \textbf{\underline{}} fRemotePull
\end{description}

\rule{\linewidth}{0.5pt}
\subsection*{\textsf{\colorbox{headtoc}{\color{white} FUNCTION}
fMonitorLogicalFileName}}

\hypertarget{ecldoc:file.fmonitorlogicalfilename}{}
\hspace{0pt} \hyperlink{ecldoc:File}{File} \textbackslash 

{\renewcommand{\arraystretch}{1.5}
\begin{tabularx}{\textwidth}{|>{\raggedright\arraybackslash}l|X|}
\hline
\hspace{0pt}\mytexttt{\color{red} varstring} & \textbf{fMonitorLogicalFileName} \\
\hline
\multicolumn{2}{|>{\raggedright\arraybackslash}X|}{\hspace{0pt}\mytexttt{\color{param} (varstring eventToFire, varstring name, integer4 shotCount=1, varstring espServerIpPort=GETENV('ws\_fs\_server'))}} \\
\hline
\end{tabularx}
}

\par
Creates a file monitor job in the DFU Server. If an appropriately named file arrives in this interval it will fire the event with the name of the triggering object as the event subtype (see the EVENT function).

\par
\begin{description}
\item [\colorbox{tagtype}{\color{white} \textbf{\textsf{PARAMETER}}}] \textbf{\underline{eventToFire}} The user-defined name of the event to fire when the filename appears. This value is used as the first parameter to the EVENT function.
\item [\colorbox{tagtype}{\color{white} \textbf{\textsf{PARAMETER}}}] \textbf{\underline{name}} The name of the logical file to monitor. This may contain wildcard characters ( * and ?)
\item [\colorbox{tagtype}{\color{white} \textbf{\textsf{PARAMETER}}}] \textbf{\underline{shotCount}} The number of times to generate the event before the monitoring job completes. A value of -1 indicates the monitoring job continues until manually aborted. The default is 1.
\item [\colorbox{tagtype}{\color{white} \textbf{\textsf{PARAMETER}}}] \textbf{\underline{espServerIpPort}} The url of the ESP file copying service. Defaults to the value of ws\_fs\_server in the environment.
\item [\colorbox{tagtype}{\color{white} \textbf{\textsf{RETURN}}}] \textbf{\underline{}} The DFU workunit id for the job.
\end{description}

\rule{\linewidth}{0.5pt}
\subsection*{\textsf{\colorbox{headtoc}{\color{white} FUNCTION}
MonitorLogicalFileName}}

\hypertarget{ecldoc:file.monitorlogicalfilename}{}
\hspace{0pt} \hyperlink{ecldoc:File}{File} \textbackslash 

{\renewcommand{\arraystretch}{1.5}
\begin{tabularx}{\textwidth}{|>{\raggedright\arraybackslash}l|X|}
\hline
\hspace{0pt}\mytexttt{\color{red} } & \textbf{MonitorLogicalFileName} \\
\hline
\multicolumn{2}{|>{\raggedright\arraybackslash}X|}{\hspace{0pt}\mytexttt{\color{param} (varstring eventToFire, varstring name, integer4 shotCount=1, varstring espServerIpPort=GETENV('ws\_fs\_server'))}} \\
\hline
\end{tabularx}
}

\par
Same as fMonitorLogicalFileName, but does not return the DFU Workunit ID.

\par
\begin{description}
\item [\colorbox{tagtype}{\color{white} \textbf{\textsf{SEE}}}] \textbf{\underline{}} fMonitorLogicalFileName
\end{description}

\rule{\linewidth}{0.5pt}
\subsection*{\textsf{\colorbox{headtoc}{\color{white} FUNCTION}
fMonitorFile}}

\hypertarget{ecldoc:file.fmonitorfile}{}
\hspace{0pt} \hyperlink{ecldoc:File}{File} \textbackslash 

{\renewcommand{\arraystretch}{1.5}
\begin{tabularx}{\textwidth}{|>{\raggedright\arraybackslash}l|X|}
\hline
\hspace{0pt}\mytexttt{\color{red} varstring} & \textbf{fMonitorFile} \\
\hline
\multicolumn{2}{|>{\raggedright\arraybackslash}X|}{\hspace{0pt}\mytexttt{\color{param} (varstring eventToFire, varstring ip, varstring filename, boolean subDirs=FALSE, integer4 shotCount=1, varstring espServerIpPort=GETENV('ws\_fs\_server'))}} \\
\hline
\end{tabularx}
}

\par
Creates a file monitor job in the DFU Server. If an appropriately named file arrives in this interval it will fire the event with the name of the triggering object as the event subtype (see the EVENT function).

\par
\begin{description}
\item [\colorbox{tagtype}{\color{white} \textbf{\textsf{PARAMETER}}}] \textbf{\underline{eventToFire}} The user-defined name of the event to fire when the filename appears. This value is used as the first parameter to the EVENT function.
\item [\colorbox{tagtype}{\color{white} \textbf{\textsf{PARAMETER}}}] \textbf{\underline{ip}} The the IP address for the file to monitor. This may be omitted if the filename parameter contains a complete URL.
\item [\colorbox{tagtype}{\color{white} \textbf{\textsf{PARAMETER}}}] \textbf{\underline{filename}} The full path of the file(s) to monitor. This may contain wildcard characters ( * and ?)
\item [\colorbox{tagtype}{\color{white} \textbf{\textsf{PARAMETER}}}] \textbf{\underline{subDirs}} Whether to include files in sub-directories (when the filename contains wildcards). Defaults to FALSE.
\item [\colorbox{tagtype}{\color{white} \textbf{\textsf{PARAMETER}}}] \textbf{\underline{shotCount}} The number of times to generate the event before the monitoring job completes. A value of -1 indicates the monitoring job continues until manually aborted. The default is 1.
\item [\colorbox{tagtype}{\color{white} \textbf{\textsf{PARAMETER}}}] \textbf{\underline{espServerIpPort}} The url of the ESP file copying service. Defaults to the value of ws\_fs\_server in the environment.
\item [\colorbox{tagtype}{\color{white} \textbf{\textsf{RETURN}}}] \textbf{\underline{}} The DFU workunit id for the job.
\end{description}

\rule{\linewidth}{0.5pt}
\subsection*{\textsf{\colorbox{headtoc}{\color{white} FUNCTION}
MonitorFile}}

\hypertarget{ecldoc:file.monitorfile}{}
\hspace{0pt} \hyperlink{ecldoc:File}{File} \textbackslash 

{\renewcommand{\arraystretch}{1.5}
\begin{tabularx}{\textwidth}{|>{\raggedright\arraybackslash}l|X|}
\hline
\hspace{0pt}\mytexttt{\color{red} } & \textbf{MonitorFile} \\
\hline
\multicolumn{2}{|>{\raggedright\arraybackslash}X|}{\hspace{0pt}\mytexttt{\color{param} (varstring eventToFire, varstring ip, varstring filename, boolean subdirs=FALSE, integer4 shotCount=1, varstring espServerIpPort=GETENV('ws\_fs\_server'))}} \\
\hline
\end{tabularx}
}

\par
Same as fMonitorFile, but does not return the DFU Workunit ID.

\par
\begin{description}
\item [\colorbox{tagtype}{\color{white} \textbf{\textsf{SEE}}}] \textbf{\underline{}} fMonitorFile
\end{description}

\rule{\linewidth}{0.5pt}
\subsection*{\textsf{\colorbox{headtoc}{\color{white} FUNCTION}
WaitDfuWorkunit}}

\hypertarget{ecldoc:file.waitdfuworkunit}{}
\hspace{0pt} \hyperlink{ecldoc:File}{File} \textbackslash 

{\renewcommand{\arraystretch}{1.5}
\begin{tabularx}{\textwidth}{|>{\raggedright\arraybackslash}l|X|}
\hline
\hspace{0pt}\mytexttt{\color{red} varstring} & \textbf{WaitDfuWorkunit} \\
\hline
\multicolumn{2}{|>{\raggedright\arraybackslash}X|}{\hspace{0pt}\mytexttt{\color{param} (varstring wuid, integer4 timeOut=-1, varstring espServerIpPort=GETENV('ws\_fs\_server'))}} \\
\hline
\end{tabularx}
}

\par
Waits for the specified DFU workunit to finish.

\par
\begin{description}
\item [\colorbox{tagtype}{\color{white} \textbf{\textsf{PARAMETER}}}] \textbf{\underline{wuid}} The dfu wfid to wait for.
\item [\colorbox{tagtype}{\color{white} \textbf{\textsf{PARAMETER}}}] \textbf{\underline{timeOut}} The time in ms to wait for the operation to complete. A value of 0 causes the call to return immediately. Defaults to no timeout (-1).
\item [\colorbox{tagtype}{\color{white} \textbf{\textsf{PARAMETER}}}] \textbf{\underline{espServerIpPort}} The url of the ESP file copying service. Defaults to the value of ws\_fs\_server in the environment.
\item [\colorbox{tagtype}{\color{white} \textbf{\textsf{RETURN}}}] \textbf{\underline{}} A string containing the final status string of the DFU workunit.
\end{description}

\rule{\linewidth}{0.5pt}
\subsection*{\textsf{\colorbox{headtoc}{\color{white} FUNCTION}
AbortDfuWorkunit}}

\hypertarget{ecldoc:file.abortdfuworkunit}{}
\hspace{0pt} \hyperlink{ecldoc:File}{File} \textbackslash 

{\renewcommand{\arraystretch}{1.5}
\begin{tabularx}{\textwidth}{|>{\raggedright\arraybackslash}l|X|}
\hline
\hspace{0pt}\mytexttt{\color{red} } & \textbf{AbortDfuWorkunit} \\
\hline
\multicolumn{2}{|>{\raggedright\arraybackslash}X|}{\hspace{0pt}\mytexttt{\color{param} (varstring wuid, varstring espServerIpPort=GETENV('ws\_fs\_server'))}} \\
\hline
\end{tabularx}
}

\par
Aborts the specified DFU workunit.

\par
\begin{description}
\item [\colorbox{tagtype}{\color{white} \textbf{\textsf{PARAMETER}}}] \textbf{\underline{wuid}} The dfu wfid to abort.
\item [\colorbox{tagtype}{\color{white} \textbf{\textsf{PARAMETER}}}] \textbf{\underline{espServerIpPort}} The url of the ESP file copying service. Defaults to the value of ws\_fs\_server in the environment.
\end{description}

\rule{\linewidth}{0.5pt}
\subsection*{\textsf{\colorbox{headtoc}{\color{white} FUNCTION}
CreateSuperFile}}

\hypertarget{ecldoc:file.createsuperfile}{}
\hspace{0pt} \hyperlink{ecldoc:File}{File} \textbackslash 

{\renewcommand{\arraystretch}{1.5}
\begin{tabularx}{\textwidth}{|>{\raggedright\arraybackslash}l|X|}
\hline
\hspace{0pt}\mytexttt{\color{red} } & \textbf{CreateSuperFile} \\
\hline
\multicolumn{2}{|>{\raggedright\arraybackslash}X|}{\hspace{0pt}\mytexttt{\color{param} (varstring superName, boolean sequentialParts=FALSE, boolean allowExist=FALSE)}} \\
\hline
\end{tabularx}
}

\par
Creates an empty superfile. This function is not included in a superfile transaction.

\par
\begin{description}
\item [\colorbox{tagtype}{\color{white} \textbf{\textsf{PARAMETER}}}] \textbf{\underline{superName}} The logical name of the superfile.
\item [\colorbox{tagtype}{\color{white} \textbf{\textsf{PARAMETER}}}] \textbf{\underline{sequentialParts}} Whether the sub-files must be sequentially ordered. Default to FALSE.
\item [\colorbox{tagtype}{\color{white} \textbf{\textsf{PARAMETER}}}] \textbf{\underline{allowExist}} Indicating whether to post an error if the superfile already exists. If TRUE, no error is posted. Defaults to FALSE.
\end{description}

\rule{\linewidth}{0.5pt}
\subsection*{\textsf{\colorbox{headtoc}{\color{white} FUNCTION}
SuperFileExists}}

\hypertarget{ecldoc:file.superfileexists}{}
\hspace{0pt} \hyperlink{ecldoc:File}{File} \textbackslash 

{\renewcommand{\arraystretch}{1.5}
\begin{tabularx}{\textwidth}{|>{\raggedright\arraybackslash}l|X|}
\hline
\hspace{0pt}\mytexttt{\color{red} boolean} & \textbf{SuperFileExists} \\
\hline
\multicolumn{2}{|>{\raggedright\arraybackslash}X|}{\hspace{0pt}\mytexttt{\color{param} (varstring superName)}} \\
\hline
\end{tabularx}
}

\par
Checks if the specified filename is present in the Distributed File Utility (DFU) and is a SuperFile.

\par
\begin{description}
\item [\colorbox{tagtype}{\color{white} \textbf{\textsf{PARAMETER}}}] \textbf{\underline{superName}} The logical name of the superfile.
\item [\colorbox{tagtype}{\color{white} \textbf{\textsf{RETURN}}}] \textbf{\underline{}} Whether the file exists.
\item [\colorbox{tagtype}{\color{white} \textbf{\textsf{SEE}}}] \textbf{\underline{}} FileExists
\end{description}

\rule{\linewidth}{0.5pt}
\subsection*{\textsf{\colorbox{headtoc}{\color{white} FUNCTION}
DeleteSuperFile}}

\hypertarget{ecldoc:file.deletesuperfile}{}
\hspace{0pt} \hyperlink{ecldoc:File}{File} \textbackslash 

{\renewcommand{\arraystretch}{1.5}
\begin{tabularx}{\textwidth}{|>{\raggedright\arraybackslash}l|X|}
\hline
\hspace{0pt}\mytexttt{\color{red} } & \textbf{DeleteSuperFile} \\
\hline
\multicolumn{2}{|>{\raggedright\arraybackslash}X|}{\hspace{0pt}\mytexttt{\color{param} (varstring superName, boolean deletesub=FALSE)}} \\
\hline
\end{tabularx}
}

\par
Deletes the superfile.

\par
\begin{description}
\item [\colorbox{tagtype}{\color{white} \textbf{\textsf{PARAMETER}}}] \textbf{\underline{superName}} The logical name of the superfile.
\item [\colorbox{tagtype}{\color{white} \textbf{\textsf{SEE}}}] \textbf{\underline{}} FileExists
\end{description}

\rule{\linewidth}{0.5pt}
\subsection*{\textsf{\colorbox{headtoc}{\color{white} FUNCTION}
GetSuperFileSubCount}}

\hypertarget{ecldoc:file.getsuperfilesubcount}{}
\hspace{0pt} \hyperlink{ecldoc:File}{File} \textbackslash 

{\renewcommand{\arraystretch}{1.5}
\begin{tabularx}{\textwidth}{|>{\raggedright\arraybackslash}l|X|}
\hline
\hspace{0pt}\mytexttt{\color{red} unsigned4} & \textbf{GetSuperFileSubCount} \\
\hline
\multicolumn{2}{|>{\raggedright\arraybackslash}X|}{\hspace{0pt}\mytexttt{\color{param} (varstring superName)}} \\
\hline
\end{tabularx}
}

\par
Returns the number of sub-files contained within a superfile.

\par
\begin{description}
\item [\colorbox{tagtype}{\color{white} \textbf{\textsf{PARAMETER}}}] \textbf{\underline{superName}} The logical name of the superfile.
\item [\colorbox{tagtype}{\color{white} \textbf{\textsf{RETURN}}}] \textbf{\underline{}} The number of sub-files within the superfile.
\end{description}

\rule{\linewidth}{0.5pt}
\subsection*{\textsf{\colorbox{headtoc}{\color{white} FUNCTION}
GetSuperFileSubName}}

\hypertarget{ecldoc:file.getsuperfilesubname}{}
\hspace{0pt} \hyperlink{ecldoc:File}{File} \textbackslash 

{\renewcommand{\arraystretch}{1.5}
\begin{tabularx}{\textwidth}{|>{\raggedright\arraybackslash}l|X|}
\hline
\hspace{0pt}\mytexttt{\color{red} varstring} & \textbf{GetSuperFileSubName} \\
\hline
\multicolumn{2}{|>{\raggedright\arraybackslash}X|}{\hspace{0pt}\mytexttt{\color{param} (varstring superName, unsigned4 fileNum, boolean absPath=FALSE)}} \\
\hline
\end{tabularx}
}

\par
Returns the name of the Nth sub-file within a superfile.

\par
\begin{description}
\item [\colorbox{tagtype}{\color{white} \textbf{\textsf{PARAMETER}}}] \textbf{\underline{superName}} The logical name of the superfile.
\item [\colorbox{tagtype}{\color{white} \textbf{\textsf{PARAMETER}}}] \textbf{\underline{fileNum}} The 1-based position of the sub-file to return the name of.
\item [\colorbox{tagtype}{\color{white} \textbf{\textsf{PARAMETER}}}] \textbf{\underline{absPath}} Whether to prepend '\~{}' to the name of the resulting logical file name.
\item [\colorbox{tagtype}{\color{white} \textbf{\textsf{RETURN}}}] \textbf{\underline{}} The logical name of the selected sub-file.
\end{description}

\rule{\linewidth}{0.5pt}
\subsection*{\textsf{\colorbox{headtoc}{\color{white} FUNCTION}
FindSuperFileSubName}}

\hypertarget{ecldoc:file.findsuperfilesubname}{}
\hspace{0pt} \hyperlink{ecldoc:File}{File} \textbackslash 

{\renewcommand{\arraystretch}{1.5}
\begin{tabularx}{\textwidth}{|>{\raggedright\arraybackslash}l|X|}
\hline
\hspace{0pt}\mytexttt{\color{red} unsigned4} & \textbf{FindSuperFileSubName} \\
\hline
\multicolumn{2}{|>{\raggedright\arraybackslash}X|}{\hspace{0pt}\mytexttt{\color{param} (varstring superName, varstring subName)}} \\
\hline
\end{tabularx}
}

\par
Returns the position of a file within a superfile.

\par
\begin{description}
\item [\colorbox{tagtype}{\color{white} \textbf{\textsf{PARAMETER}}}] \textbf{\underline{superName}} The logical name of the superfile.
\item [\colorbox{tagtype}{\color{white} \textbf{\textsf{PARAMETER}}}] \textbf{\underline{subName}} The logical name of the sub-file.
\item [\colorbox{tagtype}{\color{white} \textbf{\textsf{RETURN}}}] \textbf{\underline{}} The 1-based position of the sub-file within the superfile.
\end{description}

\rule{\linewidth}{0.5pt}
\subsection*{\textsf{\colorbox{headtoc}{\color{white} FUNCTION}
StartSuperFileTransaction}}

\hypertarget{ecldoc:file.startsuperfiletransaction}{}
\hspace{0pt} \hyperlink{ecldoc:File}{File} \textbackslash 

{\renewcommand{\arraystretch}{1.5}
\begin{tabularx}{\textwidth}{|>{\raggedright\arraybackslash}l|X|}
\hline
\hspace{0pt}\mytexttt{\color{red} } & \textbf{StartSuperFileTransaction} \\
\hline
\multicolumn{2}{|>{\raggedright\arraybackslash}X|}{\hspace{0pt}\mytexttt{\color{param} ()}} \\
\hline
\end{tabularx}
}

\par
Starts a superfile transaction. All superfile operations within the transaction will either be executed atomically or rolled back when the transaction is finished.


\rule{\linewidth}{0.5pt}
\subsection*{\textsf{\colorbox{headtoc}{\color{white} FUNCTION}
AddSuperFile}}

\hypertarget{ecldoc:file.addsuperfile}{}
\hspace{0pt} \hyperlink{ecldoc:File}{File} \textbackslash 

{\renewcommand{\arraystretch}{1.5}
\begin{tabularx}{\textwidth}{|>{\raggedright\arraybackslash}l|X|}
\hline
\hspace{0pt}\mytexttt{\color{red} } & \textbf{AddSuperFile} \\
\hline
\multicolumn{2}{|>{\raggedright\arraybackslash}X|}{\hspace{0pt}\mytexttt{\color{param} (varstring superName, varstring subName, unsigned4 atPos=0, boolean addContents=FALSE, boolean strict=FALSE)}} \\
\hline
\end{tabularx}
}

\par
Adds a file to a superfile.

\par
\begin{description}
\item [\colorbox{tagtype}{\color{white} \textbf{\textsf{PARAMETER}}}] \textbf{\underline{superName}} The logical name of the superfile.
\item [\colorbox{tagtype}{\color{white} \textbf{\textsf{PARAMETER}}}] \textbf{\underline{subName}} The name of the logical file to add.
\item [\colorbox{tagtype}{\color{white} \textbf{\textsf{PARAMETER}}}] \textbf{\underline{atPos}} The position to add the sub-file, or 0 to append. Defaults to 0.
\item [\colorbox{tagtype}{\color{white} \textbf{\textsf{PARAMETER}}}] \textbf{\underline{addContents}} Controls whether adding a superfile adds the superfile, or its contents. Defaults to FALSE (do not expand).
\item [\colorbox{tagtype}{\color{white} \textbf{\textsf{PARAMETER}}}] \textbf{\underline{strict}} Check addContents only if subName is a superfile, and ensure superfiles exist.
\end{description}

\rule{\linewidth}{0.5pt}
\subsection*{\textsf{\colorbox{headtoc}{\color{white} FUNCTION}
RemoveSuperFile}}

\hypertarget{ecldoc:file.removesuperfile}{}
\hspace{0pt} \hyperlink{ecldoc:File}{File} \textbackslash 

{\renewcommand{\arraystretch}{1.5}
\begin{tabularx}{\textwidth}{|>{\raggedright\arraybackslash}l|X|}
\hline
\hspace{0pt}\mytexttt{\color{red} } & \textbf{RemoveSuperFile} \\
\hline
\multicolumn{2}{|>{\raggedright\arraybackslash}X|}{\hspace{0pt}\mytexttt{\color{param} (varstring superName, varstring subName, boolean del=FALSE, boolean removeContents=FALSE)}} \\
\hline
\end{tabularx}
}

\par
Removes a sub-file from a superfile.

\par
\begin{description}
\item [\colorbox{tagtype}{\color{white} \textbf{\textsf{PARAMETER}}}] \textbf{\underline{superName}} The logical name of the superfile.
\item [\colorbox{tagtype}{\color{white} \textbf{\textsf{PARAMETER}}}] \textbf{\underline{subName}} The name of the sub-file to remove.
\item [\colorbox{tagtype}{\color{white} \textbf{\textsf{PARAMETER}}}] \textbf{\underline{del}} Indicates whether the sub-file should also be removed from the disk. Defaults to FALSE.
\item [\colorbox{tagtype}{\color{white} \textbf{\textsf{PARAMETER}}}] \textbf{\underline{removeContents}} Controls whether the contents of a sub-file which is a superfile should be recursively removed. Defaults to FALSE.
\end{description}

\rule{\linewidth}{0.5pt}
\subsection*{\textsf{\colorbox{headtoc}{\color{white} FUNCTION}
ClearSuperFile}}

\hypertarget{ecldoc:file.clearsuperfile}{}
\hspace{0pt} \hyperlink{ecldoc:File}{File} \textbackslash 

{\renewcommand{\arraystretch}{1.5}
\begin{tabularx}{\textwidth}{|>{\raggedright\arraybackslash}l|X|}
\hline
\hspace{0pt}\mytexttt{\color{red} } & \textbf{ClearSuperFile} \\
\hline
\multicolumn{2}{|>{\raggedright\arraybackslash}X|}{\hspace{0pt}\mytexttt{\color{param} (varstring superName, boolean del=FALSE)}} \\
\hline
\end{tabularx}
}

\par
Removes all sub-files from a superfile.

\par
\begin{description}
\item [\colorbox{tagtype}{\color{white} \textbf{\textsf{PARAMETER}}}] \textbf{\underline{superName}} The logical name of the superfile.
\item [\colorbox{tagtype}{\color{white} \textbf{\textsf{PARAMETER}}}] \textbf{\underline{del}} Indicates whether the sub-files should also be removed from the disk. Defaults to FALSE.
\end{description}

\rule{\linewidth}{0.5pt}
\subsection*{\textsf{\colorbox{headtoc}{\color{white} FUNCTION}
RemoveOwnedSubFiles}}

\hypertarget{ecldoc:file.removeownedsubfiles}{}
\hspace{0pt} \hyperlink{ecldoc:File}{File} \textbackslash 

{\renewcommand{\arraystretch}{1.5}
\begin{tabularx}{\textwidth}{|>{\raggedright\arraybackslash}l|X|}
\hline
\hspace{0pt}\mytexttt{\color{red} } & \textbf{RemoveOwnedSubFiles} \\
\hline
\multicolumn{2}{|>{\raggedright\arraybackslash}X|}{\hspace{0pt}\mytexttt{\color{param} (varstring superName, boolean del=FALSE)}} \\
\hline
\end{tabularx}
}

\par
Removes all soley-owned sub-files from a superfile. If a sub-file is also contained within another superfile then it is retained.

\par
\begin{description}
\item [\colorbox{tagtype}{\color{white} \textbf{\textsf{PARAMETER}}}] \textbf{\underline{superName}} The logical name of the superfile.
\end{description}

\rule{\linewidth}{0.5pt}
\subsection*{\textsf{\colorbox{headtoc}{\color{white} FUNCTION}
DeleteOwnedSubFiles}}

\hypertarget{ecldoc:file.deleteownedsubfiles}{}
\hspace{0pt} \hyperlink{ecldoc:File}{File} \textbackslash 

{\renewcommand{\arraystretch}{1.5}
\begin{tabularx}{\textwidth}{|>{\raggedright\arraybackslash}l|X|}
\hline
\hspace{0pt}\mytexttt{\color{red} } & \textbf{DeleteOwnedSubFiles} \\
\hline
\multicolumn{2}{|>{\raggedright\arraybackslash}X|}{\hspace{0pt}\mytexttt{\color{param} (varstring superName)}} \\
\hline
\end{tabularx}
}

\par
Legacy version of RemoveOwnedSubFiles which was incorrectly named in a previous version.

\par
\begin{description}
\item [\colorbox{tagtype}{\color{white} \textbf{\textsf{SEE}}}] \textbf{\underline{}} RemoveOwnedSubFIles
\end{description}

\rule{\linewidth}{0.5pt}
\subsection*{\textsf{\colorbox{headtoc}{\color{white} FUNCTION}
SwapSuperFile}}

\hypertarget{ecldoc:file.swapsuperfile}{}
\hspace{0pt} \hyperlink{ecldoc:File}{File} \textbackslash 

{\renewcommand{\arraystretch}{1.5}
\begin{tabularx}{\textwidth}{|>{\raggedright\arraybackslash}l|X|}
\hline
\hspace{0pt}\mytexttt{\color{red} } & \textbf{SwapSuperFile} \\
\hline
\multicolumn{2}{|>{\raggedright\arraybackslash}X|}{\hspace{0pt}\mytexttt{\color{param} (varstring superName1, varstring superName2)}} \\
\hline
\end{tabularx}
}

\par
Swap the contents of two superfiles.

\par
\begin{description}
\item [\colorbox{tagtype}{\color{white} \textbf{\textsf{PARAMETER}}}] \textbf{\underline{superName1}} The logical name of the first superfile.
\item [\colorbox{tagtype}{\color{white} \textbf{\textsf{PARAMETER}}}] \textbf{\underline{superName2}} The logical name of the second superfile.
\end{description}

\rule{\linewidth}{0.5pt}
\subsection*{\textsf{\colorbox{headtoc}{\color{white} FUNCTION}
ReplaceSuperFile}}

\hypertarget{ecldoc:file.replacesuperfile}{}
\hspace{0pt} \hyperlink{ecldoc:File}{File} \textbackslash 

{\renewcommand{\arraystretch}{1.5}
\begin{tabularx}{\textwidth}{|>{\raggedright\arraybackslash}l|X|}
\hline
\hspace{0pt}\mytexttt{\color{red} } & \textbf{ReplaceSuperFile} \\
\hline
\multicolumn{2}{|>{\raggedright\arraybackslash}X|}{\hspace{0pt}\mytexttt{\color{param} (varstring superName, varstring oldSubFile, varstring newSubFile)}} \\
\hline
\end{tabularx}
}

\par
Removes a sub-file from a superfile and replaces it with another.

\par
\begin{description}
\item [\colorbox{tagtype}{\color{white} \textbf{\textsf{PARAMETER}}}] \textbf{\underline{superName}} The logical name of the superfile.
\item [\colorbox{tagtype}{\color{white} \textbf{\textsf{PARAMETER}}}] \textbf{\underline{oldSubFile}} The logical name of the sub-file to remove.
\item [\colorbox{tagtype}{\color{white} \textbf{\textsf{PARAMETER}}}] \textbf{\underline{newSubFile}} The logical name of the sub-file to replace within the superfile.
\end{description}

\rule{\linewidth}{0.5pt}
\subsection*{\textsf{\colorbox{headtoc}{\color{white} FUNCTION}
FinishSuperFileTransaction}}

\hypertarget{ecldoc:file.finishsuperfiletransaction}{}
\hspace{0pt} \hyperlink{ecldoc:File}{File} \textbackslash 

{\renewcommand{\arraystretch}{1.5}
\begin{tabularx}{\textwidth}{|>{\raggedright\arraybackslash}l|X|}
\hline
\hspace{0pt}\mytexttt{\color{red} } & \textbf{FinishSuperFileTransaction} \\
\hline
\multicolumn{2}{|>{\raggedright\arraybackslash}X|}{\hspace{0pt}\mytexttt{\color{param} (boolean rollback=FALSE)}} \\
\hline
\end{tabularx}
}

\par
Finishes a superfile transaction. This executes all the operations since the matching StartSuperFileTransaction(). If there are any errors, then all of the operations are rolled back.


\rule{\linewidth}{0.5pt}
\subsection*{\textsf{\colorbox{headtoc}{\color{white} FUNCTION}
SuperFileContents}}

\hypertarget{ecldoc:file.superfilecontents}{}
\hspace{0pt} \hyperlink{ecldoc:File}{File} \textbackslash 

{\renewcommand{\arraystretch}{1.5}
\begin{tabularx}{\textwidth}{|>{\raggedright\arraybackslash}l|X|}
\hline
\hspace{0pt}\mytexttt{\color{red} dataset(FsLogicalFileNameRecord)} & \textbf{SuperFileContents} \\
\hline
\multicolumn{2}{|>{\raggedright\arraybackslash}X|}{\hspace{0pt}\mytexttt{\color{param} (varstring superName, boolean recurse=FALSE)}} \\
\hline
\end{tabularx}
}

\par
Returns the list of sub-files contained within a superfile.

\par
\begin{description}
\item [\colorbox{tagtype}{\color{white} \textbf{\textsf{PARAMETER}}}] \textbf{\underline{superName}} The logical name of the superfile.
\item [\colorbox{tagtype}{\color{white} \textbf{\textsf{PARAMETER}}}] \textbf{\underline{recurse}} Should the contents of child-superfiles be expanded. Default is FALSE.
\item [\colorbox{tagtype}{\color{white} \textbf{\textsf{RETURN}}}] \textbf{\underline{}} A dataset containing the names of the sub-files.
\end{description}

\rule{\linewidth}{0.5pt}
\subsection*{\textsf{\colorbox{headtoc}{\color{white} FUNCTION}
LogicalFileSuperOwners}}

\hypertarget{ecldoc:file.logicalfilesuperowners}{}
\hspace{0pt} \hyperlink{ecldoc:File}{File} \textbackslash 

{\renewcommand{\arraystretch}{1.5}
\begin{tabularx}{\textwidth}{|>{\raggedright\arraybackslash}l|X|}
\hline
\hspace{0pt}\mytexttt{\color{red} dataset(FsLogicalFileNameRecord)} & \textbf{LogicalFileSuperOwners} \\
\hline
\multicolumn{2}{|>{\raggedright\arraybackslash}X|}{\hspace{0pt}\mytexttt{\color{param} (varstring name)}} \\
\hline
\end{tabularx}
}

\par
Returns the list of superfiles that a logical file is contained within.

\par
\begin{description}
\item [\colorbox{tagtype}{\color{white} \textbf{\textsf{PARAMETER}}}] \textbf{\underline{name}} The name of the logical file.
\item [\colorbox{tagtype}{\color{white} \textbf{\textsf{RETURN}}}] \textbf{\underline{}} A dataset containing the names of the superfiles.
\end{description}

\rule{\linewidth}{0.5pt}
\subsection*{\textsf{\colorbox{headtoc}{\color{white} FUNCTION}
LogicalFileSuperSubList}}

\hypertarget{ecldoc:file.logicalfilesupersublist}{}
\hspace{0pt} \hyperlink{ecldoc:File}{File} \textbackslash 

{\renewcommand{\arraystretch}{1.5}
\begin{tabularx}{\textwidth}{|>{\raggedright\arraybackslash}l|X|}
\hline
\hspace{0pt}\mytexttt{\color{red} dataset(FsLogicalSuperSubRecord)} & \textbf{LogicalFileSuperSubList} \\
\hline
\multicolumn{2}{|>{\raggedright\arraybackslash}X|}{\hspace{0pt}\mytexttt{\color{param} ()}} \\
\hline
\end{tabularx}
}

\par
Returns the list of all the superfiles in the system and their component sub-files.

\par
\begin{description}
\item [\colorbox{tagtype}{\color{white} \textbf{\textsf{RETURN}}}] \textbf{\underline{}} A dataset containing pairs of superName,subName for each component file.
\end{description}

\rule{\linewidth}{0.5pt}
\subsection*{\textsf{\colorbox{headtoc}{\color{white} FUNCTION}
fPromoteSuperFileList}}

\hypertarget{ecldoc:file.fpromotesuperfilelist}{}
\hspace{0pt} \hyperlink{ecldoc:File}{File} \textbackslash 

{\renewcommand{\arraystretch}{1.5}
\begin{tabularx}{\textwidth}{|>{\raggedright\arraybackslash}l|X|}
\hline
\hspace{0pt}\mytexttt{\color{red} varstring} & \textbf{fPromoteSuperFileList} \\
\hline
\multicolumn{2}{|>{\raggedright\arraybackslash}X|}{\hspace{0pt}\mytexttt{\color{param} (set of varstring superNames, varstring addHead='', boolean delTail=FALSE, boolean createOnlyOne=FALSE, boolean reverse=FALSE)}} \\
\hline
\end{tabularx}
}

\par
Moves the sub-files from the first entry in the list of superfiles to the next in the list, repeating the process through the list of superfiles.

\par
\begin{description}
\item [\colorbox{tagtype}{\color{white} \textbf{\textsf{PARAMETER}}}] \textbf{\underline{superNames}} A set of the names of the superfiles to act on. Any that do not exist will be created. The contents of each superfile will be moved to the next in the list.
\item [\colorbox{tagtype}{\color{white} \textbf{\textsf{PARAMETER}}}] \textbf{\underline{addHead}} A string containing a comma-delimited list of logical file names to add to the first superfile after the promotion process is complete. Defaults to ''.
\item [\colorbox{tagtype}{\color{white} \textbf{\textsf{PARAMETER}}}] \textbf{\underline{delTail}} Indicates whether to physically delete the contents moved out of the last superfile. The default is FALSE.
\item [\colorbox{tagtype}{\color{white} \textbf{\textsf{PARAMETER}}}] \textbf{\underline{createOnlyOne}} Specifies whether to only create a single superfile (truncate the list at the first non-existent superfile). The default is FALSE.
\item [\colorbox{tagtype}{\color{white} \textbf{\textsf{PARAMETER}}}] \textbf{\underline{reverse}} Reverse the order of processing the superfiles list, effectively 'demoting' instead of 'promoting' the sub-files. The default is FALSE.
\item [\colorbox{tagtype}{\color{white} \textbf{\textsf{RETURN}}}] \textbf{\underline{}} A string containing a comma separated list of the previous sub-file contents of the emptied superfile.
\end{description}

\rule{\linewidth}{0.5pt}
\subsection*{\textsf{\colorbox{headtoc}{\color{white} FUNCTION}
PromoteSuperFileList}}

\hypertarget{ecldoc:file.promotesuperfilelist}{}
\hspace{0pt} \hyperlink{ecldoc:File}{File} \textbackslash 

{\renewcommand{\arraystretch}{1.5}
\begin{tabularx}{\textwidth}{|>{\raggedright\arraybackslash}l|X|}
\hline
\hspace{0pt}\mytexttt{\color{red} } & \textbf{PromoteSuperFileList} \\
\hline
\multicolumn{2}{|>{\raggedright\arraybackslash}X|}{\hspace{0pt}\mytexttt{\color{param} (set of varstring superNames, varstring addHead='', boolean delTail=FALSE, boolean createOnlyOne=FALSE, boolean reverse=FALSE)}} \\
\hline
\end{tabularx}
}

\par
Same as fPromoteSuperFileList, but does not return the DFU Workunit ID.

\par
\begin{description}
\item [\colorbox{tagtype}{\color{white} \textbf{\textsf{SEE}}}] \textbf{\underline{}} fPromoteSuperFileList
\end{description}

\rule{\linewidth}{0.5pt}



\chapter*{math}
\hypertarget{ecldoc:toc:math}{}

\section*{\underline{IMPORTS}}

\section*{\underline{DESCRIPTIONS}}
\subsection*{MODULE : Math}
\hypertarget{ecldoc:Math}{}
\hyperlink{ecldoc:toc:root}{Up} :

{\renewcommand{\arraystretch}{1.5}
\begin{tabularx}{\textwidth}{|>{\raggedright\arraybackslash}l|X|}
\hline
\hspace{0pt} & Math \\
\hline
\end{tabularx}
}

\par


\hyperlink{ecldoc:math.infinity}{Infinity}  |
\hyperlink{ecldoc:math.nan}{NaN}  |
\hyperlink{ecldoc:math.isinfinite}{isInfinite}  |
\hyperlink{ecldoc:math.isnan}{isNaN}  |
\hyperlink{ecldoc:math.isfinite}{isFinite}  |
\hyperlink{ecldoc:math.fmod}{FMod}  |
\hyperlink{ecldoc:math.fmatch}{FMatch}  |

\rule{\linewidth}{0.5pt}

\subsection*{ATTRIBUTE : Infinity}
\hypertarget{ecldoc:math.infinity}{}
\hyperlink{ecldoc:Math}{Up} :
\hspace{0pt} \hyperlink{ecldoc:Math}{Math} \textbackslash 

{\renewcommand{\arraystretch}{1.5}
\begin{tabularx}{\textwidth}{|>{\raggedright\arraybackslash}l|X|}
\hline
\hspace{0pt}REAL8 & Infinity \\
\hline
\end{tabularx}
}

\par
Return a real ''infinity'' value.


\rule{\linewidth}{0.5pt}
\subsection*{ATTRIBUTE : NaN}
\hypertarget{ecldoc:math.nan}{}
\hyperlink{ecldoc:Math}{Up} :
\hspace{0pt} \hyperlink{ecldoc:Math}{Math} \textbackslash 

{\renewcommand{\arraystretch}{1.5}
\begin{tabularx}{\textwidth}{|>{\raggedright\arraybackslash}l|X|}
\hline
\hspace{0pt}REAL8 & NaN \\
\hline
\end{tabularx}
}

\par
Return a non-signalling NaN (Not a Number)value.


\rule{\linewidth}{0.5pt}
\subsection*{FUNCTION : isInfinite}
\hypertarget{ecldoc:math.isinfinite}{}
\hyperlink{ecldoc:Math}{Up} :
\hspace{0pt} \hyperlink{ecldoc:Math}{Math} \textbackslash 

{\renewcommand{\arraystretch}{1.5}
\begin{tabularx}{\textwidth}{|>{\raggedright\arraybackslash}l|X|}
\hline
\hspace{0pt}BOOLEAN & isInfinite \\
\hline
\multicolumn{2}{|>{\raggedright\arraybackslash}X|}{\hspace{0pt}(REAL8 val)} \\
\hline
\end{tabularx}
}

\par
Return whether a real value is infinite (positive or negative).

\par
\begin{description}
\item [\textbf{Parameter}] val ||| The value to test.
\end{description}

\rule{\linewidth}{0.5pt}
\subsection*{FUNCTION : isNaN}
\hypertarget{ecldoc:math.isnan}{}
\hyperlink{ecldoc:Math}{Up} :
\hspace{0pt} \hyperlink{ecldoc:Math}{Math} \textbackslash 

{\renewcommand{\arraystretch}{1.5}
\begin{tabularx}{\textwidth}{|>{\raggedright\arraybackslash}l|X|}
\hline
\hspace{0pt}BOOLEAN & isNaN \\
\hline
\multicolumn{2}{|>{\raggedright\arraybackslash}X|}{\hspace{0pt}(REAL8 val)} \\
\hline
\end{tabularx}
}

\par
Return whether a real value is a NaN (not a number) value.

\par
\begin{description}
\item [\textbf{Parameter}] val ||| The value to test.
\end{description}

\rule{\linewidth}{0.5pt}
\subsection*{FUNCTION : isFinite}
\hypertarget{ecldoc:math.isfinite}{}
\hyperlink{ecldoc:Math}{Up} :
\hspace{0pt} \hyperlink{ecldoc:Math}{Math} \textbackslash 

{\renewcommand{\arraystretch}{1.5}
\begin{tabularx}{\textwidth}{|>{\raggedright\arraybackslash}l|X|}
\hline
\hspace{0pt}BOOLEAN & isFinite \\
\hline
\multicolumn{2}{|>{\raggedright\arraybackslash}X|}{\hspace{0pt}(REAL8 val)} \\
\hline
\end{tabularx}
}

\par
Return whether a real value is a valid value (neither infinite not NaN).

\par
\begin{description}
\item [\textbf{Parameter}] val ||| The value to test.
\end{description}

\rule{\linewidth}{0.5pt}
\subsection*{FUNCTION : FMod}
\hypertarget{ecldoc:math.fmod}{}
\hyperlink{ecldoc:Math}{Up} :
\hspace{0pt} \hyperlink{ecldoc:Math}{Math} \textbackslash 

{\renewcommand{\arraystretch}{1.5}
\begin{tabularx}{\textwidth}{|>{\raggedright\arraybackslash}l|X|}
\hline
\hspace{0pt}REAL8 & FMod \\
\hline
\multicolumn{2}{|>{\raggedright\arraybackslash}X|}{\hspace{0pt}(REAL8 numer, REAL8 denom)} \\
\hline
\end{tabularx}
}

\par
Returns the floating-point remainder of numer/denom (rounded towards zero). If denom is zero, the result depends on the -fdivideByZero flag: 'zero' or unset: return zero. 'nan': return a non-signalling NaN value 'fail': throw an exception

\par
\begin{description}
\item [\textbf{Parameter}] numer ||| The numerator.
\item [\textbf{Parameter}] denom ||| The numerator.
\end{description}

\rule{\linewidth}{0.5pt}
\subsection*{FUNCTION : FMatch}
\hypertarget{ecldoc:math.fmatch}{}
\hyperlink{ecldoc:Math}{Up} :
\hspace{0pt} \hyperlink{ecldoc:Math}{Math} \textbackslash 

{\renewcommand{\arraystretch}{1.5}
\begin{tabularx}{\textwidth}{|>{\raggedright\arraybackslash}l|X|}
\hline
\hspace{0pt}BOOLEAN & FMatch \\
\hline
\multicolumn{2}{|>{\raggedright\arraybackslash}X|}{\hspace{0pt}(REAL8 a, REAL8 b, REAL8 epsilon=0.0)} \\
\hline
\end{tabularx}
}

\par
Returns whether two floating point values are the same, within margin of error epsilon.

\par
\begin{description}
\item [\textbf{Parameter}] a ||| The first value.
\item [\textbf{Parameter}] b ||| The second value.
\item [\textbf{Parameter}] epsilon ||| The allowable margin of error.
\end{description}

\rule{\linewidth}{0.5pt}



\chapter*{Metaphone}
\hypertarget{ecldoc:toc:Metaphone}{}

\section*{\underline{IMPORTS}}
\begin{itemize}
\item lib\_metaphone
\end{itemize}

\section*{\underline{DESCRIPTIONS}}
\subsection*{MODULE : Metaphone}
\hypertarget{ecldoc:Metaphone}{}
\hyperlink{ecldoc:toc:root}{Up} :

{\renewcommand{\arraystretch}{1.5}
\begin{tabularx}{\textwidth}{|>{\raggedright\arraybackslash}l|X|}
\hline
\hspace{0pt} & Metaphone \\
\hline
\end{tabularx}
}

\par


\hyperlink{ecldoc:metaphone.primary}{primary}  |
\hyperlink{ecldoc:metaphone.secondary}{secondary}  |
\hyperlink{ecldoc:metaphone.double}{double}  |

\rule{\linewidth}{0.5pt}

\subsection*{FUNCTION : primary}
\hypertarget{ecldoc:metaphone.primary}{}
\hyperlink{ecldoc:Metaphone}{Up} :
\hspace{0pt} \hyperlink{ecldoc:Metaphone}{Metaphone} \textbackslash 

{\renewcommand{\arraystretch}{1.5}
\begin{tabularx}{\textwidth}{|>{\raggedright\arraybackslash}l|X|}
\hline
\hspace{0pt}String & primary \\
\hline
\multicolumn{2}{|>{\raggedright\arraybackslash}X|}{\hspace{0pt}(STRING src)} \\
\hline
\end{tabularx}
}

\par
Returns the primary metaphone value

\par
\begin{description}
\item [\textbf{Parameter}] src ||| The string whose metphone is to be calculated.
\item [\textbf{See}] http://en.wikipedia.org/wiki/Metaphone\#Double\_Metaphone
\end{description}

\rule{\linewidth}{0.5pt}
\subsection*{FUNCTION : secondary}
\hypertarget{ecldoc:metaphone.secondary}{}
\hyperlink{ecldoc:Metaphone}{Up} :
\hspace{0pt} \hyperlink{ecldoc:Metaphone}{Metaphone} \textbackslash 

{\renewcommand{\arraystretch}{1.5}
\begin{tabularx}{\textwidth}{|>{\raggedright\arraybackslash}l|X|}
\hline
\hspace{0pt}String & secondary \\
\hline
\multicolumn{2}{|>{\raggedright\arraybackslash}X|}{\hspace{0pt}(STRING src)} \\
\hline
\end{tabularx}
}

\par
Returns the secondary metaphone value

\par
\begin{description}
\item [\textbf{Parameter}] src ||| The string whose metphone is to be calculated.
\item [\textbf{See}] http://en.wikipedia.org/wiki/Metaphone\#Double\_Metaphone
\end{description}

\rule{\linewidth}{0.5pt}
\subsection*{FUNCTION : double}
\hypertarget{ecldoc:metaphone.double}{}
\hyperlink{ecldoc:Metaphone}{Up} :
\hspace{0pt} \hyperlink{ecldoc:Metaphone}{Metaphone} \textbackslash 

{\renewcommand{\arraystretch}{1.5}
\begin{tabularx}{\textwidth}{|>{\raggedright\arraybackslash}l|X|}
\hline
\hspace{0pt}String & double \\
\hline
\multicolumn{2}{|>{\raggedright\arraybackslash}X|}{\hspace{0pt}(STRING src)} \\
\hline
\end{tabularx}
}

\par
Returns the double metaphone value (primary and secondary concatenated

\par
\begin{description}
\item [\textbf{Parameter}] src ||| The string whose metphone is to be calculated.
\item [\textbf{See}] http://en.wikipedia.org/wiki/Metaphone\#Double\_Metaphone
\end{description}

\rule{\linewidth}{0.5pt}



\chapter*{str}
\hypertarget{ecldoc:toc:str}{}

\section*{\underline{IMPORTS}}
\begin{itemize}
\item lib\_stringlib
\end{itemize}

\section*{\underline{DESCRIPTIONS}}
\subsection*{MODULE : Str}
\hypertarget{ecldoc:Str}{}
\hyperlink{ecldoc:toc:root}{Up} :

{\renewcommand{\arraystretch}{1.5}
\begin{tabularx}{\textwidth}{|>{\raggedright\arraybackslash}l|X|}
\hline
\hspace{0pt} & Str \\
\hline
\end{tabularx}
}

\par


\hyperlink{ecldoc:str.compareignorecase}{CompareIgnoreCase}  |
\hyperlink{ecldoc:str.equalignorecase}{EqualIgnoreCase}  |
\hyperlink{ecldoc:str.find}{Find}  |
\hyperlink{ecldoc:str.findcount}{FindCount}  |
\hyperlink{ecldoc:str.wildmatch}{WildMatch}  |
\hyperlink{ecldoc:str.contains}{Contains}  |
\hyperlink{ecldoc:str.filterout}{FilterOut}  |
\hyperlink{ecldoc:str.filter}{Filter}  |
\hyperlink{ecldoc:str.substituteincluded}{SubstituteIncluded}  |
\hyperlink{ecldoc:str.substituteexcluded}{SubstituteExcluded}  |
\hyperlink{ecldoc:str.translate}{Translate}  |
\hyperlink{ecldoc:str.tolowercase}{ToLowerCase}  |
\hyperlink{ecldoc:str.touppercase}{ToUpperCase}  |
\hyperlink{ecldoc:str.tocapitalcase}{ToCapitalCase}  |
\hyperlink{ecldoc:str.totitlecase}{ToTitleCase}  |
\hyperlink{ecldoc:str.reverse}{Reverse}  |
\hyperlink{ecldoc:str.findreplace}{FindReplace}  |
\hyperlink{ecldoc:str.extract}{Extract}  |
\hyperlink{ecldoc:str.cleanspaces}{CleanSpaces}  |
\hyperlink{ecldoc:str.startswith}{StartsWith}  |
\hyperlink{ecldoc:str.endswith}{EndsWith}  |
\hyperlink{ecldoc:str.removesuffix}{RemoveSuffix}  |
\hyperlink{ecldoc:str.extractmultiple}{ExtractMultiple}  |
\hyperlink{ecldoc:str.countwords}{CountWords}  |
\hyperlink{ecldoc:str.splitwords}{SplitWords}  |
\hyperlink{ecldoc:str.combinewords}{CombineWords}  |
\hyperlink{ecldoc:str.editdistance}{EditDistance}  |
\hyperlink{ecldoc:str.editdistancewithinradius}{EditDistanceWithinRadius}  |
\hyperlink{ecldoc:str.wordcount}{WordCount}  |
\hyperlink{ecldoc:str.getnthword}{GetNthWord}  |
\hyperlink{ecldoc:str.excludefirstword}{ExcludeFirstWord}  |
\hyperlink{ecldoc:str.excludelastword}{ExcludeLastWord}  |
\hyperlink{ecldoc:str.excludenthword}{ExcludeNthWord}  |
\hyperlink{ecldoc:str.findword}{FindWord}  |
\hyperlink{ecldoc:str.repeat}{Repeat}  |
\hyperlink{ecldoc:str.tohexpairs}{ToHexPairs}  |
\hyperlink{ecldoc:str.fromhexpairs}{FromHexPairs}  |
\hyperlink{ecldoc:str.encodebase64}{EncodeBase64}  |
\hyperlink{ecldoc:str.decodebase64}{DecodeBase64}  |

\rule{\linewidth}{0.5pt}

\subsection*{FUNCTION : CompareIgnoreCase}
\hypertarget{ecldoc:str.compareignorecase}{}
\hyperlink{ecldoc:Str}{Up} :
\hspace{0pt} \hyperlink{ecldoc:Str}{Str} \textbackslash 

{\renewcommand{\arraystretch}{1.5}
\begin{tabularx}{\textwidth}{|>{\raggedright\arraybackslash}l|X|}
\hline
\hspace{0pt}INTEGER4 & CompareIgnoreCase \\
\hline
\multicolumn{2}{|>{\raggedright\arraybackslash}X|}{\hspace{0pt}(STRING src1, STRING src2)} \\
\hline
\end{tabularx}
}

\par
Compares the two strings case insensitively. Returns a negative integer, zero, or a positive integer according to whether the first string is less than, equal to, or greater than the second.

\par
\begin{description}
\item [\textbf{Parameter}] src1 ||| The first string to be compared.
\item [\textbf{Parameter}] src2 ||| The second string to be compared.
\item [\textbf{See}] Str.EqualIgnoreCase
\end{description}

\rule{\linewidth}{0.5pt}
\subsection*{FUNCTION : EqualIgnoreCase}
\hypertarget{ecldoc:str.equalignorecase}{}
\hyperlink{ecldoc:Str}{Up} :
\hspace{0pt} \hyperlink{ecldoc:Str}{Str} \textbackslash 

{\renewcommand{\arraystretch}{1.5}
\begin{tabularx}{\textwidth}{|>{\raggedright\arraybackslash}l|X|}
\hline
\hspace{0pt}BOOLEAN & EqualIgnoreCase \\
\hline
\multicolumn{2}{|>{\raggedright\arraybackslash}X|}{\hspace{0pt}(STRING src1, STRING src2)} \\
\hline
\end{tabularx}
}

\par
Tests whether the two strings are identical ignoring differences in case.

\par
\begin{description}
\item [\textbf{Parameter}] src1 ||| The first string to be compared.
\item [\textbf{Parameter}] src2 ||| The second string to be compared.
\item [\textbf{See}] Str.CompareIgnoreCase
\end{description}

\rule{\linewidth}{0.5pt}
\subsection*{FUNCTION : Find}
\hypertarget{ecldoc:str.find}{}
\hyperlink{ecldoc:Str}{Up} :
\hspace{0pt} \hyperlink{ecldoc:Str}{Str} \textbackslash 

{\renewcommand{\arraystretch}{1.5}
\begin{tabularx}{\textwidth}{|>{\raggedright\arraybackslash}l|X|}
\hline
\hspace{0pt}UNSIGNED4 & Find \\
\hline
\multicolumn{2}{|>{\raggedright\arraybackslash}X|}{\hspace{0pt}(STRING src, STRING sought, UNSIGNED4 instance = 1)} \\
\hline
\end{tabularx}
}

\par
Returns the character position of the nth match of the search string with the first string. If no match is found the attribute returns 0. If an instance is omitted the position of the first instance is returned.

\par
\begin{description}
\item [\textbf{Parameter}] src ||| The string that is searched
\item [\textbf{Parameter}] sought ||| The string being sought.
\item [\textbf{Parameter}] instance ||| Which match instance are we interested in?
\end{description}

\rule{\linewidth}{0.5pt}
\subsection*{FUNCTION : FindCount}
\hypertarget{ecldoc:str.findcount}{}
\hyperlink{ecldoc:Str}{Up} :
\hspace{0pt} \hyperlink{ecldoc:Str}{Str} \textbackslash 

{\renewcommand{\arraystretch}{1.5}
\begin{tabularx}{\textwidth}{|>{\raggedright\arraybackslash}l|X|}
\hline
\hspace{0pt}UNSIGNED4 & FindCount \\
\hline
\multicolumn{2}{|>{\raggedright\arraybackslash}X|}{\hspace{0pt}(STRING src, STRING sought)} \\
\hline
\end{tabularx}
}

\par
Returns the number of occurences of the second string within the first string.

\par
\begin{description}
\item [\textbf{Parameter}] src ||| The string that is searched
\item [\textbf{Parameter}] sought ||| The string being sought.
\end{description}

\rule{\linewidth}{0.5pt}
\subsection*{FUNCTION : WildMatch}
\hypertarget{ecldoc:str.wildmatch}{}
\hyperlink{ecldoc:Str}{Up} :
\hspace{0pt} \hyperlink{ecldoc:Str}{Str} \textbackslash 

{\renewcommand{\arraystretch}{1.5}
\begin{tabularx}{\textwidth}{|>{\raggedright\arraybackslash}l|X|}
\hline
\hspace{0pt}BOOLEAN & WildMatch \\
\hline
\multicolumn{2}{|>{\raggedright\arraybackslash}X|}{\hspace{0pt}(STRING src, STRING \_pattern, BOOLEAN ignore\_case)} \\
\hline
\end{tabularx}
}

\par
Tests if the search string matches the pattern. The pattern can contain wildcards '?' (single character) and '*' (multiple character).

\par
\begin{description}
\item [\textbf{Parameter}] src ||| The string that is being tested.
\item [\textbf{Parameter}] pattern ||| The pattern to match against.
\item [\textbf{Parameter}] ignore\_case ||| Whether to ignore differences in case between characters
\end{description}

\rule{\linewidth}{0.5pt}
\subsection*{FUNCTION : Contains}
\hypertarget{ecldoc:str.contains}{}
\hyperlink{ecldoc:Str}{Up} :
\hspace{0pt} \hyperlink{ecldoc:Str}{Str} \textbackslash 

{\renewcommand{\arraystretch}{1.5}
\begin{tabularx}{\textwidth}{|>{\raggedright\arraybackslash}l|X|}
\hline
\hspace{0pt}BOOLEAN & Contains \\
\hline
\multicolumn{2}{|>{\raggedright\arraybackslash}X|}{\hspace{0pt}(STRING src, STRING \_pattern, BOOLEAN ignore\_case)} \\
\hline
\end{tabularx}
}

\par
Tests if the search string contains each of the characters in the pattern. If the pattern contains duplicate characters those characters will match once for each occurence in the pattern.

\par
\begin{description}
\item [\textbf{Parameter}] src ||| The string that is being tested.
\item [\textbf{Parameter}] pattern ||| The pattern to match against.
\item [\textbf{Parameter}] ignore\_case ||| Whether to ignore differences in case between characters
\end{description}

\rule{\linewidth}{0.5pt}
\subsection*{FUNCTION : FilterOut}
\hypertarget{ecldoc:str.filterout}{}
\hyperlink{ecldoc:Str}{Up} :
\hspace{0pt} \hyperlink{ecldoc:Str}{Str} \textbackslash 

{\renewcommand{\arraystretch}{1.5}
\begin{tabularx}{\textwidth}{|>{\raggedright\arraybackslash}l|X|}
\hline
\hspace{0pt}STRING & FilterOut \\
\hline
\multicolumn{2}{|>{\raggedright\arraybackslash}X|}{\hspace{0pt}(STRING src, STRING filter)} \\
\hline
\end{tabularx}
}

\par
Returns the first string with all characters within the second string removed.

\par
\begin{description}
\item [\textbf{Parameter}] src ||| The string that is being tested.
\item [\textbf{Parameter}] filter ||| The string containing the set of characters to be excluded.
\item [\textbf{See}] Str.Filter
\end{description}

\rule{\linewidth}{0.5pt}
\subsection*{FUNCTION : Filter}
\hypertarget{ecldoc:str.filter}{}
\hyperlink{ecldoc:Str}{Up} :
\hspace{0pt} \hyperlink{ecldoc:Str}{Str} \textbackslash 

{\renewcommand{\arraystretch}{1.5}
\begin{tabularx}{\textwidth}{|>{\raggedright\arraybackslash}l|X|}
\hline
\hspace{0pt}STRING & Filter \\
\hline
\multicolumn{2}{|>{\raggedright\arraybackslash}X|}{\hspace{0pt}(STRING src, STRING filter)} \\
\hline
\end{tabularx}
}

\par
Returns the first string with all characters not within the second string removed.

\par
\begin{description}
\item [\textbf{Parameter}] src ||| The string that is being tested.
\item [\textbf{Parameter}] filter ||| The string containing the set of characters to be included.
\item [\textbf{See}] Str.FilterOut
\end{description}

\rule{\linewidth}{0.5pt}
\subsection*{FUNCTION : SubstituteIncluded}
\hypertarget{ecldoc:str.substituteincluded}{}
\hyperlink{ecldoc:Str}{Up} :
\hspace{0pt} \hyperlink{ecldoc:Str}{Str} \textbackslash 

{\renewcommand{\arraystretch}{1.5}
\begin{tabularx}{\textwidth}{|>{\raggedright\arraybackslash}l|X|}
\hline
\hspace{0pt}STRING & SubstituteIncluded \\
\hline
\multicolumn{2}{|>{\raggedright\arraybackslash}X|}{\hspace{0pt}(STRING src, STRING filter, STRING1 replace\_char)} \\
\hline
\end{tabularx}
}

\par
Returns the source string with the replacement character substituted for all characters included in the filter string. MORE: Should this be a general string substitution?

\par
\begin{description}
\item [\textbf{Parameter}] src ||| The string that is being tested.
\item [\textbf{Parameter}] filter ||| The string containing the set of characters to be included.
\item [\textbf{Parameter}] replace\_char ||| The character to be substituted into the result.
\item [\textbf{See}] Std.Str.Translate, Std.Str.SubstituteExcluded
\end{description}

\rule{\linewidth}{0.5pt}
\subsection*{FUNCTION : SubstituteExcluded}
\hypertarget{ecldoc:str.substituteexcluded}{}
\hyperlink{ecldoc:Str}{Up} :
\hspace{0pt} \hyperlink{ecldoc:Str}{Str} \textbackslash 

{\renewcommand{\arraystretch}{1.5}
\begin{tabularx}{\textwidth}{|>{\raggedright\arraybackslash}l|X|}
\hline
\hspace{0pt}STRING & SubstituteExcluded \\
\hline
\multicolumn{2}{|>{\raggedright\arraybackslash}X|}{\hspace{0pt}(STRING src, STRING filter, STRING1 replace\_char)} \\
\hline
\end{tabularx}
}

\par
Returns the source string with the replacement character substituted for all characters not included in the filter string. MORE: Should this be a general string substitution?

\par
\begin{description}
\item [\textbf{Parameter}] src ||| The string that is being tested.
\item [\textbf{Parameter}] filter ||| The string containing the set of characters to be included.
\item [\textbf{Parameter}] replace\_char ||| The character to be substituted into the result.
\item [\textbf{See}] Std.Str.SubstituteIncluded
\end{description}

\rule{\linewidth}{0.5pt}
\subsection*{FUNCTION : Translate}
\hypertarget{ecldoc:str.translate}{}
\hyperlink{ecldoc:Str}{Up} :
\hspace{0pt} \hyperlink{ecldoc:Str}{Str} \textbackslash 

{\renewcommand{\arraystretch}{1.5}
\begin{tabularx}{\textwidth}{|>{\raggedright\arraybackslash}l|X|}
\hline
\hspace{0pt}STRING & Translate \\
\hline
\multicolumn{2}{|>{\raggedright\arraybackslash}X|}{\hspace{0pt}(STRING src, STRING search, STRING replacement)} \\
\hline
\end{tabularx}
}

\par
Returns the source string with the all characters that match characters in the search string replaced with the character at the corresponding position in the replacement string.

\par
\begin{description}
\item [\textbf{Parameter}] src ||| The string that is being tested.
\item [\textbf{Parameter}] search ||| The string containing the set of characters to be included.
\item [\textbf{Parameter}] replacement ||| The string containing the characters to act as replacements.
\item [\textbf{See}] Std.Str.SubstituteIncluded
\end{description}

\rule{\linewidth}{0.5pt}
\subsection*{FUNCTION : ToLowerCase}
\hypertarget{ecldoc:str.tolowercase}{}
\hyperlink{ecldoc:Str}{Up} :
\hspace{0pt} \hyperlink{ecldoc:Str}{Str} \textbackslash 

{\renewcommand{\arraystretch}{1.5}
\begin{tabularx}{\textwidth}{|>{\raggedright\arraybackslash}l|X|}
\hline
\hspace{0pt}STRING & ToLowerCase \\
\hline
\multicolumn{2}{|>{\raggedright\arraybackslash}X|}{\hspace{0pt}(STRING src)} \\
\hline
\end{tabularx}
}

\par
Returns the argument string with all upper case characters converted to lower case.

\par
\begin{description}
\item [\textbf{Parameter}] src ||| The string that is being converted.
\end{description}

\rule{\linewidth}{0.5pt}
\subsection*{FUNCTION : ToUpperCase}
\hypertarget{ecldoc:str.touppercase}{}
\hyperlink{ecldoc:Str}{Up} :
\hspace{0pt} \hyperlink{ecldoc:Str}{Str} \textbackslash 

{\renewcommand{\arraystretch}{1.5}
\begin{tabularx}{\textwidth}{|>{\raggedright\arraybackslash}l|X|}
\hline
\hspace{0pt}STRING & ToUpperCase \\
\hline
\multicolumn{2}{|>{\raggedright\arraybackslash}X|}{\hspace{0pt}(STRING src)} \\
\hline
\end{tabularx}
}

\par
Return the argument string with all lower case characters converted to upper case.

\par
\begin{description}
\item [\textbf{Parameter}] src ||| The string that is being converted.
\end{description}

\rule{\linewidth}{0.5pt}
\subsection*{FUNCTION : ToCapitalCase}
\hypertarget{ecldoc:str.tocapitalcase}{}
\hyperlink{ecldoc:Str}{Up} :
\hspace{0pt} \hyperlink{ecldoc:Str}{Str} \textbackslash 

{\renewcommand{\arraystretch}{1.5}
\begin{tabularx}{\textwidth}{|>{\raggedright\arraybackslash}l|X|}
\hline
\hspace{0pt}STRING & ToCapitalCase \\
\hline
\multicolumn{2}{|>{\raggedright\arraybackslash}X|}{\hspace{0pt}(STRING src)} \\
\hline
\end{tabularx}
}

\par
Returns the argument string with the first letter of each word in upper case and all other letters left as-is. A contiguous sequence of alphanumeric characters is treated as a word.

\par
\begin{description}
\item [\textbf{Parameter}] src ||| The string that is being converted.
\end{description}

\rule{\linewidth}{0.5pt}
\subsection*{FUNCTION : ToTitleCase}
\hypertarget{ecldoc:str.totitlecase}{}
\hyperlink{ecldoc:Str}{Up} :
\hspace{0pt} \hyperlink{ecldoc:Str}{Str} \textbackslash 

{\renewcommand{\arraystretch}{1.5}
\begin{tabularx}{\textwidth}{|>{\raggedright\arraybackslash}l|X|}
\hline
\hspace{0pt}STRING & ToTitleCase \\
\hline
\multicolumn{2}{|>{\raggedright\arraybackslash}X|}{\hspace{0pt}(STRING src)} \\
\hline
\end{tabularx}
}

\par
Returns the argument string with the first letter of each word in upper case and all other letters lower case. A contiguous sequence of alphanumeric characters is treated as a word.

\par
\begin{description}
\item [\textbf{Parameter}] src ||| The string that is being converted.
\end{description}

\rule{\linewidth}{0.5pt}
\subsection*{FUNCTION : Reverse}
\hypertarget{ecldoc:str.reverse}{}
\hyperlink{ecldoc:Str}{Up} :
\hspace{0pt} \hyperlink{ecldoc:Str}{Str} \textbackslash 

{\renewcommand{\arraystretch}{1.5}
\begin{tabularx}{\textwidth}{|>{\raggedright\arraybackslash}l|X|}
\hline
\hspace{0pt}STRING & Reverse \\
\hline
\multicolumn{2}{|>{\raggedright\arraybackslash}X|}{\hspace{0pt}(STRING src)} \\
\hline
\end{tabularx}
}

\par
Returns the argument string with all characters in reverse order. Note the argument is not TRIMMED before it is reversed.

\par
\begin{description}
\item [\textbf{Parameter}] src ||| The string that is being reversed.
\end{description}

\rule{\linewidth}{0.5pt}
\subsection*{FUNCTION : FindReplace}
\hypertarget{ecldoc:str.findreplace}{}
\hyperlink{ecldoc:Str}{Up} :
\hspace{0pt} \hyperlink{ecldoc:Str}{Str} \textbackslash 

{\renewcommand{\arraystretch}{1.5}
\begin{tabularx}{\textwidth}{|>{\raggedright\arraybackslash}l|X|}
\hline
\hspace{0pt}STRING & FindReplace \\
\hline
\multicolumn{2}{|>{\raggedright\arraybackslash}X|}{\hspace{0pt}(STRING src, STRING sought, STRING replacement)} \\
\hline
\end{tabularx}
}

\par
Returns the source string with the replacement string substituted for all instances of the search string.

\par
\begin{description}
\item [\textbf{Parameter}] src ||| The string that is being transformed.
\item [\textbf{Parameter}] sought ||| The string to be replaced.
\item [\textbf{Parameter}] replacement ||| The string to be substituted into the result.
\end{description}

\rule{\linewidth}{0.5pt}
\subsection*{FUNCTION : Extract}
\hypertarget{ecldoc:str.extract}{}
\hyperlink{ecldoc:Str}{Up} :
\hspace{0pt} \hyperlink{ecldoc:Str}{Str} \textbackslash 

{\renewcommand{\arraystretch}{1.5}
\begin{tabularx}{\textwidth}{|>{\raggedright\arraybackslash}l|X|}
\hline
\hspace{0pt}STRING & Extract \\
\hline
\multicolumn{2}{|>{\raggedright\arraybackslash}X|}{\hspace{0pt}(STRING src, UNSIGNED4 instance)} \\
\hline
\end{tabularx}
}

\par
Returns the nth element from a comma separated string.

\par
\begin{description}
\item [\textbf{Parameter}] src ||| The string containing the comma separated list.
\item [\textbf{Parameter}] instance ||| Which item to select from the list.
\end{description}

\rule{\linewidth}{0.5pt}
\subsection*{FUNCTION : CleanSpaces}
\hypertarget{ecldoc:str.cleanspaces}{}
\hyperlink{ecldoc:Str}{Up} :
\hspace{0pt} \hyperlink{ecldoc:Str}{Str} \textbackslash 

{\renewcommand{\arraystretch}{1.5}
\begin{tabularx}{\textwidth}{|>{\raggedright\arraybackslash}l|X|}
\hline
\hspace{0pt}STRING & CleanSpaces \\
\hline
\multicolumn{2}{|>{\raggedright\arraybackslash}X|}{\hspace{0pt}(STRING src)} \\
\hline
\end{tabularx}
}

\par
Returns the source string with all instances of multiple adjacent space characters (2 or more spaces together) reduced to a single space character. Leading and trailing spaces are removed, and tab characters are converted to spaces.

\par
\begin{description}
\item [\textbf{Parameter}] src ||| The string to be cleaned.
\end{description}

\rule{\linewidth}{0.5pt}
\subsection*{FUNCTION : StartsWith}
\hypertarget{ecldoc:str.startswith}{}
\hyperlink{ecldoc:Str}{Up} :
\hspace{0pt} \hyperlink{ecldoc:Str}{Str} \textbackslash 

{\renewcommand{\arraystretch}{1.5}
\begin{tabularx}{\textwidth}{|>{\raggedright\arraybackslash}l|X|}
\hline
\hspace{0pt}BOOLEAN & StartsWith \\
\hline
\multicolumn{2}{|>{\raggedright\arraybackslash}X|}{\hspace{0pt}(STRING src, STRING prefix)} \\
\hline
\end{tabularx}
}

\par
Returns true if the prefix string matches the leading characters in the source string. Trailing spaces are stripped from the prefix before matching. // x.myString.StartsWith('x') as an alternative syntax would be even better

\par
\begin{description}
\item [\textbf{Parameter}] src ||| The string being searched in.
\item [\textbf{Parameter}] prefix ||| The prefix to search for.
\end{description}

\rule{\linewidth}{0.5pt}
\subsection*{FUNCTION : EndsWith}
\hypertarget{ecldoc:str.endswith}{}
\hyperlink{ecldoc:Str}{Up} :
\hspace{0pt} \hyperlink{ecldoc:Str}{Str} \textbackslash 

{\renewcommand{\arraystretch}{1.5}
\begin{tabularx}{\textwidth}{|>{\raggedright\arraybackslash}l|X|}
\hline
\hspace{0pt}BOOLEAN & EndsWith \\
\hline
\multicolumn{2}{|>{\raggedright\arraybackslash}X|}{\hspace{0pt}(STRING src, STRING suffix)} \\
\hline
\end{tabularx}
}

\par
Returns true if the suffix string matches the trailing characters in the source string. Trailing spaces are stripped from both strings before matching.

\par
\begin{description}
\item [\textbf{Parameter}] src ||| The string being searched in.
\item [\textbf{Parameter}] suffix ||| The prefix to search for.
\end{description}

\rule{\linewidth}{0.5pt}
\subsection*{FUNCTION : RemoveSuffix}
\hypertarget{ecldoc:str.removesuffix}{}
\hyperlink{ecldoc:Str}{Up} :
\hspace{0pt} \hyperlink{ecldoc:Str}{Str} \textbackslash 

{\renewcommand{\arraystretch}{1.5}
\begin{tabularx}{\textwidth}{|>{\raggedright\arraybackslash}l|X|}
\hline
\hspace{0pt}STRING & RemoveSuffix \\
\hline
\multicolumn{2}{|>{\raggedright\arraybackslash}X|}{\hspace{0pt}(STRING src, STRING suffix)} \\
\hline
\end{tabularx}
}

\par
Removes the suffix from the search string, if present, and returns the result. Trailing spaces are stripped from both strings before matching.

\par
\begin{description}
\item [\textbf{Parameter}] src ||| The string being searched in.
\item [\textbf{Parameter}] suffix ||| The prefix to search for.
\end{description}

\rule{\linewidth}{0.5pt}
\subsection*{FUNCTION : ExtractMultiple}
\hypertarget{ecldoc:str.extractmultiple}{}
\hyperlink{ecldoc:Str}{Up} :
\hspace{0pt} \hyperlink{ecldoc:Str}{Str} \textbackslash 

{\renewcommand{\arraystretch}{1.5}
\begin{tabularx}{\textwidth}{|>{\raggedright\arraybackslash}l|X|}
\hline
\hspace{0pt}STRING & ExtractMultiple \\
\hline
\multicolumn{2}{|>{\raggedright\arraybackslash}X|}{\hspace{0pt}(STRING src, UNSIGNED8 mask)} \\
\hline
\end{tabularx}
}

\par
Returns a string containing a list of elements from a comma separated string.

\par
\begin{description}
\item [\textbf{Parameter}] src ||| The string containing the comma separated list.
\item [\textbf{Parameter}] mask ||| A bitmask of which elements should be included. Bit 0 is item1, bit1 item 2 etc.
\end{description}

\rule{\linewidth}{0.5pt}
\subsection*{FUNCTION : CountWords}
\hypertarget{ecldoc:str.countwords}{}
\hyperlink{ecldoc:Str}{Up} :
\hspace{0pt} \hyperlink{ecldoc:Str}{Str} \textbackslash 

{\renewcommand{\arraystretch}{1.5}
\begin{tabularx}{\textwidth}{|>{\raggedright\arraybackslash}l|X|}
\hline
\hspace{0pt}UNSIGNED4 & CountWords \\
\hline
\multicolumn{2}{|>{\raggedright\arraybackslash}X|}{\hspace{0pt}(STRING src, STRING separator, BOOLEAN allow\_blank = FALSE)} \\
\hline
\end{tabularx}
}

\par
Returns the number of words that the string contains. Words are separated by one or more separator strings. No spaces are stripped from either string before matching.

\par
\begin{description}
\item [\textbf{Parameter}] src ||| The string being searched in.
\item [\textbf{Parameter}] separator ||| The string used to separate words
\item [\textbf{Parameter}] allow\_blank ||| Indicates if empty/blank string items are included in the results.
\end{description}

\rule{\linewidth}{0.5pt}
\subsection*{FUNCTION : SplitWords}
\hypertarget{ecldoc:str.splitwords}{}
\hyperlink{ecldoc:Str}{Up} :
\hspace{0pt} \hyperlink{ecldoc:Str}{Str} \textbackslash 

{\renewcommand{\arraystretch}{1.5}
\begin{tabularx}{\textwidth}{|>{\raggedright\arraybackslash}l|X|}
\hline
\hspace{0pt}SET OF STRING & SplitWords \\
\hline
\multicolumn{2}{|>{\raggedright\arraybackslash}X|}{\hspace{0pt}(STRING src, STRING separator, BOOLEAN allow\_blank = FALSE)} \\
\hline
\end{tabularx}
}

\par
Returns the list of words extracted from the string. Words are separated by one or more separator strings. No spaces are stripped from either string before matching.

\par
\begin{description}
\item [\textbf{Parameter}] src ||| The string being searched in.
\item [\textbf{Parameter}] separator ||| The string used to separate words
\item [\textbf{Parameter}] allow\_blank ||| Indicates if empty/blank string items are included in the results.
\end{description}

\rule{\linewidth}{0.5pt}
\subsection*{FUNCTION : CombineWords}
\hypertarget{ecldoc:str.combinewords}{}
\hyperlink{ecldoc:Str}{Up} :
\hspace{0pt} \hyperlink{ecldoc:Str}{Str} \textbackslash 

{\renewcommand{\arraystretch}{1.5}
\begin{tabularx}{\textwidth}{|>{\raggedright\arraybackslash}l|X|}
\hline
\hspace{0pt}STRING & CombineWords \\
\hline
\multicolumn{2}{|>{\raggedright\arraybackslash}X|}{\hspace{0pt}(SET OF STRING words, STRING separator)} \\
\hline
\end{tabularx}
}

\par
Returns the list of words extracted from the string. Words are separated by one or more separator strings. No spaces are stripped from either string before matching.

\par
\begin{description}
\item [\textbf{Parameter}] words ||| The set of strings to be combined.
\item [\textbf{Parameter}] separator ||| The string used to separate words.
\end{description}

\rule{\linewidth}{0.5pt}
\subsection*{FUNCTION : EditDistance}
\hypertarget{ecldoc:str.editdistance}{}
\hyperlink{ecldoc:Str}{Up} :
\hspace{0pt} \hyperlink{ecldoc:Str}{Str} \textbackslash 

{\renewcommand{\arraystretch}{1.5}
\begin{tabularx}{\textwidth}{|>{\raggedright\arraybackslash}l|X|}
\hline
\hspace{0pt}UNSIGNED4 & EditDistance \\
\hline
\multicolumn{2}{|>{\raggedright\arraybackslash}X|}{\hspace{0pt}(STRING \_left, STRING \_right)} \\
\hline
\end{tabularx}
}

\par
Returns the minimum edit distance between the two strings. An insert change or delete counts as a single edit. The two strings are trimmed before comparing.

\par
\begin{description}
\item [\textbf{Parameter}] \_left ||| The first string to be compared.
\item [\textbf{Parameter}] \_right ||| The second string to be compared.
\item [\textbf{Return}] The minimum edit distance between the two strings.
\end{description}

\rule{\linewidth}{0.5pt}
\subsection*{FUNCTION : EditDistanceWithinRadius}
\hypertarget{ecldoc:str.editdistancewithinradius}{}
\hyperlink{ecldoc:Str}{Up} :
\hspace{0pt} \hyperlink{ecldoc:Str}{Str} \textbackslash 

{\renewcommand{\arraystretch}{1.5}
\begin{tabularx}{\textwidth}{|>{\raggedright\arraybackslash}l|X|}
\hline
\hspace{0pt}BOOLEAN & EditDistanceWithinRadius \\
\hline
\multicolumn{2}{|>{\raggedright\arraybackslash}X|}{\hspace{0pt}(STRING \_left, STRING \_right, UNSIGNED4 radius)} \\
\hline
\end{tabularx}
}

\par
Returns true if the minimum edit distance between the two strings is with a specific range. The two strings are trimmed before comparing.

\par
\begin{description}
\item [\textbf{Parameter}] \_left ||| The first string to be compared.
\item [\textbf{Parameter}] \_right ||| The second string to be compared.
\item [\textbf{Parameter}] radius ||| The maximum edit distance that is accepable.
\item [\textbf{Return}] Whether or not the two strings are within the given specified edit distance.
\end{description}

\rule{\linewidth}{0.5pt}
\subsection*{FUNCTION : WordCount}
\hypertarget{ecldoc:str.wordcount}{}
\hyperlink{ecldoc:Str}{Up} :
\hspace{0pt} \hyperlink{ecldoc:Str}{Str} \textbackslash 

{\renewcommand{\arraystretch}{1.5}
\begin{tabularx}{\textwidth}{|>{\raggedright\arraybackslash}l|X|}
\hline
\hspace{0pt}UNSIGNED4 & WordCount \\
\hline
\multicolumn{2}{|>{\raggedright\arraybackslash}X|}{\hspace{0pt}(STRING text)} \\
\hline
\end{tabularx}
}

\par
Returns the number of words in the string. Words are separated by one or more spaces.

\par
\begin{description}
\item [\textbf{Parameter}] text ||| The string to be broken into words.
\item [\textbf{Return}] The number of words in the string.
\end{description}

\rule{\linewidth}{0.5pt}
\subsection*{FUNCTION : GetNthWord}
\hypertarget{ecldoc:str.getnthword}{}
\hyperlink{ecldoc:Str}{Up} :
\hspace{0pt} \hyperlink{ecldoc:Str}{Str} \textbackslash 

{\renewcommand{\arraystretch}{1.5}
\begin{tabularx}{\textwidth}{|>{\raggedright\arraybackslash}l|X|}
\hline
\hspace{0pt}STRING & GetNthWord \\
\hline
\multicolumn{2}{|>{\raggedright\arraybackslash}X|}{\hspace{0pt}(STRING text, UNSIGNED4 n)} \\
\hline
\end{tabularx}
}

\par
Returns the n-th word from the string. Words are separated by one or more spaces.

\par
\begin{description}
\item [\textbf{Parameter}] text ||| The string to be broken into words.
\item [\textbf{Parameter}] n ||| Which word should be returned from the function.
\item [\textbf{Return}] The number of words in the string.
\end{description}

\rule{\linewidth}{0.5pt}
\subsection*{FUNCTION : ExcludeFirstWord}
\hypertarget{ecldoc:str.excludefirstword}{}
\hyperlink{ecldoc:Str}{Up} :
\hspace{0pt} \hyperlink{ecldoc:Str}{Str} \textbackslash 

{\renewcommand{\arraystretch}{1.5}
\begin{tabularx}{\textwidth}{|>{\raggedright\arraybackslash}l|X|}
\hline
\hspace{0pt} & ExcludeFirstWord \\
\hline
\multicolumn{2}{|>{\raggedright\arraybackslash}X|}{\hspace{0pt}(STRING text)} \\
\hline
\end{tabularx}
}

\par
Returns everything except the first word from the string. Words are separated by one or more whitespace characters. Whitespace before and after the first word is also removed.

\par
\begin{description}
\item [\textbf{Parameter}] text ||| The string to be broken into words.
\item [\textbf{Return}] The string excluding the first word.
\end{description}

\rule{\linewidth}{0.5pt}
\subsection*{FUNCTION : ExcludeLastWord}
\hypertarget{ecldoc:str.excludelastword}{}
\hyperlink{ecldoc:Str}{Up} :
\hspace{0pt} \hyperlink{ecldoc:Str}{Str} \textbackslash 

{\renewcommand{\arraystretch}{1.5}
\begin{tabularx}{\textwidth}{|>{\raggedright\arraybackslash}l|X|}
\hline
\hspace{0pt} & ExcludeLastWord \\
\hline
\multicolumn{2}{|>{\raggedright\arraybackslash}X|}{\hspace{0pt}(STRING text)} \\
\hline
\end{tabularx}
}

\par
Returns everything except the last word from the string. Words are separated by one or more whitespace characters. Whitespace after a word is removed with the word and leading whitespace is removed with the first word.

\par
\begin{description}
\item [\textbf{Parameter}] text ||| The string to be broken into words.
\item [\textbf{Return}] The string excluding the last word.
\end{description}

\rule{\linewidth}{0.5pt}
\subsection*{FUNCTION : ExcludeNthWord}
\hypertarget{ecldoc:str.excludenthword}{}
\hyperlink{ecldoc:Str}{Up} :
\hspace{0pt} \hyperlink{ecldoc:Str}{Str} \textbackslash 

{\renewcommand{\arraystretch}{1.5}
\begin{tabularx}{\textwidth}{|>{\raggedright\arraybackslash}l|X|}
\hline
\hspace{0pt} & ExcludeNthWord \\
\hline
\multicolumn{2}{|>{\raggedright\arraybackslash}X|}{\hspace{0pt}(STRING text, UNSIGNED2 n)} \\
\hline
\end{tabularx}
}

\par
Returns everything except the nth word from the string. Words are separated by one or more whitespace characters. Whitespace after a word is removed with the word and leading whitespace is removed with the first word.

\par
\begin{description}
\item [\textbf{Parameter}] text ||| The string to be broken into words.
\item [\textbf{Parameter}] n ||| Which word should be returned from the function.
\item [\textbf{Return}] The string excluding the nth word.
\end{description}

\rule{\linewidth}{0.5pt}
\subsection*{FUNCTION : FindWord}
\hypertarget{ecldoc:str.findword}{}
\hyperlink{ecldoc:Str}{Up} :
\hspace{0pt} \hyperlink{ecldoc:Str}{Str} \textbackslash 

{\renewcommand{\arraystretch}{1.5}
\begin{tabularx}{\textwidth}{|>{\raggedright\arraybackslash}l|X|}
\hline
\hspace{0pt}BOOLEAN & FindWord \\
\hline
\multicolumn{2}{|>{\raggedright\arraybackslash}X|}{\hspace{0pt}(STRING src, STRING word, BOOLEAN ignore\_case=FALSE)} \\
\hline
\end{tabularx}
}

\par
Tests if the search string contains the supplied word as a whole word.

\par
\begin{description}
\item [\textbf{Parameter}] src ||| The string that is being tested.
\item [\textbf{Parameter}] word ||| The word to be searched for.
\item [\textbf{Parameter}] ignore\_case ||| Whether to ignore differences in case between characters.
\end{description}

\rule{\linewidth}{0.5pt}
\subsection*{FUNCTION : Repeat}
\hypertarget{ecldoc:str.repeat}{}
\hyperlink{ecldoc:Str}{Up} :
\hspace{0pt} \hyperlink{ecldoc:Str}{Str} \textbackslash 

{\renewcommand{\arraystretch}{1.5}
\begin{tabularx}{\textwidth}{|>{\raggedright\arraybackslash}l|X|}
\hline
\hspace{0pt}STRING & Repeat \\
\hline
\multicolumn{2}{|>{\raggedright\arraybackslash}X|}{\hspace{0pt}(STRING text, UNSIGNED4 n)} \\
\hline
\end{tabularx}
}

\par


\rule{\linewidth}{0.5pt}
\subsection*{FUNCTION : ToHexPairs}
\hypertarget{ecldoc:str.tohexpairs}{}
\hyperlink{ecldoc:Str}{Up} :
\hspace{0pt} \hyperlink{ecldoc:Str}{Str} \textbackslash 

{\renewcommand{\arraystretch}{1.5}
\begin{tabularx}{\textwidth}{|>{\raggedright\arraybackslash}l|X|}
\hline
\hspace{0pt}STRING & ToHexPairs \\
\hline
\multicolumn{2}{|>{\raggedright\arraybackslash}X|}{\hspace{0pt}(DATA value)} \\
\hline
\end{tabularx}
}

\par


\rule{\linewidth}{0.5pt}
\subsection*{FUNCTION : FromHexPairs}
\hypertarget{ecldoc:str.fromhexpairs}{}
\hyperlink{ecldoc:Str}{Up} :
\hspace{0pt} \hyperlink{ecldoc:Str}{Str} \textbackslash 

{\renewcommand{\arraystretch}{1.5}
\begin{tabularx}{\textwidth}{|>{\raggedright\arraybackslash}l|X|}
\hline
\hspace{0pt}DATA & FromHexPairs \\
\hline
\multicolumn{2}{|>{\raggedright\arraybackslash}X|}{\hspace{0pt}(STRING hex\_pairs)} \\
\hline
\end{tabularx}
}

\par


\rule{\linewidth}{0.5pt}
\subsection*{FUNCTION : EncodeBase64}
\hypertarget{ecldoc:str.encodebase64}{}
\hyperlink{ecldoc:Str}{Up} :
\hspace{0pt} \hyperlink{ecldoc:Str}{Str} \textbackslash 

{\renewcommand{\arraystretch}{1.5}
\begin{tabularx}{\textwidth}{|>{\raggedright\arraybackslash}l|X|}
\hline
\hspace{0pt}STRING & EncodeBase64 \\
\hline
\multicolumn{2}{|>{\raggedright\arraybackslash}X|}{\hspace{0pt}(DATA value)} \\
\hline
\end{tabularx}
}

\par


\rule{\linewidth}{0.5pt}
\subsection*{FUNCTION : DecodeBase64}
\hypertarget{ecldoc:str.decodebase64}{}
\hyperlink{ecldoc:Str}{Up} :
\hspace{0pt} \hyperlink{ecldoc:Str}{Str} \textbackslash 

{\renewcommand{\arraystretch}{1.5}
\begin{tabularx}{\textwidth}{|>{\raggedright\arraybackslash}l|X|}
\hline
\hspace{0pt}DATA & DecodeBase64 \\
\hline
\multicolumn{2}{|>{\raggedright\arraybackslash}X|}{\hspace{0pt}(STRING value)} \\
\hline
\end{tabularx}
}

\par


\rule{\linewidth}{0.5pt}



\chapter*{\color{headfile}
Uni
}
\hypertarget{ecldoc:toc:Uni}{}
\hyperlink{ecldoc:toc:root}{Go Up}

\section*{\underline{\textsf{IMPORTS}}}
\begin{doublespace}
{\large
lib\_unicodelib |
}
\end{doublespace}

\section*{\underline{\textsf{DESCRIPTIONS}}}
\subsection*{\textsf{\colorbox{headtoc}{\color{white} MODULE}
Uni}}

\hypertarget{ecldoc:Uni}{}

{\renewcommand{\arraystretch}{1.5}
\begin{tabularx}{\textwidth}{|>{\raggedright\arraybackslash}l|X|}
\hline
\hspace{0pt}\mytexttt{\color{red} } & \textbf{Uni} \\
\hline
\end{tabularx}
}

\par





No Documentation Found







\textbf{Children}
\begin{enumerate}
\item \hyperlink{ecldoc:uni.filterout}{FilterOut}
: Returns the first string with all characters within the second string removed
\item \hyperlink{ecldoc:uni.filter}{Filter}
: Returns the first string with all characters not within the second string removed
\item \hyperlink{ecldoc:uni.substituteincluded}{SubstituteIncluded}
: Returns the source string with the replacement character substituted for all characters included in the filter string
\item \hyperlink{ecldoc:uni.substituteexcluded}{SubstituteExcluded}
: Returns the source string with the replacement character substituted for all characters not included in the filter string
\item \hyperlink{ecldoc:uni.find}{Find}
: Returns the character position of the nth match of the search string with the first string
\item \hyperlink{ecldoc:uni.findword}{FindWord}
: Tests if the search string contains the supplied word as a whole word
\item \hyperlink{ecldoc:uni.localefind}{LocaleFind}
: Returns the character position of the nth match of the search string with the first string
\item \hyperlink{ecldoc:uni.localefindatstrength}{LocaleFindAtStrength}
: Returns the character position of the nth match of the search string with the first string
\item \hyperlink{ecldoc:uni.extract}{Extract}
: Returns the nth element from a comma separated string
\item \hyperlink{ecldoc:uni.tolowercase}{ToLowerCase}
: Returns the argument string with all upper case characters converted to lower case
\item \hyperlink{ecldoc:uni.touppercase}{ToUpperCase}
: Return the argument string with all lower case characters converted to upper case
\item \hyperlink{ecldoc:uni.totitlecase}{ToTitleCase}
: Returns the upper case variant of the string using the rules for a particular locale
\item \hyperlink{ecldoc:uni.localetolowercase}{LocaleToLowerCase}
: Returns the lower case variant of the string using the rules for a particular locale
\item \hyperlink{ecldoc:uni.localetouppercase}{LocaleToUpperCase}
: Returns the upper case variant of the string using the rules for a particular locale
\item \hyperlink{ecldoc:uni.localetotitlecase}{LocaleToTitleCase}
: Returns the upper case variant of the string using the rules for a particular locale
\item \hyperlink{ecldoc:uni.compareignorecase}{CompareIgnoreCase}
: Compares the two strings case insensitively
\item \hyperlink{ecldoc:uni.compareatstrength}{CompareAtStrength}
: Compares the two strings case insensitively
\item \hyperlink{ecldoc:uni.localecompareignorecase}{LocaleCompareIgnoreCase}
: Compares the two strings case insensitively
\item \hyperlink{ecldoc:uni.localecompareatstrength}{LocaleCompareAtStrength}
: Compares the two strings case insensitively
\item \hyperlink{ecldoc:uni.reverse}{Reverse}
: Returns the argument string with all characters in reverse order
\item \hyperlink{ecldoc:uni.findreplace}{FindReplace}
: Returns the source string with the replacement string substituted for all instances of the search string
\item \hyperlink{ecldoc:uni.localefindreplace}{LocaleFindReplace}
: Returns the source string with the replacement string substituted for all instances of the search string
\item \hyperlink{ecldoc:uni.localefindatstrengthreplace}{LocaleFindAtStrengthReplace}
: Returns the source string with the replacement string substituted for all instances of the search string
\item \hyperlink{ecldoc:uni.cleanaccents}{CleanAccents}
: Returns the source string with all accented characters replaced with unaccented
\item \hyperlink{ecldoc:uni.cleanspaces}{CleanSpaces}
: Returns the source string with all instances of multiple adjacent space characters (2 or more spaces together) reduced to a single space character
\item \hyperlink{ecldoc:uni.wildmatch}{WildMatch}
: Tests if the search string matches the pattern
\item \hyperlink{ecldoc:uni.contains}{Contains}
: Tests if the search string contains each of the characters in the pattern
\item \hyperlink{ecldoc:uni.editdistance}{EditDistance}
: Returns the minimum edit distance between the two strings
\item \hyperlink{ecldoc:uni.editdistancewithinradius}{EditDistanceWithinRadius}
: Returns true if the minimum edit distance between the two strings is with a specific range
\item \hyperlink{ecldoc:uni.wordcount}{WordCount}
: Returns the number of words in the string
\item \hyperlink{ecldoc:uni.getnthword}{GetNthWord}
: Returns the n-th word from the string
\end{enumerate}

\rule{\linewidth}{0.5pt}

\subsection*{\textsf{\colorbox{headtoc}{\color{white} FUNCTION}
FilterOut}}

\hypertarget{ecldoc:uni.filterout}{}
\hspace{0pt} \hyperlink{ecldoc:Uni}{Uni} \textbackslash 

{\renewcommand{\arraystretch}{1.5}
\begin{tabularx}{\textwidth}{|>{\raggedright\arraybackslash}l|X|}
\hline
\hspace{0pt}\mytexttt{\color{red} unicode} & \textbf{FilterOut} \\
\hline
\multicolumn{2}{|>{\raggedright\arraybackslash}X|}{\hspace{0pt}\mytexttt{\color{param} (unicode src, unicode filter)}} \\
\hline
\end{tabularx}
}

\par





Returns the first string with all characters within the second string removed.






\par
\begin{description}
\item [\colorbox{tagtype}{\color{white} \textbf{\textsf{PARAMETER}}}] \textbf{\underline{src}} ||| UNICODE --- The string that is being tested.
\item [\colorbox{tagtype}{\color{white} \textbf{\textsf{PARAMETER}}}] \textbf{\underline{filter}} ||| UNICODE --- The string containing the set of characters to be excluded.
\end{description}







\par
\begin{description}
\item [\colorbox{tagtype}{\color{white} \textbf{\textsf{RETURN}}}] \textbf{UNICODE} --- 
\end{description}






\par
\begin{description}
\item [\colorbox{tagtype}{\color{white} \textbf{\textsf{SEE}}}] Std.Uni.Filter
\end{description}




\rule{\linewidth}{0.5pt}
\subsection*{\textsf{\colorbox{headtoc}{\color{white} FUNCTION}
Filter}}

\hypertarget{ecldoc:uni.filter}{}
\hspace{0pt} \hyperlink{ecldoc:Uni}{Uni} \textbackslash 

{\renewcommand{\arraystretch}{1.5}
\begin{tabularx}{\textwidth}{|>{\raggedright\arraybackslash}l|X|}
\hline
\hspace{0pt}\mytexttt{\color{red} unicode} & \textbf{Filter} \\
\hline
\multicolumn{2}{|>{\raggedright\arraybackslash}X|}{\hspace{0pt}\mytexttt{\color{param} (unicode src, unicode filter)}} \\
\hline
\end{tabularx}
}

\par





Returns the first string with all characters not within the second string removed.






\par
\begin{description}
\item [\colorbox{tagtype}{\color{white} \textbf{\textsf{PARAMETER}}}] \textbf{\underline{src}} ||| UNICODE --- The string that is being tested.
\item [\colorbox{tagtype}{\color{white} \textbf{\textsf{PARAMETER}}}] \textbf{\underline{filter}} ||| UNICODE --- The string containing the set of characters to be included.
\end{description}







\par
\begin{description}
\item [\colorbox{tagtype}{\color{white} \textbf{\textsf{RETURN}}}] \textbf{UNICODE} --- 
\end{description}






\par
\begin{description}
\item [\colorbox{tagtype}{\color{white} \textbf{\textsf{SEE}}}] Std.Uni.FilterOut
\end{description}




\rule{\linewidth}{0.5pt}
\subsection*{\textsf{\colorbox{headtoc}{\color{white} FUNCTION}
SubstituteIncluded}}

\hypertarget{ecldoc:uni.substituteincluded}{}
\hspace{0pt} \hyperlink{ecldoc:Uni}{Uni} \textbackslash 

{\renewcommand{\arraystretch}{1.5}
\begin{tabularx}{\textwidth}{|>{\raggedright\arraybackslash}l|X|}
\hline
\hspace{0pt}\mytexttt{\color{red} unicode} & \textbf{SubstituteIncluded} \\
\hline
\multicolumn{2}{|>{\raggedright\arraybackslash}X|}{\hspace{0pt}\mytexttt{\color{param} (unicode src, unicode filter, unicode replace\_char)}} \\
\hline
\end{tabularx}
}

\par





Returns the source string with the replacement character substituted for all characters included in the filter string. MORE: Should this be a general string substitution?






\par
\begin{description}
\item [\colorbox{tagtype}{\color{white} \textbf{\textsf{PARAMETER}}}] \textbf{\underline{replace\_char}} ||| UNICODE --- The character to be substituted into the result.
\item [\colorbox{tagtype}{\color{white} \textbf{\textsf{PARAMETER}}}] \textbf{\underline{src}} ||| UNICODE --- The string that is being tested.
\item [\colorbox{tagtype}{\color{white} \textbf{\textsf{PARAMETER}}}] \textbf{\underline{filter}} ||| UNICODE --- The string containing the set of characters to be included.
\end{description}







\par
\begin{description}
\item [\colorbox{tagtype}{\color{white} \textbf{\textsf{RETURN}}}] \textbf{UNICODE} --- 
\end{description}






\par
\begin{description}
\item [\colorbox{tagtype}{\color{white} \textbf{\textsf{SEE}}}] Std.Uni.SubstituteOut
\end{description}




\rule{\linewidth}{0.5pt}
\subsection*{\textsf{\colorbox{headtoc}{\color{white} FUNCTION}
SubstituteExcluded}}

\hypertarget{ecldoc:uni.substituteexcluded}{}
\hspace{0pt} \hyperlink{ecldoc:Uni}{Uni} \textbackslash 

{\renewcommand{\arraystretch}{1.5}
\begin{tabularx}{\textwidth}{|>{\raggedright\arraybackslash}l|X|}
\hline
\hspace{0pt}\mytexttt{\color{red} unicode} & \textbf{SubstituteExcluded} \\
\hline
\multicolumn{2}{|>{\raggedright\arraybackslash}X|}{\hspace{0pt}\mytexttt{\color{param} (unicode src, unicode filter, unicode replace\_char)}} \\
\hline
\end{tabularx}
}

\par





Returns the source string with the replacement character substituted for all characters not included in the filter string. MORE: Should this be a general string substitution?






\par
\begin{description}
\item [\colorbox{tagtype}{\color{white} \textbf{\textsf{PARAMETER}}}] \textbf{\underline{replace\_char}} ||| UNICODE --- The character to be substituted into the result.
\item [\colorbox{tagtype}{\color{white} \textbf{\textsf{PARAMETER}}}] \textbf{\underline{src}} ||| UNICODE --- The string that is being tested.
\item [\colorbox{tagtype}{\color{white} \textbf{\textsf{PARAMETER}}}] \textbf{\underline{filter}} ||| UNICODE --- The string containing the set of characters to be included.
\end{description}







\par
\begin{description}
\item [\colorbox{tagtype}{\color{white} \textbf{\textsf{RETURN}}}] \textbf{UNICODE} --- 
\end{description}






\par
\begin{description}
\item [\colorbox{tagtype}{\color{white} \textbf{\textsf{SEE}}}] Std.Uni.SubstituteIncluded
\end{description}




\rule{\linewidth}{0.5pt}
\subsection*{\textsf{\colorbox{headtoc}{\color{white} FUNCTION}
Find}}

\hypertarget{ecldoc:uni.find}{}
\hspace{0pt} \hyperlink{ecldoc:Uni}{Uni} \textbackslash 

{\renewcommand{\arraystretch}{1.5}
\begin{tabularx}{\textwidth}{|>{\raggedright\arraybackslash}l|X|}
\hline
\hspace{0pt}\mytexttt{\color{red} UNSIGNED4} & \textbf{Find} \\
\hline
\multicolumn{2}{|>{\raggedright\arraybackslash}X|}{\hspace{0pt}\mytexttt{\color{param} (unicode src, unicode sought, unsigned4 instance)}} \\
\hline
\end{tabularx}
}

\par





Returns the character position of the nth match of the search string with the first string. If no match is found the attribute returns 0. If an instance is omitted the position of the first instance is returned.






\par
\begin{description}
\item [\colorbox{tagtype}{\color{white} \textbf{\textsf{PARAMETER}}}] \textbf{\underline{instance}} ||| UNSIGNED4 --- Which match instance are we interested in?
\item [\colorbox{tagtype}{\color{white} \textbf{\textsf{PARAMETER}}}] \textbf{\underline{src}} ||| UNICODE --- The string that is searched
\item [\colorbox{tagtype}{\color{white} \textbf{\textsf{PARAMETER}}}] \textbf{\underline{sought}} ||| UNICODE --- The string being sought.
\end{description}







\par
\begin{description}
\item [\colorbox{tagtype}{\color{white} \textbf{\textsf{RETURN}}}] \textbf{UNSIGNED4} --- 
\end{description}




\rule{\linewidth}{0.5pt}
\subsection*{\textsf{\colorbox{headtoc}{\color{white} FUNCTION}
FindWord}}

\hypertarget{ecldoc:uni.findword}{}
\hspace{0pt} \hyperlink{ecldoc:Uni}{Uni} \textbackslash 

{\renewcommand{\arraystretch}{1.5}
\begin{tabularx}{\textwidth}{|>{\raggedright\arraybackslash}l|X|}
\hline
\hspace{0pt}\mytexttt{\color{red} BOOLEAN} & \textbf{FindWord} \\
\hline
\multicolumn{2}{|>{\raggedright\arraybackslash}X|}{\hspace{0pt}\mytexttt{\color{param} (UNICODE src, UNICODE word, BOOLEAN ignore\_case=FALSE)}} \\
\hline
\end{tabularx}
}

\par





Tests if the search string contains the supplied word as a whole word.






\par
\begin{description}
\item [\colorbox{tagtype}{\color{white} \textbf{\textsf{PARAMETER}}}] \textbf{\underline{word}} ||| UNICODE --- The word to be searched for.
\item [\colorbox{tagtype}{\color{white} \textbf{\textsf{PARAMETER}}}] \textbf{\underline{src}} ||| UNICODE --- The string that is being tested.
\item [\colorbox{tagtype}{\color{white} \textbf{\textsf{PARAMETER}}}] \textbf{\underline{ignore\_case}} ||| BOOLEAN --- Whether to ignore differences in case between characters.
\end{description}







\par
\begin{description}
\item [\colorbox{tagtype}{\color{white} \textbf{\textsf{RETURN}}}] \textbf{BOOLEAN} --- 
\end{description}




\rule{\linewidth}{0.5pt}
\subsection*{\textsf{\colorbox{headtoc}{\color{white} FUNCTION}
LocaleFind}}

\hypertarget{ecldoc:uni.localefind}{}
\hspace{0pt} \hyperlink{ecldoc:Uni}{Uni} \textbackslash 

{\renewcommand{\arraystretch}{1.5}
\begin{tabularx}{\textwidth}{|>{\raggedright\arraybackslash}l|X|}
\hline
\hspace{0pt}\mytexttt{\color{red} UNSIGNED4} & \textbf{LocaleFind} \\
\hline
\multicolumn{2}{|>{\raggedright\arraybackslash}X|}{\hspace{0pt}\mytexttt{\color{param} (unicode src, unicode sought, unsigned4 instance, varstring locale\_name)}} \\
\hline
\end{tabularx}
}

\par





Returns the character position of the nth match of the search string with the first string. If no match is found the attribute returns 0. If an instance is omitted the position of the first instance is returned.






\par
\begin{description}
\item [\colorbox{tagtype}{\color{white} \textbf{\textsf{PARAMETER}}}] \textbf{\underline{instance}} ||| UNSIGNED4 --- Which match instance are we interested in?
\item [\colorbox{tagtype}{\color{white} \textbf{\textsf{PARAMETER}}}] \textbf{\underline{src}} ||| UNICODE --- The string that is searched
\item [\colorbox{tagtype}{\color{white} \textbf{\textsf{PARAMETER}}}] \textbf{\underline{sought}} ||| UNICODE --- The string being sought.
\item [\colorbox{tagtype}{\color{white} \textbf{\textsf{PARAMETER}}}] \textbf{\underline{locale\_name}} ||| VARSTRING --- The locale to use for the comparison
\end{description}







\par
\begin{description}
\item [\colorbox{tagtype}{\color{white} \textbf{\textsf{RETURN}}}] \textbf{UNSIGNED4} --- 
\end{description}




\rule{\linewidth}{0.5pt}
\subsection*{\textsf{\colorbox{headtoc}{\color{white} FUNCTION}
LocaleFindAtStrength}}

\hypertarget{ecldoc:uni.localefindatstrength}{}
\hspace{0pt} \hyperlink{ecldoc:Uni}{Uni} \textbackslash 

{\renewcommand{\arraystretch}{1.5}
\begin{tabularx}{\textwidth}{|>{\raggedright\arraybackslash}l|X|}
\hline
\hspace{0pt}\mytexttt{\color{red} UNSIGNED4} & \textbf{LocaleFindAtStrength} \\
\hline
\multicolumn{2}{|>{\raggedright\arraybackslash}X|}{\hspace{0pt}\mytexttt{\color{param} (unicode src, unicode tofind, unsigned4 instance, varstring locale\_name, integer1 strength)}} \\
\hline
\end{tabularx}
}

\par





Returns the character position of the nth match of the search string with the first string. If no match is found the attribute returns 0. If an instance is omitted the position of the first instance is returned.






\par
\begin{description}
\item [\colorbox{tagtype}{\color{white} \textbf{\textsf{PARAMETER}}}] \textbf{\underline{instance}} ||| UNSIGNED4 --- Which match instance are we interested in?
\item [\colorbox{tagtype}{\color{white} \textbf{\textsf{PARAMETER}}}] \textbf{\underline{strength}} ||| INTEGER1 --- The strength of the comparison 1 ignores accents and case, differentiating only between letters 2 ignores case but differentiates between accents. 3 differentiates between accents and case but ignores e.g. differences between Hiragana and Katakana 4 differentiates between accents and case and e.g. Hiragana/Katakana, but ignores e.g. Hebrew cantellation marks 5 differentiates between all strings whose canonically decomposed forms (NFDNormalization Form D) are non-identical
\item [\colorbox{tagtype}{\color{white} \textbf{\textsf{PARAMETER}}}] \textbf{\underline{src}} ||| UNICODE --- The string that is searched
\item [\colorbox{tagtype}{\color{white} \textbf{\textsf{PARAMETER}}}] \textbf{\underline{sought}} |||  --- The string being sought.
\item [\colorbox{tagtype}{\color{white} \textbf{\textsf{PARAMETER}}}] \textbf{\underline{locale\_name}} ||| VARSTRING --- The locale to use for the comparison
\item [\colorbox{tagtype}{\color{white} \textbf{\textsf{PARAMETER}}}] \textbf{\underline{tofind}} ||| UNICODE --- No Doc
\end{description}







\par
\begin{description}
\item [\colorbox{tagtype}{\color{white} \textbf{\textsf{RETURN}}}] \textbf{UNSIGNED4} --- 
\end{description}




\rule{\linewidth}{0.5pt}
\subsection*{\textsf{\colorbox{headtoc}{\color{white} FUNCTION}
Extract}}

\hypertarget{ecldoc:uni.extract}{}
\hspace{0pt} \hyperlink{ecldoc:Uni}{Uni} \textbackslash 

{\renewcommand{\arraystretch}{1.5}
\begin{tabularx}{\textwidth}{|>{\raggedright\arraybackslash}l|X|}
\hline
\hspace{0pt}\mytexttt{\color{red} unicode} & \textbf{Extract} \\
\hline
\multicolumn{2}{|>{\raggedright\arraybackslash}X|}{\hspace{0pt}\mytexttt{\color{param} (unicode src, unsigned4 instance)}} \\
\hline
\end{tabularx}
}

\par





Returns the nth element from a comma separated string.






\par
\begin{description}
\item [\colorbox{tagtype}{\color{white} \textbf{\textsf{PARAMETER}}}] \textbf{\underline{instance}} ||| UNSIGNED4 --- Which item to select from the list.
\item [\colorbox{tagtype}{\color{white} \textbf{\textsf{PARAMETER}}}] \textbf{\underline{src}} ||| UNICODE --- The string containing the comma separated list.
\end{description}







\par
\begin{description}
\item [\colorbox{tagtype}{\color{white} \textbf{\textsf{RETURN}}}] \textbf{UNICODE} --- 
\end{description}




\rule{\linewidth}{0.5pt}
\subsection*{\textsf{\colorbox{headtoc}{\color{white} FUNCTION}
ToLowerCase}}

\hypertarget{ecldoc:uni.tolowercase}{}
\hspace{0pt} \hyperlink{ecldoc:Uni}{Uni} \textbackslash 

{\renewcommand{\arraystretch}{1.5}
\begin{tabularx}{\textwidth}{|>{\raggedright\arraybackslash}l|X|}
\hline
\hspace{0pt}\mytexttt{\color{red} unicode} & \textbf{ToLowerCase} \\
\hline
\multicolumn{2}{|>{\raggedright\arraybackslash}X|}{\hspace{0pt}\mytexttt{\color{param} (unicode src)}} \\
\hline
\end{tabularx}
}

\par





Returns the argument string with all upper case characters converted to lower case.






\par
\begin{description}
\item [\colorbox{tagtype}{\color{white} \textbf{\textsf{PARAMETER}}}] \textbf{\underline{src}} ||| UNICODE --- The string that is being converted.
\end{description}







\par
\begin{description}
\item [\colorbox{tagtype}{\color{white} \textbf{\textsf{RETURN}}}] \textbf{UNICODE} --- 
\end{description}




\rule{\linewidth}{0.5pt}
\subsection*{\textsf{\colorbox{headtoc}{\color{white} FUNCTION}
ToUpperCase}}

\hypertarget{ecldoc:uni.touppercase}{}
\hspace{0pt} \hyperlink{ecldoc:Uni}{Uni} \textbackslash 

{\renewcommand{\arraystretch}{1.5}
\begin{tabularx}{\textwidth}{|>{\raggedright\arraybackslash}l|X|}
\hline
\hspace{0pt}\mytexttt{\color{red} unicode} & \textbf{ToUpperCase} \\
\hline
\multicolumn{2}{|>{\raggedright\arraybackslash}X|}{\hspace{0pt}\mytexttt{\color{param} (unicode src)}} \\
\hline
\end{tabularx}
}

\par





Return the argument string with all lower case characters converted to upper case.






\par
\begin{description}
\item [\colorbox{tagtype}{\color{white} \textbf{\textsf{PARAMETER}}}] \textbf{\underline{src}} ||| UNICODE --- The string that is being converted.
\end{description}







\par
\begin{description}
\item [\colorbox{tagtype}{\color{white} \textbf{\textsf{RETURN}}}] \textbf{UNICODE} --- 
\end{description}




\rule{\linewidth}{0.5pt}
\subsection*{\textsf{\colorbox{headtoc}{\color{white} FUNCTION}
ToTitleCase}}

\hypertarget{ecldoc:uni.totitlecase}{}
\hspace{0pt} \hyperlink{ecldoc:Uni}{Uni} \textbackslash 

{\renewcommand{\arraystretch}{1.5}
\begin{tabularx}{\textwidth}{|>{\raggedright\arraybackslash}l|X|}
\hline
\hspace{0pt}\mytexttt{\color{red} unicode} & \textbf{ToTitleCase} \\
\hline
\multicolumn{2}{|>{\raggedright\arraybackslash}X|}{\hspace{0pt}\mytexttt{\color{param} (unicode src)}} \\
\hline
\end{tabularx}
}

\par





Returns the upper case variant of the string using the rules for a particular locale.






\par
\begin{description}
\item [\colorbox{tagtype}{\color{white} \textbf{\textsf{PARAMETER}}}] \textbf{\underline{src}} ||| UNICODE --- The string that is being converted.
\item [\colorbox{tagtype}{\color{white} \textbf{\textsf{PARAMETER}}}] \textbf{\underline{locale\_name}} |||  --- The locale to use for the comparison
\end{description}







\par
\begin{description}
\item [\colorbox{tagtype}{\color{white} \textbf{\textsf{RETURN}}}] \textbf{UNICODE} --- 
\end{description}




\rule{\linewidth}{0.5pt}
\subsection*{\textsf{\colorbox{headtoc}{\color{white} FUNCTION}
LocaleToLowerCase}}

\hypertarget{ecldoc:uni.localetolowercase}{}
\hspace{0pt} \hyperlink{ecldoc:Uni}{Uni} \textbackslash 

{\renewcommand{\arraystretch}{1.5}
\begin{tabularx}{\textwidth}{|>{\raggedright\arraybackslash}l|X|}
\hline
\hspace{0pt}\mytexttt{\color{red} unicode} & \textbf{LocaleToLowerCase} \\
\hline
\multicolumn{2}{|>{\raggedright\arraybackslash}X|}{\hspace{0pt}\mytexttt{\color{param} (unicode src, varstring locale\_name)}} \\
\hline
\end{tabularx}
}

\par





Returns the lower case variant of the string using the rules for a particular locale.






\par
\begin{description}
\item [\colorbox{tagtype}{\color{white} \textbf{\textsf{PARAMETER}}}] \textbf{\underline{src}} ||| UNICODE --- The string that is being converted.
\item [\colorbox{tagtype}{\color{white} \textbf{\textsf{PARAMETER}}}] \textbf{\underline{locale\_name}} ||| VARSTRING --- The locale to use for the comparison
\end{description}







\par
\begin{description}
\item [\colorbox{tagtype}{\color{white} \textbf{\textsf{RETURN}}}] \textbf{UNICODE} --- 
\end{description}




\rule{\linewidth}{0.5pt}
\subsection*{\textsf{\colorbox{headtoc}{\color{white} FUNCTION}
LocaleToUpperCase}}

\hypertarget{ecldoc:uni.localetouppercase}{}
\hspace{0pt} \hyperlink{ecldoc:Uni}{Uni} \textbackslash 

{\renewcommand{\arraystretch}{1.5}
\begin{tabularx}{\textwidth}{|>{\raggedright\arraybackslash}l|X|}
\hline
\hspace{0pt}\mytexttt{\color{red} unicode} & \textbf{LocaleToUpperCase} \\
\hline
\multicolumn{2}{|>{\raggedright\arraybackslash}X|}{\hspace{0pt}\mytexttt{\color{param} (unicode src, varstring locale\_name)}} \\
\hline
\end{tabularx}
}

\par





Returns the upper case variant of the string using the rules for a particular locale.






\par
\begin{description}
\item [\colorbox{tagtype}{\color{white} \textbf{\textsf{PARAMETER}}}] \textbf{\underline{src}} ||| UNICODE --- The string that is being converted.
\item [\colorbox{tagtype}{\color{white} \textbf{\textsf{PARAMETER}}}] \textbf{\underline{locale\_name}} ||| VARSTRING --- The locale to use for the comparison
\end{description}







\par
\begin{description}
\item [\colorbox{tagtype}{\color{white} \textbf{\textsf{RETURN}}}] \textbf{UNICODE} --- 
\end{description}




\rule{\linewidth}{0.5pt}
\subsection*{\textsf{\colorbox{headtoc}{\color{white} FUNCTION}
LocaleToTitleCase}}

\hypertarget{ecldoc:uni.localetotitlecase}{}
\hspace{0pt} \hyperlink{ecldoc:Uni}{Uni} \textbackslash 

{\renewcommand{\arraystretch}{1.5}
\begin{tabularx}{\textwidth}{|>{\raggedright\arraybackslash}l|X|}
\hline
\hspace{0pt}\mytexttt{\color{red} unicode} & \textbf{LocaleToTitleCase} \\
\hline
\multicolumn{2}{|>{\raggedright\arraybackslash}X|}{\hspace{0pt}\mytexttt{\color{param} (unicode src, varstring locale\_name)}} \\
\hline
\end{tabularx}
}

\par





Returns the upper case variant of the string using the rules for a particular locale.






\par
\begin{description}
\item [\colorbox{tagtype}{\color{white} \textbf{\textsf{PARAMETER}}}] \textbf{\underline{src}} ||| UNICODE --- The string that is being converted.
\item [\colorbox{tagtype}{\color{white} \textbf{\textsf{PARAMETER}}}] \textbf{\underline{locale\_name}} ||| VARSTRING --- The locale to use for the comparison
\end{description}







\par
\begin{description}
\item [\colorbox{tagtype}{\color{white} \textbf{\textsf{RETURN}}}] \textbf{UNICODE} --- 
\end{description}




\rule{\linewidth}{0.5pt}
\subsection*{\textsf{\colorbox{headtoc}{\color{white} FUNCTION}
CompareIgnoreCase}}

\hypertarget{ecldoc:uni.compareignorecase}{}
\hspace{0pt} \hyperlink{ecldoc:Uni}{Uni} \textbackslash 

{\renewcommand{\arraystretch}{1.5}
\begin{tabularx}{\textwidth}{|>{\raggedright\arraybackslash}l|X|}
\hline
\hspace{0pt}\mytexttt{\color{red} integer4} & \textbf{CompareIgnoreCase} \\
\hline
\multicolumn{2}{|>{\raggedright\arraybackslash}X|}{\hspace{0pt}\mytexttt{\color{param} (unicode src1, unicode src2)}} \\
\hline
\end{tabularx}
}

\par





Compares the two strings case insensitively. Equivalent to comparing at strength 2.






\par
\begin{description}
\item [\colorbox{tagtype}{\color{white} \textbf{\textsf{PARAMETER}}}] \textbf{\underline{src2}} ||| UNICODE --- The second string to be compared.
\item [\colorbox{tagtype}{\color{white} \textbf{\textsf{PARAMETER}}}] \textbf{\underline{src1}} ||| UNICODE --- The first string to be compared.
\end{description}







\par
\begin{description}
\item [\colorbox{tagtype}{\color{white} \textbf{\textsf{RETURN}}}] \textbf{INTEGER4} --- 
\end{description}






\par
\begin{description}
\item [\colorbox{tagtype}{\color{white} \textbf{\textsf{SEE}}}] Std.Uni.CompareAtStrength
\end{description}




\rule{\linewidth}{0.5pt}
\subsection*{\textsf{\colorbox{headtoc}{\color{white} FUNCTION}
CompareAtStrength}}

\hypertarget{ecldoc:uni.compareatstrength}{}
\hspace{0pt} \hyperlink{ecldoc:Uni}{Uni} \textbackslash 

{\renewcommand{\arraystretch}{1.5}
\begin{tabularx}{\textwidth}{|>{\raggedright\arraybackslash}l|X|}
\hline
\hspace{0pt}\mytexttt{\color{red} integer4} & \textbf{CompareAtStrength} \\
\hline
\multicolumn{2}{|>{\raggedright\arraybackslash}X|}{\hspace{0pt}\mytexttt{\color{param} (unicode src1, unicode src2, integer1 strength)}} \\
\hline
\end{tabularx}
}

\par





Compares the two strings case insensitively. Equivalent to comparing at strength 2.






\par
\begin{description}
\item [\colorbox{tagtype}{\color{white} \textbf{\textsf{PARAMETER}}}] \textbf{\underline{src2}} ||| UNICODE --- The second string to be compared.
\item [\colorbox{tagtype}{\color{white} \textbf{\textsf{PARAMETER}}}] \textbf{\underline{strength}} ||| INTEGER1 --- The strength of the comparison 1 ignores accents and case, differentiating only between letters 2 ignores case but differentiates between accents. 3 differentiates between accents and case but ignores e.g. differences between Hiragana and Katakana 4 differentiates between accents and case and e.g. Hiragana/Katakana, but ignores e.g. Hebrew cantellation marks 5 differentiates between all strings whose canonically decomposed forms (NFDNormalization Form D) are non-identical
\item [\colorbox{tagtype}{\color{white} \textbf{\textsf{PARAMETER}}}] \textbf{\underline{src1}} ||| UNICODE --- The first string to be compared.
\end{description}







\par
\begin{description}
\item [\colorbox{tagtype}{\color{white} \textbf{\textsf{RETURN}}}] \textbf{INTEGER4} --- 
\end{description}






\par
\begin{description}
\item [\colorbox{tagtype}{\color{white} \textbf{\textsf{SEE}}}] Std.Uni.CompareAtStrength
\end{description}




\rule{\linewidth}{0.5pt}
\subsection*{\textsf{\colorbox{headtoc}{\color{white} FUNCTION}
LocaleCompareIgnoreCase}}

\hypertarget{ecldoc:uni.localecompareignorecase}{}
\hspace{0pt} \hyperlink{ecldoc:Uni}{Uni} \textbackslash 

{\renewcommand{\arraystretch}{1.5}
\begin{tabularx}{\textwidth}{|>{\raggedright\arraybackslash}l|X|}
\hline
\hspace{0pt}\mytexttt{\color{red} integer4} & \textbf{LocaleCompareIgnoreCase} \\
\hline
\multicolumn{2}{|>{\raggedright\arraybackslash}X|}{\hspace{0pt}\mytexttt{\color{param} (unicode src1, unicode src2, varstring locale\_name)}} \\
\hline
\end{tabularx}
}

\par





Compares the two strings case insensitively. Equivalent to comparing at strength 2.






\par
\begin{description}
\item [\colorbox{tagtype}{\color{white} \textbf{\textsf{PARAMETER}}}] \textbf{\underline{src2}} ||| UNICODE --- The second string to be compared.
\item [\colorbox{tagtype}{\color{white} \textbf{\textsf{PARAMETER}}}] \textbf{\underline{src1}} ||| UNICODE --- The first string to be compared.
\item [\colorbox{tagtype}{\color{white} \textbf{\textsf{PARAMETER}}}] \textbf{\underline{locale\_name}} ||| VARSTRING --- The locale to use for the comparison
\end{description}







\par
\begin{description}
\item [\colorbox{tagtype}{\color{white} \textbf{\textsf{RETURN}}}] \textbf{INTEGER4} --- 
\end{description}






\par
\begin{description}
\item [\colorbox{tagtype}{\color{white} \textbf{\textsf{SEE}}}] Std.Uni.CompareAtStrength
\end{description}




\rule{\linewidth}{0.5pt}
\subsection*{\textsf{\colorbox{headtoc}{\color{white} FUNCTION}
LocaleCompareAtStrength}}

\hypertarget{ecldoc:uni.localecompareatstrength}{}
\hspace{0pt} \hyperlink{ecldoc:Uni}{Uni} \textbackslash 

{\renewcommand{\arraystretch}{1.5}
\begin{tabularx}{\textwidth}{|>{\raggedright\arraybackslash}l|X|}
\hline
\hspace{0pt}\mytexttt{\color{red} integer4} & \textbf{LocaleCompareAtStrength} \\
\hline
\multicolumn{2}{|>{\raggedright\arraybackslash}X|}{\hspace{0pt}\mytexttt{\color{param} (unicode src1, unicode src2, varstring locale\_name, integer1 strength)}} \\
\hline
\end{tabularx}
}

\par





Compares the two strings case insensitively. Equivalent to comparing at strength 2.






\par
\begin{description}
\item [\colorbox{tagtype}{\color{white} \textbf{\textsf{PARAMETER}}}] \textbf{\underline{src2}} ||| UNICODE --- The second string to be compared.
\item [\colorbox{tagtype}{\color{white} \textbf{\textsf{PARAMETER}}}] \textbf{\underline{strength}} ||| INTEGER1 --- The strength of the comparison 1 ignores accents and case, differentiating only between letters 2 ignores case but differentiates between accents. 3 differentiates between accents and case but ignores e.g. differences between Hiragana and Katakana 4 differentiates between accents and case and e.g. Hiragana/Katakana, but ignores e.g. Hebrew cantellation marks 5 differentiates between all strings whose canonically decomposed forms (NFDNormalization Form D) are non-identical
\item [\colorbox{tagtype}{\color{white} \textbf{\textsf{PARAMETER}}}] \textbf{\underline{src1}} ||| UNICODE --- The first string to be compared.
\item [\colorbox{tagtype}{\color{white} \textbf{\textsf{PARAMETER}}}] \textbf{\underline{locale\_name}} ||| VARSTRING --- The locale to use for the comparison
\end{description}







\par
\begin{description}
\item [\colorbox{tagtype}{\color{white} \textbf{\textsf{RETURN}}}] \textbf{INTEGER4} --- 
\end{description}




\rule{\linewidth}{0.5pt}
\subsection*{\textsf{\colorbox{headtoc}{\color{white} FUNCTION}
Reverse}}

\hypertarget{ecldoc:uni.reverse}{}
\hspace{0pt} \hyperlink{ecldoc:Uni}{Uni} \textbackslash 

{\renewcommand{\arraystretch}{1.5}
\begin{tabularx}{\textwidth}{|>{\raggedright\arraybackslash}l|X|}
\hline
\hspace{0pt}\mytexttt{\color{red} unicode} & \textbf{Reverse} \\
\hline
\multicolumn{2}{|>{\raggedright\arraybackslash}X|}{\hspace{0pt}\mytexttt{\color{param} (unicode src)}} \\
\hline
\end{tabularx}
}

\par





Returns the argument string with all characters in reverse order. Note the argument is not TRIMMED before it is reversed.






\par
\begin{description}
\item [\colorbox{tagtype}{\color{white} \textbf{\textsf{PARAMETER}}}] \textbf{\underline{src}} ||| UNICODE --- The string that is being reversed.
\end{description}







\par
\begin{description}
\item [\colorbox{tagtype}{\color{white} \textbf{\textsf{RETURN}}}] \textbf{UNICODE} --- 
\end{description}




\rule{\linewidth}{0.5pt}
\subsection*{\textsf{\colorbox{headtoc}{\color{white} FUNCTION}
FindReplace}}

\hypertarget{ecldoc:uni.findreplace}{}
\hspace{0pt} \hyperlink{ecldoc:Uni}{Uni} \textbackslash 

{\renewcommand{\arraystretch}{1.5}
\begin{tabularx}{\textwidth}{|>{\raggedright\arraybackslash}l|X|}
\hline
\hspace{0pt}\mytexttt{\color{red} unicode} & \textbf{FindReplace} \\
\hline
\multicolumn{2}{|>{\raggedright\arraybackslash}X|}{\hspace{0pt}\mytexttt{\color{param} (unicode src, unicode sought, unicode replacement)}} \\
\hline
\end{tabularx}
}

\par





Returns the source string with the replacement string substituted for all instances of the search string.






\par
\begin{description}
\item [\colorbox{tagtype}{\color{white} \textbf{\textsf{PARAMETER}}}] \textbf{\underline{src}} ||| UNICODE --- The string that is being transformed.
\item [\colorbox{tagtype}{\color{white} \textbf{\textsf{PARAMETER}}}] \textbf{\underline{replacement}} ||| UNICODE --- The string to be substituted into the result.
\item [\colorbox{tagtype}{\color{white} \textbf{\textsf{PARAMETER}}}] \textbf{\underline{sought}} ||| UNICODE --- The string to be replaced.
\end{description}







\par
\begin{description}
\item [\colorbox{tagtype}{\color{white} \textbf{\textsf{RETURN}}}] \textbf{UNICODE} --- 
\end{description}




\rule{\linewidth}{0.5pt}
\subsection*{\textsf{\colorbox{headtoc}{\color{white} FUNCTION}
LocaleFindReplace}}

\hypertarget{ecldoc:uni.localefindreplace}{}
\hspace{0pt} \hyperlink{ecldoc:Uni}{Uni} \textbackslash 

{\renewcommand{\arraystretch}{1.5}
\begin{tabularx}{\textwidth}{|>{\raggedright\arraybackslash}l|X|}
\hline
\hspace{0pt}\mytexttt{\color{red} unicode} & \textbf{LocaleFindReplace} \\
\hline
\multicolumn{2}{|>{\raggedright\arraybackslash}X|}{\hspace{0pt}\mytexttt{\color{param} (unicode src, unicode sought, unicode replacement, varstring locale\_name)}} \\
\hline
\end{tabularx}
}

\par





Returns the source string with the replacement string substituted for all instances of the search string.






\par
\begin{description}
\item [\colorbox{tagtype}{\color{white} \textbf{\textsf{PARAMETER}}}] \textbf{\underline{src}} ||| UNICODE --- The string that is being transformed.
\item [\colorbox{tagtype}{\color{white} \textbf{\textsf{PARAMETER}}}] \textbf{\underline{replacement}} ||| UNICODE --- The string to be substituted into the result.
\item [\colorbox{tagtype}{\color{white} \textbf{\textsf{PARAMETER}}}] \textbf{\underline{sought}} ||| UNICODE --- The string to be replaced.
\item [\colorbox{tagtype}{\color{white} \textbf{\textsf{PARAMETER}}}] \textbf{\underline{locale\_name}} ||| VARSTRING --- The locale to use for the comparison
\end{description}







\par
\begin{description}
\item [\colorbox{tagtype}{\color{white} \textbf{\textsf{RETURN}}}] \textbf{UNICODE} --- 
\end{description}




\rule{\linewidth}{0.5pt}
\subsection*{\textsf{\colorbox{headtoc}{\color{white} FUNCTION}
LocaleFindAtStrengthReplace}}

\hypertarget{ecldoc:uni.localefindatstrengthreplace}{}
\hspace{0pt} \hyperlink{ecldoc:Uni}{Uni} \textbackslash 

{\renewcommand{\arraystretch}{1.5}
\begin{tabularx}{\textwidth}{|>{\raggedright\arraybackslash}l|X|}
\hline
\hspace{0pt}\mytexttt{\color{red} unicode} & \textbf{LocaleFindAtStrengthReplace} \\
\hline
\multicolumn{2}{|>{\raggedright\arraybackslash}X|}{\hspace{0pt}\mytexttt{\color{param} (unicode src, unicode sought, unicode replacement, varstring locale\_name, integer1 strength)}} \\
\hline
\end{tabularx}
}

\par





Returns the source string with the replacement string substituted for all instances of the search string.






\par
\begin{description}
\item [\colorbox{tagtype}{\color{white} \textbf{\textsf{PARAMETER}}}] \textbf{\underline{strength}} ||| INTEGER1 --- The strength of the comparison
\item [\colorbox{tagtype}{\color{white} \textbf{\textsf{PARAMETER}}}] \textbf{\underline{src}} ||| UNICODE --- The string that is being transformed.
\item [\colorbox{tagtype}{\color{white} \textbf{\textsf{PARAMETER}}}] \textbf{\underline{replacement}} ||| UNICODE --- The string to be substituted into the result.
\item [\colorbox{tagtype}{\color{white} \textbf{\textsf{PARAMETER}}}] \textbf{\underline{sought}} ||| UNICODE --- The string to be replaced.
\item [\colorbox{tagtype}{\color{white} \textbf{\textsf{PARAMETER}}}] \textbf{\underline{locale\_name}} ||| VARSTRING --- The locale to use for the comparison
\end{description}







\par
\begin{description}
\item [\colorbox{tagtype}{\color{white} \textbf{\textsf{RETURN}}}] \textbf{UNICODE} --- 
\end{description}




\rule{\linewidth}{0.5pt}
\subsection*{\textsf{\colorbox{headtoc}{\color{white} FUNCTION}
CleanAccents}}

\hypertarget{ecldoc:uni.cleanaccents}{}
\hspace{0pt} \hyperlink{ecldoc:Uni}{Uni} \textbackslash 

{\renewcommand{\arraystretch}{1.5}
\begin{tabularx}{\textwidth}{|>{\raggedright\arraybackslash}l|X|}
\hline
\hspace{0pt}\mytexttt{\color{red} unicode} & \textbf{CleanAccents} \\
\hline
\multicolumn{2}{|>{\raggedright\arraybackslash}X|}{\hspace{0pt}\mytexttt{\color{param} (unicode src)}} \\
\hline
\end{tabularx}
}

\par





Returns the source string with all accented characters replaced with unaccented.






\par
\begin{description}
\item [\colorbox{tagtype}{\color{white} \textbf{\textsf{PARAMETER}}}] \textbf{\underline{src}} ||| UNICODE --- The string that is being transformed.
\end{description}







\par
\begin{description}
\item [\colorbox{tagtype}{\color{white} \textbf{\textsf{RETURN}}}] \textbf{UNICODE} --- 
\end{description}




\rule{\linewidth}{0.5pt}
\subsection*{\textsf{\colorbox{headtoc}{\color{white} FUNCTION}
CleanSpaces}}

\hypertarget{ecldoc:uni.cleanspaces}{}
\hspace{0pt} \hyperlink{ecldoc:Uni}{Uni} \textbackslash 

{\renewcommand{\arraystretch}{1.5}
\begin{tabularx}{\textwidth}{|>{\raggedright\arraybackslash}l|X|}
\hline
\hspace{0pt}\mytexttt{\color{red} unicode} & \textbf{CleanSpaces} \\
\hline
\multicolumn{2}{|>{\raggedright\arraybackslash}X|}{\hspace{0pt}\mytexttt{\color{param} (unicode src)}} \\
\hline
\end{tabularx}
}

\par





Returns the source string with all instances of multiple adjacent space characters (2 or more spaces together) reduced to a single space character. Leading and trailing spaces are removed, and tab characters are converted to spaces.






\par
\begin{description}
\item [\colorbox{tagtype}{\color{white} \textbf{\textsf{PARAMETER}}}] \textbf{\underline{src}} ||| UNICODE --- The string to be cleaned.
\end{description}







\par
\begin{description}
\item [\colorbox{tagtype}{\color{white} \textbf{\textsf{RETURN}}}] \textbf{UNICODE} --- 
\end{description}




\rule{\linewidth}{0.5pt}
\subsection*{\textsf{\colorbox{headtoc}{\color{white} FUNCTION}
WildMatch}}

\hypertarget{ecldoc:uni.wildmatch}{}
\hspace{0pt} \hyperlink{ecldoc:Uni}{Uni} \textbackslash 

{\renewcommand{\arraystretch}{1.5}
\begin{tabularx}{\textwidth}{|>{\raggedright\arraybackslash}l|X|}
\hline
\hspace{0pt}\mytexttt{\color{red} boolean} & \textbf{WildMatch} \\
\hline
\multicolumn{2}{|>{\raggedright\arraybackslash}X|}{\hspace{0pt}\mytexttt{\color{param} (unicode src, unicode \_pattern, boolean \_noCase)}} \\
\hline
\end{tabularx}
}

\par





Tests if the search string matches the pattern. The pattern can contain wildcards '?' (single character) and '*' (multiple character).






\par
\begin{description}
\item [\colorbox{tagtype}{\color{white} \textbf{\textsf{PARAMETER}}}] \textbf{\underline{pattern}} |||  --- The pattern to match against.
\item [\colorbox{tagtype}{\color{white} \textbf{\textsf{PARAMETER}}}] \textbf{\underline{src}} ||| UNICODE --- The string that is being tested.
\item [\colorbox{tagtype}{\color{white} \textbf{\textsf{PARAMETER}}}] \textbf{\underline{ignore\_case}} |||  --- Whether to ignore differences in case between characters
\item [\colorbox{tagtype}{\color{white} \textbf{\textsf{PARAMETER}}}] \textbf{\underline{\_nocase}} ||| BOOLEAN --- No Doc
\item [\colorbox{tagtype}{\color{white} \textbf{\textsf{PARAMETER}}}] \textbf{\underline{\_pattern}} ||| UNICODE --- No Doc
\end{description}







\par
\begin{description}
\item [\colorbox{tagtype}{\color{white} \textbf{\textsf{RETURN}}}] \textbf{BOOLEAN} --- 
\end{description}




\rule{\linewidth}{0.5pt}
\subsection*{\textsf{\colorbox{headtoc}{\color{white} FUNCTION}
Contains}}

\hypertarget{ecldoc:uni.contains}{}
\hspace{0pt} \hyperlink{ecldoc:Uni}{Uni} \textbackslash 

{\renewcommand{\arraystretch}{1.5}
\begin{tabularx}{\textwidth}{|>{\raggedright\arraybackslash}l|X|}
\hline
\hspace{0pt}\mytexttt{\color{red} BOOLEAN} & \textbf{Contains} \\
\hline
\multicolumn{2}{|>{\raggedright\arraybackslash}X|}{\hspace{0pt}\mytexttt{\color{param} (unicode src, unicode \_pattern, boolean \_noCase)}} \\
\hline
\end{tabularx}
}

\par





Tests if the search string contains each of the characters in the pattern. If the pattern contains duplicate characters those characters will match once for each occurence in the pattern.






\par
\begin{description}
\item [\colorbox{tagtype}{\color{white} \textbf{\textsf{PARAMETER}}}] \textbf{\underline{pattern}} |||  --- The pattern to match against.
\item [\colorbox{tagtype}{\color{white} \textbf{\textsf{PARAMETER}}}] \textbf{\underline{src}} ||| UNICODE --- The string that is being tested.
\item [\colorbox{tagtype}{\color{white} \textbf{\textsf{PARAMETER}}}] \textbf{\underline{ignore\_case}} |||  --- Whether to ignore differences in case between characters
\item [\colorbox{tagtype}{\color{white} \textbf{\textsf{PARAMETER}}}] \textbf{\underline{\_nocase}} ||| BOOLEAN --- No Doc
\item [\colorbox{tagtype}{\color{white} \textbf{\textsf{PARAMETER}}}] \textbf{\underline{\_pattern}} ||| UNICODE --- No Doc
\end{description}







\par
\begin{description}
\item [\colorbox{tagtype}{\color{white} \textbf{\textsf{RETURN}}}] \textbf{BOOLEAN} --- 
\end{description}




\rule{\linewidth}{0.5pt}
\subsection*{\textsf{\colorbox{headtoc}{\color{white} FUNCTION}
EditDistance}}

\hypertarget{ecldoc:uni.editdistance}{}
\hspace{0pt} \hyperlink{ecldoc:Uni}{Uni} \textbackslash 

{\renewcommand{\arraystretch}{1.5}
\begin{tabularx}{\textwidth}{|>{\raggedright\arraybackslash}l|X|}
\hline
\hspace{0pt}\mytexttt{\color{red} UNSIGNED4} & \textbf{EditDistance} \\
\hline
\multicolumn{2}{|>{\raggedright\arraybackslash}X|}{\hspace{0pt}\mytexttt{\color{param} (unicode \_left, unicode \_right, varstring localename = '')}} \\
\hline
\end{tabularx}
}

\par





Returns the minimum edit distance between the two strings. An insert change or delete counts as a single edit. The two strings are trimmed before comparing.






\par
\begin{description}
\item [\colorbox{tagtype}{\color{white} \textbf{\textsf{PARAMETER}}}] \textbf{\underline{\_left}} ||| UNICODE --- The first string to be compared.
\item [\colorbox{tagtype}{\color{white} \textbf{\textsf{PARAMETER}}}] \textbf{\underline{localname}} |||  --- The locale to use for the comparison. Defaults to ''.
\item [\colorbox{tagtype}{\color{white} \textbf{\textsf{PARAMETER}}}] \textbf{\underline{\_right}} ||| UNICODE --- The second string to be compared.
\item [\colorbox{tagtype}{\color{white} \textbf{\textsf{PARAMETER}}}] \textbf{\underline{localename}} ||| VARSTRING --- No Doc
\end{description}







\par
\begin{description}
\item [\colorbox{tagtype}{\color{white} \textbf{\textsf{RETURN}}}] \textbf{UNSIGNED4} --- The minimum edit distance between the two strings.
\end{description}




\rule{\linewidth}{0.5pt}
\subsection*{\textsf{\colorbox{headtoc}{\color{white} FUNCTION}
EditDistanceWithinRadius}}

\hypertarget{ecldoc:uni.editdistancewithinradius}{}
\hspace{0pt} \hyperlink{ecldoc:Uni}{Uni} \textbackslash 

{\renewcommand{\arraystretch}{1.5}
\begin{tabularx}{\textwidth}{|>{\raggedright\arraybackslash}l|X|}
\hline
\hspace{0pt}\mytexttt{\color{red} BOOLEAN} & \textbf{EditDistanceWithinRadius} \\
\hline
\multicolumn{2}{|>{\raggedright\arraybackslash}X|}{\hspace{0pt}\mytexttt{\color{param} (unicode \_left, unicode \_right, unsigned4 radius, varstring localename = '')}} \\
\hline
\end{tabularx}
}

\par





Returns true if the minimum edit distance between the two strings is with a specific range. The two strings are trimmed before comparing.






\par
\begin{description}
\item [\colorbox{tagtype}{\color{white} \textbf{\textsf{PARAMETER}}}] \textbf{\underline{\_left}} ||| UNICODE --- The first string to be compared.
\item [\colorbox{tagtype}{\color{white} \textbf{\textsf{PARAMETER}}}] \textbf{\underline{localname}} |||  --- The locale to use for the comparison. Defaults to ''.
\item [\colorbox{tagtype}{\color{white} \textbf{\textsf{PARAMETER}}}] \textbf{\underline{\_right}} ||| UNICODE --- The second string to be compared.
\item [\colorbox{tagtype}{\color{white} \textbf{\textsf{PARAMETER}}}] \textbf{\underline{radius}} ||| UNSIGNED4 --- The maximum edit distance that is accepable.
\item [\colorbox{tagtype}{\color{white} \textbf{\textsf{PARAMETER}}}] \textbf{\underline{localename}} ||| VARSTRING --- No Doc
\end{description}







\par
\begin{description}
\item [\colorbox{tagtype}{\color{white} \textbf{\textsf{RETURN}}}] \textbf{BOOLEAN} --- Whether or not the two strings are within the given specified edit distance.
\end{description}




\rule{\linewidth}{0.5pt}
\subsection*{\textsf{\colorbox{headtoc}{\color{white} FUNCTION}
WordCount}}

\hypertarget{ecldoc:uni.wordcount}{}
\hspace{0pt} \hyperlink{ecldoc:Uni}{Uni} \textbackslash 

{\renewcommand{\arraystretch}{1.5}
\begin{tabularx}{\textwidth}{|>{\raggedright\arraybackslash}l|X|}
\hline
\hspace{0pt}\mytexttt{\color{red} unsigned4} & \textbf{WordCount} \\
\hline
\multicolumn{2}{|>{\raggedright\arraybackslash}X|}{\hspace{0pt}\mytexttt{\color{param} (unicode text, varstring localename = '')}} \\
\hline
\end{tabularx}
}

\par





Returns the number of words in the string. Word boundaries are marked by the unicode break semantics.






\par
\begin{description}
\item [\colorbox{tagtype}{\color{white} \textbf{\textsf{PARAMETER}}}] \textbf{\underline{localname}} |||  --- The locale to use for the break semantics. Defaults to ''.
\item [\colorbox{tagtype}{\color{white} \textbf{\textsf{PARAMETER}}}] \textbf{\underline{text}} ||| UNICODE --- The string to be broken into words.
\item [\colorbox{tagtype}{\color{white} \textbf{\textsf{PARAMETER}}}] \textbf{\underline{localename}} ||| VARSTRING --- No Doc
\end{description}







\par
\begin{description}
\item [\colorbox{tagtype}{\color{white} \textbf{\textsf{RETURN}}}] \textbf{UNSIGNED4} --- The number of words in the string.
\end{description}




\rule{\linewidth}{0.5pt}
\subsection*{\textsf{\colorbox{headtoc}{\color{white} FUNCTION}
GetNthWord}}

\hypertarget{ecldoc:uni.getnthword}{}
\hspace{0pt} \hyperlink{ecldoc:Uni}{Uni} \textbackslash 

{\renewcommand{\arraystretch}{1.5}
\begin{tabularx}{\textwidth}{|>{\raggedright\arraybackslash}l|X|}
\hline
\hspace{0pt}\mytexttt{\color{red} unicode} & \textbf{GetNthWord} \\
\hline
\multicolumn{2}{|>{\raggedright\arraybackslash}X|}{\hspace{0pt}\mytexttt{\color{param} (unicode text, unsigned4 n, varstring localename = '')}} \\
\hline
\end{tabularx}
}

\par





Returns the n-th word from the string. Word boundaries are marked by the unicode break semantics.






\par
\begin{description}
\item [\colorbox{tagtype}{\color{white} \textbf{\textsf{PARAMETER}}}] \textbf{\underline{localname}} |||  --- The locale to use for the break semantics. Defaults to ''.
\item [\colorbox{tagtype}{\color{white} \textbf{\textsf{PARAMETER}}}] \textbf{\underline{n}} ||| UNSIGNED4 --- Which word should be returned from the function.
\item [\colorbox{tagtype}{\color{white} \textbf{\textsf{PARAMETER}}}] \textbf{\underline{text}} ||| UNICODE --- The string to be broken into words.
\item [\colorbox{tagtype}{\color{white} \textbf{\textsf{PARAMETER}}}] \textbf{\underline{localename}} ||| VARSTRING --- No Doc
\end{description}







\par
\begin{description}
\item [\colorbox{tagtype}{\color{white} \textbf{\textsf{RETURN}}}] \textbf{UNICODE} --- The number of words in the string.
\end{description}




\rule{\linewidth}{0.5pt}



\chapter*{\color{headtoc} root}
\hypertarget{ecldoc:toc:root}{}
\hyperlink{ecldoc:toc:}{Go Up}


\section*{Table of Contents}
{\renewcommand{\arraystretch}{1.5}
\begin{longtable}{|p{\textwidth}|}
\hline
\hyperlink{ecldoc:toc:BLAS}{BLAS.ecl} \\
\hline
\hyperlink{ecldoc:toc:BundleBase}{BundleBase.ecl} \\
\hline
\hyperlink{ecldoc:toc:Date}{Date.ecl} \\
\hline
\hyperlink{ecldoc:toc:File}{File.ecl} \\
\hline
\hyperlink{ecldoc:toc:math}{math.ecl} \\
\hline
\hyperlink{ecldoc:toc:Metaphone}{Metaphone.ecl} \\
\hline
\hyperlink{ecldoc:toc:str}{str.ecl} \\
\hline
\hyperlink{ecldoc:toc:Uni}{Uni.ecl} \\
\hline
\hyperlink{ecldoc:toc:root/system}{system} \\
\hline
\end{longtable}
}

\input{root/BLAS.ecl}
\input{root/BundleBase.ecl}
\input{root/Date.ecl}
\input{root/File.ecl}
\input{root/math.ecl}
\input{root/Metaphone.ecl}
\input{root/str.ecl}
\input{root/Uni.ecl}
\input{root/system/pkg.toc}



\chapter*{\color{headtoc} root}
\hypertarget{ecldoc:toc:root}{}
\hyperlink{ecldoc:toc:}{Go Up}


\section*{Table of Contents}
{\renewcommand{\arraystretch}{1.5}
\begin{longtable}{|p{\textwidth}|}
\hline
\hyperlink{ecldoc:toc:BLAS}{BLAS.ecl} \\
\hline
\hyperlink{ecldoc:toc:BundleBase}{BundleBase.ecl} \\
\hline
\hyperlink{ecldoc:toc:Date}{Date.ecl} \\
\hline
\hyperlink{ecldoc:toc:File}{File.ecl} \\
\hline
\hyperlink{ecldoc:toc:math}{math.ecl} \\
\hline
\hyperlink{ecldoc:toc:Metaphone}{Metaphone.ecl} \\
\hline
\hyperlink{ecldoc:toc:str}{str.ecl} \\
\hline
\hyperlink{ecldoc:toc:Uni}{Uni.ecl} \\
\hline
\hyperlink{ecldoc:toc:root/system}{system} \\
\hline
\end{longtable}
}

\chapter*{\color{headfile}
BLAS
}
\hypertarget{ecldoc:toc:BLAS}{}
\hyperlink{ecldoc:toc:root}{Go Up}

\section*{\underline{\textsf{IMPORTS}}}
\begin{doublespace}
{\large
lib\_eclblas |
}
\end{doublespace}

\section*{\underline{\textsf{DESCRIPTIONS}}}
\subsection*{\textsf{\colorbox{headtoc}{\color{white} MODULE}
BLAS}}

\hypertarget{ecldoc:blas}{}

{\renewcommand{\arraystretch}{1.5}
\begin{tabularx}{\textwidth}{|>{\raggedright\arraybackslash}l|X|}
\hline
\hspace{0pt}\mytexttt{\color{red} } & \textbf{BLAS} \\
\hline
\end{tabularx}
}

\par


\textbf{Children}
\begin{enumerate}
\item \hyperlink{ecldoc:BLAS.Types}{Types}
\item \hyperlink{ecldoc:blas.icellfunc}{ICellFunc}
: Function prototype for Apply2Cell
\item \hyperlink{ecldoc:blas.apply2cells}{Apply2Cells}
: Iterate matrix and apply function to each cell
\item \hyperlink{ecldoc:blas.dasum}{dasum}
: Absolute sum, the 1 norm of a vector
\item \hyperlink{ecldoc:blas.daxpy}{daxpy}
: alpha*X + Y
\item \hyperlink{ecldoc:blas.dgemm}{dgemm}
: alpha*op(A) op(B) + beta*C where op() is transpose
\item \hyperlink{ecldoc:blas.dgetf2}{dgetf2}
: Compute LU Factorization of matrix A
\item \hyperlink{ecldoc:blas.dpotf2}{dpotf2}
: DPOTF2 computes the Cholesky factorization of a real symmetric positive definite matrix A
\item \hyperlink{ecldoc:blas.dscal}{dscal}
: Scale a vector alpha
\item \hyperlink{ecldoc:blas.dsyrk}{dsyrk}
: Implements symmetric rank update C
\item \hyperlink{ecldoc:blas.dtrsm}{dtrsm}
: Triangular matrix solver
\item \hyperlink{ecldoc:blas.extract_diag}{extract\_diag}
: Extract the diagonal of he matrix
\item \hyperlink{ecldoc:blas.extract_tri}{extract\_tri}
: Extract the upper or lower triangle
\item \hyperlink{ecldoc:blas.make_diag}{make\_diag}
: Generate a diagonal matrix
\item \hyperlink{ecldoc:blas.make_vector}{make\_vector}
: Make a vector of dimension m
\item \hyperlink{ecldoc:blas.trace}{trace}
: The trace of the input matrix
\end{enumerate}

\rule{\linewidth}{0.5pt}

\subsection*{\textsf{\colorbox{headtoc}{\color{white} MODULE}
Types}}

\hypertarget{ecldoc:BLAS.Types}{}
\hspace{0pt} \hyperlink{ecldoc:blas}{BLAS} \textbackslash 

{\renewcommand{\arraystretch}{1.5}
\begin{tabularx}{\textwidth}{|>{\raggedright\arraybackslash}l|X|}
\hline
\hspace{0pt}\mytexttt{\color{red} } & \textbf{Types} \\
\hline
\end{tabularx}
}

\par


\textbf{Children}
\begin{enumerate}
\item \hyperlink{ecldoc:blas.types.value_t}{value\_t}
\item \hyperlink{ecldoc:blas.types.dimension_t}{dimension\_t}
\item \hyperlink{ecldoc:blas.types.matrix_t}{matrix\_t}
\item \hyperlink{ecldoc:ecldoc-Triangle}{Triangle}
\item \hyperlink{ecldoc:ecldoc-Diagonal}{Diagonal}
\item \hyperlink{ecldoc:ecldoc-Side}{Side}
\end{enumerate}

\rule{\linewidth}{0.5pt}

\subsection*{\textsf{\colorbox{headtoc}{\color{white} ATTRIBUTE}
value\_t}}

\hypertarget{ecldoc:blas.types.value_t}{}
\hspace{0pt} \hyperlink{ecldoc:blas}{BLAS} \textbackslash 
\hspace{0pt} \hyperlink{ecldoc:BLAS.Types}{Types} \textbackslash 

{\renewcommand{\arraystretch}{1.5}
\begin{tabularx}{\textwidth}{|>{\raggedright\arraybackslash}l|X|}
\hline
\hspace{0pt}\mytexttt{\color{red} } & \textbf{value\_t} \\
\hline
\end{tabularx}
}

\par


\rule{\linewidth}{0.5pt}
\subsection*{\textsf{\colorbox{headtoc}{\color{white} ATTRIBUTE}
dimension\_t}}

\hypertarget{ecldoc:blas.types.dimension_t}{}
\hspace{0pt} \hyperlink{ecldoc:blas}{BLAS} \textbackslash 
\hspace{0pt} \hyperlink{ecldoc:BLAS.Types}{Types} \textbackslash 

{\renewcommand{\arraystretch}{1.5}
\begin{tabularx}{\textwidth}{|>{\raggedright\arraybackslash}l|X|}
\hline
\hspace{0pt}\mytexttt{\color{red} } & \textbf{dimension\_t} \\
\hline
\end{tabularx}
}

\par


\rule{\linewidth}{0.5pt}
\subsection*{\textsf{\colorbox{headtoc}{\color{white} ATTRIBUTE}
matrix\_t}}

\hypertarget{ecldoc:blas.types.matrix_t}{}
\hspace{0pt} \hyperlink{ecldoc:blas}{BLAS} \textbackslash 
\hspace{0pt} \hyperlink{ecldoc:BLAS.Types}{Types} \textbackslash 

{\renewcommand{\arraystretch}{1.5}
\begin{tabularx}{\textwidth}{|>{\raggedright\arraybackslash}l|X|}
\hline
\hspace{0pt}\mytexttt{\color{red} } & \textbf{matrix\_t} \\
\hline
\end{tabularx}
}

\par


\rule{\linewidth}{0.5pt}
\subsection*{\textsf{\colorbox{headtoc}{\color{white} ATTRIBUTE}
Triangle}}

\hypertarget{ecldoc:ecldoc-Triangle}{}
\hspace{0pt} \hyperlink{ecldoc:blas}{BLAS} \textbackslash 
\hspace{0pt} \hyperlink{ecldoc:BLAS.Types}{Types} \textbackslash 

{\renewcommand{\arraystretch}{1.5}
\begin{tabularx}{\textwidth}{|>{\raggedright\arraybackslash}l|X|}
\hline
\hspace{0pt}\mytexttt{\color{red} } & \textbf{Triangle} \\
\hline
\end{tabularx}
}

\par


\rule{\linewidth}{0.5pt}
\subsection*{\textsf{\colorbox{headtoc}{\color{white} ATTRIBUTE}
Diagonal}}

\hypertarget{ecldoc:ecldoc-Diagonal}{}
\hspace{0pt} \hyperlink{ecldoc:blas}{BLAS} \textbackslash 
\hspace{0pt} \hyperlink{ecldoc:BLAS.Types}{Types} \textbackslash 

{\renewcommand{\arraystretch}{1.5}
\begin{tabularx}{\textwidth}{|>{\raggedright\arraybackslash}l|X|}
\hline
\hspace{0pt}\mytexttt{\color{red} } & \textbf{Diagonal} \\
\hline
\end{tabularx}
}

\par


\rule{\linewidth}{0.5pt}
\subsection*{\textsf{\colorbox{headtoc}{\color{white} ATTRIBUTE}
Side}}

\hypertarget{ecldoc:ecldoc-Side}{}
\hspace{0pt} \hyperlink{ecldoc:blas}{BLAS} \textbackslash 
\hspace{0pt} \hyperlink{ecldoc:BLAS.Types}{Types} \textbackslash 

{\renewcommand{\arraystretch}{1.5}
\begin{tabularx}{\textwidth}{|>{\raggedright\arraybackslash}l|X|}
\hline
\hspace{0pt}\mytexttt{\color{red} } & \textbf{Side} \\
\hline
\end{tabularx}
}

\par


\rule{\linewidth}{0.5pt}


\subsection*{\textsf{\colorbox{headtoc}{\color{white} FUNCTION}
ICellFunc}}

\hypertarget{ecldoc:blas.icellfunc}{}
\hspace{0pt} \hyperlink{ecldoc:blas}{BLAS} \textbackslash 

{\renewcommand{\arraystretch}{1.5}
\begin{tabularx}{\textwidth}{|>{\raggedright\arraybackslash}l|X|}
\hline
\hspace{0pt}\mytexttt{\color{red} Types.value\_t} & \textbf{ICellFunc} \\
\hline
\multicolumn{2}{|>{\raggedright\arraybackslash}X|}{\hspace{0pt}\mytexttt{\color{param} (Types.value\_t v, Types.dimension\_t r, Types.dimension\_t c)}} \\
\hline
\end{tabularx}
}

\par
Function prototype for Apply2Cell.

\par
\begin{description}
\item [\colorbox{tagtype}{\color{white} \textbf{\textsf{PARAMETER}}}] \textbf{\underline{v}} the value
\item [\colorbox{tagtype}{\color{white} \textbf{\textsf{PARAMETER}}}] \textbf{\underline{r}} the row ordinal
\item [\colorbox{tagtype}{\color{white} \textbf{\textsf{PARAMETER}}}] \textbf{\underline{c}} the column ordinal
\item [\colorbox{tagtype}{\color{white} \textbf{\textsf{RETURN}}}] \textbf{\underline{}} the updated value
\end{description}

\rule{\linewidth}{0.5pt}
\subsection*{\textsf{\colorbox{headtoc}{\color{white} FUNCTION}
Apply2Cells}}

\hypertarget{ecldoc:blas.apply2cells}{}
\hspace{0pt} \hyperlink{ecldoc:blas}{BLAS} \textbackslash 

{\renewcommand{\arraystretch}{1.5}
\begin{tabularx}{\textwidth}{|>{\raggedright\arraybackslash}l|X|}
\hline
\hspace{0pt}\mytexttt{\color{red} Types.matrix\_t} & \textbf{Apply2Cells} \\
\hline
\multicolumn{2}{|>{\raggedright\arraybackslash}X|}{\hspace{0pt}\mytexttt{\color{param} (Types.dimension\_t m, Types.dimension\_t n, Types.matrix\_t x, ICellFunc f)}} \\
\hline
\end{tabularx}
}

\par
Iterate matrix and apply function to each cell

\par
\begin{description}
\item [\colorbox{tagtype}{\color{white} \textbf{\textsf{PARAMETER}}}] \textbf{\underline{m}} number of rows
\item [\colorbox{tagtype}{\color{white} \textbf{\textsf{PARAMETER}}}] \textbf{\underline{n}} number of columns
\item [\colorbox{tagtype}{\color{white} \textbf{\textsf{PARAMETER}}}] \textbf{\underline{x}} matrix
\item [\colorbox{tagtype}{\color{white} \textbf{\textsf{PARAMETER}}}] \textbf{\underline{f}} function to apply
\item [\colorbox{tagtype}{\color{white} \textbf{\textsf{RETURN}}}] \textbf{\underline{}} updated matrix
\end{description}

\rule{\linewidth}{0.5pt}
\subsection*{\textsf{\colorbox{headtoc}{\color{white} FUNCTION}
dasum}}

\hypertarget{ecldoc:blas.dasum}{}
\hspace{0pt} \hyperlink{ecldoc:blas}{BLAS} \textbackslash 

{\renewcommand{\arraystretch}{1.5}
\begin{tabularx}{\textwidth}{|>{\raggedright\arraybackslash}l|X|}
\hline
\hspace{0pt}\mytexttt{\color{red} Types.value\_t} & \textbf{dasum} \\
\hline
\multicolumn{2}{|>{\raggedright\arraybackslash}X|}{\hspace{0pt}\mytexttt{\color{param} (Types.dimension\_t m, Types.matrix\_t x, Types.dimension\_t incx, Types.dimension\_t skipped=0)}} \\
\hline
\end{tabularx}
}

\par
Absolute sum, the 1 norm of a vector.

\par
\begin{description}
\item [\colorbox{tagtype}{\color{white} \textbf{\textsf{PARAMETER}}}] \textbf{\underline{m}} the number of entries
\item [\colorbox{tagtype}{\color{white} \textbf{\textsf{PARAMETER}}}] \textbf{\underline{x}} the column major matrix holding the vector
\item [\colorbox{tagtype}{\color{white} \textbf{\textsf{PARAMETER}}}] \textbf{\underline{incx}} the increment for x, 1 in the case of an actual vector
\item [\colorbox{tagtype}{\color{white} \textbf{\textsf{PARAMETER}}}] \textbf{\underline{skipped}} default is zero, the number of entries stepped over to get to the first entry
\item [\colorbox{tagtype}{\color{white} \textbf{\textsf{RETURN}}}] \textbf{\underline{}} the sum of the absolute values
\end{description}

\rule{\linewidth}{0.5pt}
\subsection*{\textsf{\colorbox{headtoc}{\color{white} FUNCTION}
daxpy}}

\hypertarget{ecldoc:blas.daxpy}{}
\hspace{0pt} \hyperlink{ecldoc:blas}{BLAS} \textbackslash 

{\renewcommand{\arraystretch}{1.5}
\begin{tabularx}{\textwidth}{|>{\raggedright\arraybackslash}l|X|}
\hline
\hspace{0pt}\mytexttt{\color{red} Types.matrix\_t} & \textbf{daxpy} \\
\hline
\multicolumn{2}{|>{\raggedright\arraybackslash}X|}{\hspace{0pt}\mytexttt{\color{param} (Types.dimension\_t N, Types.value\_t alpha, Types.matrix\_t X, Types.dimension\_t incX, Types.matrix\_t Y, Types.dimension\_t incY, Types.dimension\_t x\_skipped=0, Types.dimension\_t y\_skipped=0)}} \\
\hline
\end{tabularx}
}

\par
alpha*X + Y

\par
\begin{description}
\item [\colorbox{tagtype}{\color{white} \textbf{\textsf{PARAMETER}}}] \textbf{\underline{N}} number of elements in vector
\item [\colorbox{tagtype}{\color{white} \textbf{\textsf{PARAMETER}}}] \textbf{\underline{alpha}} the scalar multiplier
\item [\colorbox{tagtype}{\color{white} \textbf{\textsf{PARAMETER}}}] \textbf{\underline{X}} the column major matrix holding the vector X
\item [\colorbox{tagtype}{\color{white} \textbf{\textsf{PARAMETER}}}] \textbf{\underline{incX}} the increment or stride for the vector
\item [\colorbox{tagtype}{\color{white} \textbf{\textsf{PARAMETER}}}] \textbf{\underline{Y}} the column major matrix holding the vector Y
\item [\colorbox{tagtype}{\color{white} \textbf{\textsf{PARAMETER}}}] \textbf{\underline{incY}} the increment or stride of Y
\item [\colorbox{tagtype}{\color{white} \textbf{\textsf{PARAMETER}}}] \textbf{\underline{x\_skipped}} number of entries skipped to get to the first X
\item [\colorbox{tagtype}{\color{white} \textbf{\textsf{PARAMETER}}}] \textbf{\underline{y\_skipped}} number of entries skipped to get to the first Y
\item [\colorbox{tagtype}{\color{white} \textbf{\textsf{RETURN}}}] \textbf{\underline{}} the updated matrix
\end{description}

\rule{\linewidth}{0.5pt}
\subsection*{\textsf{\colorbox{headtoc}{\color{white} FUNCTION}
dgemm}}

\hypertarget{ecldoc:blas.dgemm}{}
\hspace{0pt} \hyperlink{ecldoc:blas}{BLAS} \textbackslash 

{\renewcommand{\arraystretch}{1.5}
\begin{tabularx}{\textwidth}{|>{\raggedright\arraybackslash}l|X|}
\hline
\hspace{0pt}\mytexttt{\color{red} Types.matrix\_t} & \textbf{dgemm} \\
\hline
\multicolumn{2}{|>{\raggedright\arraybackslash}X|}{\hspace{0pt}\mytexttt{\color{param} (BOOLEAN transposeA, BOOLEAN transposeB, Types.dimension\_t M, Types.dimension\_t N, Types.dimension\_t K, Types.value\_t alpha, Types.matrix\_t A, Types.matrix\_t B, Types.value\_t beta=0.0, Types.matrix\_t C=[])}} \\
\hline
\end{tabularx}
}

\par
alpha*op(A) op(B) + beta*C where op() is transpose

\par
\begin{description}
\item [\colorbox{tagtype}{\color{white} \textbf{\textsf{PARAMETER}}}] \textbf{\underline{transposeA}} true when transpose of A is used
\item [\colorbox{tagtype}{\color{white} \textbf{\textsf{PARAMETER}}}] \textbf{\underline{transposeB}} true when transpose of B is used
\item [\colorbox{tagtype}{\color{white} \textbf{\textsf{PARAMETER}}}] \textbf{\underline{M}} number of rows in product
\item [\colorbox{tagtype}{\color{white} \textbf{\textsf{PARAMETER}}}] \textbf{\underline{N}} number of columns in product
\item [\colorbox{tagtype}{\color{white} \textbf{\textsf{PARAMETER}}}] \textbf{\underline{K}} number of columns/rows for the multiplier/multiplicand
\item [\colorbox{tagtype}{\color{white} \textbf{\textsf{PARAMETER}}}] \textbf{\underline{alpha}} scalar used on A
\item [\colorbox{tagtype}{\color{white} \textbf{\textsf{PARAMETER}}}] \textbf{\underline{A}} matrix A
\item [\colorbox{tagtype}{\color{white} \textbf{\textsf{PARAMETER}}}] \textbf{\underline{B}} matrix B
\item [\colorbox{tagtype}{\color{white} \textbf{\textsf{PARAMETER}}}] \textbf{\underline{beta}} scalar for matrix C
\item [\colorbox{tagtype}{\color{white} \textbf{\textsf{PARAMETER}}}] \textbf{\underline{C}} matrix C or empty
\end{description}

\rule{\linewidth}{0.5pt}
\subsection*{\textsf{\colorbox{headtoc}{\color{white} FUNCTION}
dgetf2}}

\hypertarget{ecldoc:blas.dgetf2}{}
\hspace{0pt} \hyperlink{ecldoc:blas}{BLAS} \textbackslash 

{\renewcommand{\arraystretch}{1.5}
\begin{tabularx}{\textwidth}{|>{\raggedright\arraybackslash}l|X|}
\hline
\hspace{0pt}\mytexttt{\color{red} Types.matrix\_t} & \textbf{dgetf2} \\
\hline
\multicolumn{2}{|>{\raggedright\arraybackslash}X|}{\hspace{0pt}\mytexttt{\color{param} (Types.dimension\_t m, Types.dimension\_t n, Types.matrix\_t a)}} \\
\hline
\end{tabularx}
}

\par
Compute LU Factorization of matrix A.

\par
\begin{description}
\item [\colorbox{tagtype}{\color{white} \textbf{\textsf{PARAMETER}}}] \textbf{\underline{m}} number of rows of A
\item [\colorbox{tagtype}{\color{white} \textbf{\textsf{PARAMETER}}}] \textbf{\underline{n}} number of columns of A
\item [\colorbox{tagtype}{\color{white} \textbf{\textsf{RETURN}}}] \textbf{\underline{}} composite matrix of factors, lower triangle has an implied diagonal of ones. Upper triangle has the diagonal of the composite.
\end{description}

\rule{\linewidth}{0.5pt}
\subsection*{\textsf{\colorbox{headtoc}{\color{white} FUNCTION}
dpotf2}}

\hypertarget{ecldoc:blas.dpotf2}{}
\hspace{0pt} \hyperlink{ecldoc:blas}{BLAS} \textbackslash 

{\renewcommand{\arraystretch}{1.5}
\begin{tabularx}{\textwidth}{|>{\raggedright\arraybackslash}l|X|}
\hline
\hspace{0pt}\mytexttt{\color{red} Types.matrix\_t} & \textbf{dpotf2} \\
\hline
\multicolumn{2}{|>{\raggedright\arraybackslash}X|}{\hspace{0pt}\mytexttt{\color{param} (Types.Triangle tri, Types.dimension\_t r, Types.matrix\_t A, BOOLEAN clear=TRUE)}} \\
\hline
\end{tabularx}
}

\par
DPOTF2 computes the Cholesky factorization of a real symmetric positive definite matrix A. The factorization has the form A = U**T * U , if UPLO = 'U', or A = L * L**T, if UPLO = 'L', where U is an upper triangular matrix and L is lower triangular. This is the unblocked version of the algorithm, calling Level 2 BLAS.

\par
\begin{description}
\item [\colorbox{tagtype}{\color{white} \textbf{\textsf{PARAMETER}}}] \textbf{\underline{tri}} indicate whether upper or lower triangle is used
\item [\colorbox{tagtype}{\color{white} \textbf{\textsf{PARAMETER}}}] \textbf{\underline{r}} number of rows/columns in the square matrix
\item [\colorbox{tagtype}{\color{white} \textbf{\textsf{PARAMETER}}}] \textbf{\underline{A}} the square matrix
\item [\colorbox{tagtype}{\color{white} \textbf{\textsf{PARAMETER}}}] \textbf{\underline{clear}} clears the unused triangle
\item [\colorbox{tagtype}{\color{white} \textbf{\textsf{RETURN}}}] \textbf{\underline{}} the triangular matrix requested.
\end{description}

\rule{\linewidth}{0.5pt}
\subsection*{\textsf{\colorbox{headtoc}{\color{white} FUNCTION}
dscal}}

\hypertarget{ecldoc:blas.dscal}{}
\hspace{0pt} \hyperlink{ecldoc:blas}{BLAS} \textbackslash 

{\renewcommand{\arraystretch}{1.5}
\begin{tabularx}{\textwidth}{|>{\raggedright\arraybackslash}l|X|}
\hline
\hspace{0pt}\mytexttt{\color{red} Types.matrix\_t} & \textbf{dscal} \\
\hline
\multicolumn{2}{|>{\raggedright\arraybackslash}X|}{\hspace{0pt}\mytexttt{\color{param} (Types.dimension\_t N, Types.value\_t alpha, Types.matrix\_t X, Types.dimension\_t incX, Types.dimension\_t skipped=0)}} \\
\hline
\end{tabularx}
}

\par
Scale a vector alpha

\par
\begin{description}
\item [\colorbox{tagtype}{\color{white} \textbf{\textsf{PARAMETER}}}] \textbf{\underline{N}} number of elements in the vector
\item [\colorbox{tagtype}{\color{white} \textbf{\textsf{PARAMETER}}}] \textbf{\underline{alpha}} the scaling factor
\item [\colorbox{tagtype}{\color{white} \textbf{\textsf{PARAMETER}}}] \textbf{\underline{X}} the column major matrix holding the vector
\item [\colorbox{tagtype}{\color{white} \textbf{\textsf{PARAMETER}}}] \textbf{\underline{incX}} the stride to get to the next element in the vector
\item [\colorbox{tagtype}{\color{white} \textbf{\textsf{PARAMETER}}}] \textbf{\underline{skipped}} the number of elements skipped to get to the first element
\item [\colorbox{tagtype}{\color{white} \textbf{\textsf{RETURN}}}] \textbf{\underline{}} the updated matrix
\end{description}

\rule{\linewidth}{0.5pt}
\subsection*{\textsf{\colorbox{headtoc}{\color{white} FUNCTION}
dsyrk}}

\hypertarget{ecldoc:blas.dsyrk}{}
\hspace{0pt} \hyperlink{ecldoc:blas}{BLAS} \textbackslash 

{\renewcommand{\arraystretch}{1.5}
\begin{tabularx}{\textwidth}{|>{\raggedright\arraybackslash}l|X|}
\hline
\hspace{0pt}\mytexttt{\color{red} Types.matrix\_t} & \textbf{dsyrk} \\
\hline
\multicolumn{2}{|>{\raggedright\arraybackslash}X|}{\hspace{0pt}\mytexttt{\color{param} (Types.Triangle tri, BOOLEAN transposeA, Types.dimension\_t N, Types.dimension\_t K, Types.value\_t alpha, Types.matrix\_t A, Types.value\_t beta, Types.matrix\_t C, BOOLEAN clear=FALSE)}} \\
\hline
\end{tabularx}
}

\par
Implements symmetric rank update C 

\par
\begin{description}
\item [\colorbox{tagtype}{\color{white} \textbf{\textsf{PARAMETER}}}] \textbf{\underline{tri}} update upper or lower triangle
\item [\colorbox{tagtype}{\color{white} \textbf{\textsf{PARAMETER}}}] \textbf{\underline{transposeA}} Transpose the A matrix to be NxK
\item [\colorbox{tagtype}{\color{white} \textbf{\textsf{PARAMETER}}}] \textbf{\underline{N}} number of rows
\item [\colorbox{tagtype}{\color{white} \textbf{\textsf{PARAMETER}}}] \textbf{\underline{K}} number of columns in the update matrix or transpose
\item [\colorbox{tagtype}{\color{white} \textbf{\textsf{PARAMETER}}}] \textbf{\underline{alpha}} the alpha scalar
\item [\colorbox{tagtype}{\color{white} \textbf{\textsf{PARAMETER}}}] \textbf{\underline{A}} the update matrix, either NxK or KxN
\item [\colorbox{tagtype}{\color{white} \textbf{\textsf{PARAMETER}}}] \textbf{\underline{beta}} the beta scalar
\item [\colorbox{tagtype}{\color{white} \textbf{\textsf{PARAMETER}}}] \textbf{\underline{C}} the matrix to update
\item [\colorbox{tagtype}{\color{white} \textbf{\textsf{PARAMETER}}}] \textbf{\underline{clear}} clear the triangle that is not updated. BLAS assumes that symmetric matrices have only one of the triangles and this option lets you make that true.
\end{description}

\rule{\linewidth}{0.5pt}
\subsection*{\textsf{\colorbox{headtoc}{\color{white} FUNCTION}
dtrsm}}

\hypertarget{ecldoc:blas.dtrsm}{}
\hspace{0pt} \hyperlink{ecldoc:blas}{BLAS} \textbackslash 

{\renewcommand{\arraystretch}{1.5}
\begin{tabularx}{\textwidth}{|>{\raggedright\arraybackslash}l|X|}
\hline
\hspace{0pt}\mytexttt{\color{red} Types.matrix\_t} & \textbf{dtrsm} \\
\hline
\multicolumn{2}{|>{\raggedright\arraybackslash}X|}{\hspace{0pt}\mytexttt{\color{param} (Types.Side side, Types.Triangle tri, BOOLEAN transposeA, Types.Diagonal diag, Types.dimension\_t M, Types.dimension\_t N, Types.dimension\_t lda, Types.value\_t alpha, Types.matrix\_t A, Types.matrix\_t B)}} \\
\hline
\end{tabularx}
}

\par
Triangular matrix solver. op(A) X = alpha B or X op(A) = alpha B where op is Transpose, X and B is MxN

\par
\begin{description}
\item [\colorbox{tagtype}{\color{white} \textbf{\textsf{PARAMETER}}}] \textbf{\underline{side}} side for A, Side.Ax is op(A) X = alpha B
\item [\colorbox{tagtype}{\color{white} \textbf{\textsf{PARAMETER}}}] \textbf{\underline{tri}} Says whether A is Upper or Lower triangle
\item [\colorbox{tagtype}{\color{white} \textbf{\textsf{PARAMETER}}}] \textbf{\underline{transposeA}} is op(A) the transpose of A
\item [\colorbox{tagtype}{\color{white} \textbf{\textsf{PARAMETER}}}] \textbf{\underline{diag}} is the diagonal an implied unit diagonal or supplied
\item [\colorbox{tagtype}{\color{white} \textbf{\textsf{PARAMETER}}}] \textbf{\underline{M}} number of rows
\item [\colorbox{tagtype}{\color{white} \textbf{\textsf{PARAMETER}}}] \textbf{\underline{N}} number of columns
\item [\colorbox{tagtype}{\color{white} \textbf{\textsf{PARAMETER}}}] \textbf{\underline{lda}} the leading dimension of the A matrix, either M or N
\item [\colorbox{tagtype}{\color{white} \textbf{\textsf{PARAMETER}}}] \textbf{\underline{alpha}} the scalar multiplier for B
\item [\colorbox{tagtype}{\color{white} \textbf{\textsf{PARAMETER}}}] \textbf{\underline{A}} a triangular matrix
\item [\colorbox{tagtype}{\color{white} \textbf{\textsf{PARAMETER}}}] \textbf{\underline{B}} the matrix of values for the solve
\item [\colorbox{tagtype}{\color{white} \textbf{\textsf{RETURN}}}] \textbf{\underline{}} the matrix of coefficients to get B.
\end{description}

\rule{\linewidth}{0.5pt}
\subsection*{\textsf{\colorbox{headtoc}{\color{white} FUNCTION}
extract\_diag}}

\hypertarget{ecldoc:blas.extract_diag}{}
\hspace{0pt} \hyperlink{ecldoc:blas}{BLAS} \textbackslash 

{\renewcommand{\arraystretch}{1.5}
\begin{tabularx}{\textwidth}{|>{\raggedright\arraybackslash}l|X|}
\hline
\hspace{0pt}\mytexttt{\color{red} Types.matrix\_t} & \textbf{extract\_diag} \\
\hline
\multicolumn{2}{|>{\raggedright\arraybackslash}X|}{\hspace{0pt}\mytexttt{\color{param} (Types.dimension\_t m, Types.dimension\_t n, Types.matrix\_t x)}} \\
\hline
\end{tabularx}
}

\par
Extract the diagonal of he matrix

\par
\begin{description}
\item [\colorbox{tagtype}{\color{white} \textbf{\textsf{PARAMETER}}}] \textbf{\underline{m}} number of rows
\item [\colorbox{tagtype}{\color{white} \textbf{\textsf{PARAMETER}}}] \textbf{\underline{n}} number of columns
\item [\colorbox{tagtype}{\color{white} \textbf{\textsf{PARAMETER}}}] \textbf{\underline{x}} matrix from which to extract the diagonal
\item [\colorbox{tagtype}{\color{white} \textbf{\textsf{RETURN}}}] \textbf{\underline{}} diagonal matrix
\end{description}

\rule{\linewidth}{0.5pt}
\subsection*{\textsf{\colorbox{headtoc}{\color{white} FUNCTION}
extract\_tri}}

\hypertarget{ecldoc:blas.extract_tri}{}
\hspace{0pt} \hyperlink{ecldoc:blas}{BLAS} \textbackslash 

{\renewcommand{\arraystretch}{1.5}
\begin{tabularx}{\textwidth}{|>{\raggedright\arraybackslash}l|X|}
\hline
\hspace{0pt}\mytexttt{\color{red} Types.matrix\_t} & \textbf{extract\_tri} \\
\hline
\multicolumn{2}{|>{\raggedright\arraybackslash}X|}{\hspace{0pt}\mytexttt{\color{param} (Types.dimension\_t m, Types.dimension\_t n, Types.Triangle tri, Types.Diagonal dt, Types.matrix\_t a)}} \\
\hline
\end{tabularx}
}

\par
Extract the upper or lower triangle. Diagonal can be actual or implied unit diagonal.

\par
\begin{description}
\item [\colorbox{tagtype}{\color{white} \textbf{\textsf{PARAMETER}}}] \textbf{\underline{m}} number of rows
\item [\colorbox{tagtype}{\color{white} \textbf{\textsf{PARAMETER}}}] \textbf{\underline{n}} number of columns
\item [\colorbox{tagtype}{\color{white} \textbf{\textsf{PARAMETER}}}] \textbf{\underline{tri}} Upper or Lower specifier, Triangle.Lower or Triangle.Upper
\item [\colorbox{tagtype}{\color{white} \textbf{\textsf{PARAMETER}}}] \textbf{\underline{dt}} Use Diagonal.NotUnitTri or Diagonal.UnitTri
\item [\colorbox{tagtype}{\color{white} \textbf{\textsf{PARAMETER}}}] \textbf{\underline{a}} Matrix, usually a composite from factoring
\item [\colorbox{tagtype}{\color{white} \textbf{\textsf{RETURN}}}] \textbf{\underline{}} the triangle
\end{description}

\rule{\linewidth}{0.5pt}
\subsection*{\textsf{\colorbox{headtoc}{\color{white} FUNCTION}
make\_diag}}

\hypertarget{ecldoc:blas.make_diag}{}
\hspace{0pt} \hyperlink{ecldoc:blas}{BLAS} \textbackslash 

{\renewcommand{\arraystretch}{1.5}
\begin{tabularx}{\textwidth}{|>{\raggedright\arraybackslash}l|X|}
\hline
\hspace{0pt}\mytexttt{\color{red} Types.matrix\_t} & \textbf{make\_diag} \\
\hline
\multicolumn{2}{|>{\raggedright\arraybackslash}X|}{\hspace{0pt}\mytexttt{\color{param} (Types.dimension\_t m, Types.value\_t v=1.0, Types.matrix\_t X=[])}} \\
\hline
\end{tabularx}
}

\par
Generate a diagonal matrix.

\par
\begin{description}
\item [\colorbox{tagtype}{\color{white} \textbf{\textsf{PARAMETER}}}] \textbf{\underline{m}} number of diagonal entries
\item [\colorbox{tagtype}{\color{white} \textbf{\textsf{PARAMETER}}}] \textbf{\underline{v}} option value, defaults to 1
\item [\colorbox{tagtype}{\color{white} \textbf{\textsf{PARAMETER}}}] \textbf{\underline{X}} optional input of diagonal values, multiplied by v.
\item [\colorbox{tagtype}{\color{white} \textbf{\textsf{RETURN}}}] \textbf{\underline{}} a diagonal matrix
\end{description}

\rule{\linewidth}{0.5pt}
\subsection*{\textsf{\colorbox{headtoc}{\color{white} FUNCTION}
make\_vector}}

\hypertarget{ecldoc:blas.make_vector}{}
\hspace{0pt} \hyperlink{ecldoc:blas}{BLAS} \textbackslash 

{\renewcommand{\arraystretch}{1.5}
\begin{tabularx}{\textwidth}{|>{\raggedright\arraybackslash}l|X|}
\hline
\hspace{0pt}\mytexttt{\color{red} Types.matrix\_t} & \textbf{make\_vector} \\
\hline
\multicolumn{2}{|>{\raggedright\arraybackslash}X|}{\hspace{0pt}\mytexttt{\color{param} (Types.dimension\_t m, Types.value\_t v=1.0)}} \\
\hline
\end{tabularx}
}

\par
Make a vector of dimension m

\par
\begin{description}
\item [\colorbox{tagtype}{\color{white} \textbf{\textsf{PARAMETER}}}] \textbf{\underline{m}} number of elements
\item [\colorbox{tagtype}{\color{white} \textbf{\textsf{PARAMETER}}}] \textbf{\underline{v}} the values, defaults to 1
\item [\colorbox{tagtype}{\color{white} \textbf{\textsf{RETURN}}}] \textbf{\underline{}} the vector
\end{description}

\rule{\linewidth}{0.5pt}
\subsection*{\textsf{\colorbox{headtoc}{\color{white} FUNCTION}
trace}}

\hypertarget{ecldoc:blas.trace}{}
\hspace{0pt} \hyperlink{ecldoc:blas}{BLAS} \textbackslash 

{\renewcommand{\arraystretch}{1.5}
\begin{tabularx}{\textwidth}{|>{\raggedright\arraybackslash}l|X|}
\hline
\hspace{0pt}\mytexttt{\color{red} Types.value\_t} & \textbf{trace} \\
\hline
\multicolumn{2}{|>{\raggedright\arraybackslash}X|}{\hspace{0pt}\mytexttt{\color{param} (Types.dimension\_t m, Types.dimension\_t n, Types.matrix\_t x)}} \\
\hline
\end{tabularx}
}

\par
The trace of the input matrix

\par
\begin{description}
\item [\colorbox{tagtype}{\color{white} \textbf{\textsf{PARAMETER}}}] \textbf{\underline{m}} number of rows
\item [\colorbox{tagtype}{\color{white} \textbf{\textsf{PARAMETER}}}] \textbf{\underline{n}} number of columns
\item [\colorbox{tagtype}{\color{white} \textbf{\textsf{PARAMETER}}}] \textbf{\underline{x}} the matrix
\item [\colorbox{tagtype}{\color{white} \textbf{\textsf{RETURN}}}] \textbf{\underline{}} the trace (sum of the diagonal entries)
\end{description}

\rule{\linewidth}{0.5pt}



\chapter*{\color{headfile}
BundleBase
}
\hypertarget{ecldoc:toc:BundleBase}{}
\hyperlink{ecldoc:toc:root}{Go Up}


\section*{\underline{\textsf{DESCRIPTIONS}}}
\subsection*{\textsf{\colorbox{headtoc}{\color{white} MODULE}
BundleBase}}

\hypertarget{ecldoc:BundleBase}{}

{\renewcommand{\arraystretch}{1.5}
\begin{tabularx}{\textwidth}{|>{\raggedright\arraybackslash}l|X|}
\hline
\hspace{0pt}\mytexttt{\color{red} } & \textbf{BundleBase} \\
\hline
\end{tabularx}
}

\par





No Documentation Found







\textbf{Children}
\begin{enumerate}
\item \hyperlink{ecldoc:bundlebase.propertyrecord}{PropertyRecord}
: No Documentation Found
\item \hyperlink{ecldoc:bundlebase.name}{Name}
: No Documentation Found
\item \hyperlink{ecldoc:bundlebase.description}{Description}
: No Documentation Found
\item \hyperlink{ecldoc:bundlebase.authors}{Authors}
: No Documentation Found
\item \hyperlink{ecldoc:bundlebase.license}{License}
: No Documentation Found
\item \hyperlink{ecldoc:bundlebase.copyright}{Copyright}
: No Documentation Found
\item \hyperlink{ecldoc:bundlebase.dependson}{DependsOn}
: No Documentation Found
\item \hyperlink{ecldoc:bundlebase.version}{Version}
: No Documentation Found
\item \hyperlink{ecldoc:bundlebase.properties}{Properties}
: No Documentation Found
\item \hyperlink{ecldoc:bundlebase.platformversion}{PlatformVersion}
: No Documentation Found
\end{enumerate}

\rule{\linewidth}{0.5pt}

\subsection*{\textsf{\colorbox{headtoc}{\color{white} RECORD}
PropertyRecord}}

\hypertarget{ecldoc:bundlebase.propertyrecord}{}
\hspace{0pt} \hyperlink{ecldoc:BundleBase}{BundleBase} \textbackslash 

{\renewcommand{\arraystretch}{1.5}
\begin{tabularx}{\textwidth}{|>{\raggedright\arraybackslash}l|X|}
\hline
\hspace{0pt}\mytexttt{\color{red} } & \textbf{PropertyRecord} \\
\hline
\end{tabularx}
}

\par





No Documentation Found







\par
\begin{description}
\item [\colorbox{tagtype}{\color{white} \textbf{\textsf{FIELD}}}] \textbf{\underline{value}} ||| UTF8 --- No Doc
\item [\colorbox{tagtype}{\color{white} \textbf{\textsf{FIELD}}}] \textbf{\underline{key}} ||| UTF8 --- No Doc
\end{description}





\rule{\linewidth}{0.5pt}
\subsection*{\textsf{\colorbox{headtoc}{\color{white} ATTRIBUTE}
Name}}

\hypertarget{ecldoc:bundlebase.name}{}
\hspace{0pt} \hyperlink{ecldoc:BundleBase}{BundleBase} \textbackslash 

{\renewcommand{\arraystretch}{1.5}
\begin{tabularx}{\textwidth}{|>{\raggedright\arraybackslash}l|X|}
\hline
\hspace{0pt}\mytexttt{\color{red} STRING} & \textbf{Name} \\
\hline
\end{tabularx}
}

\par





No Documentation Found








\par
\begin{description}
\item [\colorbox{tagtype}{\color{white} \textbf{\textsf{RETURN}}}] \textbf{STRING} --- 
\end{description}




\rule{\linewidth}{0.5pt}
\subsection*{\textsf{\colorbox{headtoc}{\color{white} ATTRIBUTE}
Description}}

\hypertarget{ecldoc:bundlebase.description}{}
\hspace{0pt} \hyperlink{ecldoc:BundleBase}{BundleBase} \textbackslash 

{\renewcommand{\arraystretch}{1.5}
\begin{tabularx}{\textwidth}{|>{\raggedright\arraybackslash}l|X|}
\hline
\hspace{0pt}\mytexttt{\color{red} UTF8} & \textbf{Description} \\
\hline
\end{tabularx}
}

\par





No Documentation Found








\par
\begin{description}
\item [\colorbox{tagtype}{\color{white} \textbf{\textsf{RETURN}}}] \textbf{UTF8} --- 
\end{description}




\rule{\linewidth}{0.5pt}
\subsection*{\textsf{\colorbox{headtoc}{\color{white} ATTRIBUTE}
Authors}}

\hypertarget{ecldoc:bundlebase.authors}{}
\hspace{0pt} \hyperlink{ecldoc:BundleBase}{BundleBase} \textbackslash 

{\renewcommand{\arraystretch}{1.5}
\begin{tabularx}{\textwidth}{|>{\raggedright\arraybackslash}l|X|}
\hline
\hspace{0pt}\mytexttt{\color{red} SET OF UTF8} & \textbf{Authors} \\
\hline
\end{tabularx}
}

\par





No Documentation Found








\par
\begin{description}
\item [\colorbox{tagtype}{\color{white} \textbf{\textsf{RETURN}}}] \textbf{SET ( UTF8 )} --- 
\end{description}




\rule{\linewidth}{0.5pt}
\subsection*{\textsf{\colorbox{headtoc}{\color{white} ATTRIBUTE}
License}}

\hypertarget{ecldoc:bundlebase.license}{}
\hspace{0pt} \hyperlink{ecldoc:BundleBase}{BundleBase} \textbackslash 

{\renewcommand{\arraystretch}{1.5}
\begin{tabularx}{\textwidth}{|>{\raggedright\arraybackslash}l|X|}
\hline
\hspace{0pt}\mytexttt{\color{red} UTF8} & \textbf{License} \\
\hline
\end{tabularx}
}

\par





No Documentation Found








\par
\begin{description}
\item [\colorbox{tagtype}{\color{white} \textbf{\textsf{RETURN}}}] \textbf{UTF8} --- 
\end{description}




\rule{\linewidth}{0.5pt}
\subsection*{\textsf{\colorbox{headtoc}{\color{white} ATTRIBUTE}
Copyright}}

\hypertarget{ecldoc:bundlebase.copyright}{}
\hspace{0pt} \hyperlink{ecldoc:BundleBase}{BundleBase} \textbackslash 

{\renewcommand{\arraystretch}{1.5}
\begin{tabularx}{\textwidth}{|>{\raggedright\arraybackslash}l|X|}
\hline
\hspace{0pt}\mytexttt{\color{red} UTF8} & \textbf{Copyright} \\
\hline
\end{tabularx}
}

\par





No Documentation Found








\par
\begin{description}
\item [\colorbox{tagtype}{\color{white} \textbf{\textsf{RETURN}}}] \textbf{UTF8} --- 
\end{description}




\rule{\linewidth}{0.5pt}
\subsection*{\textsf{\colorbox{headtoc}{\color{white} ATTRIBUTE}
DependsOn}}

\hypertarget{ecldoc:bundlebase.dependson}{}
\hspace{0pt} \hyperlink{ecldoc:BundleBase}{BundleBase} \textbackslash 

{\renewcommand{\arraystretch}{1.5}
\begin{tabularx}{\textwidth}{|>{\raggedright\arraybackslash}l|X|}
\hline
\hspace{0pt}\mytexttt{\color{red} SET OF STRING} & \textbf{DependsOn} \\
\hline
\end{tabularx}
}

\par





No Documentation Found








\par
\begin{description}
\item [\colorbox{tagtype}{\color{white} \textbf{\textsf{RETURN}}}] \textbf{SET ( STRING )} --- 
\end{description}




\rule{\linewidth}{0.5pt}
\subsection*{\textsf{\colorbox{headtoc}{\color{white} ATTRIBUTE}
Version}}

\hypertarget{ecldoc:bundlebase.version}{}
\hspace{0pt} \hyperlink{ecldoc:BundleBase}{BundleBase} \textbackslash 

{\renewcommand{\arraystretch}{1.5}
\begin{tabularx}{\textwidth}{|>{\raggedright\arraybackslash}l|X|}
\hline
\hspace{0pt}\mytexttt{\color{red} STRING} & \textbf{Version} \\
\hline
\end{tabularx}
}

\par





No Documentation Found








\par
\begin{description}
\item [\colorbox{tagtype}{\color{white} \textbf{\textsf{RETURN}}}] \textbf{STRING} --- 
\end{description}




\rule{\linewidth}{0.5pt}
\subsection*{\textsf{\colorbox{headtoc}{\color{white} ATTRIBUTE}
Properties}}

\hypertarget{ecldoc:bundlebase.properties}{}
\hspace{0pt} \hyperlink{ecldoc:BundleBase}{BundleBase} \textbackslash 

{\renewcommand{\arraystretch}{1.5}
\begin{tabularx}{\textwidth}{|>{\raggedright\arraybackslash}l|X|}
\hline
\hspace{0pt}\mytexttt{\color{red} } & \textbf{Properties} \\
\hline
\end{tabularx}
}

\par





No Documentation Found








\par
\begin{description}
\item [\colorbox{tagtype}{\color{white} \textbf{\textsf{RETURN}}}] \textbf{DICTIONARY ( PropertyRecord )} --- 
\end{description}




\rule{\linewidth}{0.5pt}
\subsection*{\textsf{\colorbox{headtoc}{\color{white} ATTRIBUTE}
PlatformVersion}}

\hypertarget{ecldoc:bundlebase.platformversion}{}
\hspace{0pt} \hyperlink{ecldoc:BundleBase}{BundleBase} \textbackslash 

{\renewcommand{\arraystretch}{1.5}
\begin{tabularx}{\textwidth}{|>{\raggedright\arraybackslash}l|X|}
\hline
\hspace{0pt}\mytexttt{\color{red} STRING} & \textbf{PlatformVersion} \\
\hline
\end{tabularx}
}

\par





No Documentation Found








\par
\begin{description}
\item [\colorbox{tagtype}{\color{white} \textbf{\textsf{RETURN}}}] \textbf{STRING} --- 
\end{description}




\rule{\linewidth}{0.5pt}



\chapter*{\color{headfile}
Date
}
\hypertarget{ecldoc:toc:Date}{}
\hyperlink{ecldoc:toc:root}{Go Up}

\section*{\underline{\textsf{IMPORTS}}}
\begin{doublespace}
{\large
}
\end{doublespace}

\section*{\underline{\textsf{DESCRIPTIONS}}}
\subsection*{\textsf{\colorbox{headtoc}{\color{white} MODULE}
Date}}

\hypertarget{ecldoc:Date}{}

{\renewcommand{\arraystretch}{1.5}
\begin{tabularx}{\textwidth}{|>{\raggedright\arraybackslash}l|X|}
\hline
\hspace{0pt}\mytexttt{\color{red} } & \textbf{Date} \\
\hline
\end{tabularx}
}

\par





No Documentation Found







\textbf{Children}
\begin{enumerate}
\item \hyperlink{ecldoc:date.date_rec}{Date\_rec}
: No Documentation Found
\item \hyperlink{ecldoc:date.date_t}{Date\_t}
: No Documentation Found
\item \hyperlink{ecldoc:date.days_t}{Days\_t}
: No Documentation Found
\item \hyperlink{ecldoc:date.time_rec}{Time\_rec}
: No Documentation Found
\item \hyperlink{ecldoc:date.time_t}{Time\_t}
: No Documentation Found
\item \hyperlink{ecldoc:date.seconds_t}{Seconds\_t}
: No Documentation Found
\item \hyperlink{ecldoc:date.datetime_rec}{DateTime\_rec}
: No Documentation Found
\item \hyperlink{ecldoc:date.timestamp_t}{Timestamp\_t}
: No Documentation Found
\item \hyperlink{ecldoc:date.year}{Year}
: Extracts the year from a date type
\item \hyperlink{ecldoc:date.month}{Month}
: Extracts the month from a date type
\item \hyperlink{ecldoc:date.day}{Day}
: Extracts the day of the month from a date type
\item \hyperlink{ecldoc:date.hour}{Hour}
: Extracts the hour from a time type
\item \hyperlink{ecldoc:date.minute}{Minute}
: Extracts the minutes from a time type
\item \hyperlink{ecldoc:date.second}{Second}
: Extracts the seconds from a time type
\item \hyperlink{ecldoc:date.datefromparts}{DateFromParts}
: Combines year, month day to create a date type
\item \hyperlink{ecldoc:date.timefromparts}{TimeFromParts}
: Combines hour, minute second to create a time type
\item \hyperlink{ecldoc:date.secondsfromparts}{SecondsFromParts}
: Combines date and time components to create a seconds type
\item \hyperlink{ecldoc:date.secondstoparts}{SecondsToParts}
: Converts the number of seconds since epoch to a structure containing date and time parts
\item \hyperlink{ecldoc:date.timestamptoseconds}{TimestampToSeconds}
: Converts the number of microseconds since epoch to the number of seconds since epoch
\item \hyperlink{ecldoc:date.isleapyear}{IsLeapYear}
: Tests whether the year is a leap year in the Gregorian calendar
\item \hyperlink{ecldoc:date.isdateleapyear}{IsDateLeapYear}
: Tests whether a date is a leap year in the Gregorian calendar
\item \hyperlink{ecldoc:date.fromgregorianymd}{FromGregorianYMD}
: Combines year, month, day in the Gregorian calendar to create the number days since 31st December 1BC
\item \hyperlink{ecldoc:date.togregorianymd}{ToGregorianYMD}
: Converts the number days since 31st December 1BC to a date in the Gregorian calendar
\item \hyperlink{ecldoc:date.fromgregoriandate}{FromGregorianDate}
: Converts a date in the Gregorian calendar to the number days since 31st December 1BC
\item \hyperlink{ecldoc:date.togregoriandate}{ToGregorianDate}
: Converts the number days since 31st December 1BC to a date in the Gregorian calendar
\item \hyperlink{ecldoc:date.dayofyear}{DayOfYear}
: Returns a number representing the day of the year indicated by the given date
\item \hyperlink{ecldoc:date.dayofweek}{DayOfWeek}
: Returns a number representing the day of the week indicated by the given date
\item \hyperlink{ecldoc:date.isjulianleapyear}{IsJulianLeapYear}
: Tests whether the year is a leap year in the Julian calendar
\item \hyperlink{ecldoc:date.fromjulianymd}{FromJulianYMD}
: Combines year, month, day in the Julian calendar to create the number days since 31st December 1BC
\item \hyperlink{ecldoc:date.tojulianymd}{ToJulianYMD}
: Converts the number days since 31st December 1BC to a date in the Julian calendar
\item \hyperlink{ecldoc:date.fromjuliandate}{FromJulianDate}
: Converts a date in the Julian calendar to the number days since 31st December 1BC
\item \hyperlink{ecldoc:date.tojuliandate}{ToJulianDate}
: Converts the number days since 31st December 1BC to a date in the Julian calendar
\item \hyperlink{ecldoc:date.dayssince1900}{DaysSince1900}
: Returns the number of days since 1st January 1900 (using the Gregorian Calendar)
\item \hyperlink{ecldoc:date.todayssince1900}{ToDaysSince1900}
: Returns the number of days since 1st January 1900 (using the Gregorian Calendar)
\item \hyperlink{ecldoc:date.fromdayssince1900}{FromDaysSince1900}
: Converts the number days since 1st January 1900 to a date in the Julian calendar
\item \hyperlink{ecldoc:date.yearsbetween}{YearsBetween}
: Calculate the number of whole years between two dates
\item \hyperlink{ecldoc:date.monthsbetween}{MonthsBetween}
: Calculate the number of whole months between two dates
\item \hyperlink{ecldoc:date.daysbetween}{DaysBetween}
: Calculate the number of days between two dates
\item \hyperlink{ecldoc:date.datefromdaterec}{DateFromDateRec}
: Combines the fields from a Date\_rec to create a Date\_t
\item \hyperlink{ecldoc:date.datefromrec}{DateFromRec}
: Combines the fields from a Date\_rec to create a Date\_t
\item \hyperlink{ecldoc:date.timefromtimerec}{TimeFromTimeRec}
: Combines the fields from a Time\_rec to create a Time\_t
\item \hyperlink{ecldoc:date.datefromdatetimerec}{DateFromDateTimeRec}
: Combines the date fields from a DateTime\_rec to create a Date\_t
\item \hyperlink{ecldoc:date.timefromdatetimerec}{TimeFromDateTimeRec}
: Combines the time fields from a DateTime\_rec to create a Time\_t
\item \hyperlink{ecldoc:date.secondsfromdatetimerec}{SecondsFromDateTimeRec}
: Combines the date and time fields from a DateTime\_rec to create a Seconds\_t
\item \hyperlink{ecldoc:date.fromstringtodate}{FromStringToDate}
: Converts a string to a Date\_t using the relevant string format
\item \hyperlink{ecldoc:date.fromstring}{FromString}
: Converts a string to a date using the relevant string format
\item \hyperlink{ecldoc:date.fromstringtotime}{FromStringToTime}
: Converts a string to a Time\_t using the relevant string format
\item \hyperlink{ecldoc:date.matchdatestring}{MatchDateString}
: Matches a string against a set of date string formats and returns a valid Date\_t object from the first format that successfully parses the string
\item \hyperlink{ecldoc:date.matchtimestring}{MatchTimeString}
: Matches a string against a set of time string formats and returns a valid Time\_t object from the first format that successfully parses the string
\item \hyperlink{ecldoc:date.datetostring}{DateToString}
: Formats a date as a string
\item \hyperlink{ecldoc:date.timetostring}{TimeToString}
: Formats a time as a string
\item \hyperlink{ecldoc:date.secondstostring}{SecondsToString}
: Converts a Seconds\_t value into a human-readable string using a format template
\item \hyperlink{ecldoc:date.tostring}{ToString}
: Formats a date as a string
\item \hyperlink{ecldoc:date.convertdateformat}{ConvertDateFormat}
: Converts a date from one format to another
\item \hyperlink{ecldoc:date.convertformat}{ConvertFormat}
: Converts a date from one format to another
\item \hyperlink{ecldoc:date.converttimeformat}{ConvertTimeFormat}
: Converts a time from one format to another
\item \hyperlink{ecldoc:date.convertdateformatmultiple}{ConvertDateFormatMultiple}
: Converts a date that matches one of a set of formats to another
\item \hyperlink{ecldoc:date.convertformatmultiple}{ConvertFormatMultiple}
: Converts a date that matches one of a set of formats to another
\item \hyperlink{ecldoc:date.converttimeformatmultiple}{ConvertTimeFormatMultiple}
: Converts a time that matches one of a set of formats to another
\item \hyperlink{ecldoc:date.adjustdate}{AdjustDate}
: Adjusts a date by incrementing or decrementing year, month and/or day values
\item \hyperlink{ecldoc:date.adjustdatebyseconds}{AdjustDateBySeconds}
: Adjusts a date by adding or subtracting seconds
\item \hyperlink{ecldoc:date.adjusttime}{AdjustTime}
: Adjusts a time by incrementing or decrementing hour, minute and/or second values
\item \hyperlink{ecldoc:date.adjusttimebyseconds}{AdjustTimeBySeconds}
: Adjusts a time by adding or subtracting seconds
\item \hyperlink{ecldoc:date.adjustseconds}{AdjustSeconds}
: Adjusts a Seconds\_t value by adding or subtracting years, months, days, hours, minutes and/or seconds
\item \hyperlink{ecldoc:date.adjustcalendar}{AdjustCalendar}
: Adjusts a date by incrementing or decrementing months and/or years
\item \hyperlink{ecldoc:date.islocaldaylightsavingsineffect}{IsLocalDaylightSavingsInEffect}
: Returns a boolean indicating whether daylight savings time is currently in effect locally
\item \hyperlink{ecldoc:date.localtimezoneoffset}{LocalTimeZoneOffset}
: Returns the offset (in seconds) of the time represented from UTC, with positive values indicating locations east of the Prime Meridian
\item \hyperlink{ecldoc:date.currentdate}{CurrentDate}
: Returns the current date
\item \hyperlink{ecldoc:date.today}{Today}
: Returns the current date in the local time zone
\item \hyperlink{ecldoc:date.currenttime}{CurrentTime}
: Returns the current time of day
\item \hyperlink{ecldoc:date.currentseconds}{CurrentSeconds}
: Returns the current date and time as the number of seconds since epoch
\item \hyperlink{ecldoc:date.currenttimestamp}{CurrentTimestamp}
: Returns the current date and time as the number of microseconds since epoch
\item \hyperlink{ecldoc:date.datesformonth}{DatesForMonth}
: Returns the beginning and ending dates for the month surrounding the given date
\item \hyperlink{ecldoc:date.datesforweek}{DatesForWeek}
: Returns the beginning and ending dates for the week surrounding the given date (Sunday marks the beginning of a week)
\item \hyperlink{ecldoc:date.isvaliddate}{IsValidDate}
: Tests whether a date is valid, both by range-checking the year and by validating each of the other individual components
\item \hyperlink{ecldoc:date.isvalidgregoriandate}{IsValidGregorianDate}
: Tests whether a date is valid in the Gregorian calendar
\item \hyperlink{ecldoc:date.isvalidtime}{IsValidTime}
: Tests whether a time is valid
\item \hyperlink{ecldoc:date.createdate}{CreateDate}
: A transform to create a Date\_rec from the individual elements
\item \hyperlink{ecldoc:date.createdatefromseconds}{CreateDateFromSeconds}
: A transform to create a Date\_rec from a Seconds\_t value
\item \hyperlink{ecldoc:date.createtime}{CreateTime}
: A transform to create a Time\_rec from the individual elements
\item \hyperlink{ecldoc:date.createtimefromseconds}{CreateTimeFromSeconds}
: A transform to create a Time\_rec from a Seconds\_t value
\item \hyperlink{ecldoc:date.createdatetime}{CreateDateTime}
: A transform to create a DateTime\_rec from the individual elements
\item \hyperlink{ecldoc:date.createdatetimefromseconds}{CreateDateTimeFromSeconds}
: A transform to create a DateTime\_rec from a Seconds\_t value
\end{enumerate}

\rule{\linewidth}{0.5pt}

\subsection*{\textsf{\colorbox{headtoc}{\color{white} RECORD}
Date\_rec}}

\hypertarget{ecldoc:date.date_rec}{}
\hspace{0pt} \hyperlink{ecldoc:Date}{Date} \textbackslash 

{\renewcommand{\arraystretch}{1.5}
\begin{tabularx}{\textwidth}{|>{\raggedright\arraybackslash}l|X|}
\hline
\hspace{0pt}\mytexttt{\color{red} } & \textbf{Date\_rec} \\
\hline
\end{tabularx}
}

\par





No Documentation Found







\par
\begin{description}
\item [\colorbox{tagtype}{\color{white} \textbf{\textsf{FIELD}}}] \textbf{\underline{year}} ||| INTEGER2 --- No Doc
\item [\colorbox{tagtype}{\color{white} \textbf{\textsf{FIELD}}}] \textbf{\underline{month}} ||| UNSIGNED1 --- No Doc
\item [\colorbox{tagtype}{\color{white} \textbf{\textsf{FIELD}}}] \textbf{\underline{day}} ||| UNSIGNED1 --- No Doc
\end{description}





\rule{\linewidth}{0.5pt}
\subsection*{\textsf{\colorbox{headtoc}{\color{white} ATTRIBUTE}
Date\_t}}

\hypertarget{ecldoc:date.date_t}{}
\hspace{0pt} \hyperlink{ecldoc:Date}{Date} \textbackslash 

{\renewcommand{\arraystretch}{1.5}
\begin{tabularx}{\textwidth}{|>{\raggedright\arraybackslash}l|X|}
\hline
\hspace{0pt}\mytexttt{\color{red} } & \textbf{Date\_t} \\
\hline
\end{tabularx}
}

\par





No Documentation Found








\par
\begin{description}
\item [\colorbox{tagtype}{\color{white} \textbf{\textsf{RETURN}}}] \textbf{UNSIGNED4} --- 
\end{description}




\rule{\linewidth}{0.5pt}
\subsection*{\textsf{\colorbox{headtoc}{\color{white} ATTRIBUTE}
Days\_t}}

\hypertarget{ecldoc:date.days_t}{}
\hspace{0pt} \hyperlink{ecldoc:Date}{Date} \textbackslash 

{\renewcommand{\arraystretch}{1.5}
\begin{tabularx}{\textwidth}{|>{\raggedright\arraybackslash}l|X|}
\hline
\hspace{0pt}\mytexttt{\color{red} } & \textbf{Days\_t} \\
\hline
\end{tabularx}
}

\par





No Documentation Found








\par
\begin{description}
\item [\colorbox{tagtype}{\color{white} \textbf{\textsf{RETURN}}}] \textbf{INTEGER4} --- 
\end{description}




\rule{\linewidth}{0.5pt}
\subsection*{\textsf{\colorbox{headtoc}{\color{white} RECORD}
Time\_rec}}

\hypertarget{ecldoc:date.time_rec}{}
\hspace{0pt} \hyperlink{ecldoc:Date}{Date} \textbackslash 

{\renewcommand{\arraystretch}{1.5}
\begin{tabularx}{\textwidth}{|>{\raggedright\arraybackslash}l|X|}
\hline
\hspace{0pt}\mytexttt{\color{red} } & \textbf{Time\_rec} \\
\hline
\end{tabularx}
}

\par





No Documentation Found







\par
\begin{description}
\item [\colorbox{tagtype}{\color{white} \textbf{\textsf{FIELD}}}] \textbf{\underline{minute}} ||| UNSIGNED1 --- No Doc
\item [\colorbox{tagtype}{\color{white} \textbf{\textsf{FIELD}}}] \textbf{\underline{second}} ||| UNSIGNED1 --- No Doc
\item [\colorbox{tagtype}{\color{white} \textbf{\textsf{FIELD}}}] \textbf{\underline{hour}} ||| UNSIGNED1 --- No Doc
\end{description}





\rule{\linewidth}{0.5pt}
\subsection*{\textsf{\colorbox{headtoc}{\color{white} ATTRIBUTE}
Time\_t}}

\hypertarget{ecldoc:date.time_t}{}
\hspace{0pt} \hyperlink{ecldoc:Date}{Date} \textbackslash 

{\renewcommand{\arraystretch}{1.5}
\begin{tabularx}{\textwidth}{|>{\raggedright\arraybackslash}l|X|}
\hline
\hspace{0pt}\mytexttt{\color{red} } & \textbf{Time\_t} \\
\hline
\end{tabularx}
}

\par





No Documentation Found








\par
\begin{description}
\item [\colorbox{tagtype}{\color{white} \textbf{\textsf{RETURN}}}] \textbf{UNSIGNED3} --- 
\end{description}




\rule{\linewidth}{0.5pt}
\subsection*{\textsf{\colorbox{headtoc}{\color{white} ATTRIBUTE}
Seconds\_t}}

\hypertarget{ecldoc:date.seconds_t}{}
\hspace{0pt} \hyperlink{ecldoc:Date}{Date} \textbackslash 

{\renewcommand{\arraystretch}{1.5}
\begin{tabularx}{\textwidth}{|>{\raggedright\arraybackslash}l|X|}
\hline
\hspace{0pt}\mytexttt{\color{red} } & \textbf{Seconds\_t} \\
\hline
\end{tabularx}
}

\par





No Documentation Found








\par
\begin{description}
\item [\colorbox{tagtype}{\color{white} \textbf{\textsf{RETURN}}}] \textbf{INTEGER8} --- 
\end{description}




\rule{\linewidth}{0.5pt}
\subsection*{\textsf{\colorbox{headtoc}{\color{white} RECORD}
DateTime\_rec}}

\hypertarget{ecldoc:date.datetime_rec}{}
\hspace{0pt} \hyperlink{ecldoc:Date}{Date} \textbackslash 

{\renewcommand{\arraystretch}{1.5}
\begin{tabularx}{\textwidth}{|>{\raggedright\arraybackslash}l|X|}
\hline
\hspace{0pt}\mytexttt{\color{red} } & \textbf{DateTime\_rec} \\
\hline
\end{tabularx}
}

\par





No Documentation Found







\par
\begin{description}
\item [\colorbox{tagtype}{\color{white} \textbf{\textsf{FIELD}}}] \textbf{\underline{year}} ||| INTEGER2 --- No Doc
\item [\colorbox{tagtype}{\color{white} \textbf{\textsf{FIELD}}}] \textbf{\underline{second}} ||| UNSIGNED1 --- No Doc
\item [\colorbox{tagtype}{\color{white} \textbf{\textsf{FIELD}}}] \textbf{\underline{hour}} ||| UNSIGNED1 --- No Doc
\item [\colorbox{tagtype}{\color{white} \textbf{\textsf{FIELD}}}] \textbf{\underline{minute}} ||| UNSIGNED1 --- No Doc
\item [\colorbox{tagtype}{\color{white} \textbf{\textsf{FIELD}}}] \textbf{\underline{month}} ||| UNSIGNED1 --- No Doc
\item [\colorbox{tagtype}{\color{white} \textbf{\textsf{FIELD}}}] \textbf{\underline{day}} ||| UNSIGNED1 --- No Doc
\end{description}





\rule{\linewidth}{0.5pt}
\subsection*{\textsf{\colorbox{headtoc}{\color{white} ATTRIBUTE}
Timestamp\_t}}

\hypertarget{ecldoc:date.timestamp_t}{}
\hspace{0pt} \hyperlink{ecldoc:Date}{Date} \textbackslash 

{\renewcommand{\arraystretch}{1.5}
\begin{tabularx}{\textwidth}{|>{\raggedright\arraybackslash}l|X|}
\hline
\hspace{0pt}\mytexttt{\color{red} } & \textbf{Timestamp\_t} \\
\hline
\end{tabularx}
}

\par





No Documentation Found








\par
\begin{description}
\item [\colorbox{tagtype}{\color{white} \textbf{\textsf{RETURN}}}] \textbf{INTEGER8} --- 
\end{description}




\rule{\linewidth}{0.5pt}
\subsection*{\textsf{\colorbox{headtoc}{\color{white} FUNCTION}
Year}}

\hypertarget{ecldoc:date.year}{}
\hspace{0pt} \hyperlink{ecldoc:Date}{Date} \textbackslash 

{\renewcommand{\arraystretch}{1.5}
\begin{tabularx}{\textwidth}{|>{\raggedright\arraybackslash}l|X|}
\hline
\hspace{0pt}\mytexttt{\color{red} INTEGER2} & \textbf{Year} \\
\hline
\multicolumn{2}{|>{\raggedright\arraybackslash}X|}{\hspace{0pt}\mytexttt{\color{param} (Date\_t date)}} \\
\hline
\end{tabularx}
}

\par





Extracts the year from a date type.






\par
\begin{description}
\item [\colorbox{tagtype}{\color{white} \textbf{\textsf{PARAMETER}}}] \textbf{\underline{date}} ||| UNSIGNED4 --- The date.
\end{description}







\par
\begin{description}
\item [\colorbox{tagtype}{\color{white} \textbf{\textsf{RETURN}}}] \textbf{INTEGER2} --- An integer representing the year.
\end{description}




\rule{\linewidth}{0.5pt}
\subsection*{\textsf{\colorbox{headtoc}{\color{white} FUNCTION}
Month}}

\hypertarget{ecldoc:date.month}{}
\hspace{0pt} \hyperlink{ecldoc:Date}{Date} \textbackslash 

{\renewcommand{\arraystretch}{1.5}
\begin{tabularx}{\textwidth}{|>{\raggedright\arraybackslash}l|X|}
\hline
\hspace{0pt}\mytexttt{\color{red} UNSIGNED1} & \textbf{Month} \\
\hline
\multicolumn{2}{|>{\raggedright\arraybackslash}X|}{\hspace{0pt}\mytexttt{\color{param} (Date\_t date)}} \\
\hline
\end{tabularx}
}

\par





Extracts the month from a date type.






\par
\begin{description}
\item [\colorbox{tagtype}{\color{white} \textbf{\textsf{PARAMETER}}}] \textbf{\underline{date}} ||| UNSIGNED4 --- The date.
\end{description}







\par
\begin{description}
\item [\colorbox{tagtype}{\color{white} \textbf{\textsf{RETURN}}}] \textbf{UNSIGNED1} --- An integer representing the year.
\end{description}




\rule{\linewidth}{0.5pt}
\subsection*{\textsf{\colorbox{headtoc}{\color{white} FUNCTION}
Day}}

\hypertarget{ecldoc:date.day}{}
\hspace{0pt} \hyperlink{ecldoc:Date}{Date} \textbackslash 

{\renewcommand{\arraystretch}{1.5}
\begin{tabularx}{\textwidth}{|>{\raggedright\arraybackslash}l|X|}
\hline
\hspace{0pt}\mytexttt{\color{red} UNSIGNED1} & \textbf{Day} \\
\hline
\multicolumn{2}{|>{\raggedright\arraybackslash}X|}{\hspace{0pt}\mytexttt{\color{param} (Date\_t date)}} \\
\hline
\end{tabularx}
}

\par





Extracts the day of the month from a date type.






\par
\begin{description}
\item [\colorbox{tagtype}{\color{white} \textbf{\textsf{PARAMETER}}}] \textbf{\underline{date}} ||| UNSIGNED4 --- The date.
\end{description}







\par
\begin{description}
\item [\colorbox{tagtype}{\color{white} \textbf{\textsf{RETURN}}}] \textbf{UNSIGNED1} --- An integer representing the year.
\end{description}




\rule{\linewidth}{0.5pt}
\subsection*{\textsf{\colorbox{headtoc}{\color{white} FUNCTION}
Hour}}

\hypertarget{ecldoc:date.hour}{}
\hspace{0pt} \hyperlink{ecldoc:Date}{Date} \textbackslash 

{\renewcommand{\arraystretch}{1.5}
\begin{tabularx}{\textwidth}{|>{\raggedright\arraybackslash}l|X|}
\hline
\hspace{0pt}\mytexttt{\color{red} UNSIGNED1} & \textbf{Hour} \\
\hline
\multicolumn{2}{|>{\raggedright\arraybackslash}X|}{\hspace{0pt}\mytexttt{\color{param} (Time\_t time)}} \\
\hline
\end{tabularx}
}

\par





Extracts the hour from a time type.






\par
\begin{description}
\item [\colorbox{tagtype}{\color{white} \textbf{\textsf{PARAMETER}}}] \textbf{\underline{time}} ||| UNSIGNED3 --- The time.
\end{description}







\par
\begin{description}
\item [\colorbox{tagtype}{\color{white} \textbf{\textsf{RETURN}}}] \textbf{UNSIGNED1} --- An integer representing the hour.
\end{description}




\rule{\linewidth}{0.5pt}
\subsection*{\textsf{\colorbox{headtoc}{\color{white} FUNCTION}
Minute}}

\hypertarget{ecldoc:date.minute}{}
\hspace{0pt} \hyperlink{ecldoc:Date}{Date} \textbackslash 

{\renewcommand{\arraystretch}{1.5}
\begin{tabularx}{\textwidth}{|>{\raggedright\arraybackslash}l|X|}
\hline
\hspace{0pt}\mytexttt{\color{red} UNSIGNED1} & \textbf{Minute} \\
\hline
\multicolumn{2}{|>{\raggedright\arraybackslash}X|}{\hspace{0pt}\mytexttt{\color{param} (Time\_t time)}} \\
\hline
\end{tabularx}
}

\par





Extracts the minutes from a time type.






\par
\begin{description}
\item [\colorbox{tagtype}{\color{white} \textbf{\textsf{PARAMETER}}}] \textbf{\underline{time}} ||| UNSIGNED3 --- The time.
\end{description}







\par
\begin{description}
\item [\colorbox{tagtype}{\color{white} \textbf{\textsf{RETURN}}}] \textbf{UNSIGNED1} --- An integer representing the minutes.
\end{description}




\rule{\linewidth}{0.5pt}
\subsection*{\textsf{\colorbox{headtoc}{\color{white} FUNCTION}
Second}}

\hypertarget{ecldoc:date.second}{}
\hspace{0pt} \hyperlink{ecldoc:Date}{Date} \textbackslash 

{\renewcommand{\arraystretch}{1.5}
\begin{tabularx}{\textwidth}{|>{\raggedright\arraybackslash}l|X|}
\hline
\hspace{0pt}\mytexttt{\color{red} UNSIGNED1} & \textbf{Second} \\
\hline
\multicolumn{2}{|>{\raggedright\arraybackslash}X|}{\hspace{0pt}\mytexttt{\color{param} (Time\_t time)}} \\
\hline
\end{tabularx}
}

\par





Extracts the seconds from a time type.






\par
\begin{description}
\item [\colorbox{tagtype}{\color{white} \textbf{\textsf{PARAMETER}}}] \textbf{\underline{time}} ||| UNSIGNED3 --- The time.
\end{description}







\par
\begin{description}
\item [\colorbox{tagtype}{\color{white} \textbf{\textsf{RETURN}}}] \textbf{UNSIGNED1} --- An integer representing the seconds.
\end{description}




\rule{\linewidth}{0.5pt}
\subsection*{\textsf{\colorbox{headtoc}{\color{white} FUNCTION}
DateFromParts}}

\hypertarget{ecldoc:date.datefromparts}{}
\hspace{0pt} \hyperlink{ecldoc:Date}{Date} \textbackslash 

{\renewcommand{\arraystretch}{1.5}
\begin{tabularx}{\textwidth}{|>{\raggedright\arraybackslash}l|X|}
\hline
\hspace{0pt}\mytexttt{\color{red} Date\_t} & \textbf{DateFromParts} \\
\hline
\multicolumn{2}{|>{\raggedright\arraybackslash}X|}{\hspace{0pt}\mytexttt{\color{param} (INTEGER2 year, UNSIGNED1 month, UNSIGNED1 day)}} \\
\hline
\end{tabularx}
}

\par





Combines year, month day to create a date type.






\par
\begin{description}
\item [\colorbox{tagtype}{\color{white} \textbf{\textsf{PARAMETER}}}] \textbf{\underline{year}} ||| INTEGER2 --- The year (0-9999).
\item [\colorbox{tagtype}{\color{white} \textbf{\textsf{PARAMETER}}}] \textbf{\underline{month}} ||| UNSIGNED1 --- The month (1-12).
\item [\colorbox{tagtype}{\color{white} \textbf{\textsf{PARAMETER}}}] \textbf{\underline{day}} ||| UNSIGNED1 --- The day (1..daysInMonth).
\end{description}







\par
\begin{description}
\item [\colorbox{tagtype}{\color{white} \textbf{\textsf{RETURN}}}] \textbf{UNSIGNED4} --- A date created by combining the fields.
\end{description}




\rule{\linewidth}{0.5pt}
\subsection*{\textsf{\colorbox{headtoc}{\color{white} FUNCTION}
TimeFromParts}}

\hypertarget{ecldoc:date.timefromparts}{}
\hspace{0pt} \hyperlink{ecldoc:Date}{Date} \textbackslash 

{\renewcommand{\arraystretch}{1.5}
\begin{tabularx}{\textwidth}{|>{\raggedright\arraybackslash}l|X|}
\hline
\hspace{0pt}\mytexttt{\color{red} Time\_t} & \textbf{TimeFromParts} \\
\hline
\multicolumn{2}{|>{\raggedright\arraybackslash}X|}{\hspace{0pt}\mytexttt{\color{param} (UNSIGNED1 hour, UNSIGNED1 minute, UNSIGNED1 second)}} \\
\hline
\end{tabularx}
}

\par





Combines hour, minute second to create a time type.






\par
\begin{description}
\item [\colorbox{tagtype}{\color{white} \textbf{\textsf{PARAMETER}}}] \textbf{\underline{minute}} ||| UNSIGNED1 --- The minute (0-59).
\item [\colorbox{tagtype}{\color{white} \textbf{\textsf{PARAMETER}}}] \textbf{\underline{second}} ||| UNSIGNED1 --- The second (0-59).
\item [\colorbox{tagtype}{\color{white} \textbf{\textsf{PARAMETER}}}] \textbf{\underline{hour}} ||| UNSIGNED1 --- The hour (0-23).
\end{description}







\par
\begin{description}
\item [\colorbox{tagtype}{\color{white} \textbf{\textsf{RETURN}}}] \textbf{UNSIGNED3} --- A time created by combining the fields.
\end{description}




\rule{\linewidth}{0.5pt}
\subsection*{\textsf{\colorbox{headtoc}{\color{white} FUNCTION}
SecondsFromParts}}

\hypertarget{ecldoc:date.secondsfromparts}{}
\hspace{0pt} \hyperlink{ecldoc:Date}{Date} \textbackslash 

{\renewcommand{\arraystretch}{1.5}
\begin{tabularx}{\textwidth}{|>{\raggedright\arraybackslash}l|X|}
\hline
\hspace{0pt}\mytexttt{\color{red} Seconds\_t} & \textbf{SecondsFromParts} \\
\hline
\multicolumn{2}{|>{\raggedright\arraybackslash}X|}{\hspace{0pt}\mytexttt{\color{param} (INTEGER2 year, UNSIGNED1 month, UNSIGNED1 day, UNSIGNED1 hour, UNSIGNED1 minute, UNSIGNED1 second, BOOLEAN is\_local\_time = FALSE)}} \\
\hline
\end{tabularx}
}

\par





Combines date and time components to create a seconds type. The date must be represented within the Gregorian calendar after the year 1600.






\par
\begin{description}
\item [\colorbox{tagtype}{\color{white} \textbf{\textsf{PARAMETER}}}] \textbf{\underline{year}} ||| INTEGER2 --- The year (1601-30827).
\item [\colorbox{tagtype}{\color{white} \textbf{\textsf{PARAMETER}}}] \textbf{\underline{second}} ||| UNSIGNED1 --- The second (0-59).
\item [\colorbox{tagtype}{\color{white} \textbf{\textsf{PARAMETER}}}] \textbf{\underline{hour}} ||| UNSIGNED1 --- The hour (0-23).
\item [\colorbox{tagtype}{\color{white} \textbf{\textsf{PARAMETER}}}] \textbf{\underline{minute}} ||| UNSIGNED1 --- The minute (0-59).
\item [\colorbox{tagtype}{\color{white} \textbf{\textsf{PARAMETER}}}] \textbf{\underline{month}} ||| UNSIGNED1 --- The month (1-12).
\item [\colorbox{tagtype}{\color{white} \textbf{\textsf{PARAMETER}}}] \textbf{\underline{day}} ||| UNSIGNED1 --- The day (1..daysInMonth).
\item [\colorbox{tagtype}{\color{white} \textbf{\textsf{PARAMETER}}}] \textbf{\underline{is\_local\_time}} ||| BOOLEAN --- TRUE if the datetime components are expressed in local time rather than UTC, FALSE if the components are expressed in UTC. Optional, defaults to FALSE.
\end{description}







\par
\begin{description}
\item [\colorbox{tagtype}{\color{white} \textbf{\textsf{RETURN}}}] \textbf{INTEGER8} --- A Seconds\_t value created by combining the fields.
\end{description}




\rule{\linewidth}{0.5pt}
\subsection*{\textsf{\colorbox{headtoc}{\color{white} MODULE}
SecondsToParts}}

\hypertarget{ecldoc:date.secondstoparts}{}
\hspace{0pt} \hyperlink{ecldoc:Date}{Date} \textbackslash 

{\renewcommand{\arraystretch}{1.5}
\begin{tabularx}{\textwidth}{|>{\raggedright\arraybackslash}l|X|}
\hline
\hspace{0pt}\mytexttt{\color{red} } & \textbf{SecondsToParts} \\
\hline
\multicolumn{2}{|>{\raggedright\arraybackslash}X|}{\hspace{0pt}\mytexttt{\color{param} (Seconds\_t seconds)}} \\
\hline
\end{tabularx}
}

\par





Converts the number of seconds since epoch to a structure containing date and time parts. The result must be representable within the Gregorian calendar after the year 1600.






\par
\begin{description}
\item [\colorbox{tagtype}{\color{white} \textbf{\textsf{PARAMETER}}}] \textbf{\underline{seconds}} ||| INTEGER8 --- The number of seconds since epoch.
\end{description}







\par
\begin{description}
\item [\colorbox{tagtype}{\color{white} \textbf{\textsf{RETURN}}}] \textbf{} --- Module with exported attributes for year, month, day, hour, minute, second, day\_of\_week, date and time.
\end{description}




\textbf{Children}
\begin{enumerate}
\item \hyperlink{ecldoc:date.secondstoparts.result.year}{Year}
: No Documentation Found
\item \hyperlink{ecldoc:date.secondstoparts.result.month}{Month}
: No Documentation Found
\item \hyperlink{ecldoc:date.secondstoparts.result.day}{Day}
: No Documentation Found
\item \hyperlink{ecldoc:date.secondstoparts.result.hour}{Hour}
: No Documentation Found
\item \hyperlink{ecldoc:date.secondstoparts.result.minute}{Minute}
: No Documentation Found
\item \hyperlink{ecldoc:date.secondstoparts.result.second}{Second}
: No Documentation Found
\item \hyperlink{ecldoc:date.secondstoparts.result.day_of_week}{day\_of\_week}
: No Documentation Found
\item \hyperlink{ecldoc:date.secondstoparts.result.date}{date}
: Combines year, month day to create a date type
\item \hyperlink{ecldoc:date.secondstoparts.result.time}{time}
: Combines hour, minute second to create a time type
\end{enumerate}

\rule{\linewidth}{0.5pt}

\subsection*{\textsf{\colorbox{headtoc}{\color{white} ATTRIBUTE}
Year}}

\hypertarget{ecldoc:date.secondstoparts.result.year}{}
\hspace{0pt} \hyperlink{ecldoc:Date}{Date} \textbackslash 
\hspace{0pt} \hyperlink{ecldoc:date.secondstoparts}{SecondsToParts} \textbackslash 

{\renewcommand{\arraystretch}{1.5}
\begin{tabularx}{\textwidth}{|>{\raggedright\arraybackslash}l|X|}
\hline
\hspace{0pt}\mytexttt{\color{red} INTEGER2} & \textbf{Year} \\
\hline
\end{tabularx}
}

\par





No Documentation Found








\par
\begin{description}
\item [\colorbox{tagtype}{\color{white} \textbf{\textsf{RETURN}}}] \textbf{INTEGER2} --- 
\end{description}




\rule{\linewidth}{0.5pt}
\subsection*{\textsf{\colorbox{headtoc}{\color{white} ATTRIBUTE}
Month}}

\hypertarget{ecldoc:date.secondstoparts.result.month}{}
\hspace{0pt} \hyperlink{ecldoc:Date}{Date} \textbackslash 
\hspace{0pt} \hyperlink{ecldoc:date.secondstoparts}{SecondsToParts} \textbackslash 

{\renewcommand{\arraystretch}{1.5}
\begin{tabularx}{\textwidth}{|>{\raggedright\arraybackslash}l|X|}
\hline
\hspace{0pt}\mytexttt{\color{red} UNSIGNED1} & \textbf{Month} \\
\hline
\end{tabularx}
}

\par





No Documentation Found








\par
\begin{description}
\item [\colorbox{tagtype}{\color{white} \textbf{\textsf{RETURN}}}] \textbf{UNSIGNED1} --- 
\end{description}




\rule{\linewidth}{0.5pt}
\subsection*{\textsf{\colorbox{headtoc}{\color{white} ATTRIBUTE}
Day}}

\hypertarget{ecldoc:date.secondstoparts.result.day}{}
\hspace{0pt} \hyperlink{ecldoc:Date}{Date} \textbackslash 
\hspace{0pt} \hyperlink{ecldoc:date.secondstoparts}{SecondsToParts} \textbackslash 

{\renewcommand{\arraystretch}{1.5}
\begin{tabularx}{\textwidth}{|>{\raggedright\arraybackslash}l|X|}
\hline
\hspace{0pt}\mytexttt{\color{red} UNSIGNED1} & \textbf{Day} \\
\hline
\end{tabularx}
}

\par





No Documentation Found








\par
\begin{description}
\item [\colorbox{tagtype}{\color{white} \textbf{\textsf{RETURN}}}] \textbf{UNSIGNED1} --- 
\end{description}




\rule{\linewidth}{0.5pt}
\subsection*{\textsf{\colorbox{headtoc}{\color{white} ATTRIBUTE}
Hour}}

\hypertarget{ecldoc:date.secondstoparts.result.hour}{}
\hspace{0pt} \hyperlink{ecldoc:Date}{Date} \textbackslash 
\hspace{0pt} \hyperlink{ecldoc:date.secondstoparts}{SecondsToParts} \textbackslash 

{\renewcommand{\arraystretch}{1.5}
\begin{tabularx}{\textwidth}{|>{\raggedright\arraybackslash}l|X|}
\hline
\hspace{0pt}\mytexttt{\color{red} UNSIGNED1} & \textbf{Hour} \\
\hline
\end{tabularx}
}

\par





No Documentation Found








\par
\begin{description}
\item [\colorbox{tagtype}{\color{white} \textbf{\textsf{RETURN}}}] \textbf{UNSIGNED1} --- 
\end{description}




\rule{\linewidth}{0.5pt}
\subsection*{\textsf{\colorbox{headtoc}{\color{white} ATTRIBUTE}
Minute}}

\hypertarget{ecldoc:date.secondstoparts.result.minute}{}
\hspace{0pt} \hyperlink{ecldoc:Date}{Date} \textbackslash 
\hspace{0pt} \hyperlink{ecldoc:date.secondstoparts}{SecondsToParts} \textbackslash 

{\renewcommand{\arraystretch}{1.5}
\begin{tabularx}{\textwidth}{|>{\raggedright\arraybackslash}l|X|}
\hline
\hspace{0pt}\mytexttt{\color{red} UNSIGNED1} & \textbf{Minute} \\
\hline
\end{tabularx}
}

\par





No Documentation Found








\par
\begin{description}
\item [\colorbox{tagtype}{\color{white} \textbf{\textsf{RETURN}}}] \textbf{UNSIGNED1} --- 
\end{description}




\rule{\linewidth}{0.5pt}
\subsection*{\textsf{\colorbox{headtoc}{\color{white} ATTRIBUTE}
Second}}

\hypertarget{ecldoc:date.secondstoparts.result.second}{}
\hspace{0pt} \hyperlink{ecldoc:Date}{Date} \textbackslash 
\hspace{0pt} \hyperlink{ecldoc:date.secondstoparts}{SecondsToParts} \textbackslash 

{\renewcommand{\arraystretch}{1.5}
\begin{tabularx}{\textwidth}{|>{\raggedright\arraybackslash}l|X|}
\hline
\hspace{0pt}\mytexttt{\color{red} UNSIGNED1} & \textbf{Second} \\
\hline
\end{tabularx}
}

\par





No Documentation Found








\par
\begin{description}
\item [\colorbox{tagtype}{\color{white} \textbf{\textsf{RETURN}}}] \textbf{UNSIGNED1} --- 
\end{description}




\rule{\linewidth}{0.5pt}
\subsection*{\textsf{\colorbox{headtoc}{\color{white} ATTRIBUTE}
day\_of\_week}}

\hypertarget{ecldoc:date.secondstoparts.result.day_of_week}{}
\hspace{0pt} \hyperlink{ecldoc:Date}{Date} \textbackslash 
\hspace{0pt} \hyperlink{ecldoc:date.secondstoparts}{SecondsToParts} \textbackslash 

{\renewcommand{\arraystretch}{1.5}
\begin{tabularx}{\textwidth}{|>{\raggedright\arraybackslash}l|X|}
\hline
\hspace{0pt}\mytexttt{\color{red} UNSIGNED1} & \textbf{day\_of\_week} \\
\hline
\end{tabularx}
}

\par





No Documentation Found








\par
\begin{description}
\item [\colorbox{tagtype}{\color{white} \textbf{\textsf{RETURN}}}] \textbf{UNSIGNED1} --- 
\end{description}




\rule{\linewidth}{0.5pt}
\subsection*{\textsf{\colorbox{headtoc}{\color{white} ATTRIBUTE}
date}}

\hypertarget{ecldoc:date.secondstoparts.result.date}{}
\hspace{0pt} \hyperlink{ecldoc:Date}{Date} \textbackslash 
\hspace{0pt} \hyperlink{ecldoc:date.secondstoparts}{SecondsToParts} \textbackslash 

{\renewcommand{\arraystretch}{1.5}
\begin{tabularx}{\textwidth}{|>{\raggedright\arraybackslash}l|X|}
\hline
\hspace{0pt}\mytexttt{\color{red} Date\_t} & \textbf{date} \\
\hline
\end{tabularx}
}

\par





Combines year, month day to create a date type.






\par
\begin{description}
\item [\colorbox{tagtype}{\color{white} \textbf{\textsf{PARAMETER}}}] \textbf{\underline{year}} |||  --- The year (0-9999).
\item [\colorbox{tagtype}{\color{white} \textbf{\textsf{PARAMETER}}}] \textbf{\underline{month}} |||  --- The month (1-12).
\item [\colorbox{tagtype}{\color{white} \textbf{\textsf{PARAMETER}}}] \textbf{\underline{day}} |||  --- The day (1..daysInMonth).
\end{description}







\par
\begin{description}
\item [\colorbox{tagtype}{\color{white} \textbf{\textsf{RETURN}}}] \textbf{UNSIGNED4} --- A date created by combining the fields.
\end{description}




\rule{\linewidth}{0.5pt}
\subsection*{\textsf{\colorbox{headtoc}{\color{white} ATTRIBUTE}
time}}

\hypertarget{ecldoc:date.secondstoparts.result.time}{}
\hspace{0pt} \hyperlink{ecldoc:Date}{Date} \textbackslash 
\hspace{0pt} \hyperlink{ecldoc:date.secondstoparts}{SecondsToParts} \textbackslash 

{\renewcommand{\arraystretch}{1.5}
\begin{tabularx}{\textwidth}{|>{\raggedright\arraybackslash}l|X|}
\hline
\hspace{0pt}\mytexttt{\color{red} Time\_t} & \textbf{time} \\
\hline
\end{tabularx}
}

\par





Combines hour, minute second to create a time type.






\par
\begin{description}
\item [\colorbox{tagtype}{\color{white} \textbf{\textsf{PARAMETER}}}] \textbf{\underline{minute}} |||  --- The minute (0-59).
\item [\colorbox{tagtype}{\color{white} \textbf{\textsf{PARAMETER}}}] \textbf{\underline{second}} |||  --- The second (0-59).
\item [\colorbox{tagtype}{\color{white} \textbf{\textsf{PARAMETER}}}] \textbf{\underline{hour}} |||  --- The hour (0-23).
\end{description}







\par
\begin{description}
\item [\colorbox{tagtype}{\color{white} \textbf{\textsf{RETURN}}}] \textbf{UNSIGNED3} --- A time created by combining the fields.
\end{description}




\rule{\linewidth}{0.5pt}


\subsection*{\textsf{\colorbox{headtoc}{\color{white} FUNCTION}
TimestampToSeconds}}

\hypertarget{ecldoc:date.timestamptoseconds}{}
\hspace{0pt} \hyperlink{ecldoc:Date}{Date} \textbackslash 

{\renewcommand{\arraystretch}{1.5}
\begin{tabularx}{\textwidth}{|>{\raggedright\arraybackslash}l|X|}
\hline
\hspace{0pt}\mytexttt{\color{red} Seconds\_t} & \textbf{TimestampToSeconds} \\
\hline
\multicolumn{2}{|>{\raggedright\arraybackslash}X|}{\hspace{0pt}\mytexttt{\color{param} (Timestamp\_t timestamp)}} \\
\hline
\end{tabularx}
}

\par





Converts the number of microseconds since epoch to the number of seconds since epoch.






\par
\begin{description}
\item [\colorbox{tagtype}{\color{white} \textbf{\textsf{PARAMETER}}}] \textbf{\underline{timestamp}} ||| INTEGER8 --- The number of microseconds since epoch.
\end{description}







\par
\begin{description}
\item [\colorbox{tagtype}{\color{white} \textbf{\textsf{RETURN}}}] \textbf{INTEGER8} --- The number of seconds since epoch.
\end{description}




\rule{\linewidth}{0.5pt}
\subsection*{\textsf{\colorbox{headtoc}{\color{white} FUNCTION}
IsLeapYear}}

\hypertarget{ecldoc:date.isleapyear}{}
\hspace{0pt} \hyperlink{ecldoc:Date}{Date} \textbackslash 

{\renewcommand{\arraystretch}{1.5}
\begin{tabularx}{\textwidth}{|>{\raggedright\arraybackslash}l|X|}
\hline
\hspace{0pt}\mytexttt{\color{red} BOOLEAN} & \textbf{IsLeapYear} \\
\hline
\multicolumn{2}{|>{\raggedright\arraybackslash}X|}{\hspace{0pt}\mytexttt{\color{param} (INTEGER2 year)}} \\
\hline
\end{tabularx}
}

\par





Tests whether the year is a leap year in the Gregorian calendar.






\par
\begin{description}
\item [\colorbox{tagtype}{\color{white} \textbf{\textsf{PARAMETER}}}] \textbf{\underline{year}} ||| INTEGER2 --- The year (0-9999).
\end{description}







\par
\begin{description}
\item [\colorbox{tagtype}{\color{white} \textbf{\textsf{RETURN}}}] \textbf{BOOLEAN} --- True if the year is a leap year.
\end{description}




\rule{\linewidth}{0.5pt}
\subsection*{\textsf{\colorbox{headtoc}{\color{white} FUNCTION}
IsDateLeapYear}}

\hypertarget{ecldoc:date.isdateleapyear}{}
\hspace{0pt} \hyperlink{ecldoc:Date}{Date} \textbackslash 

{\renewcommand{\arraystretch}{1.5}
\begin{tabularx}{\textwidth}{|>{\raggedright\arraybackslash}l|X|}
\hline
\hspace{0pt}\mytexttt{\color{red} BOOLEAN} & \textbf{IsDateLeapYear} \\
\hline
\multicolumn{2}{|>{\raggedright\arraybackslash}X|}{\hspace{0pt}\mytexttt{\color{param} (Date\_t date)}} \\
\hline
\end{tabularx}
}

\par





Tests whether a date is a leap year in the Gregorian calendar.






\par
\begin{description}
\item [\colorbox{tagtype}{\color{white} \textbf{\textsf{PARAMETER}}}] \textbf{\underline{date}} ||| UNSIGNED4 --- The date.
\end{description}







\par
\begin{description}
\item [\colorbox{tagtype}{\color{white} \textbf{\textsf{RETURN}}}] \textbf{BOOLEAN} --- True if the year is a leap year.
\end{description}




\rule{\linewidth}{0.5pt}
\subsection*{\textsf{\colorbox{headtoc}{\color{white} FUNCTION}
FromGregorianYMD}}

\hypertarget{ecldoc:date.fromgregorianymd}{}
\hspace{0pt} \hyperlink{ecldoc:Date}{Date} \textbackslash 

{\renewcommand{\arraystretch}{1.5}
\begin{tabularx}{\textwidth}{|>{\raggedright\arraybackslash}l|X|}
\hline
\hspace{0pt}\mytexttt{\color{red} Days\_t} & \textbf{FromGregorianYMD} \\
\hline
\multicolumn{2}{|>{\raggedright\arraybackslash}X|}{\hspace{0pt}\mytexttt{\color{param} (INTEGER2 year, UNSIGNED1 month, UNSIGNED1 day)}} \\
\hline
\end{tabularx}
}

\par





Combines year, month, day in the Gregorian calendar to create the number days since 31st December 1BC.






\par
\begin{description}
\item [\colorbox{tagtype}{\color{white} \textbf{\textsf{PARAMETER}}}] \textbf{\underline{year}} ||| INTEGER2 --- The year (-4713..9999).
\item [\colorbox{tagtype}{\color{white} \textbf{\textsf{PARAMETER}}}] \textbf{\underline{month}} ||| UNSIGNED1 --- The month (1-12). A missing value (0) is treated as 1.
\item [\colorbox{tagtype}{\color{white} \textbf{\textsf{PARAMETER}}}] \textbf{\underline{day}} ||| UNSIGNED1 --- The day (1..daysInMonth). A missing value (0) is treated as 1.
\end{description}







\par
\begin{description}
\item [\colorbox{tagtype}{\color{white} \textbf{\textsf{RETURN}}}] \textbf{INTEGER4} --- The number of elapsed days (1 Jan 1AD = 1)
\end{description}




\rule{\linewidth}{0.5pt}
\subsection*{\textsf{\colorbox{headtoc}{\color{white} MODULE}
ToGregorianYMD}}

\hypertarget{ecldoc:date.togregorianymd}{}
\hspace{0pt} \hyperlink{ecldoc:Date}{Date} \textbackslash 

{\renewcommand{\arraystretch}{1.5}
\begin{tabularx}{\textwidth}{|>{\raggedright\arraybackslash}l|X|}
\hline
\hspace{0pt}\mytexttt{\color{red} } & \textbf{ToGregorianYMD} \\
\hline
\multicolumn{2}{|>{\raggedright\arraybackslash}X|}{\hspace{0pt}\mytexttt{\color{param} (Days\_t days)}} \\
\hline
\end{tabularx}
}

\par





Converts the number days since 31st December 1BC to a date in the Gregorian calendar.






\par
\begin{description}
\item [\colorbox{tagtype}{\color{white} \textbf{\textsf{PARAMETER}}}] \textbf{\underline{days}} ||| INTEGER4 --- The number of elapsed days (1 Jan 1AD = 1)
\end{description}







\par
\begin{description}
\item [\colorbox{tagtype}{\color{white} \textbf{\textsf{RETURN}}}] \textbf{} --- Module containing Year, Month, Day in the Gregorian calendar
\end{description}




\textbf{Children}
\begin{enumerate}
\item \hyperlink{ecldoc:date.togregorianymd.result.year}{year}
: No Documentation Found
\item \hyperlink{ecldoc:date.togregorianymd.result.month}{month}
: No Documentation Found
\item \hyperlink{ecldoc:date.togregorianymd.result.day}{day}
: No Documentation Found
\end{enumerate}

\rule{\linewidth}{0.5pt}

\subsection*{\textsf{\colorbox{headtoc}{\color{white} ATTRIBUTE}
year}}

\hypertarget{ecldoc:date.togregorianymd.result.year}{}
\hspace{0pt} \hyperlink{ecldoc:Date}{Date} \textbackslash 
\hspace{0pt} \hyperlink{ecldoc:date.togregorianymd}{ToGregorianYMD} \textbackslash 

{\renewcommand{\arraystretch}{1.5}
\begin{tabularx}{\textwidth}{|>{\raggedright\arraybackslash}l|X|}
\hline
\hspace{0pt}\mytexttt{\color{red} } & \textbf{year} \\
\hline
\end{tabularx}
}

\par





No Documentation Found








\par
\begin{description}
\item [\colorbox{tagtype}{\color{white} \textbf{\textsf{RETURN}}}] \textbf{INTEGER8} --- 
\end{description}




\rule{\linewidth}{0.5pt}
\subsection*{\textsf{\colorbox{headtoc}{\color{white} ATTRIBUTE}
month}}

\hypertarget{ecldoc:date.togregorianymd.result.month}{}
\hspace{0pt} \hyperlink{ecldoc:Date}{Date} \textbackslash 
\hspace{0pt} \hyperlink{ecldoc:date.togregorianymd}{ToGregorianYMD} \textbackslash 

{\renewcommand{\arraystretch}{1.5}
\begin{tabularx}{\textwidth}{|>{\raggedright\arraybackslash}l|X|}
\hline
\hspace{0pt}\mytexttt{\color{red} } & \textbf{month} \\
\hline
\end{tabularx}
}

\par





No Documentation Found








\par
\begin{description}
\item [\colorbox{tagtype}{\color{white} \textbf{\textsf{RETURN}}}] \textbf{INTEGER8} --- 
\end{description}




\rule{\linewidth}{0.5pt}
\subsection*{\textsf{\colorbox{headtoc}{\color{white} ATTRIBUTE}
day}}

\hypertarget{ecldoc:date.togregorianymd.result.day}{}
\hspace{0pt} \hyperlink{ecldoc:Date}{Date} \textbackslash 
\hspace{0pt} \hyperlink{ecldoc:date.togregorianymd}{ToGregorianYMD} \textbackslash 

{\renewcommand{\arraystretch}{1.5}
\begin{tabularx}{\textwidth}{|>{\raggedright\arraybackslash}l|X|}
\hline
\hspace{0pt}\mytexttt{\color{red} } & \textbf{day} \\
\hline
\end{tabularx}
}

\par





No Documentation Found








\par
\begin{description}
\item [\colorbox{tagtype}{\color{white} \textbf{\textsf{RETURN}}}] \textbf{INTEGER8} --- 
\end{description}




\rule{\linewidth}{0.5pt}


\subsection*{\textsf{\colorbox{headtoc}{\color{white} FUNCTION}
FromGregorianDate}}

\hypertarget{ecldoc:date.fromgregoriandate}{}
\hspace{0pt} \hyperlink{ecldoc:Date}{Date} \textbackslash 

{\renewcommand{\arraystretch}{1.5}
\begin{tabularx}{\textwidth}{|>{\raggedright\arraybackslash}l|X|}
\hline
\hspace{0pt}\mytexttt{\color{red} Days\_t} & \textbf{FromGregorianDate} \\
\hline
\multicolumn{2}{|>{\raggedright\arraybackslash}X|}{\hspace{0pt}\mytexttt{\color{param} (Date\_t date)}} \\
\hline
\end{tabularx}
}

\par





Converts a date in the Gregorian calendar to the number days since 31st December 1BC.






\par
\begin{description}
\item [\colorbox{tagtype}{\color{white} \textbf{\textsf{PARAMETER}}}] \textbf{\underline{date}} ||| UNSIGNED4 --- The date (using the Gregorian calendar)
\end{description}







\par
\begin{description}
\item [\colorbox{tagtype}{\color{white} \textbf{\textsf{RETURN}}}] \textbf{INTEGER4} --- The number of elapsed days (1 Jan 1AD = 1)
\end{description}




\rule{\linewidth}{0.5pt}
\subsection*{\textsf{\colorbox{headtoc}{\color{white} FUNCTION}
ToGregorianDate}}

\hypertarget{ecldoc:date.togregoriandate}{}
\hspace{0pt} \hyperlink{ecldoc:Date}{Date} \textbackslash 

{\renewcommand{\arraystretch}{1.5}
\begin{tabularx}{\textwidth}{|>{\raggedright\arraybackslash}l|X|}
\hline
\hspace{0pt}\mytexttt{\color{red} Date\_t} & \textbf{ToGregorianDate} \\
\hline
\multicolumn{2}{|>{\raggedright\arraybackslash}X|}{\hspace{0pt}\mytexttt{\color{param} (Days\_t days)}} \\
\hline
\end{tabularx}
}

\par





Converts the number days since 31st December 1BC to a date in the Gregorian calendar.






\par
\begin{description}
\item [\colorbox{tagtype}{\color{white} \textbf{\textsf{PARAMETER}}}] \textbf{\underline{days}} ||| INTEGER4 --- The number of elapsed days (1 Jan 1AD = 1)
\end{description}







\par
\begin{description}
\item [\colorbox{tagtype}{\color{white} \textbf{\textsf{RETURN}}}] \textbf{UNSIGNED4} --- A Date\_t in the Gregorian calendar
\end{description}




\rule{\linewidth}{0.5pt}
\subsection*{\textsf{\colorbox{headtoc}{\color{white} FUNCTION}
DayOfYear}}

\hypertarget{ecldoc:date.dayofyear}{}
\hspace{0pt} \hyperlink{ecldoc:Date}{Date} \textbackslash 

{\renewcommand{\arraystretch}{1.5}
\begin{tabularx}{\textwidth}{|>{\raggedright\arraybackslash}l|X|}
\hline
\hspace{0pt}\mytexttt{\color{red} UNSIGNED2} & \textbf{DayOfYear} \\
\hline
\multicolumn{2}{|>{\raggedright\arraybackslash}X|}{\hspace{0pt}\mytexttt{\color{param} (Date\_t date)}} \\
\hline
\end{tabularx}
}

\par





Returns a number representing the day of the year indicated by the given date. The date must be in the Gregorian calendar after the year 1600.






\par
\begin{description}
\item [\colorbox{tagtype}{\color{white} \textbf{\textsf{PARAMETER}}}] \textbf{\underline{date}} ||| UNSIGNED4 --- A Date\_t value.
\end{description}







\par
\begin{description}
\item [\colorbox{tagtype}{\color{white} \textbf{\textsf{RETURN}}}] \textbf{UNSIGNED2} --- A number (1-366) representing the number of days since the beginning of the year.
\end{description}




\rule{\linewidth}{0.5pt}
\subsection*{\textsf{\colorbox{headtoc}{\color{white} FUNCTION}
DayOfWeek}}

\hypertarget{ecldoc:date.dayofweek}{}
\hspace{0pt} \hyperlink{ecldoc:Date}{Date} \textbackslash 

{\renewcommand{\arraystretch}{1.5}
\begin{tabularx}{\textwidth}{|>{\raggedright\arraybackslash}l|X|}
\hline
\hspace{0pt}\mytexttt{\color{red} UNSIGNED1} & \textbf{DayOfWeek} \\
\hline
\multicolumn{2}{|>{\raggedright\arraybackslash}X|}{\hspace{0pt}\mytexttt{\color{param} (Date\_t date)}} \\
\hline
\end{tabularx}
}

\par





Returns a number representing the day of the week indicated by the given date. The date must be in the Gregorian calendar after the year 1600.






\par
\begin{description}
\item [\colorbox{tagtype}{\color{white} \textbf{\textsf{PARAMETER}}}] \textbf{\underline{date}} ||| UNSIGNED4 --- A Date\_t value.
\end{description}







\par
\begin{description}
\item [\colorbox{tagtype}{\color{white} \textbf{\textsf{RETURN}}}] \textbf{UNSIGNED1} --- A number 1-7 representing the day of the week, where 1 = Sunday.
\end{description}




\rule{\linewidth}{0.5pt}
\subsection*{\textsf{\colorbox{headtoc}{\color{white} FUNCTION}
IsJulianLeapYear}}

\hypertarget{ecldoc:date.isjulianleapyear}{}
\hspace{0pt} \hyperlink{ecldoc:Date}{Date} \textbackslash 

{\renewcommand{\arraystretch}{1.5}
\begin{tabularx}{\textwidth}{|>{\raggedright\arraybackslash}l|X|}
\hline
\hspace{0pt}\mytexttt{\color{red} BOOLEAN} & \textbf{IsJulianLeapYear} \\
\hline
\multicolumn{2}{|>{\raggedright\arraybackslash}X|}{\hspace{0pt}\mytexttt{\color{param} (INTEGER2 year)}} \\
\hline
\end{tabularx}
}

\par





Tests whether the year is a leap year in the Julian calendar.






\par
\begin{description}
\item [\colorbox{tagtype}{\color{white} \textbf{\textsf{PARAMETER}}}] \textbf{\underline{year}} ||| INTEGER2 --- The year (0-9999).
\end{description}







\par
\begin{description}
\item [\colorbox{tagtype}{\color{white} \textbf{\textsf{RETURN}}}] \textbf{BOOLEAN} --- True if the year is a leap year.
\end{description}




\rule{\linewidth}{0.5pt}
\subsection*{\textsf{\colorbox{headtoc}{\color{white} FUNCTION}
FromJulianYMD}}

\hypertarget{ecldoc:date.fromjulianymd}{}
\hspace{0pt} \hyperlink{ecldoc:Date}{Date} \textbackslash 

{\renewcommand{\arraystretch}{1.5}
\begin{tabularx}{\textwidth}{|>{\raggedright\arraybackslash}l|X|}
\hline
\hspace{0pt}\mytexttt{\color{red} Days\_t} & \textbf{FromJulianYMD} \\
\hline
\multicolumn{2}{|>{\raggedright\arraybackslash}X|}{\hspace{0pt}\mytexttt{\color{param} (INTEGER2 year, UNSIGNED1 month, UNSIGNED1 day)}} \\
\hline
\end{tabularx}
}

\par





Combines year, month, day in the Julian calendar to create the number days since 31st December 1BC.






\par
\begin{description}
\item [\colorbox{tagtype}{\color{white} \textbf{\textsf{PARAMETER}}}] \textbf{\underline{year}} ||| INTEGER2 --- The year (-4800..9999).
\item [\colorbox{tagtype}{\color{white} \textbf{\textsf{PARAMETER}}}] \textbf{\underline{month}} ||| UNSIGNED1 --- The month (1-12).
\item [\colorbox{tagtype}{\color{white} \textbf{\textsf{PARAMETER}}}] \textbf{\underline{day}} ||| UNSIGNED1 --- The day (1..daysInMonth).
\end{description}







\par
\begin{description}
\item [\colorbox{tagtype}{\color{white} \textbf{\textsf{RETURN}}}] \textbf{INTEGER4} --- The number of elapsed days (1 Jan 1AD = 1)
\end{description}




\rule{\linewidth}{0.5pt}
\subsection*{\textsf{\colorbox{headtoc}{\color{white} MODULE}
ToJulianYMD}}

\hypertarget{ecldoc:date.tojulianymd}{}
\hspace{0pt} \hyperlink{ecldoc:Date}{Date} \textbackslash 

{\renewcommand{\arraystretch}{1.5}
\begin{tabularx}{\textwidth}{|>{\raggedright\arraybackslash}l|X|}
\hline
\hspace{0pt}\mytexttt{\color{red} } & \textbf{ToJulianYMD} \\
\hline
\multicolumn{2}{|>{\raggedright\arraybackslash}X|}{\hspace{0pt}\mytexttt{\color{param} (Days\_t days)}} \\
\hline
\end{tabularx}
}

\par





Converts the number days since 31st December 1BC to a date in the Julian calendar.






\par
\begin{description}
\item [\colorbox{tagtype}{\color{white} \textbf{\textsf{PARAMETER}}}] \textbf{\underline{days}} ||| INTEGER4 --- The number of elapsed days (1 Jan 1AD = 1)
\end{description}







\par
\begin{description}
\item [\colorbox{tagtype}{\color{white} \textbf{\textsf{RETURN}}}] \textbf{} --- Module containing Year, Month, Day in the Julian calendar
\end{description}




\textbf{Children}
\begin{enumerate}
\item \hyperlink{ecldoc:date.tojulianymd.result.day}{Day}
: No Documentation Found
\item \hyperlink{ecldoc:date.tojulianymd.result.month}{Month}
: No Documentation Found
\item \hyperlink{ecldoc:date.tojulianymd.result.year}{Year}
: No Documentation Found
\end{enumerate}

\rule{\linewidth}{0.5pt}

\subsection*{\textsf{\colorbox{headtoc}{\color{white} ATTRIBUTE}
Day}}

\hypertarget{ecldoc:date.tojulianymd.result.day}{}
\hspace{0pt} \hyperlink{ecldoc:Date}{Date} \textbackslash 
\hspace{0pt} \hyperlink{ecldoc:date.tojulianymd}{ToJulianYMD} \textbackslash 

{\renewcommand{\arraystretch}{1.5}
\begin{tabularx}{\textwidth}{|>{\raggedright\arraybackslash}l|X|}
\hline
\hspace{0pt}\mytexttt{\color{red} UNSIGNED1} & \textbf{Day} \\
\hline
\end{tabularx}
}

\par





No Documentation Found








\par
\begin{description}
\item [\colorbox{tagtype}{\color{white} \textbf{\textsf{RETURN}}}] \textbf{UNSIGNED1} --- 
\end{description}




\rule{\linewidth}{0.5pt}
\subsection*{\textsf{\colorbox{headtoc}{\color{white} ATTRIBUTE}
Month}}

\hypertarget{ecldoc:date.tojulianymd.result.month}{}
\hspace{0pt} \hyperlink{ecldoc:Date}{Date} \textbackslash 
\hspace{0pt} \hyperlink{ecldoc:date.tojulianymd}{ToJulianYMD} \textbackslash 

{\renewcommand{\arraystretch}{1.5}
\begin{tabularx}{\textwidth}{|>{\raggedright\arraybackslash}l|X|}
\hline
\hspace{0pt}\mytexttt{\color{red} UNSIGNED1} & \textbf{Month} \\
\hline
\end{tabularx}
}

\par





No Documentation Found








\par
\begin{description}
\item [\colorbox{tagtype}{\color{white} \textbf{\textsf{RETURN}}}] \textbf{UNSIGNED1} --- 
\end{description}




\rule{\linewidth}{0.5pt}
\subsection*{\textsf{\colorbox{headtoc}{\color{white} ATTRIBUTE}
Year}}

\hypertarget{ecldoc:date.tojulianymd.result.year}{}
\hspace{0pt} \hyperlink{ecldoc:Date}{Date} \textbackslash 
\hspace{0pt} \hyperlink{ecldoc:date.tojulianymd}{ToJulianYMD} \textbackslash 

{\renewcommand{\arraystretch}{1.5}
\begin{tabularx}{\textwidth}{|>{\raggedright\arraybackslash}l|X|}
\hline
\hspace{0pt}\mytexttt{\color{red} INTEGER2} & \textbf{Year} \\
\hline
\end{tabularx}
}

\par





No Documentation Found








\par
\begin{description}
\item [\colorbox{tagtype}{\color{white} \textbf{\textsf{RETURN}}}] \textbf{INTEGER2} --- 
\end{description}




\rule{\linewidth}{0.5pt}


\subsection*{\textsf{\colorbox{headtoc}{\color{white} FUNCTION}
FromJulianDate}}

\hypertarget{ecldoc:date.fromjuliandate}{}
\hspace{0pt} \hyperlink{ecldoc:Date}{Date} \textbackslash 

{\renewcommand{\arraystretch}{1.5}
\begin{tabularx}{\textwidth}{|>{\raggedright\arraybackslash}l|X|}
\hline
\hspace{0pt}\mytexttt{\color{red} Days\_t} & \textbf{FromJulianDate} \\
\hline
\multicolumn{2}{|>{\raggedright\arraybackslash}X|}{\hspace{0pt}\mytexttt{\color{param} (Date\_t date)}} \\
\hline
\end{tabularx}
}

\par





Converts a date in the Julian calendar to the number days since 31st December 1BC.






\par
\begin{description}
\item [\colorbox{tagtype}{\color{white} \textbf{\textsf{PARAMETER}}}] \textbf{\underline{date}} ||| UNSIGNED4 --- The date (using the Julian calendar)
\end{description}







\par
\begin{description}
\item [\colorbox{tagtype}{\color{white} \textbf{\textsf{RETURN}}}] \textbf{INTEGER4} --- The number of elapsed days (1 Jan 1AD = 1)
\end{description}




\rule{\linewidth}{0.5pt}
\subsection*{\textsf{\colorbox{headtoc}{\color{white} FUNCTION}
ToJulianDate}}

\hypertarget{ecldoc:date.tojuliandate}{}
\hspace{0pt} \hyperlink{ecldoc:Date}{Date} \textbackslash 

{\renewcommand{\arraystretch}{1.5}
\begin{tabularx}{\textwidth}{|>{\raggedright\arraybackslash}l|X|}
\hline
\hspace{0pt}\mytexttt{\color{red} Date\_t} & \textbf{ToJulianDate} \\
\hline
\multicolumn{2}{|>{\raggedright\arraybackslash}X|}{\hspace{0pt}\mytexttt{\color{param} (Days\_t days)}} \\
\hline
\end{tabularx}
}

\par





Converts the number days since 31st December 1BC to a date in the Julian calendar.






\par
\begin{description}
\item [\colorbox{tagtype}{\color{white} \textbf{\textsf{PARAMETER}}}] \textbf{\underline{days}} ||| INTEGER4 --- The number of elapsed days (1 Jan 1AD = 1)
\end{description}







\par
\begin{description}
\item [\colorbox{tagtype}{\color{white} \textbf{\textsf{RETURN}}}] \textbf{UNSIGNED4} --- A Date\_t in the Julian calendar
\end{description}




\rule{\linewidth}{0.5pt}
\subsection*{\textsf{\colorbox{headtoc}{\color{white} FUNCTION}
DaysSince1900}}

\hypertarget{ecldoc:date.dayssince1900}{}
\hspace{0pt} \hyperlink{ecldoc:Date}{Date} \textbackslash 

{\renewcommand{\arraystretch}{1.5}
\begin{tabularx}{\textwidth}{|>{\raggedright\arraybackslash}l|X|}
\hline
\hspace{0pt}\mytexttt{\color{red} Days\_t} & \textbf{DaysSince1900} \\
\hline
\multicolumn{2}{|>{\raggedright\arraybackslash}X|}{\hspace{0pt}\mytexttt{\color{param} (INTEGER2 year, UNSIGNED1 month, UNSIGNED1 day)}} \\
\hline
\end{tabularx}
}

\par





Returns the number of days since 1st January 1900 (using the Gregorian Calendar)






\par
\begin{description}
\item [\colorbox{tagtype}{\color{white} \textbf{\textsf{PARAMETER}}}] \textbf{\underline{year}} ||| INTEGER2 --- The year (-4713..9999).
\item [\colorbox{tagtype}{\color{white} \textbf{\textsf{PARAMETER}}}] \textbf{\underline{month}} ||| UNSIGNED1 --- The month (1-12). A missing value (0) is treated as 1.
\item [\colorbox{tagtype}{\color{white} \textbf{\textsf{PARAMETER}}}] \textbf{\underline{day}} ||| UNSIGNED1 --- The day (1..daysInMonth). A missing value (0) is treated as 1.
\end{description}







\par
\begin{description}
\item [\colorbox{tagtype}{\color{white} \textbf{\textsf{RETURN}}}] \textbf{INTEGER4} --- The number of elapsed days since 1st January 1900
\end{description}




\rule{\linewidth}{0.5pt}
\subsection*{\textsf{\colorbox{headtoc}{\color{white} FUNCTION}
ToDaysSince1900}}

\hypertarget{ecldoc:date.todayssince1900}{}
\hspace{0pt} \hyperlink{ecldoc:Date}{Date} \textbackslash 

{\renewcommand{\arraystretch}{1.5}
\begin{tabularx}{\textwidth}{|>{\raggedright\arraybackslash}l|X|}
\hline
\hspace{0pt}\mytexttt{\color{red} Days\_t} & \textbf{ToDaysSince1900} \\
\hline
\multicolumn{2}{|>{\raggedright\arraybackslash}X|}{\hspace{0pt}\mytexttt{\color{param} (Date\_t date)}} \\
\hline
\end{tabularx}
}

\par





Returns the number of days since 1st January 1900 (using the Gregorian Calendar)






\par
\begin{description}
\item [\colorbox{tagtype}{\color{white} \textbf{\textsf{PARAMETER}}}] \textbf{\underline{date}} ||| UNSIGNED4 --- The date
\end{description}







\par
\begin{description}
\item [\colorbox{tagtype}{\color{white} \textbf{\textsf{RETURN}}}] \textbf{INTEGER4} --- The number of elapsed days since 1st January 1900
\end{description}




\rule{\linewidth}{0.5pt}
\subsection*{\textsf{\colorbox{headtoc}{\color{white} FUNCTION}
FromDaysSince1900}}

\hypertarget{ecldoc:date.fromdayssince1900}{}
\hspace{0pt} \hyperlink{ecldoc:Date}{Date} \textbackslash 

{\renewcommand{\arraystretch}{1.5}
\begin{tabularx}{\textwidth}{|>{\raggedright\arraybackslash}l|X|}
\hline
\hspace{0pt}\mytexttt{\color{red} Date\_t} & \textbf{FromDaysSince1900} \\
\hline
\multicolumn{2}{|>{\raggedright\arraybackslash}X|}{\hspace{0pt}\mytexttt{\color{param} (Days\_t days)}} \\
\hline
\end{tabularx}
}

\par





Converts the number days since 1st January 1900 to a date in the Julian calendar.






\par
\begin{description}
\item [\colorbox{tagtype}{\color{white} \textbf{\textsf{PARAMETER}}}] \textbf{\underline{days}} ||| INTEGER4 --- The number of elapsed days since 1st Jan 1900
\end{description}







\par
\begin{description}
\item [\colorbox{tagtype}{\color{white} \textbf{\textsf{RETURN}}}] \textbf{UNSIGNED4} --- A Date\_t in the Julian calendar
\end{description}




\rule{\linewidth}{0.5pt}
\subsection*{\textsf{\colorbox{headtoc}{\color{white} FUNCTION}
YearsBetween}}

\hypertarget{ecldoc:date.yearsbetween}{}
\hspace{0pt} \hyperlink{ecldoc:Date}{Date} \textbackslash 

{\renewcommand{\arraystretch}{1.5}
\begin{tabularx}{\textwidth}{|>{\raggedright\arraybackslash}l|X|}
\hline
\hspace{0pt}\mytexttt{\color{red} INTEGER} & \textbf{YearsBetween} \\
\hline
\multicolumn{2}{|>{\raggedright\arraybackslash}X|}{\hspace{0pt}\mytexttt{\color{param} (Date\_t from, Date\_t to)}} \\
\hline
\end{tabularx}
}

\par





Calculate the number of whole years between two dates.






\par
\begin{description}
\item [\colorbox{tagtype}{\color{white} \textbf{\textsf{PARAMETER}}}] \textbf{\underline{from}} ||| UNSIGNED4 --- The first date
\item [\colorbox{tagtype}{\color{white} \textbf{\textsf{PARAMETER}}}] \textbf{\underline{to}} ||| UNSIGNED4 --- The last date
\end{description}







\par
\begin{description}
\item [\colorbox{tagtype}{\color{white} \textbf{\textsf{RETURN}}}] \textbf{INTEGER8} --- The number of years between them.
\end{description}




\rule{\linewidth}{0.5pt}
\subsection*{\textsf{\colorbox{headtoc}{\color{white} FUNCTION}
MonthsBetween}}

\hypertarget{ecldoc:date.monthsbetween}{}
\hspace{0pt} \hyperlink{ecldoc:Date}{Date} \textbackslash 

{\renewcommand{\arraystretch}{1.5}
\begin{tabularx}{\textwidth}{|>{\raggedright\arraybackslash}l|X|}
\hline
\hspace{0pt}\mytexttt{\color{red} INTEGER} & \textbf{MonthsBetween} \\
\hline
\multicolumn{2}{|>{\raggedright\arraybackslash}X|}{\hspace{0pt}\mytexttt{\color{param} (Date\_t from, Date\_t to)}} \\
\hline
\end{tabularx}
}

\par





Calculate the number of whole months between two dates.






\par
\begin{description}
\item [\colorbox{tagtype}{\color{white} \textbf{\textsf{PARAMETER}}}] \textbf{\underline{from}} ||| UNSIGNED4 --- The first date
\item [\colorbox{tagtype}{\color{white} \textbf{\textsf{PARAMETER}}}] \textbf{\underline{to}} ||| UNSIGNED4 --- The last date
\end{description}







\par
\begin{description}
\item [\colorbox{tagtype}{\color{white} \textbf{\textsf{RETURN}}}] \textbf{INTEGER8} --- The number of months between them.
\end{description}




\rule{\linewidth}{0.5pt}
\subsection*{\textsf{\colorbox{headtoc}{\color{white} FUNCTION}
DaysBetween}}

\hypertarget{ecldoc:date.daysbetween}{}
\hspace{0pt} \hyperlink{ecldoc:Date}{Date} \textbackslash 

{\renewcommand{\arraystretch}{1.5}
\begin{tabularx}{\textwidth}{|>{\raggedright\arraybackslash}l|X|}
\hline
\hspace{0pt}\mytexttt{\color{red} INTEGER} & \textbf{DaysBetween} \\
\hline
\multicolumn{2}{|>{\raggedright\arraybackslash}X|}{\hspace{0pt}\mytexttt{\color{param} (Date\_t from, Date\_t to)}} \\
\hline
\end{tabularx}
}

\par





Calculate the number of days between two dates.






\par
\begin{description}
\item [\colorbox{tagtype}{\color{white} \textbf{\textsf{PARAMETER}}}] \textbf{\underline{from}} ||| UNSIGNED4 --- The first date
\item [\colorbox{tagtype}{\color{white} \textbf{\textsf{PARAMETER}}}] \textbf{\underline{to}} ||| UNSIGNED4 --- The last date
\end{description}







\par
\begin{description}
\item [\colorbox{tagtype}{\color{white} \textbf{\textsf{RETURN}}}] \textbf{INTEGER8} --- The number of days between them.
\end{description}




\rule{\linewidth}{0.5pt}
\subsection*{\textsf{\colorbox{headtoc}{\color{white} FUNCTION}
DateFromDateRec}}

\hypertarget{ecldoc:date.datefromdaterec}{}
\hspace{0pt} \hyperlink{ecldoc:Date}{Date} \textbackslash 

{\renewcommand{\arraystretch}{1.5}
\begin{tabularx}{\textwidth}{|>{\raggedright\arraybackslash}l|X|}
\hline
\hspace{0pt}\mytexttt{\color{red} Date\_t} & \textbf{DateFromDateRec} \\
\hline
\multicolumn{2}{|>{\raggedright\arraybackslash}X|}{\hspace{0pt}\mytexttt{\color{param} (Date\_rec date)}} \\
\hline
\end{tabularx}
}

\par





Combines the fields from a Date\_rec to create a Date\_t






\par
\begin{description}
\item [\colorbox{tagtype}{\color{white} \textbf{\textsf{PARAMETER}}}] \textbf{\underline{date}} ||| ROW ( Date\_rec ) --- The row containing the date.
\end{description}







\par
\begin{description}
\item [\colorbox{tagtype}{\color{white} \textbf{\textsf{RETURN}}}] \textbf{UNSIGNED4} --- A Date\_t representing the combined values.
\end{description}




\rule{\linewidth}{0.5pt}
\subsection*{\textsf{\colorbox{headtoc}{\color{white} FUNCTION}
DateFromRec}}

\hypertarget{ecldoc:date.datefromrec}{}
\hspace{0pt} \hyperlink{ecldoc:Date}{Date} \textbackslash 

{\renewcommand{\arraystretch}{1.5}
\begin{tabularx}{\textwidth}{|>{\raggedright\arraybackslash}l|X|}
\hline
\hspace{0pt}\mytexttt{\color{red} Date\_t} & \textbf{DateFromRec} \\
\hline
\multicolumn{2}{|>{\raggedright\arraybackslash}X|}{\hspace{0pt}\mytexttt{\color{param} (Date\_rec date)}} \\
\hline
\end{tabularx}
}

\par





Combines the fields from a Date\_rec to create a Date\_t






\par
\begin{description}
\item [\colorbox{tagtype}{\color{white} \textbf{\textsf{PARAMETER}}}] \textbf{\underline{date}} ||| ROW ( Date\_rec ) --- The row containing the date.
\end{description}







\par
\begin{description}
\item [\colorbox{tagtype}{\color{white} \textbf{\textsf{RETURN}}}] \textbf{UNSIGNED4} --- A Date\_t representing the combined values.
\end{description}




\rule{\linewidth}{0.5pt}
\subsection*{\textsf{\colorbox{headtoc}{\color{white} FUNCTION}
TimeFromTimeRec}}

\hypertarget{ecldoc:date.timefromtimerec}{}
\hspace{0pt} \hyperlink{ecldoc:Date}{Date} \textbackslash 

{\renewcommand{\arraystretch}{1.5}
\begin{tabularx}{\textwidth}{|>{\raggedright\arraybackslash}l|X|}
\hline
\hspace{0pt}\mytexttt{\color{red} Time\_t} & \textbf{TimeFromTimeRec} \\
\hline
\multicolumn{2}{|>{\raggedright\arraybackslash}X|}{\hspace{0pt}\mytexttt{\color{param} (Time\_rec time)}} \\
\hline
\end{tabularx}
}

\par





Combines the fields from a Time\_rec to create a Time\_t






\par
\begin{description}
\item [\colorbox{tagtype}{\color{white} \textbf{\textsf{PARAMETER}}}] \textbf{\underline{time}} ||| ROW ( Time\_rec ) --- The row containing the time.
\end{description}







\par
\begin{description}
\item [\colorbox{tagtype}{\color{white} \textbf{\textsf{RETURN}}}] \textbf{UNSIGNED3} --- A Time\_t representing the combined values.
\end{description}




\rule{\linewidth}{0.5pt}
\subsection*{\textsf{\colorbox{headtoc}{\color{white} FUNCTION}
DateFromDateTimeRec}}

\hypertarget{ecldoc:date.datefromdatetimerec}{}
\hspace{0pt} \hyperlink{ecldoc:Date}{Date} \textbackslash 

{\renewcommand{\arraystretch}{1.5}
\begin{tabularx}{\textwidth}{|>{\raggedright\arraybackslash}l|X|}
\hline
\hspace{0pt}\mytexttt{\color{red} Date\_t} & \textbf{DateFromDateTimeRec} \\
\hline
\multicolumn{2}{|>{\raggedright\arraybackslash}X|}{\hspace{0pt}\mytexttt{\color{param} (DateTime\_rec datetime)}} \\
\hline
\end{tabularx}
}

\par





Combines the date fields from a DateTime\_rec to create a Date\_t






\par
\begin{description}
\item [\colorbox{tagtype}{\color{white} \textbf{\textsf{PARAMETER}}}] \textbf{\underline{datetime}} ||| ROW ( DateTime\_rec ) --- The row containing the datetime.
\end{description}







\par
\begin{description}
\item [\colorbox{tagtype}{\color{white} \textbf{\textsf{RETURN}}}] \textbf{UNSIGNED4} --- A Date\_t representing the combined values.
\end{description}




\rule{\linewidth}{0.5pt}
\subsection*{\textsf{\colorbox{headtoc}{\color{white} FUNCTION}
TimeFromDateTimeRec}}

\hypertarget{ecldoc:date.timefromdatetimerec}{}
\hspace{0pt} \hyperlink{ecldoc:Date}{Date} \textbackslash 

{\renewcommand{\arraystretch}{1.5}
\begin{tabularx}{\textwidth}{|>{\raggedright\arraybackslash}l|X|}
\hline
\hspace{0pt}\mytexttt{\color{red} Time\_t} & \textbf{TimeFromDateTimeRec} \\
\hline
\multicolumn{2}{|>{\raggedright\arraybackslash}X|}{\hspace{0pt}\mytexttt{\color{param} (DateTime\_rec datetime)}} \\
\hline
\end{tabularx}
}

\par





Combines the time fields from a DateTime\_rec to create a Time\_t






\par
\begin{description}
\item [\colorbox{tagtype}{\color{white} \textbf{\textsf{PARAMETER}}}] \textbf{\underline{datetime}} ||| ROW ( DateTime\_rec ) --- The row containing the datetime.
\end{description}







\par
\begin{description}
\item [\colorbox{tagtype}{\color{white} \textbf{\textsf{RETURN}}}] \textbf{UNSIGNED3} --- A Time\_t representing the combined values.
\end{description}




\rule{\linewidth}{0.5pt}
\subsection*{\textsf{\colorbox{headtoc}{\color{white} FUNCTION}
SecondsFromDateTimeRec}}

\hypertarget{ecldoc:date.secondsfromdatetimerec}{}
\hspace{0pt} \hyperlink{ecldoc:Date}{Date} \textbackslash 

{\renewcommand{\arraystretch}{1.5}
\begin{tabularx}{\textwidth}{|>{\raggedright\arraybackslash}l|X|}
\hline
\hspace{0pt}\mytexttt{\color{red} Seconds\_t} & \textbf{SecondsFromDateTimeRec} \\
\hline
\multicolumn{2}{|>{\raggedright\arraybackslash}X|}{\hspace{0pt}\mytexttt{\color{param} (DateTime\_rec datetime, BOOLEAN is\_local\_time = FALSE)}} \\
\hline
\end{tabularx}
}

\par





Combines the date and time fields from a DateTime\_rec to create a Seconds\_t






\par
\begin{description}
\item [\colorbox{tagtype}{\color{white} \textbf{\textsf{PARAMETER}}}] \textbf{\underline{datetime}} ||| ROW ( DateTime\_rec ) --- The row containing the datetime.
\item [\colorbox{tagtype}{\color{white} \textbf{\textsf{PARAMETER}}}] \textbf{\underline{is\_local\_time}} ||| BOOLEAN --- TRUE if the datetime components are expressed in local time rather than UTC, FALSE if the components are expressed in UTC. Optional, defaults to FALSE.
\end{description}







\par
\begin{description}
\item [\colorbox{tagtype}{\color{white} \textbf{\textsf{RETURN}}}] \textbf{INTEGER8} --- A Seconds\_t representing the combined values.
\end{description}




\rule{\linewidth}{0.5pt}
\subsection*{\textsf{\colorbox{headtoc}{\color{white} FUNCTION}
FromStringToDate}}

\hypertarget{ecldoc:date.fromstringtodate}{}
\hspace{0pt} \hyperlink{ecldoc:Date}{Date} \textbackslash 

{\renewcommand{\arraystretch}{1.5}
\begin{tabularx}{\textwidth}{|>{\raggedright\arraybackslash}l|X|}
\hline
\hspace{0pt}\mytexttt{\color{red} Date\_t} & \textbf{FromStringToDate} \\
\hline
\multicolumn{2}{|>{\raggedright\arraybackslash}X|}{\hspace{0pt}\mytexttt{\color{param} (STRING date\_text, VARSTRING format)}} \\
\hline
\end{tabularx}
}

\par





Converts a string to a Date\_t using the relevant string format. The resulting date must be representable within the Gregorian calendar after the year 1600.






\par
\begin{description}
\item [\colorbox{tagtype}{\color{white} \textbf{\textsf{PARAMETER}}}] \textbf{\underline{date\_text}} ||| STRING --- The string to be converted.
\item [\colorbox{tagtype}{\color{white} \textbf{\textsf{PARAMETER}}}] \textbf{\underline{format}} ||| VARSTRING --- The format of the input string. (See documentation for strftime)
\end{description}







\par
\begin{description}
\item [\colorbox{tagtype}{\color{white} \textbf{\textsf{RETURN}}}] \textbf{UNSIGNED4} --- The date that was matched in the string. Returns 0 if failed to match or if the date components match but the result is an invalid date. Supported characters: \%B Full month name \%b or \%h Abbreviated month name \%d Day of month (two digits) \%e Day of month (two digits, or a space followed by a single digit) \%m Month (two digits) \%t Whitespace \%y year within century (00-99) \%Y Full year (yyyy) \%j Julian day (1-366) Common date formats American '\%m/\%d/\%Y' mm/dd/yyyy Euro '\%d/\%m/\%Y' dd/mm/yyyy Iso format '\%Y-\%m-\%d' yyyy-mm-dd Iso basic 'Y\%m\%d' yyyymmdd '\%d-\%b-\%Y' dd-mon-yyyy e.g., '21-Mar-1954'
\end{description}




\rule{\linewidth}{0.5pt}
\subsection*{\textsf{\colorbox{headtoc}{\color{white} FUNCTION}
FromString}}

\hypertarget{ecldoc:date.fromstring}{}
\hspace{0pt} \hyperlink{ecldoc:Date}{Date} \textbackslash 

{\renewcommand{\arraystretch}{1.5}
\begin{tabularx}{\textwidth}{|>{\raggedright\arraybackslash}l|X|}
\hline
\hspace{0pt}\mytexttt{\color{red} Date\_t} & \textbf{FromString} \\
\hline
\multicolumn{2}{|>{\raggedright\arraybackslash}X|}{\hspace{0pt}\mytexttt{\color{param} (STRING date\_text, VARSTRING format)}} \\
\hline
\end{tabularx}
}

\par





Converts a string to a date using the relevant string format.






\par
\begin{description}
\item [\colorbox{tagtype}{\color{white} \textbf{\textsf{PARAMETER}}}] \textbf{\underline{date\_text}} ||| STRING --- The string to be converted.
\item [\colorbox{tagtype}{\color{white} \textbf{\textsf{PARAMETER}}}] \textbf{\underline{format}} ||| VARSTRING --- The format of the input string. (See documentation for strftime)
\end{description}







\par
\begin{description}
\item [\colorbox{tagtype}{\color{white} \textbf{\textsf{RETURN}}}] \textbf{UNSIGNED4} --- The date that was matched in the string. Returns 0 if failed to match.
\end{description}




\rule{\linewidth}{0.5pt}
\subsection*{\textsf{\colorbox{headtoc}{\color{white} FUNCTION}
FromStringToTime}}

\hypertarget{ecldoc:date.fromstringtotime}{}
\hspace{0pt} \hyperlink{ecldoc:Date}{Date} \textbackslash 

{\renewcommand{\arraystretch}{1.5}
\begin{tabularx}{\textwidth}{|>{\raggedright\arraybackslash}l|X|}
\hline
\hspace{0pt}\mytexttt{\color{red} Time\_t} & \textbf{FromStringToTime} \\
\hline
\multicolumn{2}{|>{\raggedright\arraybackslash}X|}{\hspace{0pt}\mytexttt{\color{param} (STRING time\_text, VARSTRING format)}} \\
\hline
\end{tabularx}
}

\par





Converts a string to a Time\_t using the relevant string format.






\par
\begin{description}
\item [\colorbox{tagtype}{\color{white} \textbf{\textsf{PARAMETER}}}] \textbf{\underline{date\_text}} |||  --- The string to be converted.
\item [\colorbox{tagtype}{\color{white} \textbf{\textsf{PARAMETER}}}] \textbf{\underline{format}} ||| VARSTRING --- The format of the input string. (See documentation for strftime)
\item [\colorbox{tagtype}{\color{white} \textbf{\textsf{PARAMETER}}}] \textbf{\underline{time\_text}} ||| STRING --- No Doc
\end{description}







\par
\begin{description}
\item [\colorbox{tagtype}{\color{white} \textbf{\textsf{RETURN}}}] \textbf{UNSIGNED3} --- The time that was matched in the string. Returns 0 if failed to match. Supported characters: \%H Hour (two digits) \%k (two digits, or a space followed by a single digit) \%M Minute (two digits) \%S Second (two digits) \%t Whitespace
\end{description}




\rule{\linewidth}{0.5pt}
\subsection*{\textsf{\colorbox{headtoc}{\color{white} FUNCTION}
MatchDateString}}

\hypertarget{ecldoc:date.matchdatestring}{}
\hspace{0pt} \hyperlink{ecldoc:Date}{Date} \textbackslash 

{\renewcommand{\arraystretch}{1.5}
\begin{tabularx}{\textwidth}{|>{\raggedright\arraybackslash}l|X|}
\hline
\hspace{0pt}\mytexttt{\color{red} Date\_t} & \textbf{MatchDateString} \\
\hline
\multicolumn{2}{|>{\raggedright\arraybackslash}X|}{\hspace{0pt}\mytexttt{\color{param} (STRING date\_text, SET OF VARSTRING formats)}} \\
\hline
\end{tabularx}
}

\par





Matches a string against a set of date string formats and returns a valid Date\_t object from the first format that successfully parses the string.






\par
\begin{description}
\item [\colorbox{tagtype}{\color{white} \textbf{\textsf{PARAMETER}}}] \textbf{\underline{formats}} ||| SET ( VARSTRING ) --- A set of formats to check against the string. (See documentation for strftime)
\item [\colorbox{tagtype}{\color{white} \textbf{\textsf{PARAMETER}}}] \textbf{\underline{date\_text}} ||| STRING --- The string to be converted.
\end{description}







\par
\begin{description}
\item [\colorbox{tagtype}{\color{white} \textbf{\textsf{RETURN}}}] \textbf{UNSIGNED4} --- The date that was matched in the string. Returns 0 if failed to match.
\end{description}




\rule{\linewidth}{0.5pt}
\subsection*{\textsf{\colorbox{headtoc}{\color{white} FUNCTION}
MatchTimeString}}

\hypertarget{ecldoc:date.matchtimestring}{}
\hspace{0pt} \hyperlink{ecldoc:Date}{Date} \textbackslash 

{\renewcommand{\arraystretch}{1.5}
\begin{tabularx}{\textwidth}{|>{\raggedright\arraybackslash}l|X|}
\hline
\hspace{0pt}\mytexttt{\color{red} Time\_t} & \textbf{MatchTimeString} \\
\hline
\multicolumn{2}{|>{\raggedright\arraybackslash}X|}{\hspace{0pt}\mytexttt{\color{param} (STRING time\_text, SET OF VARSTRING formats)}} \\
\hline
\end{tabularx}
}

\par





Matches a string against a set of time string formats and returns a valid Time\_t object from the first format that successfully parses the string.






\par
\begin{description}
\item [\colorbox{tagtype}{\color{white} \textbf{\textsf{PARAMETER}}}] \textbf{\underline{formats}} ||| SET ( VARSTRING ) --- A set of formats to check against the string. (See documentation for strftime)
\item [\colorbox{tagtype}{\color{white} \textbf{\textsf{PARAMETER}}}] \textbf{\underline{time\_text}} ||| STRING --- The string to be converted.
\end{description}







\par
\begin{description}
\item [\colorbox{tagtype}{\color{white} \textbf{\textsf{RETURN}}}] \textbf{UNSIGNED3} --- The time that was matched in the string. Returns 0 if failed to match.
\end{description}




\rule{\linewidth}{0.5pt}
\subsection*{\textsf{\colorbox{headtoc}{\color{white} FUNCTION}
DateToString}}

\hypertarget{ecldoc:date.datetostring}{}
\hspace{0pt} \hyperlink{ecldoc:Date}{Date} \textbackslash 

{\renewcommand{\arraystretch}{1.5}
\begin{tabularx}{\textwidth}{|>{\raggedright\arraybackslash}l|X|}
\hline
\hspace{0pt}\mytexttt{\color{red} STRING} & \textbf{DateToString} \\
\hline
\multicolumn{2}{|>{\raggedright\arraybackslash}X|}{\hspace{0pt}\mytexttt{\color{param} (Date\_t date, VARSTRING format = '\%Y-\%m-\%d')}} \\
\hline
\end{tabularx}
}

\par





Formats a date as a string.






\par
\begin{description}
\item [\colorbox{tagtype}{\color{white} \textbf{\textsf{PARAMETER}}}] \textbf{\underline{date}} ||| UNSIGNED4 --- The date to be converted.
\item [\colorbox{tagtype}{\color{white} \textbf{\textsf{PARAMETER}}}] \textbf{\underline{format}} ||| VARSTRING --- The format template to use for the conversion; see strftime() for appropriate values. The maximum length of the resulting string is 255 characters. Optional; defaults to '\%Y-\%m-\%d' which is YYYY-MM-DD.
\end{description}







\par
\begin{description}
\item [\colorbox{tagtype}{\color{white} \textbf{\textsf{RETURN}}}] \textbf{STRING} --- Blank if date cannot be formatted, or the date in the requested format.
\end{description}




\rule{\linewidth}{0.5pt}
\subsection*{\textsf{\colorbox{headtoc}{\color{white} FUNCTION}
TimeToString}}

\hypertarget{ecldoc:date.timetostring}{}
\hspace{0pt} \hyperlink{ecldoc:Date}{Date} \textbackslash 

{\renewcommand{\arraystretch}{1.5}
\begin{tabularx}{\textwidth}{|>{\raggedright\arraybackslash}l|X|}
\hline
\hspace{0pt}\mytexttt{\color{red} STRING} & \textbf{TimeToString} \\
\hline
\multicolumn{2}{|>{\raggedright\arraybackslash}X|}{\hspace{0pt}\mytexttt{\color{param} (Time\_t time, VARSTRING format = '\%H:\%M:\%S')}} \\
\hline
\end{tabularx}
}

\par





Formats a time as a string.






\par
\begin{description}
\item [\colorbox{tagtype}{\color{white} \textbf{\textsf{PARAMETER}}}] \textbf{\underline{time}} ||| UNSIGNED3 --- The time to be converted.
\item [\colorbox{tagtype}{\color{white} \textbf{\textsf{PARAMETER}}}] \textbf{\underline{format}} ||| VARSTRING --- The format template to use for the conversion; see strftime() for appropriate values. The maximum length of the resulting string is 255 characters. Optional; defaults to '\%H:\%M:\%S' which is HH:MM:SS.
\end{description}







\par
\begin{description}
\item [\colorbox{tagtype}{\color{white} \textbf{\textsf{RETURN}}}] \textbf{STRING} --- Blank if the time cannot be formatted, or the time in the requested format.
\end{description}




\rule{\linewidth}{0.5pt}
\subsection*{\textsf{\colorbox{headtoc}{\color{white} FUNCTION}
SecondsToString}}

\hypertarget{ecldoc:date.secondstostring}{}
\hspace{0pt} \hyperlink{ecldoc:Date}{Date} \textbackslash 

{\renewcommand{\arraystretch}{1.5}
\begin{tabularx}{\textwidth}{|>{\raggedright\arraybackslash}l|X|}
\hline
\hspace{0pt}\mytexttt{\color{red} STRING} & \textbf{SecondsToString} \\
\hline
\multicolumn{2}{|>{\raggedright\arraybackslash}X|}{\hspace{0pt}\mytexttt{\color{param} (Seconds\_t seconds, VARSTRING format = '\%Y-\%m-\%dT\%H:\%M:\%S')}} \\
\hline
\end{tabularx}
}

\par





Converts a Seconds\_t value into a human-readable string using a format template.






\par
\begin{description}
\item [\colorbox{tagtype}{\color{white} \textbf{\textsf{PARAMETER}}}] \textbf{\underline{format}} ||| VARSTRING --- The format template to use for the conversion; see strftime() for appropriate values. The maximum length of the resulting string is 255 characters. Optional; defaults to '\%Y-\%m-\%dT\%H:\%M:\%S' which is YYYY-MM-DDTHH:MM:SS.
\item [\colorbox{tagtype}{\color{white} \textbf{\textsf{PARAMETER}}}] \textbf{\underline{seconds}} ||| INTEGER8 --- The seconds since epoch.
\end{description}







\par
\begin{description}
\item [\colorbox{tagtype}{\color{white} \textbf{\textsf{RETURN}}}] \textbf{STRING} --- The converted seconds as a string.
\end{description}




\rule{\linewidth}{0.5pt}
\subsection*{\textsf{\colorbox{headtoc}{\color{white} FUNCTION}
ToString}}

\hypertarget{ecldoc:date.tostring}{}
\hspace{0pt} \hyperlink{ecldoc:Date}{Date} \textbackslash 

{\renewcommand{\arraystretch}{1.5}
\begin{tabularx}{\textwidth}{|>{\raggedright\arraybackslash}l|X|}
\hline
\hspace{0pt}\mytexttt{\color{red} STRING} & \textbf{ToString} \\
\hline
\multicolumn{2}{|>{\raggedright\arraybackslash}X|}{\hspace{0pt}\mytexttt{\color{param} (Date\_t date, VARSTRING format)}} \\
\hline
\end{tabularx}
}

\par





Formats a date as a string.






\par
\begin{description}
\item [\colorbox{tagtype}{\color{white} \textbf{\textsf{PARAMETER}}}] \textbf{\underline{date}} ||| UNSIGNED4 --- The date to be converted.
\item [\colorbox{tagtype}{\color{white} \textbf{\textsf{PARAMETER}}}] \textbf{\underline{format}} ||| VARSTRING --- The format the date is output in. (See documentation for strftime)
\end{description}







\par
\begin{description}
\item [\colorbox{tagtype}{\color{white} \textbf{\textsf{RETURN}}}] \textbf{STRING} --- Blank if date cannot be formatted, or the date in the requested format.
\end{description}




\rule{\linewidth}{0.5pt}
\subsection*{\textsf{\colorbox{headtoc}{\color{white} FUNCTION}
ConvertDateFormat}}

\hypertarget{ecldoc:date.convertdateformat}{}
\hspace{0pt} \hyperlink{ecldoc:Date}{Date} \textbackslash 

{\renewcommand{\arraystretch}{1.5}
\begin{tabularx}{\textwidth}{|>{\raggedright\arraybackslash}l|X|}
\hline
\hspace{0pt}\mytexttt{\color{red} STRING} & \textbf{ConvertDateFormat} \\
\hline
\multicolumn{2}{|>{\raggedright\arraybackslash}X|}{\hspace{0pt}\mytexttt{\color{param} (STRING date\_text, VARSTRING from\_format='\%m/\%d/\%Y', VARSTRING to\_format='\%Y\%m\%d')}} \\
\hline
\end{tabularx}
}

\par





Converts a date from one format to another






\par
\begin{description}
\item [\colorbox{tagtype}{\color{white} \textbf{\textsf{PARAMETER}}}] \textbf{\underline{from\_format}} ||| VARSTRING --- The format the date is to be converted from.
\item [\colorbox{tagtype}{\color{white} \textbf{\textsf{PARAMETER}}}] \textbf{\underline{date\_text}} ||| STRING --- The string containing the date to be converted.
\item [\colorbox{tagtype}{\color{white} \textbf{\textsf{PARAMETER}}}] \textbf{\underline{to\_format}} ||| VARSTRING --- The format the date is to be converted to.
\end{description}







\par
\begin{description}
\item [\colorbox{tagtype}{\color{white} \textbf{\textsf{RETURN}}}] \textbf{STRING} --- The converted string, or blank if it failed to match the format.
\end{description}




\rule{\linewidth}{0.5pt}
\subsection*{\textsf{\colorbox{headtoc}{\color{white} FUNCTION}
ConvertFormat}}

\hypertarget{ecldoc:date.convertformat}{}
\hspace{0pt} \hyperlink{ecldoc:Date}{Date} \textbackslash 

{\renewcommand{\arraystretch}{1.5}
\begin{tabularx}{\textwidth}{|>{\raggedright\arraybackslash}l|X|}
\hline
\hspace{0pt}\mytexttt{\color{red} STRING} & \textbf{ConvertFormat} \\
\hline
\multicolumn{2}{|>{\raggedright\arraybackslash}X|}{\hspace{0pt}\mytexttt{\color{param} (STRING date\_text, VARSTRING from\_format='\%m/\%d/\%Y', VARSTRING to\_format='\%Y\%m\%d')}} \\
\hline
\end{tabularx}
}

\par





Converts a date from one format to another






\par
\begin{description}
\item [\colorbox{tagtype}{\color{white} \textbf{\textsf{PARAMETER}}}] \textbf{\underline{from\_format}} ||| VARSTRING --- The format the date is to be converted from.
\item [\colorbox{tagtype}{\color{white} \textbf{\textsf{PARAMETER}}}] \textbf{\underline{date\_text}} ||| STRING --- The string containing the date to be converted.
\item [\colorbox{tagtype}{\color{white} \textbf{\textsf{PARAMETER}}}] \textbf{\underline{to\_format}} ||| VARSTRING --- The format the date is to be converted to.
\end{description}







\par
\begin{description}
\item [\colorbox{tagtype}{\color{white} \textbf{\textsf{RETURN}}}] \textbf{STRING} --- The converted string, or blank if it failed to match the format.
\end{description}




\rule{\linewidth}{0.5pt}
\subsection*{\textsf{\colorbox{headtoc}{\color{white} FUNCTION}
ConvertTimeFormat}}

\hypertarget{ecldoc:date.converttimeformat}{}
\hspace{0pt} \hyperlink{ecldoc:Date}{Date} \textbackslash 

{\renewcommand{\arraystretch}{1.5}
\begin{tabularx}{\textwidth}{|>{\raggedright\arraybackslash}l|X|}
\hline
\hspace{0pt}\mytexttt{\color{red} STRING} & \textbf{ConvertTimeFormat} \\
\hline
\multicolumn{2}{|>{\raggedright\arraybackslash}X|}{\hspace{0pt}\mytexttt{\color{param} (STRING time\_text, VARSTRING from\_format='\%H\%M\%S', VARSTRING to\_format='\%H:\%M:\%S')}} \\
\hline
\end{tabularx}
}

\par





Converts a time from one format to another






\par
\begin{description}
\item [\colorbox{tagtype}{\color{white} \textbf{\textsf{PARAMETER}}}] \textbf{\underline{from\_format}} ||| VARSTRING --- The format the time is to be converted from.
\item [\colorbox{tagtype}{\color{white} \textbf{\textsf{PARAMETER}}}] \textbf{\underline{to\_format}} ||| VARSTRING --- The format the time is to be converted to.
\item [\colorbox{tagtype}{\color{white} \textbf{\textsf{PARAMETER}}}] \textbf{\underline{time\_text}} ||| STRING --- The string containing the time to be converted.
\end{description}







\par
\begin{description}
\item [\colorbox{tagtype}{\color{white} \textbf{\textsf{RETURN}}}] \textbf{STRING} --- The converted string, or blank if it failed to match the format.
\end{description}




\rule{\linewidth}{0.5pt}
\subsection*{\textsf{\colorbox{headtoc}{\color{white} FUNCTION}
ConvertDateFormatMultiple}}

\hypertarget{ecldoc:date.convertdateformatmultiple}{}
\hspace{0pt} \hyperlink{ecldoc:Date}{Date} \textbackslash 

{\renewcommand{\arraystretch}{1.5}
\begin{tabularx}{\textwidth}{|>{\raggedright\arraybackslash}l|X|}
\hline
\hspace{0pt}\mytexttt{\color{red} STRING} & \textbf{ConvertDateFormatMultiple} \\
\hline
\multicolumn{2}{|>{\raggedright\arraybackslash}X|}{\hspace{0pt}\mytexttt{\color{param} (STRING date\_text, SET OF VARSTRING from\_formats, VARSTRING to\_format='\%Y\%m\%d')}} \\
\hline
\end{tabularx}
}

\par





Converts a date that matches one of a set of formats to another.






\par
\begin{description}
\item [\colorbox{tagtype}{\color{white} \textbf{\textsf{PARAMETER}}}] \textbf{\underline{from\_formats}} ||| SET ( VARSTRING ) --- The list of formats the date is to be converted from.
\item [\colorbox{tagtype}{\color{white} \textbf{\textsf{PARAMETER}}}] \textbf{\underline{date\_text}} ||| STRING --- The string containing the date to be converted.
\item [\colorbox{tagtype}{\color{white} \textbf{\textsf{PARAMETER}}}] \textbf{\underline{to\_format}} ||| VARSTRING --- The format the date is to be converted to.
\end{description}







\par
\begin{description}
\item [\colorbox{tagtype}{\color{white} \textbf{\textsf{RETURN}}}] \textbf{STRING} --- The converted string, or blank if it failed to match the format.
\end{description}




\rule{\linewidth}{0.5pt}
\subsection*{\textsf{\colorbox{headtoc}{\color{white} FUNCTION}
ConvertFormatMultiple}}

\hypertarget{ecldoc:date.convertformatmultiple}{}
\hspace{0pt} \hyperlink{ecldoc:Date}{Date} \textbackslash 

{\renewcommand{\arraystretch}{1.5}
\begin{tabularx}{\textwidth}{|>{\raggedright\arraybackslash}l|X|}
\hline
\hspace{0pt}\mytexttt{\color{red} STRING} & \textbf{ConvertFormatMultiple} \\
\hline
\multicolumn{2}{|>{\raggedright\arraybackslash}X|}{\hspace{0pt}\mytexttt{\color{param} (STRING date\_text, SET OF VARSTRING from\_formats, VARSTRING to\_format='\%Y\%m\%d')}} \\
\hline
\end{tabularx}
}

\par





Converts a date that matches one of a set of formats to another.






\par
\begin{description}
\item [\colorbox{tagtype}{\color{white} \textbf{\textsf{PARAMETER}}}] \textbf{\underline{from\_formats}} ||| SET ( VARSTRING ) --- The list of formats the date is to be converted from.
\item [\colorbox{tagtype}{\color{white} \textbf{\textsf{PARAMETER}}}] \textbf{\underline{date\_text}} ||| STRING --- The string containing the date to be converted.
\item [\colorbox{tagtype}{\color{white} \textbf{\textsf{PARAMETER}}}] \textbf{\underline{to\_format}} ||| VARSTRING --- The format the date is to be converted to.
\end{description}







\par
\begin{description}
\item [\colorbox{tagtype}{\color{white} \textbf{\textsf{RETURN}}}] \textbf{STRING} --- The converted string, or blank if it failed to match the format.
\end{description}




\rule{\linewidth}{0.5pt}
\subsection*{\textsf{\colorbox{headtoc}{\color{white} FUNCTION}
ConvertTimeFormatMultiple}}

\hypertarget{ecldoc:date.converttimeformatmultiple}{}
\hspace{0pt} \hyperlink{ecldoc:Date}{Date} \textbackslash 

{\renewcommand{\arraystretch}{1.5}
\begin{tabularx}{\textwidth}{|>{\raggedright\arraybackslash}l|X|}
\hline
\hspace{0pt}\mytexttt{\color{red} STRING} & \textbf{ConvertTimeFormatMultiple} \\
\hline
\multicolumn{2}{|>{\raggedright\arraybackslash}X|}{\hspace{0pt}\mytexttt{\color{param} (STRING time\_text, SET OF VARSTRING from\_formats, VARSTRING to\_format='\%H:\%m:\%s')}} \\
\hline
\end{tabularx}
}

\par





Converts a time that matches one of a set of formats to another.






\par
\begin{description}
\item [\colorbox{tagtype}{\color{white} \textbf{\textsf{PARAMETER}}}] \textbf{\underline{from\_formats}} ||| SET ( VARSTRING ) --- The list of formats the time is to be converted from.
\item [\colorbox{tagtype}{\color{white} \textbf{\textsf{PARAMETER}}}] \textbf{\underline{to\_format}} ||| VARSTRING --- The format the time is to be converted to.
\item [\colorbox{tagtype}{\color{white} \textbf{\textsf{PARAMETER}}}] \textbf{\underline{time\_text}} ||| STRING --- The string containing the time to be converted.
\end{description}







\par
\begin{description}
\item [\colorbox{tagtype}{\color{white} \textbf{\textsf{RETURN}}}] \textbf{STRING} --- The converted string, or blank if it failed to match the format.
\end{description}




\rule{\linewidth}{0.5pt}
\subsection*{\textsf{\colorbox{headtoc}{\color{white} FUNCTION}
AdjustDate}}

\hypertarget{ecldoc:date.adjustdate}{}
\hspace{0pt} \hyperlink{ecldoc:Date}{Date} \textbackslash 

{\renewcommand{\arraystretch}{1.5}
\begin{tabularx}{\textwidth}{|>{\raggedright\arraybackslash}l|X|}
\hline
\hspace{0pt}\mytexttt{\color{red} Date\_t} & \textbf{AdjustDate} \\
\hline
\multicolumn{2}{|>{\raggedright\arraybackslash}X|}{\hspace{0pt}\mytexttt{\color{param} (Date\_t date, INTEGER2 year\_delta = 0, INTEGER4 month\_delta = 0, INTEGER4 day\_delta = 0)}} \\
\hline
\end{tabularx}
}

\par





Adjusts a date by incrementing or decrementing year, month and/or day values. The date must be in the Gregorian calendar after the year 1600. If the new calculated date is invalid then it will be normalized according to mktime() rules. Example: 20140130 + 1 month = 20140302.






\par
\begin{description}
\item [\colorbox{tagtype}{\color{white} \textbf{\textsf{PARAMETER}}}] \textbf{\underline{date}} ||| UNSIGNED4 --- The date to adjust.
\item [\colorbox{tagtype}{\color{white} \textbf{\textsf{PARAMETER}}}] \textbf{\underline{month\_delta}} ||| INTEGER4 --- The requested change to the month value; optional, defaults to zero.
\item [\colorbox{tagtype}{\color{white} \textbf{\textsf{PARAMETER}}}] \textbf{\underline{year\_delta}} ||| INTEGER2 --- The requested change to the year value; optional, defaults to zero.
\item [\colorbox{tagtype}{\color{white} \textbf{\textsf{PARAMETER}}}] \textbf{\underline{day\_delta}} ||| INTEGER4 --- The requested change to the day of month value; optional, defaults to zero.
\end{description}







\par
\begin{description}
\item [\colorbox{tagtype}{\color{white} \textbf{\textsf{RETURN}}}] \textbf{UNSIGNED4} --- The adjusted Date\_t value.
\end{description}




\rule{\linewidth}{0.5pt}
\subsection*{\textsf{\colorbox{headtoc}{\color{white} FUNCTION}
AdjustDateBySeconds}}

\hypertarget{ecldoc:date.adjustdatebyseconds}{}
\hspace{0pt} \hyperlink{ecldoc:Date}{Date} \textbackslash 

{\renewcommand{\arraystretch}{1.5}
\begin{tabularx}{\textwidth}{|>{\raggedright\arraybackslash}l|X|}
\hline
\hspace{0pt}\mytexttt{\color{red} Date\_t} & \textbf{AdjustDateBySeconds} \\
\hline
\multicolumn{2}{|>{\raggedright\arraybackslash}X|}{\hspace{0pt}\mytexttt{\color{param} (Date\_t date, INTEGER4 seconds\_delta)}} \\
\hline
\end{tabularx}
}

\par





Adjusts a date by adding or subtracting seconds. The date must be in the Gregorian calendar after the year 1600. If the new calculated date is invalid then it will be normalized according to mktime() rules. Example: 20140130 + 172800 seconds = 20140201.






\par
\begin{description}
\item [\colorbox{tagtype}{\color{white} \textbf{\textsf{PARAMETER}}}] \textbf{\underline{date}} ||| UNSIGNED4 --- The date to adjust.
\item [\colorbox{tagtype}{\color{white} \textbf{\textsf{PARAMETER}}}] \textbf{\underline{seconds\_delta}} ||| INTEGER4 --- The requested change to the date, in seconds.
\end{description}







\par
\begin{description}
\item [\colorbox{tagtype}{\color{white} \textbf{\textsf{RETURN}}}] \textbf{UNSIGNED4} --- The adjusted Date\_t value.
\end{description}




\rule{\linewidth}{0.5pt}
\subsection*{\textsf{\colorbox{headtoc}{\color{white} FUNCTION}
AdjustTime}}

\hypertarget{ecldoc:date.adjusttime}{}
\hspace{0pt} \hyperlink{ecldoc:Date}{Date} \textbackslash 

{\renewcommand{\arraystretch}{1.5}
\begin{tabularx}{\textwidth}{|>{\raggedright\arraybackslash}l|X|}
\hline
\hspace{0pt}\mytexttt{\color{red} Time\_t} & \textbf{AdjustTime} \\
\hline
\multicolumn{2}{|>{\raggedright\arraybackslash}X|}{\hspace{0pt}\mytexttt{\color{param} (Time\_t time, INTEGER2 hour\_delta = 0, INTEGER4 minute\_delta = 0, INTEGER4 second\_delta = 0)}} \\
\hline
\end{tabularx}
}

\par





Adjusts a time by incrementing or decrementing hour, minute and/or second values. If the new calculated time is invalid then it will be normalized according to mktime() rules.






\par
\begin{description}
\item [\colorbox{tagtype}{\color{white} \textbf{\textsf{PARAMETER}}}] \textbf{\underline{time}} ||| UNSIGNED3 --- The time to adjust.
\item [\colorbox{tagtype}{\color{white} \textbf{\textsf{PARAMETER}}}] \textbf{\underline{second\_delta}} ||| INTEGER4 --- The requested change to the second of month value; optional, defaults to zero.
\item [\colorbox{tagtype}{\color{white} \textbf{\textsf{PARAMETER}}}] \textbf{\underline{hour\_delta}} ||| INTEGER2 --- The requested change to the hour value; optional, defaults to zero.
\item [\colorbox{tagtype}{\color{white} \textbf{\textsf{PARAMETER}}}] \textbf{\underline{minute\_delta}} ||| INTEGER4 --- The requested change to the minute value; optional, defaults to zero.
\end{description}







\par
\begin{description}
\item [\colorbox{tagtype}{\color{white} \textbf{\textsf{RETURN}}}] \textbf{UNSIGNED3} --- The adjusted Time\_t value.
\end{description}




\rule{\linewidth}{0.5pt}
\subsection*{\textsf{\colorbox{headtoc}{\color{white} FUNCTION}
AdjustTimeBySeconds}}

\hypertarget{ecldoc:date.adjusttimebyseconds}{}
\hspace{0pt} \hyperlink{ecldoc:Date}{Date} \textbackslash 

{\renewcommand{\arraystretch}{1.5}
\begin{tabularx}{\textwidth}{|>{\raggedright\arraybackslash}l|X|}
\hline
\hspace{0pt}\mytexttt{\color{red} Time\_t} & \textbf{AdjustTimeBySeconds} \\
\hline
\multicolumn{2}{|>{\raggedright\arraybackslash}X|}{\hspace{0pt}\mytexttt{\color{param} (Time\_t time, INTEGER4 seconds\_delta)}} \\
\hline
\end{tabularx}
}

\par





Adjusts a time by adding or subtracting seconds. If the new calculated time is invalid then it will be normalized according to mktime() rules.






\par
\begin{description}
\item [\colorbox{tagtype}{\color{white} \textbf{\textsf{PARAMETER}}}] \textbf{\underline{time}} ||| UNSIGNED3 --- The time to adjust.
\item [\colorbox{tagtype}{\color{white} \textbf{\textsf{PARAMETER}}}] \textbf{\underline{seconds\_delta}} ||| INTEGER4 --- The requested change to the time, in seconds.
\end{description}







\par
\begin{description}
\item [\colorbox{tagtype}{\color{white} \textbf{\textsf{RETURN}}}] \textbf{UNSIGNED3} --- The adjusted Time\_t value.
\end{description}




\rule{\linewidth}{0.5pt}
\subsection*{\textsf{\colorbox{headtoc}{\color{white} FUNCTION}
AdjustSeconds}}

\hypertarget{ecldoc:date.adjustseconds}{}
\hspace{0pt} \hyperlink{ecldoc:Date}{Date} \textbackslash 

{\renewcommand{\arraystretch}{1.5}
\begin{tabularx}{\textwidth}{|>{\raggedright\arraybackslash}l|X|}
\hline
\hspace{0pt}\mytexttt{\color{red} Seconds\_t} & \textbf{AdjustSeconds} \\
\hline
\multicolumn{2}{|>{\raggedright\arraybackslash}X|}{\hspace{0pt}\mytexttt{\color{param} (Seconds\_t seconds, INTEGER2 year\_delta = 0, INTEGER4 month\_delta = 0, INTEGER4 day\_delta = 0, INTEGER4 hour\_delta = 0, INTEGER4 minute\_delta = 0, INTEGER4 second\_delta = 0)}} \\
\hline
\end{tabularx}
}

\par





Adjusts a Seconds\_t value by adding or subtracting years, months, days, hours, minutes and/or seconds. This is performed by first converting the seconds into a full date/time structure, applying any delta values to individual date/time components, then converting the structure back to the number of seconds. This interim date must lie within Gregorian calendar after the year 1600. If the interim structure is found to have an invalid date/time then it will be normalized according to mktime() rules. Therefore, some delta values (such as ''1 month'') are actually relative to the value of the seconds argument.






\par
\begin{description}
\item [\colorbox{tagtype}{\color{white} \textbf{\textsf{PARAMETER}}}] \textbf{\underline{month\_delta}} ||| INTEGER4 --- The requested change to the month value; optional, defaults to zero.
\item [\colorbox{tagtype}{\color{white} \textbf{\textsf{PARAMETER}}}] \textbf{\underline{second\_delta}} ||| INTEGER4 --- The requested change to the second of month value; optional, defaults to zero.
\item [\colorbox{tagtype}{\color{white} \textbf{\textsf{PARAMETER}}}] \textbf{\underline{seconds}} ||| INTEGER8 --- The number of seconds to adjust.
\item [\colorbox{tagtype}{\color{white} \textbf{\textsf{PARAMETER}}}] \textbf{\underline{day\_delta}} ||| INTEGER4 --- The requested change to the day of month value; optional, defaults to zero.
\item [\colorbox{tagtype}{\color{white} \textbf{\textsf{PARAMETER}}}] \textbf{\underline{minute\_delta}} ||| INTEGER4 --- The requested change to the minute value; optional, defaults to zero.
\item [\colorbox{tagtype}{\color{white} \textbf{\textsf{PARAMETER}}}] \textbf{\underline{year\_delta}} ||| INTEGER2 --- The requested change to the year value; optional, defaults to zero.
\item [\colorbox{tagtype}{\color{white} \textbf{\textsf{PARAMETER}}}] \textbf{\underline{hour\_delta}} ||| INTEGER4 --- The requested change to the hour value; optional, defaults to zero.
\end{description}







\par
\begin{description}
\item [\colorbox{tagtype}{\color{white} \textbf{\textsf{RETURN}}}] \textbf{INTEGER8} --- The adjusted Seconds\_t value.
\end{description}




\rule{\linewidth}{0.5pt}
\subsection*{\textsf{\colorbox{headtoc}{\color{white} FUNCTION}
AdjustCalendar}}

\hypertarget{ecldoc:date.adjustcalendar}{}
\hspace{0pt} \hyperlink{ecldoc:Date}{Date} \textbackslash 

{\renewcommand{\arraystretch}{1.5}
\begin{tabularx}{\textwidth}{|>{\raggedright\arraybackslash}l|X|}
\hline
\hspace{0pt}\mytexttt{\color{red} Date\_t} & \textbf{AdjustCalendar} \\
\hline
\multicolumn{2}{|>{\raggedright\arraybackslash}X|}{\hspace{0pt}\mytexttt{\color{param} (Date\_t date, INTEGER2 year\_delta = 0, INTEGER4 month\_delta = 0, INTEGER4 day\_delta = 0)}} \\
\hline
\end{tabularx}
}

\par





Adjusts a date by incrementing or decrementing months and/or years. This routine uses the rule outlined in McGinn v. State, 46 Neb. 427, 65 N.W. 46 (1895): ''The term calendar month, whether employed in statutes or contracts, and not appearing to have been used in a different sense, denotes a period terminating with the day of the succeeding month numerically corresponding to the day of its beginning, less one. If there be no corresponding day of the succeeding month, it terminates with the last day thereof.'' The internet suggests similar legal positions exist in the Commonwealth and Germany. Note that day adjustments are performed after year and month adjustments using the preceding rules. As an example, Jan. 31, 2014 + 1 month will result in Feb. 28, 2014; Jan. 31, 2014 + 1 month + 1 day will result in Mar. 1, 2014.






\par
\begin{description}
\item [\colorbox{tagtype}{\color{white} \textbf{\textsf{PARAMETER}}}] \textbf{\underline{date}} ||| UNSIGNED4 --- The date to adjust, in the Gregorian calendar after 1600.
\item [\colorbox{tagtype}{\color{white} \textbf{\textsf{PARAMETER}}}] \textbf{\underline{month\_delta}} ||| INTEGER4 --- The requested change to the month value; optional, defaults to zero.
\item [\colorbox{tagtype}{\color{white} \textbf{\textsf{PARAMETER}}}] \textbf{\underline{year\_delta}} ||| INTEGER2 --- The requested change to the year value; optional, defaults to zero.
\item [\colorbox{tagtype}{\color{white} \textbf{\textsf{PARAMETER}}}] \textbf{\underline{day\_delta}} ||| INTEGER4 --- The requested change to the day value; optional, defaults to zero.
\end{description}







\par
\begin{description}
\item [\colorbox{tagtype}{\color{white} \textbf{\textsf{RETURN}}}] \textbf{UNSIGNED4} --- The adjusted Date\_t value.
\end{description}




\rule{\linewidth}{0.5pt}
\subsection*{\textsf{\colorbox{headtoc}{\color{white} FUNCTION}
IsLocalDaylightSavingsInEffect}}

\hypertarget{ecldoc:date.islocaldaylightsavingsineffect}{}
\hspace{0pt} \hyperlink{ecldoc:Date}{Date} \textbackslash 

{\renewcommand{\arraystretch}{1.5}
\begin{tabularx}{\textwidth}{|>{\raggedright\arraybackslash}l|X|}
\hline
\hspace{0pt}\mytexttt{\color{red} BOOLEAN} & \textbf{IsLocalDaylightSavingsInEffect} \\
\hline
\multicolumn{2}{|>{\raggedright\arraybackslash}X|}{\hspace{0pt}\mytexttt{\color{param} ()}} \\
\hline
\end{tabularx}
}

\par





Returns a boolean indicating whether daylight savings time is currently in effect locally.








\par
\begin{description}
\item [\colorbox{tagtype}{\color{white} \textbf{\textsf{RETURN}}}] \textbf{BOOLEAN} --- TRUE if daylight savings time is currently in effect, FALSE otherwise.
\end{description}




\rule{\linewidth}{0.5pt}
\subsection*{\textsf{\colorbox{headtoc}{\color{white} FUNCTION}
LocalTimeZoneOffset}}

\hypertarget{ecldoc:date.localtimezoneoffset}{}
\hspace{0pt} \hyperlink{ecldoc:Date}{Date} \textbackslash 

{\renewcommand{\arraystretch}{1.5}
\begin{tabularx}{\textwidth}{|>{\raggedright\arraybackslash}l|X|}
\hline
\hspace{0pt}\mytexttt{\color{red} INTEGER4} & \textbf{LocalTimeZoneOffset} \\
\hline
\multicolumn{2}{|>{\raggedright\arraybackslash}X|}{\hspace{0pt}\mytexttt{\color{param} ()}} \\
\hline
\end{tabularx}
}

\par





Returns the offset (in seconds) of the time represented from UTC, with positive values indicating locations east of the Prime Meridian. Given a UTC time in seconds since epoch, you can find the local time by adding the result of this function to the seconds.








\par
\begin{description}
\item [\colorbox{tagtype}{\color{white} \textbf{\textsf{RETURN}}}] \textbf{INTEGER4} --- The number of seconds offset from UTC.
\end{description}




\rule{\linewidth}{0.5pt}
\subsection*{\textsf{\colorbox{headtoc}{\color{white} FUNCTION}
CurrentDate}}

\hypertarget{ecldoc:date.currentdate}{}
\hspace{0pt} \hyperlink{ecldoc:Date}{Date} \textbackslash 

{\renewcommand{\arraystretch}{1.5}
\begin{tabularx}{\textwidth}{|>{\raggedright\arraybackslash}l|X|}
\hline
\hspace{0pt}\mytexttt{\color{red} Date\_t} & \textbf{CurrentDate} \\
\hline
\multicolumn{2}{|>{\raggedright\arraybackslash}X|}{\hspace{0pt}\mytexttt{\color{param} (BOOLEAN in\_local\_time = FALSE)}} \\
\hline
\end{tabularx}
}

\par





Returns the current date.






\par
\begin{description}
\item [\colorbox{tagtype}{\color{white} \textbf{\textsf{PARAMETER}}}] \textbf{\underline{in\_local\_time}} ||| BOOLEAN --- TRUE if the returned value should be local to the cluster computing the date, FALSE for UTC. Optional, defaults to FALSE.
\end{description}







\par
\begin{description}
\item [\colorbox{tagtype}{\color{white} \textbf{\textsf{RETURN}}}] \textbf{UNSIGNED4} --- A Date\_t representing the current date.
\end{description}




\rule{\linewidth}{0.5pt}
\subsection*{\textsf{\colorbox{headtoc}{\color{white} FUNCTION}
Today}}

\hypertarget{ecldoc:date.today}{}
\hspace{0pt} \hyperlink{ecldoc:Date}{Date} \textbackslash 

{\renewcommand{\arraystretch}{1.5}
\begin{tabularx}{\textwidth}{|>{\raggedright\arraybackslash}l|X|}
\hline
\hspace{0pt}\mytexttt{\color{red} Date\_t} & \textbf{Today} \\
\hline
\multicolumn{2}{|>{\raggedright\arraybackslash}X|}{\hspace{0pt}\mytexttt{\color{param} ()}} \\
\hline
\end{tabularx}
}

\par





Returns the current date in the local time zone.








\par
\begin{description}
\item [\colorbox{tagtype}{\color{white} \textbf{\textsf{RETURN}}}] \textbf{UNSIGNED4} --- A Date\_t representing the current date.
\end{description}




\rule{\linewidth}{0.5pt}
\subsection*{\textsf{\colorbox{headtoc}{\color{white} FUNCTION}
CurrentTime}}

\hypertarget{ecldoc:date.currenttime}{}
\hspace{0pt} \hyperlink{ecldoc:Date}{Date} \textbackslash 

{\renewcommand{\arraystretch}{1.5}
\begin{tabularx}{\textwidth}{|>{\raggedright\arraybackslash}l|X|}
\hline
\hspace{0pt}\mytexttt{\color{red} Time\_t} & \textbf{CurrentTime} \\
\hline
\multicolumn{2}{|>{\raggedright\arraybackslash}X|}{\hspace{0pt}\mytexttt{\color{param} (BOOLEAN in\_local\_time = FALSE)}} \\
\hline
\end{tabularx}
}

\par





Returns the current time of day






\par
\begin{description}
\item [\colorbox{tagtype}{\color{white} \textbf{\textsf{PARAMETER}}}] \textbf{\underline{in\_local\_time}} ||| BOOLEAN --- TRUE if the returned value should be local to the cluster computing the time, FALSE for UTC. Optional, defaults to FALSE.
\end{description}







\par
\begin{description}
\item [\colorbox{tagtype}{\color{white} \textbf{\textsf{RETURN}}}] \textbf{UNSIGNED3} --- A Time\_t representing the current time of day.
\end{description}




\rule{\linewidth}{0.5pt}
\subsection*{\textsf{\colorbox{headtoc}{\color{white} FUNCTION}
CurrentSeconds}}

\hypertarget{ecldoc:date.currentseconds}{}
\hspace{0pt} \hyperlink{ecldoc:Date}{Date} \textbackslash 

{\renewcommand{\arraystretch}{1.5}
\begin{tabularx}{\textwidth}{|>{\raggedright\arraybackslash}l|X|}
\hline
\hspace{0pt}\mytexttt{\color{red} Seconds\_t} & \textbf{CurrentSeconds} \\
\hline
\multicolumn{2}{|>{\raggedright\arraybackslash}X|}{\hspace{0pt}\mytexttt{\color{param} (BOOLEAN in\_local\_time = FALSE)}} \\
\hline
\end{tabularx}
}

\par





Returns the current date and time as the number of seconds since epoch.






\par
\begin{description}
\item [\colorbox{tagtype}{\color{white} \textbf{\textsf{PARAMETER}}}] \textbf{\underline{in\_local\_time}} ||| BOOLEAN --- TRUE if the returned value should be local to the cluster computing the time, FALSE for UTC. Optional, defaults to FALSE.
\end{description}







\par
\begin{description}
\item [\colorbox{tagtype}{\color{white} \textbf{\textsf{RETURN}}}] \textbf{INTEGER8} --- A Seconds\_t representing the current time in UTC or local time, depending on the argument.
\end{description}




\rule{\linewidth}{0.5pt}
\subsection*{\textsf{\colorbox{headtoc}{\color{white} FUNCTION}
CurrentTimestamp}}

\hypertarget{ecldoc:date.currenttimestamp}{}
\hspace{0pt} \hyperlink{ecldoc:Date}{Date} \textbackslash 

{\renewcommand{\arraystretch}{1.5}
\begin{tabularx}{\textwidth}{|>{\raggedright\arraybackslash}l|X|}
\hline
\hspace{0pt}\mytexttt{\color{red} Timestamp\_t} & \textbf{CurrentTimestamp} \\
\hline
\multicolumn{2}{|>{\raggedright\arraybackslash}X|}{\hspace{0pt}\mytexttt{\color{param} (BOOLEAN in\_local\_time = FALSE)}} \\
\hline
\end{tabularx}
}

\par





Returns the current date and time as the number of microseconds since epoch.






\par
\begin{description}
\item [\colorbox{tagtype}{\color{white} \textbf{\textsf{PARAMETER}}}] \textbf{\underline{in\_local\_time}} ||| BOOLEAN --- TRUE if the returned value should be local to the cluster computing the time, FALSE for UTC. Optional, defaults to FALSE.
\end{description}







\par
\begin{description}
\item [\colorbox{tagtype}{\color{white} \textbf{\textsf{RETURN}}}] \textbf{INTEGER8} --- A Timestamp\_t representing the current time in microseconds in UTC or local time, depending on the argument.
\end{description}




\rule{\linewidth}{0.5pt}
\subsection*{\textsf{\colorbox{headtoc}{\color{white} MODULE}
DatesForMonth}}

\hypertarget{ecldoc:date.datesformonth}{}
\hspace{0pt} \hyperlink{ecldoc:Date}{Date} \textbackslash 

{\renewcommand{\arraystretch}{1.5}
\begin{tabularx}{\textwidth}{|>{\raggedright\arraybackslash}l|X|}
\hline
\hspace{0pt}\mytexttt{\color{red} } & \textbf{DatesForMonth} \\
\hline
\multicolumn{2}{|>{\raggedright\arraybackslash}X|}{\hspace{0pt}\mytexttt{\color{param} (Date\_t as\_of\_date = CurrentDate(FALSE))}} \\
\hline
\end{tabularx}
}

\par





Returns the beginning and ending dates for the month surrounding the given date.






\par
\begin{description}
\item [\colorbox{tagtype}{\color{white} \textbf{\textsf{PARAMETER}}}] \textbf{\underline{as\_of\_date}} ||| UNSIGNED4 --- The reference date from which the month will be calculated. This date must be a date within the Gregorian calendar. Optional, defaults to the current date in UTC.
\end{description}







\par
\begin{description}
\item [\colorbox{tagtype}{\color{white} \textbf{\textsf{RETURN}}}] \textbf{} --- Module with exported attributes for startDate and endDate.
\end{description}




\textbf{Children}
\begin{enumerate}
\item \hyperlink{ecldoc:date.datesformonth.result.startdate}{startDate}
: No Documentation Found
\item \hyperlink{ecldoc:date.datesformonth.result.enddate}{endDate}
: No Documentation Found
\end{enumerate}

\rule{\linewidth}{0.5pt}

\subsection*{\textsf{\colorbox{headtoc}{\color{white} ATTRIBUTE}
startDate}}

\hypertarget{ecldoc:date.datesformonth.result.startdate}{}
\hspace{0pt} \hyperlink{ecldoc:Date}{Date} \textbackslash 
\hspace{0pt} \hyperlink{ecldoc:date.datesformonth}{DatesForMonth} \textbackslash 

{\renewcommand{\arraystretch}{1.5}
\begin{tabularx}{\textwidth}{|>{\raggedright\arraybackslash}l|X|}
\hline
\hspace{0pt}\mytexttt{\color{red} Date\_t} & \textbf{startDate} \\
\hline
\end{tabularx}
}

\par





No Documentation Found








\par
\begin{description}
\item [\colorbox{tagtype}{\color{white} \textbf{\textsf{RETURN}}}] \textbf{UNSIGNED4} --- 
\end{description}




\rule{\linewidth}{0.5pt}
\subsection*{\textsf{\colorbox{headtoc}{\color{white} ATTRIBUTE}
endDate}}

\hypertarget{ecldoc:date.datesformonth.result.enddate}{}
\hspace{0pt} \hyperlink{ecldoc:Date}{Date} \textbackslash 
\hspace{0pt} \hyperlink{ecldoc:date.datesformonth}{DatesForMonth} \textbackslash 

{\renewcommand{\arraystretch}{1.5}
\begin{tabularx}{\textwidth}{|>{\raggedright\arraybackslash}l|X|}
\hline
\hspace{0pt}\mytexttt{\color{red} Date\_t} & \textbf{endDate} \\
\hline
\end{tabularx}
}

\par





No Documentation Found








\par
\begin{description}
\item [\colorbox{tagtype}{\color{white} \textbf{\textsf{RETURN}}}] \textbf{UNSIGNED4} --- 
\end{description}




\rule{\linewidth}{0.5pt}


\subsection*{\textsf{\colorbox{headtoc}{\color{white} MODULE}
DatesForWeek}}

\hypertarget{ecldoc:date.datesforweek}{}
\hspace{0pt} \hyperlink{ecldoc:Date}{Date} \textbackslash 

{\renewcommand{\arraystretch}{1.5}
\begin{tabularx}{\textwidth}{|>{\raggedright\arraybackslash}l|X|}
\hline
\hspace{0pt}\mytexttt{\color{red} } & \textbf{DatesForWeek} \\
\hline
\multicolumn{2}{|>{\raggedright\arraybackslash}X|}{\hspace{0pt}\mytexttt{\color{param} (Date\_t as\_of\_date = CurrentDate(FALSE))}} \\
\hline
\end{tabularx}
}

\par





Returns the beginning and ending dates for the week surrounding the given date (Sunday marks the beginning of a week).






\par
\begin{description}
\item [\colorbox{tagtype}{\color{white} \textbf{\textsf{PARAMETER}}}] \textbf{\underline{as\_of\_date}} ||| UNSIGNED4 --- The reference date from which the week will be calculated. This date must be a date within the Gregorian calendar. Optional, defaults to the current date in UTC.
\end{description}







\par
\begin{description}
\item [\colorbox{tagtype}{\color{white} \textbf{\textsf{RETURN}}}] \textbf{} --- Module with exported attributes for startDate and endDate.
\end{description}




\textbf{Children}
\begin{enumerate}
\item \hyperlink{ecldoc:date.datesforweek.result.startdate}{startDate}
: No Documentation Found
\item \hyperlink{ecldoc:date.datesforweek.result.enddate}{endDate}
: No Documentation Found
\end{enumerate}

\rule{\linewidth}{0.5pt}

\subsection*{\textsf{\colorbox{headtoc}{\color{white} ATTRIBUTE}
startDate}}

\hypertarget{ecldoc:date.datesforweek.result.startdate}{}
\hspace{0pt} \hyperlink{ecldoc:Date}{Date} \textbackslash 
\hspace{0pt} \hyperlink{ecldoc:date.datesforweek}{DatesForWeek} \textbackslash 

{\renewcommand{\arraystretch}{1.5}
\begin{tabularx}{\textwidth}{|>{\raggedright\arraybackslash}l|X|}
\hline
\hspace{0pt}\mytexttt{\color{red} Date\_t} & \textbf{startDate} \\
\hline
\end{tabularx}
}

\par





No Documentation Found








\par
\begin{description}
\item [\colorbox{tagtype}{\color{white} \textbf{\textsf{RETURN}}}] \textbf{UNSIGNED4} --- 
\end{description}




\rule{\linewidth}{0.5pt}
\subsection*{\textsf{\colorbox{headtoc}{\color{white} ATTRIBUTE}
endDate}}

\hypertarget{ecldoc:date.datesforweek.result.enddate}{}
\hspace{0pt} \hyperlink{ecldoc:Date}{Date} \textbackslash 
\hspace{0pt} \hyperlink{ecldoc:date.datesforweek}{DatesForWeek} \textbackslash 

{\renewcommand{\arraystretch}{1.5}
\begin{tabularx}{\textwidth}{|>{\raggedright\arraybackslash}l|X|}
\hline
\hspace{0pt}\mytexttt{\color{red} Date\_t} & \textbf{endDate} \\
\hline
\end{tabularx}
}

\par





No Documentation Found








\par
\begin{description}
\item [\colorbox{tagtype}{\color{white} \textbf{\textsf{RETURN}}}] \textbf{UNSIGNED4} --- 
\end{description}




\rule{\linewidth}{0.5pt}


\subsection*{\textsf{\colorbox{headtoc}{\color{white} FUNCTION}
IsValidDate}}

\hypertarget{ecldoc:date.isvaliddate}{}
\hspace{0pt} \hyperlink{ecldoc:Date}{Date} \textbackslash 

{\renewcommand{\arraystretch}{1.5}
\begin{tabularx}{\textwidth}{|>{\raggedright\arraybackslash}l|X|}
\hline
\hspace{0pt}\mytexttt{\color{red} BOOLEAN} & \textbf{IsValidDate} \\
\hline
\multicolumn{2}{|>{\raggedright\arraybackslash}X|}{\hspace{0pt}\mytexttt{\color{param} (Date\_t date, INTEGER2 yearLowerBound = 1800, INTEGER2 yearUpperBound = 2100)}} \\
\hline
\end{tabularx}
}

\par





Tests whether a date is valid, both by range-checking the year and by validating each of the other individual components.






\par
\begin{description}
\item [\colorbox{tagtype}{\color{white} \textbf{\textsf{PARAMETER}}}] \textbf{\underline{date}} ||| UNSIGNED4 --- The date to validate.
\item [\colorbox{tagtype}{\color{white} \textbf{\textsf{PARAMETER}}}] \textbf{\underline{yearUpperBound}} ||| INTEGER2 --- The maximum acceptable year. Optional; defaults to 2100.
\item [\colorbox{tagtype}{\color{white} \textbf{\textsf{PARAMETER}}}] \textbf{\underline{yearLowerBound}} ||| INTEGER2 --- The minimum acceptable year. Optional; defaults to 1800.
\end{description}







\par
\begin{description}
\item [\colorbox{tagtype}{\color{white} \textbf{\textsf{RETURN}}}] \textbf{BOOLEAN} --- TRUE if the date is valid, FALSE otherwise.
\end{description}




\rule{\linewidth}{0.5pt}
\subsection*{\textsf{\colorbox{headtoc}{\color{white} FUNCTION}
IsValidGregorianDate}}

\hypertarget{ecldoc:date.isvalidgregoriandate}{}
\hspace{0pt} \hyperlink{ecldoc:Date}{Date} \textbackslash 

{\renewcommand{\arraystretch}{1.5}
\begin{tabularx}{\textwidth}{|>{\raggedright\arraybackslash}l|X|}
\hline
\hspace{0pt}\mytexttt{\color{red} BOOLEAN} & \textbf{IsValidGregorianDate} \\
\hline
\multicolumn{2}{|>{\raggedright\arraybackslash}X|}{\hspace{0pt}\mytexttt{\color{param} (Date\_t date)}} \\
\hline
\end{tabularx}
}

\par





Tests whether a date is valid in the Gregorian calendar. The year must be between 1601 and 30827.






\par
\begin{description}
\item [\colorbox{tagtype}{\color{white} \textbf{\textsf{PARAMETER}}}] \textbf{\underline{date}} ||| UNSIGNED4 --- The Date\_t to validate.
\end{description}







\par
\begin{description}
\item [\colorbox{tagtype}{\color{white} \textbf{\textsf{RETURN}}}] \textbf{BOOLEAN} --- TRUE if the date is valid, FALSE otherwise.
\end{description}




\rule{\linewidth}{0.5pt}
\subsection*{\textsf{\colorbox{headtoc}{\color{white} FUNCTION}
IsValidTime}}

\hypertarget{ecldoc:date.isvalidtime}{}
\hspace{0pt} \hyperlink{ecldoc:Date}{Date} \textbackslash 

{\renewcommand{\arraystretch}{1.5}
\begin{tabularx}{\textwidth}{|>{\raggedright\arraybackslash}l|X|}
\hline
\hspace{0pt}\mytexttt{\color{red} BOOLEAN} & \textbf{IsValidTime} \\
\hline
\multicolumn{2}{|>{\raggedright\arraybackslash}X|}{\hspace{0pt}\mytexttt{\color{param} (Time\_t time)}} \\
\hline
\end{tabularx}
}

\par





Tests whether a time is valid.






\par
\begin{description}
\item [\colorbox{tagtype}{\color{white} \textbf{\textsf{PARAMETER}}}] \textbf{\underline{time}} ||| UNSIGNED3 --- The time to validate.
\end{description}







\par
\begin{description}
\item [\colorbox{tagtype}{\color{white} \textbf{\textsf{RETURN}}}] \textbf{BOOLEAN} --- TRUE if the time is valid, FALSE otherwise.
\end{description}




\rule{\linewidth}{0.5pt}
\subsection*{\textsf{\colorbox{headtoc}{\color{white} TRANSFORM}
CreateDate}}

\hypertarget{ecldoc:date.createdate}{}
\hspace{0pt} \hyperlink{ecldoc:Date}{Date} \textbackslash 

{\renewcommand{\arraystretch}{1.5}
\begin{tabularx}{\textwidth}{|>{\raggedright\arraybackslash}l|X|}
\hline
\hspace{0pt}\mytexttt{\color{red} Date\_rec} & \textbf{CreateDate} \\
\hline
\multicolumn{2}{|>{\raggedright\arraybackslash}X|}{\hspace{0pt}\mytexttt{\color{param} (INTEGER2 year, UNSIGNED1 month, UNSIGNED1 day)}} \\
\hline
\end{tabularx}
}

\par





A transform to create a Date\_rec from the individual elements






\par
\begin{description}
\item [\colorbox{tagtype}{\color{white} \textbf{\textsf{PARAMETER}}}] \textbf{\underline{year}} ||| INTEGER2 --- The year
\item [\colorbox{tagtype}{\color{white} \textbf{\textsf{PARAMETER}}}] \textbf{\underline{month}} ||| UNSIGNED1 --- The month (1-12).
\item [\colorbox{tagtype}{\color{white} \textbf{\textsf{PARAMETER}}}] \textbf{\underline{day}} ||| UNSIGNED1 --- The day (1..daysInMonth).
\end{description}







\par
\begin{description}
\item [\colorbox{tagtype}{\color{white} \textbf{\textsf{RETURN}}}] \textbf{Date\_rec} --- A transform that creates a Date\_rec containing the date.
\end{description}




\rule{\linewidth}{0.5pt}
\subsection*{\textsf{\colorbox{headtoc}{\color{white} TRANSFORM}
CreateDateFromSeconds}}

\hypertarget{ecldoc:date.createdatefromseconds}{}
\hspace{0pt} \hyperlink{ecldoc:Date}{Date} \textbackslash 

{\renewcommand{\arraystretch}{1.5}
\begin{tabularx}{\textwidth}{|>{\raggedright\arraybackslash}l|X|}
\hline
\hspace{0pt}\mytexttt{\color{red} Date\_rec} & \textbf{CreateDateFromSeconds} \\
\hline
\multicolumn{2}{|>{\raggedright\arraybackslash}X|}{\hspace{0pt}\mytexttt{\color{param} (Seconds\_t seconds)}} \\
\hline
\end{tabularx}
}

\par





A transform to create a Date\_rec from a Seconds\_t value.






\par
\begin{description}
\item [\colorbox{tagtype}{\color{white} \textbf{\textsf{PARAMETER}}}] \textbf{\underline{seconds}} ||| INTEGER8 --- The number seconds since epoch.
\end{description}







\par
\begin{description}
\item [\colorbox{tagtype}{\color{white} \textbf{\textsf{RETURN}}}] \textbf{Date\_rec} --- A transform that creates a Date\_rec containing the date.
\end{description}




\rule{\linewidth}{0.5pt}
\subsection*{\textsf{\colorbox{headtoc}{\color{white} TRANSFORM}
CreateTime}}

\hypertarget{ecldoc:date.createtime}{}
\hspace{0pt} \hyperlink{ecldoc:Date}{Date} \textbackslash 

{\renewcommand{\arraystretch}{1.5}
\begin{tabularx}{\textwidth}{|>{\raggedright\arraybackslash}l|X|}
\hline
\hspace{0pt}\mytexttt{\color{red} Time\_rec} & \textbf{CreateTime} \\
\hline
\multicolumn{2}{|>{\raggedright\arraybackslash}X|}{\hspace{0pt}\mytexttt{\color{param} (UNSIGNED1 hour, UNSIGNED1 minute, UNSIGNED1 second)}} \\
\hline
\end{tabularx}
}

\par





A transform to create a Time\_rec from the individual elements






\par
\begin{description}
\item [\colorbox{tagtype}{\color{white} \textbf{\textsf{PARAMETER}}}] \textbf{\underline{minute}} ||| UNSIGNED1 --- The minute (0-59).
\item [\colorbox{tagtype}{\color{white} \textbf{\textsf{PARAMETER}}}] \textbf{\underline{second}} ||| UNSIGNED1 --- The second (0-59).
\item [\colorbox{tagtype}{\color{white} \textbf{\textsf{PARAMETER}}}] \textbf{\underline{hour}} ||| UNSIGNED1 --- The hour (0-23).
\end{description}







\par
\begin{description}
\item [\colorbox{tagtype}{\color{white} \textbf{\textsf{RETURN}}}] \textbf{Time\_rec} --- A transform that creates a Time\_rec containing the time of day.
\end{description}




\rule{\linewidth}{0.5pt}
\subsection*{\textsf{\colorbox{headtoc}{\color{white} TRANSFORM}
CreateTimeFromSeconds}}

\hypertarget{ecldoc:date.createtimefromseconds}{}
\hspace{0pt} \hyperlink{ecldoc:Date}{Date} \textbackslash 

{\renewcommand{\arraystretch}{1.5}
\begin{tabularx}{\textwidth}{|>{\raggedright\arraybackslash}l|X|}
\hline
\hspace{0pt}\mytexttt{\color{red} Time\_rec} & \textbf{CreateTimeFromSeconds} \\
\hline
\multicolumn{2}{|>{\raggedright\arraybackslash}X|}{\hspace{0pt}\mytexttt{\color{param} (Seconds\_t seconds)}} \\
\hline
\end{tabularx}
}

\par





A transform to create a Time\_rec from a Seconds\_t value.






\par
\begin{description}
\item [\colorbox{tagtype}{\color{white} \textbf{\textsf{PARAMETER}}}] \textbf{\underline{seconds}} ||| INTEGER8 --- The number seconds since epoch.
\end{description}







\par
\begin{description}
\item [\colorbox{tagtype}{\color{white} \textbf{\textsf{RETURN}}}] \textbf{Time\_rec} --- A transform that creates a Time\_rec containing the time of day.
\end{description}




\rule{\linewidth}{0.5pt}
\subsection*{\textsf{\colorbox{headtoc}{\color{white} TRANSFORM}
CreateDateTime}}

\hypertarget{ecldoc:date.createdatetime}{}
\hspace{0pt} \hyperlink{ecldoc:Date}{Date} \textbackslash 

{\renewcommand{\arraystretch}{1.5}
\begin{tabularx}{\textwidth}{|>{\raggedright\arraybackslash}l|X|}
\hline
\hspace{0pt}\mytexttt{\color{red} DateTime\_rec} & \textbf{CreateDateTime} \\
\hline
\multicolumn{2}{|>{\raggedright\arraybackslash}X|}{\hspace{0pt}\mytexttt{\color{param} (INTEGER2 year, UNSIGNED1 month, UNSIGNED1 day, UNSIGNED1 hour, UNSIGNED1 minute, UNSIGNED1 second)}} \\
\hline
\end{tabularx}
}

\par





A transform to create a DateTime\_rec from the individual elements






\par
\begin{description}
\item [\colorbox{tagtype}{\color{white} \textbf{\textsf{PARAMETER}}}] \textbf{\underline{year}} ||| INTEGER2 --- The year
\item [\colorbox{tagtype}{\color{white} \textbf{\textsf{PARAMETER}}}] \textbf{\underline{second}} ||| UNSIGNED1 --- The second (0-59).
\item [\colorbox{tagtype}{\color{white} \textbf{\textsf{PARAMETER}}}] \textbf{\underline{hour}} ||| UNSIGNED1 --- The hour (0-23).
\item [\colorbox{tagtype}{\color{white} \textbf{\textsf{PARAMETER}}}] \textbf{\underline{minute}} ||| UNSIGNED1 --- The minute (0-59).
\item [\colorbox{tagtype}{\color{white} \textbf{\textsf{PARAMETER}}}] \textbf{\underline{month}} ||| UNSIGNED1 --- The month (1-12).
\item [\colorbox{tagtype}{\color{white} \textbf{\textsf{PARAMETER}}}] \textbf{\underline{day}} ||| UNSIGNED1 --- The day (1..daysInMonth).
\end{description}







\par
\begin{description}
\item [\colorbox{tagtype}{\color{white} \textbf{\textsf{RETURN}}}] \textbf{DateTime\_rec} --- A transform that creates a DateTime\_rec containing date and time components.
\end{description}




\rule{\linewidth}{0.5pt}
\subsection*{\textsf{\colorbox{headtoc}{\color{white} TRANSFORM}
CreateDateTimeFromSeconds}}

\hypertarget{ecldoc:date.createdatetimefromseconds}{}
\hspace{0pt} \hyperlink{ecldoc:Date}{Date} \textbackslash 

{\renewcommand{\arraystretch}{1.5}
\begin{tabularx}{\textwidth}{|>{\raggedright\arraybackslash}l|X|}
\hline
\hspace{0pt}\mytexttt{\color{red} DateTime\_rec} & \textbf{CreateDateTimeFromSeconds} \\
\hline
\multicolumn{2}{|>{\raggedright\arraybackslash}X|}{\hspace{0pt}\mytexttt{\color{param} (Seconds\_t seconds)}} \\
\hline
\end{tabularx}
}

\par





A transform to create a DateTime\_rec from a Seconds\_t value.






\par
\begin{description}
\item [\colorbox{tagtype}{\color{white} \textbf{\textsf{PARAMETER}}}] \textbf{\underline{seconds}} ||| INTEGER8 --- The number seconds since epoch.
\end{description}







\par
\begin{description}
\item [\colorbox{tagtype}{\color{white} \textbf{\textsf{RETURN}}}] \textbf{DateTime\_rec} --- A transform that creates a DateTime\_rec containing date and time components.
\end{description}




\rule{\linewidth}{0.5pt}



\chapter*{\color{headfile}
File
}
\hypertarget{ecldoc:toc:File}{}
\hyperlink{ecldoc:toc:root}{Go Up}

\section*{\underline{\textsf{IMPORTS}}}
\begin{doublespace}
{\large
lib\_fileservices |
}
\end{doublespace}

\section*{\underline{\textsf{DESCRIPTIONS}}}
\subsection*{\textsf{\colorbox{headtoc}{\color{white} MODULE}
File}}

\hypertarget{ecldoc:File}{}

{\renewcommand{\arraystretch}{1.5}
\begin{tabularx}{\textwidth}{|>{\raggedright\arraybackslash}l|X|}
\hline
\hspace{0pt}\mytexttt{\color{red} } & \textbf{File} \\
\hline
\end{tabularx}
}

\par


\textbf{Children}
\begin{enumerate}
\item \hyperlink{ecldoc:file.fsfilenamerecord}{FsFilenameRecord}
: A record containing information about filename
\item \hyperlink{ecldoc:file.fslogicalfilename}{FsLogicalFileName}
: An alias for a logical filename that is stored in a row
\item \hyperlink{ecldoc:file.fslogicalfilenamerecord}{FsLogicalFileNameRecord}
: A record containing a logical filename
\item \hyperlink{ecldoc:file.fslogicalfileinforecord}{FsLogicalFileInfoRecord}
: A record containing information about a logical file
\item \hyperlink{ecldoc:file.fslogicalsupersubrecord}{FsLogicalSuperSubRecord}
: A record containing information about a superfile and its contents
\item \hyperlink{ecldoc:file.fsfilerelationshiprecord}{FsFileRelationshipRecord}
: A record containing information about the relationship between two files
\item \hyperlink{ecldoc:file.recfmv_recsize}{RECFMV\_RECSIZE}
: Constant that indicates IBM RECFM V format file
\item \hyperlink{ecldoc:file.recfmvb_recsize}{RECFMVB\_RECSIZE}
: Constant that indicates IBM RECFM VB format file
\item \hyperlink{ecldoc:file.prefix_variable_recsize}{PREFIX\_VARIABLE\_RECSIZE}
: Constant that indicates a variable little endian 4 byte length prefixed file
\item \hyperlink{ecldoc:file.prefix_variable_bigendian_recsize}{PREFIX\_VARIABLE\_BIGENDIAN\_RECSIZE}
: Constant that indicates a variable big endian 4 byte length prefixed file
\item \hyperlink{ecldoc:file.fileexists}{FileExists}
: Returns whether the file exists
\item \hyperlink{ecldoc:file.deletelogicalfile}{DeleteLogicalFile}
: Removes the logical file from the system, and deletes from the disk
\item \hyperlink{ecldoc:file.setreadonly}{SetReadOnly}
: Changes whether access to a file is read only or not
\item \hyperlink{ecldoc:file.renamelogicalfile}{RenameLogicalFile}
: Changes the name of a logical file
\item \hyperlink{ecldoc:file.foreignlogicalfilename}{ForeignLogicalFileName}
: Returns a logical filename that can be used to refer to a logical file in a local or remote dali
\item \hyperlink{ecldoc:file.externallogicalfilename}{ExternalLogicalFileName}
: Returns an encoded logical filename that can be used to refer to a external file
\item \hyperlink{ecldoc:file.getfiledescription}{GetFileDescription}
: Returns a string containing the description information associated with the specified filename
\item \hyperlink{ecldoc:file.setfiledescription}{SetFileDescription}
: Sets the description associated with the specified filename
\item \hyperlink{ecldoc:file.remotedirectory}{RemoteDirectory}
: Returns a dataset containing a list of files from the specified machineIP and directory
\item \hyperlink{ecldoc:file.logicalfilelist}{LogicalFileList}
: Returns a dataset of information about the logical files known to the system
\item \hyperlink{ecldoc:file.comparefiles}{CompareFiles}
: Compares two files, and returns a result indicating how well they match
\item \hyperlink{ecldoc:file.verifyfile}{VerifyFile}
: Checks the system datastore (Dali) information for the file against the physical parts on disk
\item \hyperlink{ecldoc:file.addfilerelationship}{AddFileRelationship}
: Defines the relationship between two files
\item \hyperlink{ecldoc:file.filerelationshiplist}{FileRelationshipList}
: Returns a dataset of relationships
\item \hyperlink{ecldoc:file.removefilerelationship}{RemoveFileRelationship}
: Removes a relationship between two files
\item \hyperlink{ecldoc:file.getcolumnmapping}{GetColumnMapping}
: Returns the field mappings for the file, in the same format specified for the SetColumnMapping function
\item \hyperlink{ecldoc:file.setcolumnmapping}{SetColumnMapping}
: Defines how the data in the fields of the file mist be transformed between the actual data storage format and the input format used to query that data
\item \hyperlink{ecldoc:file.encoderfsquery}{EncodeRfsQuery}
: Returns a string that can be used in a DATASET declaration to read data from an RFS (Remote File Server) instance (e.g
\item \hyperlink{ecldoc:file.rfsaction}{RfsAction}
: Sends the query to the rfs server
\item \hyperlink{ecldoc:file.moveexternalfile}{MoveExternalFile}
: Moves the single physical file between two locations on the same remote machine
\item \hyperlink{ecldoc:file.deleteexternalfile}{DeleteExternalFile}
: Removes a single physical file from a remote machine
\item \hyperlink{ecldoc:file.createexternaldirectory}{CreateExternalDirectory}
: Creates the path on the location (if it does not already exist)
\item \hyperlink{ecldoc:file.getlogicalfileattribute}{GetLogicalFileAttribute}
: Returns the value of the given attribute for the specified logicalfilename
\item \hyperlink{ecldoc:file.protectlogicalfile}{ProtectLogicalFile}
: Toggles protection on and off for the specified logicalfilename
\item \hyperlink{ecldoc:file.dfuplusexec}{DfuPlusExec}
: The DfuPlusExec action executes the specified command line just as the DfuPLus.exe program would do
\item \hyperlink{ecldoc:file.fsprayfixed}{fSprayFixed}
: Sprays a file of fixed length records from a single machine and distributes it across the nodes of the destination group
\item \hyperlink{ecldoc:file.sprayfixed}{SprayFixed}
: Same as fSprayFixed, but does not return the DFU Workunit ID
\item \hyperlink{ecldoc:file.fsprayvariable}{fSprayVariable}
\item \hyperlink{ecldoc:file.sprayvariable}{SprayVariable}
\item \hyperlink{ecldoc:file.fspraydelimited}{fSprayDelimited}
: Sprays a file of fixed delimited records from a single machine and distributes it across the nodes of the destination group
\item \hyperlink{ecldoc:file.spraydelimited}{SprayDelimited}
: Same as fSprayDelimited, but does not return the DFU Workunit ID
\item \hyperlink{ecldoc:file.fsprayxml}{fSprayXml}
: Sprays an xml file from a single machine and distributes it across the nodes of the destination group
\item \hyperlink{ecldoc:file.sprayxml}{SprayXml}
: Same as fSprayXml, but does not return the DFU Workunit ID
\item \hyperlink{ecldoc:file.fdespray}{fDespray}
: Copies a distributed file from multiple machines, and desprays it to a single file on a single machine
\item \hyperlink{ecldoc:file.despray}{Despray}
: Same as fDespray, but does not return the DFU Workunit ID
\item \hyperlink{ecldoc:file.fcopy}{fCopy}
: Copies a distributed file to another distributed file
\item \hyperlink{ecldoc:file.copy}{Copy}
: Same as fCopy, but does not return the DFU Workunit ID
\item \hyperlink{ecldoc:file.freplicate}{fReplicate}
: Ensures the specified file is replicated to its mirror copies
\item \hyperlink{ecldoc:file.replicate}{Replicate}
: Same as fReplicated, but does not return the DFU Workunit ID
\item \hyperlink{ecldoc:file.fremotepull}{fRemotePull}
: Copies a distributed file to a distributed file on remote system
\item \hyperlink{ecldoc:file.remotepull}{RemotePull}
: Same as fRemotePull, but does not return the DFU Workunit ID
\item \hyperlink{ecldoc:file.fmonitorlogicalfilename}{fMonitorLogicalFileName}
: Creates a file monitor job in the DFU Server
\item \hyperlink{ecldoc:file.monitorlogicalfilename}{MonitorLogicalFileName}
: Same as fMonitorLogicalFileName, but does not return the DFU Workunit ID
\item \hyperlink{ecldoc:file.fmonitorfile}{fMonitorFile}
: Creates a file monitor job in the DFU Server
\item \hyperlink{ecldoc:file.monitorfile}{MonitorFile}
: Same as fMonitorFile, but does not return the DFU Workunit ID
\item \hyperlink{ecldoc:file.waitdfuworkunit}{WaitDfuWorkunit}
: Waits for the specified DFU workunit to finish
\item \hyperlink{ecldoc:file.abortdfuworkunit}{AbortDfuWorkunit}
: Aborts the specified DFU workunit
\item \hyperlink{ecldoc:file.createsuperfile}{CreateSuperFile}
: Creates an empty superfile
\item \hyperlink{ecldoc:file.superfileexists}{SuperFileExists}
: Checks if the specified filename is present in the Distributed File Utility (DFU) and is a SuperFile
\item \hyperlink{ecldoc:file.deletesuperfile}{DeleteSuperFile}
: Deletes the superfile
\item \hyperlink{ecldoc:file.getsuperfilesubcount}{GetSuperFileSubCount}
: Returns the number of sub-files contained within a superfile
\item \hyperlink{ecldoc:file.getsuperfilesubname}{GetSuperFileSubName}
: Returns the name of the Nth sub-file within a superfile
\item \hyperlink{ecldoc:file.findsuperfilesubname}{FindSuperFileSubName}
: Returns the position of a file within a superfile
\item \hyperlink{ecldoc:file.startsuperfiletransaction}{StartSuperFileTransaction}
: Starts a superfile transaction
\item \hyperlink{ecldoc:file.addsuperfile}{AddSuperFile}
: Adds a file to a superfile
\item \hyperlink{ecldoc:file.removesuperfile}{RemoveSuperFile}
: Removes a sub-file from a superfile
\item \hyperlink{ecldoc:file.clearsuperfile}{ClearSuperFile}
: Removes all sub-files from a superfile
\item \hyperlink{ecldoc:file.removeownedsubfiles}{RemoveOwnedSubFiles}
: Removes all soley-owned sub-files from a superfile
\item \hyperlink{ecldoc:file.deleteownedsubfiles}{DeleteOwnedSubFiles}
: Legacy version of RemoveOwnedSubFiles which was incorrectly named in a previous version
\item \hyperlink{ecldoc:file.swapsuperfile}{SwapSuperFile}
: Swap the contents of two superfiles
\item \hyperlink{ecldoc:file.replacesuperfile}{ReplaceSuperFile}
: Removes a sub-file from a superfile and replaces it with another
\item \hyperlink{ecldoc:file.finishsuperfiletransaction}{FinishSuperFileTransaction}
: Finishes a superfile transaction
\item \hyperlink{ecldoc:file.superfilecontents}{SuperFileContents}
: Returns the list of sub-files contained within a superfile
\item \hyperlink{ecldoc:file.logicalfilesuperowners}{LogicalFileSuperOwners}
: Returns the list of superfiles that a logical file is contained within
\item \hyperlink{ecldoc:file.logicalfilesupersublist}{LogicalFileSuperSubList}
: Returns the list of all the superfiles in the system and their component sub-files
\item \hyperlink{ecldoc:file.fpromotesuperfilelist}{fPromoteSuperFileList}
: Moves the sub-files from the first entry in the list of superfiles to the next in the list, repeating the process through the list of superfiles
\item \hyperlink{ecldoc:file.promotesuperfilelist}{PromoteSuperFileList}
: Same as fPromoteSuperFileList, but does not return the DFU Workunit ID
\end{enumerate}

\rule{\linewidth}{0.5pt}

\subsection*{\textsf{\colorbox{headtoc}{\color{white} RECORD}
FsFilenameRecord}}

\hypertarget{ecldoc:file.fsfilenamerecord}{}
\hspace{0pt} \hyperlink{ecldoc:File}{File} \textbackslash 

{\renewcommand{\arraystretch}{1.5}
\begin{tabularx}{\textwidth}{|>{\raggedright\arraybackslash}l|X|}
\hline
\hspace{0pt}\mytexttt{\color{red} } & \textbf{FsFilenameRecord} \\
\hline
\end{tabularx}
}

\par
A record containing information about filename. Includes name, size and when last modified. export FsFilenameRecord := RECORD string name; integer8 size; string19 modified; END;


\rule{\linewidth}{0.5pt}
\subsection*{\textsf{\colorbox{headtoc}{\color{white} ATTRIBUTE}
FsLogicalFileName}}

\hypertarget{ecldoc:file.fslogicalfilename}{}
\hspace{0pt} \hyperlink{ecldoc:File}{File} \textbackslash 

{\renewcommand{\arraystretch}{1.5}
\begin{tabularx}{\textwidth}{|>{\raggedright\arraybackslash}l|X|}
\hline
\hspace{0pt}\mytexttt{\color{red} } & \textbf{FsLogicalFileName} \\
\hline
\end{tabularx}
}

\par
An alias for a logical filename that is stored in a row.


\rule{\linewidth}{0.5pt}
\subsection*{\textsf{\colorbox{headtoc}{\color{white} RECORD}
FsLogicalFileNameRecord}}

\hypertarget{ecldoc:file.fslogicalfilenamerecord}{}
\hspace{0pt} \hyperlink{ecldoc:File}{File} \textbackslash 

{\renewcommand{\arraystretch}{1.5}
\begin{tabularx}{\textwidth}{|>{\raggedright\arraybackslash}l|X|}
\hline
\hspace{0pt}\mytexttt{\color{red} } & \textbf{FsLogicalFileNameRecord} \\
\hline
\end{tabularx}
}

\par
A record containing a logical filename. It contains the following fields:

\par
\begin{description}
\item [\colorbox{tagtype}{\color{white} \textbf{\textsf{FIELD}}}] \textbf{\underline{name}} The logical name of the file;
\end{description}

\rule{\linewidth}{0.5pt}
\subsection*{\textsf{\colorbox{headtoc}{\color{white} RECORD}
FsLogicalFileInfoRecord}}

\hypertarget{ecldoc:file.fslogicalfileinforecord}{}
\hspace{0pt} \hyperlink{ecldoc:File}{File} \textbackslash 

{\renewcommand{\arraystretch}{1.5}
\begin{tabularx}{\textwidth}{|>{\raggedright\arraybackslash}l|X|}
\hline
\hspace{0pt}\mytexttt{\color{red} } & \textbf{FsLogicalFileInfoRecord} \\
\hline
\end{tabularx}
}

\par
A record containing information about a logical file.

\par
\begin{description}
\item [\colorbox{tagtype}{\color{white} \textbf{\textsf{FIELD}}}] \textbf{\underline{superfile}} Is this a superfile?
\item [\colorbox{tagtype}{\color{white} \textbf{\textsf{FIELD}}}] \textbf{\underline{size}} Number of bytes in the file (before compression)
\item [\colorbox{tagtype}{\color{white} \textbf{\textsf{FIELD}}}] \textbf{\underline{rowcount}} Number of rows in the file.
\end{description}

\rule{\linewidth}{0.5pt}
\subsection*{\textsf{\colorbox{headtoc}{\color{white} RECORD}
FsLogicalSuperSubRecord}}

\hypertarget{ecldoc:file.fslogicalsupersubrecord}{}
\hspace{0pt} \hyperlink{ecldoc:File}{File} \textbackslash 

{\renewcommand{\arraystretch}{1.5}
\begin{tabularx}{\textwidth}{|>{\raggedright\arraybackslash}l|X|}
\hline
\hspace{0pt}\mytexttt{\color{red} } & \textbf{FsLogicalSuperSubRecord} \\
\hline
\end{tabularx}
}

\par
A record containing information about a superfile and its contents.

\par
\begin{description}
\item [\colorbox{tagtype}{\color{white} \textbf{\textsf{FIELD}}}] \textbf{\underline{supername}} The name of the superfile
\item [\colorbox{tagtype}{\color{white} \textbf{\textsf{FIELD}}}] \textbf{\underline{subname}} The name of the sub-file
\end{description}

\rule{\linewidth}{0.5pt}
\subsection*{\textsf{\colorbox{headtoc}{\color{white} RECORD}
FsFileRelationshipRecord}}

\hypertarget{ecldoc:file.fsfilerelationshiprecord}{}
\hspace{0pt} \hyperlink{ecldoc:File}{File} \textbackslash 

{\renewcommand{\arraystretch}{1.5}
\begin{tabularx}{\textwidth}{|>{\raggedright\arraybackslash}l|X|}
\hline
\hspace{0pt}\mytexttt{\color{red} } & \textbf{FsFileRelationshipRecord} \\
\hline
\end{tabularx}
}

\par
A record containing information about the relationship between two files.

\par
\begin{description}
\item [\colorbox{tagtype}{\color{white} \textbf{\textsf{FIELD}}}] \textbf{\underline{primaryfile}} The logical filename of the primary file
\item [\colorbox{tagtype}{\color{white} \textbf{\textsf{FIELD}}}] \textbf{\underline{secondaryfile}} The logical filename of the secondary file.
\item [\colorbox{tagtype}{\color{white} \textbf{\textsf{FIELD}}}] \textbf{\underline{primaryflds}} The name of the primary key field for the primary file. The value ''\_\_fileposition\_\_'' indicates the secondary is an INDEX that must use FETCH to access non-keyed fields.
\item [\colorbox{tagtype}{\color{white} \textbf{\textsf{FIELD}}}] \textbf{\underline{secondaryflds}} The name of the foreign key field relating to the primary file.
\item [\colorbox{tagtype}{\color{white} \textbf{\textsf{FIELD}}}] \textbf{\underline{kind}} The type of relationship between the primary and secondary files. Containing either 'link' or 'view'.
\item [\colorbox{tagtype}{\color{white} \textbf{\textsf{FIELD}}}] \textbf{\underline{cardinality}} The cardinality of the relationship. The format is <primary>:<secondary>. Valid values are ''1'' or ''M''.</secondary></primary>
\item [\colorbox{tagtype}{\color{white} \textbf{\textsf{FIELD}}}] \textbf{\underline{payload}} Indicates whether the primary or secondary are payload INDEXes.
\item [\colorbox{tagtype}{\color{white} \textbf{\textsf{FIELD}}}] \textbf{\underline{description}} The description of the relationship.
\end{description}

\rule{\linewidth}{0.5pt}
\subsection*{\textsf{\colorbox{headtoc}{\color{white} ATTRIBUTE}
RECFMV\_RECSIZE}}

\hypertarget{ecldoc:file.recfmv_recsize}{}
\hspace{0pt} \hyperlink{ecldoc:File}{File} \textbackslash 

{\renewcommand{\arraystretch}{1.5}
\begin{tabularx}{\textwidth}{|>{\raggedright\arraybackslash}l|X|}
\hline
\hspace{0pt}\mytexttt{\color{red} } & \textbf{RECFMV\_RECSIZE} \\
\hline
\end{tabularx}
}

\par
Constant that indicates IBM RECFM V format file. Can be passed to SprayFixed for the record size.


\rule{\linewidth}{0.5pt}
\subsection*{\textsf{\colorbox{headtoc}{\color{white} ATTRIBUTE}
RECFMVB\_RECSIZE}}

\hypertarget{ecldoc:file.recfmvb_recsize}{}
\hspace{0pt} \hyperlink{ecldoc:File}{File} \textbackslash 

{\renewcommand{\arraystretch}{1.5}
\begin{tabularx}{\textwidth}{|>{\raggedright\arraybackslash}l|X|}
\hline
\hspace{0pt}\mytexttt{\color{red} } & \textbf{RECFMVB\_RECSIZE} \\
\hline
\end{tabularx}
}

\par
Constant that indicates IBM RECFM VB format file. Can be passed to SprayFixed for the record size.


\rule{\linewidth}{0.5pt}
\subsection*{\textsf{\colorbox{headtoc}{\color{white} ATTRIBUTE}
PREFIX\_VARIABLE\_RECSIZE}}

\hypertarget{ecldoc:file.prefix_variable_recsize}{}
\hspace{0pt} \hyperlink{ecldoc:File}{File} \textbackslash 

{\renewcommand{\arraystretch}{1.5}
\begin{tabularx}{\textwidth}{|>{\raggedright\arraybackslash}l|X|}
\hline
\hspace{0pt}\mytexttt{\color{red} INTEGER4} & \textbf{PREFIX\_VARIABLE\_RECSIZE} \\
\hline
\end{tabularx}
}

\par
Constant that indicates a variable little endian 4 byte length prefixed file. Can be passed to SprayFixed for the record size.


\rule{\linewidth}{0.5pt}
\subsection*{\textsf{\colorbox{headtoc}{\color{white} ATTRIBUTE}
PREFIX\_VARIABLE\_BIGENDIAN\_RECSIZE}}

\hypertarget{ecldoc:file.prefix_variable_bigendian_recsize}{}
\hspace{0pt} \hyperlink{ecldoc:File}{File} \textbackslash 

{\renewcommand{\arraystretch}{1.5}
\begin{tabularx}{\textwidth}{|>{\raggedright\arraybackslash}l|X|}
\hline
\hspace{0pt}\mytexttt{\color{red} INTEGER4} & \textbf{PREFIX\_VARIABLE\_BIGENDIAN\_RECSIZE} \\
\hline
\end{tabularx}
}

\par
Constant that indicates a variable big endian 4 byte length prefixed file. Can be passed to SprayFixed for the record size.


\rule{\linewidth}{0.5pt}
\subsection*{\textsf{\colorbox{headtoc}{\color{white} FUNCTION}
FileExists}}

\hypertarget{ecldoc:file.fileexists}{}
\hspace{0pt} \hyperlink{ecldoc:File}{File} \textbackslash 

{\renewcommand{\arraystretch}{1.5}
\begin{tabularx}{\textwidth}{|>{\raggedright\arraybackslash}l|X|}
\hline
\hspace{0pt}\mytexttt{\color{red} boolean} & \textbf{FileExists} \\
\hline
\multicolumn{2}{|>{\raggedright\arraybackslash}X|}{\hspace{0pt}\mytexttt{\color{param} (varstring lfn, boolean physical=FALSE)}} \\
\hline
\end{tabularx}
}

\par
Returns whether the file exists.

\par
\begin{description}
\item [\colorbox{tagtype}{\color{white} \textbf{\textsf{PARAMETER}}}] \textbf{\underline{lfn}} The logical name of the file.
\item [\colorbox{tagtype}{\color{white} \textbf{\textsf{PARAMETER}}}] \textbf{\underline{physical}} Whether to also check for the physical existence on disk. Defaults to FALSE.
\item [\colorbox{tagtype}{\color{white} \textbf{\textsf{RETURN}}}] \textbf{\underline{}} Whether the file exists.
\end{description}

\rule{\linewidth}{0.5pt}
\subsection*{\textsf{\colorbox{headtoc}{\color{white} FUNCTION}
DeleteLogicalFile}}

\hypertarget{ecldoc:file.deletelogicalfile}{}
\hspace{0pt} \hyperlink{ecldoc:File}{File} \textbackslash 

{\renewcommand{\arraystretch}{1.5}
\begin{tabularx}{\textwidth}{|>{\raggedright\arraybackslash}l|X|}
\hline
\hspace{0pt}\mytexttt{\color{red} } & \textbf{DeleteLogicalFile} \\
\hline
\multicolumn{2}{|>{\raggedright\arraybackslash}X|}{\hspace{0pt}\mytexttt{\color{param} (varstring lfn, boolean allowMissing=FALSE)}} \\
\hline
\end{tabularx}
}

\par
Removes the logical file from the system, and deletes from the disk.

\par
\begin{description}
\item [\colorbox{tagtype}{\color{white} \textbf{\textsf{PARAMETER}}}] \textbf{\underline{lfn}} The logical name of the file.
\item [\colorbox{tagtype}{\color{white} \textbf{\textsf{PARAMETER}}}] \textbf{\underline{allowMissing}} Whether to suppress an error if the filename does not exist. Defaults to FALSE.
\end{description}

\rule{\linewidth}{0.5pt}
\subsection*{\textsf{\colorbox{headtoc}{\color{white} FUNCTION}
SetReadOnly}}

\hypertarget{ecldoc:file.setreadonly}{}
\hspace{0pt} \hyperlink{ecldoc:File}{File} \textbackslash 

{\renewcommand{\arraystretch}{1.5}
\begin{tabularx}{\textwidth}{|>{\raggedright\arraybackslash}l|X|}
\hline
\hspace{0pt}\mytexttt{\color{red} } & \textbf{SetReadOnly} \\
\hline
\multicolumn{2}{|>{\raggedright\arraybackslash}X|}{\hspace{0pt}\mytexttt{\color{param} (varstring lfn, boolean ro=TRUE)}} \\
\hline
\end{tabularx}
}

\par
Changes whether access to a file is read only or not.

\par
\begin{description}
\item [\colorbox{tagtype}{\color{white} \textbf{\textsf{PARAMETER}}}] \textbf{\underline{lfn}} The logical name of the file.
\item [\colorbox{tagtype}{\color{white} \textbf{\textsf{PARAMETER}}}] \textbf{\underline{ro}} Whether updates to the file are disallowed. Defaults to TRUE.
\end{description}

\rule{\linewidth}{0.5pt}
\subsection*{\textsf{\colorbox{headtoc}{\color{white} FUNCTION}
RenameLogicalFile}}

\hypertarget{ecldoc:file.renamelogicalfile}{}
\hspace{0pt} \hyperlink{ecldoc:File}{File} \textbackslash 

{\renewcommand{\arraystretch}{1.5}
\begin{tabularx}{\textwidth}{|>{\raggedright\arraybackslash}l|X|}
\hline
\hspace{0pt}\mytexttt{\color{red} } & \textbf{RenameLogicalFile} \\
\hline
\multicolumn{2}{|>{\raggedright\arraybackslash}X|}{\hspace{0pt}\mytexttt{\color{param} (varstring oldname, varstring newname)}} \\
\hline
\end{tabularx}
}

\par
Changes the name of a logical file.

\par
\begin{description}
\item [\colorbox{tagtype}{\color{white} \textbf{\textsf{PARAMETER}}}] \textbf{\underline{oldname}} The current name of the file to be renamed.
\item [\colorbox{tagtype}{\color{white} \textbf{\textsf{PARAMETER}}}] \textbf{\underline{newname}} The new logical name of the file.
\end{description}

\rule{\linewidth}{0.5pt}
\subsection*{\textsf{\colorbox{headtoc}{\color{white} FUNCTION}
ForeignLogicalFileName}}

\hypertarget{ecldoc:file.foreignlogicalfilename}{}
\hspace{0pt} \hyperlink{ecldoc:File}{File} \textbackslash 

{\renewcommand{\arraystretch}{1.5}
\begin{tabularx}{\textwidth}{|>{\raggedright\arraybackslash}l|X|}
\hline
\hspace{0pt}\mytexttt{\color{red} varstring} & \textbf{ForeignLogicalFileName} \\
\hline
\multicolumn{2}{|>{\raggedright\arraybackslash}X|}{\hspace{0pt}\mytexttt{\color{param} (varstring name, varstring foreigndali='', boolean abspath=FALSE)}} \\
\hline
\end{tabularx}
}

\par
Returns a logical filename that can be used to refer to a logical file in a local or remote dali.

\par
\begin{description}
\item [\colorbox{tagtype}{\color{white} \textbf{\textsf{PARAMETER}}}] \textbf{\underline{name}} The logical name of the file.
\item [\colorbox{tagtype}{\color{white} \textbf{\textsf{PARAMETER}}}] \textbf{\underline{foreigndali}} The IP address of the foreign dali used to resolve the file. If blank then the file is resolved locally. Defaults to blank.
\item [\colorbox{tagtype}{\color{white} \textbf{\textsf{PARAMETER}}}] \textbf{\underline{abspath}} Should a tilde (\~{}) be prepended to the resulting logical file name. Defaults to FALSE.
\end{description}

\rule{\linewidth}{0.5pt}
\subsection*{\textsf{\colorbox{headtoc}{\color{white} FUNCTION}
ExternalLogicalFileName}}

\hypertarget{ecldoc:file.externallogicalfilename}{}
\hspace{0pt} \hyperlink{ecldoc:File}{File} \textbackslash 

{\renewcommand{\arraystretch}{1.5}
\begin{tabularx}{\textwidth}{|>{\raggedright\arraybackslash}l|X|}
\hline
\hspace{0pt}\mytexttt{\color{red} varstring} & \textbf{ExternalLogicalFileName} \\
\hline
\multicolumn{2}{|>{\raggedright\arraybackslash}X|}{\hspace{0pt}\mytexttt{\color{param} (varstring location, varstring path, boolean abspath=TRUE)}} \\
\hline
\end{tabularx}
}

\par
Returns an encoded logical filename that can be used to refer to a external file. Examples include directly reading from a landing zone. Upper case characters and other details are escaped.

\par
\begin{description}
\item [\colorbox{tagtype}{\color{white} \textbf{\textsf{PARAMETER}}}] \textbf{\underline{location}} The IP address of the remote machine. '.' can be used for the local machine.
\item [\colorbox{tagtype}{\color{white} \textbf{\textsf{PARAMETER}}}] \textbf{\underline{path}} The path/name of the file on the remote machine.
\item [\colorbox{tagtype}{\color{white} \textbf{\textsf{PARAMETER}}}] \textbf{\underline{abspath}} Should a tilde (\~{}) be prepended to the resulting logical file name. Defaults to TRUE.
\item [\colorbox{tagtype}{\color{white} \textbf{\textsf{RETURN}}}] \textbf{\underline{}} The encoded logical filename.
\end{description}

\rule{\linewidth}{0.5pt}
\subsection*{\textsf{\colorbox{headtoc}{\color{white} FUNCTION}
GetFileDescription}}

\hypertarget{ecldoc:file.getfiledescription}{}
\hspace{0pt} \hyperlink{ecldoc:File}{File} \textbackslash 

{\renewcommand{\arraystretch}{1.5}
\begin{tabularx}{\textwidth}{|>{\raggedright\arraybackslash}l|X|}
\hline
\hspace{0pt}\mytexttt{\color{red} varstring} & \textbf{GetFileDescription} \\
\hline
\multicolumn{2}{|>{\raggedright\arraybackslash}X|}{\hspace{0pt}\mytexttt{\color{param} (varstring lfn)}} \\
\hline
\end{tabularx}
}

\par
Returns a string containing the description information associated with the specified filename. This description is set either through ECL watch or by using the FileServices.SetFileDescription function.

\par
\begin{description}
\item [\colorbox{tagtype}{\color{white} \textbf{\textsf{PARAMETER}}}] \textbf{\underline{lfn}} The logical name of the file.
\end{description}

\rule{\linewidth}{0.5pt}
\subsection*{\textsf{\colorbox{headtoc}{\color{white} FUNCTION}
SetFileDescription}}

\hypertarget{ecldoc:file.setfiledescription}{}
\hspace{0pt} \hyperlink{ecldoc:File}{File} \textbackslash 

{\renewcommand{\arraystretch}{1.5}
\begin{tabularx}{\textwidth}{|>{\raggedright\arraybackslash}l|X|}
\hline
\hspace{0pt}\mytexttt{\color{red} } & \textbf{SetFileDescription} \\
\hline
\multicolumn{2}{|>{\raggedright\arraybackslash}X|}{\hspace{0pt}\mytexttt{\color{param} (varstring lfn, varstring val)}} \\
\hline
\end{tabularx}
}

\par
Sets the description associated with the specified filename.

\par
\begin{description}
\item [\colorbox{tagtype}{\color{white} \textbf{\textsf{PARAMETER}}}] \textbf{\underline{lfn}} The logical name of the file.
\item [\colorbox{tagtype}{\color{white} \textbf{\textsf{PARAMETER}}}] \textbf{\underline{val}} The description to be associated with the file.
\end{description}

\rule{\linewidth}{0.5pt}
\subsection*{\textsf{\colorbox{headtoc}{\color{white} FUNCTION}
RemoteDirectory}}

\hypertarget{ecldoc:file.remotedirectory}{}
\hspace{0pt} \hyperlink{ecldoc:File}{File} \textbackslash 

{\renewcommand{\arraystretch}{1.5}
\begin{tabularx}{\textwidth}{|>{\raggedright\arraybackslash}l|X|}
\hline
\hspace{0pt}\mytexttt{\color{red} dataset(FsFilenameRecord)} & \textbf{RemoteDirectory} \\
\hline
\multicolumn{2}{|>{\raggedright\arraybackslash}X|}{\hspace{0pt}\mytexttt{\color{param} (varstring machineIP, varstring dir, varstring mask='*', boolean recurse=FALSE)}} \\
\hline
\end{tabularx}
}

\par
Returns a dataset containing a list of files from the specified machineIP and directory.

\par
\begin{description}
\item [\colorbox{tagtype}{\color{white} \textbf{\textsf{PARAMETER}}}] \textbf{\underline{machineIP}} The IP address of the remote machine.
\item [\colorbox{tagtype}{\color{white} \textbf{\textsf{PARAMETER}}}] \textbf{\underline{directory}} The path to the directory to read. This must be in the appropriate format for the operating system running on the remote machine.
\item [\colorbox{tagtype}{\color{white} \textbf{\textsf{PARAMETER}}}] \textbf{\underline{mask}} The filemask specifying which files to include in the result. Defaults to '*' (all files).
\item [\colorbox{tagtype}{\color{white} \textbf{\textsf{PARAMETER}}}] \textbf{\underline{recurse}} Whether to include files from subdirectories under the directory. Defaults to FALSE.
\end{description}

\rule{\linewidth}{0.5pt}
\subsection*{\textsf{\colorbox{headtoc}{\color{white} FUNCTION}
LogicalFileList}}

\hypertarget{ecldoc:file.logicalfilelist}{}
\hspace{0pt} \hyperlink{ecldoc:File}{File} \textbackslash 

{\renewcommand{\arraystretch}{1.5}
\begin{tabularx}{\textwidth}{|>{\raggedright\arraybackslash}l|X|}
\hline
\hspace{0pt}\mytexttt{\color{red} dataset(FsLogicalFileInfoRecord)} & \textbf{LogicalFileList} \\
\hline
\multicolumn{2}{|>{\raggedright\arraybackslash}X|}{\hspace{0pt}\mytexttt{\color{param} (varstring namepattern='*', boolean includenormal=TRUE, boolean includesuper=FALSE, boolean unknownszero=FALSE, varstring foreigndali='')}} \\
\hline
\end{tabularx}
}

\par
Returns a dataset of information about the logical files known to the system.

\par
\begin{description}
\item [\colorbox{tagtype}{\color{white} \textbf{\textsf{PARAMETER}}}] \textbf{\underline{namepattern}} The mask of the files to list. Defaults to '*' (all files).
\item [\colorbox{tagtype}{\color{white} \textbf{\textsf{PARAMETER}}}] \textbf{\underline{includenormal}} Whether to include 'normal' files. Defaults to TRUE.
\item [\colorbox{tagtype}{\color{white} \textbf{\textsf{PARAMETER}}}] \textbf{\underline{includesuper}} Whether to include SuperFiles. Defaults to FALSE.
\item [\colorbox{tagtype}{\color{white} \textbf{\textsf{PARAMETER}}}] \textbf{\underline{unknownszero}} Whether to set file sizes that are unknown to zero(0) instead of minus-one (-1). Defaults to FALSE.
\item [\colorbox{tagtype}{\color{white} \textbf{\textsf{PARAMETER}}}] \textbf{\underline{foreigndali}} The IP address of the foreign dali used to resolve the file. If blank then the file is resolved locally. Defaults to blank.
\end{description}

\rule{\linewidth}{0.5pt}
\subsection*{\textsf{\colorbox{headtoc}{\color{white} FUNCTION}
CompareFiles}}

\hypertarget{ecldoc:file.comparefiles}{}
\hspace{0pt} \hyperlink{ecldoc:File}{File} \textbackslash 

{\renewcommand{\arraystretch}{1.5}
\begin{tabularx}{\textwidth}{|>{\raggedright\arraybackslash}l|X|}
\hline
\hspace{0pt}\mytexttt{\color{red} INTEGER4} & \textbf{CompareFiles} \\
\hline
\multicolumn{2}{|>{\raggedright\arraybackslash}X|}{\hspace{0pt}\mytexttt{\color{param} (varstring lfn1, varstring lfn2, boolean logical\_only=TRUE, boolean use\_crcs=FALSE)}} \\
\hline
\end{tabularx}
}

\par
Compares two files, and returns a result indicating how well they match.

\par
\begin{description}
\item [\colorbox{tagtype}{\color{white} \textbf{\textsf{PARAMETER}}}] \textbf{\underline{file1}} The logical name of the first file.
\item [\colorbox{tagtype}{\color{white} \textbf{\textsf{PARAMETER}}}] \textbf{\underline{file2}} The logical name of the second file.
\item [\colorbox{tagtype}{\color{white} \textbf{\textsf{PARAMETER}}}] \textbf{\underline{logical\_only}} Whether to only compare logical information in the system datastore (Dali), and ignore physical information on disk. [Default TRUE]
\item [\colorbox{tagtype}{\color{white} \textbf{\textsf{PARAMETER}}}] \textbf{\underline{use\_crcs}} Whether to compare physical CRCs of all the parts on disk. This may be slow on large files. Defaults to FALSE.
\item [\colorbox{tagtype}{\color{white} \textbf{\textsf{RETURN}}}] \textbf{\underline{}} 0 if file1 and file2 match exactly 1 if file1 and file2 contents match, but file1 is newer than file2 -1 if file1 and file2 contents match, but file2 is newer than file1 2 if file1 and file2 contents do not match and file1 is newer than file2 -2 if file1 and file2 contents do not match and file2 is newer than file1
\end{description}

\rule{\linewidth}{0.5pt}
\subsection*{\textsf{\colorbox{headtoc}{\color{white} FUNCTION}
VerifyFile}}

\hypertarget{ecldoc:file.verifyfile}{}
\hspace{0pt} \hyperlink{ecldoc:File}{File} \textbackslash 

{\renewcommand{\arraystretch}{1.5}
\begin{tabularx}{\textwidth}{|>{\raggedright\arraybackslash}l|X|}
\hline
\hspace{0pt}\mytexttt{\color{red} varstring} & \textbf{VerifyFile} \\
\hline
\multicolumn{2}{|>{\raggedright\arraybackslash}X|}{\hspace{0pt}\mytexttt{\color{param} (varstring lfn, boolean usecrcs)}} \\
\hline
\end{tabularx}
}

\par
Checks the system datastore (Dali) information for the file against the physical parts on disk.

\par
\begin{description}
\item [\colorbox{tagtype}{\color{white} \textbf{\textsf{PARAMETER}}}] \textbf{\underline{lfn}} The name of the file to check.
\item [\colorbox{tagtype}{\color{white} \textbf{\textsf{PARAMETER}}}] \textbf{\underline{use\_crcs}} Whether to compare physical CRCs of all the parts on disk. This may be slow on large files.
\item [\colorbox{tagtype}{\color{white} \textbf{\textsf{RETURN}}}] \textbf{\underline{}} 'OK' - The file parts match the datastore information 'Could not find file: <filename>' - The logical filename was not found 'Could not find part file: <partname>' - The partname was not found 'Modified time differs for: <partname>' - The partname has a different timestamp 'File size differs for: <partname>' - The partname has a file size 'File CRC differs for: <partname>' - The partname has a different CRC</partname></partname></partname></partname></filename>
\end{description}

\rule{\linewidth}{0.5pt}
\subsection*{\textsf{\colorbox{headtoc}{\color{white} FUNCTION}
AddFileRelationship}}

\hypertarget{ecldoc:file.addfilerelationship}{}
\hspace{0pt} \hyperlink{ecldoc:File}{File} \textbackslash 

{\renewcommand{\arraystretch}{1.5}
\begin{tabularx}{\textwidth}{|>{\raggedright\arraybackslash}l|X|}
\hline
\hspace{0pt}\mytexttt{\color{red} } & \textbf{AddFileRelationship} \\
\hline
\multicolumn{2}{|>{\raggedright\arraybackslash}X|}{\hspace{0pt}\mytexttt{\color{param} (varstring primary, varstring secondary, varstring primaryflds, varstring secondaryflds, varstring kind='link', varstring cardinality, boolean payload, varstring description='')}} \\
\hline
\end{tabularx}
}

\par
Defines the relationship between two files. These may be DATASETs or INDEXes. Each record in the primary file should be uniquely defined by the primaryfields (ideally), preferably efficiently. This information is used by the roxie browser to link files together.

\par
\begin{description}
\item [\colorbox{tagtype}{\color{white} \textbf{\textsf{PARAMETER}}}] \textbf{\underline{primary}} The logical filename of the primary file.
\item [\colorbox{tagtype}{\color{white} \textbf{\textsf{PARAMETER}}}] \textbf{\underline{secondary}} The logical filename of the secondary file.
\item [\colorbox{tagtype}{\color{white} \textbf{\textsf{PARAMETER}}}] \textbf{\underline{primaryfields}} The name of the primary key field for the primary file. The value ''\_\_fileposition\_\_'' indicates the secondary is an INDEX that must use FETCH to access non-keyed fields.
\item [\colorbox{tagtype}{\color{white} \textbf{\textsf{PARAMETER}}}] \textbf{\underline{secondaryfields}} The name of the foreign key field relating to the primary file.
\item [\colorbox{tagtype}{\color{white} \textbf{\textsf{PARAMETER}}}] \textbf{\underline{relationship}} The type of relationship between the primary and secondary files. Containing either 'link' or 'view'. Default is ''link''.
\item [\colorbox{tagtype}{\color{white} \textbf{\textsf{PARAMETER}}}] \textbf{\underline{cardinality}} The cardinality of the relationship. The format is <primary>:<secondary>. Valid values are ''1'' or ''M''.</secondary></primary>
\item [\colorbox{tagtype}{\color{white} \textbf{\textsf{PARAMETER}}}] \textbf{\underline{payload}} Indicates whether the primary or secondary are payload INDEXes.
\item [\colorbox{tagtype}{\color{white} \textbf{\textsf{PARAMETER}}}] \textbf{\underline{description}} The description of the relationship.
\end{description}

\rule{\linewidth}{0.5pt}
\subsection*{\textsf{\colorbox{headtoc}{\color{white} FUNCTION}
FileRelationshipList}}

\hypertarget{ecldoc:file.filerelationshiplist}{}
\hspace{0pt} \hyperlink{ecldoc:File}{File} \textbackslash 

{\renewcommand{\arraystretch}{1.5}
\begin{tabularx}{\textwidth}{|>{\raggedright\arraybackslash}l|X|}
\hline
\hspace{0pt}\mytexttt{\color{red} dataset(FsFileRelationshipRecord)} & \textbf{FileRelationshipList} \\
\hline
\multicolumn{2}{|>{\raggedright\arraybackslash}X|}{\hspace{0pt}\mytexttt{\color{param} (varstring primary, varstring secondary, varstring primflds='', varstring secondaryflds='', varstring kind='link')}} \\
\hline
\end{tabularx}
}

\par
Returns a dataset of relationships. The return records are structured in the FsFileRelationshipRecord format.

\par
\begin{description}
\item [\colorbox{tagtype}{\color{white} \textbf{\textsf{PARAMETER}}}] \textbf{\underline{primary}} The logical filename of the primary file.
\item [\colorbox{tagtype}{\color{white} \textbf{\textsf{PARAMETER}}}] \textbf{\underline{secondary}} The logical filename of the secondary file.
\item [\colorbox{tagtype}{\color{white} \textbf{\textsf{PARAMETER}}}] \textbf{\underline{primaryfields}} The name of the primary key field for the primary file.
\item [\colorbox{tagtype}{\color{white} \textbf{\textsf{PARAMETER}}}] \textbf{\underline{secondaryfields}} The name of the foreign key field relating to the primary file.
\item [\colorbox{tagtype}{\color{white} \textbf{\textsf{PARAMETER}}}] \textbf{\underline{relationship}} The type of relationship between the primary and secondary files. Containing either 'link' or 'view'. Default is ''link''.
\end{description}

\rule{\linewidth}{0.5pt}
\subsection*{\textsf{\colorbox{headtoc}{\color{white} FUNCTION}
RemoveFileRelationship}}

\hypertarget{ecldoc:file.removefilerelationship}{}
\hspace{0pt} \hyperlink{ecldoc:File}{File} \textbackslash 

{\renewcommand{\arraystretch}{1.5}
\begin{tabularx}{\textwidth}{|>{\raggedright\arraybackslash}l|X|}
\hline
\hspace{0pt}\mytexttt{\color{red} } & \textbf{RemoveFileRelationship} \\
\hline
\multicolumn{2}{|>{\raggedright\arraybackslash}X|}{\hspace{0pt}\mytexttt{\color{param} (varstring primary, varstring secondary, varstring primaryflds='', varstring secondaryflds='', varstring kind='link')}} \\
\hline
\end{tabularx}
}

\par
Removes a relationship between two files.

\par
\begin{description}
\item [\colorbox{tagtype}{\color{white} \textbf{\textsf{PARAMETER}}}] \textbf{\underline{primary}} The logical filename of the primary file.
\item [\colorbox{tagtype}{\color{white} \textbf{\textsf{PARAMETER}}}] \textbf{\underline{secondary}} The logical filename of the secondary file.
\item [\colorbox{tagtype}{\color{white} \textbf{\textsf{PARAMETER}}}] \textbf{\underline{primaryfields}} The name of the primary key field for the primary file.
\item [\colorbox{tagtype}{\color{white} \textbf{\textsf{PARAMETER}}}] \textbf{\underline{secondaryfields}} The name of the foreign key field relating to the primary file.
\item [\colorbox{tagtype}{\color{white} \textbf{\textsf{PARAMETER}}}] \textbf{\underline{relationship}} The type of relationship between the primary and secondary files. Containing either 'link' or 'view'. Default is ''link''.
\end{description}

\rule{\linewidth}{0.5pt}
\subsection*{\textsf{\colorbox{headtoc}{\color{white} FUNCTION}
GetColumnMapping}}

\hypertarget{ecldoc:file.getcolumnmapping}{}
\hspace{0pt} \hyperlink{ecldoc:File}{File} \textbackslash 

{\renewcommand{\arraystretch}{1.5}
\begin{tabularx}{\textwidth}{|>{\raggedright\arraybackslash}l|X|}
\hline
\hspace{0pt}\mytexttt{\color{red} varstring} & \textbf{GetColumnMapping} \\
\hline
\multicolumn{2}{|>{\raggedright\arraybackslash}X|}{\hspace{0pt}\mytexttt{\color{param} (varstring lfn)}} \\
\hline
\end{tabularx}
}

\par
Returns the field mappings for the file, in the same format specified for the SetColumnMapping function.

\par
\begin{description}
\item [\colorbox{tagtype}{\color{white} \textbf{\textsf{PARAMETER}}}] \textbf{\underline{lfn}} The logical filename of the primary file.
\end{description}

\rule{\linewidth}{0.5pt}
\subsection*{\textsf{\colorbox{headtoc}{\color{white} FUNCTION}
SetColumnMapping}}

\hypertarget{ecldoc:file.setcolumnmapping}{}
\hspace{0pt} \hyperlink{ecldoc:File}{File} \textbackslash 

{\renewcommand{\arraystretch}{1.5}
\begin{tabularx}{\textwidth}{|>{\raggedright\arraybackslash}l|X|}
\hline
\hspace{0pt}\mytexttt{\color{red} } & \textbf{SetColumnMapping} \\
\hline
\multicolumn{2}{|>{\raggedright\arraybackslash}X|}{\hspace{0pt}\mytexttt{\color{param} (varstring lfn, varstring mapping)}} \\
\hline
\end{tabularx}
}

\par
Defines how the data in the fields of the file mist be transformed between the actual data storage format and the input format used to query that data. This is used by the user interface of the roxie browser.

\par
\begin{description}
\item [\colorbox{tagtype}{\color{white} \textbf{\textsf{PARAMETER}}}] \textbf{\underline{lfn}} The logical filename of the primary file.
\item [\colorbox{tagtype}{\color{white} \textbf{\textsf{PARAMETER}}}] \textbf{\underline{mapping}} A string containing a comma separated list of field mappings.
\end{description}

\rule{\linewidth}{0.5pt}
\subsection*{\textsf{\colorbox{headtoc}{\color{white} FUNCTION}
EncodeRfsQuery}}

\hypertarget{ecldoc:file.encoderfsquery}{}
\hspace{0pt} \hyperlink{ecldoc:File}{File} \textbackslash 

{\renewcommand{\arraystretch}{1.5}
\begin{tabularx}{\textwidth}{|>{\raggedright\arraybackslash}l|X|}
\hline
\hspace{0pt}\mytexttt{\color{red} varstring} & \textbf{EncodeRfsQuery} \\
\hline
\multicolumn{2}{|>{\raggedright\arraybackslash}X|}{\hspace{0pt}\mytexttt{\color{param} (varstring server, varstring query)}} \\
\hline
\end{tabularx}
}

\par
Returns a string that can be used in a DATASET declaration to read data from an RFS (Remote File Server) instance (e.g. rfsmysql) on another node.

\par
\begin{description}
\item [\colorbox{tagtype}{\color{white} \textbf{\textsf{PARAMETER}}}] \textbf{\underline{server}} A string containing the ip:port address for the remote file server.
\item [\colorbox{tagtype}{\color{white} \textbf{\textsf{PARAMETER}}}] \textbf{\underline{query}} The text of the query to send to the server
\end{description}

\rule{\linewidth}{0.5pt}
\subsection*{\textsf{\colorbox{headtoc}{\color{white} FUNCTION}
RfsAction}}

\hypertarget{ecldoc:file.rfsaction}{}
\hspace{0pt} \hyperlink{ecldoc:File}{File} \textbackslash 

{\renewcommand{\arraystretch}{1.5}
\begin{tabularx}{\textwidth}{|>{\raggedright\arraybackslash}l|X|}
\hline
\hspace{0pt}\mytexttt{\color{red} } & \textbf{RfsAction} \\
\hline
\multicolumn{2}{|>{\raggedright\arraybackslash}X|}{\hspace{0pt}\mytexttt{\color{param} (varstring server, varstring query)}} \\
\hline
\end{tabularx}
}

\par
Sends the query to the rfs server.

\par
\begin{description}
\item [\colorbox{tagtype}{\color{white} \textbf{\textsf{PARAMETER}}}] \textbf{\underline{server}} A string containing the ip:port address for the remote file server.
\item [\colorbox{tagtype}{\color{white} \textbf{\textsf{PARAMETER}}}] \textbf{\underline{query}} The text of the query to send to the server
\end{description}

\rule{\linewidth}{0.5pt}
\subsection*{\textsf{\colorbox{headtoc}{\color{white} FUNCTION}
MoveExternalFile}}

\hypertarget{ecldoc:file.moveexternalfile}{}
\hspace{0pt} \hyperlink{ecldoc:File}{File} \textbackslash 

{\renewcommand{\arraystretch}{1.5}
\begin{tabularx}{\textwidth}{|>{\raggedright\arraybackslash}l|X|}
\hline
\hspace{0pt}\mytexttt{\color{red} } & \textbf{MoveExternalFile} \\
\hline
\multicolumn{2}{|>{\raggedright\arraybackslash}X|}{\hspace{0pt}\mytexttt{\color{param} (varstring location, varstring frompath, varstring topath)}} \\
\hline
\end{tabularx}
}

\par
Moves the single physical file between two locations on the same remote machine. The dafileserv utility program must be running on the location machine.

\par
\begin{description}
\item [\colorbox{tagtype}{\color{white} \textbf{\textsf{PARAMETER}}}] \textbf{\underline{location}} The IP address of the remote machine.
\item [\colorbox{tagtype}{\color{white} \textbf{\textsf{PARAMETER}}}] \textbf{\underline{frompath}} The path/name of the file to move.
\item [\colorbox{tagtype}{\color{white} \textbf{\textsf{PARAMETER}}}] \textbf{\underline{topath}} The path/name of the target file.
\end{description}

\rule{\linewidth}{0.5pt}
\subsection*{\textsf{\colorbox{headtoc}{\color{white} FUNCTION}
DeleteExternalFile}}

\hypertarget{ecldoc:file.deleteexternalfile}{}
\hspace{0pt} \hyperlink{ecldoc:File}{File} \textbackslash 

{\renewcommand{\arraystretch}{1.5}
\begin{tabularx}{\textwidth}{|>{\raggedright\arraybackslash}l|X|}
\hline
\hspace{0pt}\mytexttt{\color{red} } & \textbf{DeleteExternalFile} \\
\hline
\multicolumn{2}{|>{\raggedright\arraybackslash}X|}{\hspace{0pt}\mytexttt{\color{param} (varstring location, varstring path)}} \\
\hline
\end{tabularx}
}

\par
Removes a single physical file from a remote machine. The dafileserv utility program must be running on the location machine.

\par
\begin{description}
\item [\colorbox{tagtype}{\color{white} \textbf{\textsf{PARAMETER}}}] \textbf{\underline{location}} The IP address of the remote machine.
\item [\colorbox{tagtype}{\color{white} \textbf{\textsf{PARAMETER}}}] \textbf{\underline{path}} The path/name of the file to remove.
\end{description}

\rule{\linewidth}{0.5pt}
\subsection*{\textsf{\colorbox{headtoc}{\color{white} FUNCTION}
CreateExternalDirectory}}

\hypertarget{ecldoc:file.createexternaldirectory}{}
\hspace{0pt} \hyperlink{ecldoc:File}{File} \textbackslash 

{\renewcommand{\arraystretch}{1.5}
\begin{tabularx}{\textwidth}{|>{\raggedright\arraybackslash}l|X|}
\hline
\hspace{0pt}\mytexttt{\color{red} } & \textbf{CreateExternalDirectory} \\
\hline
\multicolumn{2}{|>{\raggedright\arraybackslash}X|}{\hspace{0pt}\mytexttt{\color{param} (varstring location, varstring path)}} \\
\hline
\end{tabularx}
}

\par
Creates the path on the location (if it does not already exist). The dafileserv utility program must be running on the location machine.

\par
\begin{description}
\item [\colorbox{tagtype}{\color{white} \textbf{\textsf{PARAMETER}}}] \textbf{\underline{location}} The IP address of the remote machine.
\item [\colorbox{tagtype}{\color{white} \textbf{\textsf{PARAMETER}}}] \textbf{\underline{path}} The path/name of the file to remove.
\end{description}

\rule{\linewidth}{0.5pt}
\subsection*{\textsf{\colorbox{headtoc}{\color{white} FUNCTION}
GetLogicalFileAttribute}}

\hypertarget{ecldoc:file.getlogicalfileattribute}{}
\hspace{0pt} \hyperlink{ecldoc:File}{File} \textbackslash 

{\renewcommand{\arraystretch}{1.5}
\begin{tabularx}{\textwidth}{|>{\raggedright\arraybackslash}l|X|}
\hline
\hspace{0pt}\mytexttt{\color{red} varstring} & \textbf{GetLogicalFileAttribute} \\
\hline
\multicolumn{2}{|>{\raggedright\arraybackslash}X|}{\hspace{0pt}\mytexttt{\color{param} (varstring lfn, varstring attrname)}} \\
\hline
\end{tabularx}
}

\par
Returns the value of the given attribute for the specified logicalfilename.

\par
\begin{description}
\item [\colorbox{tagtype}{\color{white} \textbf{\textsf{PARAMETER}}}] \textbf{\underline{lfn}} The name of the logical file.
\item [\colorbox{tagtype}{\color{white} \textbf{\textsf{PARAMETER}}}] \textbf{\underline{attrname}} The name of the file attribute to return.
\end{description}

\rule{\linewidth}{0.5pt}
\subsection*{\textsf{\colorbox{headtoc}{\color{white} FUNCTION}
ProtectLogicalFile}}

\hypertarget{ecldoc:file.protectlogicalfile}{}
\hspace{0pt} \hyperlink{ecldoc:File}{File} \textbackslash 

{\renewcommand{\arraystretch}{1.5}
\begin{tabularx}{\textwidth}{|>{\raggedright\arraybackslash}l|X|}
\hline
\hspace{0pt}\mytexttt{\color{red} } & \textbf{ProtectLogicalFile} \\
\hline
\multicolumn{2}{|>{\raggedright\arraybackslash}X|}{\hspace{0pt}\mytexttt{\color{param} (varstring lfn, boolean value=TRUE)}} \\
\hline
\end{tabularx}
}

\par
Toggles protection on and off for the specified logicalfilename.

\par
\begin{description}
\item [\colorbox{tagtype}{\color{white} \textbf{\textsf{PARAMETER}}}] \textbf{\underline{lfn}} The name of the logical file.
\item [\colorbox{tagtype}{\color{white} \textbf{\textsf{PARAMETER}}}] \textbf{\underline{value}} TRUE to enable protection, FALSE to disable.
\end{description}

\rule{\linewidth}{0.5pt}
\subsection*{\textsf{\colorbox{headtoc}{\color{white} FUNCTION}
DfuPlusExec}}

\hypertarget{ecldoc:file.dfuplusexec}{}
\hspace{0pt} \hyperlink{ecldoc:File}{File} \textbackslash 

{\renewcommand{\arraystretch}{1.5}
\begin{tabularx}{\textwidth}{|>{\raggedright\arraybackslash}l|X|}
\hline
\hspace{0pt}\mytexttt{\color{red} } & \textbf{DfuPlusExec} \\
\hline
\multicolumn{2}{|>{\raggedright\arraybackslash}X|}{\hspace{0pt}\mytexttt{\color{param} (varstring cmdline)}} \\
\hline
\end{tabularx}
}

\par
The DfuPlusExec action executes the specified command line just as the DfuPLus.exe program would do. This allows you to have all the functionality of the DfuPLus.exe program available within your ECL code. param cmdline The DFUPlus.exe command line to execute. The valid arguments are documented in the Client Tools manual, in the section describing the DfuPlus.exe program.


\rule{\linewidth}{0.5pt}
\subsection*{\textsf{\colorbox{headtoc}{\color{white} FUNCTION}
fSprayFixed}}

\hypertarget{ecldoc:file.fsprayfixed}{}
\hspace{0pt} \hyperlink{ecldoc:File}{File} \textbackslash 

{\renewcommand{\arraystretch}{1.5}
\begin{tabularx}{\textwidth}{|>{\raggedright\arraybackslash}l|X|}
\hline
\hspace{0pt}\mytexttt{\color{red} varstring} & \textbf{fSprayFixed} \\
\hline
\multicolumn{2}{|>{\raggedright\arraybackslash}X|}{\hspace{0pt}\mytexttt{\color{param} (varstring sourceIP, varstring sourcePath, integer4 recordSize, varstring destinationGroup, varstring destinationLogicalName, integer4 timeOut=-1, varstring espServerIpPort=GETENV('ws\_fs\_server'), integer4 maxConnections=-1, boolean allowOverwrite=FALSE, boolean replicate=FALSE, boolean compress=FALSE, boolean failIfNoSourceFile=FALSE, integer4 expireDays=-1)}} \\
\hline
\end{tabularx}
}

\par
Sprays a file of fixed length records from a single machine and distributes it across the nodes of the destination group.

\par
\begin{description}
\item [\colorbox{tagtype}{\color{white} \textbf{\textsf{PARAMETER}}}] \textbf{\underline{sourceIP}} The IP address of the file.
\item [\colorbox{tagtype}{\color{white} \textbf{\textsf{PARAMETER}}}] \textbf{\underline{sourcePath}} The path and name of the file.
\item [\colorbox{tagtype}{\color{white} \textbf{\textsf{PARAMETER}}}] \textbf{\underline{recordsize}} The size (in bytes) of the records in the file.
\item [\colorbox{tagtype}{\color{white} \textbf{\textsf{PARAMETER}}}] \textbf{\underline{destinationGroup}} The name of the group to distribute the file across.
\item [\colorbox{tagtype}{\color{white} \textbf{\textsf{PARAMETER}}}] \textbf{\underline{destinationLogicalName}} The logical name of the file to create.
\item [\colorbox{tagtype}{\color{white} \textbf{\textsf{PARAMETER}}}] \textbf{\underline{timeOut}} The time in ms to wait for the operation to complete. A value of 0 causes the call to return immediately. Defaults to no timeout (-1).
\item [\colorbox{tagtype}{\color{white} \textbf{\textsf{PARAMETER}}}] \textbf{\underline{espServerIpPort}} The url of the ESP file copying service. Defaults to the value of ws\_fs\_server in the environment.
\item [\colorbox{tagtype}{\color{white} \textbf{\textsf{PARAMETER}}}] \textbf{\underline{maxConnections}} The maximum number of target nodes to write to concurrently. Defaults to 1.
\item [\colorbox{tagtype}{\color{white} \textbf{\textsf{PARAMETER}}}] \textbf{\underline{allowOverwrite}} Is it valid to overwrite an existing file of the same name? Defaults to FALSE
\item [\colorbox{tagtype}{\color{white} \textbf{\textsf{PARAMETER}}}] \textbf{\underline{replicate}} Whether to replicate the new file. Defaults to FALSE.
\item [\colorbox{tagtype}{\color{white} \textbf{\textsf{PARAMETER}}}] \textbf{\underline{compress}} Whether to compress the new file. Defaults to FALSE.
\item [\colorbox{tagtype}{\color{white} \textbf{\textsf{PARAMETER}}}] \textbf{\underline{failIfNoSourceFile}} If TRUE it causes a missing source file to trigger a failure. Defaults to FALSE.
\item [\colorbox{tagtype}{\color{white} \textbf{\textsf{PARAMETER}}}] \textbf{\underline{expireDays}} Number of days to auto-remove file. Default is -1, not expire.
\item [\colorbox{tagtype}{\color{white} \textbf{\textsf{RETURN}}}] \textbf{\underline{}} The DFU workunit id for the job.
\end{description}

\rule{\linewidth}{0.5pt}
\subsection*{\textsf{\colorbox{headtoc}{\color{white} FUNCTION}
SprayFixed}}

\hypertarget{ecldoc:file.sprayfixed}{}
\hspace{0pt} \hyperlink{ecldoc:File}{File} \textbackslash 

{\renewcommand{\arraystretch}{1.5}
\begin{tabularx}{\textwidth}{|>{\raggedright\arraybackslash}l|X|}
\hline
\hspace{0pt}\mytexttt{\color{red} } & \textbf{SprayFixed} \\
\hline
\multicolumn{2}{|>{\raggedright\arraybackslash}X|}{\hspace{0pt}\mytexttt{\color{param} (varstring sourceIP, varstring sourcePath, integer4 recordSize, varstring destinationGroup, varstring destinationLogicalName, integer4 timeOut=-1, varstring espServerIpPort=GETENV('ws\_fs\_server'), integer4 maxConnections=-1, boolean allowOverwrite=FALSE, boolean replicate=FALSE, boolean compress=FALSE, boolean failIfNoSourceFile=FALSE, integer4 expireDays=-1)}} \\
\hline
\end{tabularx}
}

\par
Same as fSprayFixed, but does not return the DFU Workunit ID.

\par
\begin{description}
\item [\colorbox{tagtype}{\color{white} \textbf{\textsf{SEE}}}] \textbf{\underline{}} fSprayFixed
\end{description}

\rule{\linewidth}{0.5pt}
\subsection*{\textsf{\colorbox{headtoc}{\color{white} FUNCTION}
fSprayVariable}}

\hypertarget{ecldoc:file.fsprayvariable}{}
\hspace{0pt} \hyperlink{ecldoc:File}{File} \textbackslash 

{\renewcommand{\arraystretch}{1.5}
\begin{tabularx}{\textwidth}{|>{\raggedright\arraybackslash}l|X|}
\hline
\hspace{0pt}\mytexttt{\color{red} varstring} & \textbf{fSprayVariable} \\
\hline
\multicolumn{2}{|>{\raggedright\arraybackslash}X|}{\hspace{0pt}\mytexttt{\color{param} (varstring sourceIP, varstring sourcePath, integer4 sourceMaxRecordSize=8192, varstring sourceCsvSeparate='\textbackslash \textbackslash ,', varstring sourceCsvTerminate='\textbackslash \textbackslash n,\textbackslash \textbackslash r\textbackslash \textbackslash n', varstring sourceCsvQuote='\textbackslash ''', varstring destinationGroup, varstring destinationLogicalName, integer4 timeOut=-1, varstring espServerIpPort=GETENV('ws\_fs\_server'), integer4 maxConnections=-1, boolean allowOverwrite=FALSE, boolean replicate=FALSE, boolean compress=FALSE, varstring sourceCsvEscape='', boolean failIfNoSourceFile=FALSE, boolean recordStructurePresent=FALSE, boolean quotedTerminator=TRUE, varstring encoding='ascii', integer4 expireDays=-1)}} \\
\hline
\end{tabularx}
}

\par


\rule{\linewidth}{0.5pt}
\subsection*{\textsf{\colorbox{headtoc}{\color{white} FUNCTION}
SprayVariable}}

\hypertarget{ecldoc:file.sprayvariable}{}
\hspace{0pt} \hyperlink{ecldoc:File}{File} \textbackslash 

{\renewcommand{\arraystretch}{1.5}
\begin{tabularx}{\textwidth}{|>{\raggedright\arraybackslash}l|X|}
\hline
\hspace{0pt}\mytexttt{\color{red} } & \textbf{SprayVariable} \\
\hline
\multicolumn{2}{|>{\raggedright\arraybackslash}X|}{\hspace{0pt}\mytexttt{\color{param} (varstring sourceIP, varstring sourcePath, integer4 sourceMaxRecordSize=8192, varstring sourceCsvSeparate='\textbackslash \textbackslash ,', varstring sourceCsvTerminate='\textbackslash \textbackslash n,\textbackslash \textbackslash r\textbackslash \textbackslash n', varstring sourceCsvQuote='\textbackslash ''', varstring destinationGroup, varstring destinationLogicalName, integer4 timeOut=-1, varstring espServerIpPort=GETENV('ws\_fs\_server'), integer4 maxConnections=-1, boolean allowOverwrite=FALSE, boolean replicate=FALSE, boolean compress=FALSE, varstring sourceCsvEscape='', boolean failIfNoSourceFile=FALSE, boolean recordStructurePresent=FALSE, boolean quotedTerminator=TRUE, varstring encoding='ascii', integer4 expireDays=-1)}} \\
\hline
\end{tabularx}
}

\par


\rule{\linewidth}{0.5pt}
\subsection*{\textsf{\colorbox{headtoc}{\color{white} FUNCTION}
fSprayDelimited}}

\hypertarget{ecldoc:file.fspraydelimited}{}
\hspace{0pt} \hyperlink{ecldoc:File}{File} \textbackslash 

{\renewcommand{\arraystretch}{1.5}
\begin{tabularx}{\textwidth}{|>{\raggedright\arraybackslash}l|X|}
\hline
\hspace{0pt}\mytexttt{\color{red} varstring} & \textbf{fSprayDelimited} \\
\hline
\multicolumn{2}{|>{\raggedright\arraybackslash}X|}{\hspace{0pt}\mytexttt{\color{param} (varstring sourceIP, varstring sourcePath, integer4 sourceMaxRecordSize=8192, varstring sourceCsvSeparate='\textbackslash \textbackslash ,', varstring sourceCsvTerminate='\textbackslash \textbackslash n,\textbackslash \textbackslash r\textbackslash \textbackslash n', varstring sourceCsvQuote='\textbackslash ''', varstring destinationGroup, varstring destinationLogicalName, integer4 timeOut=-1, varstring espServerIpPort=GETENV('ws\_fs\_server'), integer4 maxConnections=-1, boolean allowOverwrite=FALSE, boolean replicate=FALSE, boolean compress=FALSE, varstring sourceCsvEscape='', boolean failIfNoSourceFile=FALSE, boolean recordStructurePresent=FALSE, boolean quotedTerminator=TRUE, varstring encoding='ascii', integer4 expireDays=-1)}} \\
\hline
\end{tabularx}
}

\par
Sprays a file of fixed delimited records from a single machine and distributes it across the nodes of the destination group.

\par
\begin{description}
\item [\colorbox{tagtype}{\color{white} \textbf{\textsf{PARAMETER}}}] \textbf{\underline{sourceIP}} The IP address of the file.
\item [\colorbox{tagtype}{\color{white} \textbf{\textsf{PARAMETER}}}] \textbf{\underline{sourcePath}} The path and name of the file.
\item [\colorbox{tagtype}{\color{white} \textbf{\textsf{PARAMETER}}}] \textbf{\underline{sourceCsvSeparate}} The character sequence which separates fields in the file.
\item [\colorbox{tagtype}{\color{white} \textbf{\textsf{PARAMETER}}}] \textbf{\underline{sourceCsvTerminate}} The character sequence which separates records in the file.
\item [\colorbox{tagtype}{\color{white} \textbf{\textsf{PARAMETER}}}] \textbf{\underline{sourceCsvQuote}} A string which can be used to delimit fields in the file.
\item [\colorbox{tagtype}{\color{white} \textbf{\textsf{PARAMETER}}}] \textbf{\underline{sourceMaxRecordSize}} The maximum size (in bytes) of the records in the file.
\item [\colorbox{tagtype}{\color{white} \textbf{\textsf{PARAMETER}}}] \textbf{\underline{destinationGroup}} The name of the group to distribute the file across.
\item [\colorbox{tagtype}{\color{white} \textbf{\textsf{PARAMETER}}}] \textbf{\underline{destinationLogicalName}} The logical name of the file to create.
\item [\colorbox{tagtype}{\color{white} \textbf{\textsf{PARAMETER}}}] \textbf{\underline{timeOut}} The time in ms to wait for the operation to complete. A value of 0 causes the call to return immediately. Defaults to no timeout (-1).
\item [\colorbox{tagtype}{\color{white} \textbf{\textsf{PARAMETER}}}] \textbf{\underline{espServerIpPort}} The url of the ESP file copying service. Defaults to the value of ws\_fs\_server in the environment.
\item [\colorbox{tagtype}{\color{white} \textbf{\textsf{PARAMETER}}}] \textbf{\underline{maxConnections}} The maximum number of target nodes to write to concurrently. Defaults to 1.
\item [\colorbox{tagtype}{\color{white} \textbf{\textsf{PARAMETER}}}] \textbf{\underline{allowOverwrite}} Is it valid to overwrite an existing file of the same name? Defaults to FALSE
\item [\colorbox{tagtype}{\color{white} \textbf{\textsf{PARAMETER}}}] \textbf{\underline{replicate}} Whether to replicate the new file. Defaults to FALSE.
\item [\colorbox{tagtype}{\color{white} \textbf{\textsf{PARAMETER}}}] \textbf{\underline{compress}} Whether to compress the new file. Defaults to FALSE.
\item [\colorbox{tagtype}{\color{white} \textbf{\textsf{PARAMETER}}}] \textbf{\underline{sourceCsvEscape}} A character that is used to escape quote characters. Defaults to none.
\item [\colorbox{tagtype}{\color{white} \textbf{\textsf{PARAMETER}}}] \textbf{\underline{failIfNoSourceFile}} If TRUE it causes a missing source file to trigger a failure. Defaults to FALSE.
\item [\colorbox{tagtype}{\color{white} \textbf{\textsf{PARAMETER}}}] \textbf{\underline{recordStructurePresent}} If TRUE derives the record structure from the header of the file.
\item [\colorbox{tagtype}{\color{white} \textbf{\textsf{PARAMETER}}}] \textbf{\underline{quotedTerminator}} Can the terminator character be included in a quoted field. Defaults to TRUE. If FALSE it allows quicker partitioning of the file (avoiding a complete file scan).
\item [\colorbox{tagtype}{\color{white} \textbf{\textsf{PARAMETER}}}] \textbf{\underline{expireDays}} Number of days to auto-remove file. Default is -1, not expire.
\item [\colorbox{tagtype}{\color{white} \textbf{\textsf{RETURN}}}] \textbf{\underline{}} The DFU workunit id for the job.
\end{description}

\rule{\linewidth}{0.5pt}
\subsection*{\textsf{\colorbox{headtoc}{\color{white} FUNCTION}
SprayDelimited}}

\hypertarget{ecldoc:file.spraydelimited}{}
\hspace{0pt} \hyperlink{ecldoc:File}{File} \textbackslash 

{\renewcommand{\arraystretch}{1.5}
\begin{tabularx}{\textwidth}{|>{\raggedright\arraybackslash}l|X|}
\hline
\hspace{0pt}\mytexttt{\color{red} } & \textbf{SprayDelimited} \\
\hline
\multicolumn{2}{|>{\raggedright\arraybackslash}X|}{\hspace{0pt}\mytexttt{\color{param} (varstring sourceIP, varstring sourcePath, integer4 sourceMaxRecordSize=8192, varstring sourceCsvSeparate='\textbackslash \textbackslash ,', varstring sourceCsvTerminate='\textbackslash \textbackslash n,\textbackslash \textbackslash r\textbackslash \textbackslash n', varstring sourceCsvQuote='\textbackslash ''', varstring destinationGroup, varstring destinationLogicalName, integer4 timeOut=-1, varstring espServerIpPort=GETENV('ws\_fs\_server'), integer4 maxConnections=-1, boolean allowOverwrite=FALSE, boolean replicate=FALSE, boolean compress=FALSE, varstring sourceCsvEscape='', boolean failIfNoSourceFile=FALSE, boolean recordStructurePresent=FALSE, boolean quotedTerminator=TRUE, const varstring encoding='ascii', integer4 expireDays=-1)}} \\
\hline
\end{tabularx}
}

\par
Same as fSprayDelimited, but does not return the DFU Workunit ID.

\par
\begin{description}
\item [\colorbox{tagtype}{\color{white} \textbf{\textsf{SEE}}}] \textbf{\underline{}} fSprayDelimited
\end{description}

\rule{\linewidth}{0.5pt}
\subsection*{\textsf{\colorbox{headtoc}{\color{white} FUNCTION}
fSprayXml}}

\hypertarget{ecldoc:file.fsprayxml}{}
\hspace{0pt} \hyperlink{ecldoc:File}{File} \textbackslash 

{\renewcommand{\arraystretch}{1.5}
\begin{tabularx}{\textwidth}{|>{\raggedright\arraybackslash}l|X|}
\hline
\hspace{0pt}\mytexttt{\color{red} varstring} & \textbf{fSprayXml} \\
\hline
\multicolumn{2}{|>{\raggedright\arraybackslash}X|}{\hspace{0pt}\mytexttt{\color{param} (varstring sourceIP, varstring sourcePath, integer4 sourceMaxRecordSize=8192, varstring sourceRowTag, varstring sourceEncoding='utf8', varstring destinationGroup, varstring destinationLogicalName, integer4 timeOut=-1, varstring espServerIpPort=GETENV('ws\_fs\_server'), integer4 maxConnections=-1, boolean allowOverwrite=FALSE, boolean replicate=FALSE, boolean compress=FALSE, boolean failIfNoSourceFile=FALSE, integer4 expireDays=-1)}} \\
\hline
\end{tabularx}
}

\par
Sprays an xml file from a single machine and distributes it across the nodes of the destination group.

\par
\begin{description}
\item [\colorbox{tagtype}{\color{white} \textbf{\textsf{PARAMETER}}}] \textbf{\underline{sourceIP}} The IP address of the file.
\item [\colorbox{tagtype}{\color{white} \textbf{\textsf{PARAMETER}}}] \textbf{\underline{sourcePath}} The path and name of the file.
\item [\colorbox{tagtype}{\color{white} \textbf{\textsf{PARAMETER}}}] \textbf{\underline{sourceMaxRecordSize}} The maximum size (in bytes) of the records in the file.
\item [\colorbox{tagtype}{\color{white} \textbf{\textsf{PARAMETER}}}] \textbf{\underline{sourceRowTag}} The xml tag that is used to delimit records in the source file. (This tag cannot recursivly nest.)
\item [\colorbox{tagtype}{\color{white} \textbf{\textsf{PARAMETER}}}] \textbf{\underline{sourceEncoding}} The unicode encoding of the file. (utf8,utf8n,utf16be,utf16le,utf32be,utf32le)
\item [\colorbox{tagtype}{\color{white} \textbf{\textsf{PARAMETER}}}] \textbf{\underline{destinationGroup}} The name of the group to distribute the file across.
\item [\colorbox{tagtype}{\color{white} \textbf{\textsf{PARAMETER}}}] \textbf{\underline{destinationLogicalName}} The logical name of the file to create.
\item [\colorbox{tagtype}{\color{white} \textbf{\textsf{PARAMETER}}}] \textbf{\underline{timeOut}} The time in ms to wait for the operation to complete. A value of 0 causes the call to return immediately. Defaults to no timeout (-1).
\item [\colorbox{tagtype}{\color{white} \textbf{\textsf{PARAMETER}}}] \textbf{\underline{espServerIpPort}} The url of the ESP file copying service. Defaults to the value of ws\_fs\_server in the environment.
\item [\colorbox{tagtype}{\color{white} \textbf{\textsf{PARAMETER}}}] \textbf{\underline{maxConnections}} The maximum number of target nodes to write to concurrently. Defaults to 1.
\item [\colorbox{tagtype}{\color{white} \textbf{\textsf{PARAMETER}}}] \textbf{\underline{allowOverwrite}} Is it valid to overwrite an existing file of the same name? Defaults to FALSE
\item [\colorbox{tagtype}{\color{white} \textbf{\textsf{PARAMETER}}}] \textbf{\underline{replicate}} Whether to replicate the new file. Defaults to FALSE.
\item [\colorbox{tagtype}{\color{white} \textbf{\textsf{PARAMETER}}}] \textbf{\underline{compress}} Whether to compress the new file. Defaults to FALSE.
\item [\colorbox{tagtype}{\color{white} \textbf{\textsf{PARAMETER}}}] \textbf{\underline{failIfNoSourceFile}} If TRUE it causes a missing source file to trigger a failure. Defaults to FALSE.
\item [\colorbox{tagtype}{\color{white} \textbf{\textsf{PARAMETER}}}] \textbf{\underline{expireDays}} Number of days to auto-remove file. Default is -1, not expire.
\item [\colorbox{tagtype}{\color{white} \textbf{\textsf{RETURN}}}] \textbf{\underline{}} The DFU workunit id for the job.
\end{description}

\rule{\linewidth}{0.5pt}
\subsection*{\textsf{\colorbox{headtoc}{\color{white} FUNCTION}
SprayXml}}

\hypertarget{ecldoc:file.sprayxml}{}
\hspace{0pt} \hyperlink{ecldoc:File}{File} \textbackslash 

{\renewcommand{\arraystretch}{1.5}
\begin{tabularx}{\textwidth}{|>{\raggedright\arraybackslash}l|X|}
\hline
\hspace{0pt}\mytexttt{\color{red} } & \textbf{SprayXml} \\
\hline
\multicolumn{2}{|>{\raggedright\arraybackslash}X|}{\hspace{0pt}\mytexttt{\color{param} (varstring sourceIP, varstring sourcePath, integer4 sourceMaxRecordSize=8192, varstring sourceRowTag, varstring sourceEncoding='utf8', varstring destinationGroup, varstring destinationLogicalName, integer4 timeOut=-1, varstring espServerIpPort=GETENV('ws\_fs\_server'), integer4 maxConnections=-1, boolean allowOverwrite=FALSE, boolean replicate=FALSE, boolean compress=FALSE, boolean failIfNoSourceFile=FALSE, integer4 expireDays=-1)}} \\
\hline
\end{tabularx}
}

\par
Same as fSprayXml, but does not return the DFU Workunit ID.

\par
\begin{description}
\item [\colorbox{tagtype}{\color{white} \textbf{\textsf{SEE}}}] \textbf{\underline{}} fSprayXml
\end{description}

\rule{\linewidth}{0.5pt}
\subsection*{\textsf{\colorbox{headtoc}{\color{white} FUNCTION}
fDespray}}

\hypertarget{ecldoc:file.fdespray}{}
\hspace{0pt} \hyperlink{ecldoc:File}{File} \textbackslash 

{\renewcommand{\arraystretch}{1.5}
\begin{tabularx}{\textwidth}{|>{\raggedright\arraybackslash}l|X|}
\hline
\hspace{0pt}\mytexttt{\color{red} varstring} & \textbf{fDespray} \\
\hline
\multicolumn{2}{|>{\raggedright\arraybackslash}X|}{\hspace{0pt}\mytexttt{\color{param} (varstring logicalName, varstring destinationIP, varstring destinationPath, integer4 timeOut=-1, varstring espServerIpPort=GETENV('ws\_fs\_server'), integer4 maxConnections=-1, boolean allowOverwrite=FALSE)}} \\
\hline
\end{tabularx}
}

\par
Copies a distributed file from multiple machines, and desprays it to a single file on a single machine.

\par
\begin{description}
\item [\colorbox{tagtype}{\color{white} \textbf{\textsf{PARAMETER}}}] \textbf{\underline{logicalName}} The name of the file to despray.
\item [\colorbox{tagtype}{\color{white} \textbf{\textsf{PARAMETER}}}] \textbf{\underline{destinationIP}} The IP of the target machine.
\item [\colorbox{tagtype}{\color{white} \textbf{\textsf{PARAMETER}}}] \textbf{\underline{destinationPath}} The path of the file to create on the destination machine.
\item [\colorbox{tagtype}{\color{white} \textbf{\textsf{PARAMETER}}}] \textbf{\underline{timeOut}} The time in ms to wait for the operation to complete. A value of 0 causes the call to return immediately. Defaults to no timeout (-1).
\item [\colorbox{tagtype}{\color{white} \textbf{\textsf{PARAMETER}}}] \textbf{\underline{espServerIpPort}} The url of the ESP file copying service. Defaults to the value of ws\_fs\_server in the environment.
\item [\colorbox{tagtype}{\color{white} \textbf{\textsf{PARAMETER}}}] \textbf{\underline{maxConnections}} The maximum number of target nodes to write to concurrently. Defaults to 1.
\item [\colorbox{tagtype}{\color{white} \textbf{\textsf{PARAMETER}}}] \textbf{\underline{allowOverwrite}} Is it valid to overwrite an existing file of the same name? Defaults to FALSE
\item [\colorbox{tagtype}{\color{white} \textbf{\textsf{RETURN}}}] \textbf{\underline{}} The DFU workunit id for the job.
\end{description}

\rule{\linewidth}{0.5pt}
\subsection*{\textsf{\colorbox{headtoc}{\color{white} FUNCTION}
Despray}}

\hypertarget{ecldoc:file.despray}{}
\hspace{0pt} \hyperlink{ecldoc:File}{File} \textbackslash 

{\renewcommand{\arraystretch}{1.5}
\begin{tabularx}{\textwidth}{|>{\raggedright\arraybackslash}l|X|}
\hline
\hspace{0pt}\mytexttt{\color{red} } & \textbf{Despray} \\
\hline
\multicolumn{2}{|>{\raggedright\arraybackslash}X|}{\hspace{0pt}\mytexttt{\color{param} (varstring logicalName, varstring destinationIP, varstring destinationPath, integer4 timeOut=-1, varstring espServerIpPort=GETENV('ws\_fs\_server'), integer4 maxConnections=-1, boolean allowOverwrite=FALSE)}} \\
\hline
\end{tabularx}
}

\par
Same as fDespray, but does not return the DFU Workunit ID.

\par
\begin{description}
\item [\colorbox{tagtype}{\color{white} \textbf{\textsf{SEE}}}] \textbf{\underline{}} fDespray
\end{description}

\rule{\linewidth}{0.5pt}
\subsection*{\textsf{\colorbox{headtoc}{\color{white} FUNCTION}
fCopy}}

\hypertarget{ecldoc:file.fcopy}{}
\hspace{0pt} \hyperlink{ecldoc:File}{File} \textbackslash 

{\renewcommand{\arraystretch}{1.5}
\begin{tabularx}{\textwidth}{|>{\raggedright\arraybackslash}l|X|}
\hline
\hspace{0pt}\mytexttt{\color{red} varstring} & \textbf{fCopy} \\
\hline
\multicolumn{2}{|>{\raggedright\arraybackslash}X|}{\hspace{0pt}\mytexttt{\color{param} (varstring sourceLogicalName, varstring destinationGroup, varstring destinationLogicalName, varstring sourceDali='', integer4 timeOut=-1, varstring espServerIpPort=GETENV('ws\_fs\_server'), integer4 maxConnections=-1, boolean allowOverwrite=FALSE, boolean replicate=FALSE, boolean asSuperfile=FALSE, boolean compress=FALSE, boolean forcePush=FALSE, integer4 transferBufferSize=0, boolean preserveCompression=TRUE)}} \\
\hline
\end{tabularx}
}

\par
Copies a distributed file to another distributed file.

\par
\begin{description}
\item [\colorbox{tagtype}{\color{white} \textbf{\textsf{PARAMETER}}}] \textbf{\underline{sourceLogicalName}} The name of the file to despray.
\item [\colorbox{tagtype}{\color{white} \textbf{\textsf{PARAMETER}}}] \textbf{\underline{destinationGroup}} The name of the group to distribute the file across.
\item [\colorbox{tagtype}{\color{white} \textbf{\textsf{PARAMETER}}}] \textbf{\underline{destinationLogicalName}} The logical name of the file to create.
\item [\colorbox{tagtype}{\color{white} \textbf{\textsf{PARAMETER}}}] \textbf{\underline{sourceDali}} The dali that contains the source file (blank implies same dali). Defaults to same dali.
\item [\colorbox{tagtype}{\color{white} \textbf{\textsf{PARAMETER}}}] \textbf{\underline{timeOut}} The time in ms to wait for the operation to complete. A value of 0 causes the call to return immediately. Defaults to no timeout (-1).
\item [\colorbox{tagtype}{\color{white} \textbf{\textsf{PARAMETER}}}] \textbf{\underline{espServerIpPort}} The url of the ESP file copying service. Defaults to the value of ws\_fs\_server in the environment.
\item [\colorbox{tagtype}{\color{white} \textbf{\textsf{PARAMETER}}}] \textbf{\underline{maxConnections}} The maximum number of target nodes to write to concurrently. Defaults to 1.
\item [\colorbox{tagtype}{\color{white} \textbf{\textsf{PARAMETER}}}] \textbf{\underline{allowOverwrite}} Is it valid to overwrite an existing file of the same name? Defaults to FALSE
\item [\colorbox{tagtype}{\color{white} \textbf{\textsf{PARAMETER}}}] \textbf{\underline{replicate}} Should the copied file also be replicated on the destination? Defaults to FALSE
\item [\colorbox{tagtype}{\color{white} \textbf{\textsf{PARAMETER}}}] \textbf{\underline{asSuperfile}} Should the file be copied as a superfile? If TRUE and source is a superfile, then the operation creates a superfile on the target, creating sub-files as needed and only overwriting existing sub-files whose content has changed. If FALSE, a single file is created. Defaults to FALSE.
\item [\colorbox{tagtype}{\color{white} \textbf{\textsf{PARAMETER}}}] \textbf{\underline{compress}} Whether to compress the new file. Defaults to FALSE.
\item [\colorbox{tagtype}{\color{white} \textbf{\textsf{PARAMETER}}}] \textbf{\underline{forcePush}} Should the copy process be executed on the source nodes (push) or on the destination nodes (pull)? Default is to pull.
\item [\colorbox{tagtype}{\color{white} \textbf{\textsf{PARAMETER}}}] \textbf{\underline{transferBufferSize}} Overrides the size (in bytes) of the internal buffer used to copy the file. Default is 64k.
\item [\colorbox{tagtype}{\color{white} \textbf{\textsf{RETURN}}}] \textbf{\underline{}} The DFU workunit id for the job.
\end{description}

\rule{\linewidth}{0.5pt}
\subsection*{\textsf{\colorbox{headtoc}{\color{white} FUNCTION}
Copy}}

\hypertarget{ecldoc:file.copy}{}
\hspace{0pt} \hyperlink{ecldoc:File}{File} \textbackslash 

{\renewcommand{\arraystretch}{1.5}
\begin{tabularx}{\textwidth}{|>{\raggedright\arraybackslash}l|X|}
\hline
\hspace{0pt}\mytexttt{\color{red} } & \textbf{Copy} \\
\hline
\multicolumn{2}{|>{\raggedright\arraybackslash}X|}{\hspace{0pt}\mytexttt{\color{param} (varstring sourceLogicalName, varstring destinationGroup, varstring destinationLogicalName, varstring sourceDali='', integer4 timeOut=-1, varstring espServerIpPort=GETENV('ws\_fs\_server'), integer4 maxConnections=-1, boolean allowOverwrite=FALSE, boolean replicate=FALSE, boolean asSuperfile=FALSE, boolean compress=FALSE, boolean forcePush=FALSE, integer4 transferBufferSize=0, boolean preserveCompression=TRUE)}} \\
\hline
\end{tabularx}
}

\par
Same as fCopy, but does not return the DFU Workunit ID.

\par
\begin{description}
\item [\colorbox{tagtype}{\color{white} \textbf{\textsf{SEE}}}] \textbf{\underline{}} fCopy
\end{description}

\rule{\linewidth}{0.5pt}
\subsection*{\textsf{\colorbox{headtoc}{\color{white} FUNCTION}
fReplicate}}

\hypertarget{ecldoc:file.freplicate}{}
\hspace{0pt} \hyperlink{ecldoc:File}{File} \textbackslash 

{\renewcommand{\arraystretch}{1.5}
\begin{tabularx}{\textwidth}{|>{\raggedright\arraybackslash}l|X|}
\hline
\hspace{0pt}\mytexttt{\color{red} varstring} & \textbf{fReplicate} \\
\hline
\multicolumn{2}{|>{\raggedright\arraybackslash}X|}{\hspace{0pt}\mytexttt{\color{param} (varstring logicalName, integer4 timeOut=-1, varstring espServerIpPort=GETENV('ws\_fs\_server'))}} \\
\hline
\end{tabularx}
}

\par
Ensures the specified file is replicated to its mirror copies.

\par
\begin{description}
\item [\colorbox{tagtype}{\color{white} \textbf{\textsf{PARAMETER}}}] \textbf{\underline{logicalName}} The name of the file to replicate.
\item [\colorbox{tagtype}{\color{white} \textbf{\textsf{PARAMETER}}}] \textbf{\underline{timeOut}} The time in ms to wait for the operation to complete. A value of 0 causes the call to return immediately. Defaults to no timeout (-1).
\item [\colorbox{tagtype}{\color{white} \textbf{\textsf{PARAMETER}}}] \textbf{\underline{espServerIpPort}} The url of the ESP file copying service. Defaults to the value of ws\_fs\_server in the environment.
\item [\colorbox{tagtype}{\color{white} \textbf{\textsf{RETURN}}}] \textbf{\underline{}} The DFU workunit id for the job.
\end{description}

\rule{\linewidth}{0.5pt}
\subsection*{\textsf{\colorbox{headtoc}{\color{white} FUNCTION}
Replicate}}

\hypertarget{ecldoc:file.replicate}{}
\hspace{0pt} \hyperlink{ecldoc:File}{File} \textbackslash 

{\renewcommand{\arraystretch}{1.5}
\begin{tabularx}{\textwidth}{|>{\raggedright\arraybackslash}l|X|}
\hline
\hspace{0pt}\mytexttt{\color{red} } & \textbf{Replicate} \\
\hline
\multicolumn{2}{|>{\raggedright\arraybackslash}X|}{\hspace{0pt}\mytexttt{\color{param} (varstring logicalName, integer4 timeOut=-1, varstring espServerIpPort=GETENV('ws\_fs\_server'))}} \\
\hline
\end{tabularx}
}

\par
Same as fReplicated, but does not return the DFU Workunit ID.

\par
\begin{description}
\item [\colorbox{tagtype}{\color{white} \textbf{\textsf{SEE}}}] \textbf{\underline{}} fReplicate
\end{description}

\rule{\linewidth}{0.5pt}
\subsection*{\textsf{\colorbox{headtoc}{\color{white} FUNCTION}
fRemotePull}}

\hypertarget{ecldoc:file.fremotepull}{}
\hspace{0pt} \hyperlink{ecldoc:File}{File} \textbackslash 

{\renewcommand{\arraystretch}{1.5}
\begin{tabularx}{\textwidth}{|>{\raggedright\arraybackslash}l|X|}
\hline
\hspace{0pt}\mytexttt{\color{red} varstring} & \textbf{fRemotePull} \\
\hline
\multicolumn{2}{|>{\raggedright\arraybackslash}X|}{\hspace{0pt}\mytexttt{\color{param} (varstring remoteEspFsURL, varstring sourceLogicalName, varstring destinationGroup, varstring destinationLogicalName, integer4 timeOut=-1, integer4 maxConnections=-1, boolean allowOverwrite=FALSE, boolean replicate=FALSE, boolean asSuperfile=FALSE, boolean forcePush=FALSE, integer4 transferBufferSize=0, boolean wrap=FALSE, boolean compress=FALSE)}} \\
\hline
\end{tabularx}
}

\par
Copies a distributed file to a distributed file on remote system. Similar to fCopy, except the copy executes remotely. Since the DFU workunit executes on the remote DFU server, the user name authentication must be the same on both systems, and the user must have rights to copy files on both systems.

\par
\begin{description}
\item [\colorbox{tagtype}{\color{white} \textbf{\textsf{PARAMETER}}}] \textbf{\underline{remoteEspFsURL}} The url of the remote ESP file copying service.
\item [\colorbox{tagtype}{\color{white} \textbf{\textsf{PARAMETER}}}] \textbf{\underline{sourceLogicalName}} The name of the file to despray.
\item [\colorbox{tagtype}{\color{white} \textbf{\textsf{PARAMETER}}}] \textbf{\underline{destinationGroup}} The name of the group to distribute the file across.
\item [\colorbox{tagtype}{\color{white} \textbf{\textsf{PARAMETER}}}] \textbf{\underline{destinationLogicalName}} The logical name of the file to create.
\item [\colorbox{tagtype}{\color{white} \textbf{\textsf{PARAMETER}}}] \textbf{\underline{timeOut}} The time in ms to wait for the operation to complete. A value of 0 causes the call to return immediately. Defaults to no timeout (-1).
\item [\colorbox{tagtype}{\color{white} \textbf{\textsf{PARAMETER}}}] \textbf{\underline{maxConnections}} The maximum number of target nodes to write to concurrently. Defaults to 1.
\item [\colorbox{tagtype}{\color{white} \textbf{\textsf{PARAMETER}}}] \textbf{\underline{allowOverwrite}} Is it valid to overwrite an existing file of the same name? Defaults to FALSE
\item [\colorbox{tagtype}{\color{white} \textbf{\textsf{PARAMETER}}}] \textbf{\underline{replicate}} Should the copied file also be replicated on the destination? Defaults to FALSE
\item [\colorbox{tagtype}{\color{white} \textbf{\textsf{PARAMETER}}}] \textbf{\underline{asSuperfile}} Should the file be copied as a superfile? If TRUE and source is a superfile, then the operation creates a superfile on the target, creating sub-files as needed and only overwriting existing sub-files whose content has changed. If FALSE a single file is created. Defaults to FALSE.
\item [\colorbox{tagtype}{\color{white} \textbf{\textsf{PARAMETER}}}] \textbf{\underline{compress}} Whether to compress the new file. Defaults to FALSE.
\item [\colorbox{tagtype}{\color{white} \textbf{\textsf{PARAMETER}}}] \textbf{\underline{forcePush}} Should the copy process should be executed on the source nodes (push) or on the destination nodes (pull)? Default is to pull.
\item [\colorbox{tagtype}{\color{white} \textbf{\textsf{PARAMETER}}}] \textbf{\underline{transferBufferSize}} Overrides the size (in bytes) of the internal buffer used to copy the file. Default is 64k.
\item [\colorbox{tagtype}{\color{white} \textbf{\textsf{PARAMETER}}}] \textbf{\underline{wrap}} Should the fileparts be wrapped when copying to a smaller sized cluster? The default is FALSE.
\item [\colorbox{tagtype}{\color{white} \textbf{\textsf{RETURN}}}] \textbf{\underline{}} The DFU workunit id for the job.
\end{description}

\rule{\linewidth}{0.5pt}
\subsection*{\textsf{\colorbox{headtoc}{\color{white} FUNCTION}
RemotePull}}

\hypertarget{ecldoc:file.remotepull}{}
\hspace{0pt} \hyperlink{ecldoc:File}{File} \textbackslash 

{\renewcommand{\arraystretch}{1.5}
\begin{tabularx}{\textwidth}{|>{\raggedright\arraybackslash}l|X|}
\hline
\hspace{0pt}\mytexttt{\color{red} } & \textbf{RemotePull} \\
\hline
\multicolumn{2}{|>{\raggedright\arraybackslash}X|}{\hspace{0pt}\mytexttt{\color{param} (varstring remoteEspFsURL, varstring sourceLogicalName, varstring destinationGroup, varstring destinationLogicalName, integer4 timeOut=-1, integer4 maxConnections=-1, boolean allowOverwrite=FALSE, boolean replicate=FALSE, boolean asSuperfile=FALSE, boolean forcePush=FALSE, integer4 transferBufferSize=0, boolean wrap=FALSE, boolean compress=FALSE)}} \\
\hline
\end{tabularx}
}

\par
Same as fRemotePull, but does not return the DFU Workunit ID.

\par
\begin{description}
\item [\colorbox{tagtype}{\color{white} \textbf{\textsf{SEE}}}] \textbf{\underline{}} fRemotePull
\end{description}

\rule{\linewidth}{0.5pt}
\subsection*{\textsf{\colorbox{headtoc}{\color{white} FUNCTION}
fMonitorLogicalFileName}}

\hypertarget{ecldoc:file.fmonitorlogicalfilename}{}
\hspace{0pt} \hyperlink{ecldoc:File}{File} \textbackslash 

{\renewcommand{\arraystretch}{1.5}
\begin{tabularx}{\textwidth}{|>{\raggedright\arraybackslash}l|X|}
\hline
\hspace{0pt}\mytexttt{\color{red} varstring} & \textbf{fMonitorLogicalFileName} \\
\hline
\multicolumn{2}{|>{\raggedright\arraybackslash}X|}{\hspace{0pt}\mytexttt{\color{param} (varstring eventToFire, varstring name, integer4 shotCount=1, varstring espServerIpPort=GETENV('ws\_fs\_server'))}} \\
\hline
\end{tabularx}
}

\par
Creates a file monitor job in the DFU Server. If an appropriately named file arrives in this interval it will fire the event with the name of the triggering object as the event subtype (see the EVENT function).

\par
\begin{description}
\item [\colorbox{tagtype}{\color{white} \textbf{\textsf{PARAMETER}}}] \textbf{\underline{eventToFire}} The user-defined name of the event to fire when the filename appears. This value is used as the first parameter to the EVENT function.
\item [\colorbox{tagtype}{\color{white} \textbf{\textsf{PARAMETER}}}] \textbf{\underline{name}} The name of the logical file to monitor. This may contain wildcard characters ( * and ?)
\item [\colorbox{tagtype}{\color{white} \textbf{\textsf{PARAMETER}}}] \textbf{\underline{shotCount}} The number of times to generate the event before the monitoring job completes. A value of -1 indicates the monitoring job continues until manually aborted. The default is 1.
\item [\colorbox{tagtype}{\color{white} \textbf{\textsf{PARAMETER}}}] \textbf{\underline{espServerIpPort}} The url of the ESP file copying service. Defaults to the value of ws\_fs\_server in the environment.
\item [\colorbox{tagtype}{\color{white} \textbf{\textsf{RETURN}}}] \textbf{\underline{}} The DFU workunit id for the job.
\end{description}

\rule{\linewidth}{0.5pt}
\subsection*{\textsf{\colorbox{headtoc}{\color{white} FUNCTION}
MonitorLogicalFileName}}

\hypertarget{ecldoc:file.monitorlogicalfilename}{}
\hspace{0pt} \hyperlink{ecldoc:File}{File} \textbackslash 

{\renewcommand{\arraystretch}{1.5}
\begin{tabularx}{\textwidth}{|>{\raggedright\arraybackslash}l|X|}
\hline
\hspace{0pt}\mytexttt{\color{red} } & \textbf{MonitorLogicalFileName} \\
\hline
\multicolumn{2}{|>{\raggedright\arraybackslash}X|}{\hspace{0pt}\mytexttt{\color{param} (varstring eventToFire, varstring name, integer4 shotCount=1, varstring espServerIpPort=GETENV('ws\_fs\_server'))}} \\
\hline
\end{tabularx}
}

\par
Same as fMonitorLogicalFileName, but does not return the DFU Workunit ID.

\par
\begin{description}
\item [\colorbox{tagtype}{\color{white} \textbf{\textsf{SEE}}}] \textbf{\underline{}} fMonitorLogicalFileName
\end{description}

\rule{\linewidth}{0.5pt}
\subsection*{\textsf{\colorbox{headtoc}{\color{white} FUNCTION}
fMonitorFile}}

\hypertarget{ecldoc:file.fmonitorfile}{}
\hspace{0pt} \hyperlink{ecldoc:File}{File} \textbackslash 

{\renewcommand{\arraystretch}{1.5}
\begin{tabularx}{\textwidth}{|>{\raggedright\arraybackslash}l|X|}
\hline
\hspace{0pt}\mytexttt{\color{red} varstring} & \textbf{fMonitorFile} \\
\hline
\multicolumn{2}{|>{\raggedright\arraybackslash}X|}{\hspace{0pt}\mytexttt{\color{param} (varstring eventToFire, varstring ip, varstring filename, boolean subDirs=FALSE, integer4 shotCount=1, varstring espServerIpPort=GETENV('ws\_fs\_server'))}} \\
\hline
\end{tabularx}
}

\par
Creates a file monitor job in the DFU Server. If an appropriately named file arrives in this interval it will fire the event with the name of the triggering object as the event subtype (see the EVENT function).

\par
\begin{description}
\item [\colorbox{tagtype}{\color{white} \textbf{\textsf{PARAMETER}}}] \textbf{\underline{eventToFire}} The user-defined name of the event to fire when the filename appears. This value is used as the first parameter to the EVENT function.
\item [\colorbox{tagtype}{\color{white} \textbf{\textsf{PARAMETER}}}] \textbf{\underline{ip}} The the IP address for the file to monitor. This may be omitted if the filename parameter contains a complete URL.
\item [\colorbox{tagtype}{\color{white} \textbf{\textsf{PARAMETER}}}] \textbf{\underline{filename}} The full path of the file(s) to monitor. This may contain wildcard characters ( * and ?)
\item [\colorbox{tagtype}{\color{white} \textbf{\textsf{PARAMETER}}}] \textbf{\underline{subDirs}} Whether to include files in sub-directories (when the filename contains wildcards). Defaults to FALSE.
\item [\colorbox{tagtype}{\color{white} \textbf{\textsf{PARAMETER}}}] \textbf{\underline{shotCount}} The number of times to generate the event before the monitoring job completes. A value of -1 indicates the monitoring job continues until manually aborted. The default is 1.
\item [\colorbox{tagtype}{\color{white} \textbf{\textsf{PARAMETER}}}] \textbf{\underline{espServerIpPort}} The url of the ESP file copying service. Defaults to the value of ws\_fs\_server in the environment.
\item [\colorbox{tagtype}{\color{white} \textbf{\textsf{RETURN}}}] \textbf{\underline{}} The DFU workunit id for the job.
\end{description}

\rule{\linewidth}{0.5pt}
\subsection*{\textsf{\colorbox{headtoc}{\color{white} FUNCTION}
MonitorFile}}

\hypertarget{ecldoc:file.monitorfile}{}
\hspace{0pt} \hyperlink{ecldoc:File}{File} \textbackslash 

{\renewcommand{\arraystretch}{1.5}
\begin{tabularx}{\textwidth}{|>{\raggedright\arraybackslash}l|X|}
\hline
\hspace{0pt}\mytexttt{\color{red} } & \textbf{MonitorFile} \\
\hline
\multicolumn{2}{|>{\raggedright\arraybackslash}X|}{\hspace{0pt}\mytexttt{\color{param} (varstring eventToFire, varstring ip, varstring filename, boolean subdirs=FALSE, integer4 shotCount=1, varstring espServerIpPort=GETENV('ws\_fs\_server'))}} \\
\hline
\end{tabularx}
}

\par
Same as fMonitorFile, but does not return the DFU Workunit ID.

\par
\begin{description}
\item [\colorbox{tagtype}{\color{white} \textbf{\textsf{SEE}}}] \textbf{\underline{}} fMonitorFile
\end{description}

\rule{\linewidth}{0.5pt}
\subsection*{\textsf{\colorbox{headtoc}{\color{white} FUNCTION}
WaitDfuWorkunit}}

\hypertarget{ecldoc:file.waitdfuworkunit}{}
\hspace{0pt} \hyperlink{ecldoc:File}{File} \textbackslash 

{\renewcommand{\arraystretch}{1.5}
\begin{tabularx}{\textwidth}{|>{\raggedright\arraybackslash}l|X|}
\hline
\hspace{0pt}\mytexttt{\color{red} varstring} & \textbf{WaitDfuWorkunit} \\
\hline
\multicolumn{2}{|>{\raggedright\arraybackslash}X|}{\hspace{0pt}\mytexttt{\color{param} (varstring wuid, integer4 timeOut=-1, varstring espServerIpPort=GETENV('ws\_fs\_server'))}} \\
\hline
\end{tabularx}
}

\par
Waits for the specified DFU workunit to finish.

\par
\begin{description}
\item [\colorbox{tagtype}{\color{white} \textbf{\textsf{PARAMETER}}}] \textbf{\underline{wuid}} The dfu wfid to wait for.
\item [\colorbox{tagtype}{\color{white} \textbf{\textsf{PARAMETER}}}] \textbf{\underline{timeOut}} The time in ms to wait for the operation to complete. A value of 0 causes the call to return immediately. Defaults to no timeout (-1).
\item [\colorbox{tagtype}{\color{white} \textbf{\textsf{PARAMETER}}}] \textbf{\underline{espServerIpPort}} The url of the ESP file copying service. Defaults to the value of ws\_fs\_server in the environment.
\item [\colorbox{tagtype}{\color{white} \textbf{\textsf{RETURN}}}] \textbf{\underline{}} A string containing the final status string of the DFU workunit.
\end{description}

\rule{\linewidth}{0.5pt}
\subsection*{\textsf{\colorbox{headtoc}{\color{white} FUNCTION}
AbortDfuWorkunit}}

\hypertarget{ecldoc:file.abortdfuworkunit}{}
\hspace{0pt} \hyperlink{ecldoc:File}{File} \textbackslash 

{\renewcommand{\arraystretch}{1.5}
\begin{tabularx}{\textwidth}{|>{\raggedright\arraybackslash}l|X|}
\hline
\hspace{0pt}\mytexttt{\color{red} } & \textbf{AbortDfuWorkunit} \\
\hline
\multicolumn{2}{|>{\raggedright\arraybackslash}X|}{\hspace{0pt}\mytexttt{\color{param} (varstring wuid, varstring espServerIpPort=GETENV('ws\_fs\_server'))}} \\
\hline
\end{tabularx}
}

\par
Aborts the specified DFU workunit.

\par
\begin{description}
\item [\colorbox{tagtype}{\color{white} \textbf{\textsf{PARAMETER}}}] \textbf{\underline{wuid}} The dfu wfid to abort.
\item [\colorbox{tagtype}{\color{white} \textbf{\textsf{PARAMETER}}}] \textbf{\underline{espServerIpPort}} The url of the ESP file copying service. Defaults to the value of ws\_fs\_server in the environment.
\end{description}

\rule{\linewidth}{0.5pt}
\subsection*{\textsf{\colorbox{headtoc}{\color{white} FUNCTION}
CreateSuperFile}}

\hypertarget{ecldoc:file.createsuperfile}{}
\hspace{0pt} \hyperlink{ecldoc:File}{File} \textbackslash 

{\renewcommand{\arraystretch}{1.5}
\begin{tabularx}{\textwidth}{|>{\raggedright\arraybackslash}l|X|}
\hline
\hspace{0pt}\mytexttt{\color{red} } & \textbf{CreateSuperFile} \\
\hline
\multicolumn{2}{|>{\raggedright\arraybackslash}X|}{\hspace{0pt}\mytexttt{\color{param} (varstring superName, boolean sequentialParts=FALSE, boolean allowExist=FALSE)}} \\
\hline
\end{tabularx}
}

\par
Creates an empty superfile. This function is not included in a superfile transaction.

\par
\begin{description}
\item [\colorbox{tagtype}{\color{white} \textbf{\textsf{PARAMETER}}}] \textbf{\underline{superName}} The logical name of the superfile.
\item [\colorbox{tagtype}{\color{white} \textbf{\textsf{PARAMETER}}}] \textbf{\underline{sequentialParts}} Whether the sub-files must be sequentially ordered. Default to FALSE.
\item [\colorbox{tagtype}{\color{white} \textbf{\textsf{PARAMETER}}}] \textbf{\underline{allowExist}} Indicating whether to post an error if the superfile already exists. If TRUE, no error is posted. Defaults to FALSE.
\end{description}

\rule{\linewidth}{0.5pt}
\subsection*{\textsf{\colorbox{headtoc}{\color{white} FUNCTION}
SuperFileExists}}

\hypertarget{ecldoc:file.superfileexists}{}
\hspace{0pt} \hyperlink{ecldoc:File}{File} \textbackslash 

{\renewcommand{\arraystretch}{1.5}
\begin{tabularx}{\textwidth}{|>{\raggedright\arraybackslash}l|X|}
\hline
\hspace{0pt}\mytexttt{\color{red} boolean} & \textbf{SuperFileExists} \\
\hline
\multicolumn{2}{|>{\raggedright\arraybackslash}X|}{\hspace{0pt}\mytexttt{\color{param} (varstring superName)}} \\
\hline
\end{tabularx}
}

\par
Checks if the specified filename is present in the Distributed File Utility (DFU) and is a SuperFile.

\par
\begin{description}
\item [\colorbox{tagtype}{\color{white} \textbf{\textsf{PARAMETER}}}] \textbf{\underline{superName}} The logical name of the superfile.
\item [\colorbox{tagtype}{\color{white} \textbf{\textsf{RETURN}}}] \textbf{\underline{}} Whether the file exists.
\item [\colorbox{tagtype}{\color{white} \textbf{\textsf{SEE}}}] \textbf{\underline{}} FileExists
\end{description}

\rule{\linewidth}{0.5pt}
\subsection*{\textsf{\colorbox{headtoc}{\color{white} FUNCTION}
DeleteSuperFile}}

\hypertarget{ecldoc:file.deletesuperfile}{}
\hspace{0pt} \hyperlink{ecldoc:File}{File} \textbackslash 

{\renewcommand{\arraystretch}{1.5}
\begin{tabularx}{\textwidth}{|>{\raggedright\arraybackslash}l|X|}
\hline
\hspace{0pt}\mytexttt{\color{red} } & \textbf{DeleteSuperFile} \\
\hline
\multicolumn{2}{|>{\raggedright\arraybackslash}X|}{\hspace{0pt}\mytexttt{\color{param} (varstring superName, boolean deletesub=FALSE)}} \\
\hline
\end{tabularx}
}

\par
Deletes the superfile.

\par
\begin{description}
\item [\colorbox{tagtype}{\color{white} \textbf{\textsf{PARAMETER}}}] \textbf{\underline{superName}} The logical name of the superfile.
\item [\colorbox{tagtype}{\color{white} \textbf{\textsf{SEE}}}] \textbf{\underline{}} FileExists
\end{description}

\rule{\linewidth}{0.5pt}
\subsection*{\textsf{\colorbox{headtoc}{\color{white} FUNCTION}
GetSuperFileSubCount}}

\hypertarget{ecldoc:file.getsuperfilesubcount}{}
\hspace{0pt} \hyperlink{ecldoc:File}{File} \textbackslash 

{\renewcommand{\arraystretch}{1.5}
\begin{tabularx}{\textwidth}{|>{\raggedright\arraybackslash}l|X|}
\hline
\hspace{0pt}\mytexttt{\color{red} unsigned4} & \textbf{GetSuperFileSubCount} \\
\hline
\multicolumn{2}{|>{\raggedright\arraybackslash}X|}{\hspace{0pt}\mytexttt{\color{param} (varstring superName)}} \\
\hline
\end{tabularx}
}

\par
Returns the number of sub-files contained within a superfile.

\par
\begin{description}
\item [\colorbox{tagtype}{\color{white} \textbf{\textsf{PARAMETER}}}] \textbf{\underline{superName}} The logical name of the superfile.
\item [\colorbox{tagtype}{\color{white} \textbf{\textsf{RETURN}}}] \textbf{\underline{}} The number of sub-files within the superfile.
\end{description}

\rule{\linewidth}{0.5pt}
\subsection*{\textsf{\colorbox{headtoc}{\color{white} FUNCTION}
GetSuperFileSubName}}

\hypertarget{ecldoc:file.getsuperfilesubname}{}
\hspace{0pt} \hyperlink{ecldoc:File}{File} \textbackslash 

{\renewcommand{\arraystretch}{1.5}
\begin{tabularx}{\textwidth}{|>{\raggedright\arraybackslash}l|X|}
\hline
\hspace{0pt}\mytexttt{\color{red} varstring} & \textbf{GetSuperFileSubName} \\
\hline
\multicolumn{2}{|>{\raggedright\arraybackslash}X|}{\hspace{0pt}\mytexttt{\color{param} (varstring superName, unsigned4 fileNum, boolean absPath=FALSE)}} \\
\hline
\end{tabularx}
}

\par
Returns the name of the Nth sub-file within a superfile.

\par
\begin{description}
\item [\colorbox{tagtype}{\color{white} \textbf{\textsf{PARAMETER}}}] \textbf{\underline{superName}} The logical name of the superfile.
\item [\colorbox{tagtype}{\color{white} \textbf{\textsf{PARAMETER}}}] \textbf{\underline{fileNum}} The 1-based position of the sub-file to return the name of.
\item [\colorbox{tagtype}{\color{white} \textbf{\textsf{PARAMETER}}}] \textbf{\underline{absPath}} Whether to prepend '\~{}' to the name of the resulting logical file name.
\item [\colorbox{tagtype}{\color{white} \textbf{\textsf{RETURN}}}] \textbf{\underline{}} The logical name of the selected sub-file.
\end{description}

\rule{\linewidth}{0.5pt}
\subsection*{\textsf{\colorbox{headtoc}{\color{white} FUNCTION}
FindSuperFileSubName}}

\hypertarget{ecldoc:file.findsuperfilesubname}{}
\hspace{0pt} \hyperlink{ecldoc:File}{File} \textbackslash 

{\renewcommand{\arraystretch}{1.5}
\begin{tabularx}{\textwidth}{|>{\raggedright\arraybackslash}l|X|}
\hline
\hspace{0pt}\mytexttt{\color{red} unsigned4} & \textbf{FindSuperFileSubName} \\
\hline
\multicolumn{2}{|>{\raggedright\arraybackslash}X|}{\hspace{0pt}\mytexttt{\color{param} (varstring superName, varstring subName)}} \\
\hline
\end{tabularx}
}

\par
Returns the position of a file within a superfile.

\par
\begin{description}
\item [\colorbox{tagtype}{\color{white} \textbf{\textsf{PARAMETER}}}] \textbf{\underline{superName}} The logical name of the superfile.
\item [\colorbox{tagtype}{\color{white} \textbf{\textsf{PARAMETER}}}] \textbf{\underline{subName}} The logical name of the sub-file.
\item [\colorbox{tagtype}{\color{white} \textbf{\textsf{RETURN}}}] \textbf{\underline{}} The 1-based position of the sub-file within the superfile.
\end{description}

\rule{\linewidth}{0.5pt}
\subsection*{\textsf{\colorbox{headtoc}{\color{white} FUNCTION}
StartSuperFileTransaction}}

\hypertarget{ecldoc:file.startsuperfiletransaction}{}
\hspace{0pt} \hyperlink{ecldoc:File}{File} \textbackslash 

{\renewcommand{\arraystretch}{1.5}
\begin{tabularx}{\textwidth}{|>{\raggedright\arraybackslash}l|X|}
\hline
\hspace{0pt}\mytexttt{\color{red} } & \textbf{StartSuperFileTransaction} \\
\hline
\multicolumn{2}{|>{\raggedright\arraybackslash}X|}{\hspace{0pt}\mytexttt{\color{param} ()}} \\
\hline
\end{tabularx}
}

\par
Starts a superfile transaction. All superfile operations within the transaction will either be executed atomically or rolled back when the transaction is finished.


\rule{\linewidth}{0.5pt}
\subsection*{\textsf{\colorbox{headtoc}{\color{white} FUNCTION}
AddSuperFile}}

\hypertarget{ecldoc:file.addsuperfile}{}
\hspace{0pt} \hyperlink{ecldoc:File}{File} \textbackslash 

{\renewcommand{\arraystretch}{1.5}
\begin{tabularx}{\textwidth}{|>{\raggedright\arraybackslash}l|X|}
\hline
\hspace{0pt}\mytexttt{\color{red} } & \textbf{AddSuperFile} \\
\hline
\multicolumn{2}{|>{\raggedright\arraybackslash}X|}{\hspace{0pt}\mytexttt{\color{param} (varstring superName, varstring subName, unsigned4 atPos=0, boolean addContents=FALSE, boolean strict=FALSE)}} \\
\hline
\end{tabularx}
}

\par
Adds a file to a superfile.

\par
\begin{description}
\item [\colorbox{tagtype}{\color{white} \textbf{\textsf{PARAMETER}}}] \textbf{\underline{superName}} The logical name of the superfile.
\item [\colorbox{tagtype}{\color{white} \textbf{\textsf{PARAMETER}}}] \textbf{\underline{subName}} The name of the logical file to add.
\item [\colorbox{tagtype}{\color{white} \textbf{\textsf{PARAMETER}}}] \textbf{\underline{atPos}} The position to add the sub-file, or 0 to append. Defaults to 0.
\item [\colorbox{tagtype}{\color{white} \textbf{\textsf{PARAMETER}}}] \textbf{\underline{addContents}} Controls whether adding a superfile adds the superfile, or its contents. Defaults to FALSE (do not expand).
\item [\colorbox{tagtype}{\color{white} \textbf{\textsf{PARAMETER}}}] \textbf{\underline{strict}} Check addContents only if subName is a superfile, and ensure superfiles exist.
\end{description}

\rule{\linewidth}{0.5pt}
\subsection*{\textsf{\colorbox{headtoc}{\color{white} FUNCTION}
RemoveSuperFile}}

\hypertarget{ecldoc:file.removesuperfile}{}
\hspace{0pt} \hyperlink{ecldoc:File}{File} \textbackslash 

{\renewcommand{\arraystretch}{1.5}
\begin{tabularx}{\textwidth}{|>{\raggedright\arraybackslash}l|X|}
\hline
\hspace{0pt}\mytexttt{\color{red} } & \textbf{RemoveSuperFile} \\
\hline
\multicolumn{2}{|>{\raggedright\arraybackslash}X|}{\hspace{0pt}\mytexttt{\color{param} (varstring superName, varstring subName, boolean del=FALSE, boolean removeContents=FALSE)}} \\
\hline
\end{tabularx}
}

\par
Removes a sub-file from a superfile.

\par
\begin{description}
\item [\colorbox{tagtype}{\color{white} \textbf{\textsf{PARAMETER}}}] \textbf{\underline{superName}} The logical name of the superfile.
\item [\colorbox{tagtype}{\color{white} \textbf{\textsf{PARAMETER}}}] \textbf{\underline{subName}} The name of the sub-file to remove.
\item [\colorbox{tagtype}{\color{white} \textbf{\textsf{PARAMETER}}}] \textbf{\underline{del}} Indicates whether the sub-file should also be removed from the disk. Defaults to FALSE.
\item [\colorbox{tagtype}{\color{white} \textbf{\textsf{PARAMETER}}}] \textbf{\underline{removeContents}} Controls whether the contents of a sub-file which is a superfile should be recursively removed. Defaults to FALSE.
\end{description}

\rule{\linewidth}{0.5pt}
\subsection*{\textsf{\colorbox{headtoc}{\color{white} FUNCTION}
ClearSuperFile}}

\hypertarget{ecldoc:file.clearsuperfile}{}
\hspace{0pt} \hyperlink{ecldoc:File}{File} \textbackslash 

{\renewcommand{\arraystretch}{1.5}
\begin{tabularx}{\textwidth}{|>{\raggedright\arraybackslash}l|X|}
\hline
\hspace{0pt}\mytexttt{\color{red} } & \textbf{ClearSuperFile} \\
\hline
\multicolumn{2}{|>{\raggedright\arraybackslash}X|}{\hspace{0pt}\mytexttt{\color{param} (varstring superName, boolean del=FALSE)}} \\
\hline
\end{tabularx}
}

\par
Removes all sub-files from a superfile.

\par
\begin{description}
\item [\colorbox{tagtype}{\color{white} \textbf{\textsf{PARAMETER}}}] \textbf{\underline{superName}} The logical name of the superfile.
\item [\colorbox{tagtype}{\color{white} \textbf{\textsf{PARAMETER}}}] \textbf{\underline{del}} Indicates whether the sub-files should also be removed from the disk. Defaults to FALSE.
\end{description}

\rule{\linewidth}{0.5pt}
\subsection*{\textsf{\colorbox{headtoc}{\color{white} FUNCTION}
RemoveOwnedSubFiles}}

\hypertarget{ecldoc:file.removeownedsubfiles}{}
\hspace{0pt} \hyperlink{ecldoc:File}{File} \textbackslash 

{\renewcommand{\arraystretch}{1.5}
\begin{tabularx}{\textwidth}{|>{\raggedright\arraybackslash}l|X|}
\hline
\hspace{0pt}\mytexttt{\color{red} } & \textbf{RemoveOwnedSubFiles} \\
\hline
\multicolumn{2}{|>{\raggedright\arraybackslash}X|}{\hspace{0pt}\mytexttt{\color{param} (varstring superName, boolean del=FALSE)}} \\
\hline
\end{tabularx}
}

\par
Removes all soley-owned sub-files from a superfile. If a sub-file is also contained within another superfile then it is retained.

\par
\begin{description}
\item [\colorbox{tagtype}{\color{white} \textbf{\textsf{PARAMETER}}}] \textbf{\underline{superName}} The logical name of the superfile.
\end{description}

\rule{\linewidth}{0.5pt}
\subsection*{\textsf{\colorbox{headtoc}{\color{white} FUNCTION}
DeleteOwnedSubFiles}}

\hypertarget{ecldoc:file.deleteownedsubfiles}{}
\hspace{0pt} \hyperlink{ecldoc:File}{File} \textbackslash 

{\renewcommand{\arraystretch}{1.5}
\begin{tabularx}{\textwidth}{|>{\raggedright\arraybackslash}l|X|}
\hline
\hspace{0pt}\mytexttt{\color{red} } & \textbf{DeleteOwnedSubFiles} \\
\hline
\multicolumn{2}{|>{\raggedright\arraybackslash}X|}{\hspace{0pt}\mytexttt{\color{param} (varstring superName)}} \\
\hline
\end{tabularx}
}

\par
Legacy version of RemoveOwnedSubFiles which was incorrectly named in a previous version.

\par
\begin{description}
\item [\colorbox{tagtype}{\color{white} \textbf{\textsf{SEE}}}] \textbf{\underline{}} RemoveOwnedSubFIles
\end{description}

\rule{\linewidth}{0.5pt}
\subsection*{\textsf{\colorbox{headtoc}{\color{white} FUNCTION}
SwapSuperFile}}

\hypertarget{ecldoc:file.swapsuperfile}{}
\hspace{0pt} \hyperlink{ecldoc:File}{File} \textbackslash 

{\renewcommand{\arraystretch}{1.5}
\begin{tabularx}{\textwidth}{|>{\raggedright\arraybackslash}l|X|}
\hline
\hspace{0pt}\mytexttt{\color{red} } & \textbf{SwapSuperFile} \\
\hline
\multicolumn{2}{|>{\raggedright\arraybackslash}X|}{\hspace{0pt}\mytexttt{\color{param} (varstring superName1, varstring superName2)}} \\
\hline
\end{tabularx}
}

\par
Swap the contents of two superfiles.

\par
\begin{description}
\item [\colorbox{tagtype}{\color{white} \textbf{\textsf{PARAMETER}}}] \textbf{\underline{superName1}} The logical name of the first superfile.
\item [\colorbox{tagtype}{\color{white} \textbf{\textsf{PARAMETER}}}] \textbf{\underline{superName2}} The logical name of the second superfile.
\end{description}

\rule{\linewidth}{0.5pt}
\subsection*{\textsf{\colorbox{headtoc}{\color{white} FUNCTION}
ReplaceSuperFile}}

\hypertarget{ecldoc:file.replacesuperfile}{}
\hspace{0pt} \hyperlink{ecldoc:File}{File} \textbackslash 

{\renewcommand{\arraystretch}{1.5}
\begin{tabularx}{\textwidth}{|>{\raggedright\arraybackslash}l|X|}
\hline
\hspace{0pt}\mytexttt{\color{red} } & \textbf{ReplaceSuperFile} \\
\hline
\multicolumn{2}{|>{\raggedright\arraybackslash}X|}{\hspace{0pt}\mytexttt{\color{param} (varstring superName, varstring oldSubFile, varstring newSubFile)}} \\
\hline
\end{tabularx}
}

\par
Removes a sub-file from a superfile and replaces it with another.

\par
\begin{description}
\item [\colorbox{tagtype}{\color{white} \textbf{\textsf{PARAMETER}}}] \textbf{\underline{superName}} The logical name of the superfile.
\item [\colorbox{tagtype}{\color{white} \textbf{\textsf{PARAMETER}}}] \textbf{\underline{oldSubFile}} The logical name of the sub-file to remove.
\item [\colorbox{tagtype}{\color{white} \textbf{\textsf{PARAMETER}}}] \textbf{\underline{newSubFile}} The logical name of the sub-file to replace within the superfile.
\end{description}

\rule{\linewidth}{0.5pt}
\subsection*{\textsf{\colorbox{headtoc}{\color{white} FUNCTION}
FinishSuperFileTransaction}}

\hypertarget{ecldoc:file.finishsuperfiletransaction}{}
\hspace{0pt} \hyperlink{ecldoc:File}{File} \textbackslash 

{\renewcommand{\arraystretch}{1.5}
\begin{tabularx}{\textwidth}{|>{\raggedright\arraybackslash}l|X|}
\hline
\hspace{0pt}\mytexttt{\color{red} } & \textbf{FinishSuperFileTransaction} \\
\hline
\multicolumn{2}{|>{\raggedright\arraybackslash}X|}{\hspace{0pt}\mytexttt{\color{param} (boolean rollback=FALSE)}} \\
\hline
\end{tabularx}
}

\par
Finishes a superfile transaction. This executes all the operations since the matching StartSuperFileTransaction(). If there are any errors, then all of the operations are rolled back.


\rule{\linewidth}{0.5pt}
\subsection*{\textsf{\colorbox{headtoc}{\color{white} FUNCTION}
SuperFileContents}}

\hypertarget{ecldoc:file.superfilecontents}{}
\hspace{0pt} \hyperlink{ecldoc:File}{File} \textbackslash 

{\renewcommand{\arraystretch}{1.5}
\begin{tabularx}{\textwidth}{|>{\raggedright\arraybackslash}l|X|}
\hline
\hspace{0pt}\mytexttt{\color{red} dataset(FsLogicalFileNameRecord)} & \textbf{SuperFileContents} \\
\hline
\multicolumn{2}{|>{\raggedright\arraybackslash}X|}{\hspace{0pt}\mytexttt{\color{param} (varstring superName, boolean recurse=FALSE)}} \\
\hline
\end{tabularx}
}

\par
Returns the list of sub-files contained within a superfile.

\par
\begin{description}
\item [\colorbox{tagtype}{\color{white} \textbf{\textsf{PARAMETER}}}] \textbf{\underline{superName}} The logical name of the superfile.
\item [\colorbox{tagtype}{\color{white} \textbf{\textsf{PARAMETER}}}] \textbf{\underline{recurse}} Should the contents of child-superfiles be expanded. Default is FALSE.
\item [\colorbox{tagtype}{\color{white} \textbf{\textsf{RETURN}}}] \textbf{\underline{}} A dataset containing the names of the sub-files.
\end{description}

\rule{\linewidth}{0.5pt}
\subsection*{\textsf{\colorbox{headtoc}{\color{white} FUNCTION}
LogicalFileSuperOwners}}

\hypertarget{ecldoc:file.logicalfilesuperowners}{}
\hspace{0pt} \hyperlink{ecldoc:File}{File} \textbackslash 

{\renewcommand{\arraystretch}{1.5}
\begin{tabularx}{\textwidth}{|>{\raggedright\arraybackslash}l|X|}
\hline
\hspace{0pt}\mytexttt{\color{red} dataset(FsLogicalFileNameRecord)} & \textbf{LogicalFileSuperOwners} \\
\hline
\multicolumn{2}{|>{\raggedright\arraybackslash}X|}{\hspace{0pt}\mytexttt{\color{param} (varstring name)}} \\
\hline
\end{tabularx}
}

\par
Returns the list of superfiles that a logical file is contained within.

\par
\begin{description}
\item [\colorbox{tagtype}{\color{white} \textbf{\textsf{PARAMETER}}}] \textbf{\underline{name}} The name of the logical file.
\item [\colorbox{tagtype}{\color{white} \textbf{\textsf{RETURN}}}] \textbf{\underline{}} A dataset containing the names of the superfiles.
\end{description}

\rule{\linewidth}{0.5pt}
\subsection*{\textsf{\colorbox{headtoc}{\color{white} FUNCTION}
LogicalFileSuperSubList}}

\hypertarget{ecldoc:file.logicalfilesupersublist}{}
\hspace{0pt} \hyperlink{ecldoc:File}{File} \textbackslash 

{\renewcommand{\arraystretch}{1.5}
\begin{tabularx}{\textwidth}{|>{\raggedright\arraybackslash}l|X|}
\hline
\hspace{0pt}\mytexttt{\color{red} dataset(FsLogicalSuperSubRecord)} & \textbf{LogicalFileSuperSubList} \\
\hline
\multicolumn{2}{|>{\raggedright\arraybackslash}X|}{\hspace{0pt}\mytexttt{\color{param} ()}} \\
\hline
\end{tabularx}
}

\par
Returns the list of all the superfiles in the system and their component sub-files.

\par
\begin{description}
\item [\colorbox{tagtype}{\color{white} \textbf{\textsf{RETURN}}}] \textbf{\underline{}} A dataset containing pairs of superName,subName for each component file.
\end{description}

\rule{\linewidth}{0.5pt}
\subsection*{\textsf{\colorbox{headtoc}{\color{white} FUNCTION}
fPromoteSuperFileList}}

\hypertarget{ecldoc:file.fpromotesuperfilelist}{}
\hspace{0pt} \hyperlink{ecldoc:File}{File} \textbackslash 

{\renewcommand{\arraystretch}{1.5}
\begin{tabularx}{\textwidth}{|>{\raggedright\arraybackslash}l|X|}
\hline
\hspace{0pt}\mytexttt{\color{red} varstring} & \textbf{fPromoteSuperFileList} \\
\hline
\multicolumn{2}{|>{\raggedright\arraybackslash}X|}{\hspace{0pt}\mytexttt{\color{param} (set of varstring superNames, varstring addHead='', boolean delTail=FALSE, boolean createOnlyOne=FALSE, boolean reverse=FALSE)}} \\
\hline
\end{tabularx}
}

\par
Moves the sub-files from the first entry in the list of superfiles to the next in the list, repeating the process through the list of superfiles.

\par
\begin{description}
\item [\colorbox{tagtype}{\color{white} \textbf{\textsf{PARAMETER}}}] \textbf{\underline{superNames}} A set of the names of the superfiles to act on. Any that do not exist will be created. The contents of each superfile will be moved to the next in the list.
\item [\colorbox{tagtype}{\color{white} \textbf{\textsf{PARAMETER}}}] \textbf{\underline{addHead}} A string containing a comma-delimited list of logical file names to add to the first superfile after the promotion process is complete. Defaults to ''.
\item [\colorbox{tagtype}{\color{white} \textbf{\textsf{PARAMETER}}}] \textbf{\underline{delTail}} Indicates whether to physically delete the contents moved out of the last superfile. The default is FALSE.
\item [\colorbox{tagtype}{\color{white} \textbf{\textsf{PARAMETER}}}] \textbf{\underline{createOnlyOne}} Specifies whether to only create a single superfile (truncate the list at the first non-existent superfile). The default is FALSE.
\item [\colorbox{tagtype}{\color{white} \textbf{\textsf{PARAMETER}}}] \textbf{\underline{reverse}} Reverse the order of processing the superfiles list, effectively 'demoting' instead of 'promoting' the sub-files. The default is FALSE.
\item [\colorbox{tagtype}{\color{white} \textbf{\textsf{RETURN}}}] \textbf{\underline{}} A string containing a comma separated list of the previous sub-file contents of the emptied superfile.
\end{description}

\rule{\linewidth}{0.5pt}
\subsection*{\textsf{\colorbox{headtoc}{\color{white} FUNCTION}
PromoteSuperFileList}}

\hypertarget{ecldoc:file.promotesuperfilelist}{}
\hspace{0pt} \hyperlink{ecldoc:File}{File} \textbackslash 

{\renewcommand{\arraystretch}{1.5}
\begin{tabularx}{\textwidth}{|>{\raggedright\arraybackslash}l|X|}
\hline
\hspace{0pt}\mytexttt{\color{red} } & \textbf{PromoteSuperFileList} \\
\hline
\multicolumn{2}{|>{\raggedright\arraybackslash}X|}{\hspace{0pt}\mytexttt{\color{param} (set of varstring superNames, varstring addHead='', boolean delTail=FALSE, boolean createOnlyOne=FALSE, boolean reverse=FALSE)}} \\
\hline
\end{tabularx}
}

\par
Same as fPromoteSuperFileList, but does not return the DFU Workunit ID.

\par
\begin{description}
\item [\colorbox{tagtype}{\color{white} \textbf{\textsf{SEE}}}] \textbf{\underline{}} fPromoteSuperFileList
\end{description}

\rule{\linewidth}{0.5pt}



\chapter*{math}
\hypertarget{ecldoc:toc:math}{}

\section*{\underline{IMPORTS}}

\section*{\underline{DESCRIPTIONS}}
\subsection*{MODULE : Math}
\hypertarget{ecldoc:Math}{}
\hyperlink{ecldoc:toc:root}{Up} :

{\renewcommand{\arraystretch}{1.5}
\begin{tabularx}{\textwidth}{|>{\raggedright\arraybackslash}l|X|}
\hline
\hspace{0pt} & Math \\
\hline
\end{tabularx}
}

\par


\hyperlink{ecldoc:math.infinity}{Infinity}  |
\hyperlink{ecldoc:math.nan}{NaN}  |
\hyperlink{ecldoc:math.isinfinite}{isInfinite}  |
\hyperlink{ecldoc:math.isnan}{isNaN}  |
\hyperlink{ecldoc:math.isfinite}{isFinite}  |
\hyperlink{ecldoc:math.fmod}{FMod}  |
\hyperlink{ecldoc:math.fmatch}{FMatch}  |

\rule{\linewidth}{0.5pt}

\subsection*{ATTRIBUTE : Infinity}
\hypertarget{ecldoc:math.infinity}{}
\hyperlink{ecldoc:Math}{Up} :
\hspace{0pt} \hyperlink{ecldoc:Math}{Math} \textbackslash 

{\renewcommand{\arraystretch}{1.5}
\begin{tabularx}{\textwidth}{|>{\raggedright\arraybackslash}l|X|}
\hline
\hspace{0pt}REAL8 & Infinity \\
\hline
\end{tabularx}
}

\par
Return a real ''infinity'' value.


\rule{\linewidth}{0.5pt}
\subsection*{ATTRIBUTE : NaN}
\hypertarget{ecldoc:math.nan}{}
\hyperlink{ecldoc:Math}{Up} :
\hspace{0pt} \hyperlink{ecldoc:Math}{Math} \textbackslash 

{\renewcommand{\arraystretch}{1.5}
\begin{tabularx}{\textwidth}{|>{\raggedright\arraybackslash}l|X|}
\hline
\hspace{0pt}REAL8 & NaN \\
\hline
\end{tabularx}
}

\par
Return a non-signalling NaN (Not a Number)value.


\rule{\linewidth}{0.5pt}
\subsection*{FUNCTION : isInfinite}
\hypertarget{ecldoc:math.isinfinite}{}
\hyperlink{ecldoc:Math}{Up} :
\hspace{0pt} \hyperlink{ecldoc:Math}{Math} \textbackslash 

{\renewcommand{\arraystretch}{1.5}
\begin{tabularx}{\textwidth}{|>{\raggedright\arraybackslash}l|X|}
\hline
\hspace{0pt}BOOLEAN & isInfinite \\
\hline
\multicolumn{2}{|>{\raggedright\arraybackslash}X|}{\hspace{0pt}(REAL8 val)} \\
\hline
\end{tabularx}
}

\par
Return whether a real value is infinite (positive or negative).

\par
\begin{description}
\item [\textbf{Parameter}] val ||| The value to test.
\end{description}

\rule{\linewidth}{0.5pt}
\subsection*{FUNCTION : isNaN}
\hypertarget{ecldoc:math.isnan}{}
\hyperlink{ecldoc:Math}{Up} :
\hspace{0pt} \hyperlink{ecldoc:Math}{Math} \textbackslash 

{\renewcommand{\arraystretch}{1.5}
\begin{tabularx}{\textwidth}{|>{\raggedright\arraybackslash}l|X|}
\hline
\hspace{0pt}BOOLEAN & isNaN \\
\hline
\multicolumn{2}{|>{\raggedright\arraybackslash}X|}{\hspace{0pt}(REAL8 val)} \\
\hline
\end{tabularx}
}

\par
Return whether a real value is a NaN (not a number) value.

\par
\begin{description}
\item [\textbf{Parameter}] val ||| The value to test.
\end{description}

\rule{\linewidth}{0.5pt}
\subsection*{FUNCTION : isFinite}
\hypertarget{ecldoc:math.isfinite}{}
\hyperlink{ecldoc:Math}{Up} :
\hspace{0pt} \hyperlink{ecldoc:Math}{Math} \textbackslash 

{\renewcommand{\arraystretch}{1.5}
\begin{tabularx}{\textwidth}{|>{\raggedright\arraybackslash}l|X|}
\hline
\hspace{0pt}BOOLEAN & isFinite \\
\hline
\multicolumn{2}{|>{\raggedright\arraybackslash}X|}{\hspace{0pt}(REAL8 val)} \\
\hline
\end{tabularx}
}

\par
Return whether a real value is a valid value (neither infinite not NaN).

\par
\begin{description}
\item [\textbf{Parameter}] val ||| The value to test.
\end{description}

\rule{\linewidth}{0.5pt}
\subsection*{FUNCTION : FMod}
\hypertarget{ecldoc:math.fmod}{}
\hyperlink{ecldoc:Math}{Up} :
\hspace{0pt} \hyperlink{ecldoc:Math}{Math} \textbackslash 

{\renewcommand{\arraystretch}{1.5}
\begin{tabularx}{\textwidth}{|>{\raggedright\arraybackslash}l|X|}
\hline
\hspace{0pt}REAL8 & FMod \\
\hline
\multicolumn{2}{|>{\raggedright\arraybackslash}X|}{\hspace{0pt}(REAL8 numer, REAL8 denom)} \\
\hline
\end{tabularx}
}

\par
Returns the floating-point remainder of numer/denom (rounded towards zero). If denom is zero, the result depends on the -fdivideByZero flag: 'zero' or unset: return zero. 'nan': return a non-signalling NaN value 'fail': throw an exception

\par
\begin{description}
\item [\textbf{Parameter}] numer ||| The numerator.
\item [\textbf{Parameter}] denom ||| The numerator.
\end{description}

\rule{\linewidth}{0.5pt}
\subsection*{FUNCTION : FMatch}
\hypertarget{ecldoc:math.fmatch}{}
\hyperlink{ecldoc:Math}{Up} :
\hspace{0pt} \hyperlink{ecldoc:Math}{Math} \textbackslash 

{\renewcommand{\arraystretch}{1.5}
\begin{tabularx}{\textwidth}{|>{\raggedright\arraybackslash}l|X|}
\hline
\hspace{0pt}BOOLEAN & FMatch \\
\hline
\multicolumn{2}{|>{\raggedright\arraybackslash}X|}{\hspace{0pt}(REAL8 a, REAL8 b, REAL8 epsilon=0.0)} \\
\hline
\end{tabularx}
}

\par
Returns whether two floating point values are the same, within margin of error epsilon.

\par
\begin{description}
\item [\textbf{Parameter}] a ||| The first value.
\item [\textbf{Parameter}] b ||| The second value.
\item [\textbf{Parameter}] epsilon ||| The allowable margin of error.
\end{description}

\rule{\linewidth}{0.5pt}



\chapter*{Metaphone}
\hypertarget{ecldoc:toc:Metaphone}{}

\section*{\underline{IMPORTS}}
\begin{itemize}
\item lib\_metaphone
\end{itemize}

\section*{\underline{DESCRIPTIONS}}
\subsection*{MODULE : Metaphone}
\hypertarget{ecldoc:Metaphone}{}
\hyperlink{ecldoc:toc:root}{Up} :

{\renewcommand{\arraystretch}{1.5}
\begin{tabularx}{\textwidth}{|>{\raggedright\arraybackslash}l|X|}
\hline
\hspace{0pt} & Metaphone \\
\hline
\end{tabularx}
}

\par


\hyperlink{ecldoc:metaphone.primary}{primary}  |
\hyperlink{ecldoc:metaphone.secondary}{secondary}  |
\hyperlink{ecldoc:metaphone.double}{double}  |

\rule{\linewidth}{0.5pt}

\subsection*{FUNCTION : primary}
\hypertarget{ecldoc:metaphone.primary}{}
\hyperlink{ecldoc:Metaphone}{Up} :
\hspace{0pt} \hyperlink{ecldoc:Metaphone}{Metaphone} \textbackslash 

{\renewcommand{\arraystretch}{1.5}
\begin{tabularx}{\textwidth}{|>{\raggedright\arraybackslash}l|X|}
\hline
\hspace{0pt}String & primary \\
\hline
\multicolumn{2}{|>{\raggedright\arraybackslash}X|}{\hspace{0pt}(STRING src)} \\
\hline
\end{tabularx}
}

\par
Returns the primary metaphone value

\par
\begin{description}
\item [\textbf{Parameter}] src ||| The string whose metphone is to be calculated.
\item [\textbf{See}] http://en.wikipedia.org/wiki/Metaphone\#Double\_Metaphone
\end{description}

\rule{\linewidth}{0.5pt}
\subsection*{FUNCTION : secondary}
\hypertarget{ecldoc:metaphone.secondary}{}
\hyperlink{ecldoc:Metaphone}{Up} :
\hspace{0pt} \hyperlink{ecldoc:Metaphone}{Metaphone} \textbackslash 

{\renewcommand{\arraystretch}{1.5}
\begin{tabularx}{\textwidth}{|>{\raggedright\arraybackslash}l|X|}
\hline
\hspace{0pt}String & secondary \\
\hline
\multicolumn{2}{|>{\raggedright\arraybackslash}X|}{\hspace{0pt}(STRING src)} \\
\hline
\end{tabularx}
}

\par
Returns the secondary metaphone value

\par
\begin{description}
\item [\textbf{Parameter}] src ||| The string whose metphone is to be calculated.
\item [\textbf{See}] http://en.wikipedia.org/wiki/Metaphone\#Double\_Metaphone
\end{description}

\rule{\linewidth}{0.5pt}
\subsection*{FUNCTION : double}
\hypertarget{ecldoc:metaphone.double}{}
\hyperlink{ecldoc:Metaphone}{Up} :
\hspace{0pt} \hyperlink{ecldoc:Metaphone}{Metaphone} \textbackslash 

{\renewcommand{\arraystretch}{1.5}
\begin{tabularx}{\textwidth}{|>{\raggedright\arraybackslash}l|X|}
\hline
\hspace{0pt}String & double \\
\hline
\multicolumn{2}{|>{\raggedright\arraybackslash}X|}{\hspace{0pt}(STRING src)} \\
\hline
\end{tabularx}
}

\par
Returns the double metaphone value (primary and secondary concatenated

\par
\begin{description}
\item [\textbf{Parameter}] src ||| The string whose metphone is to be calculated.
\item [\textbf{See}] http://en.wikipedia.org/wiki/Metaphone\#Double\_Metaphone
\end{description}

\rule{\linewidth}{0.5pt}



\chapter*{str}
\hypertarget{ecldoc:toc:str}{}

\section*{\underline{IMPORTS}}
\begin{itemize}
\item lib\_stringlib
\end{itemize}

\section*{\underline{DESCRIPTIONS}}
\subsection*{MODULE : Str}
\hypertarget{ecldoc:Str}{}
\hyperlink{ecldoc:toc:root}{Up} :

{\renewcommand{\arraystretch}{1.5}
\begin{tabularx}{\textwidth}{|>{\raggedright\arraybackslash}l|X|}
\hline
\hspace{0pt} & Str \\
\hline
\end{tabularx}
}

\par


\hyperlink{ecldoc:str.compareignorecase}{CompareIgnoreCase}  |
\hyperlink{ecldoc:str.equalignorecase}{EqualIgnoreCase}  |
\hyperlink{ecldoc:str.find}{Find}  |
\hyperlink{ecldoc:str.findcount}{FindCount}  |
\hyperlink{ecldoc:str.wildmatch}{WildMatch}  |
\hyperlink{ecldoc:str.contains}{Contains}  |
\hyperlink{ecldoc:str.filterout}{FilterOut}  |
\hyperlink{ecldoc:str.filter}{Filter}  |
\hyperlink{ecldoc:str.substituteincluded}{SubstituteIncluded}  |
\hyperlink{ecldoc:str.substituteexcluded}{SubstituteExcluded}  |
\hyperlink{ecldoc:str.translate}{Translate}  |
\hyperlink{ecldoc:str.tolowercase}{ToLowerCase}  |
\hyperlink{ecldoc:str.touppercase}{ToUpperCase}  |
\hyperlink{ecldoc:str.tocapitalcase}{ToCapitalCase}  |
\hyperlink{ecldoc:str.totitlecase}{ToTitleCase}  |
\hyperlink{ecldoc:str.reverse}{Reverse}  |
\hyperlink{ecldoc:str.findreplace}{FindReplace}  |
\hyperlink{ecldoc:str.extract}{Extract}  |
\hyperlink{ecldoc:str.cleanspaces}{CleanSpaces}  |
\hyperlink{ecldoc:str.startswith}{StartsWith}  |
\hyperlink{ecldoc:str.endswith}{EndsWith}  |
\hyperlink{ecldoc:str.removesuffix}{RemoveSuffix}  |
\hyperlink{ecldoc:str.extractmultiple}{ExtractMultiple}  |
\hyperlink{ecldoc:str.countwords}{CountWords}  |
\hyperlink{ecldoc:str.splitwords}{SplitWords}  |
\hyperlink{ecldoc:str.combinewords}{CombineWords}  |
\hyperlink{ecldoc:str.editdistance}{EditDistance}  |
\hyperlink{ecldoc:str.editdistancewithinradius}{EditDistanceWithinRadius}  |
\hyperlink{ecldoc:str.wordcount}{WordCount}  |
\hyperlink{ecldoc:str.getnthword}{GetNthWord}  |
\hyperlink{ecldoc:str.excludefirstword}{ExcludeFirstWord}  |
\hyperlink{ecldoc:str.excludelastword}{ExcludeLastWord}  |
\hyperlink{ecldoc:str.excludenthword}{ExcludeNthWord}  |
\hyperlink{ecldoc:str.findword}{FindWord}  |
\hyperlink{ecldoc:str.repeat}{Repeat}  |
\hyperlink{ecldoc:str.tohexpairs}{ToHexPairs}  |
\hyperlink{ecldoc:str.fromhexpairs}{FromHexPairs}  |
\hyperlink{ecldoc:str.encodebase64}{EncodeBase64}  |
\hyperlink{ecldoc:str.decodebase64}{DecodeBase64}  |

\rule{\linewidth}{0.5pt}

\subsection*{FUNCTION : CompareIgnoreCase}
\hypertarget{ecldoc:str.compareignorecase}{}
\hyperlink{ecldoc:Str}{Up} :
\hspace{0pt} \hyperlink{ecldoc:Str}{Str} \textbackslash 

{\renewcommand{\arraystretch}{1.5}
\begin{tabularx}{\textwidth}{|>{\raggedright\arraybackslash}l|X|}
\hline
\hspace{0pt}INTEGER4 & CompareIgnoreCase \\
\hline
\multicolumn{2}{|>{\raggedright\arraybackslash}X|}{\hspace{0pt}(STRING src1, STRING src2)} \\
\hline
\end{tabularx}
}

\par
Compares the two strings case insensitively. Returns a negative integer, zero, or a positive integer according to whether the first string is less than, equal to, or greater than the second.

\par
\begin{description}
\item [\textbf{Parameter}] src1 ||| The first string to be compared.
\item [\textbf{Parameter}] src2 ||| The second string to be compared.
\item [\textbf{See}] Str.EqualIgnoreCase
\end{description}

\rule{\linewidth}{0.5pt}
\subsection*{FUNCTION : EqualIgnoreCase}
\hypertarget{ecldoc:str.equalignorecase}{}
\hyperlink{ecldoc:Str}{Up} :
\hspace{0pt} \hyperlink{ecldoc:Str}{Str} \textbackslash 

{\renewcommand{\arraystretch}{1.5}
\begin{tabularx}{\textwidth}{|>{\raggedright\arraybackslash}l|X|}
\hline
\hspace{0pt}BOOLEAN & EqualIgnoreCase \\
\hline
\multicolumn{2}{|>{\raggedright\arraybackslash}X|}{\hspace{0pt}(STRING src1, STRING src2)} \\
\hline
\end{tabularx}
}

\par
Tests whether the two strings are identical ignoring differences in case.

\par
\begin{description}
\item [\textbf{Parameter}] src1 ||| The first string to be compared.
\item [\textbf{Parameter}] src2 ||| The second string to be compared.
\item [\textbf{See}] Str.CompareIgnoreCase
\end{description}

\rule{\linewidth}{0.5pt}
\subsection*{FUNCTION : Find}
\hypertarget{ecldoc:str.find}{}
\hyperlink{ecldoc:Str}{Up} :
\hspace{0pt} \hyperlink{ecldoc:Str}{Str} \textbackslash 

{\renewcommand{\arraystretch}{1.5}
\begin{tabularx}{\textwidth}{|>{\raggedright\arraybackslash}l|X|}
\hline
\hspace{0pt}UNSIGNED4 & Find \\
\hline
\multicolumn{2}{|>{\raggedright\arraybackslash}X|}{\hspace{0pt}(STRING src, STRING sought, UNSIGNED4 instance = 1)} \\
\hline
\end{tabularx}
}

\par
Returns the character position of the nth match of the search string with the first string. If no match is found the attribute returns 0. If an instance is omitted the position of the first instance is returned.

\par
\begin{description}
\item [\textbf{Parameter}] src ||| The string that is searched
\item [\textbf{Parameter}] sought ||| The string being sought.
\item [\textbf{Parameter}] instance ||| Which match instance are we interested in?
\end{description}

\rule{\linewidth}{0.5pt}
\subsection*{FUNCTION : FindCount}
\hypertarget{ecldoc:str.findcount}{}
\hyperlink{ecldoc:Str}{Up} :
\hspace{0pt} \hyperlink{ecldoc:Str}{Str} \textbackslash 

{\renewcommand{\arraystretch}{1.5}
\begin{tabularx}{\textwidth}{|>{\raggedright\arraybackslash}l|X|}
\hline
\hspace{0pt}UNSIGNED4 & FindCount \\
\hline
\multicolumn{2}{|>{\raggedright\arraybackslash}X|}{\hspace{0pt}(STRING src, STRING sought)} \\
\hline
\end{tabularx}
}

\par
Returns the number of occurences of the second string within the first string.

\par
\begin{description}
\item [\textbf{Parameter}] src ||| The string that is searched
\item [\textbf{Parameter}] sought ||| The string being sought.
\end{description}

\rule{\linewidth}{0.5pt}
\subsection*{FUNCTION : WildMatch}
\hypertarget{ecldoc:str.wildmatch}{}
\hyperlink{ecldoc:Str}{Up} :
\hspace{0pt} \hyperlink{ecldoc:Str}{Str} \textbackslash 

{\renewcommand{\arraystretch}{1.5}
\begin{tabularx}{\textwidth}{|>{\raggedright\arraybackslash}l|X|}
\hline
\hspace{0pt}BOOLEAN & WildMatch \\
\hline
\multicolumn{2}{|>{\raggedright\arraybackslash}X|}{\hspace{0pt}(STRING src, STRING \_pattern, BOOLEAN ignore\_case)} \\
\hline
\end{tabularx}
}

\par
Tests if the search string matches the pattern. The pattern can contain wildcards '?' (single character) and '*' (multiple character).

\par
\begin{description}
\item [\textbf{Parameter}] src ||| The string that is being tested.
\item [\textbf{Parameter}] pattern ||| The pattern to match against.
\item [\textbf{Parameter}] ignore\_case ||| Whether to ignore differences in case between characters
\end{description}

\rule{\linewidth}{0.5pt}
\subsection*{FUNCTION : Contains}
\hypertarget{ecldoc:str.contains}{}
\hyperlink{ecldoc:Str}{Up} :
\hspace{0pt} \hyperlink{ecldoc:Str}{Str} \textbackslash 

{\renewcommand{\arraystretch}{1.5}
\begin{tabularx}{\textwidth}{|>{\raggedright\arraybackslash}l|X|}
\hline
\hspace{0pt}BOOLEAN & Contains \\
\hline
\multicolumn{2}{|>{\raggedright\arraybackslash}X|}{\hspace{0pt}(STRING src, STRING \_pattern, BOOLEAN ignore\_case)} \\
\hline
\end{tabularx}
}

\par
Tests if the search string contains each of the characters in the pattern. If the pattern contains duplicate characters those characters will match once for each occurence in the pattern.

\par
\begin{description}
\item [\textbf{Parameter}] src ||| The string that is being tested.
\item [\textbf{Parameter}] pattern ||| The pattern to match against.
\item [\textbf{Parameter}] ignore\_case ||| Whether to ignore differences in case between characters
\end{description}

\rule{\linewidth}{0.5pt}
\subsection*{FUNCTION : FilterOut}
\hypertarget{ecldoc:str.filterout}{}
\hyperlink{ecldoc:Str}{Up} :
\hspace{0pt} \hyperlink{ecldoc:Str}{Str} \textbackslash 

{\renewcommand{\arraystretch}{1.5}
\begin{tabularx}{\textwidth}{|>{\raggedright\arraybackslash}l|X|}
\hline
\hspace{0pt}STRING & FilterOut \\
\hline
\multicolumn{2}{|>{\raggedright\arraybackslash}X|}{\hspace{0pt}(STRING src, STRING filter)} \\
\hline
\end{tabularx}
}

\par
Returns the first string with all characters within the second string removed.

\par
\begin{description}
\item [\textbf{Parameter}] src ||| The string that is being tested.
\item [\textbf{Parameter}] filter ||| The string containing the set of characters to be excluded.
\item [\textbf{See}] Str.Filter
\end{description}

\rule{\linewidth}{0.5pt}
\subsection*{FUNCTION : Filter}
\hypertarget{ecldoc:str.filter}{}
\hyperlink{ecldoc:Str}{Up} :
\hspace{0pt} \hyperlink{ecldoc:Str}{Str} \textbackslash 

{\renewcommand{\arraystretch}{1.5}
\begin{tabularx}{\textwidth}{|>{\raggedright\arraybackslash}l|X|}
\hline
\hspace{0pt}STRING & Filter \\
\hline
\multicolumn{2}{|>{\raggedright\arraybackslash}X|}{\hspace{0pt}(STRING src, STRING filter)} \\
\hline
\end{tabularx}
}

\par
Returns the first string with all characters not within the second string removed.

\par
\begin{description}
\item [\textbf{Parameter}] src ||| The string that is being tested.
\item [\textbf{Parameter}] filter ||| The string containing the set of characters to be included.
\item [\textbf{See}] Str.FilterOut
\end{description}

\rule{\linewidth}{0.5pt}
\subsection*{FUNCTION : SubstituteIncluded}
\hypertarget{ecldoc:str.substituteincluded}{}
\hyperlink{ecldoc:Str}{Up} :
\hspace{0pt} \hyperlink{ecldoc:Str}{Str} \textbackslash 

{\renewcommand{\arraystretch}{1.5}
\begin{tabularx}{\textwidth}{|>{\raggedright\arraybackslash}l|X|}
\hline
\hspace{0pt}STRING & SubstituteIncluded \\
\hline
\multicolumn{2}{|>{\raggedright\arraybackslash}X|}{\hspace{0pt}(STRING src, STRING filter, STRING1 replace\_char)} \\
\hline
\end{tabularx}
}

\par
Returns the source string with the replacement character substituted for all characters included in the filter string. MORE: Should this be a general string substitution?

\par
\begin{description}
\item [\textbf{Parameter}] src ||| The string that is being tested.
\item [\textbf{Parameter}] filter ||| The string containing the set of characters to be included.
\item [\textbf{Parameter}] replace\_char ||| The character to be substituted into the result.
\item [\textbf{See}] Std.Str.Translate, Std.Str.SubstituteExcluded
\end{description}

\rule{\linewidth}{0.5pt}
\subsection*{FUNCTION : SubstituteExcluded}
\hypertarget{ecldoc:str.substituteexcluded}{}
\hyperlink{ecldoc:Str}{Up} :
\hspace{0pt} \hyperlink{ecldoc:Str}{Str} \textbackslash 

{\renewcommand{\arraystretch}{1.5}
\begin{tabularx}{\textwidth}{|>{\raggedright\arraybackslash}l|X|}
\hline
\hspace{0pt}STRING & SubstituteExcluded \\
\hline
\multicolumn{2}{|>{\raggedright\arraybackslash}X|}{\hspace{0pt}(STRING src, STRING filter, STRING1 replace\_char)} \\
\hline
\end{tabularx}
}

\par
Returns the source string with the replacement character substituted for all characters not included in the filter string. MORE: Should this be a general string substitution?

\par
\begin{description}
\item [\textbf{Parameter}] src ||| The string that is being tested.
\item [\textbf{Parameter}] filter ||| The string containing the set of characters to be included.
\item [\textbf{Parameter}] replace\_char ||| The character to be substituted into the result.
\item [\textbf{See}] Std.Str.SubstituteIncluded
\end{description}

\rule{\linewidth}{0.5pt}
\subsection*{FUNCTION : Translate}
\hypertarget{ecldoc:str.translate}{}
\hyperlink{ecldoc:Str}{Up} :
\hspace{0pt} \hyperlink{ecldoc:Str}{Str} \textbackslash 

{\renewcommand{\arraystretch}{1.5}
\begin{tabularx}{\textwidth}{|>{\raggedright\arraybackslash}l|X|}
\hline
\hspace{0pt}STRING & Translate \\
\hline
\multicolumn{2}{|>{\raggedright\arraybackslash}X|}{\hspace{0pt}(STRING src, STRING search, STRING replacement)} \\
\hline
\end{tabularx}
}

\par
Returns the source string with the all characters that match characters in the search string replaced with the character at the corresponding position in the replacement string.

\par
\begin{description}
\item [\textbf{Parameter}] src ||| The string that is being tested.
\item [\textbf{Parameter}] search ||| The string containing the set of characters to be included.
\item [\textbf{Parameter}] replacement ||| The string containing the characters to act as replacements.
\item [\textbf{See}] Std.Str.SubstituteIncluded
\end{description}

\rule{\linewidth}{0.5pt}
\subsection*{FUNCTION : ToLowerCase}
\hypertarget{ecldoc:str.tolowercase}{}
\hyperlink{ecldoc:Str}{Up} :
\hspace{0pt} \hyperlink{ecldoc:Str}{Str} \textbackslash 

{\renewcommand{\arraystretch}{1.5}
\begin{tabularx}{\textwidth}{|>{\raggedright\arraybackslash}l|X|}
\hline
\hspace{0pt}STRING & ToLowerCase \\
\hline
\multicolumn{2}{|>{\raggedright\arraybackslash}X|}{\hspace{0pt}(STRING src)} \\
\hline
\end{tabularx}
}

\par
Returns the argument string with all upper case characters converted to lower case.

\par
\begin{description}
\item [\textbf{Parameter}] src ||| The string that is being converted.
\end{description}

\rule{\linewidth}{0.5pt}
\subsection*{FUNCTION : ToUpperCase}
\hypertarget{ecldoc:str.touppercase}{}
\hyperlink{ecldoc:Str}{Up} :
\hspace{0pt} \hyperlink{ecldoc:Str}{Str} \textbackslash 

{\renewcommand{\arraystretch}{1.5}
\begin{tabularx}{\textwidth}{|>{\raggedright\arraybackslash}l|X|}
\hline
\hspace{0pt}STRING & ToUpperCase \\
\hline
\multicolumn{2}{|>{\raggedright\arraybackslash}X|}{\hspace{0pt}(STRING src)} \\
\hline
\end{tabularx}
}

\par
Return the argument string with all lower case characters converted to upper case.

\par
\begin{description}
\item [\textbf{Parameter}] src ||| The string that is being converted.
\end{description}

\rule{\linewidth}{0.5pt}
\subsection*{FUNCTION : ToCapitalCase}
\hypertarget{ecldoc:str.tocapitalcase}{}
\hyperlink{ecldoc:Str}{Up} :
\hspace{0pt} \hyperlink{ecldoc:Str}{Str} \textbackslash 

{\renewcommand{\arraystretch}{1.5}
\begin{tabularx}{\textwidth}{|>{\raggedright\arraybackslash}l|X|}
\hline
\hspace{0pt}STRING & ToCapitalCase \\
\hline
\multicolumn{2}{|>{\raggedright\arraybackslash}X|}{\hspace{0pt}(STRING src)} \\
\hline
\end{tabularx}
}

\par
Returns the argument string with the first letter of each word in upper case and all other letters left as-is. A contiguous sequence of alphanumeric characters is treated as a word.

\par
\begin{description}
\item [\textbf{Parameter}] src ||| The string that is being converted.
\end{description}

\rule{\linewidth}{0.5pt}
\subsection*{FUNCTION : ToTitleCase}
\hypertarget{ecldoc:str.totitlecase}{}
\hyperlink{ecldoc:Str}{Up} :
\hspace{0pt} \hyperlink{ecldoc:Str}{Str} \textbackslash 

{\renewcommand{\arraystretch}{1.5}
\begin{tabularx}{\textwidth}{|>{\raggedright\arraybackslash}l|X|}
\hline
\hspace{0pt}STRING & ToTitleCase \\
\hline
\multicolumn{2}{|>{\raggedright\arraybackslash}X|}{\hspace{0pt}(STRING src)} \\
\hline
\end{tabularx}
}

\par
Returns the argument string with the first letter of each word in upper case and all other letters lower case. A contiguous sequence of alphanumeric characters is treated as a word.

\par
\begin{description}
\item [\textbf{Parameter}] src ||| The string that is being converted.
\end{description}

\rule{\linewidth}{0.5pt}
\subsection*{FUNCTION : Reverse}
\hypertarget{ecldoc:str.reverse}{}
\hyperlink{ecldoc:Str}{Up} :
\hspace{0pt} \hyperlink{ecldoc:Str}{Str} \textbackslash 

{\renewcommand{\arraystretch}{1.5}
\begin{tabularx}{\textwidth}{|>{\raggedright\arraybackslash}l|X|}
\hline
\hspace{0pt}STRING & Reverse \\
\hline
\multicolumn{2}{|>{\raggedright\arraybackslash}X|}{\hspace{0pt}(STRING src)} \\
\hline
\end{tabularx}
}

\par
Returns the argument string with all characters in reverse order. Note the argument is not TRIMMED before it is reversed.

\par
\begin{description}
\item [\textbf{Parameter}] src ||| The string that is being reversed.
\end{description}

\rule{\linewidth}{0.5pt}
\subsection*{FUNCTION : FindReplace}
\hypertarget{ecldoc:str.findreplace}{}
\hyperlink{ecldoc:Str}{Up} :
\hspace{0pt} \hyperlink{ecldoc:Str}{Str} \textbackslash 

{\renewcommand{\arraystretch}{1.5}
\begin{tabularx}{\textwidth}{|>{\raggedright\arraybackslash}l|X|}
\hline
\hspace{0pt}STRING & FindReplace \\
\hline
\multicolumn{2}{|>{\raggedright\arraybackslash}X|}{\hspace{0pt}(STRING src, STRING sought, STRING replacement)} \\
\hline
\end{tabularx}
}

\par
Returns the source string with the replacement string substituted for all instances of the search string.

\par
\begin{description}
\item [\textbf{Parameter}] src ||| The string that is being transformed.
\item [\textbf{Parameter}] sought ||| The string to be replaced.
\item [\textbf{Parameter}] replacement ||| The string to be substituted into the result.
\end{description}

\rule{\linewidth}{0.5pt}
\subsection*{FUNCTION : Extract}
\hypertarget{ecldoc:str.extract}{}
\hyperlink{ecldoc:Str}{Up} :
\hspace{0pt} \hyperlink{ecldoc:Str}{Str} \textbackslash 

{\renewcommand{\arraystretch}{1.5}
\begin{tabularx}{\textwidth}{|>{\raggedright\arraybackslash}l|X|}
\hline
\hspace{0pt}STRING & Extract \\
\hline
\multicolumn{2}{|>{\raggedright\arraybackslash}X|}{\hspace{0pt}(STRING src, UNSIGNED4 instance)} \\
\hline
\end{tabularx}
}

\par
Returns the nth element from a comma separated string.

\par
\begin{description}
\item [\textbf{Parameter}] src ||| The string containing the comma separated list.
\item [\textbf{Parameter}] instance ||| Which item to select from the list.
\end{description}

\rule{\linewidth}{0.5pt}
\subsection*{FUNCTION : CleanSpaces}
\hypertarget{ecldoc:str.cleanspaces}{}
\hyperlink{ecldoc:Str}{Up} :
\hspace{0pt} \hyperlink{ecldoc:Str}{Str} \textbackslash 

{\renewcommand{\arraystretch}{1.5}
\begin{tabularx}{\textwidth}{|>{\raggedright\arraybackslash}l|X|}
\hline
\hspace{0pt}STRING & CleanSpaces \\
\hline
\multicolumn{2}{|>{\raggedright\arraybackslash}X|}{\hspace{0pt}(STRING src)} \\
\hline
\end{tabularx}
}

\par
Returns the source string with all instances of multiple adjacent space characters (2 or more spaces together) reduced to a single space character. Leading and trailing spaces are removed, and tab characters are converted to spaces.

\par
\begin{description}
\item [\textbf{Parameter}] src ||| The string to be cleaned.
\end{description}

\rule{\linewidth}{0.5pt}
\subsection*{FUNCTION : StartsWith}
\hypertarget{ecldoc:str.startswith}{}
\hyperlink{ecldoc:Str}{Up} :
\hspace{0pt} \hyperlink{ecldoc:Str}{Str} \textbackslash 

{\renewcommand{\arraystretch}{1.5}
\begin{tabularx}{\textwidth}{|>{\raggedright\arraybackslash}l|X|}
\hline
\hspace{0pt}BOOLEAN & StartsWith \\
\hline
\multicolumn{2}{|>{\raggedright\arraybackslash}X|}{\hspace{0pt}(STRING src, STRING prefix)} \\
\hline
\end{tabularx}
}

\par
Returns true if the prefix string matches the leading characters in the source string. Trailing spaces are stripped from the prefix before matching. // x.myString.StartsWith('x') as an alternative syntax would be even better

\par
\begin{description}
\item [\textbf{Parameter}] src ||| The string being searched in.
\item [\textbf{Parameter}] prefix ||| The prefix to search for.
\end{description}

\rule{\linewidth}{0.5pt}
\subsection*{FUNCTION : EndsWith}
\hypertarget{ecldoc:str.endswith}{}
\hyperlink{ecldoc:Str}{Up} :
\hspace{0pt} \hyperlink{ecldoc:Str}{Str} \textbackslash 

{\renewcommand{\arraystretch}{1.5}
\begin{tabularx}{\textwidth}{|>{\raggedright\arraybackslash}l|X|}
\hline
\hspace{0pt}BOOLEAN & EndsWith \\
\hline
\multicolumn{2}{|>{\raggedright\arraybackslash}X|}{\hspace{0pt}(STRING src, STRING suffix)} \\
\hline
\end{tabularx}
}

\par
Returns true if the suffix string matches the trailing characters in the source string. Trailing spaces are stripped from both strings before matching.

\par
\begin{description}
\item [\textbf{Parameter}] src ||| The string being searched in.
\item [\textbf{Parameter}] suffix ||| The prefix to search for.
\end{description}

\rule{\linewidth}{0.5pt}
\subsection*{FUNCTION : RemoveSuffix}
\hypertarget{ecldoc:str.removesuffix}{}
\hyperlink{ecldoc:Str}{Up} :
\hspace{0pt} \hyperlink{ecldoc:Str}{Str} \textbackslash 

{\renewcommand{\arraystretch}{1.5}
\begin{tabularx}{\textwidth}{|>{\raggedright\arraybackslash}l|X|}
\hline
\hspace{0pt}STRING & RemoveSuffix \\
\hline
\multicolumn{2}{|>{\raggedright\arraybackslash}X|}{\hspace{0pt}(STRING src, STRING suffix)} \\
\hline
\end{tabularx}
}

\par
Removes the suffix from the search string, if present, and returns the result. Trailing spaces are stripped from both strings before matching.

\par
\begin{description}
\item [\textbf{Parameter}] src ||| The string being searched in.
\item [\textbf{Parameter}] suffix ||| The prefix to search for.
\end{description}

\rule{\linewidth}{0.5pt}
\subsection*{FUNCTION : ExtractMultiple}
\hypertarget{ecldoc:str.extractmultiple}{}
\hyperlink{ecldoc:Str}{Up} :
\hspace{0pt} \hyperlink{ecldoc:Str}{Str} \textbackslash 

{\renewcommand{\arraystretch}{1.5}
\begin{tabularx}{\textwidth}{|>{\raggedright\arraybackslash}l|X|}
\hline
\hspace{0pt}STRING & ExtractMultiple \\
\hline
\multicolumn{2}{|>{\raggedright\arraybackslash}X|}{\hspace{0pt}(STRING src, UNSIGNED8 mask)} \\
\hline
\end{tabularx}
}

\par
Returns a string containing a list of elements from a comma separated string.

\par
\begin{description}
\item [\textbf{Parameter}] src ||| The string containing the comma separated list.
\item [\textbf{Parameter}] mask ||| A bitmask of which elements should be included. Bit 0 is item1, bit1 item 2 etc.
\end{description}

\rule{\linewidth}{0.5pt}
\subsection*{FUNCTION : CountWords}
\hypertarget{ecldoc:str.countwords}{}
\hyperlink{ecldoc:Str}{Up} :
\hspace{0pt} \hyperlink{ecldoc:Str}{Str} \textbackslash 

{\renewcommand{\arraystretch}{1.5}
\begin{tabularx}{\textwidth}{|>{\raggedright\arraybackslash}l|X|}
\hline
\hspace{0pt}UNSIGNED4 & CountWords \\
\hline
\multicolumn{2}{|>{\raggedright\arraybackslash}X|}{\hspace{0pt}(STRING src, STRING separator, BOOLEAN allow\_blank = FALSE)} \\
\hline
\end{tabularx}
}

\par
Returns the number of words that the string contains. Words are separated by one or more separator strings. No spaces are stripped from either string before matching.

\par
\begin{description}
\item [\textbf{Parameter}] src ||| The string being searched in.
\item [\textbf{Parameter}] separator ||| The string used to separate words
\item [\textbf{Parameter}] allow\_blank ||| Indicates if empty/blank string items are included in the results.
\end{description}

\rule{\linewidth}{0.5pt}
\subsection*{FUNCTION : SplitWords}
\hypertarget{ecldoc:str.splitwords}{}
\hyperlink{ecldoc:Str}{Up} :
\hspace{0pt} \hyperlink{ecldoc:Str}{Str} \textbackslash 

{\renewcommand{\arraystretch}{1.5}
\begin{tabularx}{\textwidth}{|>{\raggedright\arraybackslash}l|X|}
\hline
\hspace{0pt}SET OF STRING & SplitWords \\
\hline
\multicolumn{2}{|>{\raggedright\arraybackslash}X|}{\hspace{0pt}(STRING src, STRING separator, BOOLEAN allow\_blank = FALSE)} \\
\hline
\end{tabularx}
}

\par
Returns the list of words extracted from the string. Words are separated by one or more separator strings. No spaces are stripped from either string before matching.

\par
\begin{description}
\item [\textbf{Parameter}] src ||| The string being searched in.
\item [\textbf{Parameter}] separator ||| The string used to separate words
\item [\textbf{Parameter}] allow\_blank ||| Indicates if empty/blank string items are included in the results.
\end{description}

\rule{\linewidth}{0.5pt}
\subsection*{FUNCTION : CombineWords}
\hypertarget{ecldoc:str.combinewords}{}
\hyperlink{ecldoc:Str}{Up} :
\hspace{0pt} \hyperlink{ecldoc:Str}{Str} \textbackslash 

{\renewcommand{\arraystretch}{1.5}
\begin{tabularx}{\textwidth}{|>{\raggedright\arraybackslash}l|X|}
\hline
\hspace{0pt}STRING & CombineWords \\
\hline
\multicolumn{2}{|>{\raggedright\arraybackslash}X|}{\hspace{0pt}(SET OF STRING words, STRING separator)} \\
\hline
\end{tabularx}
}

\par
Returns the list of words extracted from the string. Words are separated by one or more separator strings. No spaces are stripped from either string before matching.

\par
\begin{description}
\item [\textbf{Parameter}] words ||| The set of strings to be combined.
\item [\textbf{Parameter}] separator ||| The string used to separate words.
\end{description}

\rule{\linewidth}{0.5pt}
\subsection*{FUNCTION : EditDistance}
\hypertarget{ecldoc:str.editdistance}{}
\hyperlink{ecldoc:Str}{Up} :
\hspace{0pt} \hyperlink{ecldoc:Str}{Str} \textbackslash 

{\renewcommand{\arraystretch}{1.5}
\begin{tabularx}{\textwidth}{|>{\raggedright\arraybackslash}l|X|}
\hline
\hspace{0pt}UNSIGNED4 & EditDistance \\
\hline
\multicolumn{2}{|>{\raggedright\arraybackslash}X|}{\hspace{0pt}(STRING \_left, STRING \_right)} \\
\hline
\end{tabularx}
}

\par
Returns the minimum edit distance between the two strings. An insert change or delete counts as a single edit. The two strings are trimmed before comparing.

\par
\begin{description}
\item [\textbf{Parameter}] \_left ||| The first string to be compared.
\item [\textbf{Parameter}] \_right ||| The second string to be compared.
\item [\textbf{Return}] The minimum edit distance between the two strings.
\end{description}

\rule{\linewidth}{0.5pt}
\subsection*{FUNCTION : EditDistanceWithinRadius}
\hypertarget{ecldoc:str.editdistancewithinradius}{}
\hyperlink{ecldoc:Str}{Up} :
\hspace{0pt} \hyperlink{ecldoc:Str}{Str} \textbackslash 

{\renewcommand{\arraystretch}{1.5}
\begin{tabularx}{\textwidth}{|>{\raggedright\arraybackslash}l|X|}
\hline
\hspace{0pt}BOOLEAN & EditDistanceWithinRadius \\
\hline
\multicolumn{2}{|>{\raggedright\arraybackslash}X|}{\hspace{0pt}(STRING \_left, STRING \_right, UNSIGNED4 radius)} \\
\hline
\end{tabularx}
}

\par
Returns true if the minimum edit distance between the two strings is with a specific range. The two strings are trimmed before comparing.

\par
\begin{description}
\item [\textbf{Parameter}] \_left ||| The first string to be compared.
\item [\textbf{Parameter}] \_right ||| The second string to be compared.
\item [\textbf{Parameter}] radius ||| The maximum edit distance that is accepable.
\item [\textbf{Return}] Whether or not the two strings are within the given specified edit distance.
\end{description}

\rule{\linewidth}{0.5pt}
\subsection*{FUNCTION : WordCount}
\hypertarget{ecldoc:str.wordcount}{}
\hyperlink{ecldoc:Str}{Up} :
\hspace{0pt} \hyperlink{ecldoc:Str}{Str} \textbackslash 

{\renewcommand{\arraystretch}{1.5}
\begin{tabularx}{\textwidth}{|>{\raggedright\arraybackslash}l|X|}
\hline
\hspace{0pt}UNSIGNED4 & WordCount \\
\hline
\multicolumn{2}{|>{\raggedright\arraybackslash}X|}{\hspace{0pt}(STRING text)} \\
\hline
\end{tabularx}
}

\par
Returns the number of words in the string. Words are separated by one or more spaces.

\par
\begin{description}
\item [\textbf{Parameter}] text ||| The string to be broken into words.
\item [\textbf{Return}] The number of words in the string.
\end{description}

\rule{\linewidth}{0.5pt}
\subsection*{FUNCTION : GetNthWord}
\hypertarget{ecldoc:str.getnthword}{}
\hyperlink{ecldoc:Str}{Up} :
\hspace{0pt} \hyperlink{ecldoc:Str}{Str} \textbackslash 

{\renewcommand{\arraystretch}{1.5}
\begin{tabularx}{\textwidth}{|>{\raggedright\arraybackslash}l|X|}
\hline
\hspace{0pt}STRING & GetNthWord \\
\hline
\multicolumn{2}{|>{\raggedright\arraybackslash}X|}{\hspace{0pt}(STRING text, UNSIGNED4 n)} \\
\hline
\end{tabularx}
}

\par
Returns the n-th word from the string. Words are separated by one or more spaces.

\par
\begin{description}
\item [\textbf{Parameter}] text ||| The string to be broken into words.
\item [\textbf{Parameter}] n ||| Which word should be returned from the function.
\item [\textbf{Return}] The number of words in the string.
\end{description}

\rule{\linewidth}{0.5pt}
\subsection*{FUNCTION : ExcludeFirstWord}
\hypertarget{ecldoc:str.excludefirstword}{}
\hyperlink{ecldoc:Str}{Up} :
\hspace{0pt} \hyperlink{ecldoc:Str}{Str} \textbackslash 

{\renewcommand{\arraystretch}{1.5}
\begin{tabularx}{\textwidth}{|>{\raggedright\arraybackslash}l|X|}
\hline
\hspace{0pt} & ExcludeFirstWord \\
\hline
\multicolumn{2}{|>{\raggedright\arraybackslash}X|}{\hspace{0pt}(STRING text)} \\
\hline
\end{tabularx}
}

\par
Returns everything except the first word from the string. Words are separated by one or more whitespace characters. Whitespace before and after the first word is also removed.

\par
\begin{description}
\item [\textbf{Parameter}] text ||| The string to be broken into words.
\item [\textbf{Return}] The string excluding the first word.
\end{description}

\rule{\linewidth}{0.5pt}
\subsection*{FUNCTION : ExcludeLastWord}
\hypertarget{ecldoc:str.excludelastword}{}
\hyperlink{ecldoc:Str}{Up} :
\hspace{0pt} \hyperlink{ecldoc:Str}{Str} \textbackslash 

{\renewcommand{\arraystretch}{1.5}
\begin{tabularx}{\textwidth}{|>{\raggedright\arraybackslash}l|X|}
\hline
\hspace{0pt} & ExcludeLastWord \\
\hline
\multicolumn{2}{|>{\raggedright\arraybackslash}X|}{\hspace{0pt}(STRING text)} \\
\hline
\end{tabularx}
}

\par
Returns everything except the last word from the string. Words are separated by one or more whitespace characters. Whitespace after a word is removed with the word and leading whitespace is removed with the first word.

\par
\begin{description}
\item [\textbf{Parameter}] text ||| The string to be broken into words.
\item [\textbf{Return}] The string excluding the last word.
\end{description}

\rule{\linewidth}{0.5pt}
\subsection*{FUNCTION : ExcludeNthWord}
\hypertarget{ecldoc:str.excludenthword}{}
\hyperlink{ecldoc:Str}{Up} :
\hspace{0pt} \hyperlink{ecldoc:Str}{Str} \textbackslash 

{\renewcommand{\arraystretch}{1.5}
\begin{tabularx}{\textwidth}{|>{\raggedright\arraybackslash}l|X|}
\hline
\hspace{0pt} & ExcludeNthWord \\
\hline
\multicolumn{2}{|>{\raggedright\arraybackslash}X|}{\hspace{0pt}(STRING text, UNSIGNED2 n)} \\
\hline
\end{tabularx}
}

\par
Returns everything except the nth word from the string. Words are separated by one or more whitespace characters. Whitespace after a word is removed with the word and leading whitespace is removed with the first word.

\par
\begin{description}
\item [\textbf{Parameter}] text ||| The string to be broken into words.
\item [\textbf{Parameter}] n ||| Which word should be returned from the function.
\item [\textbf{Return}] The string excluding the nth word.
\end{description}

\rule{\linewidth}{0.5pt}
\subsection*{FUNCTION : FindWord}
\hypertarget{ecldoc:str.findword}{}
\hyperlink{ecldoc:Str}{Up} :
\hspace{0pt} \hyperlink{ecldoc:Str}{Str} \textbackslash 

{\renewcommand{\arraystretch}{1.5}
\begin{tabularx}{\textwidth}{|>{\raggedright\arraybackslash}l|X|}
\hline
\hspace{0pt}BOOLEAN & FindWord \\
\hline
\multicolumn{2}{|>{\raggedright\arraybackslash}X|}{\hspace{0pt}(STRING src, STRING word, BOOLEAN ignore\_case=FALSE)} \\
\hline
\end{tabularx}
}

\par
Tests if the search string contains the supplied word as a whole word.

\par
\begin{description}
\item [\textbf{Parameter}] src ||| The string that is being tested.
\item [\textbf{Parameter}] word ||| The word to be searched for.
\item [\textbf{Parameter}] ignore\_case ||| Whether to ignore differences in case between characters.
\end{description}

\rule{\linewidth}{0.5pt}
\subsection*{FUNCTION : Repeat}
\hypertarget{ecldoc:str.repeat}{}
\hyperlink{ecldoc:Str}{Up} :
\hspace{0pt} \hyperlink{ecldoc:Str}{Str} \textbackslash 

{\renewcommand{\arraystretch}{1.5}
\begin{tabularx}{\textwidth}{|>{\raggedright\arraybackslash}l|X|}
\hline
\hspace{0pt}STRING & Repeat \\
\hline
\multicolumn{2}{|>{\raggedright\arraybackslash}X|}{\hspace{0pt}(STRING text, UNSIGNED4 n)} \\
\hline
\end{tabularx}
}

\par


\rule{\linewidth}{0.5pt}
\subsection*{FUNCTION : ToHexPairs}
\hypertarget{ecldoc:str.tohexpairs}{}
\hyperlink{ecldoc:Str}{Up} :
\hspace{0pt} \hyperlink{ecldoc:Str}{Str} \textbackslash 

{\renewcommand{\arraystretch}{1.5}
\begin{tabularx}{\textwidth}{|>{\raggedright\arraybackslash}l|X|}
\hline
\hspace{0pt}STRING & ToHexPairs \\
\hline
\multicolumn{2}{|>{\raggedright\arraybackslash}X|}{\hspace{0pt}(DATA value)} \\
\hline
\end{tabularx}
}

\par


\rule{\linewidth}{0.5pt}
\subsection*{FUNCTION : FromHexPairs}
\hypertarget{ecldoc:str.fromhexpairs}{}
\hyperlink{ecldoc:Str}{Up} :
\hspace{0pt} \hyperlink{ecldoc:Str}{Str} \textbackslash 

{\renewcommand{\arraystretch}{1.5}
\begin{tabularx}{\textwidth}{|>{\raggedright\arraybackslash}l|X|}
\hline
\hspace{0pt}DATA & FromHexPairs \\
\hline
\multicolumn{2}{|>{\raggedright\arraybackslash}X|}{\hspace{0pt}(STRING hex\_pairs)} \\
\hline
\end{tabularx}
}

\par


\rule{\linewidth}{0.5pt}
\subsection*{FUNCTION : EncodeBase64}
\hypertarget{ecldoc:str.encodebase64}{}
\hyperlink{ecldoc:Str}{Up} :
\hspace{0pt} \hyperlink{ecldoc:Str}{Str} \textbackslash 

{\renewcommand{\arraystretch}{1.5}
\begin{tabularx}{\textwidth}{|>{\raggedright\arraybackslash}l|X|}
\hline
\hspace{0pt}STRING & EncodeBase64 \\
\hline
\multicolumn{2}{|>{\raggedright\arraybackslash}X|}{\hspace{0pt}(DATA value)} \\
\hline
\end{tabularx}
}

\par


\rule{\linewidth}{0.5pt}
\subsection*{FUNCTION : DecodeBase64}
\hypertarget{ecldoc:str.decodebase64}{}
\hyperlink{ecldoc:Str}{Up} :
\hspace{0pt} \hyperlink{ecldoc:Str}{Str} \textbackslash 

{\renewcommand{\arraystretch}{1.5}
\begin{tabularx}{\textwidth}{|>{\raggedright\arraybackslash}l|X|}
\hline
\hspace{0pt}DATA & DecodeBase64 \\
\hline
\multicolumn{2}{|>{\raggedright\arraybackslash}X|}{\hspace{0pt}(STRING value)} \\
\hline
\end{tabularx}
}

\par


\rule{\linewidth}{0.5pt}



\chapter*{\color{headfile}
Uni
}
\hypertarget{ecldoc:toc:Uni}{}
\hyperlink{ecldoc:toc:root}{Go Up}

\section*{\underline{\textsf{IMPORTS}}}
\begin{doublespace}
{\large
lib\_unicodelib |
}
\end{doublespace}

\section*{\underline{\textsf{DESCRIPTIONS}}}
\subsection*{\textsf{\colorbox{headtoc}{\color{white} MODULE}
Uni}}

\hypertarget{ecldoc:Uni}{}

{\renewcommand{\arraystretch}{1.5}
\begin{tabularx}{\textwidth}{|>{\raggedright\arraybackslash}l|X|}
\hline
\hspace{0pt}\mytexttt{\color{red} } & \textbf{Uni} \\
\hline
\end{tabularx}
}

\par





No Documentation Found







\textbf{Children}
\begin{enumerate}
\item \hyperlink{ecldoc:uni.filterout}{FilterOut}
: Returns the first string with all characters within the second string removed
\item \hyperlink{ecldoc:uni.filter}{Filter}
: Returns the first string with all characters not within the second string removed
\item \hyperlink{ecldoc:uni.substituteincluded}{SubstituteIncluded}
: Returns the source string with the replacement character substituted for all characters included in the filter string
\item \hyperlink{ecldoc:uni.substituteexcluded}{SubstituteExcluded}
: Returns the source string with the replacement character substituted for all characters not included in the filter string
\item \hyperlink{ecldoc:uni.find}{Find}
: Returns the character position of the nth match of the search string with the first string
\item \hyperlink{ecldoc:uni.findword}{FindWord}
: Tests if the search string contains the supplied word as a whole word
\item \hyperlink{ecldoc:uni.localefind}{LocaleFind}
: Returns the character position of the nth match of the search string with the first string
\item \hyperlink{ecldoc:uni.localefindatstrength}{LocaleFindAtStrength}
: Returns the character position of the nth match of the search string with the first string
\item \hyperlink{ecldoc:uni.extract}{Extract}
: Returns the nth element from a comma separated string
\item \hyperlink{ecldoc:uni.tolowercase}{ToLowerCase}
: Returns the argument string with all upper case characters converted to lower case
\item \hyperlink{ecldoc:uni.touppercase}{ToUpperCase}
: Return the argument string with all lower case characters converted to upper case
\item \hyperlink{ecldoc:uni.totitlecase}{ToTitleCase}
: Returns the upper case variant of the string using the rules for a particular locale
\item \hyperlink{ecldoc:uni.localetolowercase}{LocaleToLowerCase}
: Returns the lower case variant of the string using the rules for a particular locale
\item \hyperlink{ecldoc:uni.localetouppercase}{LocaleToUpperCase}
: Returns the upper case variant of the string using the rules for a particular locale
\item \hyperlink{ecldoc:uni.localetotitlecase}{LocaleToTitleCase}
: Returns the upper case variant of the string using the rules for a particular locale
\item \hyperlink{ecldoc:uni.compareignorecase}{CompareIgnoreCase}
: Compares the two strings case insensitively
\item \hyperlink{ecldoc:uni.compareatstrength}{CompareAtStrength}
: Compares the two strings case insensitively
\item \hyperlink{ecldoc:uni.localecompareignorecase}{LocaleCompareIgnoreCase}
: Compares the two strings case insensitively
\item \hyperlink{ecldoc:uni.localecompareatstrength}{LocaleCompareAtStrength}
: Compares the two strings case insensitively
\item \hyperlink{ecldoc:uni.reverse}{Reverse}
: Returns the argument string with all characters in reverse order
\item \hyperlink{ecldoc:uni.findreplace}{FindReplace}
: Returns the source string with the replacement string substituted for all instances of the search string
\item \hyperlink{ecldoc:uni.localefindreplace}{LocaleFindReplace}
: Returns the source string with the replacement string substituted for all instances of the search string
\item \hyperlink{ecldoc:uni.localefindatstrengthreplace}{LocaleFindAtStrengthReplace}
: Returns the source string with the replacement string substituted for all instances of the search string
\item \hyperlink{ecldoc:uni.cleanaccents}{CleanAccents}
: Returns the source string with all accented characters replaced with unaccented
\item \hyperlink{ecldoc:uni.cleanspaces}{CleanSpaces}
: Returns the source string with all instances of multiple adjacent space characters (2 or more spaces together) reduced to a single space character
\item \hyperlink{ecldoc:uni.wildmatch}{WildMatch}
: Tests if the search string matches the pattern
\item \hyperlink{ecldoc:uni.contains}{Contains}
: Tests if the search string contains each of the characters in the pattern
\item \hyperlink{ecldoc:uni.editdistance}{EditDistance}
: Returns the minimum edit distance between the two strings
\item \hyperlink{ecldoc:uni.editdistancewithinradius}{EditDistanceWithinRadius}
: Returns true if the minimum edit distance between the two strings is with a specific range
\item \hyperlink{ecldoc:uni.wordcount}{WordCount}
: Returns the number of words in the string
\item \hyperlink{ecldoc:uni.getnthword}{GetNthWord}
: Returns the n-th word from the string
\end{enumerate}

\rule{\linewidth}{0.5pt}

\subsection*{\textsf{\colorbox{headtoc}{\color{white} FUNCTION}
FilterOut}}

\hypertarget{ecldoc:uni.filterout}{}
\hspace{0pt} \hyperlink{ecldoc:Uni}{Uni} \textbackslash 

{\renewcommand{\arraystretch}{1.5}
\begin{tabularx}{\textwidth}{|>{\raggedright\arraybackslash}l|X|}
\hline
\hspace{0pt}\mytexttt{\color{red} unicode} & \textbf{FilterOut} \\
\hline
\multicolumn{2}{|>{\raggedright\arraybackslash}X|}{\hspace{0pt}\mytexttt{\color{param} (unicode src, unicode filter)}} \\
\hline
\end{tabularx}
}

\par





Returns the first string with all characters within the second string removed.






\par
\begin{description}
\item [\colorbox{tagtype}{\color{white} \textbf{\textsf{PARAMETER}}}] \textbf{\underline{src}} ||| UNICODE --- The string that is being tested.
\item [\colorbox{tagtype}{\color{white} \textbf{\textsf{PARAMETER}}}] \textbf{\underline{filter}} ||| UNICODE --- The string containing the set of characters to be excluded.
\end{description}







\par
\begin{description}
\item [\colorbox{tagtype}{\color{white} \textbf{\textsf{RETURN}}}] \textbf{UNICODE} --- 
\end{description}






\par
\begin{description}
\item [\colorbox{tagtype}{\color{white} \textbf{\textsf{SEE}}}] Std.Uni.Filter
\end{description}




\rule{\linewidth}{0.5pt}
\subsection*{\textsf{\colorbox{headtoc}{\color{white} FUNCTION}
Filter}}

\hypertarget{ecldoc:uni.filter}{}
\hspace{0pt} \hyperlink{ecldoc:Uni}{Uni} \textbackslash 

{\renewcommand{\arraystretch}{1.5}
\begin{tabularx}{\textwidth}{|>{\raggedright\arraybackslash}l|X|}
\hline
\hspace{0pt}\mytexttt{\color{red} unicode} & \textbf{Filter} \\
\hline
\multicolumn{2}{|>{\raggedright\arraybackslash}X|}{\hspace{0pt}\mytexttt{\color{param} (unicode src, unicode filter)}} \\
\hline
\end{tabularx}
}

\par





Returns the first string with all characters not within the second string removed.






\par
\begin{description}
\item [\colorbox{tagtype}{\color{white} \textbf{\textsf{PARAMETER}}}] \textbf{\underline{src}} ||| UNICODE --- The string that is being tested.
\item [\colorbox{tagtype}{\color{white} \textbf{\textsf{PARAMETER}}}] \textbf{\underline{filter}} ||| UNICODE --- The string containing the set of characters to be included.
\end{description}







\par
\begin{description}
\item [\colorbox{tagtype}{\color{white} \textbf{\textsf{RETURN}}}] \textbf{UNICODE} --- 
\end{description}






\par
\begin{description}
\item [\colorbox{tagtype}{\color{white} \textbf{\textsf{SEE}}}] Std.Uni.FilterOut
\end{description}




\rule{\linewidth}{0.5pt}
\subsection*{\textsf{\colorbox{headtoc}{\color{white} FUNCTION}
SubstituteIncluded}}

\hypertarget{ecldoc:uni.substituteincluded}{}
\hspace{0pt} \hyperlink{ecldoc:Uni}{Uni} \textbackslash 

{\renewcommand{\arraystretch}{1.5}
\begin{tabularx}{\textwidth}{|>{\raggedright\arraybackslash}l|X|}
\hline
\hspace{0pt}\mytexttt{\color{red} unicode} & \textbf{SubstituteIncluded} \\
\hline
\multicolumn{2}{|>{\raggedright\arraybackslash}X|}{\hspace{0pt}\mytexttt{\color{param} (unicode src, unicode filter, unicode replace\_char)}} \\
\hline
\end{tabularx}
}

\par





Returns the source string with the replacement character substituted for all characters included in the filter string. MORE: Should this be a general string substitution?






\par
\begin{description}
\item [\colorbox{tagtype}{\color{white} \textbf{\textsf{PARAMETER}}}] \textbf{\underline{replace\_char}} ||| UNICODE --- The character to be substituted into the result.
\item [\colorbox{tagtype}{\color{white} \textbf{\textsf{PARAMETER}}}] \textbf{\underline{src}} ||| UNICODE --- The string that is being tested.
\item [\colorbox{tagtype}{\color{white} \textbf{\textsf{PARAMETER}}}] \textbf{\underline{filter}} ||| UNICODE --- The string containing the set of characters to be included.
\end{description}







\par
\begin{description}
\item [\colorbox{tagtype}{\color{white} \textbf{\textsf{RETURN}}}] \textbf{UNICODE} --- 
\end{description}






\par
\begin{description}
\item [\colorbox{tagtype}{\color{white} \textbf{\textsf{SEE}}}] Std.Uni.SubstituteOut
\end{description}




\rule{\linewidth}{0.5pt}
\subsection*{\textsf{\colorbox{headtoc}{\color{white} FUNCTION}
SubstituteExcluded}}

\hypertarget{ecldoc:uni.substituteexcluded}{}
\hspace{0pt} \hyperlink{ecldoc:Uni}{Uni} \textbackslash 

{\renewcommand{\arraystretch}{1.5}
\begin{tabularx}{\textwidth}{|>{\raggedright\arraybackslash}l|X|}
\hline
\hspace{0pt}\mytexttt{\color{red} unicode} & \textbf{SubstituteExcluded} \\
\hline
\multicolumn{2}{|>{\raggedright\arraybackslash}X|}{\hspace{0pt}\mytexttt{\color{param} (unicode src, unicode filter, unicode replace\_char)}} \\
\hline
\end{tabularx}
}

\par





Returns the source string with the replacement character substituted for all characters not included in the filter string. MORE: Should this be a general string substitution?






\par
\begin{description}
\item [\colorbox{tagtype}{\color{white} \textbf{\textsf{PARAMETER}}}] \textbf{\underline{replace\_char}} ||| UNICODE --- The character to be substituted into the result.
\item [\colorbox{tagtype}{\color{white} \textbf{\textsf{PARAMETER}}}] \textbf{\underline{src}} ||| UNICODE --- The string that is being tested.
\item [\colorbox{tagtype}{\color{white} \textbf{\textsf{PARAMETER}}}] \textbf{\underline{filter}} ||| UNICODE --- The string containing the set of characters to be included.
\end{description}







\par
\begin{description}
\item [\colorbox{tagtype}{\color{white} \textbf{\textsf{RETURN}}}] \textbf{UNICODE} --- 
\end{description}






\par
\begin{description}
\item [\colorbox{tagtype}{\color{white} \textbf{\textsf{SEE}}}] Std.Uni.SubstituteIncluded
\end{description}




\rule{\linewidth}{0.5pt}
\subsection*{\textsf{\colorbox{headtoc}{\color{white} FUNCTION}
Find}}

\hypertarget{ecldoc:uni.find}{}
\hspace{0pt} \hyperlink{ecldoc:Uni}{Uni} \textbackslash 

{\renewcommand{\arraystretch}{1.5}
\begin{tabularx}{\textwidth}{|>{\raggedright\arraybackslash}l|X|}
\hline
\hspace{0pt}\mytexttt{\color{red} UNSIGNED4} & \textbf{Find} \\
\hline
\multicolumn{2}{|>{\raggedright\arraybackslash}X|}{\hspace{0pt}\mytexttt{\color{param} (unicode src, unicode sought, unsigned4 instance)}} \\
\hline
\end{tabularx}
}

\par





Returns the character position of the nth match of the search string with the first string. If no match is found the attribute returns 0. If an instance is omitted the position of the first instance is returned.






\par
\begin{description}
\item [\colorbox{tagtype}{\color{white} \textbf{\textsf{PARAMETER}}}] \textbf{\underline{instance}} ||| UNSIGNED4 --- Which match instance are we interested in?
\item [\colorbox{tagtype}{\color{white} \textbf{\textsf{PARAMETER}}}] \textbf{\underline{src}} ||| UNICODE --- The string that is searched
\item [\colorbox{tagtype}{\color{white} \textbf{\textsf{PARAMETER}}}] \textbf{\underline{sought}} ||| UNICODE --- The string being sought.
\end{description}







\par
\begin{description}
\item [\colorbox{tagtype}{\color{white} \textbf{\textsf{RETURN}}}] \textbf{UNSIGNED4} --- 
\end{description}




\rule{\linewidth}{0.5pt}
\subsection*{\textsf{\colorbox{headtoc}{\color{white} FUNCTION}
FindWord}}

\hypertarget{ecldoc:uni.findword}{}
\hspace{0pt} \hyperlink{ecldoc:Uni}{Uni} \textbackslash 

{\renewcommand{\arraystretch}{1.5}
\begin{tabularx}{\textwidth}{|>{\raggedright\arraybackslash}l|X|}
\hline
\hspace{0pt}\mytexttt{\color{red} BOOLEAN} & \textbf{FindWord} \\
\hline
\multicolumn{2}{|>{\raggedright\arraybackslash}X|}{\hspace{0pt}\mytexttt{\color{param} (UNICODE src, UNICODE word, BOOLEAN ignore\_case=FALSE)}} \\
\hline
\end{tabularx}
}

\par





Tests if the search string contains the supplied word as a whole word.






\par
\begin{description}
\item [\colorbox{tagtype}{\color{white} \textbf{\textsf{PARAMETER}}}] \textbf{\underline{word}} ||| UNICODE --- The word to be searched for.
\item [\colorbox{tagtype}{\color{white} \textbf{\textsf{PARAMETER}}}] \textbf{\underline{src}} ||| UNICODE --- The string that is being tested.
\item [\colorbox{tagtype}{\color{white} \textbf{\textsf{PARAMETER}}}] \textbf{\underline{ignore\_case}} ||| BOOLEAN --- Whether to ignore differences in case between characters.
\end{description}







\par
\begin{description}
\item [\colorbox{tagtype}{\color{white} \textbf{\textsf{RETURN}}}] \textbf{BOOLEAN} --- 
\end{description}




\rule{\linewidth}{0.5pt}
\subsection*{\textsf{\colorbox{headtoc}{\color{white} FUNCTION}
LocaleFind}}

\hypertarget{ecldoc:uni.localefind}{}
\hspace{0pt} \hyperlink{ecldoc:Uni}{Uni} \textbackslash 

{\renewcommand{\arraystretch}{1.5}
\begin{tabularx}{\textwidth}{|>{\raggedright\arraybackslash}l|X|}
\hline
\hspace{0pt}\mytexttt{\color{red} UNSIGNED4} & \textbf{LocaleFind} \\
\hline
\multicolumn{2}{|>{\raggedright\arraybackslash}X|}{\hspace{0pt}\mytexttt{\color{param} (unicode src, unicode sought, unsigned4 instance, varstring locale\_name)}} \\
\hline
\end{tabularx}
}

\par





Returns the character position of the nth match of the search string with the first string. If no match is found the attribute returns 0. If an instance is omitted the position of the first instance is returned.






\par
\begin{description}
\item [\colorbox{tagtype}{\color{white} \textbf{\textsf{PARAMETER}}}] \textbf{\underline{instance}} ||| UNSIGNED4 --- Which match instance are we interested in?
\item [\colorbox{tagtype}{\color{white} \textbf{\textsf{PARAMETER}}}] \textbf{\underline{src}} ||| UNICODE --- The string that is searched
\item [\colorbox{tagtype}{\color{white} \textbf{\textsf{PARAMETER}}}] \textbf{\underline{sought}} ||| UNICODE --- The string being sought.
\item [\colorbox{tagtype}{\color{white} \textbf{\textsf{PARAMETER}}}] \textbf{\underline{locale\_name}} ||| VARSTRING --- The locale to use for the comparison
\end{description}







\par
\begin{description}
\item [\colorbox{tagtype}{\color{white} \textbf{\textsf{RETURN}}}] \textbf{UNSIGNED4} --- 
\end{description}




\rule{\linewidth}{0.5pt}
\subsection*{\textsf{\colorbox{headtoc}{\color{white} FUNCTION}
LocaleFindAtStrength}}

\hypertarget{ecldoc:uni.localefindatstrength}{}
\hspace{0pt} \hyperlink{ecldoc:Uni}{Uni} \textbackslash 

{\renewcommand{\arraystretch}{1.5}
\begin{tabularx}{\textwidth}{|>{\raggedright\arraybackslash}l|X|}
\hline
\hspace{0pt}\mytexttt{\color{red} UNSIGNED4} & \textbf{LocaleFindAtStrength} \\
\hline
\multicolumn{2}{|>{\raggedright\arraybackslash}X|}{\hspace{0pt}\mytexttt{\color{param} (unicode src, unicode tofind, unsigned4 instance, varstring locale\_name, integer1 strength)}} \\
\hline
\end{tabularx}
}

\par





Returns the character position of the nth match of the search string with the first string. If no match is found the attribute returns 0. If an instance is omitted the position of the first instance is returned.






\par
\begin{description}
\item [\colorbox{tagtype}{\color{white} \textbf{\textsf{PARAMETER}}}] \textbf{\underline{instance}} ||| UNSIGNED4 --- Which match instance are we interested in?
\item [\colorbox{tagtype}{\color{white} \textbf{\textsf{PARAMETER}}}] \textbf{\underline{strength}} ||| INTEGER1 --- The strength of the comparison 1 ignores accents and case, differentiating only between letters 2 ignores case but differentiates between accents. 3 differentiates between accents and case but ignores e.g. differences between Hiragana and Katakana 4 differentiates between accents and case and e.g. Hiragana/Katakana, but ignores e.g. Hebrew cantellation marks 5 differentiates between all strings whose canonically decomposed forms (NFDNormalization Form D) are non-identical
\item [\colorbox{tagtype}{\color{white} \textbf{\textsf{PARAMETER}}}] \textbf{\underline{src}} ||| UNICODE --- The string that is searched
\item [\colorbox{tagtype}{\color{white} \textbf{\textsf{PARAMETER}}}] \textbf{\underline{sought}} |||  --- The string being sought.
\item [\colorbox{tagtype}{\color{white} \textbf{\textsf{PARAMETER}}}] \textbf{\underline{locale\_name}} ||| VARSTRING --- The locale to use for the comparison
\item [\colorbox{tagtype}{\color{white} \textbf{\textsf{PARAMETER}}}] \textbf{\underline{tofind}} ||| UNICODE --- No Doc
\end{description}







\par
\begin{description}
\item [\colorbox{tagtype}{\color{white} \textbf{\textsf{RETURN}}}] \textbf{UNSIGNED4} --- 
\end{description}




\rule{\linewidth}{0.5pt}
\subsection*{\textsf{\colorbox{headtoc}{\color{white} FUNCTION}
Extract}}

\hypertarget{ecldoc:uni.extract}{}
\hspace{0pt} \hyperlink{ecldoc:Uni}{Uni} \textbackslash 

{\renewcommand{\arraystretch}{1.5}
\begin{tabularx}{\textwidth}{|>{\raggedright\arraybackslash}l|X|}
\hline
\hspace{0pt}\mytexttt{\color{red} unicode} & \textbf{Extract} \\
\hline
\multicolumn{2}{|>{\raggedright\arraybackslash}X|}{\hspace{0pt}\mytexttt{\color{param} (unicode src, unsigned4 instance)}} \\
\hline
\end{tabularx}
}

\par





Returns the nth element from a comma separated string.






\par
\begin{description}
\item [\colorbox{tagtype}{\color{white} \textbf{\textsf{PARAMETER}}}] \textbf{\underline{instance}} ||| UNSIGNED4 --- Which item to select from the list.
\item [\colorbox{tagtype}{\color{white} \textbf{\textsf{PARAMETER}}}] \textbf{\underline{src}} ||| UNICODE --- The string containing the comma separated list.
\end{description}







\par
\begin{description}
\item [\colorbox{tagtype}{\color{white} \textbf{\textsf{RETURN}}}] \textbf{UNICODE} --- 
\end{description}




\rule{\linewidth}{0.5pt}
\subsection*{\textsf{\colorbox{headtoc}{\color{white} FUNCTION}
ToLowerCase}}

\hypertarget{ecldoc:uni.tolowercase}{}
\hspace{0pt} \hyperlink{ecldoc:Uni}{Uni} \textbackslash 

{\renewcommand{\arraystretch}{1.5}
\begin{tabularx}{\textwidth}{|>{\raggedright\arraybackslash}l|X|}
\hline
\hspace{0pt}\mytexttt{\color{red} unicode} & \textbf{ToLowerCase} \\
\hline
\multicolumn{2}{|>{\raggedright\arraybackslash}X|}{\hspace{0pt}\mytexttt{\color{param} (unicode src)}} \\
\hline
\end{tabularx}
}

\par





Returns the argument string with all upper case characters converted to lower case.






\par
\begin{description}
\item [\colorbox{tagtype}{\color{white} \textbf{\textsf{PARAMETER}}}] \textbf{\underline{src}} ||| UNICODE --- The string that is being converted.
\end{description}







\par
\begin{description}
\item [\colorbox{tagtype}{\color{white} \textbf{\textsf{RETURN}}}] \textbf{UNICODE} --- 
\end{description}




\rule{\linewidth}{0.5pt}
\subsection*{\textsf{\colorbox{headtoc}{\color{white} FUNCTION}
ToUpperCase}}

\hypertarget{ecldoc:uni.touppercase}{}
\hspace{0pt} \hyperlink{ecldoc:Uni}{Uni} \textbackslash 

{\renewcommand{\arraystretch}{1.5}
\begin{tabularx}{\textwidth}{|>{\raggedright\arraybackslash}l|X|}
\hline
\hspace{0pt}\mytexttt{\color{red} unicode} & \textbf{ToUpperCase} \\
\hline
\multicolumn{2}{|>{\raggedright\arraybackslash}X|}{\hspace{0pt}\mytexttt{\color{param} (unicode src)}} \\
\hline
\end{tabularx}
}

\par





Return the argument string with all lower case characters converted to upper case.






\par
\begin{description}
\item [\colorbox{tagtype}{\color{white} \textbf{\textsf{PARAMETER}}}] \textbf{\underline{src}} ||| UNICODE --- The string that is being converted.
\end{description}







\par
\begin{description}
\item [\colorbox{tagtype}{\color{white} \textbf{\textsf{RETURN}}}] \textbf{UNICODE} --- 
\end{description}




\rule{\linewidth}{0.5pt}
\subsection*{\textsf{\colorbox{headtoc}{\color{white} FUNCTION}
ToTitleCase}}

\hypertarget{ecldoc:uni.totitlecase}{}
\hspace{0pt} \hyperlink{ecldoc:Uni}{Uni} \textbackslash 

{\renewcommand{\arraystretch}{1.5}
\begin{tabularx}{\textwidth}{|>{\raggedright\arraybackslash}l|X|}
\hline
\hspace{0pt}\mytexttt{\color{red} unicode} & \textbf{ToTitleCase} \\
\hline
\multicolumn{2}{|>{\raggedright\arraybackslash}X|}{\hspace{0pt}\mytexttt{\color{param} (unicode src)}} \\
\hline
\end{tabularx}
}

\par





Returns the upper case variant of the string using the rules for a particular locale.






\par
\begin{description}
\item [\colorbox{tagtype}{\color{white} \textbf{\textsf{PARAMETER}}}] \textbf{\underline{src}} ||| UNICODE --- The string that is being converted.
\item [\colorbox{tagtype}{\color{white} \textbf{\textsf{PARAMETER}}}] \textbf{\underline{locale\_name}} |||  --- The locale to use for the comparison
\end{description}







\par
\begin{description}
\item [\colorbox{tagtype}{\color{white} \textbf{\textsf{RETURN}}}] \textbf{UNICODE} --- 
\end{description}




\rule{\linewidth}{0.5pt}
\subsection*{\textsf{\colorbox{headtoc}{\color{white} FUNCTION}
LocaleToLowerCase}}

\hypertarget{ecldoc:uni.localetolowercase}{}
\hspace{0pt} \hyperlink{ecldoc:Uni}{Uni} \textbackslash 

{\renewcommand{\arraystretch}{1.5}
\begin{tabularx}{\textwidth}{|>{\raggedright\arraybackslash}l|X|}
\hline
\hspace{0pt}\mytexttt{\color{red} unicode} & \textbf{LocaleToLowerCase} \\
\hline
\multicolumn{2}{|>{\raggedright\arraybackslash}X|}{\hspace{0pt}\mytexttt{\color{param} (unicode src, varstring locale\_name)}} \\
\hline
\end{tabularx}
}

\par





Returns the lower case variant of the string using the rules for a particular locale.






\par
\begin{description}
\item [\colorbox{tagtype}{\color{white} \textbf{\textsf{PARAMETER}}}] \textbf{\underline{src}} ||| UNICODE --- The string that is being converted.
\item [\colorbox{tagtype}{\color{white} \textbf{\textsf{PARAMETER}}}] \textbf{\underline{locale\_name}} ||| VARSTRING --- The locale to use for the comparison
\end{description}







\par
\begin{description}
\item [\colorbox{tagtype}{\color{white} \textbf{\textsf{RETURN}}}] \textbf{UNICODE} --- 
\end{description}




\rule{\linewidth}{0.5pt}
\subsection*{\textsf{\colorbox{headtoc}{\color{white} FUNCTION}
LocaleToUpperCase}}

\hypertarget{ecldoc:uni.localetouppercase}{}
\hspace{0pt} \hyperlink{ecldoc:Uni}{Uni} \textbackslash 

{\renewcommand{\arraystretch}{1.5}
\begin{tabularx}{\textwidth}{|>{\raggedright\arraybackslash}l|X|}
\hline
\hspace{0pt}\mytexttt{\color{red} unicode} & \textbf{LocaleToUpperCase} \\
\hline
\multicolumn{2}{|>{\raggedright\arraybackslash}X|}{\hspace{0pt}\mytexttt{\color{param} (unicode src, varstring locale\_name)}} \\
\hline
\end{tabularx}
}

\par





Returns the upper case variant of the string using the rules for a particular locale.






\par
\begin{description}
\item [\colorbox{tagtype}{\color{white} \textbf{\textsf{PARAMETER}}}] \textbf{\underline{src}} ||| UNICODE --- The string that is being converted.
\item [\colorbox{tagtype}{\color{white} \textbf{\textsf{PARAMETER}}}] \textbf{\underline{locale\_name}} ||| VARSTRING --- The locale to use for the comparison
\end{description}







\par
\begin{description}
\item [\colorbox{tagtype}{\color{white} \textbf{\textsf{RETURN}}}] \textbf{UNICODE} --- 
\end{description}




\rule{\linewidth}{0.5pt}
\subsection*{\textsf{\colorbox{headtoc}{\color{white} FUNCTION}
LocaleToTitleCase}}

\hypertarget{ecldoc:uni.localetotitlecase}{}
\hspace{0pt} \hyperlink{ecldoc:Uni}{Uni} \textbackslash 

{\renewcommand{\arraystretch}{1.5}
\begin{tabularx}{\textwidth}{|>{\raggedright\arraybackslash}l|X|}
\hline
\hspace{0pt}\mytexttt{\color{red} unicode} & \textbf{LocaleToTitleCase} \\
\hline
\multicolumn{2}{|>{\raggedright\arraybackslash}X|}{\hspace{0pt}\mytexttt{\color{param} (unicode src, varstring locale\_name)}} \\
\hline
\end{tabularx}
}

\par





Returns the upper case variant of the string using the rules for a particular locale.






\par
\begin{description}
\item [\colorbox{tagtype}{\color{white} \textbf{\textsf{PARAMETER}}}] \textbf{\underline{src}} ||| UNICODE --- The string that is being converted.
\item [\colorbox{tagtype}{\color{white} \textbf{\textsf{PARAMETER}}}] \textbf{\underline{locale\_name}} ||| VARSTRING --- The locale to use for the comparison
\end{description}







\par
\begin{description}
\item [\colorbox{tagtype}{\color{white} \textbf{\textsf{RETURN}}}] \textbf{UNICODE} --- 
\end{description}




\rule{\linewidth}{0.5pt}
\subsection*{\textsf{\colorbox{headtoc}{\color{white} FUNCTION}
CompareIgnoreCase}}

\hypertarget{ecldoc:uni.compareignorecase}{}
\hspace{0pt} \hyperlink{ecldoc:Uni}{Uni} \textbackslash 

{\renewcommand{\arraystretch}{1.5}
\begin{tabularx}{\textwidth}{|>{\raggedright\arraybackslash}l|X|}
\hline
\hspace{0pt}\mytexttt{\color{red} integer4} & \textbf{CompareIgnoreCase} \\
\hline
\multicolumn{2}{|>{\raggedright\arraybackslash}X|}{\hspace{0pt}\mytexttt{\color{param} (unicode src1, unicode src2)}} \\
\hline
\end{tabularx}
}

\par





Compares the two strings case insensitively. Equivalent to comparing at strength 2.






\par
\begin{description}
\item [\colorbox{tagtype}{\color{white} \textbf{\textsf{PARAMETER}}}] \textbf{\underline{src2}} ||| UNICODE --- The second string to be compared.
\item [\colorbox{tagtype}{\color{white} \textbf{\textsf{PARAMETER}}}] \textbf{\underline{src1}} ||| UNICODE --- The first string to be compared.
\end{description}







\par
\begin{description}
\item [\colorbox{tagtype}{\color{white} \textbf{\textsf{RETURN}}}] \textbf{INTEGER4} --- 
\end{description}






\par
\begin{description}
\item [\colorbox{tagtype}{\color{white} \textbf{\textsf{SEE}}}] Std.Uni.CompareAtStrength
\end{description}




\rule{\linewidth}{0.5pt}
\subsection*{\textsf{\colorbox{headtoc}{\color{white} FUNCTION}
CompareAtStrength}}

\hypertarget{ecldoc:uni.compareatstrength}{}
\hspace{0pt} \hyperlink{ecldoc:Uni}{Uni} \textbackslash 

{\renewcommand{\arraystretch}{1.5}
\begin{tabularx}{\textwidth}{|>{\raggedright\arraybackslash}l|X|}
\hline
\hspace{0pt}\mytexttt{\color{red} integer4} & \textbf{CompareAtStrength} \\
\hline
\multicolumn{2}{|>{\raggedright\arraybackslash}X|}{\hspace{0pt}\mytexttt{\color{param} (unicode src1, unicode src2, integer1 strength)}} \\
\hline
\end{tabularx}
}

\par





Compares the two strings case insensitively. Equivalent to comparing at strength 2.






\par
\begin{description}
\item [\colorbox{tagtype}{\color{white} \textbf{\textsf{PARAMETER}}}] \textbf{\underline{src2}} ||| UNICODE --- The second string to be compared.
\item [\colorbox{tagtype}{\color{white} \textbf{\textsf{PARAMETER}}}] \textbf{\underline{strength}} ||| INTEGER1 --- The strength of the comparison 1 ignores accents and case, differentiating only between letters 2 ignores case but differentiates between accents. 3 differentiates between accents and case but ignores e.g. differences between Hiragana and Katakana 4 differentiates between accents and case and e.g. Hiragana/Katakana, but ignores e.g. Hebrew cantellation marks 5 differentiates between all strings whose canonically decomposed forms (NFDNormalization Form D) are non-identical
\item [\colorbox{tagtype}{\color{white} \textbf{\textsf{PARAMETER}}}] \textbf{\underline{src1}} ||| UNICODE --- The first string to be compared.
\end{description}







\par
\begin{description}
\item [\colorbox{tagtype}{\color{white} \textbf{\textsf{RETURN}}}] \textbf{INTEGER4} --- 
\end{description}






\par
\begin{description}
\item [\colorbox{tagtype}{\color{white} \textbf{\textsf{SEE}}}] Std.Uni.CompareAtStrength
\end{description}




\rule{\linewidth}{0.5pt}
\subsection*{\textsf{\colorbox{headtoc}{\color{white} FUNCTION}
LocaleCompareIgnoreCase}}

\hypertarget{ecldoc:uni.localecompareignorecase}{}
\hspace{0pt} \hyperlink{ecldoc:Uni}{Uni} \textbackslash 

{\renewcommand{\arraystretch}{1.5}
\begin{tabularx}{\textwidth}{|>{\raggedright\arraybackslash}l|X|}
\hline
\hspace{0pt}\mytexttt{\color{red} integer4} & \textbf{LocaleCompareIgnoreCase} \\
\hline
\multicolumn{2}{|>{\raggedright\arraybackslash}X|}{\hspace{0pt}\mytexttt{\color{param} (unicode src1, unicode src2, varstring locale\_name)}} \\
\hline
\end{tabularx}
}

\par





Compares the two strings case insensitively. Equivalent to comparing at strength 2.






\par
\begin{description}
\item [\colorbox{tagtype}{\color{white} \textbf{\textsf{PARAMETER}}}] \textbf{\underline{src2}} ||| UNICODE --- The second string to be compared.
\item [\colorbox{tagtype}{\color{white} \textbf{\textsf{PARAMETER}}}] \textbf{\underline{src1}} ||| UNICODE --- The first string to be compared.
\item [\colorbox{tagtype}{\color{white} \textbf{\textsf{PARAMETER}}}] \textbf{\underline{locale\_name}} ||| VARSTRING --- The locale to use for the comparison
\end{description}







\par
\begin{description}
\item [\colorbox{tagtype}{\color{white} \textbf{\textsf{RETURN}}}] \textbf{INTEGER4} --- 
\end{description}






\par
\begin{description}
\item [\colorbox{tagtype}{\color{white} \textbf{\textsf{SEE}}}] Std.Uni.CompareAtStrength
\end{description}




\rule{\linewidth}{0.5pt}
\subsection*{\textsf{\colorbox{headtoc}{\color{white} FUNCTION}
LocaleCompareAtStrength}}

\hypertarget{ecldoc:uni.localecompareatstrength}{}
\hspace{0pt} \hyperlink{ecldoc:Uni}{Uni} \textbackslash 

{\renewcommand{\arraystretch}{1.5}
\begin{tabularx}{\textwidth}{|>{\raggedright\arraybackslash}l|X|}
\hline
\hspace{0pt}\mytexttt{\color{red} integer4} & \textbf{LocaleCompareAtStrength} \\
\hline
\multicolumn{2}{|>{\raggedright\arraybackslash}X|}{\hspace{0pt}\mytexttt{\color{param} (unicode src1, unicode src2, varstring locale\_name, integer1 strength)}} \\
\hline
\end{tabularx}
}

\par





Compares the two strings case insensitively. Equivalent to comparing at strength 2.






\par
\begin{description}
\item [\colorbox{tagtype}{\color{white} \textbf{\textsf{PARAMETER}}}] \textbf{\underline{src2}} ||| UNICODE --- The second string to be compared.
\item [\colorbox{tagtype}{\color{white} \textbf{\textsf{PARAMETER}}}] \textbf{\underline{strength}} ||| INTEGER1 --- The strength of the comparison 1 ignores accents and case, differentiating only between letters 2 ignores case but differentiates between accents. 3 differentiates between accents and case but ignores e.g. differences between Hiragana and Katakana 4 differentiates between accents and case and e.g. Hiragana/Katakana, but ignores e.g. Hebrew cantellation marks 5 differentiates between all strings whose canonically decomposed forms (NFDNormalization Form D) are non-identical
\item [\colorbox{tagtype}{\color{white} \textbf{\textsf{PARAMETER}}}] \textbf{\underline{src1}} ||| UNICODE --- The first string to be compared.
\item [\colorbox{tagtype}{\color{white} \textbf{\textsf{PARAMETER}}}] \textbf{\underline{locale\_name}} ||| VARSTRING --- The locale to use for the comparison
\end{description}







\par
\begin{description}
\item [\colorbox{tagtype}{\color{white} \textbf{\textsf{RETURN}}}] \textbf{INTEGER4} --- 
\end{description}




\rule{\linewidth}{0.5pt}
\subsection*{\textsf{\colorbox{headtoc}{\color{white} FUNCTION}
Reverse}}

\hypertarget{ecldoc:uni.reverse}{}
\hspace{0pt} \hyperlink{ecldoc:Uni}{Uni} \textbackslash 

{\renewcommand{\arraystretch}{1.5}
\begin{tabularx}{\textwidth}{|>{\raggedright\arraybackslash}l|X|}
\hline
\hspace{0pt}\mytexttt{\color{red} unicode} & \textbf{Reverse} \\
\hline
\multicolumn{2}{|>{\raggedright\arraybackslash}X|}{\hspace{0pt}\mytexttt{\color{param} (unicode src)}} \\
\hline
\end{tabularx}
}

\par





Returns the argument string with all characters in reverse order. Note the argument is not TRIMMED before it is reversed.






\par
\begin{description}
\item [\colorbox{tagtype}{\color{white} \textbf{\textsf{PARAMETER}}}] \textbf{\underline{src}} ||| UNICODE --- The string that is being reversed.
\end{description}







\par
\begin{description}
\item [\colorbox{tagtype}{\color{white} \textbf{\textsf{RETURN}}}] \textbf{UNICODE} --- 
\end{description}




\rule{\linewidth}{0.5pt}
\subsection*{\textsf{\colorbox{headtoc}{\color{white} FUNCTION}
FindReplace}}

\hypertarget{ecldoc:uni.findreplace}{}
\hspace{0pt} \hyperlink{ecldoc:Uni}{Uni} \textbackslash 

{\renewcommand{\arraystretch}{1.5}
\begin{tabularx}{\textwidth}{|>{\raggedright\arraybackslash}l|X|}
\hline
\hspace{0pt}\mytexttt{\color{red} unicode} & \textbf{FindReplace} \\
\hline
\multicolumn{2}{|>{\raggedright\arraybackslash}X|}{\hspace{0pt}\mytexttt{\color{param} (unicode src, unicode sought, unicode replacement)}} \\
\hline
\end{tabularx}
}

\par





Returns the source string with the replacement string substituted for all instances of the search string.






\par
\begin{description}
\item [\colorbox{tagtype}{\color{white} \textbf{\textsf{PARAMETER}}}] \textbf{\underline{src}} ||| UNICODE --- The string that is being transformed.
\item [\colorbox{tagtype}{\color{white} \textbf{\textsf{PARAMETER}}}] \textbf{\underline{replacement}} ||| UNICODE --- The string to be substituted into the result.
\item [\colorbox{tagtype}{\color{white} \textbf{\textsf{PARAMETER}}}] \textbf{\underline{sought}} ||| UNICODE --- The string to be replaced.
\end{description}







\par
\begin{description}
\item [\colorbox{tagtype}{\color{white} \textbf{\textsf{RETURN}}}] \textbf{UNICODE} --- 
\end{description}




\rule{\linewidth}{0.5pt}
\subsection*{\textsf{\colorbox{headtoc}{\color{white} FUNCTION}
LocaleFindReplace}}

\hypertarget{ecldoc:uni.localefindreplace}{}
\hspace{0pt} \hyperlink{ecldoc:Uni}{Uni} \textbackslash 

{\renewcommand{\arraystretch}{1.5}
\begin{tabularx}{\textwidth}{|>{\raggedright\arraybackslash}l|X|}
\hline
\hspace{0pt}\mytexttt{\color{red} unicode} & \textbf{LocaleFindReplace} \\
\hline
\multicolumn{2}{|>{\raggedright\arraybackslash}X|}{\hspace{0pt}\mytexttt{\color{param} (unicode src, unicode sought, unicode replacement, varstring locale\_name)}} \\
\hline
\end{tabularx}
}

\par





Returns the source string with the replacement string substituted for all instances of the search string.






\par
\begin{description}
\item [\colorbox{tagtype}{\color{white} \textbf{\textsf{PARAMETER}}}] \textbf{\underline{src}} ||| UNICODE --- The string that is being transformed.
\item [\colorbox{tagtype}{\color{white} \textbf{\textsf{PARAMETER}}}] \textbf{\underline{replacement}} ||| UNICODE --- The string to be substituted into the result.
\item [\colorbox{tagtype}{\color{white} \textbf{\textsf{PARAMETER}}}] \textbf{\underline{sought}} ||| UNICODE --- The string to be replaced.
\item [\colorbox{tagtype}{\color{white} \textbf{\textsf{PARAMETER}}}] \textbf{\underline{locale\_name}} ||| VARSTRING --- The locale to use for the comparison
\end{description}







\par
\begin{description}
\item [\colorbox{tagtype}{\color{white} \textbf{\textsf{RETURN}}}] \textbf{UNICODE} --- 
\end{description}




\rule{\linewidth}{0.5pt}
\subsection*{\textsf{\colorbox{headtoc}{\color{white} FUNCTION}
LocaleFindAtStrengthReplace}}

\hypertarget{ecldoc:uni.localefindatstrengthreplace}{}
\hspace{0pt} \hyperlink{ecldoc:Uni}{Uni} \textbackslash 

{\renewcommand{\arraystretch}{1.5}
\begin{tabularx}{\textwidth}{|>{\raggedright\arraybackslash}l|X|}
\hline
\hspace{0pt}\mytexttt{\color{red} unicode} & \textbf{LocaleFindAtStrengthReplace} \\
\hline
\multicolumn{2}{|>{\raggedright\arraybackslash}X|}{\hspace{0pt}\mytexttt{\color{param} (unicode src, unicode sought, unicode replacement, varstring locale\_name, integer1 strength)}} \\
\hline
\end{tabularx}
}

\par





Returns the source string with the replacement string substituted for all instances of the search string.






\par
\begin{description}
\item [\colorbox{tagtype}{\color{white} \textbf{\textsf{PARAMETER}}}] \textbf{\underline{strength}} ||| INTEGER1 --- The strength of the comparison
\item [\colorbox{tagtype}{\color{white} \textbf{\textsf{PARAMETER}}}] \textbf{\underline{src}} ||| UNICODE --- The string that is being transformed.
\item [\colorbox{tagtype}{\color{white} \textbf{\textsf{PARAMETER}}}] \textbf{\underline{replacement}} ||| UNICODE --- The string to be substituted into the result.
\item [\colorbox{tagtype}{\color{white} \textbf{\textsf{PARAMETER}}}] \textbf{\underline{sought}} ||| UNICODE --- The string to be replaced.
\item [\colorbox{tagtype}{\color{white} \textbf{\textsf{PARAMETER}}}] \textbf{\underline{locale\_name}} ||| VARSTRING --- The locale to use for the comparison
\end{description}







\par
\begin{description}
\item [\colorbox{tagtype}{\color{white} \textbf{\textsf{RETURN}}}] \textbf{UNICODE} --- 
\end{description}




\rule{\linewidth}{0.5pt}
\subsection*{\textsf{\colorbox{headtoc}{\color{white} FUNCTION}
CleanAccents}}

\hypertarget{ecldoc:uni.cleanaccents}{}
\hspace{0pt} \hyperlink{ecldoc:Uni}{Uni} \textbackslash 

{\renewcommand{\arraystretch}{1.5}
\begin{tabularx}{\textwidth}{|>{\raggedright\arraybackslash}l|X|}
\hline
\hspace{0pt}\mytexttt{\color{red} unicode} & \textbf{CleanAccents} \\
\hline
\multicolumn{2}{|>{\raggedright\arraybackslash}X|}{\hspace{0pt}\mytexttt{\color{param} (unicode src)}} \\
\hline
\end{tabularx}
}

\par





Returns the source string with all accented characters replaced with unaccented.






\par
\begin{description}
\item [\colorbox{tagtype}{\color{white} \textbf{\textsf{PARAMETER}}}] \textbf{\underline{src}} ||| UNICODE --- The string that is being transformed.
\end{description}







\par
\begin{description}
\item [\colorbox{tagtype}{\color{white} \textbf{\textsf{RETURN}}}] \textbf{UNICODE} --- 
\end{description}




\rule{\linewidth}{0.5pt}
\subsection*{\textsf{\colorbox{headtoc}{\color{white} FUNCTION}
CleanSpaces}}

\hypertarget{ecldoc:uni.cleanspaces}{}
\hspace{0pt} \hyperlink{ecldoc:Uni}{Uni} \textbackslash 

{\renewcommand{\arraystretch}{1.5}
\begin{tabularx}{\textwidth}{|>{\raggedright\arraybackslash}l|X|}
\hline
\hspace{0pt}\mytexttt{\color{red} unicode} & \textbf{CleanSpaces} \\
\hline
\multicolumn{2}{|>{\raggedright\arraybackslash}X|}{\hspace{0pt}\mytexttt{\color{param} (unicode src)}} \\
\hline
\end{tabularx}
}

\par





Returns the source string with all instances of multiple adjacent space characters (2 or more spaces together) reduced to a single space character. Leading and trailing spaces are removed, and tab characters are converted to spaces.






\par
\begin{description}
\item [\colorbox{tagtype}{\color{white} \textbf{\textsf{PARAMETER}}}] \textbf{\underline{src}} ||| UNICODE --- The string to be cleaned.
\end{description}







\par
\begin{description}
\item [\colorbox{tagtype}{\color{white} \textbf{\textsf{RETURN}}}] \textbf{UNICODE} --- 
\end{description}




\rule{\linewidth}{0.5pt}
\subsection*{\textsf{\colorbox{headtoc}{\color{white} FUNCTION}
WildMatch}}

\hypertarget{ecldoc:uni.wildmatch}{}
\hspace{0pt} \hyperlink{ecldoc:Uni}{Uni} \textbackslash 

{\renewcommand{\arraystretch}{1.5}
\begin{tabularx}{\textwidth}{|>{\raggedright\arraybackslash}l|X|}
\hline
\hspace{0pt}\mytexttt{\color{red} boolean} & \textbf{WildMatch} \\
\hline
\multicolumn{2}{|>{\raggedright\arraybackslash}X|}{\hspace{0pt}\mytexttt{\color{param} (unicode src, unicode \_pattern, boolean \_noCase)}} \\
\hline
\end{tabularx}
}

\par





Tests if the search string matches the pattern. The pattern can contain wildcards '?' (single character) and '*' (multiple character).






\par
\begin{description}
\item [\colorbox{tagtype}{\color{white} \textbf{\textsf{PARAMETER}}}] \textbf{\underline{pattern}} |||  --- The pattern to match against.
\item [\colorbox{tagtype}{\color{white} \textbf{\textsf{PARAMETER}}}] \textbf{\underline{src}} ||| UNICODE --- The string that is being tested.
\item [\colorbox{tagtype}{\color{white} \textbf{\textsf{PARAMETER}}}] \textbf{\underline{ignore\_case}} |||  --- Whether to ignore differences in case between characters
\item [\colorbox{tagtype}{\color{white} \textbf{\textsf{PARAMETER}}}] \textbf{\underline{\_nocase}} ||| BOOLEAN --- No Doc
\item [\colorbox{tagtype}{\color{white} \textbf{\textsf{PARAMETER}}}] \textbf{\underline{\_pattern}} ||| UNICODE --- No Doc
\end{description}







\par
\begin{description}
\item [\colorbox{tagtype}{\color{white} \textbf{\textsf{RETURN}}}] \textbf{BOOLEAN} --- 
\end{description}




\rule{\linewidth}{0.5pt}
\subsection*{\textsf{\colorbox{headtoc}{\color{white} FUNCTION}
Contains}}

\hypertarget{ecldoc:uni.contains}{}
\hspace{0pt} \hyperlink{ecldoc:Uni}{Uni} \textbackslash 

{\renewcommand{\arraystretch}{1.5}
\begin{tabularx}{\textwidth}{|>{\raggedright\arraybackslash}l|X|}
\hline
\hspace{0pt}\mytexttt{\color{red} BOOLEAN} & \textbf{Contains} \\
\hline
\multicolumn{2}{|>{\raggedright\arraybackslash}X|}{\hspace{0pt}\mytexttt{\color{param} (unicode src, unicode \_pattern, boolean \_noCase)}} \\
\hline
\end{tabularx}
}

\par





Tests if the search string contains each of the characters in the pattern. If the pattern contains duplicate characters those characters will match once for each occurence in the pattern.






\par
\begin{description}
\item [\colorbox{tagtype}{\color{white} \textbf{\textsf{PARAMETER}}}] \textbf{\underline{pattern}} |||  --- The pattern to match against.
\item [\colorbox{tagtype}{\color{white} \textbf{\textsf{PARAMETER}}}] \textbf{\underline{src}} ||| UNICODE --- The string that is being tested.
\item [\colorbox{tagtype}{\color{white} \textbf{\textsf{PARAMETER}}}] \textbf{\underline{ignore\_case}} |||  --- Whether to ignore differences in case between characters
\item [\colorbox{tagtype}{\color{white} \textbf{\textsf{PARAMETER}}}] \textbf{\underline{\_nocase}} ||| BOOLEAN --- No Doc
\item [\colorbox{tagtype}{\color{white} \textbf{\textsf{PARAMETER}}}] \textbf{\underline{\_pattern}} ||| UNICODE --- No Doc
\end{description}







\par
\begin{description}
\item [\colorbox{tagtype}{\color{white} \textbf{\textsf{RETURN}}}] \textbf{BOOLEAN} --- 
\end{description}




\rule{\linewidth}{0.5pt}
\subsection*{\textsf{\colorbox{headtoc}{\color{white} FUNCTION}
EditDistance}}

\hypertarget{ecldoc:uni.editdistance}{}
\hspace{0pt} \hyperlink{ecldoc:Uni}{Uni} \textbackslash 

{\renewcommand{\arraystretch}{1.5}
\begin{tabularx}{\textwidth}{|>{\raggedright\arraybackslash}l|X|}
\hline
\hspace{0pt}\mytexttt{\color{red} UNSIGNED4} & \textbf{EditDistance} \\
\hline
\multicolumn{2}{|>{\raggedright\arraybackslash}X|}{\hspace{0pt}\mytexttt{\color{param} (unicode \_left, unicode \_right, varstring localename = '')}} \\
\hline
\end{tabularx}
}

\par





Returns the minimum edit distance between the two strings. An insert change or delete counts as a single edit. The two strings are trimmed before comparing.






\par
\begin{description}
\item [\colorbox{tagtype}{\color{white} \textbf{\textsf{PARAMETER}}}] \textbf{\underline{\_left}} ||| UNICODE --- The first string to be compared.
\item [\colorbox{tagtype}{\color{white} \textbf{\textsf{PARAMETER}}}] \textbf{\underline{localname}} |||  --- The locale to use for the comparison. Defaults to ''.
\item [\colorbox{tagtype}{\color{white} \textbf{\textsf{PARAMETER}}}] \textbf{\underline{\_right}} ||| UNICODE --- The second string to be compared.
\item [\colorbox{tagtype}{\color{white} \textbf{\textsf{PARAMETER}}}] \textbf{\underline{localename}} ||| VARSTRING --- No Doc
\end{description}







\par
\begin{description}
\item [\colorbox{tagtype}{\color{white} \textbf{\textsf{RETURN}}}] \textbf{UNSIGNED4} --- The minimum edit distance between the two strings.
\end{description}




\rule{\linewidth}{0.5pt}
\subsection*{\textsf{\colorbox{headtoc}{\color{white} FUNCTION}
EditDistanceWithinRadius}}

\hypertarget{ecldoc:uni.editdistancewithinradius}{}
\hspace{0pt} \hyperlink{ecldoc:Uni}{Uni} \textbackslash 

{\renewcommand{\arraystretch}{1.5}
\begin{tabularx}{\textwidth}{|>{\raggedright\arraybackslash}l|X|}
\hline
\hspace{0pt}\mytexttt{\color{red} BOOLEAN} & \textbf{EditDistanceWithinRadius} \\
\hline
\multicolumn{2}{|>{\raggedright\arraybackslash}X|}{\hspace{0pt}\mytexttt{\color{param} (unicode \_left, unicode \_right, unsigned4 radius, varstring localename = '')}} \\
\hline
\end{tabularx}
}

\par





Returns true if the minimum edit distance between the two strings is with a specific range. The two strings are trimmed before comparing.






\par
\begin{description}
\item [\colorbox{tagtype}{\color{white} \textbf{\textsf{PARAMETER}}}] \textbf{\underline{\_left}} ||| UNICODE --- The first string to be compared.
\item [\colorbox{tagtype}{\color{white} \textbf{\textsf{PARAMETER}}}] \textbf{\underline{localname}} |||  --- The locale to use for the comparison. Defaults to ''.
\item [\colorbox{tagtype}{\color{white} \textbf{\textsf{PARAMETER}}}] \textbf{\underline{\_right}} ||| UNICODE --- The second string to be compared.
\item [\colorbox{tagtype}{\color{white} \textbf{\textsf{PARAMETER}}}] \textbf{\underline{radius}} ||| UNSIGNED4 --- The maximum edit distance that is accepable.
\item [\colorbox{tagtype}{\color{white} \textbf{\textsf{PARAMETER}}}] \textbf{\underline{localename}} ||| VARSTRING --- No Doc
\end{description}







\par
\begin{description}
\item [\colorbox{tagtype}{\color{white} \textbf{\textsf{RETURN}}}] \textbf{BOOLEAN} --- Whether or not the two strings are within the given specified edit distance.
\end{description}




\rule{\linewidth}{0.5pt}
\subsection*{\textsf{\colorbox{headtoc}{\color{white} FUNCTION}
WordCount}}

\hypertarget{ecldoc:uni.wordcount}{}
\hspace{0pt} \hyperlink{ecldoc:Uni}{Uni} \textbackslash 

{\renewcommand{\arraystretch}{1.5}
\begin{tabularx}{\textwidth}{|>{\raggedright\arraybackslash}l|X|}
\hline
\hspace{0pt}\mytexttt{\color{red} unsigned4} & \textbf{WordCount} \\
\hline
\multicolumn{2}{|>{\raggedright\arraybackslash}X|}{\hspace{0pt}\mytexttt{\color{param} (unicode text, varstring localename = '')}} \\
\hline
\end{tabularx}
}

\par





Returns the number of words in the string. Word boundaries are marked by the unicode break semantics.






\par
\begin{description}
\item [\colorbox{tagtype}{\color{white} \textbf{\textsf{PARAMETER}}}] \textbf{\underline{localname}} |||  --- The locale to use for the break semantics. Defaults to ''.
\item [\colorbox{tagtype}{\color{white} \textbf{\textsf{PARAMETER}}}] \textbf{\underline{text}} ||| UNICODE --- The string to be broken into words.
\item [\colorbox{tagtype}{\color{white} \textbf{\textsf{PARAMETER}}}] \textbf{\underline{localename}} ||| VARSTRING --- No Doc
\end{description}







\par
\begin{description}
\item [\colorbox{tagtype}{\color{white} \textbf{\textsf{RETURN}}}] \textbf{UNSIGNED4} --- The number of words in the string.
\end{description}




\rule{\linewidth}{0.5pt}
\subsection*{\textsf{\colorbox{headtoc}{\color{white} FUNCTION}
GetNthWord}}

\hypertarget{ecldoc:uni.getnthword}{}
\hspace{0pt} \hyperlink{ecldoc:Uni}{Uni} \textbackslash 

{\renewcommand{\arraystretch}{1.5}
\begin{tabularx}{\textwidth}{|>{\raggedright\arraybackslash}l|X|}
\hline
\hspace{0pt}\mytexttt{\color{red} unicode} & \textbf{GetNthWord} \\
\hline
\multicolumn{2}{|>{\raggedright\arraybackslash}X|}{\hspace{0pt}\mytexttt{\color{param} (unicode text, unsigned4 n, varstring localename = '')}} \\
\hline
\end{tabularx}
}

\par





Returns the n-th word from the string. Word boundaries are marked by the unicode break semantics.






\par
\begin{description}
\item [\colorbox{tagtype}{\color{white} \textbf{\textsf{PARAMETER}}}] \textbf{\underline{localname}} |||  --- The locale to use for the break semantics. Defaults to ''.
\item [\colorbox{tagtype}{\color{white} \textbf{\textsf{PARAMETER}}}] \textbf{\underline{n}} ||| UNSIGNED4 --- Which word should be returned from the function.
\item [\colorbox{tagtype}{\color{white} \textbf{\textsf{PARAMETER}}}] \textbf{\underline{text}} ||| UNICODE --- The string to be broken into words.
\item [\colorbox{tagtype}{\color{white} \textbf{\textsf{PARAMETER}}}] \textbf{\underline{localename}} ||| VARSTRING --- No Doc
\end{description}







\par
\begin{description}
\item [\colorbox{tagtype}{\color{white} \textbf{\textsf{RETURN}}}] \textbf{UNICODE} --- The number of words in the string.
\end{description}




\rule{\linewidth}{0.5pt}



\chapter*{\color{headtoc} root}
\hypertarget{ecldoc:toc:root}{}
\hyperlink{ecldoc:toc:}{Go Up}


\section*{Table of Contents}
{\renewcommand{\arraystretch}{1.5}
\begin{longtable}{|p{\textwidth}|}
\hline
\hyperlink{ecldoc:toc:BLAS}{BLAS.ecl} \\
\hline
\hyperlink{ecldoc:toc:BundleBase}{BundleBase.ecl} \\
\hline
\hyperlink{ecldoc:toc:Date}{Date.ecl} \\
\hline
\hyperlink{ecldoc:toc:File}{File.ecl} \\
\hline
\hyperlink{ecldoc:toc:math}{math.ecl} \\
\hline
\hyperlink{ecldoc:toc:Metaphone}{Metaphone.ecl} \\
\hline
\hyperlink{ecldoc:toc:str}{str.ecl} \\
\hline
\hyperlink{ecldoc:toc:Uni}{Uni.ecl} \\
\hline
\hyperlink{ecldoc:toc:root/system}{system} \\
\hline
\end{longtable}
}

\chapter*{\color{headfile}
BLAS
}
\hypertarget{ecldoc:toc:BLAS}{}
\hyperlink{ecldoc:toc:root}{Go Up}

\section*{\underline{\textsf{IMPORTS}}}
\begin{doublespace}
{\large
lib\_eclblas |
}
\end{doublespace}

\section*{\underline{\textsf{DESCRIPTIONS}}}
\subsection*{\textsf{\colorbox{headtoc}{\color{white} MODULE}
BLAS}}

\hypertarget{ecldoc:blas}{}

{\renewcommand{\arraystretch}{1.5}
\begin{tabularx}{\textwidth}{|>{\raggedright\arraybackslash}l|X|}
\hline
\hspace{0pt}\mytexttt{\color{red} } & \textbf{BLAS} \\
\hline
\end{tabularx}
}

\par


\textbf{Children}
\begin{enumerate}
\item \hyperlink{ecldoc:BLAS.Types}{Types}
\item \hyperlink{ecldoc:blas.icellfunc}{ICellFunc}
: Function prototype for Apply2Cell
\item \hyperlink{ecldoc:blas.apply2cells}{Apply2Cells}
: Iterate matrix and apply function to each cell
\item \hyperlink{ecldoc:blas.dasum}{dasum}
: Absolute sum, the 1 norm of a vector
\item \hyperlink{ecldoc:blas.daxpy}{daxpy}
: alpha*X + Y
\item \hyperlink{ecldoc:blas.dgemm}{dgemm}
: alpha*op(A) op(B) + beta*C where op() is transpose
\item \hyperlink{ecldoc:blas.dgetf2}{dgetf2}
: Compute LU Factorization of matrix A
\item \hyperlink{ecldoc:blas.dpotf2}{dpotf2}
: DPOTF2 computes the Cholesky factorization of a real symmetric positive definite matrix A
\item \hyperlink{ecldoc:blas.dscal}{dscal}
: Scale a vector alpha
\item \hyperlink{ecldoc:blas.dsyrk}{dsyrk}
: Implements symmetric rank update C
\item \hyperlink{ecldoc:blas.dtrsm}{dtrsm}
: Triangular matrix solver
\item \hyperlink{ecldoc:blas.extract_diag}{extract\_diag}
: Extract the diagonal of he matrix
\item \hyperlink{ecldoc:blas.extract_tri}{extract\_tri}
: Extract the upper or lower triangle
\item \hyperlink{ecldoc:blas.make_diag}{make\_diag}
: Generate a diagonal matrix
\item \hyperlink{ecldoc:blas.make_vector}{make\_vector}
: Make a vector of dimension m
\item \hyperlink{ecldoc:blas.trace}{trace}
: The trace of the input matrix
\end{enumerate}

\rule{\linewidth}{0.5pt}

\subsection*{\textsf{\colorbox{headtoc}{\color{white} MODULE}
Types}}

\hypertarget{ecldoc:BLAS.Types}{}
\hspace{0pt} \hyperlink{ecldoc:blas}{BLAS} \textbackslash 

{\renewcommand{\arraystretch}{1.5}
\begin{tabularx}{\textwidth}{|>{\raggedright\arraybackslash}l|X|}
\hline
\hspace{0pt}\mytexttt{\color{red} } & \textbf{Types} \\
\hline
\end{tabularx}
}

\par


\textbf{Children}
\begin{enumerate}
\item \hyperlink{ecldoc:blas.types.value_t}{value\_t}
\item \hyperlink{ecldoc:blas.types.dimension_t}{dimension\_t}
\item \hyperlink{ecldoc:blas.types.matrix_t}{matrix\_t}
\item \hyperlink{ecldoc:ecldoc-Triangle}{Triangle}
\item \hyperlink{ecldoc:ecldoc-Diagonal}{Diagonal}
\item \hyperlink{ecldoc:ecldoc-Side}{Side}
\end{enumerate}

\rule{\linewidth}{0.5pt}

\subsection*{\textsf{\colorbox{headtoc}{\color{white} ATTRIBUTE}
value\_t}}

\hypertarget{ecldoc:blas.types.value_t}{}
\hspace{0pt} \hyperlink{ecldoc:blas}{BLAS} \textbackslash 
\hspace{0pt} \hyperlink{ecldoc:BLAS.Types}{Types} \textbackslash 

{\renewcommand{\arraystretch}{1.5}
\begin{tabularx}{\textwidth}{|>{\raggedright\arraybackslash}l|X|}
\hline
\hspace{0pt}\mytexttt{\color{red} } & \textbf{value\_t} \\
\hline
\end{tabularx}
}

\par


\rule{\linewidth}{0.5pt}
\subsection*{\textsf{\colorbox{headtoc}{\color{white} ATTRIBUTE}
dimension\_t}}

\hypertarget{ecldoc:blas.types.dimension_t}{}
\hspace{0pt} \hyperlink{ecldoc:blas}{BLAS} \textbackslash 
\hspace{0pt} \hyperlink{ecldoc:BLAS.Types}{Types} \textbackslash 

{\renewcommand{\arraystretch}{1.5}
\begin{tabularx}{\textwidth}{|>{\raggedright\arraybackslash}l|X|}
\hline
\hspace{0pt}\mytexttt{\color{red} } & \textbf{dimension\_t} \\
\hline
\end{tabularx}
}

\par


\rule{\linewidth}{0.5pt}
\subsection*{\textsf{\colorbox{headtoc}{\color{white} ATTRIBUTE}
matrix\_t}}

\hypertarget{ecldoc:blas.types.matrix_t}{}
\hspace{0pt} \hyperlink{ecldoc:blas}{BLAS} \textbackslash 
\hspace{0pt} \hyperlink{ecldoc:BLAS.Types}{Types} \textbackslash 

{\renewcommand{\arraystretch}{1.5}
\begin{tabularx}{\textwidth}{|>{\raggedright\arraybackslash}l|X|}
\hline
\hspace{0pt}\mytexttt{\color{red} } & \textbf{matrix\_t} \\
\hline
\end{tabularx}
}

\par


\rule{\linewidth}{0.5pt}
\subsection*{\textsf{\colorbox{headtoc}{\color{white} ATTRIBUTE}
Triangle}}

\hypertarget{ecldoc:ecldoc-Triangle}{}
\hspace{0pt} \hyperlink{ecldoc:blas}{BLAS} \textbackslash 
\hspace{0pt} \hyperlink{ecldoc:BLAS.Types}{Types} \textbackslash 

{\renewcommand{\arraystretch}{1.5}
\begin{tabularx}{\textwidth}{|>{\raggedright\arraybackslash}l|X|}
\hline
\hspace{0pt}\mytexttt{\color{red} } & \textbf{Triangle} \\
\hline
\end{tabularx}
}

\par


\rule{\linewidth}{0.5pt}
\subsection*{\textsf{\colorbox{headtoc}{\color{white} ATTRIBUTE}
Diagonal}}

\hypertarget{ecldoc:ecldoc-Diagonal}{}
\hspace{0pt} \hyperlink{ecldoc:blas}{BLAS} \textbackslash 
\hspace{0pt} \hyperlink{ecldoc:BLAS.Types}{Types} \textbackslash 

{\renewcommand{\arraystretch}{1.5}
\begin{tabularx}{\textwidth}{|>{\raggedright\arraybackslash}l|X|}
\hline
\hspace{0pt}\mytexttt{\color{red} } & \textbf{Diagonal} \\
\hline
\end{tabularx}
}

\par


\rule{\linewidth}{0.5pt}
\subsection*{\textsf{\colorbox{headtoc}{\color{white} ATTRIBUTE}
Side}}

\hypertarget{ecldoc:ecldoc-Side}{}
\hspace{0pt} \hyperlink{ecldoc:blas}{BLAS} \textbackslash 
\hspace{0pt} \hyperlink{ecldoc:BLAS.Types}{Types} \textbackslash 

{\renewcommand{\arraystretch}{1.5}
\begin{tabularx}{\textwidth}{|>{\raggedright\arraybackslash}l|X|}
\hline
\hspace{0pt}\mytexttt{\color{red} } & \textbf{Side} \\
\hline
\end{tabularx}
}

\par


\rule{\linewidth}{0.5pt}


\subsection*{\textsf{\colorbox{headtoc}{\color{white} FUNCTION}
ICellFunc}}

\hypertarget{ecldoc:blas.icellfunc}{}
\hspace{0pt} \hyperlink{ecldoc:blas}{BLAS} \textbackslash 

{\renewcommand{\arraystretch}{1.5}
\begin{tabularx}{\textwidth}{|>{\raggedright\arraybackslash}l|X|}
\hline
\hspace{0pt}\mytexttt{\color{red} Types.value\_t} & \textbf{ICellFunc} \\
\hline
\multicolumn{2}{|>{\raggedright\arraybackslash}X|}{\hspace{0pt}\mytexttt{\color{param} (Types.value\_t v, Types.dimension\_t r, Types.dimension\_t c)}} \\
\hline
\end{tabularx}
}

\par
Function prototype for Apply2Cell.

\par
\begin{description}
\item [\colorbox{tagtype}{\color{white} \textbf{\textsf{PARAMETER}}}] \textbf{\underline{v}} the value
\item [\colorbox{tagtype}{\color{white} \textbf{\textsf{PARAMETER}}}] \textbf{\underline{r}} the row ordinal
\item [\colorbox{tagtype}{\color{white} \textbf{\textsf{PARAMETER}}}] \textbf{\underline{c}} the column ordinal
\item [\colorbox{tagtype}{\color{white} \textbf{\textsf{RETURN}}}] \textbf{\underline{}} the updated value
\end{description}

\rule{\linewidth}{0.5pt}
\subsection*{\textsf{\colorbox{headtoc}{\color{white} FUNCTION}
Apply2Cells}}

\hypertarget{ecldoc:blas.apply2cells}{}
\hspace{0pt} \hyperlink{ecldoc:blas}{BLAS} \textbackslash 

{\renewcommand{\arraystretch}{1.5}
\begin{tabularx}{\textwidth}{|>{\raggedright\arraybackslash}l|X|}
\hline
\hspace{0pt}\mytexttt{\color{red} Types.matrix\_t} & \textbf{Apply2Cells} \\
\hline
\multicolumn{2}{|>{\raggedright\arraybackslash}X|}{\hspace{0pt}\mytexttt{\color{param} (Types.dimension\_t m, Types.dimension\_t n, Types.matrix\_t x, ICellFunc f)}} \\
\hline
\end{tabularx}
}

\par
Iterate matrix and apply function to each cell

\par
\begin{description}
\item [\colorbox{tagtype}{\color{white} \textbf{\textsf{PARAMETER}}}] \textbf{\underline{m}} number of rows
\item [\colorbox{tagtype}{\color{white} \textbf{\textsf{PARAMETER}}}] \textbf{\underline{n}} number of columns
\item [\colorbox{tagtype}{\color{white} \textbf{\textsf{PARAMETER}}}] \textbf{\underline{x}} matrix
\item [\colorbox{tagtype}{\color{white} \textbf{\textsf{PARAMETER}}}] \textbf{\underline{f}} function to apply
\item [\colorbox{tagtype}{\color{white} \textbf{\textsf{RETURN}}}] \textbf{\underline{}} updated matrix
\end{description}

\rule{\linewidth}{0.5pt}
\subsection*{\textsf{\colorbox{headtoc}{\color{white} FUNCTION}
dasum}}

\hypertarget{ecldoc:blas.dasum}{}
\hspace{0pt} \hyperlink{ecldoc:blas}{BLAS} \textbackslash 

{\renewcommand{\arraystretch}{1.5}
\begin{tabularx}{\textwidth}{|>{\raggedright\arraybackslash}l|X|}
\hline
\hspace{0pt}\mytexttt{\color{red} Types.value\_t} & \textbf{dasum} \\
\hline
\multicolumn{2}{|>{\raggedright\arraybackslash}X|}{\hspace{0pt}\mytexttt{\color{param} (Types.dimension\_t m, Types.matrix\_t x, Types.dimension\_t incx, Types.dimension\_t skipped=0)}} \\
\hline
\end{tabularx}
}

\par
Absolute sum, the 1 norm of a vector.

\par
\begin{description}
\item [\colorbox{tagtype}{\color{white} \textbf{\textsf{PARAMETER}}}] \textbf{\underline{m}} the number of entries
\item [\colorbox{tagtype}{\color{white} \textbf{\textsf{PARAMETER}}}] \textbf{\underline{x}} the column major matrix holding the vector
\item [\colorbox{tagtype}{\color{white} \textbf{\textsf{PARAMETER}}}] \textbf{\underline{incx}} the increment for x, 1 in the case of an actual vector
\item [\colorbox{tagtype}{\color{white} \textbf{\textsf{PARAMETER}}}] \textbf{\underline{skipped}} default is zero, the number of entries stepped over to get to the first entry
\item [\colorbox{tagtype}{\color{white} \textbf{\textsf{RETURN}}}] \textbf{\underline{}} the sum of the absolute values
\end{description}

\rule{\linewidth}{0.5pt}
\subsection*{\textsf{\colorbox{headtoc}{\color{white} FUNCTION}
daxpy}}

\hypertarget{ecldoc:blas.daxpy}{}
\hspace{0pt} \hyperlink{ecldoc:blas}{BLAS} \textbackslash 

{\renewcommand{\arraystretch}{1.5}
\begin{tabularx}{\textwidth}{|>{\raggedright\arraybackslash}l|X|}
\hline
\hspace{0pt}\mytexttt{\color{red} Types.matrix\_t} & \textbf{daxpy} \\
\hline
\multicolumn{2}{|>{\raggedright\arraybackslash}X|}{\hspace{0pt}\mytexttt{\color{param} (Types.dimension\_t N, Types.value\_t alpha, Types.matrix\_t X, Types.dimension\_t incX, Types.matrix\_t Y, Types.dimension\_t incY, Types.dimension\_t x\_skipped=0, Types.dimension\_t y\_skipped=0)}} \\
\hline
\end{tabularx}
}

\par
alpha*X + Y

\par
\begin{description}
\item [\colorbox{tagtype}{\color{white} \textbf{\textsf{PARAMETER}}}] \textbf{\underline{N}} number of elements in vector
\item [\colorbox{tagtype}{\color{white} \textbf{\textsf{PARAMETER}}}] \textbf{\underline{alpha}} the scalar multiplier
\item [\colorbox{tagtype}{\color{white} \textbf{\textsf{PARAMETER}}}] \textbf{\underline{X}} the column major matrix holding the vector X
\item [\colorbox{tagtype}{\color{white} \textbf{\textsf{PARAMETER}}}] \textbf{\underline{incX}} the increment or stride for the vector
\item [\colorbox{tagtype}{\color{white} \textbf{\textsf{PARAMETER}}}] \textbf{\underline{Y}} the column major matrix holding the vector Y
\item [\colorbox{tagtype}{\color{white} \textbf{\textsf{PARAMETER}}}] \textbf{\underline{incY}} the increment or stride of Y
\item [\colorbox{tagtype}{\color{white} \textbf{\textsf{PARAMETER}}}] \textbf{\underline{x\_skipped}} number of entries skipped to get to the first X
\item [\colorbox{tagtype}{\color{white} \textbf{\textsf{PARAMETER}}}] \textbf{\underline{y\_skipped}} number of entries skipped to get to the first Y
\item [\colorbox{tagtype}{\color{white} \textbf{\textsf{RETURN}}}] \textbf{\underline{}} the updated matrix
\end{description}

\rule{\linewidth}{0.5pt}
\subsection*{\textsf{\colorbox{headtoc}{\color{white} FUNCTION}
dgemm}}

\hypertarget{ecldoc:blas.dgemm}{}
\hspace{0pt} \hyperlink{ecldoc:blas}{BLAS} \textbackslash 

{\renewcommand{\arraystretch}{1.5}
\begin{tabularx}{\textwidth}{|>{\raggedright\arraybackslash}l|X|}
\hline
\hspace{0pt}\mytexttt{\color{red} Types.matrix\_t} & \textbf{dgemm} \\
\hline
\multicolumn{2}{|>{\raggedright\arraybackslash}X|}{\hspace{0pt}\mytexttt{\color{param} (BOOLEAN transposeA, BOOLEAN transposeB, Types.dimension\_t M, Types.dimension\_t N, Types.dimension\_t K, Types.value\_t alpha, Types.matrix\_t A, Types.matrix\_t B, Types.value\_t beta=0.0, Types.matrix\_t C=[])}} \\
\hline
\end{tabularx}
}

\par
alpha*op(A) op(B) + beta*C where op() is transpose

\par
\begin{description}
\item [\colorbox{tagtype}{\color{white} \textbf{\textsf{PARAMETER}}}] \textbf{\underline{transposeA}} true when transpose of A is used
\item [\colorbox{tagtype}{\color{white} \textbf{\textsf{PARAMETER}}}] \textbf{\underline{transposeB}} true when transpose of B is used
\item [\colorbox{tagtype}{\color{white} \textbf{\textsf{PARAMETER}}}] \textbf{\underline{M}} number of rows in product
\item [\colorbox{tagtype}{\color{white} \textbf{\textsf{PARAMETER}}}] \textbf{\underline{N}} number of columns in product
\item [\colorbox{tagtype}{\color{white} \textbf{\textsf{PARAMETER}}}] \textbf{\underline{K}} number of columns/rows for the multiplier/multiplicand
\item [\colorbox{tagtype}{\color{white} \textbf{\textsf{PARAMETER}}}] \textbf{\underline{alpha}} scalar used on A
\item [\colorbox{tagtype}{\color{white} \textbf{\textsf{PARAMETER}}}] \textbf{\underline{A}} matrix A
\item [\colorbox{tagtype}{\color{white} \textbf{\textsf{PARAMETER}}}] \textbf{\underline{B}} matrix B
\item [\colorbox{tagtype}{\color{white} \textbf{\textsf{PARAMETER}}}] \textbf{\underline{beta}} scalar for matrix C
\item [\colorbox{tagtype}{\color{white} \textbf{\textsf{PARAMETER}}}] \textbf{\underline{C}} matrix C or empty
\end{description}

\rule{\linewidth}{0.5pt}
\subsection*{\textsf{\colorbox{headtoc}{\color{white} FUNCTION}
dgetf2}}

\hypertarget{ecldoc:blas.dgetf2}{}
\hspace{0pt} \hyperlink{ecldoc:blas}{BLAS} \textbackslash 

{\renewcommand{\arraystretch}{1.5}
\begin{tabularx}{\textwidth}{|>{\raggedright\arraybackslash}l|X|}
\hline
\hspace{0pt}\mytexttt{\color{red} Types.matrix\_t} & \textbf{dgetf2} \\
\hline
\multicolumn{2}{|>{\raggedright\arraybackslash}X|}{\hspace{0pt}\mytexttt{\color{param} (Types.dimension\_t m, Types.dimension\_t n, Types.matrix\_t a)}} \\
\hline
\end{tabularx}
}

\par
Compute LU Factorization of matrix A.

\par
\begin{description}
\item [\colorbox{tagtype}{\color{white} \textbf{\textsf{PARAMETER}}}] \textbf{\underline{m}} number of rows of A
\item [\colorbox{tagtype}{\color{white} \textbf{\textsf{PARAMETER}}}] \textbf{\underline{n}} number of columns of A
\item [\colorbox{tagtype}{\color{white} \textbf{\textsf{RETURN}}}] \textbf{\underline{}} composite matrix of factors, lower triangle has an implied diagonal of ones. Upper triangle has the diagonal of the composite.
\end{description}

\rule{\linewidth}{0.5pt}
\subsection*{\textsf{\colorbox{headtoc}{\color{white} FUNCTION}
dpotf2}}

\hypertarget{ecldoc:blas.dpotf2}{}
\hspace{0pt} \hyperlink{ecldoc:blas}{BLAS} \textbackslash 

{\renewcommand{\arraystretch}{1.5}
\begin{tabularx}{\textwidth}{|>{\raggedright\arraybackslash}l|X|}
\hline
\hspace{0pt}\mytexttt{\color{red} Types.matrix\_t} & \textbf{dpotf2} \\
\hline
\multicolumn{2}{|>{\raggedright\arraybackslash}X|}{\hspace{0pt}\mytexttt{\color{param} (Types.Triangle tri, Types.dimension\_t r, Types.matrix\_t A, BOOLEAN clear=TRUE)}} \\
\hline
\end{tabularx}
}

\par
DPOTF2 computes the Cholesky factorization of a real symmetric positive definite matrix A. The factorization has the form A = U**T * U , if UPLO = 'U', or A = L * L**T, if UPLO = 'L', where U is an upper triangular matrix and L is lower triangular. This is the unblocked version of the algorithm, calling Level 2 BLAS.

\par
\begin{description}
\item [\colorbox{tagtype}{\color{white} \textbf{\textsf{PARAMETER}}}] \textbf{\underline{tri}} indicate whether upper or lower triangle is used
\item [\colorbox{tagtype}{\color{white} \textbf{\textsf{PARAMETER}}}] \textbf{\underline{r}} number of rows/columns in the square matrix
\item [\colorbox{tagtype}{\color{white} \textbf{\textsf{PARAMETER}}}] \textbf{\underline{A}} the square matrix
\item [\colorbox{tagtype}{\color{white} \textbf{\textsf{PARAMETER}}}] \textbf{\underline{clear}} clears the unused triangle
\item [\colorbox{tagtype}{\color{white} \textbf{\textsf{RETURN}}}] \textbf{\underline{}} the triangular matrix requested.
\end{description}

\rule{\linewidth}{0.5pt}
\subsection*{\textsf{\colorbox{headtoc}{\color{white} FUNCTION}
dscal}}

\hypertarget{ecldoc:blas.dscal}{}
\hspace{0pt} \hyperlink{ecldoc:blas}{BLAS} \textbackslash 

{\renewcommand{\arraystretch}{1.5}
\begin{tabularx}{\textwidth}{|>{\raggedright\arraybackslash}l|X|}
\hline
\hspace{0pt}\mytexttt{\color{red} Types.matrix\_t} & \textbf{dscal} \\
\hline
\multicolumn{2}{|>{\raggedright\arraybackslash}X|}{\hspace{0pt}\mytexttt{\color{param} (Types.dimension\_t N, Types.value\_t alpha, Types.matrix\_t X, Types.dimension\_t incX, Types.dimension\_t skipped=0)}} \\
\hline
\end{tabularx}
}

\par
Scale a vector alpha

\par
\begin{description}
\item [\colorbox{tagtype}{\color{white} \textbf{\textsf{PARAMETER}}}] \textbf{\underline{N}} number of elements in the vector
\item [\colorbox{tagtype}{\color{white} \textbf{\textsf{PARAMETER}}}] \textbf{\underline{alpha}} the scaling factor
\item [\colorbox{tagtype}{\color{white} \textbf{\textsf{PARAMETER}}}] \textbf{\underline{X}} the column major matrix holding the vector
\item [\colorbox{tagtype}{\color{white} \textbf{\textsf{PARAMETER}}}] \textbf{\underline{incX}} the stride to get to the next element in the vector
\item [\colorbox{tagtype}{\color{white} \textbf{\textsf{PARAMETER}}}] \textbf{\underline{skipped}} the number of elements skipped to get to the first element
\item [\colorbox{tagtype}{\color{white} \textbf{\textsf{RETURN}}}] \textbf{\underline{}} the updated matrix
\end{description}

\rule{\linewidth}{0.5pt}
\subsection*{\textsf{\colorbox{headtoc}{\color{white} FUNCTION}
dsyrk}}

\hypertarget{ecldoc:blas.dsyrk}{}
\hspace{0pt} \hyperlink{ecldoc:blas}{BLAS} \textbackslash 

{\renewcommand{\arraystretch}{1.5}
\begin{tabularx}{\textwidth}{|>{\raggedright\arraybackslash}l|X|}
\hline
\hspace{0pt}\mytexttt{\color{red} Types.matrix\_t} & \textbf{dsyrk} \\
\hline
\multicolumn{2}{|>{\raggedright\arraybackslash}X|}{\hspace{0pt}\mytexttt{\color{param} (Types.Triangle tri, BOOLEAN transposeA, Types.dimension\_t N, Types.dimension\_t K, Types.value\_t alpha, Types.matrix\_t A, Types.value\_t beta, Types.matrix\_t C, BOOLEAN clear=FALSE)}} \\
\hline
\end{tabularx}
}

\par
Implements symmetric rank update C 

\par
\begin{description}
\item [\colorbox{tagtype}{\color{white} \textbf{\textsf{PARAMETER}}}] \textbf{\underline{tri}} update upper or lower triangle
\item [\colorbox{tagtype}{\color{white} \textbf{\textsf{PARAMETER}}}] \textbf{\underline{transposeA}} Transpose the A matrix to be NxK
\item [\colorbox{tagtype}{\color{white} \textbf{\textsf{PARAMETER}}}] \textbf{\underline{N}} number of rows
\item [\colorbox{tagtype}{\color{white} \textbf{\textsf{PARAMETER}}}] \textbf{\underline{K}} number of columns in the update matrix or transpose
\item [\colorbox{tagtype}{\color{white} \textbf{\textsf{PARAMETER}}}] \textbf{\underline{alpha}} the alpha scalar
\item [\colorbox{tagtype}{\color{white} \textbf{\textsf{PARAMETER}}}] \textbf{\underline{A}} the update matrix, either NxK or KxN
\item [\colorbox{tagtype}{\color{white} \textbf{\textsf{PARAMETER}}}] \textbf{\underline{beta}} the beta scalar
\item [\colorbox{tagtype}{\color{white} \textbf{\textsf{PARAMETER}}}] \textbf{\underline{C}} the matrix to update
\item [\colorbox{tagtype}{\color{white} \textbf{\textsf{PARAMETER}}}] \textbf{\underline{clear}} clear the triangle that is not updated. BLAS assumes that symmetric matrices have only one of the triangles and this option lets you make that true.
\end{description}

\rule{\linewidth}{0.5pt}
\subsection*{\textsf{\colorbox{headtoc}{\color{white} FUNCTION}
dtrsm}}

\hypertarget{ecldoc:blas.dtrsm}{}
\hspace{0pt} \hyperlink{ecldoc:blas}{BLAS} \textbackslash 

{\renewcommand{\arraystretch}{1.5}
\begin{tabularx}{\textwidth}{|>{\raggedright\arraybackslash}l|X|}
\hline
\hspace{0pt}\mytexttt{\color{red} Types.matrix\_t} & \textbf{dtrsm} \\
\hline
\multicolumn{2}{|>{\raggedright\arraybackslash}X|}{\hspace{0pt}\mytexttt{\color{param} (Types.Side side, Types.Triangle tri, BOOLEAN transposeA, Types.Diagonal diag, Types.dimension\_t M, Types.dimension\_t N, Types.dimension\_t lda, Types.value\_t alpha, Types.matrix\_t A, Types.matrix\_t B)}} \\
\hline
\end{tabularx}
}

\par
Triangular matrix solver. op(A) X = alpha B or X op(A) = alpha B where op is Transpose, X and B is MxN

\par
\begin{description}
\item [\colorbox{tagtype}{\color{white} \textbf{\textsf{PARAMETER}}}] \textbf{\underline{side}} side for A, Side.Ax is op(A) X = alpha B
\item [\colorbox{tagtype}{\color{white} \textbf{\textsf{PARAMETER}}}] \textbf{\underline{tri}} Says whether A is Upper or Lower triangle
\item [\colorbox{tagtype}{\color{white} \textbf{\textsf{PARAMETER}}}] \textbf{\underline{transposeA}} is op(A) the transpose of A
\item [\colorbox{tagtype}{\color{white} \textbf{\textsf{PARAMETER}}}] \textbf{\underline{diag}} is the diagonal an implied unit diagonal or supplied
\item [\colorbox{tagtype}{\color{white} \textbf{\textsf{PARAMETER}}}] \textbf{\underline{M}} number of rows
\item [\colorbox{tagtype}{\color{white} \textbf{\textsf{PARAMETER}}}] \textbf{\underline{N}} number of columns
\item [\colorbox{tagtype}{\color{white} \textbf{\textsf{PARAMETER}}}] \textbf{\underline{lda}} the leading dimension of the A matrix, either M or N
\item [\colorbox{tagtype}{\color{white} \textbf{\textsf{PARAMETER}}}] \textbf{\underline{alpha}} the scalar multiplier for B
\item [\colorbox{tagtype}{\color{white} \textbf{\textsf{PARAMETER}}}] \textbf{\underline{A}} a triangular matrix
\item [\colorbox{tagtype}{\color{white} \textbf{\textsf{PARAMETER}}}] \textbf{\underline{B}} the matrix of values for the solve
\item [\colorbox{tagtype}{\color{white} \textbf{\textsf{RETURN}}}] \textbf{\underline{}} the matrix of coefficients to get B.
\end{description}

\rule{\linewidth}{0.5pt}
\subsection*{\textsf{\colorbox{headtoc}{\color{white} FUNCTION}
extract\_diag}}

\hypertarget{ecldoc:blas.extract_diag}{}
\hspace{0pt} \hyperlink{ecldoc:blas}{BLAS} \textbackslash 

{\renewcommand{\arraystretch}{1.5}
\begin{tabularx}{\textwidth}{|>{\raggedright\arraybackslash}l|X|}
\hline
\hspace{0pt}\mytexttt{\color{red} Types.matrix\_t} & \textbf{extract\_diag} \\
\hline
\multicolumn{2}{|>{\raggedright\arraybackslash}X|}{\hspace{0pt}\mytexttt{\color{param} (Types.dimension\_t m, Types.dimension\_t n, Types.matrix\_t x)}} \\
\hline
\end{tabularx}
}

\par
Extract the diagonal of he matrix

\par
\begin{description}
\item [\colorbox{tagtype}{\color{white} \textbf{\textsf{PARAMETER}}}] \textbf{\underline{m}} number of rows
\item [\colorbox{tagtype}{\color{white} \textbf{\textsf{PARAMETER}}}] \textbf{\underline{n}} number of columns
\item [\colorbox{tagtype}{\color{white} \textbf{\textsf{PARAMETER}}}] \textbf{\underline{x}} matrix from which to extract the diagonal
\item [\colorbox{tagtype}{\color{white} \textbf{\textsf{RETURN}}}] \textbf{\underline{}} diagonal matrix
\end{description}

\rule{\linewidth}{0.5pt}
\subsection*{\textsf{\colorbox{headtoc}{\color{white} FUNCTION}
extract\_tri}}

\hypertarget{ecldoc:blas.extract_tri}{}
\hspace{0pt} \hyperlink{ecldoc:blas}{BLAS} \textbackslash 

{\renewcommand{\arraystretch}{1.5}
\begin{tabularx}{\textwidth}{|>{\raggedright\arraybackslash}l|X|}
\hline
\hspace{0pt}\mytexttt{\color{red} Types.matrix\_t} & \textbf{extract\_tri} \\
\hline
\multicolumn{2}{|>{\raggedright\arraybackslash}X|}{\hspace{0pt}\mytexttt{\color{param} (Types.dimension\_t m, Types.dimension\_t n, Types.Triangle tri, Types.Diagonal dt, Types.matrix\_t a)}} \\
\hline
\end{tabularx}
}

\par
Extract the upper or lower triangle. Diagonal can be actual or implied unit diagonal.

\par
\begin{description}
\item [\colorbox{tagtype}{\color{white} \textbf{\textsf{PARAMETER}}}] \textbf{\underline{m}} number of rows
\item [\colorbox{tagtype}{\color{white} \textbf{\textsf{PARAMETER}}}] \textbf{\underline{n}} number of columns
\item [\colorbox{tagtype}{\color{white} \textbf{\textsf{PARAMETER}}}] \textbf{\underline{tri}} Upper or Lower specifier, Triangle.Lower or Triangle.Upper
\item [\colorbox{tagtype}{\color{white} \textbf{\textsf{PARAMETER}}}] \textbf{\underline{dt}} Use Diagonal.NotUnitTri or Diagonal.UnitTri
\item [\colorbox{tagtype}{\color{white} \textbf{\textsf{PARAMETER}}}] \textbf{\underline{a}} Matrix, usually a composite from factoring
\item [\colorbox{tagtype}{\color{white} \textbf{\textsf{RETURN}}}] \textbf{\underline{}} the triangle
\end{description}

\rule{\linewidth}{0.5pt}
\subsection*{\textsf{\colorbox{headtoc}{\color{white} FUNCTION}
make\_diag}}

\hypertarget{ecldoc:blas.make_diag}{}
\hspace{0pt} \hyperlink{ecldoc:blas}{BLAS} \textbackslash 

{\renewcommand{\arraystretch}{1.5}
\begin{tabularx}{\textwidth}{|>{\raggedright\arraybackslash}l|X|}
\hline
\hspace{0pt}\mytexttt{\color{red} Types.matrix\_t} & \textbf{make\_diag} \\
\hline
\multicolumn{2}{|>{\raggedright\arraybackslash}X|}{\hspace{0pt}\mytexttt{\color{param} (Types.dimension\_t m, Types.value\_t v=1.0, Types.matrix\_t X=[])}} \\
\hline
\end{tabularx}
}

\par
Generate a diagonal matrix.

\par
\begin{description}
\item [\colorbox{tagtype}{\color{white} \textbf{\textsf{PARAMETER}}}] \textbf{\underline{m}} number of diagonal entries
\item [\colorbox{tagtype}{\color{white} \textbf{\textsf{PARAMETER}}}] \textbf{\underline{v}} option value, defaults to 1
\item [\colorbox{tagtype}{\color{white} \textbf{\textsf{PARAMETER}}}] \textbf{\underline{X}} optional input of diagonal values, multiplied by v.
\item [\colorbox{tagtype}{\color{white} \textbf{\textsf{RETURN}}}] \textbf{\underline{}} a diagonal matrix
\end{description}

\rule{\linewidth}{0.5pt}
\subsection*{\textsf{\colorbox{headtoc}{\color{white} FUNCTION}
make\_vector}}

\hypertarget{ecldoc:blas.make_vector}{}
\hspace{0pt} \hyperlink{ecldoc:blas}{BLAS} \textbackslash 

{\renewcommand{\arraystretch}{1.5}
\begin{tabularx}{\textwidth}{|>{\raggedright\arraybackslash}l|X|}
\hline
\hspace{0pt}\mytexttt{\color{red} Types.matrix\_t} & \textbf{make\_vector} \\
\hline
\multicolumn{2}{|>{\raggedright\arraybackslash}X|}{\hspace{0pt}\mytexttt{\color{param} (Types.dimension\_t m, Types.value\_t v=1.0)}} \\
\hline
\end{tabularx}
}

\par
Make a vector of dimension m

\par
\begin{description}
\item [\colorbox{tagtype}{\color{white} \textbf{\textsf{PARAMETER}}}] \textbf{\underline{m}} number of elements
\item [\colorbox{tagtype}{\color{white} \textbf{\textsf{PARAMETER}}}] \textbf{\underline{v}} the values, defaults to 1
\item [\colorbox{tagtype}{\color{white} \textbf{\textsf{RETURN}}}] \textbf{\underline{}} the vector
\end{description}

\rule{\linewidth}{0.5pt}
\subsection*{\textsf{\colorbox{headtoc}{\color{white} FUNCTION}
trace}}

\hypertarget{ecldoc:blas.trace}{}
\hspace{0pt} \hyperlink{ecldoc:blas}{BLAS} \textbackslash 

{\renewcommand{\arraystretch}{1.5}
\begin{tabularx}{\textwidth}{|>{\raggedright\arraybackslash}l|X|}
\hline
\hspace{0pt}\mytexttt{\color{red} Types.value\_t} & \textbf{trace} \\
\hline
\multicolumn{2}{|>{\raggedright\arraybackslash}X|}{\hspace{0pt}\mytexttt{\color{param} (Types.dimension\_t m, Types.dimension\_t n, Types.matrix\_t x)}} \\
\hline
\end{tabularx}
}

\par
The trace of the input matrix

\par
\begin{description}
\item [\colorbox{tagtype}{\color{white} \textbf{\textsf{PARAMETER}}}] \textbf{\underline{m}} number of rows
\item [\colorbox{tagtype}{\color{white} \textbf{\textsf{PARAMETER}}}] \textbf{\underline{n}} number of columns
\item [\colorbox{tagtype}{\color{white} \textbf{\textsf{PARAMETER}}}] \textbf{\underline{x}} the matrix
\item [\colorbox{tagtype}{\color{white} \textbf{\textsf{RETURN}}}] \textbf{\underline{}} the trace (sum of the diagonal entries)
\end{description}

\rule{\linewidth}{0.5pt}



\chapter*{\color{headfile}
BundleBase
}
\hypertarget{ecldoc:toc:BundleBase}{}
\hyperlink{ecldoc:toc:root}{Go Up}


\section*{\underline{\textsf{DESCRIPTIONS}}}
\subsection*{\textsf{\colorbox{headtoc}{\color{white} MODULE}
BundleBase}}

\hypertarget{ecldoc:BundleBase}{}

{\renewcommand{\arraystretch}{1.5}
\begin{tabularx}{\textwidth}{|>{\raggedright\arraybackslash}l|X|}
\hline
\hspace{0pt}\mytexttt{\color{red} } & \textbf{BundleBase} \\
\hline
\end{tabularx}
}

\par





No Documentation Found







\textbf{Children}
\begin{enumerate}
\item \hyperlink{ecldoc:bundlebase.propertyrecord}{PropertyRecord}
: No Documentation Found
\item \hyperlink{ecldoc:bundlebase.name}{Name}
: No Documentation Found
\item \hyperlink{ecldoc:bundlebase.description}{Description}
: No Documentation Found
\item \hyperlink{ecldoc:bundlebase.authors}{Authors}
: No Documentation Found
\item \hyperlink{ecldoc:bundlebase.license}{License}
: No Documentation Found
\item \hyperlink{ecldoc:bundlebase.copyright}{Copyright}
: No Documentation Found
\item \hyperlink{ecldoc:bundlebase.dependson}{DependsOn}
: No Documentation Found
\item \hyperlink{ecldoc:bundlebase.version}{Version}
: No Documentation Found
\item \hyperlink{ecldoc:bundlebase.properties}{Properties}
: No Documentation Found
\item \hyperlink{ecldoc:bundlebase.platformversion}{PlatformVersion}
: No Documentation Found
\end{enumerate}

\rule{\linewidth}{0.5pt}

\subsection*{\textsf{\colorbox{headtoc}{\color{white} RECORD}
PropertyRecord}}

\hypertarget{ecldoc:bundlebase.propertyrecord}{}
\hspace{0pt} \hyperlink{ecldoc:BundleBase}{BundleBase} \textbackslash 

{\renewcommand{\arraystretch}{1.5}
\begin{tabularx}{\textwidth}{|>{\raggedright\arraybackslash}l|X|}
\hline
\hspace{0pt}\mytexttt{\color{red} } & \textbf{PropertyRecord} \\
\hline
\end{tabularx}
}

\par





No Documentation Found







\par
\begin{description}
\item [\colorbox{tagtype}{\color{white} \textbf{\textsf{FIELD}}}] \textbf{\underline{value}} ||| UTF8 --- No Doc
\item [\colorbox{tagtype}{\color{white} \textbf{\textsf{FIELD}}}] \textbf{\underline{key}} ||| UTF8 --- No Doc
\end{description}





\rule{\linewidth}{0.5pt}
\subsection*{\textsf{\colorbox{headtoc}{\color{white} ATTRIBUTE}
Name}}

\hypertarget{ecldoc:bundlebase.name}{}
\hspace{0pt} \hyperlink{ecldoc:BundleBase}{BundleBase} \textbackslash 

{\renewcommand{\arraystretch}{1.5}
\begin{tabularx}{\textwidth}{|>{\raggedright\arraybackslash}l|X|}
\hline
\hspace{0pt}\mytexttt{\color{red} STRING} & \textbf{Name} \\
\hline
\end{tabularx}
}

\par





No Documentation Found








\par
\begin{description}
\item [\colorbox{tagtype}{\color{white} \textbf{\textsf{RETURN}}}] \textbf{STRING} --- 
\end{description}




\rule{\linewidth}{0.5pt}
\subsection*{\textsf{\colorbox{headtoc}{\color{white} ATTRIBUTE}
Description}}

\hypertarget{ecldoc:bundlebase.description}{}
\hspace{0pt} \hyperlink{ecldoc:BundleBase}{BundleBase} \textbackslash 

{\renewcommand{\arraystretch}{1.5}
\begin{tabularx}{\textwidth}{|>{\raggedright\arraybackslash}l|X|}
\hline
\hspace{0pt}\mytexttt{\color{red} UTF8} & \textbf{Description} \\
\hline
\end{tabularx}
}

\par





No Documentation Found








\par
\begin{description}
\item [\colorbox{tagtype}{\color{white} \textbf{\textsf{RETURN}}}] \textbf{UTF8} --- 
\end{description}




\rule{\linewidth}{0.5pt}
\subsection*{\textsf{\colorbox{headtoc}{\color{white} ATTRIBUTE}
Authors}}

\hypertarget{ecldoc:bundlebase.authors}{}
\hspace{0pt} \hyperlink{ecldoc:BundleBase}{BundleBase} \textbackslash 

{\renewcommand{\arraystretch}{1.5}
\begin{tabularx}{\textwidth}{|>{\raggedright\arraybackslash}l|X|}
\hline
\hspace{0pt}\mytexttt{\color{red} SET OF UTF8} & \textbf{Authors} \\
\hline
\end{tabularx}
}

\par





No Documentation Found








\par
\begin{description}
\item [\colorbox{tagtype}{\color{white} \textbf{\textsf{RETURN}}}] \textbf{SET ( UTF8 )} --- 
\end{description}




\rule{\linewidth}{0.5pt}
\subsection*{\textsf{\colorbox{headtoc}{\color{white} ATTRIBUTE}
License}}

\hypertarget{ecldoc:bundlebase.license}{}
\hspace{0pt} \hyperlink{ecldoc:BundleBase}{BundleBase} \textbackslash 

{\renewcommand{\arraystretch}{1.5}
\begin{tabularx}{\textwidth}{|>{\raggedright\arraybackslash}l|X|}
\hline
\hspace{0pt}\mytexttt{\color{red} UTF8} & \textbf{License} \\
\hline
\end{tabularx}
}

\par





No Documentation Found








\par
\begin{description}
\item [\colorbox{tagtype}{\color{white} \textbf{\textsf{RETURN}}}] \textbf{UTF8} --- 
\end{description}




\rule{\linewidth}{0.5pt}
\subsection*{\textsf{\colorbox{headtoc}{\color{white} ATTRIBUTE}
Copyright}}

\hypertarget{ecldoc:bundlebase.copyright}{}
\hspace{0pt} \hyperlink{ecldoc:BundleBase}{BundleBase} \textbackslash 

{\renewcommand{\arraystretch}{1.5}
\begin{tabularx}{\textwidth}{|>{\raggedright\arraybackslash}l|X|}
\hline
\hspace{0pt}\mytexttt{\color{red} UTF8} & \textbf{Copyright} \\
\hline
\end{tabularx}
}

\par





No Documentation Found








\par
\begin{description}
\item [\colorbox{tagtype}{\color{white} \textbf{\textsf{RETURN}}}] \textbf{UTF8} --- 
\end{description}




\rule{\linewidth}{0.5pt}
\subsection*{\textsf{\colorbox{headtoc}{\color{white} ATTRIBUTE}
DependsOn}}

\hypertarget{ecldoc:bundlebase.dependson}{}
\hspace{0pt} \hyperlink{ecldoc:BundleBase}{BundleBase} \textbackslash 

{\renewcommand{\arraystretch}{1.5}
\begin{tabularx}{\textwidth}{|>{\raggedright\arraybackslash}l|X|}
\hline
\hspace{0pt}\mytexttt{\color{red} SET OF STRING} & \textbf{DependsOn} \\
\hline
\end{tabularx}
}

\par





No Documentation Found








\par
\begin{description}
\item [\colorbox{tagtype}{\color{white} \textbf{\textsf{RETURN}}}] \textbf{SET ( STRING )} --- 
\end{description}




\rule{\linewidth}{0.5pt}
\subsection*{\textsf{\colorbox{headtoc}{\color{white} ATTRIBUTE}
Version}}

\hypertarget{ecldoc:bundlebase.version}{}
\hspace{0pt} \hyperlink{ecldoc:BundleBase}{BundleBase} \textbackslash 

{\renewcommand{\arraystretch}{1.5}
\begin{tabularx}{\textwidth}{|>{\raggedright\arraybackslash}l|X|}
\hline
\hspace{0pt}\mytexttt{\color{red} STRING} & \textbf{Version} \\
\hline
\end{tabularx}
}

\par





No Documentation Found








\par
\begin{description}
\item [\colorbox{tagtype}{\color{white} \textbf{\textsf{RETURN}}}] \textbf{STRING} --- 
\end{description}




\rule{\linewidth}{0.5pt}
\subsection*{\textsf{\colorbox{headtoc}{\color{white} ATTRIBUTE}
Properties}}

\hypertarget{ecldoc:bundlebase.properties}{}
\hspace{0pt} \hyperlink{ecldoc:BundleBase}{BundleBase} \textbackslash 

{\renewcommand{\arraystretch}{1.5}
\begin{tabularx}{\textwidth}{|>{\raggedright\arraybackslash}l|X|}
\hline
\hspace{0pt}\mytexttt{\color{red} } & \textbf{Properties} \\
\hline
\end{tabularx}
}

\par





No Documentation Found








\par
\begin{description}
\item [\colorbox{tagtype}{\color{white} \textbf{\textsf{RETURN}}}] \textbf{DICTIONARY ( PropertyRecord )} --- 
\end{description}




\rule{\linewidth}{0.5pt}
\subsection*{\textsf{\colorbox{headtoc}{\color{white} ATTRIBUTE}
PlatformVersion}}

\hypertarget{ecldoc:bundlebase.platformversion}{}
\hspace{0pt} \hyperlink{ecldoc:BundleBase}{BundleBase} \textbackslash 

{\renewcommand{\arraystretch}{1.5}
\begin{tabularx}{\textwidth}{|>{\raggedright\arraybackslash}l|X|}
\hline
\hspace{0pt}\mytexttt{\color{red} STRING} & \textbf{PlatformVersion} \\
\hline
\end{tabularx}
}

\par





No Documentation Found








\par
\begin{description}
\item [\colorbox{tagtype}{\color{white} \textbf{\textsf{RETURN}}}] \textbf{STRING} --- 
\end{description}




\rule{\linewidth}{0.5pt}



\chapter*{\color{headfile}
Date
}
\hypertarget{ecldoc:toc:Date}{}
\hyperlink{ecldoc:toc:root}{Go Up}

\section*{\underline{\textsf{IMPORTS}}}
\begin{doublespace}
{\large
}
\end{doublespace}

\section*{\underline{\textsf{DESCRIPTIONS}}}
\subsection*{\textsf{\colorbox{headtoc}{\color{white} MODULE}
Date}}

\hypertarget{ecldoc:Date}{}

{\renewcommand{\arraystretch}{1.5}
\begin{tabularx}{\textwidth}{|>{\raggedright\arraybackslash}l|X|}
\hline
\hspace{0pt}\mytexttt{\color{red} } & \textbf{Date} \\
\hline
\end{tabularx}
}

\par





No Documentation Found







\textbf{Children}
\begin{enumerate}
\item \hyperlink{ecldoc:date.date_rec}{Date\_rec}
: No Documentation Found
\item \hyperlink{ecldoc:date.date_t}{Date\_t}
: No Documentation Found
\item \hyperlink{ecldoc:date.days_t}{Days\_t}
: No Documentation Found
\item \hyperlink{ecldoc:date.time_rec}{Time\_rec}
: No Documentation Found
\item \hyperlink{ecldoc:date.time_t}{Time\_t}
: No Documentation Found
\item \hyperlink{ecldoc:date.seconds_t}{Seconds\_t}
: No Documentation Found
\item \hyperlink{ecldoc:date.datetime_rec}{DateTime\_rec}
: No Documentation Found
\item \hyperlink{ecldoc:date.timestamp_t}{Timestamp\_t}
: No Documentation Found
\item \hyperlink{ecldoc:date.year}{Year}
: Extracts the year from a date type
\item \hyperlink{ecldoc:date.month}{Month}
: Extracts the month from a date type
\item \hyperlink{ecldoc:date.day}{Day}
: Extracts the day of the month from a date type
\item \hyperlink{ecldoc:date.hour}{Hour}
: Extracts the hour from a time type
\item \hyperlink{ecldoc:date.minute}{Minute}
: Extracts the minutes from a time type
\item \hyperlink{ecldoc:date.second}{Second}
: Extracts the seconds from a time type
\item \hyperlink{ecldoc:date.datefromparts}{DateFromParts}
: Combines year, month day to create a date type
\item \hyperlink{ecldoc:date.timefromparts}{TimeFromParts}
: Combines hour, minute second to create a time type
\item \hyperlink{ecldoc:date.secondsfromparts}{SecondsFromParts}
: Combines date and time components to create a seconds type
\item \hyperlink{ecldoc:date.secondstoparts}{SecondsToParts}
: Converts the number of seconds since epoch to a structure containing date and time parts
\item \hyperlink{ecldoc:date.timestamptoseconds}{TimestampToSeconds}
: Converts the number of microseconds since epoch to the number of seconds since epoch
\item \hyperlink{ecldoc:date.isleapyear}{IsLeapYear}
: Tests whether the year is a leap year in the Gregorian calendar
\item \hyperlink{ecldoc:date.isdateleapyear}{IsDateLeapYear}
: Tests whether a date is a leap year in the Gregorian calendar
\item \hyperlink{ecldoc:date.fromgregorianymd}{FromGregorianYMD}
: Combines year, month, day in the Gregorian calendar to create the number days since 31st December 1BC
\item \hyperlink{ecldoc:date.togregorianymd}{ToGregorianYMD}
: Converts the number days since 31st December 1BC to a date in the Gregorian calendar
\item \hyperlink{ecldoc:date.fromgregoriandate}{FromGregorianDate}
: Converts a date in the Gregorian calendar to the number days since 31st December 1BC
\item \hyperlink{ecldoc:date.togregoriandate}{ToGregorianDate}
: Converts the number days since 31st December 1BC to a date in the Gregorian calendar
\item \hyperlink{ecldoc:date.dayofyear}{DayOfYear}
: Returns a number representing the day of the year indicated by the given date
\item \hyperlink{ecldoc:date.dayofweek}{DayOfWeek}
: Returns a number representing the day of the week indicated by the given date
\item \hyperlink{ecldoc:date.isjulianleapyear}{IsJulianLeapYear}
: Tests whether the year is a leap year in the Julian calendar
\item \hyperlink{ecldoc:date.fromjulianymd}{FromJulianYMD}
: Combines year, month, day in the Julian calendar to create the number days since 31st December 1BC
\item \hyperlink{ecldoc:date.tojulianymd}{ToJulianYMD}
: Converts the number days since 31st December 1BC to a date in the Julian calendar
\item \hyperlink{ecldoc:date.fromjuliandate}{FromJulianDate}
: Converts a date in the Julian calendar to the number days since 31st December 1BC
\item \hyperlink{ecldoc:date.tojuliandate}{ToJulianDate}
: Converts the number days since 31st December 1BC to a date in the Julian calendar
\item \hyperlink{ecldoc:date.dayssince1900}{DaysSince1900}
: Returns the number of days since 1st January 1900 (using the Gregorian Calendar)
\item \hyperlink{ecldoc:date.todayssince1900}{ToDaysSince1900}
: Returns the number of days since 1st January 1900 (using the Gregorian Calendar)
\item \hyperlink{ecldoc:date.fromdayssince1900}{FromDaysSince1900}
: Converts the number days since 1st January 1900 to a date in the Julian calendar
\item \hyperlink{ecldoc:date.yearsbetween}{YearsBetween}
: Calculate the number of whole years between two dates
\item \hyperlink{ecldoc:date.monthsbetween}{MonthsBetween}
: Calculate the number of whole months between two dates
\item \hyperlink{ecldoc:date.daysbetween}{DaysBetween}
: Calculate the number of days between two dates
\item \hyperlink{ecldoc:date.datefromdaterec}{DateFromDateRec}
: Combines the fields from a Date\_rec to create a Date\_t
\item \hyperlink{ecldoc:date.datefromrec}{DateFromRec}
: Combines the fields from a Date\_rec to create a Date\_t
\item \hyperlink{ecldoc:date.timefromtimerec}{TimeFromTimeRec}
: Combines the fields from a Time\_rec to create a Time\_t
\item \hyperlink{ecldoc:date.datefromdatetimerec}{DateFromDateTimeRec}
: Combines the date fields from a DateTime\_rec to create a Date\_t
\item \hyperlink{ecldoc:date.timefromdatetimerec}{TimeFromDateTimeRec}
: Combines the time fields from a DateTime\_rec to create a Time\_t
\item \hyperlink{ecldoc:date.secondsfromdatetimerec}{SecondsFromDateTimeRec}
: Combines the date and time fields from a DateTime\_rec to create a Seconds\_t
\item \hyperlink{ecldoc:date.fromstringtodate}{FromStringToDate}
: Converts a string to a Date\_t using the relevant string format
\item \hyperlink{ecldoc:date.fromstring}{FromString}
: Converts a string to a date using the relevant string format
\item \hyperlink{ecldoc:date.fromstringtotime}{FromStringToTime}
: Converts a string to a Time\_t using the relevant string format
\item \hyperlink{ecldoc:date.matchdatestring}{MatchDateString}
: Matches a string against a set of date string formats and returns a valid Date\_t object from the first format that successfully parses the string
\item \hyperlink{ecldoc:date.matchtimestring}{MatchTimeString}
: Matches a string against a set of time string formats and returns a valid Time\_t object from the first format that successfully parses the string
\item \hyperlink{ecldoc:date.datetostring}{DateToString}
: Formats a date as a string
\item \hyperlink{ecldoc:date.timetostring}{TimeToString}
: Formats a time as a string
\item \hyperlink{ecldoc:date.secondstostring}{SecondsToString}
: Converts a Seconds\_t value into a human-readable string using a format template
\item \hyperlink{ecldoc:date.tostring}{ToString}
: Formats a date as a string
\item \hyperlink{ecldoc:date.convertdateformat}{ConvertDateFormat}
: Converts a date from one format to another
\item \hyperlink{ecldoc:date.convertformat}{ConvertFormat}
: Converts a date from one format to another
\item \hyperlink{ecldoc:date.converttimeformat}{ConvertTimeFormat}
: Converts a time from one format to another
\item \hyperlink{ecldoc:date.convertdateformatmultiple}{ConvertDateFormatMultiple}
: Converts a date that matches one of a set of formats to another
\item \hyperlink{ecldoc:date.convertformatmultiple}{ConvertFormatMultiple}
: Converts a date that matches one of a set of formats to another
\item \hyperlink{ecldoc:date.converttimeformatmultiple}{ConvertTimeFormatMultiple}
: Converts a time that matches one of a set of formats to another
\item \hyperlink{ecldoc:date.adjustdate}{AdjustDate}
: Adjusts a date by incrementing or decrementing year, month and/or day values
\item \hyperlink{ecldoc:date.adjustdatebyseconds}{AdjustDateBySeconds}
: Adjusts a date by adding or subtracting seconds
\item \hyperlink{ecldoc:date.adjusttime}{AdjustTime}
: Adjusts a time by incrementing or decrementing hour, minute and/or second values
\item \hyperlink{ecldoc:date.adjusttimebyseconds}{AdjustTimeBySeconds}
: Adjusts a time by adding or subtracting seconds
\item \hyperlink{ecldoc:date.adjustseconds}{AdjustSeconds}
: Adjusts a Seconds\_t value by adding or subtracting years, months, days, hours, minutes and/or seconds
\item \hyperlink{ecldoc:date.adjustcalendar}{AdjustCalendar}
: Adjusts a date by incrementing or decrementing months and/or years
\item \hyperlink{ecldoc:date.islocaldaylightsavingsineffect}{IsLocalDaylightSavingsInEffect}
: Returns a boolean indicating whether daylight savings time is currently in effect locally
\item \hyperlink{ecldoc:date.localtimezoneoffset}{LocalTimeZoneOffset}
: Returns the offset (in seconds) of the time represented from UTC, with positive values indicating locations east of the Prime Meridian
\item \hyperlink{ecldoc:date.currentdate}{CurrentDate}
: Returns the current date
\item \hyperlink{ecldoc:date.today}{Today}
: Returns the current date in the local time zone
\item \hyperlink{ecldoc:date.currenttime}{CurrentTime}
: Returns the current time of day
\item \hyperlink{ecldoc:date.currentseconds}{CurrentSeconds}
: Returns the current date and time as the number of seconds since epoch
\item \hyperlink{ecldoc:date.currenttimestamp}{CurrentTimestamp}
: Returns the current date and time as the number of microseconds since epoch
\item \hyperlink{ecldoc:date.datesformonth}{DatesForMonth}
: Returns the beginning and ending dates for the month surrounding the given date
\item \hyperlink{ecldoc:date.datesforweek}{DatesForWeek}
: Returns the beginning and ending dates for the week surrounding the given date (Sunday marks the beginning of a week)
\item \hyperlink{ecldoc:date.isvaliddate}{IsValidDate}
: Tests whether a date is valid, both by range-checking the year and by validating each of the other individual components
\item \hyperlink{ecldoc:date.isvalidgregoriandate}{IsValidGregorianDate}
: Tests whether a date is valid in the Gregorian calendar
\item \hyperlink{ecldoc:date.isvalidtime}{IsValidTime}
: Tests whether a time is valid
\item \hyperlink{ecldoc:date.createdate}{CreateDate}
: A transform to create a Date\_rec from the individual elements
\item \hyperlink{ecldoc:date.createdatefromseconds}{CreateDateFromSeconds}
: A transform to create a Date\_rec from a Seconds\_t value
\item \hyperlink{ecldoc:date.createtime}{CreateTime}
: A transform to create a Time\_rec from the individual elements
\item \hyperlink{ecldoc:date.createtimefromseconds}{CreateTimeFromSeconds}
: A transform to create a Time\_rec from a Seconds\_t value
\item \hyperlink{ecldoc:date.createdatetime}{CreateDateTime}
: A transform to create a DateTime\_rec from the individual elements
\item \hyperlink{ecldoc:date.createdatetimefromseconds}{CreateDateTimeFromSeconds}
: A transform to create a DateTime\_rec from a Seconds\_t value
\end{enumerate}

\rule{\linewidth}{0.5pt}

\subsection*{\textsf{\colorbox{headtoc}{\color{white} RECORD}
Date\_rec}}

\hypertarget{ecldoc:date.date_rec}{}
\hspace{0pt} \hyperlink{ecldoc:Date}{Date} \textbackslash 

{\renewcommand{\arraystretch}{1.5}
\begin{tabularx}{\textwidth}{|>{\raggedright\arraybackslash}l|X|}
\hline
\hspace{0pt}\mytexttt{\color{red} } & \textbf{Date\_rec} \\
\hline
\end{tabularx}
}

\par





No Documentation Found







\par
\begin{description}
\item [\colorbox{tagtype}{\color{white} \textbf{\textsf{FIELD}}}] \textbf{\underline{year}} ||| INTEGER2 --- No Doc
\item [\colorbox{tagtype}{\color{white} \textbf{\textsf{FIELD}}}] \textbf{\underline{month}} ||| UNSIGNED1 --- No Doc
\item [\colorbox{tagtype}{\color{white} \textbf{\textsf{FIELD}}}] \textbf{\underline{day}} ||| UNSIGNED1 --- No Doc
\end{description}





\rule{\linewidth}{0.5pt}
\subsection*{\textsf{\colorbox{headtoc}{\color{white} ATTRIBUTE}
Date\_t}}

\hypertarget{ecldoc:date.date_t}{}
\hspace{0pt} \hyperlink{ecldoc:Date}{Date} \textbackslash 

{\renewcommand{\arraystretch}{1.5}
\begin{tabularx}{\textwidth}{|>{\raggedright\arraybackslash}l|X|}
\hline
\hspace{0pt}\mytexttt{\color{red} } & \textbf{Date\_t} \\
\hline
\end{tabularx}
}

\par





No Documentation Found








\par
\begin{description}
\item [\colorbox{tagtype}{\color{white} \textbf{\textsf{RETURN}}}] \textbf{UNSIGNED4} --- 
\end{description}




\rule{\linewidth}{0.5pt}
\subsection*{\textsf{\colorbox{headtoc}{\color{white} ATTRIBUTE}
Days\_t}}

\hypertarget{ecldoc:date.days_t}{}
\hspace{0pt} \hyperlink{ecldoc:Date}{Date} \textbackslash 

{\renewcommand{\arraystretch}{1.5}
\begin{tabularx}{\textwidth}{|>{\raggedright\arraybackslash}l|X|}
\hline
\hspace{0pt}\mytexttt{\color{red} } & \textbf{Days\_t} \\
\hline
\end{tabularx}
}

\par





No Documentation Found








\par
\begin{description}
\item [\colorbox{tagtype}{\color{white} \textbf{\textsf{RETURN}}}] \textbf{INTEGER4} --- 
\end{description}




\rule{\linewidth}{0.5pt}
\subsection*{\textsf{\colorbox{headtoc}{\color{white} RECORD}
Time\_rec}}

\hypertarget{ecldoc:date.time_rec}{}
\hspace{0pt} \hyperlink{ecldoc:Date}{Date} \textbackslash 

{\renewcommand{\arraystretch}{1.5}
\begin{tabularx}{\textwidth}{|>{\raggedright\arraybackslash}l|X|}
\hline
\hspace{0pt}\mytexttt{\color{red} } & \textbf{Time\_rec} \\
\hline
\end{tabularx}
}

\par





No Documentation Found







\par
\begin{description}
\item [\colorbox{tagtype}{\color{white} \textbf{\textsf{FIELD}}}] \textbf{\underline{minute}} ||| UNSIGNED1 --- No Doc
\item [\colorbox{tagtype}{\color{white} \textbf{\textsf{FIELD}}}] \textbf{\underline{second}} ||| UNSIGNED1 --- No Doc
\item [\colorbox{tagtype}{\color{white} \textbf{\textsf{FIELD}}}] \textbf{\underline{hour}} ||| UNSIGNED1 --- No Doc
\end{description}





\rule{\linewidth}{0.5pt}
\subsection*{\textsf{\colorbox{headtoc}{\color{white} ATTRIBUTE}
Time\_t}}

\hypertarget{ecldoc:date.time_t}{}
\hspace{0pt} \hyperlink{ecldoc:Date}{Date} \textbackslash 

{\renewcommand{\arraystretch}{1.5}
\begin{tabularx}{\textwidth}{|>{\raggedright\arraybackslash}l|X|}
\hline
\hspace{0pt}\mytexttt{\color{red} } & \textbf{Time\_t} \\
\hline
\end{tabularx}
}

\par





No Documentation Found








\par
\begin{description}
\item [\colorbox{tagtype}{\color{white} \textbf{\textsf{RETURN}}}] \textbf{UNSIGNED3} --- 
\end{description}




\rule{\linewidth}{0.5pt}
\subsection*{\textsf{\colorbox{headtoc}{\color{white} ATTRIBUTE}
Seconds\_t}}

\hypertarget{ecldoc:date.seconds_t}{}
\hspace{0pt} \hyperlink{ecldoc:Date}{Date} \textbackslash 

{\renewcommand{\arraystretch}{1.5}
\begin{tabularx}{\textwidth}{|>{\raggedright\arraybackslash}l|X|}
\hline
\hspace{0pt}\mytexttt{\color{red} } & \textbf{Seconds\_t} \\
\hline
\end{tabularx}
}

\par





No Documentation Found








\par
\begin{description}
\item [\colorbox{tagtype}{\color{white} \textbf{\textsf{RETURN}}}] \textbf{INTEGER8} --- 
\end{description}




\rule{\linewidth}{0.5pt}
\subsection*{\textsf{\colorbox{headtoc}{\color{white} RECORD}
DateTime\_rec}}

\hypertarget{ecldoc:date.datetime_rec}{}
\hspace{0pt} \hyperlink{ecldoc:Date}{Date} \textbackslash 

{\renewcommand{\arraystretch}{1.5}
\begin{tabularx}{\textwidth}{|>{\raggedright\arraybackslash}l|X|}
\hline
\hspace{0pt}\mytexttt{\color{red} } & \textbf{DateTime\_rec} \\
\hline
\end{tabularx}
}

\par





No Documentation Found







\par
\begin{description}
\item [\colorbox{tagtype}{\color{white} \textbf{\textsf{FIELD}}}] \textbf{\underline{year}} ||| INTEGER2 --- No Doc
\item [\colorbox{tagtype}{\color{white} \textbf{\textsf{FIELD}}}] \textbf{\underline{second}} ||| UNSIGNED1 --- No Doc
\item [\colorbox{tagtype}{\color{white} \textbf{\textsf{FIELD}}}] \textbf{\underline{hour}} ||| UNSIGNED1 --- No Doc
\item [\colorbox{tagtype}{\color{white} \textbf{\textsf{FIELD}}}] \textbf{\underline{minute}} ||| UNSIGNED1 --- No Doc
\item [\colorbox{tagtype}{\color{white} \textbf{\textsf{FIELD}}}] \textbf{\underline{month}} ||| UNSIGNED1 --- No Doc
\item [\colorbox{tagtype}{\color{white} \textbf{\textsf{FIELD}}}] \textbf{\underline{day}} ||| UNSIGNED1 --- No Doc
\end{description}





\rule{\linewidth}{0.5pt}
\subsection*{\textsf{\colorbox{headtoc}{\color{white} ATTRIBUTE}
Timestamp\_t}}

\hypertarget{ecldoc:date.timestamp_t}{}
\hspace{0pt} \hyperlink{ecldoc:Date}{Date} \textbackslash 

{\renewcommand{\arraystretch}{1.5}
\begin{tabularx}{\textwidth}{|>{\raggedright\arraybackslash}l|X|}
\hline
\hspace{0pt}\mytexttt{\color{red} } & \textbf{Timestamp\_t} \\
\hline
\end{tabularx}
}

\par





No Documentation Found








\par
\begin{description}
\item [\colorbox{tagtype}{\color{white} \textbf{\textsf{RETURN}}}] \textbf{INTEGER8} --- 
\end{description}




\rule{\linewidth}{0.5pt}
\subsection*{\textsf{\colorbox{headtoc}{\color{white} FUNCTION}
Year}}

\hypertarget{ecldoc:date.year}{}
\hspace{0pt} \hyperlink{ecldoc:Date}{Date} \textbackslash 

{\renewcommand{\arraystretch}{1.5}
\begin{tabularx}{\textwidth}{|>{\raggedright\arraybackslash}l|X|}
\hline
\hspace{0pt}\mytexttt{\color{red} INTEGER2} & \textbf{Year} \\
\hline
\multicolumn{2}{|>{\raggedright\arraybackslash}X|}{\hspace{0pt}\mytexttt{\color{param} (Date\_t date)}} \\
\hline
\end{tabularx}
}

\par





Extracts the year from a date type.






\par
\begin{description}
\item [\colorbox{tagtype}{\color{white} \textbf{\textsf{PARAMETER}}}] \textbf{\underline{date}} ||| UNSIGNED4 --- The date.
\end{description}







\par
\begin{description}
\item [\colorbox{tagtype}{\color{white} \textbf{\textsf{RETURN}}}] \textbf{INTEGER2} --- An integer representing the year.
\end{description}




\rule{\linewidth}{0.5pt}
\subsection*{\textsf{\colorbox{headtoc}{\color{white} FUNCTION}
Month}}

\hypertarget{ecldoc:date.month}{}
\hspace{0pt} \hyperlink{ecldoc:Date}{Date} \textbackslash 

{\renewcommand{\arraystretch}{1.5}
\begin{tabularx}{\textwidth}{|>{\raggedright\arraybackslash}l|X|}
\hline
\hspace{0pt}\mytexttt{\color{red} UNSIGNED1} & \textbf{Month} \\
\hline
\multicolumn{2}{|>{\raggedright\arraybackslash}X|}{\hspace{0pt}\mytexttt{\color{param} (Date\_t date)}} \\
\hline
\end{tabularx}
}

\par





Extracts the month from a date type.






\par
\begin{description}
\item [\colorbox{tagtype}{\color{white} \textbf{\textsf{PARAMETER}}}] \textbf{\underline{date}} ||| UNSIGNED4 --- The date.
\end{description}







\par
\begin{description}
\item [\colorbox{tagtype}{\color{white} \textbf{\textsf{RETURN}}}] \textbf{UNSIGNED1} --- An integer representing the year.
\end{description}




\rule{\linewidth}{0.5pt}
\subsection*{\textsf{\colorbox{headtoc}{\color{white} FUNCTION}
Day}}

\hypertarget{ecldoc:date.day}{}
\hspace{0pt} \hyperlink{ecldoc:Date}{Date} \textbackslash 

{\renewcommand{\arraystretch}{1.5}
\begin{tabularx}{\textwidth}{|>{\raggedright\arraybackslash}l|X|}
\hline
\hspace{0pt}\mytexttt{\color{red} UNSIGNED1} & \textbf{Day} \\
\hline
\multicolumn{2}{|>{\raggedright\arraybackslash}X|}{\hspace{0pt}\mytexttt{\color{param} (Date\_t date)}} \\
\hline
\end{tabularx}
}

\par





Extracts the day of the month from a date type.






\par
\begin{description}
\item [\colorbox{tagtype}{\color{white} \textbf{\textsf{PARAMETER}}}] \textbf{\underline{date}} ||| UNSIGNED4 --- The date.
\end{description}







\par
\begin{description}
\item [\colorbox{tagtype}{\color{white} \textbf{\textsf{RETURN}}}] \textbf{UNSIGNED1} --- An integer representing the year.
\end{description}




\rule{\linewidth}{0.5pt}
\subsection*{\textsf{\colorbox{headtoc}{\color{white} FUNCTION}
Hour}}

\hypertarget{ecldoc:date.hour}{}
\hspace{0pt} \hyperlink{ecldoc:Date}{Date} \textbackslash 

{\renewcommand{\arraystretch}{1.5}
\begin{tabularx}{\textwidth}{|>{\raggedright\arraybackslash}l|X|}
\hline
\hspace{0pt}\mytexttt{\color{red} UNSIGNED1} & \textbf{Hour} \\
\hline
\multicolumn{2}{|>{\raggedright\arraybackslash}X|}{\hspace{0pt}\mytexttt{\color{param} (Time\_t time)}} \\
\hline
\end{tabularx}
}

\par





Extracts the hour from a time type.






\par
\begin{description}
\item [\colorbox{tagtype}{\color{white} \textbf{\textsf{PARAMETER}}}] \textbf{\underline{time}} ||| UNSIGNED3 --- The time.
\end{description}







\par
\begin{description}
\item [\colorbox{tagtype}{\color{white} \textbf{\textsf{RETURN}}}] \textbf{UNSIGNED1} --- An integer representing the hour.
\end{description}




\rule{\linewidth}{0.5pt}
\subsection*{\textsf{\colorbox{headtoc}{\color{white} FUNCTION}
Minute}}

\hypertarget{ecldoc:date.minute}{}
\hspace{0pt} \hyperlink{ecldoc:Date}{Date} \textbackslash 

{\renewcommand{\arraystretch}{1.5}
\begin{tabularx}{\textwidth}{|>{\raggedright\arraybackslash}l|X|}
\hline
\hspace{0pt}\mytexttt{\color{red} UNSIGNED1} & \textbf{Minute} \\
\hline
\multicolumn{2}{|>{\raggedright\arraybackslash}X|}{\hspace{0pt}\mytexttt{\color{param} (Time\_t time)}} \\
\hline
\end{tabularx}
}

\par





Extracts the minutes from a time type.






\par
\begin{description}
\item [\colorbox{tagtype}{\color{white} \textbf{\textsf{PARAMETER}}}] \textbf{\underline{time}} ||| UNSIGNED3 --- The time.
\end{description}







\par
\begin{description}
\item [\colorbox{tagtype}{\color{white} \textbf{\textsf{RETURN}}}] \textbf{UNSIGNED1} --- An integer representing the minutes.
\end{description}




\rule{\linewidth}{0.5pt}
\subsection*{\textsf{\colorbox{headtoc}{\color{white} FUNCTION}
Second}}

\hypertarget{ecldoc:date.second}{}
\hspace{0pt} \hyperlink{ecldoc:Date}{Date} \textbackslash 

{\renewcommand{\arraystretch}{1.5}
\begin{tabularx}{\textwidth}{|>{\raggedright\arraybackslash}l|X|}
\hline
\hspace{0pt}\mytexttt{\color{red} UNSIGNED1} & \textbf{Second} \\
\hline
\multicolumn{2}{|>{\raggedright\arraybackslash}X|}{\hspace{0pt}\mytexttt{\color{param} (Time\_t time)}} \\
\hline
\end{tabularx}
}

\par





Extracts the seconds from a time type.






\par
\begin{description}
\item [\colorbox{tagtype}{\color{white} \textbf{\textsf{PARAMETER}}}] \textbf{\underline{time}} ||| UNSIGNED3 --- The time.
\end{description}







\par
\begin{description}
\item [\colorbox{tagtype}{\color{white} \textbf{\textsf{RETURN}}}] \textbf{UNSIGNED1} --- An integer representing the seconds.
\end{description}




\rule{\linewidth}{0.5pt}
\subsection*{\textsf{\colorbox{headtoc}{\color{white} FUNCTION}
DateFromParts}}

\hypertarget{ecldoc:date.datefromparts}{}
\hspace{0pt} \hyperlink{ecldoc:Date}{Date} \textbackslash 

{\renewcommand{\arraystretch}{1.5}
\begin{tabularx}{\textwidth}{|>{\raggedright\arraybackslash}l|X|}
\hline
\hspace{0pt}\mytexttt{\color{red} Date\_t} & \textbf{DateFromParts} \\
\hline
\multicolumn{2}{|>{\raggedright\arraybackslash}X|}{\hspace{0pt}\mytexttt{\color{param} (INTEGER2 year, UNSIGNED1 month, UNSIGNED1 day)}} \\
\hline
\end{tabularx}
}

\par





Combines year, month day to create a date type.






\par
\begin{description}
\item [\colorbox{tagtype}{\color{white} \textbf{\textsf{PARAMETER}}}] \textbf{\underline{year}} ||| INTEGER2 --- The year (0-9999).
\item [\colorbox{tagtype}{\color{white} \textbf{\textsf{PARAMETER}}}] \textbf{\underline{month}} ||| UNSIGNED1 --- The month (1-12).
\item [\colorbox{tagtype}{\color{white} \textbf{\textsf{PARAMETER}}}] \textbf{\underline{day}} ||| UNSIGNED1 --- The day (1..daysInMonth).
\end{description}







\par
\begin{description}
\item [\colorbox{tagtype}{\color{white} \textbf{\textsf{RETURN}}}] \textbf{UNSIGNED4} --- A date created by combining the fields.
\end{description}




\rule{\linewidth}{0.5pt}
\subsection*{\textsf{\colorbox{headtoc}{\color{white} FUNCTION}
TimeFromParts}}

\hypertarget{ecldoc:date.timefromparts}{}
\hspace{0pt} \hyperlink{ecldoc:Date}{Date} \textbackslash 

{\renewcommand{\arraystretch}{1.5}
\begin{tabularx}{\textwidth}{|>{\raggedright\arraybackslash}l|X|}
\hline
\hspace{0pt}\mytexttt{\color{red} Time\_t} & \textbf{TimeFromParts} \\
\hline
\multicolumn{2}{|>{\raggedright\arraybackslash}X|}{\hspace{0pt}\mytexttt{\color{param} (UNSIGNED1 hour, UNSIGNED1 minute, UNSIGNED1 second)}} \\
\hline
\end{tabularx}
}

\par





Combines hour, minute second to create a time type.






\par
\begin{description}
\item [\colorbox{tagtype}{\color{white} \textbf{\textsf{PARAMETER}}}] \textbf{\underline{minute}} ||| UNSIGNED1 --- The minute (0-59).
\item [\colorbox{tagtype}{\color{white} \textbf{\textsf{PARAMETER}}}] \textbf{\underline{second}} ||| UNSIGNED1 --- The second (0-59).
\item [\colorbox{tagtype}{\color{white} \textbf{\textsf{PARAMETER}}}] \textbf{\underline{hour}} ||| UNSIGNED1 --- The hour (0-23).
\end{description}







\par
\begin{description}
\item [\colorbox{tagtype}{\color{white} \textbf{\textsf{RETURN}}}] \textbf{UNSIGNED3} --- A time created by combining the fields.
\end{description}




\rule{\linewidth}{0.5pt}
\subsection*{\textsf{\colorbox{headtoc}{\color{white} FUNCTION}
SecondsFromParts}}

\hypertarget{ecldoc:date.secondsfromparts}{}
\hspace{0pt} \hyperlink{ecldoc:Date}{Date} \textbackslash 

{\renewcommand{\arraystretch}{1.5}
\begin{tabularx}{\textwidth}{|>{\raggedright\arraybackslash}l|X|}
\hline
\hspace{0pt}\mytexttt{\color{red} Seconds\_t} & \textbf{SecondsFromParts} \\
\hline
\multicolumn{2}{|>{\raggedright\arraybackslash}X|}{\hspace{0pt}\mytexttt{\color{param} (INTEGER2 year, UNSIGNED1 month, UNSIGNED1 day, UNSIGNED1 hour, UNSIGNED1 minute, UNSIGNED1 second, BOOLEAN is\_local\_time = FALSE)}} \\
\hline
\end{tabularx}
}

\par





Combines date and time components to create a seconds type. The date must be represented within the Gregorian calendar after the year 1600.






\par
\begin{description}
\item [\colorbox{tagtype}{\color{white} \textbf{\textsf{PARAMETER}}}] \textbf{\underline{year}} ||| INTEGER2 --- The year (1601-30827).
\item [\colorbox{tagtype}{\color{white} \textbf{\textsf{PARAMETER}}}] \textbf{\underline{second}} ||| UNSIGNED1 --- The second (0-59).
\item [\colorbox{tagtype}{\color{white} \textbf{\textsf{PARAMETER}}}] \textbf{\underline{hour}} ||| UNSIGNED1 --- The hour (0-23).
\item [\colorbox{tagtype}{\color{white} \textbf{\textsf{PARAMETER}}}] \textbf{\underline{minute}} ||| UNSIGNED1 --- The minute (0-59).
\item [\colorbox{tagtype}{\color{white} \textbf{\textsf{PARAMETER}}}] \textbf{\underline{month}} ||| UNSIGNED1 --- The month (1-12).
\item [\colorbox{tagtype}{\color{white} \textbf{\textsf{PARAMETER}}}] \textbf{\underline{day}} ||| UNSIGNED1 --- The day (1..daysInMonth).
\item [\colorbox{tagtype}{\color{white} \textbf{\textsf{PARAMETER}}}] \textbf{\underline{is\_local\_time}} ||| BOOLEAN --- TRUE if the datetime components are expressed in local time rather than UTC, FALSE if the components are expressed in UTC. Optional, defaults to FALSE.
\end{description}







\par
\begin{description}
\item [\colorbox{tagtype}{\color{white} \textbf{\textsf{RETURN}}}] \textbf{INTEGER8} --- A Seconds\_t value created by combining the fields.
\end{description}




\rule{\linewidth}{0.5pt}
\subsection*{\textsf{\colorbox{headtoc}{\color{white} MODULE}
SecondsToParts}}

\hypertarget{ecldoc:date.secondstoparts}{}
\hspace{0pt} \hyperlink{ecldoc:Date}{Date} \textbackslash 

{\renewcommand{\arraystretch}{1.5}
\begin{tabularx}{\textwidth}{|>{\raggedright\arraybackslash}l|X|}
\hline
\hspace{0pt}\mytexttt{\color{red} } & \textbf{SecondsToParts} \\
\hline
\multicolumn{2}{|>{\raggedright\arraybackslash}X|}{\hspace{0pt}\mytexttt{\color{param} (Seconds\_t seconds)}} \\
\hline
\end{tabularx}
}

\par





Converts the number of seconds since epoch to a structure containing date and time parts. The result must be representable within the Gregorian calendar after the year 1600.






\par
\begin{description}
\item [\colorbox{tagtype}{\color{white} \textbf{\textsf{PARAMETER}}}] \textbf{\underline{seconds}} ||| INTEGER8 --- The number of seconds since epoch.
\end{description}







\par
\begin{description}
\item [\colorbox{tagtype}{\color{white} \textbf{\textsf{RETURN}}}] \textbf{} --- Module with exported attributes for year, month, day, hour, minute, second, day\_of\_week, date and time.
\end{description}




\textbf{Children}
\begin{enumerate}
\item \hyperlink{ecldoc:date.secondstoparts.result.year}{Year}
: No Documentation Found
\item \hyperlink{ecldoc:date.secondstoparts.result.month}{Month}
: No Documentation Found
\item \hyperlink{ecldoc:date.secondstoparts.result.day}{Day}
: No Documentation Found
\item \hyperlink{ecldoc:date.secondstoparts.result.hour}{Hour}
: No Documentation Found
\item \hyperlink{ecldoc:date.secondstoparts.result.minute}{Minute}
: No Documentation Found
\item \hyperlink{ecldoc:date.secondstoparts.result.second}{Second}
: No Documentation Found
\item \hyperlink{ecldoc:date.secondstoparts.result.day_of_week}{day\_of\_week}
: No Documentation Found
\item \hyperlink{ecldoc:date.secondstoparts.result.date}{date}
: Combines year, month day to create a date type
\item \hyperlink{ecldoc:date.secondstoparts.result.time}{time}
: Combines hour, minute second to create a time type
\end{enumerate}

\rule{\linewidth}{0.5pt}

\subsection*{\textsf{\colorbox{headtoc}{\color{white} ATTRIBUTE}
Year}}

\hypertarget{ecldoc:date.secondstoparts.result.year}{}
\hspace{0pt} \hyperlink{ecldoc:Date}{Date} \textbackslash 
\hspace{0pt} \hyperlink{ecldoc:date.secondstoparts}{SecondsToParts} \textbackslash 

{\renewcommand{\arraystretch}{1.5}
\begin{tabularx}{\textwidth}{|>{\raggedright\arraybackslash}l|X|}
\hline
\hspace{0pt}\mytexttt{\color{red} INTEGER2} & \textbf{Year} \\
\hline
\end{tabularx}
}

\par





No Documentation Found








\par
\begin{description}
\item [\colorbox{tagtype}{\color{white} \textbf{\textsf{RETURN}}}] \textbf{INTEGER2} --- 
\end{description}




\rule{\linewidth}{0.5pt}
\subsection*{\textsf{\colorbox{headtoc}{\color{white} ATTRIBUTE}
Month}}

\hypertarget{ecldoc:date.secondstoparts.result.month}{}
\hspace{0pt} \hyperlink{ecldoc:Date}{Date} \textbackslash 
\hspace{0pt} \hyperlink{ecldoc:date.secondstoparts}{SecondsToParts} \textbackslash 

{\renewcommand{\arraystretch}{1.5}
\begin{tabularx}{\textwidth}{|>{\raggedright\arraybackslash}l|X|}
\hline
\hspace{0pt}\mytexttt{\color{red} UNSIGNED1} & \textbf{Month} \\
\hline
\end{tabularx}
}

\par





No Documentation Found








\par
\begin{description}
\item [\colorbox{tagtype}{\color{white} \textbf{\textsf{RETURN}}}] \textbf{UNSIGNED1} --- 
\end{description}




\rule{\linewidth}{0.5pt}
\subsection*{\textsf{\colorbox{headtoc}{\color{white} ATTRIBUTE}
Day}}

\hypertarget{ecldoc:date.secondstoparts.result.day}{}
\hspace{0pt} \hyperlink{ecldoc:Date}{Date} \textbackslash 
\hspace{0pt} \hyperlink{ecldoc:date.secondstoparts}{SecondsToParts} \textbackslash 

{\renewcommand{\arraystretch}{1.5}
\begin{tabularx}{\textwidth}{|>{\raggedright\arraybackslash}l|X|}
\hline
\hspace{0pt}\mytexttt{\color{red} UNSIGNED1} & \textbf{Day} \\
\hline
\end{tabularx}
}

\par





No Documentation Found








\par
\begin{description}
\item [\colorbox{tagtype}{\color{white} \textbf{\textsf{RETURN}}}] \textbf{UNSIGNED1} --- 
\end{description}




\rule{\linewidth}{0.5pt}
\subsection*{\textsf{\colorbox{headtoc}{\color{white} ATTRIBUTE}
Hour}}

\hypertarget{ecldoc:date.secondstoparts.result.hour}{}
\hspace{0pt} \hyperlink{ecldoc:Date}{Date} \textbackslash 
\hspace{0pt} \hyperlink{ecldoc:date.secondstoparts}{SecondsToParts} \textbackslash 

{\renewcommand{\arraystretch}{1.5}
\begin{tabularx}{\textwidth}{|>{\raggedright\arraybackslash}l|X|}
\hline
\hspace{0pt}\mytexttt{\color{red} UNSIGNED1} & \textbf{Hour} \\
\hline
\end{tabularx}
}

\par





No Documentation Found








\par
\begin{description}
\item [\colorbox{tagtype}{\color{white} \textbf{\textsf{RETURN}}}] \textbf{UNSIGNED1} --- 
\end{description}




\rule{\linewidth}{0.5pt}
\subsection*{\textsf{\colorbox{headtoc}{\color{white} ATTRIBUTE}
Minute}}

\hypertarget{ecldoc:date.secondstoparts.result.minute}{}
\hspace{0pt} \hyperlink{ecldoc:Date}{Date} \textbackslash 
\hspace{0pt} \hyperlink{ecldoc:date.secondstoparts}{SecondsToParts} \textbackslash 

{\renewcommand{\arraystretch}{1.5}
\begin{tabularx}{\textwidth}{|>{\raggedright\arraybackslash}l|X|}
\hline
\hspace{0pt}\mytexttt{\color{red} UNSIGNED1} & \textbf{Minute} \\
\hline
\end{tabularx}
}

\par





No Documentation Found








\par
\begin{description}
\item [\colorbox{tagtype}{\color{white} \textbf{\textsf{RETURN}}}] \textbf{UNSIGNED1} --- 
\end{description}




\rule{\linewidth}{0.5pt}
\subsection*{\textsf{\colorbox{headtoc}{\color{white} ATTRIBUTE}
Second}}

\hypertarget{ecldoc:date.secondstoparts.result.second}{}
\hspace{0pt} \hyperlink{ecldoc:Date}{Date} \textbackslash 
\hspace{0pt} \hyperlink{ecldoc:date.secondstoparts}{SecondsToParts} \textbackslash 

{\renewcommand{\arraystretch}{1.5}
\begin{tabularx}{\textwidth}{|>{\raggedright\arraybackslash}l|X|}
\hline
\hspace{0pt}\mytexttt{\color{red} UNSIGNED1} & \textbf{Second} \\
\hline
\end{tabularx}
}

\par





No Documentation Found








\par
\begin{description}
\item [\colorbox{tagtype}{\color{white} \textbf{\textsf{RETURN}}}] \textbf{UNSIGNED1} --- 
\end{description}




\rule{\linewidth}{0.5pt}
\subsection*{\textsf{\colorbox{headtoc}{\color{white} ATTRIBUTE}
day\_of\_week}}

\hypertarget{ecldoc:date.secondstoparts.result.day_of_week}{}
\hspace{0pt} \hyperlink{ecldoc:Date}{Date} \textbackslash 
\hspace{0pt} \hyperlink{ecldoc:date.secondstoparts}{SecondsToParts} \textbackslash 

{\renewcommand{\arraystretch}{1.5}
\begin{tabularx}{\textwidth}{|>{\raggedright\arraybackslash}l|X|}
\hline
\hspace{0pt}\mytexttt{\color{red} UNSIGNED1} & \textbf{day\_of\_week} \\
\hline
\end{tabularx}
}

\par





No Documentation Found








\par
\begin{description}
\item [\colorbox{tagtype}{\color{white} \textbf{\textsf{RETURN}}}] \textbf{UNSIGNED1} --- 
\end{description}




\rule{\linewidth}{0.5pt}
\subsection*{\textsf{\colorbox{headtoc}{\color{white} ATTRIBUTE}
date}}

\hypertarget{ecldoc:date.secondstoparts.result.date}{}
\hspace{0pt} \hyperlink{ecldoc:Date}{Date} \textbackslash 
\hspace{0pt} \hyperlink{ecldoc:date.secondstoparts}{SecondsToParts} \textbackslash 

{\renewcommand{\arraystretch}{1.5}
\begin{tabularx}{\textwidth}{|>{\raggedright\arraybackslash}l|X|}
\hline
\hspace{0pt}\mytexttt{\color{red} Date\_t} & \textbf{date} \\
\hline
\end{tabularx}
}

\par





Combines year, month day to create a date type.






\par
\begin{description}
\item [\colorbox{tagtype}{\color{white} \textbf{\textsf{PARAMETER}}}] \textbf{\underline{year}} |||  --- The year (0-9999).
\item [\colorbox{tagtype}{\color{white} \textbf{\textsf{PARAMETER}}}] \textbf{\underline{month}} |||  --- The month (1-12).
\item [\colorbox{tagtype}{\color{white} \textbf{\textsf{PARAMETER}}}] \textbf{\underline{day}} |||  --- The day (1..daysInMonth).
\end{description}







\par
\begin{description}
\item [\colorbox{tagtype}{\color{white} \textbf{\textsf{RETURN}}}] \textbf{UNSIGNED4} --- A date created by combining the fields.
\end{description}




\rule{\linewidth}{0.5pt}
\subsection*{\textsf{\colorbox{headtoc}{\color{white} ATTRIBUTE}
time}}

\hypertarget{ecldoc:date.secondstoparts.result.time}{}
\hspace{0pt} \hyperlink{ecldoc:Date}{Date} \textbackslash 
\hspace{0pt} \hyperlink{ecldoc:date.secondstoparts}{SecondsToParts} \textbackslash 

{\renewcommand{\arraystretch}{1.5}
\begin{tabularx}{\textwidth}{|>{\raggedright\arraybackslash}l|X|}
\hline
\hspace{0pt}\mytexttt{\color{red} Time\_t} & \textbf{time} \\
\hline
\end{tabularx}
}

\par





Combines hour, minute second to create a time type.






\par
\begin{description}
\item [\colorbox{tagtype}{\color{white} \textbf{\textsf{PARAMETER}}}] \textbf{\underline{minute}} |||  --- The minute (0-59).
\item [\colorbox{tagtype}{\color{white} \textbf{\textsf{PARAMETER}}}] \textbf{\underline{second}} |||  --- The second (0-59).
\item [\colorbox{tagtype}{\color{white} \textbf{\textsf{PARAMETER}}}] \textbf{\underline{hour}} |||  --- The hour (0-23).
\end{description}







\par
\begin{description}
\item [\colorbox{tagtype}{\color{white} \textbf{\textsf{RETURN}}}] \textbf{UNSIGNED3} --- A time created by combining the fields.
\end{description}




\rule{\linewidth}{0.5pt}


\subsection*{\textsf{\colorbox{headtoc}{\color{white} FUNCTION}
TimestampToSeconds}}

\hypertarget{ecldoc:date.timestamptoseconds}{}
\hspace{0pt} \hyperlink{ecldoc:Date}{Date} \textbackslash 

{\renewcommand{\arraystretch}{1.5}
\begin{tabularx}{\textwidth}{|>{\raggedright\arraybackslash}l|X|}
\hline
\hspace{0pt}\mytexttt{\color{red} Seconds\_t} & \textbf{TimestampToSeconds} \\
\hline
\multicolumn{2}{|>{\raggedright\arraybackslash}X|}{\hspace{0pt}\mytexttt{\color{param} (Timestamp\_t timestamp)}} \\
\hline
\end{tabularx}
}

\par





Converts the number of microseconds since epoch to the number of seconds since epoch.






\par
\begin{description}
\item [\colorbox{tagtype}{\color{white} \textbf{\textsf{PARAMETER}}}] \textbf{\underline{timestamp}} ||| INTEGER8 --- The number of microseconds since epoch.
\end{description}







\par
\begin{description}
\item [\colorbox{tagtype}{\color{white} \textbf{\textsf{RETURN}}}] \textbf{INTEGER8} --- The number of seconds since epoch.
\end{description}




\rule{\linewidth}{0.5pt}
\subsection*{\textsf{\colorbox{headtoc}{\color{white} FUNCTION}
IsLeapYear}}

\hypertarget{ecldoc:date.isleapyear}{}
\hspace{0pt} \hyperlink{ecldoc:Date}{Date} \textbackslash 

{\renewcommand{\arraystretch}{1.5}
\begin{tabularx}{\textwidth}{|>{\raggedright\arraybackslash}l|X|}
\hline
\hspace{0pt}\mytexttt{\color{red} BOOLEAN} & \textbf{IsLeapYear} \\
\hline
\multicolumn{2}{|>{\raggedright\arraybackslash}X|}{\hspace{0pt}\mytexttt{\color{param} (INTEGER2 year)}} \\
\hline
\end{tabularx}
}

\par





Tests whether the year is a leap year in the Gregorian calendar.






\par
\begin{description}
\item [\colorbox{tagtype}{\color{white} \textbf{\textsf{PARAMETER}}}] \textbf{\underline{year}} ||| INTEGER2 --- The year (0-9999).
\end{description}







\par
\begin{description}
\item [\colorbox{tagtype}{\color{white} \textbf{\textsf{RETURN}}}] \textbf{BOOLEAN} --- True if the year is a leap year.
\end{description}




\rule{\linewidth}{0.5pt}
\subsection*{\textsf{\colorbox{headtoc}{\color{white} FUNCTION}
IsDateLeapYear}}

\hypertarget{ecldoc:date.isdateleapyear}{}
\hspace{0pt} \hyperlink{ecldoc:Date}{Date} \textbackslash 

{\renewcommand{\arraystretch}{1.5}
\begin{tabularx}{\textwidth}{|>{\raggedright\arraybackslash}l|X|}
\hline
\hspace{0pt}\mytexttt{\color{red} BOOLEAN} & \textbf{IsDateLeapYear} \\
\hline
\multicolumn{2}{|>{\raggedright\arraybackslash}X|}{\hspace{0pt}\mytexttt{\color{param} (Date\_t date)}} \\
\hline
\end{tabularx}
}

\par





Tests whether a date is a leap year in the Gregorian calendar.






\par
\begin{description}
\item [\colorbox{tagtype}{\color{white} \textbf{\textsf{PARAMETER}}}] \textbf{\underline{date}} ||| UNSIGNED4 --- The date.
\end{description}







\par
\begin{description}
\item [\colorbox{tagtype}{\color{white} \textbf{\textsf{RETURN}}}] \textbf{BOOLEAN} --- True if the year is a leap year.
\end{description}




\rule{\linewidth}{0.5pt}
\subsection*{\textsf{\colorbox{headtoc}{\color{white} FUNCTION}
FromGregorianYMD}}

\hypertarget{ecldoc:date.fromgregorianymd}{}
\hspace{0pt} \hyperlink{ecldoc:Date}{Date} \textbackslash 

{\renewcommand{\arraystretch}{1.5}
\begin{tabularx}{\textwidth}{|>{\raggedright\arraybackslash}l|X|}
\hline
\hspace{0pt}\mytexttt{\color{red} Days\_t} & \textbf{FromGregorianYMD} \\
\hline
\multicolumn{2}{|>{\raggedright\arraybackslash}X|}{\hspace{0pt}\mytexttt{\color{param} (INTEGER2 year, UNSIGNED1 month, UNSIGNED1 day)}} \\
\hline
\end{tabularx}
}

\par





Combines year, month, day in the Gregorian calendar to create the number days since 31st December 1BC.






\par
\begin{description}
\item [\colorbox{tagtype}{\color{white} \textbf{\textsf{PARAMETER}}}] \textbf{\underline{year}} ||| INTEGER2 --- The year (-4713..9999).
\item [\colorbox{tagtype}{\color{white} \textbf{\textsf{PARAMETER}}}] \textbf{\underline{month}} ||| UNSIGNED1 --- The month (1-12). A missing value (0) is treated as 1.
\item [\colorbox{tagtype}{\color{white} \textbf{\textsf{PARAMETER}}}] \textbf{\underline{day}} ||| UNSIGNED1 --- The day (1..daysInMonth). A missing value (0) is treated as 1.
\end{description}







\par
\begin{description}
\item [\colorbox{tagtype}{\color{white} \textbf{\textsf{RETURN}}}] \textbf{INTEGER4} --- The number of elapsed days (1 Jan 1AD = 1)
\end{description}




\rule{\linewidth}{0.5pt}
\subsection*{\textsf{\colorbox{headtoc}{\color{white} MODULE}
ToGregorianYMD}}

\hypertarget{ecldoc:date.togregorianymd}{}
\hspace{0pt} \hyperlink{ecldoc:Date}{Date} \textbackslash 

{\renewcommand{\arraystretch}{1.5}
\begin{tabularx}{\textwidth}{|>{\raggedright\arraybackslash}l|X|}
\hline
\hspace{0pt}\mytexttt{\color{red} } & \textbf{ToGregorianYMD} \\
\hline
\multicolumn{2}{|>{\raggedright\arraybackslash}X|}{\hspace{0pt}\mytexttt{\color{param} (Days\_t days)}} \\
\hline
\end{tabularx}
}

\par





Converts the number days since 31st December 1BC to a date in the Gregorian calendar.






\par
\begin{description}
\item [\colorbox{tagtype}{\color{white} \textbf{\textsf{PARAMETER}}}] \textbf{\underline{days}} ||| INTEGER4 --- The number of elapsed days (1 Jan 1AD = 1)
\end{description}







\par
\begin{description}
\item [\colorbox{tagtype}{\color{white} \textbf{\textsf{RETURN}}}] \textbf{} --- Module containing Year, Month, Day in the Gregorian calendar
\end{description}




\textbf{Children}
\begin{enumerate}
\item \hyperlink{ecldoc:date.togregorianymd.result.year}{year}
: No Documentation Found
\item \hyperlink{ecldoc:date.togregorianymd.result.month}{month}
: No Documentation Found
\item \hyperlink{ecldoc:date.togregorianymd.result.day}{day}
: No Documentation Found
\end{enumerate}

\rule{\linewidth}{0.5pt}

\subsection*{\textsf{\colorbox{headtoc}{\color{white} ATTRIBUTE}
year}}

\hypertarget{ecldoc:date.togregorianymd.result.year}{}
\hspace{0pt} \hyperlink{ecldoc:Date}{Date} \textbackslash 
\hspace{0pt} \hyperlink{ecldoc:date.togregorianymd}{ToGregorianYMD} \textbackslash 

{\renewcommand{\arraystretch}{1.5}
\begin{tabularx}{\textwidth}{|>{\raggedright\arraybackslash}l|X|}
\hline
\hspace{0pt}\mytexttt{\color{red} } & \textbf{year} \\
\hline
\end{tabularx}
}

\par





No Documentation Found








\par
\begin{description}
\item [\colorbox{tagtype}{\color{white} \textbf{\textsf{RETURN}}}] \textbf{INTEGER8} --- 
\end{description}




\rule{\linewidth}{0.5pt}
\subsection*{\textsf{\colorbox{headtoc}{\color{white} ATTRIBUTE}
month}}

\hypertarget{ecldoc:date.togregorianymd.result.month}{}
\hspace{0pt} \hyperlink{ecldoc:Date}{Date} \textbackslash 
\hspace{0pt} \hyperlink{ecldoc:date.togregorianymd}{ToGregorianYMD} \textbackslash 

{\renewcommand{\arraystretch}{1.5}
\begin{tabularx}{\textwidth}{|>{\raggedright\arraybackslash}l|X|}
\hline
\hspace{0pt}\mytexttt{\color{red} } & \textbf{month} \\
\hline
\end{tabularx}
}

\par





No Documentation Found








\par
\begin{description}
\item [\colorbox{tagtype}{\color{white} \textbf{\textsf{RETURN}}}] \textbf{INTEGER8} --- 
\end{description}




\rule{\linewidth}{0.5pt}
\subsection*{\textsf{\colorbox{headtoc}{\color{white} ATTRIBUTE}
day}}

\hypertarget{ecldoc:date.togregorianymd.result.day}{}
\hspace{0pt} \hyperlink{ecldoc:Date}{Date} \textbackslash 
\hspace{0pt} \hyperlink{ecldoc:date.togregorianymd}{ToGregorianYMD} \textbackslash 

{\renewcommand{\arraystretch}{1.5}
\begin{tabularx}{\textwidth}{|>{\raggedright\arraybackslash}l|X|}
\hline
\hspace{0pt}\mytexttt{\color{red} } & \textbf{day} \\
\hline
\end{tabularx}
}

\par





No Documentation Found








\par
\begin{description}
\item [\colorbox{tagtype}{\color{white} \textbf{\textsf{RETURN}}}] \textbf{INTEGER8} --- 
\end{description}




\rule{\linewidth}{0.5pt}


\subsection*{\textsf{\colorbox{headtoc}{\color{white} FUNCTION}
FromGregorianDate}}

\hypertarget{ecldoc:date.fromgregoriandate}{}
\hspace{0pt} \hyperlink{ecldoc:Date}{Date} \textbackslash 

{\renewcommand{\arraystretch}{1.5}
\begin{tabularx}{\textwidth}{|>{\raggedright\arraybackslash}l|X|}
\hline
\hspace{0pt}\mytexttt{\color{red} Days\_t} & \textbf{FromGregorianDate} \\
\hline
\multicolumn{2}{|>{\raggedright\arraybackslash}X|}{\hspace{0pt}\mytexttt{\color{param} (Date\_t date)}} \\
\hline
\end{tabularx}
}

\par





Converts a date in the Gregorian calendar to the number days since 31st December 1BC.






\par
\begin{description}
\item [\colorbox{tagtype}{\color{white} \textbf{\textsf{PARAMETER}}}] \textbf{\underline{date}} ||| UNSIGNED4 --- The date (using the Gregorian calendar)
\end{description}







\par
\begin{description}
\item [\colorbox{tagtype}{\color{white} \textbf{\textsf{RETURN}}}] \textbf{INTEGER4} --- The number of elapsed days (1 Jan 1AD = 1)
\end{description}




\rule{\linewidth}{0.5pt}
\subsection*{\textsf{\colorbox{headtoc}{\color{white} FUNCTION}
ToGregorianDate}}

\hypertarget{ecldoc:date.togregoriandate}{}
\hspace{0pt} \hyperlink{ecldoc:Date}{Date} \textbackslash 

{\renewcommand{\arraystretch}{1.5}
\begin{tabularx}{\textwidth}{|>{\raggedright\arraybackslash}l|X|}
\hline
\hspace{0pt}\mytexttt{\color{red} Date\_t} & \textbf{ToGregorianDate} \\
\hline
\multicolumn{2}{|>{\raggedright\arraybackslash}X|}{\hspace{0pt}\mytexttt{\color{param} (Days\_t days)}} \\
\hline
\end{tabularx}
}

\par





Converts the number days since 31st December 1BC to a date in the Gregorian calendar.






\par
\begin{description}
\item [\colorbox{tagtype}{\color{white} \textbf{\textsf{PARAMETER}}}] \textbf{\underline{days}} ||| INTEGER4 --- The number of elapsed days (1 Jan 1AD = 1)
\end{description}







\par
\begin{description}
\item [\colorbox{tagtype}{\color{white} \textbf{\textsf{RETURN}}}] \textbf{UNSIGNED4} --- A Date\_t in the Gregorian calendar
\end{description}




\rule{\linewidth}{0.5pt}
\subsection*{\textsf{\colorbox{headtoc}{\color{white} FUNCTION}
DayOfYear}}

\hypertarget{ecldoc:date.dayofyear}{}
\hspace{0pt} \hyperlink{ecldoc:Date}{Date} \textbackslash 

{\renewcommand{\arraystretch}{1.5}
\begin{tabularx}{\textwidth}{|>{\raggedright\arraybackslash}l|X|}
\hline
\hspace{0pt}\mytexttt{\color{red} UNSIGNED2} & \textbf{DayOfYear} \\
\hline
\multicolumn{2}{|>{\raggedright\arraybackslash}X|}{\hspace{0pt}\mytexttt{\color{param} (Date\_t date)}} \\
\hline
\end{tabularx}
}

\par





Returns a number representing the day of the year indicated by the given date. The date must be in the Gregorian calendar after the year 1600.






\par
\begin{description}
\item [\colorbox{tagtype}{\color{white} \textbf{\textsf{PARAMETER}}}] \textbf{\underline{date}} ||| UNSIGNED4 --- A Date\_t value.
\end{description}







\par
\begin{description}
\item [\colorbox{tagtype}{\color{white} \textbf{\textsf{RETURN}}}] \textbf{UNSIGNED2} --- A number (1-366) representing the number of days since the beginning of the year.
\end{description}




\rule{\linewidth}{0.5pt}
\subsection*{\textsf{\colorbox{headtoc}{\color{white} FUNCTION}
DayOfWeek}}

\hypertarget{ecldoc:date.dayofweek}{}
\hspace{0pt} \hyperlink{ecldoc:Date}{Date} \textbackslash 

{\renewcommand{\arraystretch}{1.5}
\begin{tabularx}{\textwidth}{|>{\raggedright\arraybackslash}l|X|}
\hline
\hspace{0pt}\mytexttt{\color{red} UNSIGNED1} & \textbf{DayOfWeek} \\
\hline
\multicolumn{2}{|>{\raggedright\arraybackslash}X|}{\hspace{0pt}\mytexttt{\color{param} (Date\_t date)}} \\
\hline
\end{tabularx}
}

\par





Returns a number representing the day of the week indicated by the given date. The date must be in the Gregorian calendar after the year 1600.






\par
\begin{description}
\item [\colorbox{tagtype}{\color{white} \textbf{\textsf{PARAMETER}}}] \textbf{\underline{date}} ||| UNSIGNED4 --- A Date\_t value.
\end{description}







\par
\begin{description}
\item [\colorbox{tagtype}{\color{white} \textbf{\textsf{RETURN}}}] \textbf{UNSIGNED1} --- A number 1-7 representing the day of the week, where 1 = Sunday.
\end{description}




\rule{\linewidth}{0.5pt}
\subsection*{\textsf{\colorbox{headtoc}{\color{white} FUNCTION}
IsJulianLeapYear}}

\hypertarget{ecldoc:date.isjulianleapyear}{}
\hspace{0pt} \hyperlink{ecldoc:Date}{Date} \textbackslash 

{\renewcommand{\arraystretch}{1.5}
\begin{tabularx}{\textwidth}{|>{\raggedright\arraybackslash}l|X|}
\hline
\hspace{0pt}\mytexttt{\color{red} BOOLEAN} & \textbf{IsJulianLeapYear} \\
\hline
\multicolumn{2}{|>{\raggedright\arraybackslash}X|}{\hspace{0pt}\mytexttt{\color{param} (INTEGER2 year)}} \\
\hline
\end{tabularx}
}

\par





Tests whether the year is a leap year in the Julian calendar.






\par
\begin{description}
\item [\colorbox{tagtype}{\color{white} \textbf{\textsf{PARAMETER}}}] \textbf{\underline{year}} ||| INTEGER2 --- The year (0-9999).
\end{description}







\par
\begin{description}
\item [\colorbox{tagtype}{\color{white} \textbf{\textsf{RETURN}}}] \textbf{BOOLEAN} --- True if the year is a leap year.
\end{description}




\rule{\linewidth}{0.5pt}
\subsection*{\textsf{\colorbox{headtoc}{\color{white} FUNCTION}
FromJulianYMD}}

\hypertarget{ecldoc:date.fromjulianymd}{}
\hspace{0pt} \hyperlink{ecldoc:Date}{Date} \textbackslash 

{\renewcommand{\arraystretch}{1.5}
\begin{tabularx}{\textwidth}{|>{\raggedright\arraybackslash}l|X|}
\hline
\hspace{0pt}\mytexttt{\color{red} Days\_t} & \textbf{FromJulianYMD} \\
\hline
\multicolumn{2}{|>{\raggedright\arraybackslash}X|}{\hspace{0pt}\mytexttt{\color{param} (INTEGER2 year, UNSIGNED1 month, UNSIGNED1 day)}} \\
\hline
\end{tabularx}
}

\par





Combines year, month, day in the Julian calendar to create the number days since 31st December 1BC.






\par
\begin{description}
\item [\colorbox{tagtype}{\color{white} \textbf{\textsf{PARAMETER}}}] \textbf{\underline{year}} ||| INTEGER2 --- The year (-4800..9999).
\item [\colorbox{tagtype}{\color{white} \textbf{\textsf{PARAMETER}}}] \textbf{\underline{month}} ||| UNSIGNED1 --- The month (1-12).
\item [\colorbox{tagtype}{\color{white} \textbf{\textsf{PARAMETER}}}] \textbf{\underline{day}} ||| UNSIGNED1 --- The day (1..daysInMonth).
\end{description}







\par
\begin{description}
\item [\colorbox{tagtype}{\color{white} \textbf{\textsf{RETURN}}}] \textbf{INTEGER4} --- The number of elapsed days (1 Jan 1AD = 1)
\end{description}




\rule{\linewidth}{0.5pt}
\subsection*{\textsf{\colorbox{headtoc}{\color{white} MODULE}
ToJulianYMD}}

\hypertarget{ecldoc:date.tojulianymd}{}
\hspace{0pt} \hyperlink{ecldoc:Date}{Date} \textbackslash 

{\renewcommand{\arraystretch}{1.5}
\begin{tabularx}{\textwidth}{|>{\raggedright\arraybackslash}l|X|}
\hline
\hspace{0pt}\mytexttt{\color{red} } & \textbf{ToJulianYMD} \\
\hline
\multicolumn{2}{|>{\raggedright\arraybackslash}X|}{\hspace{0pt}\mytexttt{\color{param} (Days\_t days)}} \\
\hline
\end{tabularx}
}

\par





Converts the number days since 31st December 1BC to a date in the Julian calendar.






\par
\begin{description}
\item [\colorbox{tagtype}{\color{white} \textbf{\textsf{PARAMETER}}}] \textbf{\underline{days}} ||| INTEGER4 --- The number of elapsed days (1 Jan 1AD = 1)
\end{description}







\par
\begin{description}
\item [\colorbox{tagtype}{\color{white} \textbf{\textsf{RETURN}}}] \textbf{} --- Module containing Year, Month, Day in the Julian calendar
\end{description}




\textbf{Children}
\begin{enumerate}
\item \hyperlink{ecldoc:date.tojulianymd.result.day}{Day}
: No Documentation Found
\item \hyperlink{ecldoc:date.tojulianymd.result.month}{Month}
: No Documentation Found
\item \hyperlink{ecldoc:date.tojulianymd.result.year}{Year}
: No Documentation Found
\end{enumerate}

\rule{\linewidth}{0.5pt}

\subsection*{\textsf{\colorbox{headtoc}{\color{white} ATTRIBUTE}
Day}}

\hypertarget{ecldoc:date.tojulianymd.result.day}{}
\hspace{0pt} \hyperlink{ecldoc:Date}{Date} \textbackslash 
\hspace{0pt} \hyperlink{ecldoc:date.tojulianymd}{ToJulianYMD} \textbackslash 

{\renewcommand{\arraystretch}{1.5}
\begin{tabularx}{\textwidth}{|>{\raggedright\arraybackslash}l|X|}
\hline
\hspace{0pt}\mytexttt{\color{red} UNSIGNED1} & \textbf{Day} \\
\hline
\end{tabularx}
}

\par





No Documentation Found








\par
\begin{description}
\item [\colorbox{tagtype}{\color{white} \textbf{\textsf{RETURN}}}] \textbf{UNSIGNED1} --- 
\end{description}




\rule{\linewidth}{0.5pt}
\subsection*{\textsf{\colorbox{headtoc}{\color{white} ATTRIBUTE}
Month}}

\hypertarget{ecldoc:date.tojulianymd.result.month}{}
\hspace{0pt} \hyperlink{ecldoc:Date}{Date} \textbackslash 
\hspace{0pt} \hyperlink{ecldoc:date.tojulianymd}{ToJulianYMD} \textbackslash 

{\renewcommand{\arraystretch}{1.5}
\begin{tabularx}{\textwidth}{|>{\raggedright\arraybackslash}l|X|}
\hline
\hspace{0pt}\mytexttt{\color{red} UNSIGNED1} & \textbf{Month} \\
\hline
\end{tabularx}
}

\par





No Documentation Found








\par
\begin{description}
\item [\colorbox{tagtype}{\color{white} \textbf{\textsf{RETURN}}}] \textbf{UNSIGNED1} --- 
\end{description}




\rule{\linewidth}{0.5pt}
\subsection*{\textsf{\colorbox{headtoc}{\color{white} ATTRIBUTE}
Year}}

\hypertarget{ecldoc:date.tojulianymd.result.year}{}
\hspace{0pt} \hyperlink{ecldoc:Date}{Date} \textbackslash 
\hspace{0pt} \hyperlink{ecldoc:date.tojulianymd}{ToJulianYMD} \textbackslash 

{\renewcommand{\arraystretch}{1.5}
\begin{tabularx}{\textwidth}{|>{\raggedright\arraybackslash}l|X|}
\hline
\hspace{0pt}\mytexttt{\color{red} INTEGER2} & \textbf{Year} \\
\hline
\end{tabularx}
}

\par





No Documentation Found








\par
\begin{description}
\item [\colorbox{tagtype}{\color{white} \textbf{\textsf{RETURN}}}] \textbf{INTEGER2} --- 
\end{description}




\rule{\linewidth}{0.5pt}


\subsection*{\textsf{\colorbox{headtoc}{\color{white} FUNCTION}
FromJulianDate}}

\hypertarget{ecldoc:date.fromjuliandate}{}
\hspace{0pt} \hyperlink{ecldoc:Date}{Date} \textbackslash 

{\renewcommand{\arraystretch}{1.5}
\begin{tabularx}{\textwidth}{|>{\raggedright\arraybackslash}l|X|}
\hline
\hspace{0pt}\mytexttt{\color{red} Days\_t} & \textbf{FromJulianDate} \\
\hline
\multicolumn{2}{|>{\raggedright\arraybackslash}X|}{\hspace{0pt}\mytexttt{\color{param} (Date\_t date)}} \\
\hline
\end{tabularx}
}

\par





Converts a date in the Julian calendar to the number days since 31st December 1BC.






\par
\begin{description}
\item [\colorbox{tagtype}{\color{white} \textbf{\textsf{PARAMETER}}}] \textbf{\underline{date}} ||| UNSIGNED4 --- The date (using the Julian calendar)
\end{description}







\par
\begin{description}
\item [\colorbox{tagtype}{\color{white} \textbf{\textsf{RETURN}}}] \textbf{INTEGER4} --- The number of elapsed days (1 Jan 1AD = 1)
\end{description}




\rule{\linewidth}{0.5pt}
\subsection*{\textsf{\colorbox{headtoc}{\color{white} FUNCTION}
ToJulianDate}}

\hypertarget{ecldoc:date.tojuliandate}{}
\hspace{0pt} \hyperlink{ecldoc:Date}{Date} \textbackslash 

{\renewcommand{\arraystretch}{1.5}
\begin{tabularx}{\textwidth}{|>{\raggedright\arraybackslash}l|X|}
\hline
\hspace{0pt}\mytexttt{\color{red} Date\_t} & \textbf{ToJulianDate} \\
\hline
\multicolumn{2}{|>{\raggedright\arraybackslash}X|}{\hspace{0pt}\mytexttt{\color{param} (Days\_t days)}} \\
\hline
\end{tabularx}
}

\par





Converts the number days since 31st December 1BC to a date in the Julian calendar.






\par
\begin{description}
\item [\colorbox{tagtype}{\color{white} \textbf{\textsf{PARAMETER}}}] \textbf{\underline{days}} ||| INTEGER4 --- The number of elapsed days (1 Jan 1AD = 1)
\end{description}







\par
\begin{description}
\item [\colorbox{tagtype}{\color{white} \textbf{\textsf{RETURN}}}] \textbf{UNSIGNED4} --- A Date\_t in the Julian calendar
\end{description}




\rule{\linewidth}{0.5pt}
\subsection*{\textsf{\colorbox{headtoc}{\color{white} FUNCTION}
DaysSince1900}}

\hypertarget{ecldoc:date.dayssince1900}{}
\hspace{0pt} \hyperlink{ecldoc:Date}{Date} \textbackslash 

{\renewcommand{\arraystretch}{1.5}
\begin{tabularx}{\textwidth}{|>{\raggedright\arraybackslash}l|X|}
\hline
\hspace{0pt}\mytexttt{\color{red} Days\_t} & \textbf{DaysSince1900} \\
\hline
\multicolumn{2}{|>{\raggedright\arraybackslash}X|}{\hspace{0pt}\mytexttt{\color{param} (INTEGER2 year, UNSIGNED1 month, UNSIGNED1 day)}} \\
\hline
\end{tabularx}
}

\par





Returns the number of days since 1st January 1900 (using the Gregorian Calendar)






\par
\begin{description}
\item [\colorbox{tagtype}{\color{white} \textbf{\textsf{PARAMETER}}}] \textbf{\underline{year}} ||| INTEGER2 --- The year (-4713..9999).
\item [\colorbox{tagtype}{\color{white} \textbf{\textsf{PARAMETER}}}] \textbf{\underline{month}} ||| UNSIGNED1 --- The month (1-12). A missing value (0) is treated as 1.
\item [\colorbox{tagtype}{\color{white} \textbf{\textsf{PARAMETER}}}] \textbf{\underline{day}} ||| UNSIGNED1 --- The day (1..daysInMonth). A missing value (0) is treated as 1.
\end{description}







\par
\begin{description}
\item [\colorbox{tagtype}{\color{white} \textbf{\textsf{RETURN}}}] \textbf{INTEGER4} --- The number of elapsed days since 1st January 1900
\end{description}




\rule{\linewidth}{0.5pt}
\subsection*{\textsf{\colorbox{headtoc}{\color{white} FUNCTION}
ToDaysSince1900}}

\hypertarget{ecldoc:date.todayssince1900}{}
\hspace{0pt} \hyperlink{ecldoc:Date}{Date} \textbackslash 

{\renewcommand{\arraystretch}{1.5}
\begin{tabularx}{\textwidth}{|>{\raggedright\arraybackslash}l|X|}
\hline
\hspace{0pt}\mytexttt{\color{red} Days\_t} & \textbf{ToDaysSince1900} \\
\hline
\multicolumn{2}{|>{\raggedright\arraybackslash}X|}{\hspace{0pt}\mytexttt{\color{param} (Date\_t date)}} \\
\hline
\end{tabularx}
}

\par





Returns the number of days since 1st January 1900 (using the Gregorian Calendar)






\par
\begin{description}
\item [\colorbox{tagtype}{\color{white} \textbf{\textsf{PARAMETER}}}] \textbf{\underline{date}} ||| UNSIGNED4 --- The date
\end{description}







\par
\begin{description}
\item [\colorbox{tagtype}{\color{white} \textbf{\textsf{RETURN}}}] \textbf{INTEGER4} --- The number of elapsed days since 1st January 1900
\end{description}




\rule{\linewidth}{0.5pt}
\subsection*{\textsf{\colorbox{headtoc}{\color{white} FUNCTION}
FromDaysSince1900}}

\hypertarget{ecldoc:date.fromdayssince1900}{}
\hspace{0pt} \hyperlink{ecldoc:Date}{Date} \textbackslash 

{\renewcommand{\arraystretch}{1.5}
\begin{tabularx}{\textwidth}{|>{\raggedright\arraybackslash}l|X|}
\hline
\hspace{0pt}\mytexttt{\color{red} Date\_t} & \textbf{FromDaysSince1900} \\
\hline
\multicolumn{2}{|>{\raggedright\arraybackslash}X|}{\hspace{0pt}\mytexttt{\color{param} (Days\_t days)}} \\
\hline
\end{tabularx}
}

\par





Converts the number days since 1st January 1900 to a date in the Julian calendar.






\par
\begin{description}
\item [\colorbox{tagtype}{\color{white} \textbf{\textsf{PARAMETER}}}] \textbf{\underline{days}} ||| INTEGER4 --- The number of elapsed days since 1st Jan 1900
\end{description}







\par
\begin{description}
\item [\colorbox{tagtype}{\color{white} \textbf{\textsf{RETURN}}}] \textbf{UNSIGNED4} --- A Date\_t in the Julian calendar
\end{description}




\rule{\linewidth}{0.5pt}
\subsection*{\textsf{\colorbox{headtoc}{\color{white} FUNCTION}
YearsBetween}}

\hypertarget{ecldoc:date.yearsbetween}{}
\hspace{0pt} \hyperlink{ecldoc:Date}{Date} \textbackslash 

{\renewcommand{\arraystretch}{1.5}
\begin{tabularx}{\textwidth}{|>{\raggedright\arraybackslash}l|X|}
\hline
\hspace{0pt}\mytexttt{\color{red} INTEGER} & \textbf{YearsBetween} \\
\hline
\multicolumn{2}{|>{\raggedright\arraybackslash}X|}{\hspace{0pt}\mytexttt{\color{param} (Date\_t from, Date\_t to)}} \\
\hline
\end{tabularx}
}

\par





Calculate the number of whole years between two dates.






\par
\begin{description}
\item [\colorbox{tagtype}{\color{white} \textbf{\textsf{PARAMETER}}}] \textbf{\underline{from}} ||| UNSIGNED4 --- The first date
\item [\colorbox{tagtype}{\color{white} \textbf{\textsf{PARAMETER}}}] \textbf{\underline{to}} ||| UNSIGNED4 --- The last date
\end{description}







\par
\begin{description}
\item [\colorbox{tagtype}{\color{white} \textbf{\textsf{RETURN}}}] \textbf{INTEGER8} --- The number of years between them.
\end{description}




\rule{\linewidth}{0.5pt}
\subsection*{\textsf{\colorbox{headtoc}{\color{white} FUNCTION}
MonthsBetween}}

\hypertarget{ecldoc:date.monthsbetween}{}
\hspace{0pt} \hyperlink{ecldoc:Date}{Date} \textbackslash 

{\renewcommand{\arraystretch}{1.5}
\begin{tabularx}{\textwidth}{|>{\raggedright\arraybackslash}l|X|}
\hline
\hspace{0pt}\mytexttt{\color{red} INTEGER} & \textbf{MonthsBetween} \\
\hline
\multicolumn{2}{|>{\raggedright\arraybackslash}X|}{\hspace{0pt}\mytexttt{\color{param} (Date\_t from, Date\_t to)}} \\
\hline
\end{tabularx}
}

\par





Calculate the number of whole months between two dates.






\par
\begin{description}
\item [\colorbox{tagtype}{\color{white} \textbf{\textsf{PARAMETER}}}] \textbf{\underline{from}} ||| UNSIGNED4 --- The first date
\item [\colorbox{tagtype}{\color{white} \textbf{\textsf{PARAMETER}}}] \textbf{\underline{to}} ||| UNSIGNED4 --- The last date
\end{description}







\par
\begin{description}
\item [\colorbox{tagtype}{\color{white} \textbf{\textsf{RETURN}}}] \textbf{INTEGER8} --- The number of months between them.
\end{description}




\rule{\linewidth}{0.5pt}
\subsection*{\textsf{\colorbox{headtoc}{\color{white} FUNCTION}
DaysBetween}}

\hypertarget{ecldoc:date.daysbetween}{}
\hspace{0pt} \hyperlink{ecldoc:Date}{Date} \textbackslash 

{\renewcommand{\arraystretch}{1.5}
\begin{tabularx}{\textwidth}{|>{\raggedright\arraybackslash}l|X|}
\hline
\hspace{0pt}\mytexttt{\color{red} INTEGER} & \textbf{DaysBetween} \\
\hline
\multicolumn{2}{|>{\raggedright\arraybackslash}X|}{\hspace{0pt}\mytexttt{\color{param} (Date\_t from, Date\_t to)}} \\
\hline
\end{tabularx}
}

\par





Calculate the number of days between two dates.






\par
\begin{description}
\item [\colorbox{tagtype}{\color{white} \textbf{\textsf{PARAMETER}}}] \textbf{\underline{from}} ||| UNSIGNED4 --- The first date
\item [\colorbox{tagtype}{\color{white} \textbf{\textsf{PARAMETER}}}] \textbf{\underline{to}} ||| UNSIGNED4 --- The last date
\end{description}







\par
\begin{description}
\item [\colorbox{tagtype}{\color{white} \textbf{\textsf{RETURN}}}] \textbf{INTEGER8} --- The number of days between them.
\end{description}




\rule{\linewidth}{0.5pt}
\subsection*{\textsf{\colorbox{headtoc}{\color{white} FUNCTION}
DateFromDateRec}}

\hypertarget{ecldoc:date.datefromdaterec}{}
\hspace{0pt} \hyperlink{ecldoc:Date}{Date} \textbackslash 

{\renewcommand{\arraystretch}{1.5}
\begin{tabularx}{\textwidth}{|>{\raggedright\arraybackslash}l|X|}
\hline
\hspace{0pt}\mytexttt{\color{red} Date\_t} & \textbf{DateFromDateRec} \\
\hline
\multicolumn{2}{|>{\raggedright\arraybackslash}X|}{\hspace{0pt}\mytexttt{\color{param} (Date\_rec date)}} \\
\hline
\end{tabularx}
}

\par





Combines the fields from a Date\_rec to create a Date\_t






\par
\begin{description}
\item [\colorbox{tagtype}{\color{white} \textbf{\textsf{PARAMETER}}}] \textbf{\underline{date}} ||| ROW ( Date\_rec ) --- The row containing the date.
\end{description}







\par
\begin{description}
\item [\colorbox{tagtype}{\color{white} \textbf{\textsf{RETURN}}}] \textbf{UNSIGNED4} --- A Date\_t representing the combined values.
\end{description}




\rule{\linewidth}{0.5pt}
\subsection*{\textsf{\colorbox{headtoc}{\color{white} FUNCTION}
DateFromRec}}

\hypertarget{ecldoc:date.datefromrec}{}
\hspace{0pt} \hyperlink{ecldoc:Date}{Date} \textbackslash 

{\renewcommand{\arraystretch}{1.5}
\begin{tabularx}{\textwidth}{|>{\raggedright\arraybackslash}l|X|}
\hline
\hspace{0pt}\mytexttt{\color{red} Date\_t} & \textbf{DateFromRec} \\
\hline
\multicolumn{2}{|>{\raggedright\arraybackslash}X|}{\hspace{0pt}\mytexttt{\color{param} (Date\_rec date)}} \\
\hline
\end{tabularx}
}

\par





Combines the fields from a Date\_rec to create a Date\_t






\par
\begin{description}
\item [\colorbox{tagtype}{\color{white} \textbf{\textsf{PARAMETER}}}] \textbf{\underline{date}} ||| ROW ( Date\_rec ) --- The row containing the date.
\end{description}







\par
\begin{description}
\item [\colorbox{tagtype}{\color{white} \textbf{\textsf{RETURN}}}] \textbf{UNSIGNED4} --- A Date\_t representing the combined values.
\end{description}




\rule{\linewidth}{0.5pt}
\subsection*{\textsf{\colorbox{headtoc}{\color{white} FUNCTION}
TimeFromTimeRec}}

\hypertarget{ecldoc:date.timefromtimerec}{}
\hspace{0pt} \hyperlink{ecldoc:Date}{Date} \textbackslash 

{\renewcommand{\arraystretch}{1.5}
\begin{tabularx}{\textwidth}{|>{\raggedright\arraybackslash}l|X|}
\hline
\hspace{0pt}\mytexttt{\color{red} Time\_t} & \textbf{TimeFromTimeRec} \\
\hline
\multicolumn{2}{|>{\raggedright\arraybackslash}X|}{\hspace{0pt}\mytexttt{\color{param} (Time\_rec time)}} \\
\hline
\end{tabularx}
}

\par





Combines the fields from a Time\_rec to create a Time\_t






\par
\begin{description}
\item [\colorbox{tagtype}{\color{white} \textbf{\textsf{PARAMETER}}}] \textbf{\underline{time}} ||| ROW ( Time\_rec ) --- The row containing the time.
\end{description}







\par
\begin{description}
\item [\colorbox{tagtype}{\color{white} \textbf{\textsf{RETURN}}}] \textbf{UNSIGNED3} --- A Time\_t representing the combined values.
\end{description}




\rule{\linewidth}{0.5pt}
\subsection*{\textsf{\colorbox{headtoc}{\color{white} FUNCTION}
DateFromDateTimeRec}}

\hypertarget{ecldoc:date.datefromdatetimerec}{}
\hspace{0pt} \hyperlink{ecldoc:Date}{Date} \textbackslash 

{\renewcommand{\arraystretch}{1.5}
\begin{tabularx}{\textwidth}{|>{\raggedright\arraybackslash}l|X|}
\hline
\hspace{0pt}\mytexttt{\color{red} Date\_t} & \textbf{DateFromDateTimeRec} \\
\hline
\multicolumn{2}{|>{\raggedright\arraybackslash}X|}{\hspace{0pt}\mytexttt{\color{param} (DateTime\_rec datetime)}} \\
\hline
\end{tabularx}
}

\par





Combines the date fields from a DateTime\_rec to create a Date\_t






\par
\begin{description}
\item [\colorbox{tagtype}{\color{white} \textbf{\textsf{PARAMETER}}}] \textbf{\underline{datetime}} ||| ROW ( DateTime\_rec ) --- The row containing the datetime.
\end{description}







\par
\begin{description}
\item [\colorbox{tagtype}{\color{white} \textbf{\textsf{RETURN}}}] \textbf{UNSIGNED4} --- A Date\_t representing the combined values.
\end{description}




\rule{\linewidth}{0.5pt}
\subsection*{\textsf{\colorbox{headtoc}{\color{white} FUNCTION}
TimeFromDateTimeRec}}

\hypertarget{ecldoc:date.timefromdatetimerec}{}
\hspace{0pt} \hyperlink{ecldoc:Date}{Date} \textbackslash 

{\renewcommand{\arraystretch}{1.5}
\begin{tabularx}{\textwidth}{|>{\raggedright\arraybackslash}l|X|}
\hline
\hspace{0pt}\mytexttt{\color{red} Time\_t} & \textbf{TimeFromDateTimeRec} \\
\hline
\multicolumn{2}{|>{\raggedright\arraybackslash}X|}{\hspace{0pt}\mytexttt{\color{param} (DateTime\_rec datetime)}} \\
\hline
\end{tabularx}
}

\par





Combines the time fields from a DateTime\_rec to create a Time\_t






\par
\begin{description}
\item [\colorbox{tagtype}{\color{white} \textbf{\textsf{PARAMETER}}}] \textbf{\underline{datetime}} ||| ROW ( DateTime\_rec ) --- The row containing the datetime.
\end{description}







\par
\begin{description}
\item [\colorbox{tagtype}{\color{white} \textbf{\textsf{RETURN}}}] \textbf{UNSIGNED3} --- A Time\_t representing the combined values.
\end{description}




\rule{\linewidth}{0.5pt}
\subsection*{\textsf{\colorbox{headtoc}{\color{white} FUNCTION}
SecondsFromDateTimeRec}}

\hypertarget{ecldoc:date.secondsfromdatetimerec}{}
\hspace{0pt} \hyperlink{ecldoc:Date}{Date} \textbackslash 

{\renewcommand{\arraystretch}{1.5}
\begin{tabularx}{\textwidth}{|>{\raggedright\arraybackslash}l|X|}
\hline
\hspace{0pt}\mytexttt{\color{red} Seconds\_t} & \textbf{SecondsFromDateTimeRec} \\
\hline
\multicolumn{2}{|>{\raggedright\arraybackslash}X|}{\hspace{0pt}\mytexttt{\color{param} (DateTime\_rec datetime, BOOLEAN is\_local\_time = FALSE)}} \\
\hline
\end{tabularx}
}

\par





Combines the date and time fields from a DateTime\_rec to create a Seconds\_t






\par
\begin{description}
\item [\colorbox{tagtype}{\color{white} \textbf{\textsf{PARAMETER}}}] \textbf{\underline{datetime}} ||| ROW ( DateTime\_rec ) --- The row containing the datetime.
\item [\colorbox{tagtype}{\color{white} \textbf{\textsf{PARAMETER}}}] \textbf{\underline{is\_local\_time}} ||| BOOLEAN --- TRUE if the datetime components are expressed in local time rather than UTC, FALSE if the components are expressed in UTC. Optional, defaults to FALSE.
\end{description}







\par
\begin{description}
\item [\colorbox{tagtype}{\color{white} \textbf{\textsf{RETURN}}}] \textbf{INTEGER8} --- A Seconds\_t representing the combined values.
\end{description}




\rule{\linewidth}{0.5pt}
\subsection*{\textsf{\colorbox{headtoc}{\color{white} FUNCTION}
FromStringToDate}}

\hypertarget{ecldoc:date.fromstringtodate}{}
\hspace{0pt} \hyperlink{ecldoc:Date}{Date} \textbackslash 

{\renewcommand{\arraystretch}{1.5}
\begin{tabularx}{\textwidth}{|>{\raggedright\arraybackslash}l|X|}
\hline
\hspace{0pt}\mytexttt{\color{red} Date\_t} & \textbf{FromStringToDate} \\
\hline
\multicolumn{2}{|>{\raggedright\arraybackslash}X|}{\hspace{0pt}\mytexttt{\color{param} (STRING date\_text, VARSTRING format)}} \\
\hline
\end{tabularx}
}

\par





Converts a string to a Date\_t using the relevant string format. The resulting date must be representable within the Gregorian calendar after the year 1600.






\par
\begin{description}
\item [\colorbox{tagtype}{\color{white} \textbf{\textsf{PARAMETER}}}] \textbf{\underline{date\_text}} ||| STRING --- The string to be converted.
\item [\colorbox{tagtype}{\color{white} \textbf{\textsf{PARAMETER}}}] \textbf{\underline{format}} ||| VARSTRING --- The format of the input string. (See documentation for strftime)
\end{description}







\par
\begin{description}
\item [\colorbox{tagtype}{\color{white} \textbf{\textsf{RETURN}}}] \textbf{UNSIGNED4} --- The date that was matched in the string. Returns 0 if failed to match or if the date components match but the result is an invalid date. Supported characters: \%B Full month name \%b or \%h Abbreviated month name \%d Day of month (two digits) \%e Day of month (two digits, or a space followed by a single digit) \%m Month (two digits) \%t Whitespace \%y year within century (00-99) \%Y Full year (yyyy) \%j Julian day (1-366) Common date formats American '\%m/\%d/\%Y' mm/dd/yyyy Euro '\%d/\%m/\%Y' dd/mm/yyyy Iso format '\%Y-\%m-\%d' yyyy-mm-dd Iso basic 'Y\%m\%d' yyyymmdd '\%d-\%b-\%Y' dd-mon-yyyy e.g., '21-Mar-1954'
\end{description}




\rule{\linewidth}{0.5pt}
\subsection*{\textsf{\colorbox{headtoc}{\color{white} FUNCTION}
FromString}}

\hypertarget{ecldoc:date.fromstring}{}
\hspace{0pt} \hyperlink{ecldoc:Date}{Date} \textbackslash 

{\renewcommand{\arraystretch}{1.5}
\begin{tabularx}{\textwidth}{|>{\raggedright\arraybackslash}l|X|}
\hline
\hspace{0pt}\mytexttt{\color{red} Date\_t} & \textbf{FromString} \\
\hline
\multicolumn{2}{|>{\raggedright\arraybackslash}X|}{\hspace{0pt}\mytexttt{\color{param} (STRING date\_text, VARSTRING format)}} \\
\hline
\end{tabularx}
}

\par





Converts a string to a date using the relevant string format.






\par
\begin{description}
\item [\colorbox{tagtype}{\color{white} \textbf{\textsf{PARAMETER}}}] \textbf{\underline{date\_text}} ||| STRING --- The string to be converted.
\item [\colorbox{tagtype}{\color{white} \textbf{\textsf{PARAMETER}}}] \textbf{\underline{format}} ||| VARSTRING --- The format of the input string. (See documentation for strftime)
\end{description}







\par
\begin{description}
\item [\colorbox{tagtype}{\color{white} \textbf{\textsf{RETURN}}}] \textbf{UNSIGNED4} --- The date that was matched in the string. Returns 0 if failed to match.
\end{description}




\rule{\linewidth}{0.5pt}
\subsection*{\textsf{\colorbox{headtoc}{\color{white} FUNCTION}
FromStringToTime}}

\hypertarget{ecldoc:date.fromstringtotime}{}
\hspace{0pt} \hyperlink{ecldoc:Date}{Date} \textbackslash 

{\renewcommand{\arraystretch}{1.5}
\begin{tabularx}{\textwidth}{|>{\raggedright\arraybackslash}l|X|}
\hline
\hspace{0pt}\mytexttt{\color{red} Time\_t} & \textbf{FromStringToTime} \\
\hline
\multicolumn{2}{|>{\raggedright\arraybackslash}X|}{\hspace{0pt}\mytexttt{\color{param} (STRING time\_text, VARSTRING format)}} \\
\hline
\end{tabularx}
}

\par





Converts a string to a Time\_t using the relevant string format.






\par
\begin{description}
\item [\colorbox{tagtype}{\color{white} \textbf{\textsf{PARAMETER}}}] \textbf{\underline{date\_text}} |||  --- The string to be converted.
\item [\colorbox{tagtype}{\color{white} \textbf{\textsf{PARAMETER}}}] \textbf{\underline{format}} ||| VARSTRING --- The format of the input string. (See documentation for strftime)
\item [\colorbox{tagtype}{\color{white} \textbf{\textsf{PARAMETER}}}] \textbf{\underline{time\_text}} ||| STRING --- No Doc
\end{description}







\par
\begin{description}
\item [\colorbox{tagtype}{\color{white} \textbf{\textsf{RETURN}}}] \textbf{UNSIGNED3} --- The time that was matched in the string. Returns 0 if failed to match. Supported characters: \%H Hour (two digits) \%k (two digits, or a space followed by a single digit) \%M Minute (two digits) \%S Second (two digits) \%t Whitespace
\end{description}




\rule{\linewidth}{0.5pt}
\subsection*{\textsf{\colorbox{headtoc}{\color{white} FUNCTION}
MatchDateString}}

\hypertarget{ecldoc:date.matchdatestring}{}
\hspace{0pt} \hyperlink{ecldoc:Date}{Date} \textbackslash 

{\renewcommand{\arraystretch}{1.5}
\begin{tabularx}{\textwidth}{|>{\raggedright\arraybackslash}l|X|}
\hline
\hspace{0pt}\mytexttt{\color{red} Date\_t} & \textbf{MatchDateString} \\
\hline
\multicolumn{2}{|>{\raggedright\arraybackslash}X|}{\hspace{0pt}\mytexttt{\color{param} (STRING date\_text, SET OF VARSTRING formats)}} \\
\hline
\end{tabularx}
}

\par





Matches a string against a set of date string formats and returns a valid Date\_t object from the first format that successfully parses the string.






\par
\begin{description}
\item [\colorbox{tagtype}{\color{white} \textbf{\textsf{PARAMETER}}}] \textbf{\underline{formats}} ||| SET ( VARSTRING ) --- A set of formats to check against the string. (See documentation for strftime)
\item [\colorbox{tagtype}{\color{white} \textbf{\textsf{PARAMETER}}}] \textbf{\underline{date\_text}} ||| STRING --- The string to be converted.
\end{description}







\par
\begin{description}
\item [\colorbox{tagtype}{\color{white} \textbf{\textsf{RETURN}}}] \textbf{UNSIGNED4} --- The date that was matched in the string. Returns 0 if failed to match.
\end{description}




\rule{\linewidth}{0.5pt}
\subsection*{\textsf{\colorbox{headtoc}{\color{white} FUNCTION}
MatchTimeString}}

\hypertarget{ecldoc:date.matchtimestring}{}
\hspace{0pt} \hyperlink{ecldoc:Date}{Date} \textbackslash 

{\renewcommand{\arraystretch}{1.5}
\begin{tabularx}{\textwidth}{|>{\raggedright\arraybackslash}l|X|}
\hline
\hspace{0pt}\mytexttt{\color{red} Time\_t} & \textbf{MatchTimeString} \\
\hline
\multicolumn{2}{|>{\raggedright\arraybackslash}X|}{\hspace{0pt}\mytexttt{\color{param} (STRING time\_text, SET OF VARSTRING formats)}} \\
\hline
\end{tabularx}
}

\par





Matches a string against a set of time string formats and returns a valid Time\_t object from the first format that successfully parses the string.






\par
\begin{description}
\item [\colorbox{tagtype}{\color{white} \textbf{\textsf{PARAMETER}}}] \textbf{\underline{formats}} ||| SET ( VARSTRING ) --- A set of formats to check against the string. (See documentation for strftime)
\item [\colorbox{tagtype}{\color{white} \textbf{\textsf{PARAMETER}}}] \textbf{\underline{time\_text}} ||| STRING --- The string to be converted.
\end{description}







\par
\begin{description}
\item [\colorbox{tagtype}{\color{white} \textbf{\textsf{RETURN}}}] \textbf{UNSIGNED3} --- The time that was matched in the string. Returns 0 if failed to match.
\end{description}




\rule{\linewidth}{0.5pt}
\subsection*{\textsf{\colorbox{headtoc}{\color{white} FUNCTION}
DateToString}}

\hypertarget{ecldoc:date.datetostring}{}
\hspace{0pt} \hyperlink{ecldoc:Date}{Date} \textbackslash 

{\renewcommand{\arraystretch}{1.5}
\begin{tabularx}{\textwidth}{|>{\raggedright\arraybackslash}l|X|}
\hline
\hspace{0pt}\mytexttt{\color{red} STRING} & \textbf{DateToString} \\
\hline
\multicolumn{2}{|>{\raggedright\arraybackslash}X|}{\hspace{0pt}\mytexttt{\color{param} (Date\_t date, VARSTRING format = '\%Y-\%m-\%d')}} \\
\hline
\end{tabularx}
}

\par





Formats a date as a string.






\par
\begin{description}
\item [\colorbox{tagtype}{\color{white} \textbf{\textsf{PARAMETER}}}] \textbf{\underline{date}} ||| UNSIGNED4 --- The date to be converted.
\item [\colorbox{tagtype}{\color{white} \textbf{\textsf{PARAMETER}}}] \textbf{\underline{format}} ||| VARSTRING --- The format template to use for the conversion; see strftime() for appropriate values. The maximum length of the resulting string is 255 characters. Optional; defaults to '\%Y-\%m-\%d' which is YYYY-MM-DD.
\end{description}







\par
\begin{description}
\item [\colorbox{tagtype}{\color{white} \textbf{\textsf{RETURN}}}] \textbf{STRING} --- Blank if date cannot be formatted, or the date in the requested format.
\end{description}




\rule{\linewidth}{0.5pt}
\subsection*{\textsf{\colorbox{headtoc}{\color{white} FUNCTION}
TimeToString}}

\hypertarget{ecldoc:date.timetostring}{}
\hspace{0pt} \hyperlink{ecldoc:Date}{Date} \textbackslash 

{\renewcommand{\arraystretch}{1.5}
\begin{tabularx}{\textwidth}{|>{\raggedright\arraybackslash}l|X|}
\hline
\hspace{0pt}\mytexttt{\color{red} STRING} & \textbf{TimeToString} \\
\hline
\multicolumn{2}{|>{\raggedright\arraybackslash}X|}{\hspace{0pt}\mytexttt{\color{param} (Time\_t time, VARSTRING format = '\%H:\%M:\%S')}} \\
\hline
\end{tabularx}
}

\par





Formats a time as a string.






\par
\begin{description}
\item [\colorbox{tagtype}{\color{white} \textbf{\textsf{PARAMETER}}}] \textbf{\underline{time}} ||| UNSIGNED3 --- The time to be converted.
\item [\colorbox{tagtype}{\color{white} \textbf{\textsf{PARAMETER}}}] \textbf{\underline{format}} ||| VARSTRING --- The format template to use for the conversion; see strftime() for appropriate values. The maximum length of the resulting string is 255 characters. Optional; defaults to '\%H:\%M:\%S' which is HH:MM:SS.
\end{description}







\par
\begin{description}
\item [\colorbox{tagtype}{\color{white} \textbf{\textsf{RETURN}}}] \textbf{STRING} --- Blank if the time cannot be formatted, or the time in the requested format.
\end{description}




\rule{\linewidth}{0.5pt}
\subsection*{\textsf{\colorbox{headtoc}{\color{white} FUNCTION}
SecondsToString}}

\hypertarget{ecldoc:date.secondstostring}{}
\hspace{0pt} \hyperlink{ecldoc:Date}{Date} \textbackslash 

{\renewcommand{\arraystretch}{1.5}
\begin{tabularx}{\textwidth}{|>{\raggedright\arraybackslash}l|X|}
\hline
\hspace{0pt}\mytexttt{\color{red} STRING} & \textbf{SecondsToString} \\
\hline
\multicolumn{2}{|>{\raggedright\arraybackslash}X|}{\hspace{0pt}\mytexttt{\color{param} (Seconds\_t seconds, VARSTRING format = '\%Y-\%m-\%dT\%H:\%M:\%S')}} \\
\hline
\end{tabularx}
}

\par





Converts a Seconds\_t value into a human-readable string using a format template.






\par
\begin{description}
\item [\colorbox{tagtype}{\color{white} \textbf{\textsf{PARAMETER}}}] \textbf{\underline{format}} ||| VARSTRING --- The format template to use for the conversion; see strftime() for appropriate values. The maximum length of the resulting string is 255 characters. Optional; defaults to '\%Y-\%m-\%dT\%H:\%M:\%S' which is YYYY-MM-DDTHH:MM:SS.
\item [\colorbox{tagtype}{\color{white} \textbf{\textsf{PARAMETER}}}] \textbf{\underline{seconds}} ||| INTEGER8 --- The seconds since epoch.
\end{description}







\par
\begin{description}
\item [\colorbox{tagtype}{\color{white} \textbf{\textsf{RETURN}}}] \textbf{STRING} --- The converted seconds as a string.
\end{description}




\rule{\linewidth}{0.5pt}
\subsection*{\textsf{\colorbox{headtoc}{\color{white} FUNCTION}
ToString}}

\hypertarget{ecldoc:date.tostring}{}
\hspace{0pt} \hyperlink{ecldoc:Date}{Date} \textbackslash 

{\renewcommand{\arraystretch}{1.5}
\begin{tabularx}{\textwidth}{|>{\raggedright\arraybackslash}l|X|}
\hline
\hspace{0pt}\mytexttt{\color{red} STRING} & \textbf{ToString} \\
\hline
\multicolumn{2}{|>{\raggedright\arraybackslash}X|}{\hspace{0pt}\mytexttt{\color{param} (Date\_t date, VARSTRING format)}} \\
\hline
\end{tabularx}
}

\par





Formats a date as a string.






\par
\begin{description}
\item [\colorbox{tagtype}{\color{white} \textbf{\textsf{PARAMETER}}}] \textbf{\underline{date}} ||| UNSIGNED4 --- The date to be converted.
\item [\colorbox{tagtype}{\color{white} \textbf{\textsf{PARAMETER}}}] \textbf{\underline{format}} ||| VARSTRING --- The format the date is output in. (See documentation for strftime)
\end{description}







\par
\begin{description}
\item [\colorbox{tagtype}{\color{white} \textbf{\textsf{RETURN}}}] \textbf{STRING} --- Blank if date cannot be formatted, or the date in the requested format.
\end{description}




\rule{\linewidth}{0.5pt}
\subsection*{\textsf{\colorbox{headtoc}{\color{white} FUNCTION}
ConvertDateFormat}}

\hypertarget{ecldoc:date.convertdateformat}{}
\hspace{0pt} \hyperlink{ecldoc:Date}{Date} \textbackslash 

{\renewcommand{\arraystretch}{1.5}
\begin{tabularx}{\textwidth}{|>{\raggedright\arraybackslash}l|X|}
\hline
\hspace{0pt}\mytexttt{\color{red} STRING} & \textbf{ConvertDateFormat} \\
\hline
\multicolumn{2}{|>{\raggedright\arraybackslash}X|}{\hspace{0pt}\mytexttt{\color{param} (STRING date\_text, VARSTRING from\_format='\%m/\%d/\%Y', VARSTRING to\_format='\%Y\%m\%d')}} \\
\hline
\end{tabularx}
}

\par





Converts a date from one format to another






\par
\begin{description}
\item [\colorbox{tagtype}{\color{white} \textbf{\textsf{PARAMETER}}}] \textbf{\underline{from\_format}} ||| VARSTRING --- The format the date is to be converted from.
\item [\colorbox{tagtype}{\color{white} \textbf{\textsf{PARAMETER}}}] \textbf{\underline{date\_text}} ||| STRING --- The string containing the date to be converted.
\item [\colorbox{tagtype}{\color{white} \textbf{\textsf{PARAMETER}}}] \textbf{\underline{to\_format}} ||| VARSTRING --- The format the date is to be converted to.
\end{description}







\par
\begin{description}
\item [\colorbox{tagtype}{\color{white} \textbf{\textsf{RETURN}}}] \textbf{STRING} --- The converted string, or blank if it failed to match the format.
\end{description}




\rule{\linewidth}{0.5pt}
\subsection*{\textsf{\colorbox{headtoc}{\color{white} FUNCTION}
ConvertFormat}}

\hypertarget{ecldoc:date.convertformat}{}
\hspace{0pt} \hyperlink{ecldoc:Date}{Date} \textbackslash 

{\renewcommand{\arraystretch}{1.5}
\begin{tabularx}{\textwidth}{|>{\raggedright\arraybackslash}l|X|}
\hline
\hspace{0pt}\mytexttt{\color{red} STRING} & \textbf{ConvertFormat} \\
\hline
\multicolumn{2}{|>{\raggedright\arraybackslash}X|}{\hspace{0pt}\mytexttt{\color{param} (STRING date\_text, VARSTRING from\_format='\%m/\%d/\%Y', VARSTRING to\_format='\%Y\%m\%d')}} \\
\hline
\end{tabularx}
}

\par





Converts a date from one format to another






\par
\begin{description}
\item [\colorbox{tagtype}{\color{white} \textbf{\textsf{PARAMETER}}}] \textbf{\underline{from\_format}} ||| VARSTRING --- The format the date is to be converted from.
\item [\colorbox{tagtype}{\color{white} \textbf{\textsf{PARAMETER}}}] \textbf{\underline{date\_text}} ||| STRING --- The string containing the date to be converted.
\item [\colorbox{tagtype}{\color{white} \textbf{\textsf{PARAMETER}}}] \textbf{\underline{to\_format}} ||| VARSTRING --- The format the date is to be converted to.
\end{description}







\par
\begin{description}
\item [\colorbox{tagtype}{\color{white} \textbf{\textsf{RETURN}}}] \textbf{STRING} --- The converted string, or blank if it failed to match the format.
\end{description}




\rule{\linewidth}{0.5pt}
\subsection*{\textsf{\colorbox{headtoc}{\color{white} FUNCTION}
ConvertTimeFormat}}

\hypertarget{ecldoc:date.converttimeformat}{}
\hspace{0pt} \hyperlink{ecldoc:Date}{Date} \textbackslash 

{\renewcommand{\arraystretch}{1.5}
\begin{tabularx}{\textwidth}{|>{\raggedright\arraybackslash}l|X|}
\hline
\hspace{0pt}\mytexttt{\color{red} STRING} & \textbf{ConvertTimeFormat} \\
\hline
\multicolumn{2}{|>{\raggedright\arraybackslash}X|}{\hspace{0pt}\mytexttt{\color{param} (STRING time\_text, VARSTRING from\_format='\%H\%M\%S', VARSTRING to\_format='\%H:\%M:\%S')}} \\
\hline
\end{tabularx}
}

\par





Converts a time from one format to another






\par
\begin{description}
\item [\colorbox{tagtype}{\color{white} \textbf{\textsf{PARAMETER}}}] \textbf{\underline{from\_format}} ||| VARSTRING --- The format the time is to be converted from.
\item [\colorbox{tagtype}{\color{white} \textbf{\textsf{PARAMETER}}}] \textbf{\underline{to\_format}} ||| VARSTRING --- The format the time is to be converted to.
\item [\colorbox{tagtype}{\color{white} \textbf{\textsf{PARAMETER}}}] \textbf{\underline{time\_text}} ||| STRING --- The string containing the time to be converted.
\end{description}







\par
\begin{description}
\item [\colorbox{tagtype}{\color{white} \textbf{\textsf{RETURN}}}] \textbf{STRING} --- The converted string, or blank if it failed to match the format.
\end{description}




\rule{\linewidth}{0.5pt}
\subsection*{\textsf{\colorbox{headtoc}{\color{white} FUNCTION}
ConvertDateFormatMultiple}}

\hypertarget{ecldoc:date.convertdateformatmultiple}{}
\hspace{0pt} \hyperlink{ecldoc:Date}{Date} \textbackslash 

{\renewcommand{\arraystretch}{1.5}
\begin{tabularx}{\textwidth}{|>{\raggedright\arraybackslash}l|X|}
\hline
\hspace{0pt}\mytexttt{\color{red} STRING} & \textbf{ConvertDateFormatMultiple} \\
\hline
\multicolumn{2}{|>{\raggedright\arraybackslash}X|}{\hspace{0pt}\mytexttt{\color{param} (STRING date\_text, SET OF VARSTRING from\_formats, VARSTRING to\_format='\%Y\%m\%d')}} \\
\hline
\end{tabularx}
}

\par





Converts a date that matches one of a set of formats to another.






\par
\begin{description}
\item [\colorbox{tagtype}{\color{white} \textbf{\textsf{PARAMETER}}}] \textbf{\underline{from\_formats}} ||| SET ( VARSTRING ) --- The list of formats the date is to be converted from.
\item [\colorbox{tagtype}{\color{white} \textbf{\textsf{PARAMETER}}}] \textbf{\underline{date\_text}} ||| STRING --- The string containing the date to be converted.
\item [\colorbox{tagtype}{\color{white} \textbf{\textsf{PARAMETER}}}] \textbf{\underline{to\_format}} ||| VARSTRING --- The format the date is to be converted to.
\end{description}







\par
\begin{description}
\item [\colorbox{tagtype}{\color{white} \textbf{\textsf{RETURN}}}] \textbf{STRING} --- The converted string, or blank if it failed to match the format.
\end{description}




\rule{\linewidth}{0.5pt}
\subsection*{\textsf{\colorbox{headtoc}{\color{white} FUNCTION}
ConvertFormatMultiple}}

\hypertarget{ecldoc:date.convertformatmultiple}{}
\hspace{0pt} \hyperlink{ecldoc:Date}{Date} \textbackslash 

{\renewcommand{\arraystretch}{1.5}
\begin{tabularx}{\textwidth}{|>{\raggedright\arraybackslash}l|X|}
\hline
\hspace{0pt}\mytexttt{\color{red} STRING} & \textbf{ConvertFormatMultiple} \\
\hline
\multicolumn{2}{|>{\raggedright\arraybackslash}X|}{\hspace{0pt}\mytexttt{\color{param} (STRING date\_text, SET OF VARSTRING from\_formats, VARSTRING to\_format='\%Y\%m\%d')}} \\
\hline
\end{tabularx}
}

\par





Converts a date that matches one of a set of formats to another.






\par
\begin{description}
\item [\colorbox{tagtype}{\color{white} \textbf{\textsf{PARAMETER}}}] \textbf{\underline{from\_formats}} ||| SET ( VARSTRING ) --- The list of formats the date is to be converted from.
\item [\colorbox{tagtype}{\color{white} \textbf{\textsf{PARAMETER}}}] \textbf{\underline{date\_text}} ||| STRING --- The string containing the date to be converted.
\item [\colorbox{tagtype}{\color{white} \textbf{\textsf{PARAMETER}}}] \textbf{\underline{to\_format}} ||| VARSTRING --- The format the date is to be converted to.
\end{description}







\par
\begin{description}
\item [\colorbox{tagtype}{\color{white} \textbf{\textsf{RETURN}}}] \textbf{STRING} --- The converted string, or blank if it failed to match the format.
\end{description}




\rule{\linewidth}{0.5pt}
\subsection*{\textsf{\colorbox{headtoc}{\color{white} FUNCTION}
ConvertTimeFormatMultiple}}

\hypertarget{ecldoc:date.converttimeformatmultiple}{}
\hspace{0pt} \hyperlink{ecldoc:Date}{Date} \textbackslash 

{\renewcommand{\arraystretch}{1.5}
\begin{tabularx}{\textwidth}{|>{\raggedright\arraybackslash}l|X|}
\hline
\hspace{0pt}\mytexttt{\color{red} STRING} & \textbf{ConvertTimeFormatMultiple} \\
\hline
\multicolumn{2}{|>{\raggedright\arraybackslash}X|}{\hspace{0pt}\mytexttt{\color{param} (STRING time\_text, SET OF VARSTRING from\_formats, VARSTRING to\_format='\%H:\%m:\%s')}} \\
\hline
\end{tabularx}
}

\par





Converts a time that matches one of a set of formats to another.






\par
\begin{description}
\item [\colorbox{tagtype}{\color{white} \textbf{\textsf{PARAMETER}}}] \textbf{\underline{from\_formats}} ||| SET ( VARSTRING ) --- The list of formats the time is to be converted from.
\item [\colorbox{tagtype}{\color{white} \textbf{\textsf{PARAMETER}}}] \textbf{\underline{to\_format}} ||| VARSTRING --- The format the time is to be converted to.
\item [\colorbox{tagtype}{\color{white} \textbf{\textsf{PARAMETER}}}] \textbf{\underline{time\_text}} ||| STRING --- The string containing the time to be converted.
\end{description}







\par
\begin{description}
\item [\colorbox{tagtype}{\color{white} \textbf{\textsf{RETURN}}}] \textbf{STRING} --- The converted string, or blank if it failed to match the format.
\end{description}




\rule{\linewidth}{0.5pt}
\subsection*{\textsf{\colorbox{headtoc}{\color{white} FUNCTION}
AdjustDate}}

\hypertarget{ecldoc:date.adjustdate}{}
\hspace{0pt} \hyperlink{ecldoc:Date}{Date} \textbackslash 

{\renewcommand{\arraystretch}{1.5}
\begin{tabularx}{\textwidth}{|>{\raggedright\arraybackslash}l|X|}
\hline
\hspace{0pt}\mytexttt{\color{red} Date\_t} & \textbf{AdjustDate} \\
\hline
\multicolumn{2}{|>{\raggedright\arraybackslash}X|}{\hspace{0pt}\mytexttt{\color{param} (Date\_t date, INTEGER2 year\_delta = 0, INTEGER4 month\_delta = 0, INTEGER4 day\_delta = 0)}} \\
\hline
\end{tabularx}
}

\par





Adjusts a date by incrementing or decrementing year, month and/or day values. The date must be in the Gregorian calendar after the year 1600. If the new calculated date is invalid then it will be normalized according to mktime() rules. Example: 20140130 + 1 month = 20140302.






\par
\begin{description}
\item [\colorbox{tagtype}{\color{white} \textbf{\textsf{PARAMETER}}}] \textbf{\underline{date}} ||| UNSIGNED4 --- The date to adjust.
\item [\colorbox{tagtype}{\color{white} \textbf{\textsf{PARAMETER}}}] \textbf{\underline{month\_delta}} ||| INTEGER4 --- The requested change to the month value; optional, defaults to zero.
\item [\colorbox{tagtype}{\color{white} \textbf{\textsf{PARAMETER}}}] \textbf{\underline{year\_delta}} ||| INTEGER2 --- The requested change to the year value; optional, defaults to zero.
\item [\colorbox{tagtype}{\color{white} \textbf{\textsf{PARAMETER}}}] \textbf{\underline{day\_delta}} ||| INTEGER4 --- The requested change to the day of month value; optional, defaults to zero.
\end{description}







\par
\begin{description}
\item [\colorbox{tagtype}{\color{white} \textbf{\textsf{RETURN}}}] \textbf{UNSIGNED4} --- The adjusted Date\_t value.
\end{description}




\rule{\linewidth}{0.5pt}
\subsection*{\textsf{\colorbox{headtoc}{\color{white} FUNCTION}
AdjustDateBySeconds}}

\hypertarget{ecldoc:date.adjustdatebyseconds}{}
\hspace{0pt} \hyperlink{ecldoc:Date}{Date} \textbackslash 

{\renewcommand{\arraystretch}{1.5}
\begin{tabularx}{\textwidth}{|>{\raggedright\arraybackslash}l|X|}
\hline
\hspace{0pt}\mytexttt{\color{red} Date\_t} & \textbf{AdjustDateBySeconds} \\
\hline
\multicolumn{2}{|>{\raggedright\arraybackslash}X|}{\hspace{0pt}\mytexttt{\color{param} (Date\_t date, INTEGER4 seconds\_delta)}} \\
\hline
\end{tabularx}
}

\par





Adjusts a date by adding or subtracting seconds. The date must be in the Gregorian calendar after the year 1600. If the new calculated date is invalid then it will be normalized according to mktime() rules. Example: 20140130 + 172800 seconds = 20140201.






\par
\begin{description}
\item [\colorbox{tagtype}{\color{white} \textbf{\textsf{PARAMETER}}}] \textbf{\underline{date}} ||| UNSIGNED4 --- The date to adjust.
\item [\colorbox{tagtype}{\color{white} \textbf{\textsf{PARAMETER}}}] \textbf{\underline{seconds\_delta}} ||| INTEGER4 --- The requested change to the date, in seconds.
\end{description}







\par
\begin{description}
\item [\colorbox{tagtype}{\color{white} \textbf{\textsf{RETURN}}}] \textbf{UNSIGNED4} --- The adjusted Date\_t value.
\end{description}




\rule{\linewidth}{0.5pt}
\subsection*{\textsf{\colorbox{headtoc}{\color{white} FUNCTION}
AdjustTime}}

\hypertarget{ecldoc:date.adjusttime}{}
\hspace{0pt} \hyperlink{ecldoc:Date}{Date} \textbackslash 

{\renewcommand{\arraystretch}{1.5}
\begin{tabularx}{\textwidth}{|>{\raggedright\arraybackslash}l|X|}
\hline
\hspace{0pt}\mytexttt{\color{red} Time\_t} & \textbf{AdjustTime} \\
\hline
\multicolumn{2}{|>{\raggedright\arraybackslash}X|}{\hspace{0pt}\mytexttt{\color{param} (Time\_t time, INTEGER2 hour\_delta = 0, INTEGER4 minute\_delta = 0, INTEGER4 second\_delta = 0)}} \\
\hline
\end{tabularx}
}

\par





Adjusts a time by incrementing or decrementing hour, minute and/or second values. If the new calculated time is invalid then it will be normalized according to mktime() rules.






\par
\begin{description}
\item [\colorbox{tagtype}{\color{white} \textbf{\textsf{PARAMETER}}}] \textbf{\underline{time}} ||| UNSIGNED3 --- The time to adjust.
\item [\colorbox{tagtype}{\color{white} \textbf{\textsf{PARAMETER}}}] \textbf{\underline{second\_delta}} ||| INTEGER4 --- The requested change to the second of month value; optional, defaults to zero.
\item [\colorbox{tagtype}{\color{white} \textbf{\textsf{PARAMETER}}}] \textbf{\underline{hour\_delta}} ||| INTEGER2 --- The requested change to the hour value; optional, defaults to zero.
\item [\colorbox{tagtype}{\color{white} \textbf{\textsf{PARAMETER}}}] \textbf{\underline{minute\_delta}} ||| INTEGER4 --- The requested change to the minute value; optional, defaults to zero.
\end{description}







\par
\begin{description}
\item [\colorbox{tagtype}{\color{white} \textbf{\textsf{RETURN}}}] \textbf{UNSIGNED3} --- The adjusted Time\_t value.
\end{description}




\rule{\linewidth}{0.5pt}
\subsection*{\textsf{\colorbox{headtoc}{\color{white} FUNCTION}
AdjustTimeBySeconds}}

\hypertarget{ecldoc:date.adjusttimebyseconds}{}
\hspace{0pt} \hyperlink{ecldoc:Date}{Date} \textbackslash 

{\renewcommand{\arraystretch}{1.5}
\begin{tabularx}{\textwidth}{|>{\raggedright\arraybackslash}l|X|}
\hline
\hspace{0pt}\mytexttt{\color{red} Time\_t} & \textbf{AdjustTimeBySeconds} \\
\hline
\multicolumn{2}{|>{\raggedright\arraybackslash}X|}{\hspace{0pt}\mytexttt{\color{param} (Time\_t time, INTEGER4 seconds\_delta)}} \\
\hline
\end{tabularx}
}

\par





Adjusts a time by adding or subtracting seconds. If the new calculated time is invalid then it will be normalized according to mktime() rules.






\par
\begin{description}
\item [\colorbox{tagtype}{\color{white} \textbf{\textsf{PARAMETER}}}] \textbf{\underline{time}} ||| UNSIGNED3 --- The time to adjust.
\item [\colorbox{tagtype}{\color{white} \textbf{\textsf{PARAMETER}}}] \textbf{\underline{seconds\_delta}} ||| INTEGER4 --- The requested change to the time, in seconds.
\end{description}







\par
\begin{description}
\item [\colorbox{tagtype}{\color{white} \textbf{\textsf{RETURN}}}] \textbf{UNSIGNED3} --- The adjusted Time\_t value.
\end{description}




\rule{\linewidth}{0.5pt}
\subsection*{\textsf{\colorbox{headtoc}{\color{white} FUNCTION}
AdjustSeconds}}

\hypertarget{ecldoc:date.adjustseconds}{}
\hspace{0pt} \hyperlink{ecldoc:Date}{Date} \textbackslash 

{\renewcommand{\arraystretch}{1.5}
\begin{tabularx}{\textwidth}{|>{\raggedright\arraybackslash}l|X|}
\hline
\hspace{0pt}\mytexttt{\color{red} Seconds\_t} & \textbf{AdjustSeconds} \\
\hline
\multicolumn{2}{|>{\raggedright\arraybackslash}X|}{\hspace{0pt}\mytexttt{\color{param} (Seconds\_t seconds, INTEGER2 year\_delta = 0, INTEGER4 month\_delta = 0, INTEGER4 day\_delta = 0, INTEGER4 hour\_delta = 0, INTEGER4 minute\_delta = 0, INTEGER4 second\_delta = 0)}} \\
\hline
\end{tabularx}
}

\par





Adjusts a Seconds\_t value by adding or subtracting years, months, days, hours, minutes and/or seconds. This is performed by first converting the seconds into a full date/time structure, applying any delta values to individual date/time components, then converting the structure back to the number of seconds. This interim date must lie within Gregorian calendar after the year 1600. If the interim structure is found to have an invalid date/time then it will be normalized according to mktime() rules. Therefore, some delta values (such as ''1 month'') are actually relative to the value of the seconds argument.






\par
\begin{description}
\item [\colorbox{tagtype}{\color{white} \textbf{\textsf{PARAMETER}}}] \textbf{\underline{month\_delta}} ||| INTEGER4 --- The requested change to the month value; optional, defaults to zero.
\item [\colorbox{tagtype}{\color{white} \textbf{\textsf{PARAMETER}}}] \textbf{\underline{second\_delta}} ||| INTEGER4 --- The requested change to the second of month value; optional, defaults to zero.
\item [\colorbox{tagtype}{\color{white} \textbf{\textsf{PARAMETER}}}] \textbf{\underline{seconds}} ||| INTEGER8 --- The number of seconds to adjust.
\item [\colorbox{tagtype}{\color{white} \textbf{\textsf{PARAMETER}}}] \textbf{\underline{day\_delta}} ||| INTEGER4 --- The requested change to the day of month value; optional, defaults to zero.
\item [\colorbox{tagtype}{\color{white} \textbf{\textsf{PARAMETER}}}] \textbf{\underline{minute\_delta}} ||| INTEGER4 --- The requested change to the minute value; optional, defaults to zero.
\item [\colorbox{tagtype}{\color{white} \textbf{\textsf{PARAMETER}}}] \textbf{\underline{year\_delta}} ||| INTEGER2 --- The requested change to the year value; optional, defaults to zero.
\item [\colorbox{tagtype}{\color{white} \textbf{\textsf{PARAMETER}}}] \textbf{\underline{hour\_delta}} ||| INTEGER4 --- The requested change to the hour value; optional, defaults to zero.
\end{description}







\par
\begin{description}
\item [\colorbox{tagtype}{\color{white} \textbf{\textsf{RETURN}}}] \textbf{INTEGER8} --- The adjusted Seconds\_t value.
\end{description}




\rule{\linewidth}{0.5pt}
\subsection*{\textsf{\colorbox{headtoc}{\color{white} FUNCTION}
AdjustCalendar}}

\hypertarget{ecldoc:date.adjustcalendar}{}
\hspace{0pt} \hyperlink{ecldoc:Date}{Date} \textbackslash 

{\renewcommand{\arraystretch}{1.5}
\begin{tabularx}{\textwidth}{|>{\raggedright\arraybackslash}l|X|}
\hline
\hspace{0pt}\mytexttt{\color{red} Date\_t} & \textbf{AdjustCalendar} \\
\hline
\multicolumn{2}{|>{\raggedright\arraybackslash}X|}{\hspace{0pt}\mytexttt{\color{param} (Date\_t date, INTEGER2 year\_delta = 0, INTEGER4 month\_delta = 0, INTEGER4 day\_delta = 0)}} \\
\hline
\end{tabularx}
}

\par





Adjusts a date by incrementing or decrementing months and/or years. This routine uses the rule outlined in McGinn v. State, 46 Neb. 427, 65 N.W. 46 (1895): ''The term calendar month, whether employed in statutes or contracts, and not appearing to have been used in a different sense, denotes a period terminating with the day of the succeeding month numerically corresponding to the day of its beginning, less one. If there be no corresponding day of the succeeding month, it terminates with the last day thereof.'' The internet suggests similar legal positions exist in the Commonwealth and Germany. Note that day adjustments are performed after year and month adjustments using the preceding rules. As an example, Jan. 31, 2014 + 1 month will result in Feb. 28, 2014; Jan. 31, 2014 + 1 month + 1 day will result in Mar. 1, 2014.






\par
\begin{description}
\item [\colorbox{tagtype}{\color{white} \textbf{\textsf{PARAMETER}}}] \textbf{\underline{date}} ||| UNSIGNED4 --- The date to adjust, in the Gregorian calendar after 1600.
\item [\colorbox{tagtype}{\color{white} \textbf{\textsf{PARAMETER}}}] \textbf{\underline{month\_delta}} ||| INTEGER4 --- The requested change to the month value; optional, defaults to zero.
\item [\colorbox{tagtype}{\color{white} \textbf{\textsf{PARAMETER}}}] \textbf{\underline{year\_delta}} ||| INTEGER2 --- The requested change to the year value; optional, defaults to zero.
\item [\colorbox{tagtype}{\color{white} \textbf{\textsf{PARAMETER}}}] \textbf{\underline{day\_delta}} ||| INTEGER4 --- The requested change to the day value; optional, defaults to zero.
\end{description}







\par
\begin{description}
\item [\colorbox{tagtype}{\color{white} \textbf{\textsf{RETURN}}}] \textbf{UNSIGNED4} --- The adjusted Date\_t value.
\end{description}




\rule{\linewidth}{0.5pt}
\subsection*{\textsf{\colorbox{headtoc}{\color{white} FUNCTION}
IsLocalDaylightSavingsInEffect}}

\hypertarget{ecldoc:date.islocaldaylightsavingsineffect}{}
\hspace{0pt} \hyperlink{ecldoc:Date}{Date} \textbackslash 

{\renewcommand{\arraystretch}{1.5}
\begin{tabularx}{\textwidth}{|>{\raggedright\arraybackslash}l|X|}
\hline
\hspace{0pt}\mytexttt{\color{red} BOOLEAN} & \textbf{IsLocalDaylightSavingsInEffect} \\
\hline
\multicolumn{2}{|>{\raggedright\arraybackslash}X|}{\hspace{0pt}\mytexttt{\color{param} ()}} \\
\hline
\end{tabularx}
}

\par





Returns a boolean indicating whether daylight savings time is currently in effect locally.








\par
\begin{description}
\item [\colorbox{tagtype}{\color{white} \textbf{\textsf{RETURN}}}] \textbf{BOOLEAN} --- TRUE if daylight savings time is currently in effect, FALSE otherwise.
\end{description}




\rule{\linewidth}{0.5pt}
\subsection*{\textsf{\colorbox{headtoc}{\color{white} FUNCTION}
LocalTimeZoneOffset}}

\hypertarget{ecldoc:date.localtimezoneoffset}{}
\hspace{0pt} \hyperlink{ecldoc:Date}{Date} \textbackslash 

{\renewcommand{\arraystretch}{1.5}
\begin{tabularx}{\textwidth}{|>{\raggedright\arraybackslash}l|X|}
\hline
\hspace{0pt}\mytexttt{\color{red} INTEGER4} & \textbf{LocalTimeZoneOffset} \\
\hline
\multicolumn{2}{|>{\raggedright\arraybackslash}X|}{\hspace{0pt}\mytexttt{\color{param} ()}} \\
\hline
\end{tabularx}
}

\par





Returns the offset (in seconds) of the time represented from UTC, with positive values indicating locations east of the Prime Meridian. Given a UTC time in seconds since epoch, you can find the local time by adding the result of this function to the seconds.








\par
\begin{description}
\item [\colorbox{tagtype}{\color{white} \textbf{\textsf{RETURN}}}] \textbf{INTEGER4} --- The number of seconds offset from UTC.
\end{description}




\rule{\linewidth}{0.5pt}
\subsection*{\textsf{\colorbox{headtoc}{\color{white} FUNCTION}
CurrentDate}}

\hypertarget{ecldoc:date.currentdate}{}
\hspace{0pt} \hyperlink{ecldoc:Date}{Date} \textbackslash 

{\renewcommand{\arraystretch}{1.5}
\begin{tabularx}{\textwidth}{|>{\raggedright\arraybackslash}l|X|}
\hline
\hspace{0pt}\mytexttt{\color{red} Date\_t} & \textbf{CurrentDate} \\
\hline
\multicolumn{2}{|>{\raggedright\arraybackslash}X|}{\hspace{0pt}\mytexttt{\color{param} (BOOLEAN in\_local\_time = FALSE)}} \\
\hline
\end{tabularx}
}

\par





Returns the current date.






\par
\begin{description}
\item [\colorbox{tagtype}{\color{white} \textbf{\textsf{PARAMETER}}}] \textbf{\underline{in\_local\_time}} ||| BOOLEAN --- TRUE if the returned value should be local to the cluster computing the date, FALSE for UTC. Optional, defaults to FALSE.
\end{description}







\par
\begin{description}
\item [\colorbox{tagtype}{\color{white} \textbf{\textsf{RETURN}}}] \textbf{UNSIGNED4} --- A Date\_t representing the current date.
\end{description}




\rule{\linewidth}{0.5pt}
\subsection*{\textsf{\colorbox{headtoc}{\color{white} FUNCTION}
Today}}

\hypertarget{ecldoc:date.today}{}
\hspace{0pt} \hyperlink{ecldoc:Date}{Date} \textbackslash 

{\renewcommand{\arraystretch}{1.5}
\begin{tabularx}{\textwidth}{|>{\raggedright\arraybackslash}l|X|}
\hline
\hspace{0pt}\mytexttt{\color{red} Date\_t} & \textbf{Today} \\
\hline
\multicolumn{2}{|>{\raggedright\arraybackslash}X|}{\hspace{0pt}\mytexttt{\color{param} ()}} \\
\hline
\end{tabularx}
}

\par





Returns the current date in the local time zone.








\par
\begin{description}
\item [\colorbox{tagtype}{\color{white} \textbf{\textsf{RETURN}}}] \textbf{UNSIGNED4} --- A Date\_t representing the current date.
\end{description}




\rule{\linewidth}{0.5pt}
\subsection*{\textsf{\colorbox{headtoc}{\color{white} FUNCTION}
CurrentTime}}

\hypertarget{ecldoc:date.currenttime}{}
\hspace{0pt} \hyperlink{ecldoc:Date}{Date} \textbackslash 

{\renewcommand{\arraystretch}{1.5}
\begin{tabularx}{\textwidth}{|>{\raggedright\arraybackslash}l|X|}
\hline
\hspace{0pt}\mytexttt{\color{red} Time\_t} & \textbf{CurrentTime} \\
\hline
\multicolumn{2}{|>{\raggedright\arraybackslash}X|}{\hspace{0pt}\mytexttt{\color{param} (BOOLEAN in\_local\_time = FALSE)}} \\
\hline
\end{tabularx}
}

\par





Returns the current time of day






\par
\begin{description}
\item [\colorbox{tagtype}{\color{white} \textbf{\textsf{PARAMETER}}}] \textbf{\underline{in\_local\_time}} ||| BOOLEAN --- TRUE if the returned value should be local to the cluster computing the time, FALSE for UTC. Optional, defaults to FALSE.
\end{description}







\par
\begin{description}
\item [\colorbox{tagtype}{\color{white} \textbf{\textsf{RETURN}}}] \textbf{UNSIGNED3} --- A Time\_t representing the current time of day.
\end{description}




\rule{\linewidth}{0.5pt}
\subsection*{\textsf{\colorbox{headtoc}{\color{white} FUNCTION}
CurrentSeconds}}

\hypertarget{ecldoc:date.currentseconds}{}
\hspace{0pt} \hyperlink{ecldoc:Date}{Date} \textbackslash 

{\renewcommand{\arraystretch}{1.5}
\begin{tabularx}{\textwidth}{|>{\raggedright\arraybackslash}l|X|}
\hline
\hspace{0pt}\mytexttt{\color{red} Seconds\_t} & \textbf{CurrentSeconds} \\
\hline
\multicolumn{2}{|>{\raggedright\arraybackslash}X|}{\hspace{0pt}\mytexttt{\color{param} (BOOLEAN in\_local\_time = FALSE)}} \\
\hline
\end{tabularx}
}

\par





Returns the current date and time as the number of seconds since epoch.






\par
\begin{description}
\item [\colorbox{tagtype}{\color{white} \textbf{\textsf{PARAMETER}}}] \textbf{\underline{in\_local\_time}} ||| BOOLEAN --- TRUE if the returned value should be local to the cluster computing the time, FALSE for UTC. Optional, defaults to FALSE.
\end{description}







\par
\begin{description}
\item [\colorbox{tagtype}{\color{white} \textbf{\textsf{RETURN}}}] \textbf{INTEGER8} --- A Seconds\_t representing the current time in UTC or local time, depending on the argument.
\end{description}




\rule{\linewidth}{0.5pt}
\subsection*{\textsf{\colorbox{headtoc}{\color{white} FUNCTION}
CurrentTimestamp}}

\hypertarget{ecldoc:date.currenttimestamp}{}
\hspace{0pt} \hyperlink{ecldoc:Date}{Date} \textbackslash 

{\renewcommand{\arraystretch}{1.5}
\begin{tabularx}{\textwidth}{|>{\raggedright\arraybackslash}l|X|}
\hline
\hspace{0pt}\mytexttt{\color{red} Timestamp\_t} & \textbf{CurrentTimestamp} \\
\hline
\multicolumn{2}{|>{\raggedright\arraybackslash}X|}{\hspace{0pt}\mytexttt{\color{param} (BOOLEAN in\_local\_time = FALSE)}} \\
\hline
\end{tabularx}
}

\par





Returns the current date and time as the number of microseconds since epoch.






\par
\begin{description}
\item [\colorbox{tagtype}{\color{white} \textbf{\textsf{PARAMETER}}}] \textbf{\underline{in\_local\_time}} ||| BOOLEAN --- TRUE if the returned value should be local to the cluster computing the time, FALSE for UTC. Optional, defaults to FALSE.
\end{description}







\par
\begin{description}
\item [\colorbox{tagtype}{\color{white} \textbf{\textsf{RETURN}}}] \textbf{INTEGER8} --- A Timestamp\_t representing the current time in microseconds in UTC or local time, depending on the argument.
\end{description}




\rule{\linewidth}{0.5pt}
\subsection*{\textsf{\colorbox{headtoc}{\color{white} MODULE}
DatesForMonth}}

\hypertarget{ecldoc:date.datesformonth}{}
\hspace{0pt} \hyperlink{ecldoc:Date}{Date} \textbackslash 

{\renewcommand{\arraystretch}{1.5}
\begin{tabularx}{\textwidth}{|>{\raggedright\arraybackslash}l|X|}
\hline
\hspace{0pt}\mytexttt{\color{red} } & \textbf{DatesForMonth} \\
\hline
\multicolumn{2}{|>{\raggedright\arraybackslash}X|}{\hspace{0pt}\mytexttt{\color{param} (Date\_t as\_of\_date = CurrentDate(FALSE))}} \\
\hline
\end{tabularx}
}

\par





Returns the beginning and ending dates for the month surrounding the given date.






\par
\begin{description}
\item [\colorbox{tagtype}{\color{white} \textbf{\textsf{PARAMETER}}}] \textbf{\underline{as\_of\_date}} ||| UNSIGNED4 --- The reference date from which the month will be calculated. This date must be a date within the Gregorian calendar. Optional, defaults to the current date in UTC.
\end{description}







\par
\begin{description}
\item [\colorbox{tagtype}{\color{white} \textbf{\textsf{RETURN}}}] \textbf{} --- Module with exported attributes for startDate and endDate.
\end{description}




\textbf{Children}
\begin{enumerate}
\item \hyperlink{ecldoc:date.datesformonth.result.startdate}{startDate}
: No Documentation Found
\item \hyperlink{ecldoc:date.datesformonth.result.enddate}{endDate}
: No Documentation Found
\end{enumerate}

\rule{\linewidth}{0.5pt}

\subsection*{\textsf{\colorbox{headtoc}{\color{white} ATTRIBUTE}
startDate}}

\hypertarget{ecldoc:date.datesformonth.result.startdate}{}
\hspace{0pt} \hyperlink{ecldoc:Date}{Date} \textbackslash 
\hspace{0pt} \hyperlink{ecldoc:date.datesformonth}{DatesForMonth} \textbackslash 

{\renewcommand{\arraystretch}{1.5}
\begin{tabularx}{\textwidth}{|>{\raggedright\arraybackslash}l|X|}
\hline
\hspace{0pt}\mytexttt{\color{red} Date\_t} & \textbf{startDate} \\
\hline
\end{tabularx}
}

\par





No Documentation Found








\par
\begin{description}
\item [\colorbox{tagtype}{\color{white} \textbf{\textsf{RETURN}}}] \textbf{UNSIGNED4} --- 
\end{description}




\rule{\linewidth}{0.5pt}
\subsection*{\textsf{\colorbox{headtoc}{\color{white} ATTRIBUTE}
endDate}}

\hypertarget{ecldoc:date.datesformonth.result.enddate}{}
\hspace{0pt} \hyperlink{ecldoc:Date}{Date} \textbackslash 
\hspace{0pt} \hyperlink{ecldoc:date.datesformonth}{DatesForMonth} \textbackslash 

{\renewcommand{\arraystretch}{1.5}
\begin{tabularx}{\textwidth}{|>{\raggedright\arraybackslash}l|X|}
\hline
\hspace{0pt}\mytexttt{\color{red} Date\_t} & \textbf{endDate} \\
\hline
\end{tabularx}
}

\par





No Documentation Found








\par
\begin{description}
\item [\colorbox{tagtype}{\color{white} \textbf{\textsf{RETURN}}}] \textbf{UNSIGNED4} --- 
\end{description}




\rule{\linewidth}{0.5pt}


\subsection*{\textsf{\colorbox{headtoc}{\color{white} MODULE}
DatesForWeek}}

\hypertarget{ecldoc:date.datesforweek}{}
\hspace{0pt} \hyperlink{ecldoc:Date}{Date} \textbackslash 

{\renewcommand{\arraystretch}{1.5}
\begin{tabularx}{\textwidth}{|>{\raggedright\arraybackslash}l|X|}
\hline
\hspace{0pt}\mytexttt{\color{red} } & \textbf{DatesForWeek} \\
\hline
\multicolumn{2}{|>{\raggedright\arraybackslash}X|}{\hspace{0pt}\mytexttt{\color{param} (Date\_t as\_of\_date = CurrentDate(FALSE))}} \\
\hline
\end{tabularx}
}

\par





Returns the beginning and ending dates for the week surrounding the given date (Sunday marks the beginning of a week).






\par
\begin{description}
\item [\colorbox{tagtype}{\color{white} \textbf{\textsf{PARAMETER}}}] \textbf{\underline{as\_of\_date}} ||| UNSIGNED4 --- The reference date from which the week will be calculated. This date must be a date within the Gregorian calendar. Optional, defaults to the current date in UTC.
\end{description}







\par
\begin{description}
\item [\colorbox{tagtype}{\color{white} \textbf{\textsf{RETURN}}}] \textbf{} --- Module with exported attributes for startDate and endDate.
\end{description}




\textbf{Children}
\begin{enumerate}
\item \hyperlink{ecldoc:date.datesforweek.result.startdate}{startDate}
: No Documentation Found
\item \hyperlink{ecldoc:date.datesforweek.result.enddate}{endDate}
: No Documentation Found
\end{enumerate}

\rule{\linewidth}{0.5pt}

\subsection*{\textsf{\colorbox{headtoc}{\color{white} ATTRIBUTE}
startDate}}

\hypertarget{ecldoc:date.datesforweek.result.startdate}{}
\hspace{0pt} \hyperlink{ecldoc:Date}{Date} \textbackslash 
\hspace{0pt} \hyperlink{ecldoc:date.datesforweek}{DatesForWeek} \textbackslash 

{\renewcommand{\arraystretch}{1.5}
\begin{tabularx}{\textwidth}{|>{\raggedright\arraybackslash}l|X|}
\hline
\hspace{0pt}\mytexttt{\color{red} Date\_t} & \textbf{startDate} \\
\hline
\end{tabularx}
}

\par





No Documentation Found








\par
\begin{description}
\item [\colorbox{tagtype}{\color{white} \textbf{\textsf{RETURN}}}] \textbf{UNSIGNED4} --- 
\end{description}




\rule{\linewidth}{0.5pt}
\subsection*{\textsf{\colorbox{headtoc}{\color{white} ATTRIBUTE}
endDate}}

\hypertarget{ecldoc:date.datesforweek.result.enddate}{}
\hspace{0pt} \hyperlink{ecldoc:Date}{Date} \textbackslash 
\hspace{0pt} \hyperlink{ecldoc:date.datesforweek}{DatesForWeek} \textbackslash 

{\renewcommand{\arraystretch}{1.5}
\begin{tabularx}{\textwidth}{|>{\raggedright\arraybackslash}l|X|}
\hline
\hspace{0pt}\mytexttt{\color{red} Date\_t} & \textbf{endDate} \\
\hline
\end{tabularx}
}

\par





No Documentation Found








\par
\begin{description}
\item [\colorbox{tagtype}{\color{white} \textbf{\textsf{RETURN}}}] \textbf{UNSIGNED4} --- 
\end{description}




\rule{\linewidth}{0.5pt}


\subsection*{\textsf{\colorbox{headtoc}{\color{white} FUNCTION}
IsValidDate}}

\hypertarget{ecldoc:date.isvaliddate}{}
\hspace{0pt} \hyperlink{ecldoc:Date}{Date} \textbackslash 

{\renewcommand{\arraystretch}{1.5}
\begin{tabularx}{\textwidth}{|>{\raggedright\arraybackslash}l|X|}
\hline
\hspace{0pt}\mytexttt{\color{red} BOOLEAN} & \textbf{IsValidDate} \\
\hline
\multicolumn{2}{|>{\raggedright\arraybackslash}X|}{\hspace{0pt}\mytexttt{\color{param} (Date\_t date, INTEGER2 yearLowerBound = 1800, INTEGER2 yearUpperBound = 2100)}} \\
\hline
\end{tabularx}
}

\par





Tests whether a date is valid, both by range-checking the year and by validating each of the other individual components.






\par
\begin{description}
\item [\colorbox{tagtype}{\color{white} \textbf{\textsf{PARAMETER}}}] \textbf{\underline{date}} ||| UNSIGNED4 --- The date to validate.
\item [\colorbox{tagtype}{\color{white} \textbf{\textsf{PARAMETER}}}] \textbf{\underline{yearUpperBound}} ||| INTEGER2 --- The maximum acceptable year. Optional; defaults to 2100.
\item [\colorbox{tagtype}{\color{white} \textbf{\textsf{PARAMETER}}}] \textbf{\underline{yearLowerBound}} ||| INTEGER2 --- The minimum acceptable year. Optional; defaults to 1800.
\end{description}







\par
\begin{description}
\item [\colorbox{tagtype}{\color{white} \textbf{\textsf{RETURN}}}] \textbf{BOOLEAN} --- TRUE if the date is valid, FALSE otherwise.
\end{description}




\rule{\linewidth}{0.5pt}
\subsection*{\textsf{\colorbox{headtoc}{\color{white} FUNCTION}
IsValidGregorianDate}}

\hypertarget{ecldoc:date.isvalidgregoriandate}{}
\hspace{0pt} \hyperlink{ecldoc:Date}{Date} \textbackslash 

{\renewcommand{\arraystretch}{1.5}
\begin{tabularx}{\textwidth}{|>{\raggedright\arraybackslash}l|X|}
\hline
\hspace{0pt}\mytexttt{\color{red} BOOLEAN} & \textbf{IsValidGregorianDate} \\
\hline
\multicolumn{2}{|>{\raggedright\arraybackslash}X|}{\hspace{0pt}\mytexttt{\color{param} (Date\_t date)}} \\
\hline
\end{tabularx}
}

\par





Tests whether a date is valid in the Gregorian calendar. The year must be between 1601 and 30827.






\par
\begin{description}
\item [\colorbox{tagtype}{\color{white} \textbf{\textsf{PARAMETER}}}] \textbf{\underline{date}} ||| UNSIGNED4 --- The Date\_t to validate.
\end{description}







\par
\begin{description}
\item [\colorbox{tagtype}{\color{white} \textbf{\textsf{RETURN}}}] \textbf{BOOLEAN} --- TRUE if the date is valid, FALSE otherwise.
\end{description}




\rule{\linewidth}{0.5pt}
\subsection*{\textsf{\colorbox{headtoc}{\color{white} FUNCTION}
IsValidTime}}

\hypertarget{ecldoc:date.isvalidtime}{}
\hspace{0pt} \hyperlink{ecldoc:Date}{Date} \textbackslash 

{\renewcommand{\arraystretch}{1.5}
\begin{tabularx}{\textwidth}{|>{\raggedright\arraybackslash}l|X|}
\hline
\hspace{0pt}\mytexttt{\color{red} BOOLEAN} & \textbf{IsValidTime} \\
\hline
\multicolumn{2}{|>{\raggedright\arraybackslash}X|}{\hspace{0pt}\mytexttt{\color{param} (Time\_t time)}} \\
\hline
\end{tabularx}
}

\par





Tests whether a time is valid.






\par
\begin{description}
\item [\colorbox{tagtype}{\color{white} \textbf{\textsf{PARAMETER}}}] \textbf{\underline{time}} ||| UNSIGNED3 --- The time to validate.
\end{description}







\par
\begin{description}
\item [\colorbox{tagtype}{\color{white} \textbf{\textsf{RETURN}}}] \textbf{BOOLEAN} --- TRUE if the time is valid, FALSE otherwise.
\end{description}




\rule{\linewidth}{0.5pt}
\subsection*{\textsf{\colorbox{headtoc}{\color{white} TRANSFORM}
CreateDate}}

\hypertarget{ecldoc:date.createdate}{}
\hspace{0pt} \hyperlink{ecldoc:Date}{Date} \textbackslash 

{\renewcommand{\arraystretch}{1.5}
\begin{tabularx}{\textwidth}{|>{\raggedright\arraybackslash}l|X|}
\hline
\hspace{0pt}\mytexttt{\color{red} Date\_rec} & \textbf{CreateDate} \\
\hline
\multicolumn{2}{|>{\raggedright\arraybackslash}X|}{\hspace{0pt}\mytexttt{\color{param} (INTEGER2 year, UNSIGNED1 month, UNSIGNED1 day)}} \\
\hline
\end{tabularx}
}

\par





A transform to create a Date\_rec from the individual elements






\par
\begin{description}
\item [\colorbox{tagtype}{\color{white} \textbf{\textsf{PARAMETER}}}] \textbf{\underline{year}} ||| INTEGER2 --- The year
\item [\colorbox{tagtype}{\color{white} \textbf{\textsf{PARAMETER}}}] \textbf{\underline{month}} ||| UNSIGNED1 --- The month (1-12).
\item [\colorbox{tagtype}{\color{white} \textbf{\textsf{PARAMETER}}}] \textbf{\underline{day}} ||| UNSIGNED1 --- The day (1..daysInMonth).
\end{description}







\par
\begin{description}
\item [\colorbox{tagtype}{\color{white} \textbf{\textsf{RETURN}}}] \textbf{Date\_rec} --- A transform that creates a Date\_rec containing the date.
\end{description}




\rule{\linewidth}{0.5pt}
\subsection*{\textsf{\colorbox{headtoc}{\color{white} TRANSFORM}
CreateDateFromSeconds}}

\hypertarget{ecldoc:date.createdatefromseconds}{}
\hspace{0pt} \hyperlink{ecldoc:Date}{Date} \textbackslash 

{\renewcommand{\arraystretch}{1.5}
\begin{tabularx}{\textwidth}{|>{\raggedright\arraybackslash}l|X|}
\hline
\hspace{0pt}\mytexttt{\color{red} Date\_rec} & \textbf{CreateDateFromSeconds} \\
\hline
\multicolumn{2}{|>{\raggedright\arraybackslash}X|}{\hspace{0pt}\mytexttt{\color{param} (Seconds\_t seconds)}} \\
\hline
\end{tabularx}
}

\par





A transform to create a Date\_rec from a Seconds\_t value.






\par
\begin{description}
\item [\colorbox{tagtype}{\color{white} \textbf{\textsf{PARAMETER}}}] \textbf{\underline{seconds}} ||| INTEGER8 --- The number seconds since epoch.
\end{description}







\par
\begin{description}
\item [\colorbox{tagtype}{\color{white} \textbf{\textsf{RETURN}}}] \textbf{Date\_rec} --- A transform that creates a Date\_rec containing the date.
\end{description}




\rule{\linewidth}{0.5pt}
\subsection*{\textsf{\colorbox{headtoc}{\color{white} TRANSFORM}
CreateTime}}

\hypertarget{ecldoc:date.createtime}{}
\hspace{0pt} \hyperlink{ecldoc:Date}{Date} \textbackslash 

{\renewcommand{\arraystretch}{1.5}
\begin{tabularx}{\textwidth}{|>{\raggedright\arraybackslash}l|X|}
\hline
\hspace{0pt}\mytexttt{\color{red} Time\_rec} & \textbf{CreateTime} \\
\hline
\multicolumn{2}{|>{\raggedright\arraybackslash}X|}{\hspace{0pt}\mytexttt{\color{param} (UNSIGNED1 hour, UNSIGNED1 minute, UNSIGNED1 second)}} \\
\hline
\end{tabularx}
}

\par





A transform to create a Time\_rec from the individual elements






\par
\begin{description}
\item [\colorbox{tagtype}{\color{white} \textbf{\textsf{PARAMETER}}}] \textbf{\underline{minute}} ||| UNSIGNED1 --- The minute (0-59).
\item [\colorbox{tagtype}{\color{white} \textbf{\textsf{PARAMETER}}}] \textbf{\underline{second}} ||| UNSIGNED1 --- The second (0-59).
\item [\colorbox{tagtype}{\color{white} \textbf{\textsf{PARAMETER}}}] \textbf{\underline{hour}} ||| UNSIGNED1 --- The hour (0-23).
\end{description}







\par
\begin{description}
\item [\colorbox{tagtype}{\color{white} \textbf{\textsf{RETURN}}}] \textbf{Time\_rec} --- A transform that creates a Time\_rec containing the time of day.
\end{description}




\rule{\linewidth}{0.5pt}
\subsection*{\textsf{\colorbox{headtoc}{\color{white} TRANSFORM}
CreateTimeFromSeconds}}

\hypertarget{ecldoc:date.createtimefromseconds}{}
\hspace{0pt} \hyperlink{ecldoc:Date}{Date} \textbackslash 

{\renewcommand{\arraystretch}{1.5}
\begin{tabularx}{\textwidth}{|>{\raggedright\arraybackslash}l|X|}
\hline
\hspace{0pt}\mytexttt{\color{red} Time\_rec} & \textbf{CreateTimeFromSeconds} \\
\hline
\multicolumn{2}{|>{\raggedright\arraybackslash}X|}{\hspace{0pt}\mytexttt{\color{param} (Seconds\_t seconds)}} \\
\hline
\end{tabularx}
}

\par





A transform to create a Time\_rec from a Seconds\_t value.






\par
\begin{description}
\item [\colorbox{tagtype}{\color{white} \textbf{\textsf{PARAMETER}}}] \textbf{\underline{seconds}} ||| INTEGER8 --- The number seconds since epoch.
\end{description}







\par
\begin{description}
\item [\colorbox{tagtype}{\color{white} \textbf{\textsf{RETURN}}}] \textbf{Time\_rec} --- A transform that creates a Time\_rec containing the time of day.
\end{description}




\rule{\linewidth}{0.5pt}
\subsection*{\textsf{\colorbox{headtoc}{\color{white} TRANSFORM}
CreateDateTime}}

\hypertarget{ecldoc:date.createdatetime}{}
\hspace{0pt} \hyperlink{ecldoc:Date}{Date} \textbackslash 

{\renewcommand{\arraystretch}{1.5}
\begin{tabularx}{\textwidth}{|>{\raggedright\arraybackslash}l|X|}
\hline
\hspace{0pt}\mytexttt{\color{red} DateTime\_rec} & \textbf{CreateDateTime} \\
\hline
\multicolumn{2}{|>{\raggedright\arraybackslash}X|}{\hspace{0pt}\mytexttt{\color{param} (INTEGER2 year, UNSIGNED1 month, UNSIGNED1 day, UNSIGNED1 hour, UNSIGNED1 minute, UNSIGNED1 second)}} \\
\hline
\end{tabularx}
}

\par





A transform to create a DateTime\_rec from the individual elements






\par
\begin{description}
\item [\colorbox{tagtype}{\color{white} \textbf{\textsf{PARAMETER}}}] \textbf{\underline{year}} ||| INTEGER2 --- The year
\item [\colorbox{tagtype}{\color{white} \textbf{\textsf{PARAMETER}}}] \textbf{\underline{second}} ||| UNSIGNED1 --- The second (0-59).
\item [\colorbox{tagtype}{\color{white} \textbf{\textsf{PARAMETER}}}] \textbf{\underline{hour}} ||| UNSIGNED1 --- The hour (0-23).
\item [\colorbox{tagtype}{\color{white} \textbf{\textsf{PARAMETER}}}] \textbf{\underline{minute}} ||| UNSIGNED1 --- The minute (0-59).
\item [\colorbox{tagtype}{\color{white} \textbf{\textsf{PARAMETER}}}] \textbf{\underline{month}} ||| UNSIGNED1 --- The month (1-12).
\item [\colorbox{tagtype}{\color{white} \textbf{\textsf{PARAMETER}}}] \textbf{\underline{day}} ||| UNSIGNED1 --- The day (1..daysInMonth).
\end{description}







\par
\begin{description}
\item [\colorbox{tagtype}{\color{white} \textbf{\textsf{RETURN}}}] \textbf{DateTime\_rec} --- A transform that creates a DateTime\_rec containing date and time components.
\end{description}




\rule{\linewidth}{0.5pt}
\subsection*{\textsf{\colorbox{headtoc}{\color{white} TRANSFORM}
CreateDateTimeFromSeconds}}

\hypertarget{ecldoc:date.createdatetimefromseconds}{}
\hspace{0pt} \hyperlink{ecldoc:Date}{Date} \textbackslash 

{\renewcommand{\arraystretch}{1.5}
\begin{tabularx}{\textwidth}{|>{\raggedright\arraybackslash}l|X|}
\hline
\hspace{0pt}\mytexttt{\color{red} DateTime\_rec} & \textbf{CreateDateTimeFromSeconds} \\
\hline
\multicolumn{2}{|>{\raggedright\arraybackslash}X|}{\hspace{0pt}\mytexttt{\color{param} (Seconds\_t seconds)}} \\
\hline
\end{tabularx}
}

\par





A transform to create a DateTime\_rec from a Seconds\_t value.






\par
\begin{description}
\item [\colorbox{tagtype}{\color{white} \textbf{\textsf{PARAMETER}}}] \textbf{\underline{seconds}} ||| INTEGER8 --- The number seconds since epoch.
\end{description}







\par
\begin{description}
\item [\colorbox{tagtype}{\color{white} \textbf{\textsf{RETURN}}}] \textbf{DateTime\_rec} --- A transform that creates a DateTime\_rec containing date and time components.
\end{description}




\rule{\linewidth}{0.5pt}



\chapter*{\color{headfile}
File
}
\hypertarget{ecldoc:toc:File}{}
\hyperlink{ecldoc:toc:root}{Go Up}

\section*{\underline{\textsf{IMPORTS}}}
\begin{doublespace}
{\large
lib\_fileservices |
}
\end{doublespace}

\section*{\underline{\textsf{DESCRIPTIONS}}}
\subsection*{\textsf{\colorbox{headtoc}{\color{white} MODULE}
File}}

\hypertarget{ecldoc:File}{}

{\renewcommand{\arraystretch}{1.5}
\begin{tabularx}{\textwidth}{|>{\raggedright\arraybackslash}l|X|}
\hline
\hspace{0pt}\mytexttt{\color{red} } & \textbf{File} \\
\hline
\end{tabularx}
}

\par


\textbf{Children}
\begin{enumerate}
\item \hyperlink{ecldoc:file.fsfilenamerecord}{FsFilenameRecord}
: A record containing information about filename
\item \hyperlink{ecldoc:file.fslogicalfilename}{FsLogicalFileName}
: An alias for a logical filename that is stored in a row
\item \hyperlink{ecldoc:file.fslogicalfilenamerecord}{FsLogicalFileNameRecord}
: A record containing a logical filename
\item \hyperlink{ecldoc:file.fslogicalfileinforecord}{FsLogicalFileInfoRecord}
: A record containing information about a logical file
\item \hyperlink{ecldoc:file.fslogicalsupersubrecord}{FsLogicalSuperSubRecord}
: A record containing information about a superfile and its contents
\item \hyperlink{ecldoc:file.fsfilerelationshiprecord}{FsFileRelationshipRecord}
: A record containing information about the relationship between two files
\item \hyperlink{ecldoc:file.recfmv_recsize}{RECFMV\_RECSIZE}
: Constant that indicates IBM RECFM V format file
\item \hyperlink{ecldoc:file.recfmvb_recsize}{RECFMVB\_RECSIZE}
: Constant that indicates IBM RECFM VB format file
\item \hyperlink{ecldoc:file.prefix_variable_recsize}{PREFIX\_VARIABLE\_RECSIZE}
: Constant that indicates a variable little endian 4 byte length prefixed file
\item \hyperlink{ecldoc:file.prefix_variable_bigendian_recsize}{PREFIX\_VARIABLE\_BIGENDIAN\_RECSIZE}
: Constant that indicates a variable big endian 4 byte length prefixed file
\item \hyperlink{ecldoc:file.fileexists}{FileExists}
: Returns whether the file exists
\item \hyperlink{ecldoc:file.deletelogicalfile}{DeleteLogicalFile}
: Removes the logical file from the system, and deletes from the disk
\item \hyperlink{ecldoc:file.setreadonly}{SetReadOnly}
: Changes whether access to a file is read only or not
\item \hyperlink{ecldoc:file.renamelogicalfile}{RenameLogicalFile}
: Changes the name of a logical file
\item \hyperlink{ecldoc:file.foreignlogicalfilename}{ForeignLogicalFileName}
: Returns a logical filename that can be used to refer to a logical file in a local or remote dali
\item \hyperlink{ecldoc:file.externallogicalfilename}{ExternalLogicalFileName}
: Returns an encoded logical filename that can be used to refer to a external file
\item \hyperlink{ecldoc:file.getfiledescription}{GetFileDescription}
: Returns a string containing the description information associated with the specified filename
\item \hyperlink{ecldoc:file.setfiledescription}{SetFileDescription}
: Sets the description associated with the specified filename
\item \hyperlink{ecldoc:file.remotedirectory}{RemoteDirectory}
: Returns a dataset containing a list of files from the specified machineIP and directory
\item \hyperlink{ecldoc:file.logicalfilelist}{LogicalFileList}
: Returns a dataset of information about the logical files known to the system
\item \hyperlink{ecldoc:file.comparefiles}{CompareFiles}
: Compares two files, and returns a result indicating how well they match
\item \hyperlink{ecldoc:file.verifyfile}{VerifyFile}
: Checks the system datastore (Dali) information for the file against the physical parts on disk
\item \hyperlink{ecldoc:file.addfilerelationship}{AddFileRelationship}
: Defines the relationship between two files
\item \hyperlink{ecldoc:file.filerelationshiplist}{FileRelationshipList}
: Returns a dataset of relationships
\item \hyperlink{ecldoc:file.removefilerelationship}{RemoveFileRelationship}
: Removes a relationship between two files
\item \hyperlink{ecldoc:file.getcolumnmapping}{GetColumnMapping}
: Returns the field mappings for the file, in the same format specified for the SetColumnMapping function
\item \hyperlink{ecldoc:file.setcolumnmapping}{SetColumnMapping}
: Defines how the data in the fields of the file mist be transformed between the actual data storage format and the input format used to query that data
\item \hyperlink{ecldoc:file.encoderfsquery}{EncodeRfsQuery}
: Returns a string that can be used in a DATASET declaration to read data from an RFS (Remote File Server) instance (e.g
\item \hyperlink{ecldoc:file.rfsaction}{RfsAction}
: Sends the query to the rfs server
\item \hyperlink{ecldoc:file.moveexternalfile}{MoveExternalFile}
: Moves the single physical file between two locations on the same remote machine
\item \hyperlink{ecldoc:file.deleteexternalfile}{DeleteExternalFile}
: Removes a single physical file from a remote machine
\item \hyperlink{ecldoc:file.createexternaldirectory}{CreateExternalDirectory}
: Creates the path on the location (if it does not already exist)
\item \hyperlink{ecldoc:file.getlogicalfileattribute}{GetLogicalFileAttribute}
: Returns the value of the given attribute for the specified logicalfilename
\item \hyperlink{ecldoc:file.protectlogicalfile}{ProtectLogicalFile}
: Toggles protection on and off for the specified logicalfilename
\item \hyperlink{ecldoc:file.dfuplusexec}{DfuPlusExec}
: The DfuPlusExec action executes the specified command line just as the DfuPLus.exe program would do
\item \hyperlink{ecldoc:file.fsprayfixed}{fSprayFixed}
: Sprays a file of fixed length records from a single machine and distributes it across the nodes of the destination group
\item \hyperlink{ecldoc:file.sprayfixed}{SprayFixed}
: Same as fSprayFixed, but does not return the DFU Workunit ID
\item \hyperlink{ecldoc:file.fsprayvariable}{fSprayVariable}
\item \hyperlink{ecldoc:file.sprayvariable}{SprayVariable}
\item \hyperlink{ecldoc:file.fspraydelimited}{fSprayDelimited}
: Sprays a file of fixed delimited records from a single machine and distributes it across the nodes of the destination group
\item \hyperlink{ecldoc:file.spraydelimited}{SprayDelimited}
: Same as fSprayDelimited, but does not return the DFU Workunit ID
\item \hyperlink{ecldoc:file.fsprayxml}{fSprayXml}
: Sprays an xml file from a single machine and distributes it across the nodes of the destination group
\item \hyperlink{ecldoc:file.sprayxml}{SprayXml}
: Same as fSprayXml, but does not return the DFU Workunit ID
\item \hyperlink{ecldoc:file.fdespray}{fDespray}
: Copies a distributed file from multiple machines, and desprays it to a single file on a single machine
\item \hyperlink{ecldoc:file.despray}{Despray}
: Same as fDespray, but does not return the DFU Workunit ID
\item \hyperlink{ecldoc:file.fcopy}{fCopy}
: Copies a distributed file to another distributed file
\item \hyperlink{ecldoc:file.copy}{Copy}
: Same as fCopy, but does not return the DFU Workunit ID
\item \hyperlink{ecldoc:file.freplicate}{fReplicate}
: Ensures the specified file is replicated to its mirror copies
\item \hyperlink{ecldoc:file.replicate}{Replicate}
: Same as fReplicated, but does not return the DFU Workunit ID
\item \hyperlink{ecldoc:file.fremotepull}{fRemotePull}
: Copies a distributed file to a distributed file on remote system
\item \hyperlink{ecldoc:file.remotepull}{RemotePull}
: Same as fRemotePull, but does not return the DFU Workunit ID
\item \hyperlink{ecldoc:file.fmonitorlogicalfilename}{fMonitorLogicalFileName}
: Creates a file monitor job in the DFU Server
\item \hyperlink{ecldoc:file.monitorlogicalfilename}{MonitorLogicalFileName}
: Same as fMonitorLogicalFileName, but does not return the DFU Workunit ID
\item \hyperlink{ecldoc:file.fmonitorfile}{fMonitorFile}
: Creates a file monitor job in the DFU Server
\item \hyperlink{ecldoc:file.monitorfile}{MonitorFile}
: Same as fMonitorFile, but does not return the DFU Workunit ID
\item \hyperlink{ecldoc:file.waitdfuworkunit}{WaitDfuWorkunit}
: Waits for the specified DFU workunit to finish
\item \hyperlink{ecldoc:file.abortdfuworkunit}{AbortDfuWorkunit}
: Aborts the specified DFU workunit
\item \hyperlink{ecldoc:file.createsuperfile}{CreateSuperFile}
: Creates an empty superfile
\item \hyperlink{ecldoc:file.superfileexists}{SuperFileExists}
: Checks if the specified filename is present in the Distributed File Utility (DFU) and is a SuperFile
\item \hyperlink{ecldoc:file.deletesuperfile}{DeleteSuperFile}
: Deletes the superfile
\item \hyperlink{ecldoc:file.getsuperfilesubcount}{GetSuperFileSubCount}
: Returns the number of sub-files contained within a superfile
\item \hyperlink{ecldoc:file.getsuperfilesubname}{GetSuperFileSubName}
: Returns the name of the Nth sub-file within a superfile
\item \hyperlink{ecldoc:file.findsuperfilesubname}{FindSuperFileSubName}
: Returns the position of a file within a superfile
\item \hyperlink{ecldoc:file.startsuperfiletransaction}{StartSuperFileTransaction}
: Starts a superfile transaction
\item \hyperlink{ecldoc:file.addsuperfile}{AddSuperFile}
: Adds a file to a superfile
\item \hyperlink{ecldoc:file.removesuperfile}{RemoveSuperFile}
: Removes a sub-file from a superfile
\item \hyperlink{ecldoc:file.clearsuperfile}{ClearSuperFile}
: Removes all sub-files from a superfile
\item \hyperlink{ecldoc:file.removeownedsubfiles}{RemoveOwnedSubFiles}
: Removes all soley-owned sub-files from a superfile
\item \hyperlink{ecldoc:file.deleteownedsubfiles}{DeleteOwnedSubFiles}
: Legacy version of RemoveOwnedSubFiles which was incorrectly named in a previous version
\item \hyperlink{ecldoc:file.swapsuperfile}{SwapSuperFile}
: Swap the contents of two superfiles
\item \hyperlink{ecldoc:file.replacesuperfile}{ReplaceSuperFile}
: Removes a sub-file from a superfile and replaces it with another
\item \hyperlink{ecldoc:file.finishsuperfiletransaction}{FinishSuperFileTransaction}
: Finishes a superfile transaction
\item \hyperlink{ecldoc:file.superfilecontents}{SuperFileContents}
: Returns the list of sub-files contained within a superfile
\item \hyperlink{ecldoc:file.logicalfilesuperowners}{LogicalFileSuperOwners}
: Returns the list of superfiles that a logical file is contained within
\item \hyperlink{ecldoc:file.logicalfilesupersublist}{LogicalFileSuperSubList}
: Returns the list of all the superfiles in the system and their component sub-files
\item \hyperlink{ecldoc:file.fpromotesuperfilelist}{fPromoteSuperFileList}
: Moves the sub-files from the first entry in the list of superfiles to the next in the list, repeating the process through the list of superfiles
\item \hyperlink{ecldoc:file.promotesuperfilelist}{PromoteSuperFileList}
: Same as fPromoteSuperFileList, but does not return the DFU Workunit ID
\end{enumerate}

\rule{\linewidth}{0.5pt}

\subsection*{\textsf{\colorbox{headtoc}{\color{white} RECORD}
FsFilenameRecord}}

\hypertarget{ecldoc:file.fsfilenamerecord}{}
\hspace{0pt} \hyperlink{ecldoc:File}{File} \textbackslash 

{\renewcommand{\arraystretch}{1.5}
\begin{tabularx}{\textwidth}{|>{\raggedright\arraybackslash}l|X|}
\hline
\hspace{0pt}\mytexttt{\color{red} } & \textbf{FsFilenameRecord} \\
\hline
\end{tabularx}
}

\par
A record containing information about filename. Includes name, size and when last modified. export FsFilenameRecord := RECORD string name; integer8 size; string19 modified; END;


\rule{\linewidth}{0.5pt}
\subsection*{\textsf{\colorbox{headtoc}{\color{white} ATTRIBUTE}
FsLogicalFileName}}

\hypertarget{ecldoc:file.fslogicalfilename}{}
\hspace{0pt} \hyperlink{ecldoc:File}{File} \textbackslash 

{\renewcommand{\arraystretch}{1.5}
\begin{tabularx}{\textwidth}{|>{\raggedright\arraybackslash}l|X|}
\hline
\hspace{0pt}\mytexttt{\color{red} } & \textbf{FsLogicalFileName} \\
\hline
\end{tabularx}
}

\par
An alias for a logical filename that is stored in a row.


\rule{\linewidth}{0.5pt}
\subsection*{\textsf{\colorbox{headtoc}{\color{white} RECORD}
FsLogicalFileNameRecord}}

\hypertarget{ecldoc:file.fslogicalfilenamerecord}{}
\hspace{0pt} \hyperlink{ecldoc:File}{File} \textbackslash 

{\renewcommand{\arraystretch}{1.5}
\begin{tabularx}{\textwidth}{|>{\raggedright\arraybackslash}l|X|}
\hline
\hspace{0pt}\mytexttt{\color{red} } & \textbf{FsLogicalFileNameRecord} \\
\hline
\end{tabularx}
}

\par
A record containing a logical filename. It contains the following fields:

\par
\begin{description}
\item [\colorbox{tagtype}{\color{white} \textbf{\textsf{FIELD}}}] \textbf{\underline{name}} The logical name of the file;
\end{description}

\rule{\linewidth}{0.5pt}
\subsection*{\textsf{\colorbox{headtoc}{\color{white} RECORD}
FsLogicalFileInfoRecord}}

\hypertarget{ecldoc:file.fslogicalfileinforecord}{}
\hspace{0pt} \hyperlink{ecldoc:File}{File} \textbackslash 

{\renewcommand{\arraystretch}{1.5}
\begin{tabularx}{\textwidth}{|>{\raggedright\arraybackslash}l|X|}
\hline
\hspace{0pt}\mytexttt{\color{red} } & \textbf{FsLogicalFileInfoRecord} \\
\hline
\end{tabularx}
}

\par
A record containing information about a logical file.

\par
\begin{description}
\item [\colorbox{tagtype}{\color{white} \textbf{\textsf{FIELD}}}] \textbf{\underline{superfile}} Is this a superfile?
\item [\colorbox{tagtype}{\color{white} \textbf{\textsf{FIELD}}}] \textbf{\underline{size}} Number of bytes in the file (before compression)
\item [\colorbox{tagtype}{\color{white} \textbf{\textsf{FIELD}}}] \textbf{\underline{rowcount}} Number of rows in the file.
\end{description}

\rule{\linewidth}{0.5pt}
\subsection*{\textsf{\colorbox{headtoc}{\color{white} RECORD}
FsLogicalSuperSubRecord}}

\hypertarget{ecldoc:file.fslogicalsupersubrecord}{}
\hspace{0pt} \hyperlink{ecldoc:File}{File} \textbackslash 

{\renewcommand{\arraystretch}{1.5}
\begin{tabularx}{\textwidth}{|>{\raggedright\arraybackslash}l|X|}
\hline
\hspace{0pt}\mytexttt{\color{red} } & \textbf{FsLogicalSuperSubRecord} \\
\hline
\end{tabularx}
}

\par
A record containing information about a superfile and its contents.

\par
\begin{description}
\item [\colorbox{tagtype}{\color{white} \textbf{\textsf{FIELD}}}] \textbf{\underline{supername}} The name of the superfile
\item [\colorbox{tagtype}{\color{white} \textbf{\textsf{FIELD}}}] \textbf{\underline{subname}} The name of the sub-file
\end{description}

\rule{\linewidth}{0.5pt}
\subsection*{\textsf{\colorbox{headtoc}{\color{white} RECORD}
FsFileRelationshipRecord}}

\hypertarget{ecldoc:file.fsfilerelationshiprecord}{}
\hspace{0pt} \hyperlink{ecldoc:File}{File} \textbackslash 

{\renewcommand{\arraystretch}{1.5}
\begin{tabularx}{\textwidth}{|>{\raggedright\arraybackslash}l|X|}
\hline
\hspace{0pt}\mytexttt{\color{red} } & \textbf{FsFileRelationshipRecord} \\
\hline
\end{tabularx}
}

\par
A record containing information about the relationship between two files.

\par
\begin{description}
\item [\colorbox{tagtype}{\color{white} \textbf{\textsf{FIELD}}}] \textbf{\underline{primaryfile}} The logical filename of the primary file
\item [\colorbox{tagtype}{\color{white} \textbf{\textsf{FIELD}}}] \textbf{\underline{secondaryfile}} The logical filename of the secondary file.
\item [\colorbox{tagtype}{\color{white} \textbf{\textsf{FIELD}}}] \textbf{\underline{primaryflds}} The name of the primary key field for the primary file. The value ''\_\_fileposition\_\_'' indicates the secondary is an INDEX that must use FETCH to access non-keyed fields.
\item [\colorbox{tagtype}{\color{white} \textbf{\textsf{FIELD}}}] \textbf{\underline{secondaryflds}} The name of the foreign key field relating to the primary file.
\item [\colorbox{tagtype}{\color{white} \textbf{\textsf{FIELD}}}] \textbf{\underline{kind}} The type of relationship between the primary and secondary files. Containing either 'link' or 'view'.
\item [\colorbox{tagtype}{\color{white} \textbf{\textsf{FIELD}}}] \textbf{\underline{cardinality}} The cardinality of the relationship. The format is <primary>:<secondary>. Valid values are ''1'' or ''M''.</secondary></primary>
\item [\colorbox{tagtype}{\color{white} \textbf{\textsf{FIELD}}}] \textbf{\underline{payload}} Indicates whether the primary or secondary are payload INDEXes.
\item [\colorbox{tagtype}{\color{white} \textbf{\textsf{FIELD}}}] \textbf{\underline{description}} The description of the relationship.
\end{description}

\rule{\linewidth}{0.5pt}
\subsection*{\textsf{\colorbox{headtoc}{\color{white} ATTRIBUTE}
RECFMV\_RECSIZE}}

\hypertarget{ecldoc:file.recfmv_recsize}{}
\hspace{0pt} \hyperlink{ecldoc:File}{File} \textbackslash 

{\renewcommand{\arraystretch}{1.5}
\begin{tabularx}{\textwidth}{|>{\raggedright\arraybackslash}l|X|}
\hline
\hspace{0pt}\mytexttt{\color{red} } & \textbf{RECFMV\_RECSIZE} \\
\hline
\end{tabularx}
}

\par
Constant that indicates IBM RECFM V format file. Can be passed to SprayFixed for the record size.


\rule{\linewidth}{0.5pt}
\subsection*{\textsf{\colorbox{headtoc}{\color{white} ATTRIBUTE}
RECFMVB\_RECSIZE}}

\hypertarget{ecldoc:file.recfmvb_recsize}{}
\hspace{0pt} \hyperlink{ecldoc:File}{File} \textbackslash 

{\renewcommand{\arraystretch}{1.5}
\begin{tabularx}{\textwidth}{|>{\raggedright\arraybackslash}l|X|}
\hline
\hspace{0pt}\mytexttt{\color{red} } & \textbf{RECFMVB\_RECSIZE} \\
\hline
\end{tabularx}
}

\par
Constant that indicates IBM RECFM VB format file. Can be passed to SprayFixed for the record size.


\rule{\linewidth}{0.5pt}
\subsection*{\textsf{\colorbox{headtoc}{\color{white} ATTRIBUTE}
PREFIX\_VARIABLE\_RECSIZE}}

\hypertarget{ecldoc:file.prefix_variable_recsize}{}
\hspace{0pt} \hyperlink{ecldoc:File}{File} \textbackslash 

{\renewcommand{\arraystretch}{1.5}
\begin{tabularx}{\textwidth}{|>{\raggedright\arraybackslash}l|X|}
\hline
\hspace{0pt}\mytexttt{\color{red} INTEGER4} & \textbf{PREFIX\_VARIABLE\_RECSIZE} \\
\hline
\end{tabularx}
}

\par
Constant that indicates a variable little endian 4 byte length prefixed file. Can be passed to SprayFixed for the record size.


\rule{\linewidth}{0.5pt}
\subsection*{\textsf{\colorbox{headtoc}{\color{white} ATTRIBUTE}
PREFIX\_VARIABLE\_BIGENDIAN\_RECSIZE}}

\hypertarget{ecldoc:file.prefix_variable_bigendian_recsize}{}
\hspace{0pt} \hyperlink{ecldoc:File}{File} \textbackslash 

{\renewcommand{\arraystretch}{1.5}
\begin{tabularx}{\textwidth}{|>{\raggedright\arraybackslash}l|X|}
\hline
\hspace{0pt}\mytexttt{\color{red} INTEGER4} & \textbf{PREFIX\_VARIABLE\_BIGENDIAN\_RECSIZE} \\
\hline
\end{tabularx}
}

\par
Constant that indicates a variable big endian 4 byte length prefixed file. Can be passed to SprayFixed for the record size.


\rule{\linewidth}{0.5pt}
\subsection*{\textsf{\colorbox{headtoc}{\color{white} FUNCTION}
FileExists}}

\hypertarget{ecldoc:file.fileexists}{}
\hspace{0pt} \hyperlink{ecldoc:File}{File} \textbackslash 

{\renewcommand{\arraystretch}{1.5}
\begin{tabularx}{\textwidth}{|>{\raggedright\arraybackslash}l|X|}
\hline
\hspace{0pt}\mytexttt{\color{red} boolean} & \textbf{FileExists} \\
\hline
\multicolumn{2}{|>{\raggedright\arraybackslash}X|}{\hspace{0pt}\mytexttt{\color{param} (varstring lfn, boolean physical=FALSE)}} \\
\hline
\end{tabularx}
}

\par
Returns whether the file exists.

\par
\begin{description}
\item [\colorbox{tagtype}{\color{white} \textbf{\textsf{PARAMETER}}}] \textbf{\underline{lfn}} The logical name of the file.
\item [\colorbox{tagtype}{\color{white} \textbf{\textsf{PARAMETER}}}] \textbf{\underline{physical}} Whether to also check for the physical existence on disk. Defaults to FALSE.
\item [\colorbox{tagtype}{\color{white} \textbf{\textsf{RETURN}}}] \textbf{\underline{}} Whether the file exists.
\end{description}

\rule{\linewidth}{0.5pt}
\subsection*{\textsf{\colorbox{headtoc}{\color{white} FUNCTION}
DeleteLogicalFile}}

\hypertarget{ecldoc:file.deletelogicalfile}{}
\hspace{0pt} \hyperlink{ecldoc:File}{File} \textbackslash 

{\renewcommand{\arraystretch}{1.5}
\begin{tabularx}{\textwidth}{|>{\raggedright\arraybackslash}l|X|}
\hline
\hspace{0pt}\mytexttt{\color{red} } & \textbf{DeleteLogicalFile} \\
\hline
\multicolumn{2}{|>{\raggedright\arraybackslash}X|}{\hspace{0pt}\mytexttt{\color{param} (varstring lfn, boolean allowMissing=FALSE)}} \\
\hline
\end{tabularx}
}

\par
Removes the logical file from the system, and deletes from the disk.

\par
\begin{description}
\item [\colorbox{tagtype}{\color{white} \textbf{\textsf{PARAMETER}}}] \textbf{\underline{lfn}} The logical name of the file.
\item [\colorbox{tagtype}{\color{white} \textbf{\textsf{PARAMETER}}}] \textbf{\underline{allowMissing}} Whether to suppress an error if the filename does not exist. Defaults to FALSE.
\end{description}

\rule{\linewidth}{0.5pt}
\subsection*{\textsf{\colorbox{headtoc}{\color{white} FUNCTION}
SetReadOnly}}

\hypertarget{ecldoc:file.setreadonly}{}
\hspace{0pt} \hyperlink{ecldoc:File}{File} \textbackslash 

{\renewcommand{\arraystretch}{1.5}
\begin{tabularx}{\textwidth}{|>{\raggedright\arraybackslash}l|X|}
\hline
\hspace{0pt}\mytexttt{\color{red} } & \textbf{SetReadOnly} \\
\hline
\multicolumn{2}{|>{\raggedright\arraybackslash}X|}{\hspace{0pt}\mytexttt{\color{param} (varstring lfn, boolean ro=TRUE)}} \\
\hline
\end{tabularx}
}

\par
Changes whether access to a file is read only or not.

\par
\begin{description}
\item [\colorbox{tagtype}{\color{white} \textbf{\textsf{PARAMETER}}}] \textbf{\underline{lfn}} The logical name of the file.
\item [\colorbox{tagtype}{\color{white} \textbf{\textsf{PARAMETER}}}] \textbf{\underline{ro}} Whether updates to the file are disallowed. Defaults to TRUE.
\end{description}

\rule{\linewidth}{0.5pt}
\subsection*{\textsf{\colorbox{headtoc}{\color{white} FUNCTION}
RenameLogicalFile}}

\hypertarget{ecldoc:file.renamelogicalfile}{}
\hspace{0pt} \hyperlink{ecldoc:File}{File} \textbackslash 

{\renewcommand{\arraystretch}{1.5}
\begin{tabularx}{\textwidth}{|>{\raggedright\arraybackslash}l|X|}
\hline
\hspace{0pt}\mytexttt{\color{red} } & \textbf{RenameLogicalFile} \\
\hline
\multicolumn{2}{|>{\raggedright\arraybackslash}X|}{\hspace{0pt}\mytexttt{\color{param} (varstring oldname, varstring newname)}} \\
\hline
\end{tabularx}
}

\par
Changes the name of a logical file.

\par
\begin{description}
\item [\colorbox{tagtype}{\color{white} \textbf{\textsf{PARAMETER}}}] \textbf{\underline{oldname}} The current name of the file to be renamed.
\item [\colorbox{tagtype}{\color{white} \textbf{\textsf{PARAMETER}}}] \textbf{\underline{newname}} The new logical name of the file.
\end{description}

\rule{\linewidth}{0.5pt}
\subsection*{\textsf{\colorbox{headtoc}{\color{white} FUNCTION}
ForeignLogicalFileName}}

\hypertarget{ecldoc:file.foreignlogicalfilename}{}
\hspace{0pt} \hyperlink{ecldoc:File}{File} \textbackslash 

{\renewcommand{\arraystretch}{1.5}
\begin{tabularx}{\textwidth}{|>{\raggedright\arraybackslash}l|X|}
\hline
\hspace{0pt}\mytexttt{\color{red} varstring} & \textbf{ForeignLogicalFileName} \\
\hline
\multicolumn{2}{|>{\raggedright\arraybackslash}X|}{\hspace{0pt}\mytexttt{\color{param} (varstring name, varstring foreigndali='', boolean abspath=FALSE)}} \\
\hline
\end{tabularx}
}

\par
Returns a logical filename that can be used to refer to a logical file in a local or remote dali.

\par
\begin{description}
\item [\colorbox{tagtype}{\color{white} \textbf{\textsf{PARAMETER}}}] \textbf{\underline{name}} The logical name of the file.
\item [\colorbox{tagtype}{\color{white} \textbf{\textsf{PARAMETER}}}] \textbf{\underline{foreigndali}} The IP address of the foreign dali used to resolve the file. If blank then the file is resolved locally. Defaults to blank.
\item [\colorbox{tagtype}{\color{white} \textbf{\textsf{PARAMETER}}}] \textbf{\underline{abspath}} Should a tilde (\~{}) be prepended to the resulting logical file name. Defaults to FALSE.
\end{description}

\rule{\linewidth}{0.5pt}
\subsection*{\textsf{\colorbox{headtoc}{\color{white} FUNCTION}
ExternalLogicalFileName}}

\hypertarget{ecldoc:file.externallogicalfilename}{}
\hspace{0pt} \hyperlink{ecldoc:File}{File} \textbackslash 

{\renewcommand{\arraystretch}{1.5}
\begin{tabularx}{\textwidth}{|>{\raggedright\arraybackslash}l|X|}
\hline
\hspace{0pt}\mytexttt{\color{red} varstring} & \textbf{ExternalLogicalFileName} \\
\hline
\multicolumn{2}{|>{\raggedright\arraybackslash}X|}{\hspace{0pt}\mytexttt{\color{param} (varstring location, varstring path, boolean abspath=TRUE)}} \\
\hline
\end{tabularx}
}

\par
Returns an encoded logical filename that can be used to refer to a external file. Examples include directly reading from a landing zone. Upper case characters and other details are escaped.

\par
\begin{description}
\item [\colorbox{tagtype}{\color{white} \textbf{\textsf{PARAMETER}}}] \textbf{\underline{location}} The IP address of the remote machine. '.' can be used for the local machine.
\item [\colorbox{tagtype}{\color{white} \textbf{\textsf{PARAMETER}}}] \textbf{\underline{path}} The path/name of the file on the remote machine.
\item [\colorbox{tagtype}{\color{white} \textbf{\textsf{PARAMETER}}}] \textbf{\underline{abspath}} Should a tilde (\~{}) be prepended to the resulting logical file name. Defaults to TRUE.
\item [\colorbox{tagtype}{\color{white} \textbf{\textsf{RETURN}}}] \textbf{\underline{}} The encoded logical filename.
\end{description}

\rule{\linewidth}{0.5pt}
\subsection*{\textsf{\colorbox{headtoc}{\color{white} FUNCTION}
GetFileDescription}}

\hypertarget{ecldoc:file.getfiledescription}{}
\hspace{0pt} \hyperlink{ecldoc:File}{File} \textbackslash 

{\renewcommand{\arraystretch}{1.5}
\begin{tabularx}{\textwidth}{|>{\raggedright\arraybackslash}l|X|}
\hline
\hspace{0pt}\mytexttt{\color{red} varstring} & \textbf{GetFileDescription} \\
\hline
\multicolumn{2}{|>{\raggedright\arraybackslash}X|}{\hspace{0pt}\mytexttt{\color{param} (varstring lfn)}} \\
\hline
\end{tabularx}
}

\par
Returns a string containing the description information associated with the specified filename. This description is set either through ECL watch or by using the FileServices.SetFileDescription function.

\par
\begin{description}
\item [\colorbox{tagtype}{\color{white} \textbf{\textsf{PARAMETER}}}] \textbf{\underline{lfn}} The logical name of the file.
\end{description}

\rule{\linewidth}{0.5pt}
\subsection*{\textsf{\colorbox{headtoc}{\color{white} FUNCTION}
SetFileDescription}}

\hypertarget{ecldoc:file.setfiledescription}{}
\hspace{0pt} \hyperlink{ecldoc:File}{File} \textbackslash 

{\renewcommand{\arraystretch}{1.5}
\begin{tabularx}{\textwidth}{|>{\raggedright\arraybackslash}l|X|}
\hline
\hspace{0pt}\mytexttt{\color{red} } & \textbf{SetFileDescription} \\
\hline
\multicolumn{2}{|>{\raggedright\arraybackslash}X|}{\hspace{0pt}\mytexttt{\color{param} (varstring lfn, varstring val)}} \\
\hline
\end{tabularx}
}

\par
Sets the description associated with the specified filename.

\par
\begin{description}
\item [\colorbox{tagtype}{\color{white} \textbf{\textsf{PARAMETER}}}] \textbf{\underline{lfn}} The logical name of the file.
\item [\colorbox{tagtype}{\color{white} \textbf{\textsf{PARAMETER}}}] \textbf{\underline{val}} The description to be associated with the file.
\end{description}

\rule{\linewidth}{0.5pt}
\subsection*{\textsf{\colorbox{headtoc}{\color{white} FUNCTION}
RemoteDirectory}}

\hypertarget{ecldoc:file.remotedirectory}{}
\hspace{0pt} \hyperlink{ecldoc:File}{File} \textbackslash 

{\renewcommand{\arraystretch}{1.5}
\begin{tabularx}{\textwidth}{|>{\raggedright\arraybackslash}l|X|}
\hline
\hspace{0pt}\mytexttt{\color{red} dataset(FsFilenameRecord)} & \textbf{RemoteDirectory} \\
\hline
\multicolumn{2}{|>{\raggedright\arraybackslash}X|}{\hspace{0pt}\mytexttt{\color{param} (varstring machineIP, varstring dir, varstring mask='*', boolean recurse=FALSE)}} \\
\hline
\end{tabularx}
}

\par
Returns a dataset containing a list of files from the specified machineIP and directory.

\par
\begin{description}
\item [\colorbox{tagtype}{\color{white} \textbf{\textsf{PARAMETER}}}] \textbf{\underline{machineIP}} The IP address of the remote machine.
\item [\colorbox{tagtype}{\color{white} \textbf{\textsf{PARAMETER}}}] \textbf{\underline{directory}} The path to the directory to read. This must be in the appropriate format for the operating system running on the remote machine.
\item [\colorbox{tagtype}{\color{white} \textbf{\textsf{PARAMETER}}}] \textbf{\underline{mask}} The filemask specifying which files to include in the result. Defaults to '*' (all files).
\item [\colorbox{tagtype}{\color{white} \textbf{\textsf{PARAMETER}}}] \textbf{\underline{recurse}} Whether to include files from subdirectories under the directory. Defaults to FALSE.
\end{description}

\rule{\linewidth}{0.5pt}
\subsection*{\textsf{\colorbox{headtoc}{\color{white} FUNCTION}
LogicalFileList}}

\hypertarget{ecldoc:file.logicalfilelist}{}
\hspace{0pt} \hyperlink{ecldoc:File}{File} \textbackslash 

{\renewcommand{\arraystretch}{1.5}
\begin{tabularx}{\textwidth}{|>{\raggedright\arraybackslash}l|X|}
\hline
\hspace{0pt}\mytexttt{\color{red} dataset(FsLogicalFileInfoRecord)} & \textbf{LogicalFileList} \\
\hline
\multicolumn{2}{|>{\raggedright\arraybackslash}X|}{\hspace{0pt}\mytexttt{\color{param} (varstring namepattern='*', boolean includenormal=TRUE, boolean includesuper=FALSE, boolean unknownszero=FALSE, varstring foreigndali='')}} \\
\hline
\end{tabularx}
}

\par
Returns a dataset of information about the logical files known to the system.

\par
\begin{description}
\item [\colorbox{tagtype}{\color{white} \textbf{\textsf{PARAMETER}}}] \textbf{\underline{namepattern}} The mask of the files to list. Defaults to '*' (all files).
\item [\colorbox{tagtype}{\color{white} \textbf{\textsf{PARAMETER}}}] \textbf{\underline{includenormal}} Whether to include 'normal' files. Defaults to TRUE.
\item [\colorbox{tagtype}{\color{white} \textbf{\textsf{PARAMETER}}}] \textbf{\underline{includesuper}} Whether to include SuperFiles. Defaults to FALSE.
\item [\colorbox{tagtype}{\color{white} \textbf{\textsf{PARAMETER}}}] \textbf{\underline{unknownszero}} Whether to set file sizes that are unknown to zero(0) instead of minus-one (-1). Defaults to FALSE.
\item [\colorbox{tagtype}{\color{white} \textbf{\textsf{PARAMETER}}}] \textbf{\underline{foreigndali}} The IP address of the foreign dali used to resolve the file. If blank then the file is resolved locally. Defaults to blank.
\end{description}

\rule{\linewidth}{0.5pt}
\subsection*{\textsf{\colorbox{headtoc}{\color{white} FUNCTION}
CompareFiles}}

\hypertarget{ecldoc:file.comparefiles}{}
\hspace{0pt} \hyperlink{ecldoc:File}{File} \textbackslash 

{\renewcommand{\arraystretch}{1.5}
\begin{tabularx}{\textwidth}{|>{\raggedright\arraybackslash}l|X|}
\hline
\hspace{0pt}\mytexttt{\color{red} INTEGER4} & \textbf{CompareFiles} \\
\hline
\multicolumn{2}{|>{\raggedright\arraybackslash}X|}{\hspace{0pt}\mytexttt{\color{param} (varstring lfn1, varstring lfn2, boolean logical\_only=TRUE, boolean use\_crcs=FALSE)}} \\
\hline
\end{tabularx}
}

\par
Compares two files, and returns a result indicating how well they match.

\par
\begin{description}
\item [\colorbox{tagtype}{\color{white} \textbf{\textsf{PARAMETER}}}] \textbf{\underline{file1}} The logical name of the first file.
\item [\colorbox{tagtype}{\color{white} \textbf{\textsf{PARAMETER}}}] \textbf{\underline{file2}} The logical name of the second file.
\item [\colorbox{tagtype}{\color{white} \textbf{\textsf{PARAMETER}}}] \textbf{\underline{logical\_only}} Whether to only compare logical information in the system datastore (Dali), and ignore physical information on disk. [Default TRUE]
\item [\colorbox{tagtype}{\color{white} \textbf{\textsf{PARAMETER}}}] \textbf{\underline{use\_crcs}} Whether to compare physical CRCs of all the parts on disk. This may be slow on large files. Defaults to FALSE.
\item [\colorbox{tagtype}{\color{white} \textbf{\textsf{RETURN}}}] \textbf{\underline{}} 0 if file1 and file2 match exactly 1 if file1 and file2 contents match, but file1 is newer than file2 -1 if file1 and file2 contents match, but file2 is newer than file1 2 if file1 and file2 contents do not match and file1 is newer than file2 -2 if file1 and file2 contents do not match and file2 is newer than file1
\end{description}

\rule{\linewidth}{0.5pt}
\subsection*{\textsf{\colorbox{headtoc}{\color{white} FUNCTION}
VerifyFile}}

\hypertarget{ecldoc:file.verifyfile}{}
\hspace{0pt} \hyperlink{ecldoc:File}{File} \textbackslash 

{\renewcommand{\arraystretch}{1.5}
\begin{tabularx}{\textwidth}{|>{\raggedright\arraybackslash}l|X|}
\hline
\hspace{0pt}\mytexttt{\color{red} varstring} & \textbf{VerifyFile} \\
\hline
\multicolumn{2}{|>{\raggedright\arraybackslash}X|}{\hspace{0pt}\mytexttt{\color{param} (varstring lfn, boolean usecrcs)}} \\
\hline
\end{tabularx}
}

\par
Checks the system datastore (Dali) information for the file against the physical parts on disk.

\par
\begin{description}
\item [\colorbox{tagtype}{\color{white} \textbf{\textsf{PARAMETER}}}] \textbf{\underline{lfn}} The name of the file to check.
\item [\colorbox{tagtype}{\color{white} \textbf{\textsf{PARAMETER}}}] \textbf{\underline{use\_crcs}} Whether to compare physical CRCs of all the parts on disk. This may be slow on large files.
\item [\colorbox{tagtype}{\color{white} \textbf{\textsf{RETURN}}}] \textbf{\underline{}} 'OK' - The file parts match the datastore information 'Could not find file: <filename>' - The logical filename was not found 'Could not find part file: <partname>' - The partname was not found 'Modified time differs for: <partname>' - The partname has a different timestamp 'File size differs for: <partname>' - The partname has a file size 'File CRC differs for: <partname>' - The partname has a different CRC</partname></partname></partname></partname></filename>
\end{description}

\rule{\linewidth}{0.5pt}
\subsection*{\textsf{\colorbox{headtoc}{\color{white} FUNCTION}
AddFileRelationship}}

\hypertarget{ecldoc:file.addfilerelationship}{}
\hspace{0pt} \hyperlink{ecldoc:File}{File} \textbackslash 

{\renewcommand{\arraystretch}{1.5}
\begin{tabularx}{\textwidth}{|>{\raggedright\arraybackslash}l|X|}
\hline
\hspace{0pt}\mytexttt{\color{red} } & \textbf{AddFileRelationship} \\
\hline
\multicolumn{2}{|>{\raggedright\arraybackslash}X|}{\hspace{0pt}\mytexttt{\color{param} (varstring primary, varstring secondary, varstring primaryflds, varstring secondaryflds, varstring kind='link', varstring cardinality, boolean payload, varstring description='')}} \\
\hline
\end{tabularx}
}

\par
Defines the relationship between two files. These may be DATASETs or INDEXes. Each record in the primary file should be uniquely defined by the primaryfields (ideally), preferably efficiently. This information is used by the roxie browser to link files together.

\par
\begin{description}
\item [\colorbox{tagtype}{\color{white} \textbf{\textsf{PARAMETER}}}] \textbf{\underline{primary}} The logical filename of the primary file.
\item [\colorbox{tagtype}{\color{white} \textbf{\textsf{PARAMETER}}}] \textbf{\underline{secondary}} The logical filename of the secondary file.
\item [\colorbox{tagtype}{\color{white} \textbf{\textsf{PARAMETER}}}] \textbf{\underline{primaryfields}} The name of the primary key field for the primary file. The value ''\_\_fileposition\_\_'' indicates the secondary is an INDEX that must use FETCH to access non-keyed fields.
\item [\colorbox{tagtype}{\color{white} \textbf{\textsf{PARAMETER}}}] \textbf{\underline{secondaryfields}} The name of the foreign key field relating to the primary file.
\item [\colorbox{tagtype}{\color{white} \textbf{\textsf{PARAMETER}}}] \textbf{\underline{relationship}} The type of relationship between the primary and secondary files. Containing either 'link' or 'view'. Default is ''link''.
\item [\colorbox{tagtype}{\color{white} \textbf{\textsf{PARAMETER}}}] \textbf{\underline{cardinality}} The cardinality of the relationship. The format is <primary>:<secondary>. Valid values are ''1'' or ''M''.</secondary></primary>
\item [\colorbox{tagtype}{\color{white} \textbf{\textsf{PARAMETER}}}] \textbf{\underline{payload}} Indicates whether the primary or secondary are payload INDEXes.
\item [\colorbox{tagtype}{\color{white} \textbf{\textsf{PARAMETER}}}] \textbf{\underline{description}} The description of the relationship.
\end{description}

\rule{\linewidth}{0.5pt}
\subsection*{\textsf{\colorbox{headtoc}{\color{white} FUNCTION}
FileRelationshipList}}

\hypertarget{ecldoc:file.filerelationshiplist}{}
\hspace{0pt} \hyperlink{ecldoc:File}{File} \textbackslash 

{\renewcommand{\arraystretch}{1.5}
\begin{tabularx}{\textwidth}{|>{\raggedright\arraybackslash}l|X|}
\hline
\hspace{0pt}\mytexttt{\color{red} dataset(FsFileRelationshipRecord)} & \textbf{FileRelationshipList} \\
\hline
\multicolumn{2}{|>{\raggedright\arraybackslash}X|}{\hspace{0pt}\mytexttt{\color{param} (varstring primary, varstring secondary, varstring primflds='', varstring secondaryflds='', varstring kind='link')}} \\
\hline
\end{tabularx}
}

\par
Returns a dataset of relationships. The return records are structured in the FsFileRelationshipRecord format.

\par
\begin{description}
\item [\colorbox{tagtype}{\color{white} \textbf{\textsf{PARAMETER}}}] \textbf{\underline{primary}} The logical filename of the primary file.
\item [\colorbox{tagtype}{\color{white} \textbf{\textsf{PARAMETER}}}] \textbf{\underline{secondary}} The logical filename of the secondary file.
\item [\colorbox{tagtype}{\color{white} \textbf{\textsf{PARAMETER}}}] \textbf{\underline{primaryfields}} The name of the primary key field for the primary file.
\item [\colorbox{tagtype}{\color{white} \textbf{\textsf{PARAMETER}}}] \textbf{\underline{secondaryfields}} The name of the foreign key field relating to the primary file.
\item [\colorbox{tagtype}{\color{white} \textbf{\textsf{PARAMETER}}}] \textbf{\underline{relationship}} The type of relationship between the primary and secondary files. Containing either 'link' or 'view'. Default is ''link''.
\end{description}

\rule{\linewidth}{0.5pt}
\subsection*{\textsf{\colorbox{headtoc}{\color{white} FUNCTION}
RemoveFileRelationship}}

\hypertarget{ecldoc:file.removefilerelationship}{}
\hspace{0pt} \hyperlink{ecldoc:File}{File} \textbackslash 

{\renewcommand{\arraystretch}{1.5}
\begin{tabularx}{\textwidth}{|>{\raggedright\arraybackslash}l|X|}
\hline
\hspace{0pt}\mytexttt{\color{red} } & \textbf{RemoveFileRelationship} \\
\hline
\multicolumn{2}{|>{\raggedright\arraybackslash}X|}{\hspace{0pt}\mytexttt{\color{param} (varstring primary, varstring secondary, varstring primaryflds='', varstring secondaryflds='', varstring kind='link')}} \\
\hline
\end{tabularx}
}

\par
Removes a relationship between two files.

\par
\begin{description}
\item [\colorbox{tagtype}{\color{white} \textbf{\textsf{PARAMETER}}}] \textbf{\underline{primary}} The logical filename of the primary file.
\item [\colorbox{tagtype}{\color{white} \textbf{\textsf{PARAMETER}}}] \textbf{\underline{secondary}} The logical filename of the secondary file.
\item [\colorbox{tagtype}{\color{white} \textbf{\textsf{PARAMETER}}}] \textbf{\underline{primaryfields}} The name of the primary key field for the primary file.
\item [\colorbox{tagtype}{\color{white} \textbf{\textsf{PARAMETER}}}] \textbf{\underline{secondaryfields}} The name of the foreign key field relating to the primary file.
\item [\colorbox{tagtype}{\color{white} \textbf{\textsf{PARAMETER}}}] \textbf{\underline{relationship}} The type of relationship between the primary and secondary files. Containing either 'link' or 'view'. Default is ''link''.
\end{description}

\rule{\linewidth}{0.5pt}
\subsection*{\textsf{\colorbox{headtoc}{\color{white} FUNCTION}
GetColumnMapping}}

\hypertarget{ecldoc:file.getcolumnmapping}{}
\hspace{0pt} \hyperlink{ecldoc:File}{File} \textbackslash 

{\renewcommand{\arraystretch}{1.5}
\begin{tabularx}{\textwidth}{|>{\raggedright\arraybackslash}l|X|}
\hline
\hspace{0pt}\mytexttt{\color{red} varstring} & \textbf{GetColumnMapping} \\
\hline
\multicolumn{2}{|>{\raggedright\arraybackslash}X|}{\hspace{0pt}\mytexttt{\color{param} (varstring lfn)}} \\
\hline
\end{tabularx}
}

\par
Returns the field mappings for the file, in the same format specified for the SetColumnMapping function.

\par
\begin{description}
\item [\colorbox{tagtype}{\color{white} \textbf{\textsf{PARAMETER}}}] \textbf{\underline{lfn}} The logical filename of the primary file.
\end{description}

\rule{\linewidth}{0.5pt}
\subsection*{\textsf{\colorbox{headtoc}{\color{white} FUNCTION}
SetColumnMapping}}

\hypertarget{ecldoc:file.setcolumnmapping}{}
\hspace{0pt} \hyperlink{ecldoc:File}{File} \textbackslash 

{\renewcommand{\arraystretch}{1.5}
\begin{tabularx}{\textwidth}{|>{\raggedright\arraybackslash}l|X|}
\hline
\hspace{0pt}\mytexttt{\color{red} } & \textbf{SetColumnMapping} \\
\hline
\multicolumn{2}{|>{\raggedright\arraybackslash}X|}{\hspace{0pt}\mytexttt{\color{param} (varstring lfn, varstring mapping)}} \\
\hline
\end{tabularx}
}

\par
Defines how the data in the fields of the file mist be transformed between the actual data storage format and the input format used to query that data. This is used by the user interface of the roxie browser.

\par
\begin{description}
\item [\colorbox{tagtype}{\color{white} \textbf{\textsf{PARAMETER}}}] \textbf{\underline{lfn}} The logical filename of the primary file.
\item [\colorbox{tagtype}{\color{white} \textbf{\textsf{PARAMETER}}}] \textbf{\underline{mapping}} A string containing a comma separated list of field mappings.
\end{description}

\rule{\linewidth}{0.5pt}
\subsection*{\textsf{\colorbox{headtoc}{\color{white} FUNCTION}
EncodeRfsQuery}}

\hypertarget{ecldoc:file.encoderfsquery}{}
\hspace{0pt} \hyperlink{ecldoc:File}{File} \textbackslash 

{\renewcommand{\arraystretch}{1.5}
\begin{tabularx}{\textwidth}{|>{\raggedright\arraybackslash}l|X|}
\hline
\hspace{0pt}\mytexttt{\color{red} varstring} & \textbf{EncodeRfsQuery} \\
\hline
\multicolumn{2}{|>{\raggedright\arraybackslash}X|}{\hspace{0pt}\mytexttt{\color{param} (varstring server, varstring query)}} \\
\hline
\end{tabularx}
}

\par
Returns a string that can be used in a DATASET declaration to read data from an RFS (Remote File Server) instance (e.g. rfsmysql) on another node.

\par
\begin{description}
\item [\colorbox{tagtype}{\color{white} \textbf{\textsf{PARAMETER}}}] \textbf{\underline{server}} A string containing the ip:port address for the remote file server.
\item [\colorbox{tagtype}{\color{white} \textbf{\textsf{PARAMETER}}}] \textbf{\underline{query}} The text of the query to send to the server
\end{description}

\rule{\linewidth}{0.5pt}
\subsection*{\textsf{\colorbox{headtoc}{\color{white} FUNCTION}
RfsAction}}

\hypertarget{ecldoc:file.rfsaction}{}
\hspace{0pt} \hyperlink{ecldoc:File}{File} \textbackslash 

{\renewcommand{\arraystretch}{1.5}
\begin{tabularx}{\textwidth}{|>{\raggedright\arraybackslash}l|X|}
\hline
\hspace{0pt}\mytexttt{\color{red} } & \textbf{RfsAction} \\
\hline
\multicolumn{2}{|>{\raggedright\arraybackslash}X|}{\hspace{0pt}\mytexttt{\color{param} (varstring server, varstring query)}} \\
\hline
\end{tabularx}
}

\par
Sends the query to the rfs server.

\par
\begin{description}
\item [\colorbox{tagtype}{\color{white} \textbf{\textsf{PARAMETER}}}] \textbf{\underline{server}} A string containing the ip:port address for the remote file server.
\item [\colorbox{tagtype}{\color{white} \textbf{\textsf{PARAMETER}}}] \textbf{\underline{query}} The text of the query to send to the server
\end{description}

\rule{\linewidth}{0.5pt}
\subsection*{\textsf{\colorbox{headtoc}{\color{white} FUNCTION}
MoveExternalFile}}

\hypertarget{ecldoc:file.moveexternalfile}{}
\hspace{0pt} \hyperlink{ecldoc:File}{File} \textbackslash 

{\renewcommand{\arraystretch}{1.5}
\begin{tabularx}{\textwidth}{|>{\raggedright\arraybackslash}l|X|}
\hline
\hspace{0pt}\mytexttt{\color{red} } & \textbf{MoveExternalFile} \\
\hline
\multicolumn{2}{|>{\raggedright\arraybackslash}X|}{\hspace{0pt}\mytexttt{\color{param} (varstring location, varstring frompath, varstring topath)}} \\
\hline
\end{tabularx}
}

\par
Moves the single physical file between two locations on the same remote machine. The dafileserv utility program must be running on the location machine.

\par
\begin{description}
\item [\colorbox{tagtype}{\color{white} \textbf{\textsf{PARAMETER}}}] \textbf{\underline{location}} The IP address of the remote machine.
\item [\colorbox{tagtype}{\color{white} \textbf{\textsf{PARAMETER}}}] \textbf{\underline{frompath}} The path/name of the file to move.
\item [\colorbox{tagtype}{\color{white} \textbf{\textsf{PARAMETER}}}] \textbf{\underline{topath}} The path/name of the target file.
\end{description}

\rule{\linewidth}{0.5pt}
\subsection*{\textsf{\colorbox{headtoc}{\color{white} FUNCTION}
DeleteExternalFile}}

\hypertarget{ecldoc:file.deleteexternalfile}{}
\hspace{0pt} \hyperlink{ecldoc:File}{File} \textbackslash 

{\renewcommand{\arraystretch}{1.5}
\begin{tabularx}{\textwidth}{|>{\raggedright\arraybackslash}l|X|}
\hline
\hspace{0pt}\mytexttt{\color{red} } & \textbf{DeleteExternalFile} \\
\hline
\multicolumn{2}{|>{\raggedright\arraybackslash}X|}{\hspace{0pt}\mytexttt{\color{param} (varstring location, varstring path)}} \\
\hline
\end{tabularx}
}

\par
Removes a single physical file from a remote machine. The dafileserv utility program must be running on the location machine.

\par
\begin{description}
\item [\colorbox{tagtype}{\color{white} \textbf{\textsf{PARAMETER}}}] \textbf{\underline{location}} The IP address of the remote machine.
\item [\colorbox{tagtype}{\color{white} \textbf{\textsf{PARAMETER}}}] \textbf{\underline{path}} The path/name of the file to remove.
\end{description}

\rule{\linewidth}{0.5pt}
\subsection*{\textsf{\colorbox{headtoc}{\color{white} FUNCTION}
CreateExternalDirectory}}

\hypertarget{ecldoc:file.createexternaldirectory}{}
\hspace{0pt} \hyperlink{ecldoc:File}{File} \textbackslash 

{\renewcommand{\arraystretch}{1.5}
\begin{tabularx}{\textwidth}{|>{\raggedright\arraybackslash}l|X|}
\hline
\hspace{0pt}\mytexttt{\color{red} } & \textbf{CreateExternalDirectory} \\
\hline
\multicolumn{2}{|>{\raggedright\arraybackslash}X|}{\hspace{0pt}\mytexttt{\color{param} (varstring location, varstring path)}} \\
\hline
\end{tabularx}
}

\par
Creates the path on the location (if it does not already exist). The dafileserv utility program must be running on the location machine.

\par
\begin{description}
\item [\colorbox{tagtype}{\color{white} \textbf{\textsf{PARAMETER}}}] \textbf{\underline{location}} The IP address of the remote machine.
\item [\colorbox{tagtype}{\color{white} \textbf{\textsf{PARAMETER}}}] \textbf{\underline{path}} The path/name of the file to remove.
\end{description}

\rule{\linewidth}{0.5pt}
\subsection*{\textsf{\colorbox{headtoc}{\color{white} FUNCTION}
GetLogicalFileAttribute}}

\hypertarget{ecldoc:file.getlogicalfileattribute}{}
\hspace{0pt} \hyperlink{ecldoc:File}{File} \textbackslash 

{\renewcommand{\arraystretch}{1.5}
\begin{tabularx}{\textwidth}{|>{\raggedright\arraybackslash}l|X|}
\hline
\hspace{0pt}\mytexttt{\color{red} varstring} & \textbf{GetLogicalFileAttribute} \\
\hline
\multicolumn{2}{|>{\raggedright\arraybackslash}X|}{\hspace{0pt}\mytexttt{\color{param} (varstring lfn, varstring attrname)}} \\
\hline
\end{tabularx}
}

\par
Returns the value of the given attribute for the specified logicalfilename.

\par
\begin{description}
\item [\colorbox{tagtype}{\color{white} \textbf{\textsf{PARAMETER}}}] \textbf{\underline{lfn}} The name of the logical file.
\item [\colorbox{tagtype}{\color{white} \textbf{\textsf{PARAMETER}}}] \textbf{\underline{attrname}} The name of the file attribute to return.
\end{description}

\rule{\linewidth}{0.5pt}
\subsection*{\textsf{\colorbox{headtoc}{\color{white} FUNCTION}
ProtectLogicalFile}}

\hypertarget{ecldoc:file.protectlogicalfile}{}
\hspace{0pt} \hyperlink{ecldoc:File}{File} \textbackslash 

{\renewcommand{\arraystretch}{1.5}
\begin{tabularx}{\textwidth}{|>{\raggedright\arraybackslash}l|X|}
\hline
\hspace{0pt}\mytexttt{\color{red} } & \textbf{ProtectLogicalFile} \\
\hline
\multicolumn{2}{|>{\raggedright\arraybackslash}X|}{\hspace{0pt}\mytexttt{\color{param} (varstring lfn, boolean value=TRUE)}} \\
\hline
\end{tabularx}
}

\par
Toggles protection on and off for the specified logicalfilename.

\par
\begin{description}
\item [\colorbox{tagtype}{\color{white} \textbf{\textsf{PARAMETER}}}] \textbf{\underline{lfn}} The name of the logical file.
\item [\colorbox{tagtype}{\color{white} \textbf{\textsf{PARAMETER}}}] \textbf{\underline{value}} TRUE to enable protection, FALSE to disable.
\end{description}

\rule{\linewidth}{0.5pt}
\subsection*{\textsf{\colorbox{headtoc}{\color{white} FUNCTION}
DfuPlusExec}}

\hypertarget{ecldoc:file.dfuplusexec}{}
\hspace{0pt} \hyperlink{ecldoc:File}{File} \textbackslash 

{\renewcommand{\arraystretch}{1.5}
\begin{tabularx}{\textwidth}{|>{\raggedright\arraybackslash}l|X|}
\hline
\hspace{0pt}\mytexttt{\color{red} } & \textbf{DfuPlusExec} \\
\hline
\multicolumn{2}{|>{\raggedright\arraybackslash}X|}{\hspace{0pt}\mytexttt{\color{param} (varstring cmdline)}} \\
\hline
\end{tabularx}
}

\par
The DfuPlusExec action executes the specified command line just as the DfuPLus.exe program would do. This allows you to have all the functionality of the DfuPLus.exe program available within your ECL code. param cmdline The DFUPlus.exe command line to execute. The valid arguments are documented in the Client Tools manual, in the section describing the DfuPlus.exe program.


\rule{\linewidth}{0.5pt}
\subsection*{\textsf{\colorbox{headtoc}{\color{white} FUNCTION}
fSprayFixed}}

\hypertarget{ecldoc:file.fsprayfixed}{}
\hspace{0pt} \hyperlink{ecldoc:File}{File} \textbackslash 

{\renewcommand{\arraystretch}{1.5}
\begin{tabularx}{\textwidth}{|>{\raggedright\arraybackslash}l|X|}
\hline
\hspace{0pt}\mytexttt{\color{red} varstring} & \textbf{fSprayFixed} \\
\hline
\multicolumn{2}{|>{\raggedright\arraybackslash}X|}{\hspace{0pt}\mytexttt{\color{param} (varstring sourceIP, varstring sourcePath, integer4 recordSize, varstring destinationGroup, varstring destinationLogicalName, integer4 timeOut=-1, varstring espServerIpPort=GETENV('ws\_fs\_server'), integer4 maxConnections=-1, boolean allowOverwrite=FALSE, boolean replicate=FALSE, boolean compress=FALSE, boolean failIfNoSourceFile=FALSE, integer4 expireDays=-1)}} \\
\hline
\end{tabularx}
}

\par
Sprays a file of fixed length records from a single machine and distributes it across the nodes of the destination group.

\par
\begin{description}
\item [\colorbox{tagtype}{\color{white} \textbf{\textsf{PARAMETER}}}] \textbf{\underline{sourceIP}} The IP address of the file.
\item [\colorbox{tagtype}{\color{white} \textbf{\textsf{PARAMETER}}}] \textbf{\underline{sourcePath}} The path and name of the file.
\item [\colorbox{tagtype}{\color{white} \textbf{\textsf{PARAMETER}}}] \textbf{\underline{recordsize}} The size (in bytes) of the records in the file.
\item [\colorbox{tagtype}{\color{white} \textbf{\textsf{PARAMETER}}}] \textbf{\underline{destinationGroup}} The name of the group to distribute the file across.
\item [\colorbox{tagtype}{\color{white} \textbf{\textsf{PARAMETER}}}] \textbf{\underline{destinationLogicalName}} The logical name of the file to create.
\item [\colorbox{tagtype}{\color{white} \textbf{\textsf{PARAMETER}}}] \textbf{\underline{timeOut}} The time in ms to wait for the operation to complete. A value of 0 causes the call to return immediately. Defaults to no timeout (-1).
\item [\colorbox{tagtype}{\color{white} \textbf{\textsf{PARAMETER}}}] \textbf{\underline{espServerIpPort}} The url of the ESP file copying service. Defaults to the value of ws\_fs\_server in the environment.
\item [\colorbox{tagtype}{\color{white} \textbf{\textsf{PARAMETER}}}] \textbf{\underline{maxConnections}} The maximum number of target nodes to write to concurrently. Defaults to 1.
\item [\colorbox{tagtype}{\color{white} \textbf{\textsf{PARAMETER}}}] \textbf{\underline{allowOverwrite}} Is it valid to overwrite an existing file of the same name? Defaults to FALSE
\item [\colorbox{tagtype}{\color{white} \textbf{\textsf{PARAMETER}}}] \textbf{\underline{replicate}} Whether to replicate the new file. Defaults to FALSE.
\item [\colorbox{tagtype}{\color{white} \textbf{\textsf{PARAMETER}}}] \textbf{\underline{compress}} Whether to compress the new file. Defaults to FALSE.
\item [\colorbox{tagtype}{\color{white} \textbf{\textsf{PARAMETER}}}] \textbf{\underline{failIfNoSourceFile}} If TRUE it causes a missing source file to trigger a failure. Defaults to FALSE.
\item [\colorbox{tagtype}{\color{white} \textbf{\textsf{PARAMETER}}}] \textbf{\underline{expireDays}} Number of days to auto-remove file. Default is -1, not expire.
\item [\colorbox{tagtype}{\color{white} \textbf{\textsf{RETURN}}}] \textbf{\underline{}} The DFU workunit id for the job.
\end{description}

\rule{\linewidth}{0.5pt}
\subsection*{\textsf{\colorbox{headtoc}{\color{white} FUNCTION}
SprayFixed}}

\hypertarget{ecldoc:file.sprayfixed}{}
\hspace{0pt} \hyperlink{ecldoc:File}{File} \textbackslash 

{\renewcommand{\arraystretch}{1.5}
\begin{tabularx}{\textwidth}{|>{\raggedright\arraybackslash}l|X|}
\hline
\hspace{0pt}\mytexttt{\color{red} } & \textbf{SprayFixed} \\
\hline
\multicolumn{2}{|>{\raggedright\arraybackslash}X|}{\hspace{0pt}\mytexttt{\color{param} (varstring sourceIP, varstring sourcePath, integer4 recordSize, varstring destinationGroup, varstring destinationLogicalName, integer4 timeOut=-1, varstring espServerIpPort=GETENV('ws\_fs\_server'), integer4 maxConnections=-1, boolean allowOverwrite=FALSE, boolean replicate=FALSE, boolean compress=FALSE, boolean failIfNoSourceFile=FALSE, integer4 expireDays=-1)}} \\
\hline
\end{tabularx}
}

\par
Same as fSprayFixed, but does not return the DFU Workunit ID.

\par
\begin{description}
\item [\colorbox{tagtype}{\color{white} \textbf{\textsf{SEE}}}] \textbf{\underline{}} fSprayFixed
\end{description}

\rule{\linewidth}{0.5pt}
\subsection*{\textsf{\colorbox{headtoc}{\color{white} FUNCTION}
fSprayVariable}}

\hypertarget{ecldoc:file.fsprayvariable}{}
\hspace{0pt} \hyperlink{ecldoc:File}{File} \textbackslash 

{\renewcommand{\arraystretch}{1.5}
\begin{tabularx}{\textwidth}{|>{\raggedright\arraybackslash}l|X|}
\hline
\hspace{0pt}\mytexttt{\color{red} varstring} & \textbf{fSprayVariable} \\
\hline
\multicolumn{2}{|>{\raggedright\arraybackslash}X|}{\hspace{0pt}\mytexttt{\color{param} (varstring sourceIP, varstring sourcePath, integer4 sourceMaxRecordSize=8192, varstring sourceCsvSeparate='\textbackslash \textbackslash ,', varstring sourceCsvTerminate='\textbackslash \textbackslash n,\textbackslash \textbackslash r\textbackslash \textbackslash n', varstring sourceCsvQuote='\textbackslash ''', varstring destinationGroup, varstring destinationLogicalName, integer4 timeOut=-1, varstring espServerIpPort=GETENV('ws\_fs\_server'), integer4 maxConnections=-1, boolean allowOverwrite=FALSE, boolean replicate=FALSE, boolean compress=FALSE, varstring sourceCsvEscape='', boolean failIfNoSourceFile=FALSE, boolean recordStructurePresent=FALSE, boolean quotedTerminator=TRUE, varstring encoding='ascii', integer4 expireDays=-1)}} \\
\hline
\end{tabularx}
}

\par


\rule{\linewidth}{0.5pt}
\subsection*{\textsf{\colorbox{headtoc}{\color{white} FUNCTION}
SprayVariable}}

\hypertarget{ecldoc:file.sprayvariable}{}
\hspace{0pt} \hyperlink{ecldoc:File}{File} \textbackslash 

{\renewcommand{\arraystretch}{1.5}
\begin{tabularx}{\textwidth}{|>{\raggedright\arraybackslash}l|X|}
\hline
\hspace{0pt}\mytexttt{\color{red} } & \textbf{SprayVariable} \\
\hline
\multicolumn{2}{|>{\raggedright\arraybackslash}X|}{\hspace{0pt}\mytexttt{\color{param} (varstring sourceIP, varstring sourcePath, integer4 sourceMaxRecordSize=8192, varstring sourceCsvSeparate='\textbackslash \textbackslash ,', varstring sourceCsvTerminate='\textbackslash \textbackslash n,\textbackslash \textbackslash r\textbackslash \textbackslash n', varstring sourceCsvQuote='\textbackslash ''', varstring destinationGroup, varstring destinationLogicalName, integer4 timeOut=-1, varstring espServerIpPort=GETENV('ws\_fs\_server'), integer4 maxConnections=-1, boolean allowOverwrite=FALSE, boolean replicate=FALSE, boolean compress=FALSE, varstring sourceCsvEscape='', boolean failIfNoSourceFile=FALSE, boolean recordStructurePresent=FALSE, boolean quotedTerminator=TRUE, varstring encoding='ascii', integer4 expireDays=-1)}} \\
\hline
\end{tabularx}
}

\par


\rule{\linewidth}{0.5pt}
\subsection*{\textsf{\colorbox{headtoc}{\color{white} FUNCTION}
fSprayDelimited}}

\hypertarget{ecldoc:file.fspraydelimited}{}
\hspace{0pt} \hyperlink{ecldoc:File}{File} \textbackslash 

{\renewcommand{\arraystretch}{1.5}
\begin{tabularx}{\textwidth}{|>{\raggedright\arraybackslash}l|X|}
\hline
\hspace{0pt}\mytexttt{\color{red} varstring} & \textbf{fSprayDelimited} \\
\hline
\multicolumn{2}{|>{\raggedright\arraybackslash}X|}{\hspace{0pt}\mytexttt{\color{param} (varstring sourceIP, varstring sourcePath, integer4 sourceMaxRecordSize=8192, varstring sourceCsvSeparate='\textbackslash \textbackslash ,', varstring sourceCsvTerminate='\textbackslash \textbackslash n,\textbackslash \textbackslash r\textbackslash \textbackslash n', varstring sourceCsvQuote='\textbackslash ''', varstring destinationGroup, varstring destinationLogicalName, integer4 timeOut=-1, varstring espServerIpPort=GETENV('ws\_fs\_server'), integer4 maxConnections=-1, boolean allowOverwrite=FALSE, boolean replicate=FALSE, boolean compress=FALSE, varstring sourceCsvEscape='', boolean failIfNoSourceFile=FALSE, boolean recordStructurePresent=FALSE, boolean quotedTerminator=TRUE, varstring encoding='ascii', integer4 expireDays=-1)}} \\
\hline
\end{tabularx}
}

\par
Sprays a file of fixed delimited records from a single machine and distributes it across the nodes of the destination group.

\par
\begin{description}
\item [\colorbox{tagtype}{\color{white} \textbf{\textsf{PARAMETER}}}] \textbf{\underline{sourceIP}} The IP address of the file.
\item [\colorbox{tagtype}{\color{white} \textbf{\textsf{PARAMETER}}}] \textbf{\underline{sourcePath}} The path and name of the file.
\item [\colorbox{tagtype}{\color{white} \textbf{\textsf{PARAMETER}}}] \textbf{\underline{sourceCsvSeparate}} The character sequence which separates fields in the file.
\item [\colorbox{tagtype}{\color{white} \textbf{\textsf{PARAMETER}}}] \textbf{\underline{sourceCsvTerminate}} The character sequence which separates records in the file.
\item [\colorbox{tagtype}{\color{white} \textbf{\textsf{PARAMETER}}}] \textbf{\underline{sourceCsvQuote}} A string which can be used to delimit fields in the file.
\item [\colorbox{tagtype}{\color{white} \textbf{\textsf{PARAMETER}}}] \textbf{\underline{sourceMaxRecordSize}} The maximum size (in bytes) of the records in the file.
\item [\colorbox{tagtype}{\color{white} \textbf{\textsf{PARAMETER}}}] \textbf{\underline{destinationGroup}} The name of the group to distribute the file across.
\item [\colorbox{tagtype}{\color{white} \textbf{\textsf{PARAMETER}}}] \textbf{\underline{destinationLogicalName}} The logical name of the file to create.
\item [\colorbox{tagtype}{\color{white} \textbf{\textsf{PARAMETER}}}] \textbf{\underline{timeOut}} The time in ms to wait for the operation to complete. A value of 0 causes the call to return immediately. Defaults to no timeout (-1).
\item [\colorbox{tagtype}{\color{white} \textbf{\textsf{PARAMETER}}}] \textbf{\underline{espServerIpPort}} The url of the ESP file copying service. Defaults to the value of ws\_fs\_server in the environment.
\item [\colorbox{tagtype}{\color{white} \textbf{\textsf{PARAMETER}}}] \textbf{\underline{maxConnections}} The maximum number of target nodes to write to concurrently. Defaults to 1.
\item [\colorbox{tagtype}{\color{white} \textbf{\textsf{PARAMETER}}}] \textbf{\underline{allowOverwrite}} Is it valid to overwrite an existing file of the same name? Defaults to FALSE
\item [\colorbox{tagtype}{\color{white} \textbf{\textsf{PARAMETER}}}] \textbf{\underline{replicate}} Whether to replicate the new file. Defaults to FALSE.
\item [\colorbox{tagtype}{\color{white} \textbf{\textsf{PARAMETER}}}] \textbf{\underline{compress}} Whether to compress the new file. Defaults to FALSE.
\item [\colorbox{tagtype}{\color{white} \textbf{\textsf{PARAMETER}}}] \textbf{\underline{sourceCsvEscape}} A character that is used to escape quote characters. Defaults to none.
\item [\colorbox{tagtype}{\color{white} \textbf{\textsf{PARAMETER}}}] \textbf{\underline{failIfNoSourceFile}} If TRUE it causes a missing source file to trigger a failure. Defaults to FALSE.
\item [\colorbox{tagtype}{\color{white} \textbf{\textsf{PARAMETER}}}] \textbf{\underline{recordStructurePresent}} If TRUE derives the record structure from the header of the file.
\item [\colorbox{tagtype}{\color{white} \textbf{\textsf{PARAMETER}}}] \textbf{\underline{quotedTerminator}} Can the terminator character be included in a quoted field. Defaults to TRUE. If FALSE it allows quicker partitioning of the file (avoiding a complete file scan).
\item [\colorbox{tagtype}{\color{white} \textbf{\textsf{PARAMETER}}}] \textbf{\underline{expireDays}} Number of days to auto-remove file. Default is -1, not expire.
\item [\colorbox{tagtype}{\color{white} \textbf{\textsf{RETURN}}}] \textbf{\underline{}} The DFU workunit id for the job.
\end{description}

\rule{\linewidth}{0.5pt}
\subsection*{\textsf{\colorbox{headtoc}{\color{white} FUNCTION}
SprayDelimited}}

\hypertarget{ecldoc:file.spraydelimited}{}
\hspace{0pt} \hyperlink{ecldoc:File}{File} \textbackslash 

{\renewcommand{\arraystretch}{1.5}
\begin{tabularx}{\textwidth}{|>{\raggedright\arraybackslash}l|X|}
\hline
\hspace{0pt}\mytexttt{\color{red} } & \textbf{SprayDelimited} \\
\hline
\multicolumn{2}{|>{\raggedright\arraybackslash}X|}{\hspace{0pt}\mytexttt{\color{param} (varstring sourceIP, varstring sourcePath, integer4 sourceMaxRecordSize=8192, varstring sourceCsvSeparate='\textbackslash \textbackslash ,', varstring sourceCsvTerminate='\textbackslash \textbackslash n,\textbackslash \textbackslash r\textbackslash \textbackslash n', varstring sourceCsvQuote='\textbackslash ''', varstring destinationGroup, varstring destinationLogicalName, integer4 timeOut=-1, varstring espServerIpPort=GETENV('ws\_fs\_server'), integer4 maxConnections=-1, boolean allowOverwrite=FALSE, boolean replicate=FALSE, boolean compress=FALSE, varstring sourceCsvEscape='', boolean failIfNoSourceFile=FALSE, boolean recordStructurePresent=FALSE, boolean quotedTerminator=TRUE, const varstring encoding='ascii', integer4 expireDays=-1)}} \\
\hline
\end{tabularx}
}

\par
Same as fSprayDelimited, but does not return the DFU Workunit ID.

\par
\begin{description}
\item [\colorbox{tagtype}{\color{white} \textbf{\textsf{SEE}}}] \textbf{\underline{}} fSprayDelimited
\end{description}

\rule{\linewidth}{0.5pt}
\subsection*{\textsf{\colorbox{headtoc}{\color{white} FUNCTION}
fSprayXml}}

\hypertarget{ecldoc:file.fsprayxml}{}
\hspace{0pt} \hyperlink{ecldoc:File}{File} \textbackslash 

{\renewcommand{\arraystretch}{1.5}
\begin{tabularx}{\textwidth}{|>{\raggedright\arraybackslash}l|X|}
\hline
\hspace{0pt}\mytexttt{\color{red} varstring} & \textbf{fSprayXml} \\
\hline
\multicolumn{2}{|>{\raggedright\arraybackslash}X|}{\hspace{0pt}\mytexttt{\color{param} (varstring sourceIP, varstring sourcePath, integer4 sourceMaxRecordSize=8192, varstring sourceRowTag, varstring sourceEncoding='utf8', varstring destinationGroup, varstring destinationLogicalName, integer4 timeOut=-1, varstring espServerIpPort=GETENV('ws\_fs\_server'), integer4 maxConnections=-1, boolean allowOverwrite=FALSE, boolean replicate=FALSE, boolean compress=FALSE, boolean failIfNoSourceFile=FALSE, integer4 expireDays=-1)}} \\
\hline
\end{tabularx}
}

\par
Sprays an xml file from a single machine and distributes it across the nodes of the destination group.

\par
\begin{description}
\item [\colorbox{tagtype}{\color{white} \textbf{\textsf{PARAMETER}}}] \textbf{\underline{sourceIP}} The IP address of the file.
\item [\colorbox{tagtype}{\color{white} \textbf{\textsf{PARAMETER}}}] \textbf{\underline{sourcePath}} The path and name of the file.
\item [\colorbox{tagtype}{\color{white} \textbf{\textsf{PARAMETER}}}] \textbf{\underline{sourceMaxRecordSize}} The maximum size (in bytes) of the records in the file.
\item [\colorbox{tagtype}{\color{white} \textbf{\textsf{PARAMETER}}}] \textbf{\underline{sourceRowTag}} The xml tag that is used to delimit records in the source file. (This tag cannot recursivly nest.)
\item [\colorbox{tagtype}{\color{white} \textbf{\textsf{PARAMETER}}}] \textbf{\underline{sourceEncoding}} The unicode encoding of the file. (utf8,utf8n,utf16be,utf16le,utf32be,utf32le)
\item [\colorbox{tagtype}{\color{white} \textbf{\textsf{PARAMETER}}}] \textbf{\underline{destinationGroup}} The name of the group to distribute the file across.
\item [\colorbox{tagtype}{\color{white} \textbf{\textsf{PARAMETER}}}] \textbf{\underline{destinationLogicalName}} The logical name of the file to create.
\item [\colorbox{tagtype}{\color{white} \textbf{\textsf{PARAMETER}}}] \textbf{\underline{timeOut}} The time in ms to wait for the operation to complete. A value of 0 causes the call to return immediately. Defaults to no timeout (-1).
\item [\colorbox{tagtype}{\color{white} \textbf{\textsf{PARAMETER}}}] \textbf{\underline{espServerIpPort}} The url of the ESP file copying service. Defaults to the value of ws\_fs\_server in the environment.
\item [\colorbox{tagtype}{\color{white} \textbf{\textsf{PARAMETER}}}] \textbf{\underline{maxConnections}} The maximum number of target nodes to write to concurrently. Defaults to 1.
\item [\colorbox{tagtype}{\color{white} \textbf{\textsf{PARAMETER}}}] \textbf{\underline{allowOverwrite}} Is it valid to overwrite an existing file of the same name? Defaults to FALSE
\item [\colorbox{tagtype}{\color{white} \textbf{\textsf{PARAMETER}}}] \textbf{\underline{replicate}} Whether to replicate the new file. Defaults to FALSE.
\item [\colorbox{tagtype}{\color{white} \textbf{\textsf{PARAMETER}}}] \textbf{\underline{compress}} Whether to compress the new file. Defaults to FALSE.
\item [\colorbox{tagtype}{\color{white} \textbf{\textsf{PARAMETER}}}] \textbf{\underline{failIfNoSourceFile}} If TRUE it causes a missing source file to trigger a failure. Defaults to FALSE.
\item [\colorbox{tagtype}{\color{white} \textbf{\textsf{PARAMETER}}}] \textbf{\underline{expireDays}} Number of days to auto-remove file. Default is -1, not expire.
\item [\colorbox{tagtype}{\color{white} \textbf{\textsf{RETURN}}}] \textbf{\underline{}} The DFU workunit id for the job.
\end{description}

\rule{\linewidth}{0.5pt}
\subsection*{\textsf{\colorbox{headtoc}{\color{white} FUNCTION}
SprayXml}}

\hypertarget{ecldoc:file.sprayxml}{}
\hspace{0pt} \hyperlink{ecldoc:File}{File} \textbackslash 

{\renewcommand{\arraystretch}{1.5}
\begin{tabularx}{\textwidth}{|>{\raggedright\arraybackslash}l|X|}
\hline
\hspace{0pt}\mytexttt{\color{red} } & \textbf{SprayXml} \\
\hline
\multicolumn{2}{|>{\raggedright\arraybackslash}X|}{\hspace{0pt}\mytexttt{\color{param} (varstring sourceIP, varstring sourcePath, integer4 sourceMaxRecordSize=8192, varstring sourceRowTag, varstring sourceEncoding='utf8', varstring destinationGroup, varstring destinationLogicalName, integer4 timeOut=-1, varstring espServerIpPort=GETENV('ws\_fs\_server'), integer4 maxConnections=-1, boolean allowOverwrite=FALSE, boolean replicate=FALSE, boolean compress=FALSE, boolean failIfNoSourceFile=FALSE, integer4 expireDays=-1)}} \\
\hline
\end{tabularx}
}

\par
Same as fSprayXml, but does not return the DFU Workunit ID.

\par
\begin{description}
\item [\colorbox{tagtype}{\color{white} \textbf{\textsf{SEE}}}] \textbf{\underline{}} fSprayXml
\end{description}

\rule{\linewidth}{0.5pt}
\subsection*{\textsf{\colorbox{headtoc}{\color{white} FUNCTION}
fDespray}}

\hypertarget{ecldoc:file.fdespray}{}
\hspace{0pt} \hyperlink{ecldoc:File}{File} \textbackslash 

{\renewcommand{\arraystretch}{1.5}
\begin{tabularx}{\textwidth}{|>{\raggedright\arraybackslash}l|X|}
\hline
\hspace{0pt}\mytexttt{\color{red} varstring} & \textbf{fDespray} \\
\hline
\multicolumn{2}{|>{\raggedright\arraybackslash}X|}{\hspace{0pt}\mytexttt{\color{param} (varstring logicalName, varstring destinationIP, varstring destinationPath, integer4 timeOut=-1, varstring espServerIpPort=GETENV('ws\_fs\_server'), integer4 maxConnections=-1, boolean allowOverwrite=FALSE)}} \\
\hline
\end{tabularx}
}

\par
Copies a distributed file from multiple machines, and desprays it to a single file on a single machine.

\par
\begin{description}
\item [\colorbox{tagtype}{\color{white} \textbf{\textsf{PARAMETER}}}] \textbf{\underline{logicalName}} The name of the file to despray.
\item [\colorbox{tagtype}{\color{white} \textbf{\textsf{PARAMETER}}}] \textbf{\underline{destinationIP}} The IP of the target machine.
\item [\colorbox{tagtype}{\color{white} \textbf{\textsf{PARAMETER}}}] \textbf{\underline{destinationPath}} The path of the file to create on the destination machine.
\item [\colorbox{tagtype}{\color{white} \textbf{\textsf{PARAMETER}}}] \textbf{\underline{timeOut}} The time in ms to wait for the operation to complete. A value of 0 causes the call to return immediately. Defaults to no timeout (-1).
\item [\colorbox{tagtype}{\color{white} \textbf{\textsf{PARAMETER}}}] \textbf{\underline{espServerIpPort}} The url of the ESP file copying service. Defaults to the value of ws\_fs\_server in the environment.
\item [\colorbox{tagtype}{\color{white} \textbf{\textsf{PARAMETER}}}] \textbf{\underline{maxConnections}} The maximum number of target nodes to write to concurrently. Defaults to 1.
\item [\colorbox{tagtype}{\color{white} \textbf{\textsf{PARAMETER}}}] \textbf{\underline{allowOverwrite}} Is it valid to overwrite an existing file of the same name? Defaults to FALSE
\item [\colorbox{tagtype}{\color{white} \textbf{\textsf{RETURN}}}] \textbf{\underline{}} The DFU workunit id for the job.
\end{description}

\rule{\linewidth}{0.5pt}
\subsection*{\textsf{\colorbox{headtoc}{\color{white} FUNCTION}
Despray}}

\hypertarget{ecldoc:file.despray}{}
\hspace{0pt} \hyperlink{ecldoc:File}{File} \textbackslash 

{\renewcommand{\arraystretch}{1.5}
\begin{tabularx}{\textwidth}{|>{\raggedright\arraybackslash}l|X|}
\hline
\hspace{0pt}\mytexttt{\color{red} } & \textbf{Despray} \\
\hline
\multicolumn{2}{|>{\raggedright\arraybackslash}X|}{\hspace{0pt}\mytexttt{\color{param} (varstring logicalName, varstring destinationIP, varstring destinationPath, integer4 timeOut=-1, varstring espServerIpPort=GETENV('ws\_fs\_server'), integer4 maxConnections=-1, boolean allowOverwrite=FALSE)}} \\
\hline
\end{tabularx}
}

\par
Same as fDespray, but does not return the DFU Workunit ID.

\par
\begin{description}
\item [\colorbox{tagtype}{\color{white} \textbf{\textsf{SEE}}}] \textbf{\underline{}} fDespray
\end{description}

\rule{\linewidth}{0.5pt}
\subsection*{\textsf{\colorbox{headtoc}{\color{white} FUNCTION}
fCopy}}

\hypertarget{ecldoc:file.fcopy}{}
\hspace{0pt} \hyperlink{ecldoc:File}{File} \textbackslash 

{\renewcommand{\arraystretch}{1.5}
\begin{tabularx}{\textwidth}{|>{\raggedright\arraybackslash}l|X|}
\hline
\hspace{0pt}\mytexttt{\color{red} varstring} & \textbf{fCopy} \\
\hline
\multicolumn{2}{|>{\raggedright\arraybackslash}X|}{\hspace{0pt}\mytexttt{\color{param} (varstring sourceLogicalName, varstring destinationGroup, varstring destinationLogicalName, varstring sourceDali='', integer4 timeOut=-1, varstring espServerIpPort=GETENV('ws\_fs\_server'), integer4 maxConnections=-1, boolean allowOverwrite=FALSE, boolean replicate=FALSE, boolean asSuperfile=FALSE, boolean compress=FALSE, boolean forcePush=FALSE, integer4 transferBufferSize=0, boolean preserveCompression=TRUE)}} \\
\hline
\end{tabularx}
}

\par
Copies a distributed file to another distributed file.

\par
\begin{description}
\item [\colorbox{tagtype}{\color{white} \textbf{\textsf{PARAMETER}}}] \textbf{\underline{sourceLogicalName}} The name of the file to despray.
\item [\colorbox{tagtype}{\color{white} \textbf{\textsf{PARAMETER}}}] \textbf{\underline{destinationGroup}} The name of the group to distribute the file across.
\item [\colorbox{tagtype}{\color{white} \textbf{\textsf{PARAMETER}}}] \textbf{\underline{destinationLogicalName}} The logical name of the file to create.
\item [\colorbox{tagtype}{\color{white} \textbf{\textsf{PARAMETER}}}] \textbf{\underline{sourceDali}} The dali that contains the source file (blank implies same dali). Defaults to same dali.
\item [\colorbox{tagtype}{\color{white} \textbf{\textsf{PARAMETER}}}] \textbf{\underline{timeOut}} The time in ms to wait for the operation to complete. A value of 0 causes the call to return immediately. Defaults to no timeout (-1).
\item [\colorbox{tagtype}{\color{white} \textbf{\textsf{PARAMETER}}}] \textbf{\underline{espServerIpPort}} The url of the ESP file copying service. Defaults to the value of ws\_fs\_server in the environment.
\item [\colorbox{tagtype}{\color{white} \textbf{\textsf{PARAMETER}}}] \textbf{\underline{maxConnections}} The maximum number of target nodes to write to concurrently. Defaults to 1.
\item [\colorbox{tagtype}{\color{white} \textbf{\textsf{PARAMETER}}}] \textbf{\underline{allowOverwrite}} Is it valid to overwrite an existing file of the same name? Defaults to FALSE
\item [\colorbox{tagtype}{\color{white} \textbf{\textsf{PARAMETER}}}] \textbf{\underline{replicate}} Should the copied file also be replicated on the destination? Defaults to FALSE
\item [\colorbox{tagtype}{\color{white} \textbf{\textsf{PARAMETER}}}] \textbf{\underline{asSuperfile}} Should the file be copied as a superfile? If TRUE and source is a superfile, then the operation creates a superfile on the target, creating sub-files as needed and only overwriting existing sub-files whose content has changed. If FALSE, a single file is created. Defaults to FALSE.
\item [\colorbox{tagtype}{\color{white} \textbf{\textsf{PARAMETER}}}] \textbf{\underline{compress}} Whether to compress the new file. Defaults to FALSE.
\item [\colorbox{tagtype}{\color{white} \textbf{\textsf{PARAMETER}}}] \textbf{\underline{forcePush}} Should the copy process be executed on the source nodes (push) or on the destination nodes (pull)? Default is to pull.
\item [\colorbox{tagtype}{\color{white} \textbf{\textsf{PARAMETER}}}] \textbf{\underline{transferBufferSize}} Overrides the size (in bytes) of the internal buffer used to copy the file. Default is 64k.
\item [\colorbox{tagtype}{\color{white} \textbf{\textsf{RETURN}}}] \textbf{\underline{}} The DFU workunit id for the job.
\end{description}

\rule{\linewidth}{0.5pt}
\subsection*{\textsf{\colorbox{headtoc}{\color{white} FUNCTION}
Copy}}

\hypertarget{ecldoc:file.copy}{}
\hspace{0pt} \hyperlink{ecldoc:File}{File} \textbackslash 

{\renewcommand{\arraystretch}{1.5}
\begin{tabularx}{\textwidth}{|>{\raggedright\arraybackslash}l|X|}
\hline
\hspace{0pt}\mytexttt{\color{red} } & \textbf{Copy} \\
\hline
\multicolumn{2}{|>{\raggedright\arraybackslash}X|}{\hspace{0pt}\mytexttt{\color{param} (varstring sourceLogicalName, varstring destinationGroup, varstring destinationLogicalName, varstring sourceDali='', integer4 timeOut=-1, varstring espServerIpPort=GETENV('ws\_fs\_server'), integer4 maxConnections=-1, boolean allowOverwrite=FALSE, boolean replicate=FALSE, boolean asSuperfile=FALSE, boolean compress=FALSE, boolean forcePush=FALSE, integer4 transferBufferSize=0, boolean preserveCompression=TRUE)}} \\
\hline
\end{tabularx}
}

\par
Same as fCopy, but does not return the DFU Workunit ID.

\par
\begin{description}
\item [\colorbox{tagtype}{\color{white} \textbf{\textsf{SEE}}}] \textbf{\underline{}} fCopy
\end{description}

\rule{\linewidth}{0.5pt}
\subsection*{\textsf{\colorbox{headtoc}{\color{white} FUNCTION}
fReplicate}}

\hypertarget{ecldoc:file.freplicate}{}
\hspace{0pt} \hyperlink{ecldoc:File}{File} \textbackslash 

{\renewcommand{\arraystretch}{1.5}
\begin{tabularx}{\textwidth}{|>{\raggedright\arraybackslash}l|X|}
\hline
\hspace{0pt}\mytexttt{\color{red} varstring} & \textbf{fReplicate} \\
\hline
\multicolumn{2}{|>{\raggedright\arraybackslash}X|}{\hspace{0pt}\mytexttt{\color{param} (varstring logicalName, integer4 timeOut=-1, varstring espServerIpPort=GETENV('ws\_fs\_server'))}} \\
\hline
\end{tabularx}
}

\par
Ensures the specified file is replicated to its mirror copies.

\par
\begin{description}
\item [\colorbox{tagtype}{\color{white} \textbf{\textsf{PARAMETER}}}] \textbf{\underline{logicalName}} The name of the file to replicate.
\item [\colorbox{tagtype}{\color{white} \textbf{\textsf{PARAMETER}}}] \textbf{\underline{timeOut}} The time in ms to wait for the operation to complete. A value of 0 causes the call to return immediately. Defaults to no timeout (-1).
\item [\colorbox{tagtype}{\color{white} \textbf{\textsf{PARAMETER}}}] \textbf{\underline{espServerIpPort}} The url of the ESP file copying service. Defaults to the value of ws\_fs\_server in the environment.
\item [\colorbox{tagtype}{\color{white} \textbf{\textsf{RETURN}}}] \textbf{\underline{}} The DFU workunit id for the job.
\end{description}

\rule{\linewidth}{0.5pt}
\subsection*{\textsf{\colorbox{headtoc}{\color{white} FUNCTION}
Replicate}}

\hypertarget{ecldoc:file.replicate}{}
\hspace{0pt} \hyperlink{ecldoc:File}{File} \textbackslash 

{\renewcommand{\arraystretch}{1.5}
\begin{tabularx}{\textwidth}{|>{\raggedright\arraybackslash}l|X|}
\hline
\hspace{0pt}\mytexttt{\color{red} } & \textbf{Replicate} \\
\hline
\multicolumn{2}{|>{\raggedright\arraybackslash}X|}{\hspace{0pt}\mytexttt{\color{param} (varstring logicalName, integer4 timeOut=-1, varstring espServerIpPort=GETENV('ws\_fs\_server'))}} \\
\hline
\end{tabularx}
}

\par
Same as fReplicated, but does not return the DFU Workunit ID.

\par
\begin{description}
\item [\colorbox{tagtype}{\color{white} \textbf{\textsf{SEE}}}] \textbf{\underline{}} fReplicate
\end{description}

\rule{\linewidth}{0.5pt}
\subsection*{\textsf{\colorbox{headtoc}{\color{white} FUNCTION}
fRemotePull}}

\hypertarget{ecldoc:file.fremotepull}{}
\hspace{0pt} \hyperlink{ecldoc:File}{File} \textbackslash 

{\renewcommand{\arraystretch}{1.5}
\begin{tabularx}{\textwidth}{|>{\raggedright\arraybackslash}l|X|}
\hline
\hspace{0pt}\mytexttt{\color{red} varstring} & \textbf{fRemotePull} \\
\hline
\multicolumn{2}{|>{\raggedright\arraybackslash}X|}{\hspace{0pt}\mytexttt{\color{param} (varstring remoteEspFsURL, varstring sourceLogicalName, varstring destinationGroup, varstring destinationLogicalName, integer4 timeOut=-1, integer4 maxConnections=-1, boolean allowOverwrite=FALSE, boolean replicate=FALSE, boolean asSuperfile=FALSE, boolean forcePush=FALSE, integer4 transferBufferSize=0, boolean wrap=FALSE, boolean compress=FALSE)}} \\
\hline
\end{tabularx}
}

\par
Copies a distributed file to a distributed file on remote system. Similar to fCopy, except the copy executes remotely. Since the DFU workunit executes on the remote DFU server, the user name authentication must be the same on both systems, and the user must have rights to copy files on both systems.

\par
\begin{description}
\item [\colorbox{tagtype}{\color{white} \textbf{\textsf{PARAMETER}}}] \textbf{\underline{remoteEspFsURL}} The url of the remote ESP file copying service.
\item [\colorbox{tagtype}{\color{white} \textbf{\textsf{PARAMETER}}}] \textbf{\underline{sourceLogicalName}} The name of the file to despray.
\item [\colorbox{tagtype}{\color{white} \textbf{\textsf{PARAMETER}}}] \textbf{\underline{destinationGroup}} The name of the group to distribute the file across.
\item [\colorbox{tagtype}{\color{white} \textbf{\textsf{PARAMETER}}}] \textbf{\underline{destinationLogicalName}} The logical name of the file to create.
\item [\colorbox{tagtype}{\color{white} \textbf{\textsf{PARAMETER}}}] \textbf{\underline{timeOut}} The time in ms to wait for the operation to complete. A value of 0 causes the call to return immediately. Defaults to no timeout (-1).
\item [\colorbox{tagtype}{\color{white} \textbf{\textsf{PARAMETER}}}] \textbf{\underline{maxConnections}} The maximum number of target nodes to write to concurrently. Defaults to 1.
\item [\colorbox{tagtype}{\color{white} \textbf{\textsf{PARAMETER}}}] \textbf{\underline{allowOverwrite}} Is it valid to overwrite an existing file of the same name? Defaults to FALSE
\item [\colorbox{tagtype}{\color{white} \textbf{\textsf{PARAMETER}}}] \textbf{\underline{replicate}} Should the copied file also be replicated on the destination? Defaults to FALSE
\item [\colorbox{tagtype}{\color{white} \textbf{\textsf{PARAMETER}}}] \textbf{\underline{asSuperfile}} Should the file be copied as a superfile? If TRUE and source is a superfile, then the operation creates a superfile on the target, creating sub-files as needed and only overwriting existing sub-files whose content has changed. If FALSE a single file is created. Defaults to FALSE.
\item [\colorbox{tagtype}{\color{white} \textbf{\textsf{PARAMETER}}}] \textbf{\underline{compress}} Whether to compress the new file. Defaults to FALSE.
\item [\colorbox{tagtype}{\color{white} \textbf{\textsf{PARAMETER}}}] \textbf{\underline{forcePush}} Should the copy process should be executed on the source nodes (push) or on the destination nodes (pull)? Default is to pull.
\item [\colorbox{tagtype}{\color{white} \textbf{\textsf{PARAMETER}}}] \textbf{\underline{transferBufferSize}} Overrides the size (in bytes) of the internal buffer used to copy the file. Default is 64k.
\item [\colorbox{tagtype}{\color{white} \textbf{\textsf{PARAMETER}}}] \textbf{\underline{wrap}} Should the fileparts be wrapped when copying to a smaller sized cluster? The default is FALSE.
\item [\colorbox{tagtype}{\color{white} \textbf{\textsf{RETURN}}}] \textbf{\underline{}} The DFU workunit id for the job.
\end{description}

\rule{\linewidth}{0.5pt}
\subsection*{\textsf{\colorbox{headtoc}{\color{white} FUNCTION}
RemotePull}}

\hypertarget{ecldoc:file.remotepull}{}
\hspace{0pt} \hyperlink{ecldoc:File}{File} \textbackslash 

{\renewcommand{\arraystretch}{1.5}
\begin{tabularx}{\textwidth}{|>{\raggedright\arraybackslash}l|X|}
\hline
\hspace{0pt}\mytexttt{\color{red} } & \textbf{RemotePull} \\
\hline
\multicolumn{2}{|>{\raggedright\arraybackslash}X|}{\hspace{0pt}\mytexttt{\color{param} (varstring remoteEspFsURL, varstring sourceLogicalName, varstring destinationGroup, varstring destinationLogicalName, integer4 timeOut=-1, integer4 maxConnections=-1, boolean allowOverwrite=FALSE, boolean replicate=FALSE, boolean asSuperfile=FALSE, boolean forcePush=FALSE, integer4 transferBufferSize=0, boolean wrap=FALSE, boolean compress=FALSE)}} \\
\hline
\end{tabularx}
}

\par
Same as fRemotePull, but does not return the DFU Workunit ID.

\par
\begin{description}
\item [\colorbox{tagtype}{\color{white} \textbf{\textsf{SEE}}}] \textbf{\underline{}} fRemotePull
\end{description}

\rule{\linewidth}{0.5pt}
\subsection*{\textsf{\colorbox{headtoc}{\color{white} FUNCTION}
fMonitorLogicalFileName}}

\hypertarget{ecldoc:file.fmonitorlogicalfilename}{}
\hspace{0pt} \hyperlink{ecldoc:File}{File} \textbackslash 

{\renewcommand{\arraystretch}{1.5}
\begin{tabularx}{\textwidth}{|>{\raggedright\arraybackslash}l|X|}
\hline
\hspace{0pt}\mytexttt{\color{red} varstring} & \textbf{fMonitorLogicalFileName} \\
\hline
\multicolumn{2}{|>{\raggedright\arraybackslash}X|}{\hspace{0pt}\mytexttt{\color{param} (varstring eventToFire, varstring name, integer4 shotCount=1, varstring espServerIpPort=GETENV('ws\_fs\_server'))}} \\
\hline
\end{tabularx}
}

\par
Creates a file monitor job in the DFU Server. If an appropriately named file arrives in this interval it will fire the event with the name of the triggering object as the event subtype (see the EVENT function).

\par
\begin{description}
\item [\colorbox{tagtype}{\color{white} \textbf{\textsf{PARAMETER}}}] \textbf{\underline{eventToFire}} The user-defined name of the event to fire when the filename appears. This value is used as the first parameter to the EVENT function.
\item [\colorbox{tagtype}{\color{white} \textbf{\textsf{PARAMETER}}}] \textbf{\underline{name}} The name of the logical file to monitor. This may contain wildcard characters ( * and ?)
\item [\colorbox{tagtype}{\color{white} \textbf{\textsf{PARAMETER}}}] \textbf{\underline{shotCount}} The number of times to generate the event before the monitoring job completes. A value of -1 indicates the monitoring job continues until manually aborted. The default is 1.
\item [\colorbox{tagtype}{\color{white} \textbf{\textsf{PARAMETER}}}] \textbf{\underline{espServerIpPort}} The url of the ESP file copying service. Defaults to the value of ws\_fs\_server in the environment.
\item [\colorbox{tagtype}{\color{white} \textbf{\textsf{RETURN}}}] \textbf{\underline{}} The DFU workunit id for the job.
\end{description}

\rule{\linewidth}{0.5pt}
\subsection*{\textsf{\colorbox{headtoc}{\color{white} FUNCTION}
MonitorLogicalFileName}}

\hypertarget{ecldoc:file.monitorlogicalfilename}{}
\hspace{0pt} \hyperlink{ecldoc:File}{File} \textbackslash 

{\renewcommand{\arraystretch}{1.5}
\begin{tabularx}{\textwidth}{|>{\raggedright\arraybackslash}l|X|}
\hline
\hspace{0pt}\mytexttt{\color{red} } & \textbf{MonitorLogicalFileName} \\
\hline
\multicolumn{2}{|>{\raggedright\arraybackslash}X|}{\hspace{0pt}\mytexttt{\color{param} (varstring eventToFire, varstring name, integer4 shotCount=1, varstring espServerIpPort=GETENV('ws\_fs\_server'))}} \\
\hline
\end{tabularx}
}

\par
Same as fMonitorLogicalFileName, but does not return the DFU Workunit ID.

\par
\begin{description}
\item [\colorbox{tagtype}{\color{white} \textbf{\textsf{SEE}}}] \textbf{\underline{}} fMonitorLogicalFileName
\end{description}

\rule{\linewidth}{0.5pt}
\subsection*{\textsf{\colorbox{headtoc}{\color{white} FUNCTION}
fMonitorFile}}

\hypertarget{ecldoc:file.fmonitorfile}{}
\hspace{0pt} \hyperlink{ecldoc:File}{File} \textbackslash 

{\renewcommand{\arraystretch}{1.5}
\begin{tabularx}{\textwidth}{|>{\raggedright\arraybackslash}l|X|}
\hline
\hspace{0pt}\mytexttt{\color{red} varstring} & \textbf{fMonitorFile} \\
\hline
\multicolumn{2}{|>{\raggedright\arraybackslash}X|}{\hspace{0pt}\mytexttt{\color{param} (varstring eventToFire, varstring ip, varstring filename, boolean subDirs=FALSE, integer4 shotCount=1, varstring espServerIpPort=GETENV('ws\_fs\_server'))}} \\
\hline
\end{tabularx}
}

\par
Creates a file monitor job in the DFU Server. If an appropriately named file arrives in this interval it will fire the event with the name of the triggering object as the event subtype (see the EVENT function).

\par
\begin{description}
\item [\colorbox{tagtype}{\color{white} \textbf{\textsf{PARAMETER}}}] \textbf{\underline{eventToFire}} The user-defined name of the event to fire when the filename appears. This value is used as the first parameter to the EVENT function.
\item [\colorbox{tagtype}{\color{white} \textbf{\textsf{PARAMETER}}}] \textbf{\underline{ip}} The the IP address for the file to monitor. This may be omitted if the filename parameter contains a complete URL.
\item [\colorbox{tagtype}{\color{white} \textbf{\textsf{PARAMETER}}}] \textbf{\underline{filename}} The full path of the file(s) to monitor. This may contain wildcard characters ( * and ?)
\item [\colorbox{tagtype}{\color{white} \textbf{\textsf{PARAMETER}}}] \textbf{\underline{subDirs}} Whether to include files in sub-directories (when the filename contains wildcards). Defaults to FALSE.
\item [\colorbox{tagtype}{\color{white} \textbf{\textsf{PARAMETER}}}] \textbf{\underline{shotCount}} The number of times to generate the event before the monitoring job completes. A value of -1 indicates the monitoring job continues until manually aborted. The default is 1.
\item [\colorbox{tagtype}{\color{white} \textbf{\textsf{PARAMETER}}}] \textbf{\underline{espServerIpPort}} The url of the ESP file copying service. Defaults to the value of ws\_fs\_server in the environment.
\item [\colorbox{tagtype}{\color{white} \textbf{\textsf{RETURN}}}] \textbf{\underline{}} The DFU workunit id for the job.
\end{description}

\rule{\linewidth}{0.5pt}
\subsection*{\textsf{\colorbox{headtoc}{\color{white} FUNCTION}
MonitorFile}}

\hypertarget{ecldoc:file.monitorfile}{}
\hspace{0pt} \hyperlink{ecldoc:File}{File} \textbackslash 

{\renewcommand{\arraystretch}{1.5}
\begin{tabularx}{\textwidth}{|>{\raggedright\arraybackslash}l|X|}
\hline
\hspace{0pt}\mytexttt{\color{red} } & \textbf{MonitorFile} \\
\hline
\multicolumn{2}{|>{\raggedright\arraybackslash}X|}{\hspace{0pt}\mytexttt{\color{param} (varstring eventToFire, varstring ip, varstring filename, boolean subdirs=FALSE, integer4 shotCount=1, varstring espServerIpPort=GETENV('ws\_fs\_server'))}} \\
\hline
\end{tabularx}
}

\par
Same as fMonitorFile, but does not return the DFU Workunit ID.

\par
\begin{description}
\item [\colorbox{tagtype}{\color{white} \textbf{\textsf{SEE}}}] \textbf{\underline{}} fMonitorFile
\end{description}

\rule{\linewidth}{0.5pt}
\subsection*{\textsf{\colorbox{headtoc}{\color{white} FUNCTION}
WaitDfuWorkunit}}

\hypertarget{ecldoc:file.waitdfuworkunit}{}
\hspace{0pt} \hyperlink{ecldoc:File}{File} \textbackslash 

{\renewcommand{\arraystretch}{1.5}
\begin{tabularx}{\textwidth}{|>{\raggedright\arraybackslash}l|X|}
\hline
\hspace{0pt}\mytexttt{\color{red} varstring} & \textbf{WaitDfuWorkunit} \\
\hline
\multicolumn{2}{|>{\raggedright\arraybackslash}X|}{\hspace{0pt}\mytexttt{\color{param} (varstring wuid, integer4 timeOut=-1, varstring espServerIpPort=GETENV('ws\_fs\_server'))}} \\
\hline
\end{tabularx}
}

\par
Waits for the specified DFU workunit to finish.

\par
\begin{description}
\item [\colorbox{tagtype}{\color{white} \textbf{\textsf{PARAMETER}}}] \textbf{\underline{wuid}} The dfu wfid to wait for.
\item [\colorbox{tagtype}{\color{white} \textbf{\textsf{PARAMETER}}}] \textbf{\underline{timeOut}} The time in ms to wait for the operation to complete. A value of 0 causes the call to return immediately. Defaults to no timeout (-1).
\item [\colorbox{tagtype}{\color{white} \textbf{\textsf{PARAMETER}}}] \textbf{\underline{espServerIpPort}} The url of the ESP file copying service. Defaults to the value of ws\_fs\_server in the environment.
\item [\colorbox{tagtype}{\color{white} \textbf{\textsf{RETURN}}}] \textbf{\underline{}} A string containing the final status string of the DFU workunit.
\end{description}

\rule{\linewidth}{0.5pt}
\subsection*{\textsf{\colorbox{headtoc}{\color{white} FUNCTION}
AbortDfuWorkunit}}

\hypertarget{ecldoc:file.abortdfuworkunit}{}
\hspace{0pt} \hyperlink{ecldoc:File}{File} \textbackslash 

{\renewcommand{\arraystretch}{1.5}
\begin{tabularx}{\textwidth}{|>{\raggedright\arraybackslash}l|X|}
\hline
\hspace{0pt}\mytexttt{\color{red} } & \textbf{AbortDfuWorkunit} \\
\hline
\multicolumn{2}{|>{\raggedright\arraybackslash}X|}{\hspace{0pt}\mytexttt{\color{param} (varstring wuid, varstring espServerIpPort=GETENV('ws\_fs\_server'))}} \\
\hline
\end{tabularx}
}

\par
Aborts the specified DFU workunit.

\par
\begin{description}
\item [\colorbox{tagtype}{\color{white} \textbf{\textsf{PARAMETER}}}] \textbf{\underline{wuid}} The dfu wfid to abort.
\item [\colorbox{tagtype}{\color{white} \textbf{\textsf{PARAMETER}}}] \textbf{\underline{espServerIpPort}} The url of the ESP file copying service. Defaults to the value of ws\_fs\_server in the environment.
\end{description}

\rule{\linewidth}{0.5pt}
\subsection*{\textsf{\colorbox{headtoc}{\color{white} FUNCTION}
CreateSuperFile}}

\hypertarget{ecldoc:file.createsuperfile}{}
\hspace{0pt} \hyperlink{ecldoc:File}{File} \textbackslash 

{\renewcommand{\arraystretch}{1.5}
\begin{tabularx}{\textwidth}{|>{\raggedright\arraybackslash}l|X|}
\hline
\hspace{0pt}\mytexttt{\color{red} } & \textbf{CreateSuperFile} \\
\hline
\multicolumn{2}{|>{\raggedright\arraybackslash}X|}{\hspace{0pt}\mytexttt{\color{param} (varstring superName, boolean sequentialParts=FALSE, boolean allowExist=FALSE)}} \\
\hline
\end{tabularx}
}

\par
Creates an empty superfile. This function is not included in a superfile transaction.

\par
\begin{description}
\item [\colorbox{tagtype}{\color{white} \textbf{\textsf{PARAMETER}}}] \textbf{\underline{superName}} The logical name of the superfile.
\item [\colorbox{tagtype}{\color{white} \textbf{\textsf{PARAMETER}}}] \textbf{\underline{sequentialParts}} Whether the sub-files must be sequentially ordered. Default to FALSE.
\item [\colorbox{tagtype}{\color{white} \textbf{\textsf{PARAMETER}}}] \textbf{\underline{allowExist}} Indicating whether to post an error if the superfile already exists. If TRUE, no error is posted. Defaults to FALSE.
\end{description}

\rule{\linewidth}{0.5pt}
\subsection*{\textsf{\colorbox{headtoc}{\color{white} FUNCTION}
SuperFileExists}}

\hypertarget{ecldoc:file.superfileexists}{}
\hspace{0pt} \hyperlink{ecldoc:File}{File} \textbackslash 

{\renewcommand{\arraystretch}{1.5}
\begin{tabularx}{\textwidth}{|>{\raggedright\arraybackslash}l|X|}
\hline
\hspace{0pt}\mytexttt{\color{red} boolean} & \textbf{SuperFileExists} \\
\hline
\multicolumn{2}{|>{\raggedright\arraybackslash}X|}{\hspace{0pt}\mytexttt{\color{param} (varstring superName)}} \\
\hline
\end{tabularx}
}

\par
Checks if the specified filename is present in the Distributed File Utility (DFU) and is a SuperFile.

\par
\begin{description}
\item [\colorbox{tagtype}{\color{white} \textbf{\textsf{PARAMETER}}}] \textbf{\underline{superName}} The logical name of the superfile.
\item [\colorbox{tagtype}{\color{white} \textbf{\textsf{RETURN}}}] \textbf{\underline{}} Whether the file exists.
\item [\colorbox{tagtype}{\color{white} \textbf{\textsf{SEE}}}] \textbf{\underline{}} FileExists
\end{description}

\rule{\linewidth}{0.5pt}
\subsection*{\textsf{\colorbox{headtoc}{\color{white} FUNCTION}
DeleteSuperFile}}

\hypertarget{ecldoc:file.deletesuperfile}{}
\hspace{0pt} \hyperlink{ecldoc:File}{File} \textbackslash 

{\renewcommand{\arraystretch}{1.5}
\begin{tabularx}{\textwidth}{|>{\raggedright\arraybackslash}l|X|}
\hline
\hspace{0pt}\mytexttt{\color{red} } & \textbf{DeleteSuperFile} \\
\hline
\multicolumn{2}{|>{\raggedright\arraybackslash}X|}{\hspace{0pt}\mytexttt{\color{param} (varstring superName, boolean deletesub=FALSE)}} \\
\hline
\end{tabularx}
}

\par
Deletes the superfile.

\par
\begin{description}
\item [\colorbox{tagtype}{\color{white} \textbf{\textsf{PARAMETER}}}] \textbf{\underline{superName}} The logical name of the superfile.
\item [\colorbox{tagtype}{\color{white} \textbf{\textsf{SEE}}}] \textbf{\underline{}} FileExists
\end{description}

\rule{\linewidth}{0.5pt}
\subsection*{\textsf{\colorbox{headtoc}{\color{white} FUNCTION}
GetSuperFileSubCount}}

\hypertarget{ecldoc:file.getsuperfilesubcount}{}
\hspace{0pt} \hyperlink{ecldoc:File}{File} \textbackslash 

{\renewcommand{\arraystretch}{1.5}
\begin{tabularx}{\textwidth}{|>{\raggedright\arraybackslash}l|X|}
\hline
\hspace{0pt}\mytexttt{\color{red} unsigned4} & \textbf{GetSuperFileSubCount} \\
\hline
\multicolumn{2}{|>{\raggedright\arraybackslash}X|}{\hspace{0pt}\mytexttt{\color{param} (varstring superName)}} \\
\hline
\end{tabularx}
}

\par
Returns the number of sub-files contained within a superfile.

\par
\begin{description}
\item [\colorbox{tagtype}{\color{white} \textbf{\textsf{PARAMETER}}}] \textbf{\underline{superName}} The logical name of the superfile.
\item [\colorbox{tagtype}{\color{white} \textbf{\textsf{RETURN}}}] \textbf{\underline{}} The number of sub-files within the superfile.
\end{description}

\rule{\linewidth}{0.5pt}
\subsection*{\textsf{\colorbox{headtoc}{\color{white} FUNCTION}
GetSuperFileSubName}}

\hypertarget{ecldoc:file.getsuperfilesubname}{}
\hspace{0pt} \hyperlink{ecldoc:File}{File} \textbackslash 

{\renewcommand{\arraystretch}{1.5}
\begin{tabularx}{\textwidth}{|>{\raggedright\arraybackslash}l|X|}
\hline
\hspace{0pt}\mytexttt{\color{red} varstring} & \textbf{GetSuperFileSubName} \\
\hline
\multicolumn{2}{|>{\raggedright\arraybackslash}X|}{\hspace{0pt}\mytexttt{\color{param} (varstring superName, unsigned4 fileNum, boolean absPath=FALSE)}} \\
\hline
\end{tabularx}
}

\par
Returns the name of the Nth sub-file within a superfile.

\par
\begin{description}
\item [\colorbox{tagtype}{\color{white} \textbf{\textsf{PARAMETER}}}] \textbf{\underline{superName}} The logical name of the superfile.
\item [\colorbox{tagtype}{\color{white} \textbf{\textsf{PARAMETER}}}] \textbf{\underline{fileNum}} The 1-based position of the sub-file to return the name of.
\item [\colorbox{tagtype}{\color{white} \textbf{\textsf{PARAMETER}}}] \textbf{\underline{absPath}} Whether to prepend '\~{}' to the name of the resulting logical file name.
\item [\colorbox{tagtype}{\color{white} \textbf{\textsf{RETURN}}}] \textbf{\underline{}} The logical name of the selected sub-file.
\end{description}

\rule{\linewidth}{0.5pt}
\subsection*{\textsf{\colorbox{headtoc}{\color{white} FUNCTION}
FindSuperFileSubName}}

\hypertarget{ecldoc:file.findsuperfilesubname}{}
\hspace{0pt} \hyperlink{ecldoc:File}{File} \textbackslash 

{\renewcommand{\arraystretch}{1.5}
\begin{tabularx}{\textwidth}{|>{\raggedright\arraybackslash}l|X|}
\hline
\hspace{0pt}\mytexttt{\color{red} unsigned4} & \textbf{FindSuperFileSubName} \\
\hline
\multicolumn{2}{|>{\raggedright\arraybackslash}X|}{\hspace{0pt}\mytexttt{\color{param} (varstring superName, varstring subName)}} \\
\hline
\end{tabularx}
}

\par
Returns the position of a file within a superfile.

\par
\begin{description}
\item [\colorbox{tagtype}{\color{white} \textbf{\textsf{PARAMETER}}}] \textbf{\underline{superName}} The logical name of the superfile.
\item [\colorbox{tagtype}{\color{white} \textbf{\textsf{PARAMETER}}}] \textbf{\underline{subName}} The logical name of the sub-file.
\item [\colorbox{tagtype}{\color{white} \textbf{\textsf{RETURN}}}] \textbf{\underline{}} The 1-based position of the sub-file within the superfile.
\end{description}

\rule{\linewidth}{0.5pt}
\subsection*{\textsf{\colorbox{headtoc}{\color{white} FUNCTION}
StartSuperFileTransaction}}

\hypertarget{ecldoc:file.startsuperfiletransaction}{}
\hspace{0pt} \hyperlink{ecldoc:File}{File} \textbackslash 

{\renewcommand{\arraystretch}{1.5}
\begin{tabularx}{\textwidth}{|>{\raggedright\arraybackslash}l|X|}
\hline
\hspace{0pt}\mytexttt{\color{red} } & \textbf{StartSuperFileTransaction} \\
\hline
\multicolumn{2}{|>{\raggedright\arraybackslash}X|}{\hspace{0pt}\mytexttt{\color{param} ()}} \\
\hline
\end{tabularx}
}

\par
Starts a superfile transaction. All superfile operations within the transaction will either be executed atomically or rolled back when the transaction is finished.


\rule{\linewidth}{0.5pt}
\subsection*{\textsf{\colorbox{headtoc}{\color{white} FUNCTION}
AddSuperFile}}

\hypertarget{ecldoc:file.addsuperfile}{}
\hspace{0pt} \hyperlink{ecldoc:File}{File} \textbackslash 

{\renewcommand{\arraystretch}{1.5}
\begin{tabularx}{\textwidth}{|>{\raggedright\arraybackslash}l|X|}
\hline
\hspace{0pt}\mytexttt{\color{red} } & \textbf{AddSuperFile} \\
\hline
\multicolumn{2}{|>{\raggedright\arraybackslash}X|}{\hspace{0pt}\mytexttt{\color{param} (varstring superName, varstring subName, unsigned4 atPos=0, boolean addContents=FALSE, boolean strict=FALSE)}} \\
\hline
\end{tabularx}
}

\par
Adds a file to a superfile.

\par
\begin{description}
\item [\colorbox{tagtype}{\color{white} \textbf{\textsf{PARAMETER}}}] \textbf{\underline{superName}} The logical name of the superfile.
\item [\colorbox{tagtype}{\color{white} \textbf{\textsf{PARAMETER}}}] \textbf{\underline{subName}} The name of the logical file to add.
\item [\colorbox{tagtype}{\color{white} \textbf{\textsf{PARAMETER}}}] \textbf{\underline{atPos}} The position to add the sub-file, or 0 to append. Defaults to 0.
\item [\colorbox{tagtype}{\color{white} \textbf{\textsf{PARAMETER}}}] \textbf{\underline{addContents}} Controls whether adding a superfile adds the superfile, or its contents. Defaults to FALSE (do not expand).
\item [\colorbox{tagtype}{\color{white} \textbf{\textsf{PARAMETER}}}] \textbf{\underline{strict}} Check addContents only if subName is a superfile, and ensure superfiles exist.
\end{description}

\rule{\linewidth}{0.5pt}
\subsection*{\textsf{\colorbox{headtoc}{\color{white} FUNCTION}
RemoveSuperFile}}

\hypertarget{ecldoc:file.removesuperfile}{}
\hspace{0pt} \hyperlink{ecldoc:File}{File} \textbackslash 

{\renewcommand{\arraystretch}{1.5}
\begin{tabularx}{\textwidth}{|>{\raggedright\arraybackslash}l|X|}
\hline
\hspace{0pt}\mytexttt{\color{red} } & \textbf{RemoveSuperFile} \\
\hline
\multicolumn{2}{|>{\raggedright\arraybackslash}X|}{\hspace{0pt}\mytexttt{\color{param} (varstring superName, varstring subName, boolean del=FALSE, boolean removeContents=FALSE)}} \\
\hline
\end{tabularx}
}

\par
Removes a sub-file from a superfile.

\par
\begin{description}
\item [\colorbox{tagtype}{\color{white} \textbf{\textsf{PARAMETER}}}] \textbf{\underline{superName}} The logical name of the superfile.
\item [\colorbox{tagtype}{\color{white} \textbf{\textsf{PARAMETER}}}] \textbf{\underline{subName}} The name of the sub-file to remove.
\item [\colorbox{tagtype}{\color{white} \textbf{\textsf{PARAMETER}}}] \textbf{\underline{del}} Indicates whether the sub-file should also be removed from the disk. Defaults to FALSE.
\item [\colorbox{tagtype}{\color{white} \textbf{\textsf{PARAMETER}}}] \textbf{\underline{removeContents}} Controls whether the contents of a sub-file which is a superfile should be recursively removed. Defaults to FALSE.
\end{description}

\rule{\linewidth}{0.5pt}
\subsection*{\textsf{\colorbox{headtoc}{\color{white} FUNCTION}
ClearSuperFile}}

\hypertarget{ecldoc:file.clearsuperfile}{}
\hspace{0pt} \hyperlink{ecldoc:File}{File} \textbackslash 

{\renewcommand{\arraystretch}{1.5}
\begin{tabularx}{\textwidth}{|>{\raggedright\arraybackslash}l|X|}
\hline
\hspace{0pt}\mytexttt{\color{red} } & \textbf{ClearSuperFile} \\
\hline
\multicolumn{2}{|>{\raggedright\arraybackslash}X|}{\hspace{0pt}\mytexttt{\color{param} (varstring superName, boolean del=FALSE)}} \\
\hline
\end{tabularx}
}

\par
Removes all sub-files from a superfile.

\par
\begin{description}
\item [\colorbox{tagtype}{\color{white} \textbf{\textsf{PARAMETER}}}] \textbf{\underline{superName}} The logical name of the superfile.
\item [\colorbox{tagtype}{\color{white} \textbf{\textsf{PARAMETER}}}] \textbf{\underline{del}} Indicates whether the sub-files should also be removed from the disk. Defaults to FALSE.
\end{description}

\rule{\linewidth}{0.5pt}
\subsection*{\textsf{\colorbox{headtoc}{\color{white} FUNCTION}
RemoveOwnedSubFiles}}

\hypertarget{ecldoc:file.removeownedsubfiles}{}
\hspace{0pt} \hyperlink{ecldoc:File}{File} \textbackslash 

{\renewcommand{\arraystretch}{1.5}
\begin{tabularx}{\textwidth}{|>{\raggedright\arraybackslash}l|X|}
\hline
\hspace{0pt}\mytexttt{\color{red} } & \textbf{RemoveOwnedSubFiles} \\
\hline
\multicolumn{2}{|>{\raggedright\arraybackslash}X|}{\hspace{0pt}\mytexttt{\color{param} (varstring superName, boolean del=FALSE)}} \\
\hline
\end{tabularx}
}

\par
Removes all soley-owned sub-files from a superfile. If a sub-file is also contained within another superfile then it is retained.

\par
\begin{description}
\item [\colorbox{tagtype}{\color{white} \textbf{\textsf{PARAMETER}}}] \textbf{\underline{superName}} The logical name of the superfile.
\end{description}

\rule{\linewidth}{0.5pt}
\subsection*{\textsf{\colorbox{headtoc}{\color{white} FUNCTION}
DeleteOwnedSubFiles}}

\hypertarget{ecldoc:file.deleteownedsubfiles}{}
\hspace{0pt} \hyperlink{ecldoc:File}{File} \textbackslash 

{\renewcommand{\arraystretch}{1.5}
\begin{tabularx}{\textwidth}{|>{\raggedright\arraybackslash}l|X|}
\hline
\hspace{0pt}\mytexttt{\color{red} } & \textbf{DeleteOwnedSubFiles} \\
\hline
\multicolumn{2}{|>{\raggedright\arraybackslash}X|}{\hspace{0pt}\mytexttt{\color{param} (varstring superName)}} \\
\hline
\end{tabularx}
}

\par
Legacy version of RemoveOwnedSubFiles which was incorrectly named in a previous version.

\par
\begin{description}
\item [\colorbox{tagtype}{\color{white} \textbf{\textsf{SEE}}}] \textbf{\underline{}} RemoveOwnedSubFIles
\end{description}

\rule{\linewidth}{0.5pt}
\subsection*{\textsf{\colorbox{headtoc}{\color{white} FUNCTION}
SwapSuperFile}}

\hypertarget{ecldoc:file.swapsuperfile}{}
\hspace{0pt} \hyperlink{ecldoc:File}{File} \textbackslash 

{\renewcommand{\arraystretch}{1.5}
\begin{tabularx}{\textwidth}{|>{\raggedright\arraybackslash}l|X|}
\hline
\hspace{0pt}\mytexttt{\color{red} } & \textbf{SwapSuperFile} \\
\hline
\multicolumn{2}{|>{\raggedright\arraybackslash}X|}{\hspace{0pt}\mytexttt{\color{param} (varstring superName1, varstring superName2)}} \\
\hline
\end{tabularx}
}

\par
Swap the contents of two superfiles.

\par
\begin{description}
\item [\colorbox{tagtype}{\color{white} \textbf{\textsf{PARAMETER}}}] \textbf{\underline{superName1}} The logical name of the first superfile.
\item [\colorbox{tagtype}{\color{white} \textbf{\textsf{PARAMETER}}}] \textbf{\underline{superName2}} The logical name of the second superfile.
\end{description}

\rule{\linewidth}{0.5pt}
\subsection*{\textsf{\colorbox{headtoc}{\color{white} FUNCTION}
ReplaceSuperFile}}

\hypertarget{ecldoc:file.replacesuperfile}{}
\hspace{0pt} \hyperlink{ecldoc:File}{File} \textbackslash 

{\renewcommand{\arraystretch}{1.5}
\begin{tabularx}{\textwidth}{|>{\raggedright\arraybackslash}l|X|}
\hline
\hspace{0pt}\mytexttt{\color{red} } & \textbf{ReplaceSuperFile} \\
\hline
\multicolumn{2}{|>{\raggedright\arraybackslash}X|}{\hspace{0pt}\mytexttt{\color{param} (varstring superName, varstring oldSubFile, varstring newSubFile)}} \\
\hline
\end{tabularx}
}

\par
Removes a sub-file from a superfile and replaces it with another.

\par
\begin{description}
\item [\colorbox{tagtype}{\color{white} \textbf{\textsf{PARAMETER}}}] \textbf{\underline{superName}} The logical name of the superfile.
\item [\colorbox{tagtype}{\color{white} \textbf{\textsf{PARAMETER}}}] \textbf{\underline{oldSubFile}} The logical name of the sub-file to remove.
\item [\colorbox{tagtype}{\color{white} \textbf{\textsf{PARAMETER}}}] \textbf{\underline{newSubFile}} The logical name of the sub-file to replace within the superfile.
\end{description}

\rule{\linewidth}{0.5pt}
\subsection*{\textsf{\colorbox{headtoc}{\color{white} FUNCTION}
FinishSuperFileTransaction}}

\hypertarget{ecldoc:file.finishsuperfiletransaction}{}
\hspace{0pt} \hyperlink{ecldoc:File}{File} \textbackslash 

{\renewcommand{\arraystretch}{1.5}
\begin{tabularx}{\textwidth}{|>{\raggedright\arraybackslash}l|X|}
\hline
\hspace{0pt}\mytexttt{\color{red} } & \textbf{FinishSuperFileTransaction} \\
\hline
\multicolumn{2}{|>{\raggedright\arraybackslash}X|}{\hspace{0pt}\mytexttt{\color{param} (boolean rollback=FALSE)}} \\
\hline
\end{tabularx}
}

\par
Finishes a superfile transaction. This executes all the operations since the matching StartSuperFileTransaction(). If there are any errors, then all of the operations are rolled back.


\rule{\linewidth}{0.5pt}
\subsection*{\textsf{\colorbox{headtoc}{\color{white} FUNCTION}
SuperFileContents}}

\hypertarget{ecldoc:file.superfilecontents}{}
\hspace{0pt} \hyperlink{ecldoc:File}{File} \textbackslash 

{\renewcommand{\arraystretch}{1.5}
\begin{tabularx}{\textwidth}{|>{\raggedright\arraybackslash}l|X|}
\hline
\hspace{0pt}\mytexttt{\color{red} dataset(FsLogicalFileNameRecord)} & \textbf{SuperFileContents} \\
\hline
\multicolumn{2}{|>{\raggedright\arraybackslash}X|}{\hspace{0pt}\mytexttt{\color{param} (varstring superName, boolean recurse=FALSE)}} \\
\hline
\end{tabularx}
}

\par
Returns the list of sub-files contained within a superfile.

\par
\begin{description}
\item [\colorbox{tagtype}{\color{white} \textbf{\textsf{PARAMETER}}}] \textbf{\underline{superName}} The logical name of the superfile.
\item [\colorbox{tagtype}{\color{white} \textbf{\textsf{PARAMETER}}}] \textbf{\underline{recurse}} Should the contents of child-superfiles be expanded. Default is FALSE.
\item [\colorbox{tagtype}{\color{white} \textbf{\textsf{RETURN}}}] \textbf{\underline{}} A dataset containing the names of the sub-files.
\end{description}

\rule{\linewidth}{0.5pt}
\subsection*{\textsf{\colorbox{headtoc}{\color{white} FUNCTION}
LogicalFileSuperOwners}}

\hypertarget{ecldoc:file.logicalfilesuperowners}{}
\hspace{0pt} \hyperlink{ecldoc:File}{File} \textbackslash 

{\renewcommand{\arraystretch}{1.5}
\begin{tabularx}{\textwidth}{|>{\raggedright\arraybackslash}l|X|}
\hline
\hspace{0pt}\mytexttt{\color{red} dataset(FsLogicalFileNameRecord)} & \textbf{LogicalFileSuperOwners} \\
\hline
\multicolumn{2}{|>{\raggedright\arraybackslash}X|}{\hspace{0pt}\mytexttt{\color{param} (varstring name)}} \\
\hline
\end{tabularx}
}

\par
Returns the list of superfiles that a logical file is contained within.

\par
\begin{description}
\item [\colorbox{tagtype}{\color{white} \textbf{\textsf{PARAMETER}}}] \textbf{\underline{name}} The name of the logical file.
\item [\colorbox{tagtype}{\color{white} \textbf{\textsf{RETURN}}}] \textbf{\underline{}} A dataset containing the names of the superfiles.
\end{description}

\rule{\linewidth}{0.5pt}
\subsection*{\textsf{\colorbox{headtoc}{\color{white} FUNCTION}
LogicalFileSuperSubList}}

\hypertarget{ecldoc:file.logicalfilesupersublist}{}
\hspace{0pt} \hyperlink{ecldoc:File}{File} \textbackslash 

{\renewcommand{\arraystretch}{1.5}
\begin{tabularx}{\textwidth}{|>{\raggedright\arraybackslash}l|X|}
\hline
\hspace{0pt}\mytexttt{\color{red} dataset(FsLogicalSuperSubRecord)} & \textbf{LogicalFileSuperSubList} \\
\hline
\multicolumn{2}{|>{\raggedright\arraybackslash}X|}{\hspace{0pt}\mytexttt{\color{param} ()}} \\
\hline
\end{tabularx}
}

\par
Returns the list of all the superfiles in the system and their component sub-files.

\par
\begin{description}
\item [\colorbox{tagtype}{\color{white} \textbf{\textsf{RETURN}}}] \textbf{\underline{}} A dataset containing pairs of superName,subName for each component file.
\end{description}

\rule{\linewidth}{0.5pt}
\subsection*{\textsf{\colorbox{headtoc}{\color{white} FUNCTION}
fPromoteSuperFileList}}

\hypertarget{ecldoc:file.fpromotesuperfilelist}{}
\hspace{0pt} \hyperlink{ecldoc:File}{File} \textbackslash 

{\renewcommand{\arraystretch}{1.5}
\begin{tabularx}{\textwidth}{|>{\raggedright\arraybackslash}l|X|}
\hline
\hspace{0pt}\mytexttt{\color{red} varstring} & \textbf{fPromoteSuperFileList} \\
\hline
\multicolumn{2}{|>{\raggedright\arraybackslash}X|}{\hspace{0pt}\mytexttt{\color{param} (set of varstring superNames, varstring addHead='', boolean delTail=FALSE, boolean createOnlyOne=FALSE, boolean reverse=FALSE)}} \\
\hline
\end{tabularx}
}

\par
Moves the sub-files from the first entry in the list of superfiles to the next in the list, repeating the process through the list of superfiles.

\par
\begin{description}
\item [\colorbox{tagtype}{\color{white} \textbf{\textsf{PARAMETER}}}] \textbf{\underline{superNames}} A set of the names of the superfiles to act on. Any that do not exist will be created. The contents of each superfile will be moved to the next in the list.
\item [\colorbox{tagtype}{\color{white} \textbf{\textsf{PARAMETER}}}] \textbf{\underline{addHead}} A string containing a comma-delimited list of logical file names to add to the first superfile after the promotion process is complete. Defaults to ''.
\item [\colorbox{tagtype}{\color{white} \textbf{\textsf{PARAMETER}}}] \textbf{\underline{delTail}} Indicates whether to physically delete the contents moved out of the last superfile. The default is FALSE.
\item [\colorbox{tagtype}{\color{white} \textbf{\textsf{PARAMETER}}}] \textbf{\underline{createOnlyOne}} Specifies whether to only create a single superfile (truncate the list at the first non-existent superfile). The default is FALSE.
\item [\colorbox{tagtype}{\color{white} \textbf{\textsf{PARAMETER}}}] \textbf{\underline{reverse}} Reverse the order of processing the superfiles list, effectively 'demoting' instead of 'promoting' the sub-files. The default is FALSE.
\item [\colorbox{tagtype}{\color{white} \textbf{\textsf{RETURN}}}] \textbf{\underline{}} A string containing a comma separated list of the previous sub-file contents of the emptied superfile.
\end{description}

\rule{\linewidth}{0.5pt}
\subsection*{\textsf{\colorbox{headtoc}{\color{white} FUNCTION}
PromoteSuperFileList}}

\hypertarget{ecldoc:file.promotesuperfilelist}{}
\hspace{0pt} \hyperlink{ecldoc:File}{File} \textbackslash 

{\renewcommand{\arraystretch}{1.5}
\begin{tabularx}{\textwidth}{|>{\raggedright\arraybackslash}l|X|}
\hline
\hspace{0pt}\mytexttt{\color{red} } & \textbf{PromoteSuperFileList} \\
\hline
\multicolumn{2}{|>{\raggedright\arraybackslash}X|}{\hspace{0pt}\mytexttt{\color{param} (set of varstring superNames, varstring addHead='', boolean delTail=FALSE, boolean createOnlyOne=FALSE, boolean reverse=FALSE)}} \\
\hline
\end{tabularx}
}

\par
Same as fPromoteSuperFileList, but does not return the DFU Workunit ID.

\par
\begin{description}
\item [\colorbox{tagtype}{\color{white} \textbf{\textsf{SEE}}}] \textbf{\underline{}} fPromoteSuperFileList
\end{description}

\rule{\linewidth}{0.5pt}



\chapter*{math}
\hypertarget{ecldoc:toc:math}{}

\section*{\underline{IMPORTS}}

\section*{\underline{DESCRIPTIONS}}
\subsection*{MODULE : Math}
\hypertarget{ecldoc:Math}{}
\hyperlink{ecldoc:toc:root}{Up} :

{\renewcommand{\arraystretch}{1.5}
\begin{tabularx}{\textwidth}{|>{\raggedright\arraybackslash}l|X|}
\hline
\hspace{0pt} & Math \\
\hline
\end{tabularx}
}

\par


\hyperlink{ecldoc:math.infinity}{Infinity}  |
\hyperlink{ecldoc:math.nan}{NaN}  |
\hyperlink{ecldoc:math.isinfinite}{isInfinite}  |
\hyperlink{ecldoc:math.isnan}{isNaN}  |
\hyperlink{ecldoc:math.isfinite}{isFinite}  |
\hyperlink{ecldoc:math.fmod}{FMod}  |
\hyperlink{ecldoc:math.fmatch}{FMatch}  |

\rule{\linewidth}{0.5pt}

\subsection*{ATTRIBUTE : Infinity}
\hypertarget{ecldoc:math.infinity}{}
\hyperlink{ecldoc:Math}{Up} :
\hspace{0pt} \hyperlink{ecldoc:Math}{Math} \textbackslash 

{\renewcommand{\arraystretch}{1.5}
\begin{tabularx}{\textwidth}{|>{\raggedright\arraybackslash}l|X|}
\hline
\hspace{0pt}REAL8 & Infinity \\
\hline
\end{tabularx}
}

\par
Return a real ''infinity'' value.


\rule{\linewidth}{0.5pt}
\subsection*{ATTRIBUTE : NaN}
\hypertarget{ecldoc:math.nan}{}
\hyperlink{ecldoc:Math}{Up} :
\hspace{0pt} \hyperlink{ecldoc:Math}{Math} \textbackslash 

{\renewcommand{\arraystretch}{1.5}
\begin{tabularx}{\textwidth}{|>{\raggedright\arraybackslash}l|X|}
\hline
\hspace{0pt}REAL8 & NaN \\
\hline
\end{tabularx}
}

\par
Return a non-signalling NaN (Not a Number)value.


\rule{\linewidth}{0.5pt}
\subsection*{FUNCTION : isInfinite}
\hypertarget{ecldoc:math.isinfinite}{}
\hyperlink{ecldoc:Math}{Up} :
\hspace{0pt} \hyperlink{ecldoc:Math}{Math} \textbackslash 

{\renewcommand{\arraystretch}{1.5}
\begin{tabularx}{\textwidth}{|>{\raggedright\arraybackslash}l|X|}
\hline
\hspace{0pt}BOOLEAN & isInfinite \\
\hline
\multicolumn{2}{|>{\raggedright\arraybackslash}X|}{\hspace{0pt}(REAL8 val)} \\
\hline
\end{tabularx}
}

\par
Return whether a real value is infinite (positive or negative).

\par
\begin{description}
\item [\textbf{Parameter}] val ||| The value to test.
\end{description}

\rule{\linewidth}{0.5pt}
\subsection*{FUNCTION : isNaN}
\hypertarget{ecldoc:math.isnan}{}
\hyperlink{ecldoc:Math}{Up} :
\hspace{0pt} \hyperlink{ecldoc:Math}{Math} \textbackslash 

{\renewcommand{\arraystretch}{1.5}
\begin{tabularx}{\textwidth}{|>{\raggedright\arraybackslash}l|X|}
\hline
\hspace{0pt}BOOLEAN & isNaN \\
\hline
\multicolumn{2}{|>{\raggedright\arraybackslash}X|}{\hspace{0pt}(REAL8 val)} \\
\hline
\end{tabularx}
}

\par
Return whether a real value is a NaN (not a number) value.

\par
\begin{description}
\item [\textbf{Parameter}] val ||| The value to test.
\end{description}

\rule{\linewidth}{0.5pt}
\subsection*{FUNCTION : isFinite}
\hypertarget{ecldoc:math.isfinite}{}
\hyperlink{ecldoc:Math}{Up} :
\hspace{0pt} \hyperlink{ecldoc:Math}{Math} \textbackslash 

{\renewcommand{\arraystretch}{1.5}
\begin{tabularx}{\textwidth}{|>{\raggedright\arraybackslash}l|X|}
\hline
\hspace{0pt}BOOLEAN & isFinite \\
\hline
\multicolumn{2}{|>{\raggedright\arraybackslash}X|}{\hspace{0pt}(REAL8 val)} \\
\hline
\end{tabularx}
}

\par
Return whether a real value is a valid value (neither infinite not NaN).

\par
\begin{description}
\item [\textbf{Parameter}] val ||| The value to test.
\end{description}

\rule{\linewidth}{0.5pt}
\subsection*{FUNCTION : FMod}
\hypertarget{ecldoc:math.fmod}{}
\hyperlink{ecldoc:Math}{Up} :
\hspace{0pt} \hyperlink{ecldoc:Math}{Math} \textbackslash 

{\renewcommand{\arraystretch}{1.5}
\begin{tabularx}{\textwidth}{|>{\raggedright\arraybackslash}l|X|}
\hline
\hspace{0pt}REAL8 & FMod \\
\hline
\multicolumn{2}{|>{\raggedright\arraybackslash}X|}{\hspace{0pt}(REAL8 numer, REAL8 denom)} \\
\hline
\end{tabularx}
}

\par
Returns the floating-point remainder of numer/denom (rounded towards zero). If denom is zero, the result depends on the -fdivideByZero flag: 'zero' or unset: return zero. 'nan': return a non-signalling NaN value 'fail': throw an exception

\par
\begin{description}
\item [\textbf{Parameter}] numer ||| The numerator.
\item [\textbf{Parameter}] denom ||| The numerator.
\end{description}

\rule{\linewidth}{0.5pt}
\subsection*{FUNCTION : FMatch}
\hypertarget{ecldoc:math.fmatch}{}
\hyperlink{ecldoc:Math}{Up} :
\hspace{0pt} \hyperlink{ecldoc:Math}{Math} \textbackslash 

{\renewcommand{\arraystretch}{1.5}
\begin{tabularx}{\textwidth}{|>{\raggedright\arraybackslash}l|X|}
\hline
\hspace{0pt}BOOLEAN & FMatch \\
\hline
\multicolumn{2}{|>{\raggedright\arraybackslash}X|}{\hspace{0pt}(REAL8 a, REAL8 b, REAL8 epsilon=0.0)} \\
\hline
\end{tabularx}
}

\par
Returns whether two floating point values are the same, within margin of error epsilon.

\par
\begin{description}
\item [\textbf{Parameter}] a ||| The first value.
\item [\textbf{Parameter}] b ||| The second value.
\item [\textbf{Parameter}] epsilon ||| The allowable margin of error.
\end{description}

\rule{\linewidth}{0.5pt}



\chapter*{Metaphone}
\hypertarget{ecldoc:toc:Metaphone}{}

\section*{\underline{IMPORTS}}
\begin{itemize}
\item lib\_metaphone
\end{itemize}

\section*{\underline{DESCRIPTIONS}}
\subsection*{MODULE : Metaphone}
\hypertarget{ecldoc:Metaphone}{}
\hyperlink{ecldoc:toc:root}{Up} :

{\renewcommand{\arraystretch}{1.5}
\begin{tabularx}{\textwidth}{|>{\raggedright\arraybackslash}l|X|}
\hline
\hspace{0pt} & Metaphone \\
\hline
\end{tabularx}
}

\par


\hyperlink{ecldoc:metaphone.primary}{primary}  |
\hyperlink{ecldoc:metaphone.secondary}{secondary}  |
\hyperlink{ecldoc:metaphone.double}{double}  |

\rule{\linewidth}{0.5pt}

\subsection*{FUNCTION : primary}
\hypertarget{ecldoc:metaphone.primary}{}
\hyperlink{ecldoc:Metaphone}{Up} :
\hspace{0pt} \hyperlink{ecldoc:Metaphone}{Metaphone} \textbackslash 

{\renewcommand{\arraystretch}{1.5}
\begin{tabularx}{\textwidth}{|>{\raggedright\arraybackslash}l|X|}
\hline
\hspace{0pt}String & primary \\
\hline
\multicolumn{2}{|>{\raggedright\arraybackslash}X|}{\hspace{0pt}(STRING src)} \\
\hline
\end{tabularx}
}

\par
Returns the primary metaphone value

\par
\begin{description}
\item [\textbf{Parameter}] src ||| The string whose metphone is to be calculated.
\item [\textbf{See}] http://en.wikipedia.org/wiki/Metaphone\#Double\_Metaphone
\end{description}

\rule{\linewidth}{0.5pt}
\subsection*{FUNCTION : secondary}
\hypertarget{ecldoc:metaphone.secondary}{}
\hyperlink{ecldoc:Metaphone}{Up} :
\hspace{0pt} \hyperlink{ecldoc:Metaphone}{Metaphone} \textbackslash 

{\renewcommand{\arraystretch}{1.5}
\begin{tabularx}{\textwidth}{|>{\raggedright\arraybackslash}l|X|}
\hline
\hspace{0pt}String & secondary \\
\hline
\multicolumn{2}{|>{\raggedright\arraybackslash}X|}{\hspace{0pt}(STRING src)} \\
\hline
\end{tabularx}
}

\par
Returns the secondary metaphone value

\par
\begin{description}
\item [\textbf{Parameter}] src ||| The string whose metphone is to be calculated.
\item [\textbf{See}] http://en.wikipedia.org/wiki/Metaphone\#Double\_Metaphone
\end{description}

\rule{\linewidth}{0.5pt}
\subsection*{FUNCTION : double}
\hypertarget{ecldoc:metaphone.double}{}
\hyperlink{ecldoc:Metaphone}{Up} :
\hspace{0pt} \hyperlink{ecldoc:Metaphone}{Metaphone} \textbackslash 

{\renewcommand{\arraystretch}{1.5}
\begin{tabularx}{\textwidth}{|>{\raggedright\arraybackslash}l|X|}
\hline
\hspace{0pt}String & double \\
\hline
\multicolumn{2}{|>{\raggedright\arraybackslash}X|}{\hspace{0pt}(STRING src)} \\
\hline
\end{tabularx}
}

\par
Returns the double metaphone value (primary and secondary concatenated

\par
\begin{description}
\item [\textbf{Parameter}] src ||| The string whose metphone is to be calculated.
\item [\textbf{See}] http://en.wikipedia.org/wiki/Metaphone\#Double\_Metaphone
\end{description}

\rule{\linewidth}{0.5pt}



\chapter*{str}
\hypertarget{ecldoc:toc:str}{}

\section*{\underline{IMPORTS}}
\begin{itemize}
\item lib\_stringlib
\end{itemize}

\section*{\underline{DESCRIPTIONS}}
\subsection*{MODULE : Str}
\hypertarget{ecldoc:Str}{}
\hyperlink{ecldoc:toc:root}{Up} :

{\renewcommand{\arraystretch}{1.5}
\begin{tabularx}{\textwidth}{|>{\raggedright\arraybackslash}l|X|}
\hline
\hspace{0pt} & Str \\
\hline
\end{tabularx}
}

\par


\hyperlink{ecldoc:str.compareignorecase}{CompareIgnoreCase}  |
\hyperlink{ecldoc:str.equalignorecase}{EqualIgnoreCase}  |
\hyperlink{ecldoc:str.find}{Find}  |
\hyperlink{ecldoc:str.findcount}{FindCount}  |
\hyperlink{ecldoc:str.wildmatch}{WildMatch}  |
\hyperlink{ecldoc:str.contains}{Contains}  |
\hyperlink{ecldoc:str.filterout}{FilterOut}  |
\hyperlink{ecldoc:str.filter}{Filter}  |
\hyperlink{ecldoc:str.substituteincluded}{SubstituteIncluded}  |
\hyperlink{ecldoc:str.substituteexcluded}{SubstituteExcluded}  |
\hyperlink{ecldoc:str.translate}{Translate}  |
\hyperlink{ecldoc:str.tolowercase}{ToLowerCase}  |
\hyperlink{ecldoc:str.touppercase}{ToUpperCase}  |
\hyperlink{ecldoc:str.tocapitalcase}{ToCapitalCase}  |
\hyperlink{ecldoc:str.totitlecase}{ToTitleCase}  |
\hyperlink{ecldoc:str.reverse}{Reverse}  |
\hyperlink{ecldoc:str.findreplace}{FindReplace}  |
\hyperlink{ecldoc:str.extract}{Extract}  |
\hyperlink{ecldoc:str.cleanspaces}{CleanSpaces}  |
\hyperlink{ecldoc:str.startswith}{StartsWith}  |
\hyperlink{ecldoc:str.endswith}{EndsWith}  |
\hyperlink{ecldoc:str.removesuffix}{RemoveSuffix}  |
\hyperlink{ecldoc:str.extractmultiple}{ExtractMultiple}  |
\hyperlink{ecldoc:str.countwords}{CountWords}  |
\hyperlink{ecldoc:str.splitwords}{SplitWords}  |
\hyperlink{ecldoc:str.combinewords}{CombineWords}  |
\hyperlink{ecldoc:str.editdistance}{EditDistance}  |
\hyperlink{ecldoc:str.editdistancewithinradius}{EditDistanceWithinRadius}  |
\hyperlink{ecldoc:str.wordcount}{WordCount}  |
\hyperlink{ecldoc:str.getnthword}{GetNthWord}  |
\hyperlink{ecldoc:str.excludefirstword}{ExcludeFirstWord}  |
\hyperlink{ecldoc:str.excludelastword}{ExcludeLastWord}  |
\hyperlink{ecldoc:str.excludenthword}{ExcludeNthWord}  |
\hyperlink{ecldoc:str.findword}{FindWord}  |
\hyperlink{ecldoc:str.repeat}{Repeat}  |
\hyperlink{ecldoc:str.tohexpairs}{ToHexPairs}  |
\hyperlink{ecldoc:str.fromhexpairs}{FromHexPairs}  |
\hyperlink{ecldoc:str.encodebase64}{EncodeBase64}  |
\hyperlink{ecldoc:str.decodebase64}{DecodeBase64}  |

\rule{\linewidth}{0.5pt}

\subsection*{FUNCTION : CompareIgnoreCase}
\hypertarget{ecldoc:str.compareignorecase}{}
\hyperlink{ecldoc:Str}{Up} :
\hspace{0pt} \hyperlink{ecldoc:Str}{Str} \textbackslash 

{\renewcommand{\arraystretch}{1.5}
\begin{tabularx}{\textwidth}{|>{\raggedright\arraybackslash}l|X|}
\hline
\hspace{0pt}INTEGER4 & CompareIgnoreCase \\
\hline
\multicolumn{2}{|>{\raggedright\arraybackslash}X|}{\hspace{0pt}(STRING src1, STRING src2)} \\
\hline
\end{tabularx}
}

\par
Compares the two strings case insensitively. Returns a negative integer, zero, or a positive integer according to whether the first string is less than, equal to, or greater than the second.

\par
\begin{description}
\item [\textbf{Parameter}] src1 ||| The first string to be compared.
\item [\textbf{Parameter}] src2 ||| The second string to be compared.
\item [\textbf{See}] Str.EqualIgnoreCase
\end{description}

\rule{\linewidth}{0.5pt}
\subsection*{FUNCTION : EqualIgnoreCase}
\hypertarget{ecldoc:str.equalignorecase}{}
\hyperlink{ecldoc:Str}{Up} :
\hspace{0pt} \hyperlink{ecldoc:Str}{Str} \textbackslash 

{\renewcommand{\arraystretch}{1.5}
\begin{tabularx}{\textwidth}{|>{\raggedright\arraybackslash}l|X|}
\hline
\hspace{0pt}BOOLEAN & EqualIgnoreCase \\
\hline
\multicolumn{2}{|>{\raggedright\arraybackslash}X|}{\hspace{0pt}(STRING src1, STRING src2)} \\
\hline
\end{tabularx}
}

\par
Tests whether the two strings are identical ignoring differences in case.

\par
\begin{description}
\item [\textbf{Parameter}] src1 ||| The first string to be compared.
\item [\textbf{Parameter}] src2 ||| The second string to be compared.
\item [\textbf{See}] Str.CompareIgnoreCase
\end{description}

\rule{\linewidth}{0.5pt}
\subsection*{FUNCTION : Find}
\hypertarget{ecldoc:str.find}{}
\hyperlink{ecldoc:Str}{Up} :
\hspace{0pt} \hyperlink{ecldoc:Str}{Str} \textbackslash 

{\renewcommand{\arraystretch}{1.5}
\begin{tabularx}{\textwidth}{|>{\raggedright\arraybackslash}l|X|}
\hline
\hspace{0pt}UNSIGNED4 & Find \\
\hline
\multicolumn{2}{|>{\raggedright\arraybackslash}X|}{\hspace{0pt}(STRING src, STRING sought, UNSIGNED4 instance = 1)} \\
\hline
\end{tabularx}
}

\par
Returns the character position of the nth match of the search string with the first string. If no match is found the attribute returns 0. If an instance is omitted the position of the first instance is returned.

\par
\begin{description}
\item [\textbf{Parameter}] src ||| The string that is searched
\item [\textbf{Parameter}] sought ||| The string being sought.
\item [\textbf{Parameter}] instance ||| Which match instance are we interested in?
\end{description}

\rule{\linewidth}{0.5pt}
\subsection*{FUNCTION : FindCount}
\hypertarget{ecldoc:str.findcount}{}
\hyperlink{ecldoc:Str}{Up} :
\hspace{0pt} \hyperlink{ecldoc:Str}{Str} \textbackslash 

{\renewcommand{\arraystretch}{1.5}
\begin{tabularx}{\textwidth}{|>{\raggedright\arraybackslash}l|X|}
\hline
\hspace{0pt}UNSIGNED4 & FindCount \\
\hline
\multicolumn{2}{|>{\raggedright\arraybackslash}X|}{\hspace{0pt}(STRING src, STRING sought)} \\
\hline
\end{tabularx}
}

\par
Returns the number of occurences of the second string within the first string.

\par
\begin{description}
\item [\textbf{Parameter}] src ||| The string that is searched
\item [\textbf{Parameter}] sought ||| The string being sought.
\end{description}

\rule{\linewidth}{0.5pt}
\subsection*{FUNCTION : WildMatch}
\hypertarget{ecldoc:str.wildmatch}{}
\hyperlink{ecldoc:Str}{Up} :
\hspace{0pt} \hyperlink{ecldoc:Str}{Str} \textbackslash 

{\renewcommand{\arraystretch}{1.5}
\begin{tabularx}{\textwidth}{|>{\raggedright\arraybackslash}l|X|}
\hline
\hspace{0pt}BOOLEAN & WildMatch \\
\hline
\multicolumn{2}{|>{\raggedright\arraybackslash}X|}{\hspace{0pt}(STRING src, STRING \_pattern, BOOLEAN ignore\_case)} \\
\hline
\end{tabularx}
}

\par
Tests if the search string matches the pattern. The pattern can contain wildcards '?' (single character) and '*' (multiple character).

\par
\begin{description}
\item [\textbf{Parameter}] src ||| The string that is being tested.
\item [\textbf{Parameter}] pattern ||| The pattern to match against.
\item [\textbf{Parameter}] ignore\_case ||| Whether to ignore differences in case between characters
\end{description}

\rule{\linewidth}{0.5pt}
\subsection*{FUNCTION : Contains}
\hypertarget{ecldoc:str.contains}{}
\hyperlink{ecldoc:Str}{Up} :
\hspace{0pt} \hyperlink{ecldoc:Str}{Str} \textbackslash 

{\renewcommand{\arraystretch}{1.5}
\begin{tabularx}{\textwidth}{|>{\raggedright\arraybackslash}l|X|}
\hline
\hspace{0pt}BOOLEAN & Contains \\
\hline
\multicolumn{2}{|>{\raggedright\arraybackslash}X|}{\hspace{0pt}(STRING src, STRING \_pattern, BOOLEAN ignore\_case)} \\
\hline
\end{tabularx}
}

\par
Tests if the search string contains each of the characters in the pattern. If the pattern contains duplicate characters those characters will match once for each occurence in the pattern.

\par
\begin{description}
\item [\textbf{Parameter}] src ||| The string that is being tested.
\item [\textbf{Parameter}] pattern ||| The pattern to match against.
\item [\textbf{Parameter}] ignore\_case ||| Whether to ignore differences in case between characters
\end{description}

\rule{\linewidth}{0.5pt}
\subsection*{FUNCTION : FilterOut}
\hypertarget{ecldoc:str.filterout}{}
\hyperlink{ecldoc:Str}{Up} :
\hspace{0pt} \hyperlink{ecldoc:Str}{Str} \textbackslash 

{\renewcommand{\arraystretch}{1.5}
\begin{tabularx}{\textwidth}{|>{\raggedright\arraybackslash}l|X|}
\hline
\hspace{0pt}STRING & FilterOut \\
\hline
\multicolumn{2}{|>{\raggedright\arraybackslash}X|}{\hspace{0pt}(STRING src, STRING filter)} \\
\hline
\end{tabularx}
}

\par
Returns the first string with all characters within the second string removed.

\par
\begin{description}
\item [\textbf{Parameter}] src ||| The string that is being tested.
\item [\textbf{Parameter}] filter ||| The string containing the set of characters to be excluded.
\item [\textbf{See}] Str.Filter
\end{description}

\rule{\linewidth}{0.5pt}
\subsection*{FUNCTION : Filter}
\hypertarget{ecldoc:str.filter}{}
\hyperlink{ecldoc:Str}{Up} :
\hspace{0pt} \hyperlink{ecldoc:Str}{Str} \textbackslash 

{\renewcommand{\arraystretch}{1.5}
\begin{tabularx}{\textwidth}{|>{\raggedright\arraybackslash}l|X|}
\hline
\hspace{0pt}STRING & Filter \\
\hline
\multicolumn{2}{|>{\raggedright\arraybackslash}X|}{\hspace{0pt}(STRING src, STRING filter)} \\
\hline
\end{tabularx}
}

\par
Returns the first string with all characters not within the second string removed.

\par
\begin{description}
\item [\textbf{Parameter}] src ||| The string that is being tested.
\item [\textbf{Parameter}] filter ||| The string containing the set of characters to be included.
\item [\textbf{See}] Str.FilterOut
\end{description}

\rule{\linewidth}{0.5pt}
\subsection*{FUNCTION : SubstituteIncluded}
\hypertarget{ecldoc:str.substituteincluded}{}
\hyperlink{ecldoc:Str}{Up} :
\hspace{0pt} \hyperlink{ecldoc:Str}{Str} \textbackslash 

{\renewcommand{\arraystretch}{1.5}
\begin{tabularx}{\textwidth}{|>{\raggedright\arraybackslash}l|X|}
\hline
\hspace{0pt}STRING & SubstituteIncluded \\
\hline
\multicolumn{2}{|>{\raggedright\arraybackslash}X|}{\hspace{0pt}(STRING src, STRING filter, STRING1 replace\_char)} \\
\hline
\end{tabularx}
}

\par
Returns the source string with the replacement character substituted for all characters included in the filter string. MORE: Should this be a general string substitution?

\par
\begin{description}
\item [\textbf{Parameter}] src ||| The string that is being tested.
\item [\textbf{Parameter}] filter ||| The string containing the set of characters to be included.
\item [\textbf{Parameter}] replace\_char ||| The character to be substituted into the result.
\item [\textbf{See}] Std.Str.Translate, Std.Str.SubstituteExcluded
\end{description}

\rule{\linewidth}{0.5pt}
\subsection*{FUNCTION : SubstituteExcluded}
\hypertarget{ecldoc:str.substituteexcluded}{}
\hyperlink{ecldoc:Str}{Up} :
\hspace{0pt} \hyperlink{ecldoc:Str}{Str} \textbackslash 

{\renewcommand{\arraystretch}{1.5}
\begin{tabularx}{\textwidth}{|>{\raggedright\arraybackslash}l|X|}
\hline
\hspace{0pt}STRING & SubstituteExcluded \\
\hline
\multicolumn{2}{|>{\raggedright\arraybackslash}X|}{\hspace{0pt}(STRING src, STRING filter, STRING1 replace\_char)} \\
\hline
\end{tabularx}
}

\par
Returns the source string with the replacement character substituted for all characters not included in the filter string. MORE: Should this be a general string substitution?

\par
\begin{description}
\item [\textbf{Parameter}] src ||| The string that is being tested.
\item [\textbf{Parameter}] filter ||| The string containing the set of characters to be included.
\item [\textbf{Parameter}] replace\_char ||| The character to be substituted into the result.
\item [\textbf{See}] Std.Str.SubstituteIncluded
\end{description}

\rule{\linewidth}{0.5pt}
\subsection*{FUNCTION : Translate}
\hypertarget{ecldoc:str.translate}{}
\hyperlink{ecldoc:Str}{Up} :
\hspace{0pt} \hyperlink{ecldoc:Str}{Str} \textbackslash 

{\renewcommand{\arraystretch}{1.5}
\begin{tabularx}{\textwidth}{|>{\raggedright\arraybackslash}l|X|}
\hline
\hspace{0pt}STRING & Translate \\
\hline
\multicolumn{2}{|>{\raggedright\arraybackslash}X|}{\hspace{0pt}(STRING src, STRING search, STRING replacement)} \\
\hline
\end{tabularx}
}

\par
Returns the source string with the all characters that match characters in the search string replaced with the character at the corresponding position in the replacement string.

\par
\begin{description}
\item [\textbf{Parameter}] src ||| The string that is being tested.
\item [\textbf{Parameter}] search ||| The string containing the set of characters to be included.
\item [\textbf{Parameter}] replacement ||| The string containing the characters to act as replacements.
\item [\textbf{See}] Std.Str.SubstituteIncluded
\end{description}

\rule{\linewidth}{0.5pt}
\subsection*{FUNCTION : ToLowerCase}
\hypertarget{ecldoc:str.tolowercase}{}
\hyperlink{ecldoc:Str}{Up} :
\hspace{0pt} \hyperlink{ecldoc:Str}{Str} \textbackslash 

{\renewcommand{\arraystretch}{1.5}
\begin{tabularx}{\textwidth}{|>{\raggedright\arraybackslash}l|X|}
\hline
\hspace{0pt}STRING & ToLowerCase \\
\hline
\multicolumn{2}{|>{\raggedright\arraybackslash}X|}{\hspace{0pt}(STRING src)} \\
\hline
\end{tabularx}
}

\par
Returns the argument string with all upper case characters converted to lower case.

\par
\begin{description}
\item [\textbf{Parameter}] src ||| The string that is being converted.
\end{description}

\rule{\linewidth}{0.5pt}
\subsection*{FUNCTION : ToUpperCase}
\hypertarget{ecldoc:str.touppercase}{}
\hyperlink{ecldoc:Str}{Up} :
\hspace{0pt} \hyperlink{ecldoc:Str}{Str} \textbackslash 

{\renewcommand{\arraystretch}{1.5}
\begin{tabularx}{\textwidth}{|>{\raggedright\arraybackslash}l|X|}
\hline
\hspace{0pt}STRING & ToUpperCase \\
\hline
\multicolumn{2}{|>{\raggedright\arraybackslash}X|}{\hspace{0pt}(STRING src)} \\
\hline
\end{tabularx}
}

\par
Return the argument string with all lower case characters converted to upper case.

\par
\begin{description}
\item [\textbf{Parameter}] src ||| The string that is being converted.
\end{description}

\rule{\linewidth}{0.5pt}
\subsection*{FUNCTION : ToCapitalCase}
\hypertarget{ecldoc:str.tocapitalcase}{}
\hyperlink{ecldoc:Str}{Up} :
\hspace{0pt} \hyperlink{ecldoc:Str}{Str} \textbackslash 

{\renewcommand{\arraystretch}{1.5}
\begin{tabularx}{\textwidth}{|>{\raggedright\arraybackslash}l|X|}
\hline
\hspace{0pt}STRING & ToCapitalCase \\
\hline
\multicolumn{2}{|>{\raggedright\arraybackslash}X|}{\hspace{0pt}(STRING src)} \\
\hline
\end{tabularx}
}

\par
Returns the argument string with the first letter of each word in upper case and all other letters left as-is. A contiguous sequence of alphanumeric characters is treated as a word.

\par
\begin{description}
\item [\textbf{Parameter}] src ||| The string that is being converted.
\end{description}

\rule{\linewidth}{0.5pt}
\subsection*{FUNCTION : ToTitleCase}
\hypertarget{ecldoc:str.totitlecase}{}
\hyperlink{ecldoc:Str}{Up} :
\hspace{0pt} \hyperlink{ecldoc:Str}{Str} \textbackslash 

{\renewcommand{\arraystretch}{1.5}
\begin{tabularx}{\textwidth}{|>{\raggedright\arraybackslash}l|X|}
\hline
\hspace{0pt}STRING & ToTitleCase \\
\hline
\multicolumn{2}{|>{\raggedright\arraybackslash}X|}{\hspace{0pt}(STRING src)} \\
\hline
\end{tabularx}
}

\par
Returns the argument string with the first letter of each word in upper case and all other letters lower case. A contiguous sequence of alphanumeric characters is treated as a word.

\par
\begin{description}
\item [\textbf{Parameter}] src ||| The string that is being converted.
\end{description}

\rule{\linewidth}{0.5pt}
\subsection*{FUNCTION : Reverse}
\hypertarget{ecldoc:str.reverse}{}
\hyperlink{ecldoc:Str}{Up} :
\hspace{0pt} \hyperlink{ecldoc:Str}{Str} \textbackslash 

{\renewcommand{\arraystretch}{1.5}
\begin{tabularx}{\textwidth}{|>{\raggedright\arraybackslash}l|X|}
\hline
\hspace{0pt}STRING & Reverse \\
\hline
\multicolumn{2}{|>{\raggedright\arraybackslash}X|}{\hspace{0pt}(STRING src)} \\
\hline
\end{tabularx}
}

\par
Returns the argument string with all characters in reverse order. Note the argument is not TRIMMED before it is reversed.

\par
\begin{description}
\item [\textbf{Parameter}] src ||| The string that is being reversed.
\end{description}

\rule{\linewidth}{0.5pt}
\subsection*{FUNCTION : FindReplace}
\hypertarget{ecldoc:str.findreplace}{}
\hyperlink{ecldoc:Str}{Up} :
\hspace{0pt} \hyperlink{ecldoc:Str}{Str} \textbackslash 

{\renewcommand{\arraystretch}{1.5}
\begin{tabularx}{\textwidth}{|>{\raggedright\arraybackslash}l|X|}
\hline
\hspace{0pt}STRING & FindReplace \\
\hline
\multicolumn{2}{|>{\raggedright\arraybackslash}X|}{\hspace{0pt}(STRING src, STRING sought, STRING replacement)} \\
\hline
\end{tabularx}
}

\par
Returns the source string with the replacement string substituted for all instances of the search string.

\par
\begin{description}
\item [\textbf{Parameter}] src ||| The string that is being transformed.
\item [\textbf{Parameter}] sought ||| The string to be replaced.
\item [\textbf{Parameter}] replacement ||| The string to be substituted into the result.
\end{description}

\rule{\linewidth}{0.5pt}
\subsection*{FUNCTION : Extract}
\hypertarget{ecldoc:str.extract}{}
\hyperlink{ecldoc:Str}{Up} :
\hspace{0pt} \hyperlink{ecldoc:Str}{Str} \textbackslash 

{\renewcommand{\arraystretch}{1.5}
\begin{tabularx}{\textwidth}{|>{\raggedright\arraybackslash}l|X|}
\hline
\hspace{0pt}STRING & Extract \\
\hline
\multicolumn{2}{|>{\raggedright\arraybackslash}X|}{\hspace{0pt}(STRING src, UNSIGNED4 instance)} \\
\hline
\end{tabularx}
}

\par
Returns the nth element from a comma separated string.

\par
\begin{description}
\item [\textbf{Parameter}] src ||| The string containing the comma separated list.
\item [\textbf{Parameter}] instance ||| Which item to select from the list.
\end{description}

\rule{\linewidth}{0.5pt}
\subsection*{FUNCTION : CleanSpaces}
\hypertarget{ecldoc:str.cleanspaces}{}
\hyperlink{ecldoc:Str}{Up} :
\hspace{0pt} \hyperlink{ecldoc:Str}{Str} \textbackslash 

{\renewcommand{\arraystretch}{1.5}
\begin{tabularx}{\textwidth}{|>{\raggedright\arraybackslash}l|X|}
\hline
\hspace{0pt}STRING & CleanSpaces \\
\hline
\multicolumn{2}{|>{\raggedright\arraybackslash}X|}{\hspace{0pt}(STRING src)} \\
\hline
\end{tabularx}
}

\par
Returns the source string with all instances of multiple adjacent space characters (2 or more spaces together) reduced to a single space character. Leading and trailing spaces are removed, and tab characters are converted to spaces.

\par
\begin{description}
\item [\textbf{Parameter}] src ||| The string to be cleaned.
\end{description}

\rule{\linewidth}{0.5pt}
\subsection*{FUNCTION : StartsWith}
\hypertarget{ecldoc:str.startswith}{}
\hyperlink{ecldoc:Str}{Up} :
\hspace{0pt} \hyperlink{ecldoc:Str}{Str} \textbackslash 

{\renewcommand{\arraystretch}{1.5}
\begin{tabularx}{\textwidth}{|>{\raggedright\arraybackslash}l|X|}
\hline
\hspace{0pt}BOOLEAN & StartsWith \\
\hline
\multicolumn{2}{|>{\raggedright\arraybackslash}X|}{\hspace{0pt}(STRING src, STRING prefix)} \\
\hline
\end{tabularx}
}

\par
Returns true if the prefix string matches the leading characters in the source string. Trailing spaces are stripped from the prefix before matching. // x.myString.StartsWith('x') as an alternative syntax would be even better

\par
\begin{description}
\item [\textbf{Parameter}] src ||| The string being searched in.
\item [\textbf{Parameter}] prefix ||| The prefix to search for.
\end{description}

\rule{\linewidth}{0.5pt}
\subsection*{FUNCTION : EndsWith}
\hypertarget{ecldoc:str.endswith}{}
\hyperlink{ecldoc:Str}{Up} :
\hspace{0pt} \hyperlink{ecldoc:Str}{Str} \textbackslash 

{\renewcommand{\arraystretch}{1.5}
\begin{tabularx}{\textwidth}{|>{\raggedright\arraybackslash}l|X|}
\hline
\hspace{0pt}BOOLEAN & EndsWith \\
\hline
\multicolumn{2}{|>{\raggedright\arraybackslash}X|}{\hspace{0pt}(STRING src, STRING suffix)} \\
\hline
\end{tabularx}
}

\par
Returns true if the suffix string matches the trailing characters in the source string. Trailing spaces are stripped from both strings before matching.

\par
\begin{description}
\item [\textbf{Parameter}] src ||| The string being searched in.
\item [\textbf{Parameter}] suffix ||| The prefix to search for.
\end{description}

\rule{\linewidth}{0.5pt}
\subsection*{FUNCTION : RemoveSuffix}
\hypertarget{ecldoc:str.removesuffix}{}
\hyperlink{ecldoc:Str}{Up} :
\hspace{0pt} \hyperlink{ecldoc:Str}{Str} \textbackslash 

{\renewcommand{\arraystretch}{1.5}
\begin{tabularx}{\textwidth}{|>{\raggedright\arraybackslash}l|X|}
\hline
\hspace{0pt}STRING & RemoveSuffix \\
\hline
\multicolumn{2}{|>{\raggedright\arraybackslash}X|}{\hspace{0pt}(STRING src, STRING suffix)} \\
\hline
\end{tabularx}
}

\par
Removes the suffix from the search string, if present, and returns the result. Trailing spaces are stripped from both strings before matching.

\par
\begin{description}
\item [\textbf{Parameter}] src ||| The string being searched in.
\item [\textbf{Parameter}] suffix ||| The prefix to search for.
\end{description}

\rule{\linewidth}{0.5pt}
\subsection*{FUNCTION : ExtractMultiple}
\hypertarget{ecldoc:str.extractmultiple}{}
\hyperlink{ecldoc:Str}{Up} :
\hspace{0pt} \hyperlink{ecldoc:Str}{Str} \textbackslash 

{\renewcommand{\arraystretch}{1.5}
\begin{tabularx}{\textwidth}{|>{\raggedright\arraybackslash}l|X|}
\hline
\hspace{0pt}STRING & ExtractMultiple \\
\hline
\multicolumn{2}{|>{\raggedright\arraybackslash}X|}{\hspace{0pt}(STRING src, UNSIGNED8 mask)} \\
\hline
\end{tabularx}
}

\par
Returns a string containing a list of elements from a comma separated string.

\par
\begin{description}
\item [\textbf{Parameter}] src ||| The string containing the comma separated list.
\item [\textbf{Parameter}] mask ||| A bitmask of which elements should be included. Bit 0 is item1, bit1 item 2 etc.
\end{description}

\rule{\linewidth}{0.5pt}
\subsection*{FUNCTION : CountWords}
\hypertarget{ecldoc:str.countwords}{}
\hyperlink{ecldoc:Str}{Up} :
\hspace{0pt} \hyperlink{ecldoc:Str}{Str} \textbackslash 

{\renewcommand{\arraystretch}{1.5}
\begin{tabularx}{\textwidth}{|>{\raggedright\arraybackslash}l|X|}
\hline
\hspace{0pt}UNSIGNED4 & CountWords \\
\hline
\multicolumn{2}{|>{\raggedright\arraybackslash}X|}{\hspace{0pt}(STRING src, STRING separator, BOOLEAN allow\_blank = FALSE)} \\
\hline
\end{tabularx}
}

\par
Returns the number of words that the string contains. Words are separated by one or more separator strings. No spaces are stripped from either string before matching.

\par
\begin{description}
\item [\textbf{Parameter}] src ||| The string being searched in.
\item [\textbf{Parameter}] separator ||| The string used to separate words
\item [\textbf{Parameter}] allow\_blank ||| Indicates if empty/blank string items are included in the results.
\end{description}

\rule{\linewidth}{0.5pt}
\subsection*{FUNCTION : SplitWords}
\hypertarget{ecldoc:str.splitwords}{}
\hyperlink{ecldoc:Str}{Up} :
\hspace{0pt} \hyperlink{ecldoc:Str}{Str} \textbackslash 

{\renewcommand{\arraystretch}{1.5}
\begin{tabularx}{\textwidth}{|>{\raggedright\arraybackslash}l|X|}
\hline
\hspace{0pt}SET OF STRING & SplitWords \\
\hline
\multicolumn{2}{|>{\raggedright\arraybackslash}X|}{\hspace{0pt}(STRING src, STRING separator, BOOLEAN allow\_blank = FALSE)} \\
\hline
\end{tabularx}
}

\par
Returns the list of words extracted from the string. Words are separated by one or more separator strings. No spaces are stripped from either string before matching.

\par
\begin{description}
\item [\textbf{Parameter}] src ||| The string being searched in.
\item [\textbf{Parameter}] separator ||| The string used to separate words
\item [\textbf{Parameter}] allow\_blank ||| Indicates if empty/blank string items are included in the results.
\end{description}

\rule{\linewidth}{0.5pt}
\subsection*{FUNCTION : CombineWords}
\hypertarget{ecldoc:str.combinewords}{}
\hyperlink{ecldoc:Str}{Up} :
\hspace{0pt} \hyperlink{ecldoc:Str}{Str} \textbackslash 

{\renewcommand{\arraystretch}{1.5}
\begin{tabularx}{\textwidth}{|>{\raggedright\arraybackslash}l|X|}
\hline
\hspace{0pt}STRING & CombineWords \\
\hline
\multicolumn{2}{|>{\raggedright\arraybackslash}X|}{\hspace{0pt}(SET OF STRING words, STRING separator)} \\
\hline
\end{tabularx}
}

\par
Returns the list of words extracted from the string. Words are separated by one or more separator strings. No spaces are stripped from either string before matching.

\par
\begin{description}
\item [\textbf{Parameter}] words ||| The set of strings to be combined.
\item [\textbf{Parameter}] separator ||| The string used to separate words.
\end{description}

\rule{\linewidth}{0.5pt}
\subsection*{FUNCTION : EditDistance}
\hypertarget{ecldoc:str.editdistance}{}
\hyperlink{ecldoc:Str}{Up} :
\hspace{0pt} \hyperlink{ecldoc:Str}{Str} \textbackslash 

{\renewcommand{\arraystretch}{1.5}
\begin{tabularx}{\textwidth}{|>{\raggedright\arraybackslash}l|X|}
\hline
\hspace{0pt}UNSIGNED4 & EditDistance \\
\hline
\multicolumn{2}{|>{\raggedright\arraybackslash}X|}{\hspace{0pt}(STRING \_left, STRING \_right)} \\
\hline
\end{tabularx}
}

\par
Returns the minimum edit distance between the two strings. An insert change or delete counts as a single edit. The two strings are trimmed before comparing.

\par
\begin{description}
\item [\textbf{Parameter}] \_left ||| The first string to be compared.
\item [\textbf{Parameter}] \_right ||| The second string to be compared.
\item [\textbf{Return}] The minimum edit distance between the two strings.
\end{description}

\rule{\linewidth}{0.5pt}
\subsection*{FUNCTION : EditDistanceWithinRadius}
\hypertarget{ecldoc:str.editdistancewithinradius}{}
\hyperlink{ecldoc:Str}{Up} :
\hspace{0pt} \hyperlink{ecldoc:Str}{Str} \textbackslash 

{\renewcommand{\arraystretch}{1.5}
\begin{tabularx}{\textwidth}{|>{\raggedright\arraybackslash}l|X|}
\hline
\hspace{0pt}BOOLEAN & EditDistanceWithinRadius \\
\hline
\multicolumn{2}{|>{\raggedright\arraybackslash}X|}{\hspace{0pt}(STRING \_left, STRING \_right, UNSIGNED4 radius)} \\
\hline
\end{tabularx}
}

\par
Returns true if the minimum edit distance between the two strings is with a specific range. The two strings are trimmed before comparing.

\par
\begin{description}
\item [\textbf{Parameter}] \_left ||| The first string to be compared.
\item [\textbf{Parameter}] \_right ||| The second string to be compared.
\item [\textbf{Parameter}] radius ||| The maximum edit distance that is accepable.
\item [\textbf{Return}] Whether or not the two strings are within the given specified edit distance.
\end{description}

\rule{\linewidth}{0.5pt}
\subsection*{FUNCTION : WordCount}
\hypertarget{ecldoc:str.wordcount}{}
\hyperlink{ecldoc:Str}{Up} :
\hspace{0pt} \hyperlink{ecldoc:Str}{Str} \textbackslash 

{\renewcommand{\arraystretch}{1.5}
\begin{tabularx}{\textwidth}{|>{\raggedright\arraybackslash}l|X|}
\hline
\hspace{0pt}UNSIGNED4 & WordCount \\
\hline
\multicolumn{2}{|>{\raggedright\arraybackslash}X|}{\hspace{0pt}(STRING text)} \\
\hline
\end{tabularx}
}

\par
Returns the number of words in the string. Words are separated by one or more spaces.

\par
\begin{description}
\item [\textbf{Parameter}] text ||| The string to be broken into words.
\item [\textbf{Return}] The number of words in the string.
\end{description}

\rule{\linewidth}{0.5pt}
\subsection*{FUNCTION : GetNthWord}
\hypertarget{ecldoc:str.getnthword}{}
\hyperlink{ecldoc:Str}{Up} :
\hspace{0pt} \hyperlink{ecldoc:Str}{Str} \textbackslash 

{\renewcommand{\arraystretch}{1.5}
\begin{tabularx}{\textwidth}{|>{\raggedright\arraybackslash}l|X|}
\hline
\hspace{0pt}STRING & GetNthWord \\
\hline
\multicolumn{2}{|>{\raggedright\arraybackslash}X|}{\hspace{0pt}(STRING text, UNSIGNED4 n)} \\
\hline
\end{tabularx}
}

\par
Returns the n-th word from the string. Words are separated by one or more spaces.

\par
\begin{description}
\item [\textbf{Parameter}] text ||| The string to be broken into words.
\item [\textbf{Parameter}] n ||| Which word should be returned from the function.
\item [\textbf{Return}] The number of words in the string.
\end{description}

\rule{\linewidth}{0.5pt}
\subsection*{FUNCTION : ExcludeFirstWord}
\hypertarget{ecldoc:str.excludefirstword}{}
\hyperlink{ecldoc:Str}{Up} :
\hspace{0pt} \hyperlink{ecldoc:Str}{Str} \textbackslash 

{\renewcommand{\arraystretch}{1.5}
\begin{tabularx}{\textwidth}{|>{\raggedright\arraybackslash}l|X|}
\hline
\hspace{0pt} & ExcludeFirstWord \\
\hline
\multicolumn{2}{|>{\raggedright\arraybackslash}X|}{\hspace{0pt}(STRING text)} \\
\hline
\end{tabularx}
}

\par
Returns everything except the first word from the string. Words are separated by one or more whitespace characters. Whitespace before and after the first word is also removed.

\par
\begin{description}
\item [\textbf{Parameter}] text ||| The string to be broken into words.
\item [\textbf{Return}] The string excluding the first word.
\end{description}

\rule{\linewidth}{0.5pt}
\subsection*{FUNCTION : ExcludeLastWord}
\hypertarget{ecldoc:str.excludelastword}{}
\hyperlink{ecldoc:Str}{Up} :
\hspace{0pt} \hyperlink{ecldoc:Str}{Str} \textbackslash 

{\renewcommand{\arraystretch}{1.5}
\begin{tabularx}{\textwidth}{|>{\raggedright\arraybackslash}l|X|}
\hline
\hspace{0pt} & ExcludeLastWord \\
\hline
\multicolumn{2}{|>{\raggedright\arraybackslash}X|}{\hspace{0pt}(STRING text)} \\
\hline
\end{tabularx}
}

\par
Returns everything except the last word from the string. Words are separated by one or more whitespace characters. Whitespace after a word is removed with the word and leading whitespace is removed with the first word.

\par
\begin{description}
\item [\textbf{Parameter}] text ||| The string to be broken into words.
\item [\textbf{Return}] The string excluding the last word.
\end{description}

\rule{\linewidth}{0.5pt}
\subsection*{FUNCTION : ExcludeNthWord}
\hypertarget{ecldoc:str.excludenthword}{}
\hyperlink{ecldoc:Str}{Up} :
\hspace{0pt} \hyperlink{ecldoc:Str}{Str} \textbackslash 

{\renewcommand{\arraystretch}{1.5}
\begin{tabularx}{\textwidth}{|>{\raggedright\arraybackslash}l|X|}
\hline
\hspace{0pt} & ExcludeNthWord \\
\hline
\multicolumn{2}{|>{\raggedright\arraybackslash}X|}{\hspace{0pt}(STRING text, UNSIGNED2 n)} \\
\hline
\end{tabularx}
}

\par
Returns everything except the nth word from the string. Words are separated by one or more whitespace characters. Whitespace after a word is removed with the word and leading whitespace is removed with the first word.

\par
\begin{description}
\item [\textbf{Parameter}] text ||| The string to be broken into words.
\item [\textbf{Parameter}] n ||| Which word should be returned from the function.
\item [\textbf{Return}] The string excluding the nth word.
\end{description}

\rule{\linewidth}{0.5pt}
\subsection*{FUNCTION : FindWord}
\hypertarget{ecldoc:str.findword}{}
\hyperlink{ecldoc:Str}{Up} :
\hspace{0pt} \hyperlink{ecldoc:Str}{Str} \textbackslash 

{\renewcommand{\arraystretch}{1.5}
\begin{tabularx}{\textwidth}{|>{\raggedright\arraybackslash}l|X|}
\hline
\hspace{0pt}BOOLEAN & FindWord \\
\hline
\multicolumn{2}{|>{\raggedright\arraybackslash}X|}{\hspace{0pt}(STRING src, STRING word, BOOLEAN ignore\_case=FALSE)} \\
\hline
\end{tabularx}
}

\par
Tests if the search string contains the supplied word as a whole word.

\par
\begin{description}
\item [\textbf{Parameter}] src ||| The string that is being tested.
\item [\textbf{Parameter}] word ||| The word to be searched for.
\item [\textbf{Parameter}] ignore\_case ||| Whether to ignore differences in case between characters.
\end{description}

\rule{\linewidth}{0.5pt}
\subsection*{FUNCTION : Repeat}
\hypertarget{ecldoc:str.repeat}{}
\hyperlink{ecldoc:Str}{Up} :
\hspace{0pt} \hyperlink{ecldoc:Str}{Str} \textbackslash 

{\renewcommand{\arraystretch}{1.5}
\begin{tabularx}{\textwidth}{|>{\raggedright\arraybackslash}l|X|}
\hline
\hspace{0pt}STRING & Repeat \\
\hline
\multicolumn{2}{|>{\raggedright\arraybackslash}X|}{\hspace{0pt}(STRING text, UNSIGNED4 n)} \\
\hline
\end{tabularx}
}

\par


\rule{\linewidth}{0.5pt}
\subsection*{FUNCTION : ToHexPairs}
\hypertarget{ecldoc:str.tohexpairs}{}
\hyperlink{ecldoc:Str}{Up} :
\hspace{0pt} \hyperlink{ecldoc:Str}{Str} \textbackslash 

{\renewcommand{\arraystretch}{1.5}
\begin{tabularx}{\textwidth}{|>{\raggedright\arraybackslash}l|X|}
\hline
\hspace{0pt}STRING & ToHexPairs \\
\hline
\multicolumn{2}{|>{\raggedright\arraybackslash}X|}{\hspace{0pt}(DATA value)} \\
\hline
\end{tabularx}
}

\par


\rule{\linewidth}{0.5pt}
\subsection*{FUNCTION : FromHexPairs}
\hypertarget{ecldoc:str.fromhexpairs}{}
\hyperlink{ecldoc:Str}{Up} :
\hspace{0pt} \hyperlink{ecldoc:Str}{Str} \textbackslash 

{\renewcommand{\arraystretch}{1.5}
\begin{tabularx}{\textwidth}{|>{\raggedright\arraybackslash}l|X|}
\hline
\hspace{0pt}DATA & FromHexPairs \\
\hline
\multicolumn{2}{|>{\raggedright\arraybackslash}X|}{\hspace{0pt}(STRING hex\_pairs)} \\
\hline
\end{tabularx}
}

\par


\rule{\linewidth}{0.5pt}
\subsection*{FUNCTION : EncodeBase64}
\hypertarget{ecldoc:str.encodebase64}{}
\hyperlink{ecldoc:Str}{Up} :
\hspace{0pt} \hyperlink{ecldoc:Str}{Str} \textbackslash 

{\renewcommand{\arraystretch}{1.5}
\begin{tabularx}{\textwidth}{|>{\raggedright\arraybackslash}l|X|}
\hline
\hspace{0pt}STRING & EncodeBase64 \\
\hline
\multicolumn{2}{|>{\raggedright\arraybackslash}X|}{\hspace{0pt}(DATA value)} \\
\hline
\end{tabularx}
}

\par


\rule{\linewidth}{0.5pt}
\subsection*{FUNCTION : DecodeBase64}
\hypertarget{ecldoc:str.decodebase64}{}
\hyperlink{ecldoc:Str}{Up} :
\hspace{0pt} \hyperlink{ecldoc:Str}{Str} \textbackslash 

{\renewcommand{\arraystretch}{1.5}
\begin{tabularx}{\textwidth}{|>{\raggedright\arraybackslash}l|X|}
\hline
\hspace{0pt}DATA & DecodeBase64 \\
\hline
\multicolumn{2}{|>{\raggedright\arraybackslash}X|}{\hspace{0pt}(STRING value)} \\
\hline
\end{tabularx}
}

\par


\rule{\linewidth}{0.5pt}



\chapter*{\color{headfile}
Uni
}
\hypertarget{ecldoc:toc:Uni}{}
\hyperlink{ecldoc:toc:root}{Go Up}

\section*{\underline{\textsf{IMPORTS}}}
\begin{doublespace}
{\large
lib\_unicodelib |
}
\end{doublespace}

\section*{\underline{\textsf{DESCRIPTIONS}}}
\subsection*{\textsf{\colorbox{headtoc}{\color{white} MODULE}
Uni}}

\hypertarget{ecldoc:Uni}{}

{\renewcommand{\arraystretch}{1.5}
\begin{tabularx}{\textwidth}{|>{\raggedright\arraybackslash}l|X|}
\hline
\hspace{0pt}\mytexttt{\color{red} } & \textbf{Uni} \\
\hline
\end{tabularx}
}

\par





No Documentation Found







\textbf{Children}
\begin{enumerate}
\item \hyperlink{ecldoc:uni.filterout}{FilterOut}
: Returns the first string with all characters within the second string removed
\item \hyperlink{ecldoc:uni.filter}{Filter}
: Returns the first string with all characters not within the second string removed
\item \hyperlink{ecldoc:uni.substituteincluded}{SubstituteIncluded}
: Returns the source string with the replacement character substituted for all characters included in the filter string
\item \hyperlink{ecldoc:uni.substituteexcluded}{SubstituteExcluded}
: Returns the source string with the replacement character substituted for all characters not included in the filter string
\item \hyperlink{ecldoc:uni.find}{Find}
: Returns the character position of the nth match of the search string with the first string
\item \hyperlink{ecldoc:uni.findword}{FindWord}
: Tests if the search string contains the supplied word as a whole word
\item \hyperlink{ecldoc:uni.localefind}{LocaleFind}
: Returns the character position of the nth match of the search string with the first string
\item \hyperlink{ecldoc:uni.localefindatstrength}{LocaleFindAtStrength}
: Returns the character position of the nth match of the search string with the first string
\item \hyperlink{ecldoc:uni.extract}{Extract}
: Returns the nth element from a comma separated string
\item \hyperlink{ecldoc:uni.tolowercase}{ToLowerCase}
: Returns the argument string with all upper case characters converted to lower case
\item \hyperlink{ecldoc:uni.touppercase}{ToUpperCase}
: Return the argument string with all lower case characters converted to upper case
\item \hyperlink{ecldoc:uni.totitlecase}{ToTitleCase}
: Returns the upper case variant of the string using the rules for a particular locale
\item \hyperlink{ecldoc:uni.localetolowercase}{LocaleToLowerCase}
: Returns the lower case variant of the string using the rules for a particular locale
\item \hyperlink{ecldoc:uni.localetouppercase}{LocaleToUpperCase}
: Returns the upper case variant of the string using the rules for a particular locale
\item \hyperlink{ecldoc:uni.localetotitlecase}{LocaleToTitleCase}
: Returns the upper case variant of the string using the rules for a particular locale
\item \hyperlink{ecldoc:uni.compareignorecase}{CompareIgnoreCase}
: Compares the two strings case insensitively
\item \hyperlink{ecldoc:uni.compareatstrength}{CompareAtStrength}
: Compares the two strings case insensitively
\item \hyperlink{ecldoc:uni.localecompareignorecase}{LocaleCompareIgnoreCase}
: Compares the two strings case insensitively
\item \hyperlink{ecldoc:uni.localecompareatstrength}{LocaleCompareAtStrength}
: Compares the two strings case insensitively
\item \hyperlink{ecldoc:uni.reverse}{Reverse}
: Returns the argument string with all characters in reverse order
\item \hyperlink{ecldoc:uni.findreplace}{FindReplace}
: Returns the source string with the replacement string substituted for all instances of the search string
\item \hyperlink{ecldoc:uni.localefindreplace}{LocaleFindReplace}
: Returns the source string with the replacement string substituted for all instances of the search string
\item \hyperlink{ecldoc:uni.localefindatstrengthreplace}{LocaleFindAtStrengthReplace}
: Returns the source string with the replacement string substituted for all instances of the search string
\item \hyperlink{ecldoc:uni.cleanaccents}{CleanAccents}
: Returns the source string with all accented characters replaced with unaccented
\item \hyperlink{ecldoc:uni.cleanspaces}{CleanSpaces}
: Returns the source string with all instances of multiple adjacent space characters (2 or more spaces together) reduced to a single space character
\item \hyperlink{ecldoc:uni.wildmatch}{WildMatch}
: Tests if the search string matches the pattern
\item \hyperlink{ecldoc:uni.contains}{Contains}
: Tests if the search string contains each of the characters in the pattern
\item \hyperlink{ecldoc:uni.editdistance}{EditDistance}
: Returns the minimum edit distance between the two strings
\item \hyperlink{ecldoc:uni.editdistancewithinradius}{EditDistanceWithinRadius}
: Returns true if the minimum edit distance between the two strings is with a specific range
\item \hyperlink{ecldoc:uni.wordcount}{WordCount}
: Returns the number of words in the string
\item \hyperlink{ecldoc:uni.getnthword}{GetNthWord}
: Returns the n-th word from the string
\end{enumerate}

\rule{\linewidth}{0.5pt}

\subsection*{\textsf{\colorbox{headtoc}{\color{white} FUNCTION}
FilterOut}}

\hypertarget{ecldoc:uni.filterout}{}
\hspace{0pt} \hyperlink{ecldoc:Uni}{Uni} \textbackslash 

{\renewcommand{\arraystretch}{1.5}
\begin{tabularx}{\textwidth}{|>{\raggedright\arraybackslash}l|X|}
\hline
\hspace{0pt}\mytexttt{\color{red} unicode} & \textbf{FilterOut} \\
\hline
\multicolumn{2}{|>{\raggedright\arraybackslash}X|}{\hspace{0pt}\mytexttt{\color{param} (unicode src, unicode filter)}} \\
\hline
\end{tabularx}
}

\par





Returns the first string with all characters within the second string removed.






\par
\begin{description}
\item [\colorbox{tagtype}{\color{white} \textbf{\textsf{PARAMETER}}}] \textbf{\underline{src}} ||| UNICODE --- The string that is being tested.
\item [\colorbox{tagtype}{\color{white} \textbf{\textsf{PARAMETER}}}] \textbf{\underline{filter}} ||| UNICODE --- The string containing the set of characters to be excluded.
\end{description}







\par
\begin{description}
\item [\colorbox{tagtype}{\color{white} \textbf{\textsf{RETURN}}}] \textbf{UNICODE} --- 
\end{description}






\par
\begin{description}
\item [\colorbox{tagtype}{\color{white} \textbf{\textsf{SEE}}}] Std.Uni.Filter
\end{description}




\rule{\linewidth}{0.5pt}
\subsection*{\textsf{\colorbox{headtoc}{\color{white} FUNCTION}
Filter}}

\hypertarget{ecldoc:uni.filter}{}
\hspace{0pt} \hyperlink{ecldoc:Uni}{Uni} \textbackslash 

{\renewcommand{\arraystretch}{1.5}
\begin{tabularx}{\textwidth}{|>{\raggedright\arraybackslash}l|X|}
\hline
\hspace{0pt}\mytexttt{\color{red} unicode} & \textbf{Filter} \\
\hline
\multicolumn{2}{|>{\raggedright\arraybackslash}X|}{\hspace{0pt}\mytexttt{\color{param} (unicode src, unicode filter)}} \\
\hline
\end{tabularx}
}

\par





Returns the first string with all characters not within the second string removed.






\par
\begin{description}
\item [\colorbox{tagtype}{\color{white} \textbf{\textsf{PARAMETER}}}] \textbf{\underline{src}} ||| UNICODE --- The string that is being tested.
\item [\colorbox{tagtype}{\color{white} \textbf{\textsf{PARAMETER}}}] \textbf{\underline{filter}} ||| UNICODE --- The string containing the set of characters to be included.
\end{description}







\par
\begin{description}
\item [\colorbox{tagtype}{\color{white} \textbf{\textsf{RETURN}}}] \textbf{UNICODE} --- 
\end{description}






\par
\begin{description}
\item [\colorbox{tagtype}{\color{white} \textbf{\textsf{SEE}}}] Std.Uni.FilterOut
\end{description}




\rule{\linewidth}{0.5pt}
\subsection*{\textsf{\colorbox{headtoc}{\color{white} FUNCTION}
SubstituteIncluded}}

\hypertarget{ecldoc:uni.substituteincluded}{}
\hspace{0pt} \hyperlink{ecldoc:Uni}{Uni} \textbackslash 

{\renewcommand{\arraystretch}{1.5}
\begin{tabularx}{\textwidth}{|>{\raggedright\arraybackslash}l|X|}
\hline
\hspace{0pt}\mytexttt{\color{red} unicode} & \textbf{SubstituteIncluded} \\
\hline
\multicolumn{2}{|>{\raggedright\arraybackslash}X|}{\hspace{0pt}\mytexttt{\color{param} (unicode src, unicode filter, unicode replace\_char)}} \\
\hline
\end{tabularx}
}

\par





Returns the source string with the replacement character substituted for all characters included in the filter string. MORE: Should this be a general string substitution?






\par
\begin{description}
\item [\colorbox{tagtype}{\color{white} \textbf{\textsf{PARAMETER}}}] \textbf{\underline{replace\_char}} ||| UNICODE --- The character to be substituted into the result.
\item [\colorbox{tagtype}{\color{white} \textbf{\textsf{PARAMETER}}}] \textbf{\underline{src}} ||| UNICODE --- The string that is being tested.
\item [\colorbox{tagtype}{\color{white} \textbf{\textsf{PARAMETER}}}] \textbf{\underline{filter}} ||| UNICODE --- The string containing the set of characters to be included.
\end{description}







\par
\begin{description}
\item [\colorbox{tagtype}{\color{white} \textbf{\textsf{RETURN}}}] \textbf{UNICODE} --- 
\end{description}






\par
\begin{description}
\item [\colorbox{tagtype}{\color{white} \textbf{\textsf{SEE}}}] Std.Uni.SubstituteOut
\end{description}




\rule{\linewidth}{0.5pt}
\subsection*{\textsf{\colorbox{headtoc}{\color{white} FUNCTION}
SubstituteExcluded}}

\hypertarget{ecldoc:uni.substituteexcluded}{}
\hspace{0pt} \hyperlink{ecldoc:Uni}{Uni} \textbackslash 

{\renewcommand{\arraystretch}{1.5}
\begin{tabularx}{\textwidth}{|>{\raggedright\arraybackslash}l|X|}
\hline
\hspace{0pt}\mytexttt{\color{red} unicode} & \textbf{SubstituteExcluded} \\
\hline
\multicolumn{2}{|>{\raggedright\arraybackslash}X|}{\hspace{0pt}\mytexttt{\color{param} (unicode src, unicode filter, unicode replace\_char)}} \\
\hline
\end{tabularx}
}

\par





Returns the source string with the replacement character substituted for all characters not included in the filter string. MORE: Should this be a general string substitution?






\par
\begin{description}
\item [\colorbox{tagtype}{\color{white} \textbf{\textsf{PARAMETER}}}] \textbf{\underline{replace\_char}} ||| UNICODE --- The character to be substituted into the result.
\item [\colorbox{tagtype}{\color{white} \textbf{\textsf{PARAMETER}}}] \textbf{\underline{src}} ||| UNICODE --- The string that is being tested.
\item [\colorbox{tagtype}{\color{white} \textbf{\textsf{PARAMETER}}}] \textbf{\underline{filter}} ||| UNICODE --- The string containing the set of characters to be included.
\end{description}







\par
\begin{description}
\item [\colorbox{tagtype}{\color{white} \textbf{\textsf{RETURN}}}] \textbf{UNICODE} --- 
\end{description}






\par
\begin{description}
\item [\colorbox{tagtype}{\color{white} \textbf{\textsf{SEE}}}] Std.Uni.SubstituteIncluded
\end{description}




\rule{\linewidth}{0.5pt}
\subsection*{\textsf{\colorbox{headtoc}{\color{white} FUNCTION}
Find}}

\hypertarget{ecldoc:uni.find}{}
\hspace{0pt} \hyperlink{ecldoc:Uni}{Uni} \textbackslash 

{\renewcommand{\arraystretch}{1.5}
\begin{tabularx}{\textwidth}{|>{\raggedright\arraybackslash}l|X|}
\hline
\hspace{0pt}\mytexttt{\color{red} UNSIGNED4} & \textbf{Find} \\
\hline
\multicolumn{2}{|>{\raggedright\arraybackslash}X|}{\hspace{0pt}\mytexttt{\color{param} (unicode src, unicode sought, unsigned4 instance)}} \\
\hline
\end{tabularx}
}

\par





Returns the character position of the nth match of the search string with the first string. If no match is found the attribute returns 0. If an instance is omitted the position of the first instance is returned.






\par
\begin{description}
\item [\colorbox{tagtype}{\color{white} \textbf{\textsf{PARAMETER}}}] \textbf{\underline{instance}} ||| UNSIGNED4 --- Which match instance are we interested in?
\item [\colorbox{tagtype}{\color{white} \textbf{\textsf{PARAMETER}}}] \textbf{\underline{src}} ||| UNICODE --- The string that is searched
\item [\colorbox{tagtype}{\color{white} \textbf{\textsf{PARAMETER}}}] \textbf{\underline{sought}} ||| UNICODE --- The string being sought.
\end{description}







\par
\begin{description}
\item [\colorbox{tagtype}{\color{white} \textbf{\textsf{RETURN}}}] \textbf{UNSIGNED4} --- 
\end{description}




\rule{\linewidth}{0.5pt}
\subsection*{\textsf{\colorbox{headtoc}{\color{white} FUNCTION}
FindWord}}

\hypertarget{ecldoc:uni.findword}{}
\hspace{0pt} \hyperlink{ecldoc:Uni}{Uni} \textbackslash 

{\renewcommand{\arraystretch}{1.5}
\begin{tabularx}{\textwidth}{|>{\raggedright\arraybackslash}l|X|}
\hline
\hspace{0pt}\mytexttt{\color{red} BOOLEAN} & \textbf{FindWord} \\
\hline
\multicolumn{2}{|>{\raggedright\arraybackslash}X|}{\hspace{0pt}\mytexttt{\color{param} (UNICODE src, UNICODE word, BOOLEAN ignore\_case=FALSE)}} \\
\hline
\end{tabularx}
}

\par





Tests if the search string contains the supplied word as a whole word.






\par
\begin{description}
\item [\colorbox{tagtype}{\color{white} \textbf{\textsf{PARAMETER}}}] \textbf{\underline{word}} ||| UNICODE --- The word to be searched for.
\item [\colorbox{tagtype}{\color{white} \textbf{\textsf{PARAMETER}}}] \textbf{\underline{src}} ||| UNICODE --- The string that is being tested.
\item [\colorbox{tagtype}{\color{white} \textbf{\textsf{PARAMETER}}}] \textbf{\underline{ignore\_case}} ||| BOOLEAN --- Whether to ignore differences in case between characters.
\end{description}







\par
\begin{description}
\item [\colorbox{tagtype}{\color{white} \textbf{\textsf{RETURN}}}] \textbf{BOOLEAN} --- 
\end{description}




\rule{\linewidth}{0.5pt}
\subsection*{\textsf{\colorbox{headtoc}{\color{white} FUNCTION}
LocaleFind}}

\hypertarget{ecldoc:uni.localefind}{}
\hspace{0pt} \hyperlink{ecldoc:Uni}{Uni} \textbackslash 

{\renewcommand{\arraystretch}{1.5}
\begin{tabularx}{\textwidth}{|>{\raggedright\arraybackslash}l|X|}
\hline
\hspace{0pt}\mytexttt{\color{red} UNSIGNED4} & \textbf{LocaleFind} \\
\hline
\multicolumn{2}{|>{\raggedright\arraybackslash}X|}{\hspace{0pt}\mytexttt{\color{param} (unicode src, unicode sought, unsigned4 instance, varstring locale\_name)}} \\
\hline
\end{tabularx}
}

\par





Returns the character position of the nth match of the search string with the first string. If no match is found the attribute returns 0. If an instance is omitted the position of the first instance is returned.






\par
\begin{description}
\item [\colorbox{tagtype}{\color{white} \textbf{\textsf{PARAMETER}}}] \textbf{\underline{instance}} ||| UNSIGNED4 --- Which match instance are we interested in?
\item [\colorbox{tagtype}{\color{white} \textbf{\textsf{PARAMETER}}}] \textbf{\underline{src}} ||| UNICODE --- The string that is searched
\item [\colorbox{tagtype}{\color{white} \textbf{\textsf{PARAMETER}}}] \textbf{\underline{sought}} ||| UNICODE --- The string being sought.
\item [\colorbox{tagtype}{\color{white} \textbf{\textsf{PARAMETER}}}] \textbf{\underline{locale\_name}} ||| VARSTRING --- The locale to use for the comparison
\end{description}







\par
\begin{description}
\item [\colorbox{tagtype}{\color{white} \textbf{\textsf{RETURN}}}] \textbf{UNSIGNED4} --- 
\end{description}




\rule{\linewidth}{0.5pt}
\subsection*{\textsf{\colorbox{headtoc}{\color{white} FUNCTION}
LocaleFindAtStrength}}

\hypertarget{ecldoc:uni.localefindatstrength}{}
\hspace{0pt} \hyperlink{ecldoc:Uni}{Uni} \textbackslash 

{\renewcommand{\arraystretch}{1.5}
\begin{tabularx}{\textwidth}{|>{\raggedright\arraybackslash}l|X|}
\hline
\hspace{0pt}\mytexttt{\color{red} UNSIGNED4} & \textbf{LocaleFindAtStrength} \\
\hline
\multicolumn{2}{|>{\raggedright\arraybackslash}X|}{\hspace{0pt}\mytexttt{\color{param} (unicode src, unicode tofind, unsigned4 instance, varstring locale\_name, integer1 strength)}} \\
\hline
\end{tabularx}
}

\par





Returns the character position of the nth match of the search string with the first string. If no match is found the attribute returns 0. If an instance is omitted the position of the first instance is returned.






\par
\begin{description}
\item [\colorbox{tagtype}{\color{white} \textbf{\textsf{PARAMETER}}}] \textbf{\underline{instance}} ||| UNSIGNED4 --- Which match instance are we interested in?
\item [\colorbox{tagtype}{\color{white} \textbf{\textsf{PARAMETER}}}] \textbf{\underline{strength}} ||| INTEGER1 --- The strength of the comparison 1 ignores accents and case, differentiating only between letters 2 ignores case but differentiates between accents. 3 differentiates between accents and case but ignores e.g. differences between Hiragana and Katakana 4 differentiates between accents and case and e.g. Hiragana/Katakana, but ignores e.g. Hebrew cantellation marks 5 differentiates between all strings whose canonically decomposed forms (NFDNormalization Form D) are non-identical
\item [\colorbox{tagtype}{\color{white} \textbf{\textsf{PARAMETER}}}] \textbf{\underline{src}} ||| UNICODE --- The string that is searched
\item [\colorbox{tagtype}{\color{white} \textbf{\textsf{PARAMETER}}}] \textbf{\underline{sought}} |||  --- The string being sought.
\item [\colorbox{tagtype}{\color{white} \textbf{\textsf{PARAMETER}}}] \textbf{\underline{locale\_name}} ||| VARSTRING --- The locale to use for the comparison
\item [\colorbox{tagtype}{\color{white} \textbf{\textsf{PARAMETER}}}] \textbf{\underline{tofind}} ||| UNICODE --- No Doc
\end{description}







\par
\begin{description}
\item [\colorbox{tagtype}{\color{white} \textbf{\textsf{RETURN}}}] \textbf{UNSIGNED4} --- 
\end{description}




\rule{\linewidth}{0.5pt}
\subsection*{\textsf{\colorbox{headtoc}{\color{white} FUNCTION}
Extract}}

\hypertarget{ecldoc:uni.extract}{}
\hspace{0pt} \hyperlink{ecldoc:Uni}{Uni} \textbackslash 

{\renewcommand{\arraystretch}{1.5}
\begin{tabularx}{\textwidth}{|>{\raggedright\arraybackslash}l|X|}
\hline
\hspace{0pt}\mytexttt{\color{red} unicode} & \textbf{Extract} \\
\hline
\multicolumn{2}{|>{\raggedright\arraybackslash}X|}{\hspace{0pt}\mytexttt{\color{param} (unicode src, unsigned4 instance)}} \\
\hline
\end{tabularx}
}

\par





Returns the nth element from a comma separated string.






\par
\begin{description}
\item [\colorbox{tagtype}{\color{white} \textbf{\textsf{PARAMETER}}}] \textbf{\underline{instance}} ||| UNSIGNED4 --- Which item to select from the list.
\item [\colorbox{tagtype}{\color{white} \textbf{\textsf{PARAMETER}}}] \textbf{\underline{src}} ||| UNICODE --- The string containing the comma separated list.
\end{description}







\par
\begin{description}
\item [\colorbox{tagtype}{\color{white} \textbf{\textsf{RETURN}}}] \textbf{UNICODE} --- 
\end{description}




\rule{\linewidth}{0.5pt}
\subsection*{\textsf{\colorbox{headtoc}{\color{white} FUNCTION}
ToLowerCase}}

\hypertarget{ecldoc:uni.tolowercase}{}
\hspace{0pt} \hyperlink{ecldoc:Uni}{Uni} \textbackslash 

{\renewcommand{\arraystretch}{1.5}
\begin{tabularx}{\textwidth}{|>{\raggedright\arraybackslash}l|X|}
\hline
\hspace{0pt}\mytexttt{\color{red} unicode} & \textbf{ToLowerCase} \\
\hline
\multicolumn{2}{|>{\raggedright\arraybackslash}X|}{\hspace{0pt}\mytexttt{\color{param} (unicode src)}} \\
\hline
\end{tabularx}
}

\par





Returns the argument string with all upper case characters converted to lower case.






\par
\begin{description}
\item [\colorbox{tagtype}{\color{white} \textbf{\textsf{PARAMETER}}}] \textbf{\underline{src}} ||| UNICODE --- The string that is being converted.
\end{description}







\par
\begin{description}
\item [\colorbox{tagtype}{\color{white} \textbf{\textsf{RETURN}}}] \textbf{UNICODE} --- 
\end{description}




\rule{\linewidth}{0.5pt}
\subsection*{\textsf{\colorbox{headtoc}{\color{white} FUNCTION}
ToUpperCase}}

\hypertarget{ecldoc:uni.touppercase}{}
\hspace{0pt} \hyperlink{ecldoc:Uni}{Uni} \textbackslash 

{\renewcommand{\arraystretch}{1.5}
\begin{tabularx}{\textwidth}{|>{\raggedright\arraybackslash}l|X|}
\hline
\hspace{0pt}\mytexttt{\color{red} unicode} & \textbf{ToUpperCase} \\
\hline
\multicolumn{2}{|>{\raggedright\arraybackslash}X|}{\hspace{0pt}\mytexttt{\color{param} (unicode src)}} \\
\hline
\end{tabularx}
}

\par





Return the argument string with all lower case characters converted to upper case.






\par
\begin{description}
\item [\colorbox{tagtype}{\color{white} \textbf{\textsf{PARAMETER}}}] \textbf{\underline{src}} ||| UNICODE --- The string that is being converted.
\end{description}







\par
\begin{description}
\item [\colorbox{tagtype}{\color{white} \textbf{\textsf{RETURN}}}] \textbf{UNICODE} --- 
\end{description}




\rule{\linewidth}{0.5pt}
\subsection*{\textsf{\colorbox{headtoc}{\color{white} FUNCTION}
ToTitleCase}}

\hypertarget{ecldoc:uni.totitlecase}{}
\hspace{0pt} \hyperlink{ecldoc:Uni}{Uni} \textbackslash 

{\renewcommand{\arraystretch}{1.5}
\begin{tabularx}{\textwidth}{|>{\raggedright\arraybackslash}l|X|}
\hline
\hspace{0pt}\mytexttt{\color{red} unicode} & \textbf{ToTitleCase} \\
\hline
\multicolumn{2}{|>{\raggedright\arraybackslash}X|}{\hspace{0pt}\mytexttt{\color{param} (unicode src)}} \\
\hline
\end{tabularx}
}

\par





Returns the upper case variant of the string using the rules for a particular locale.






\par
\begin{description}
\item [\colorbox{tagtype}{\color{white} \textbf{\textsf{PARAMETER}}}] \textbf{\underline{src}} ||| UNICODE --- The string that is being converted.
\item [\colorbox{tagtype}{\color{white} \textbf{\textsf{PARAMETER}}}] \textbf{\underline{locale\_name}} |||  --- The locale to use for the comparison
\end{description}







\par
\begin{description}
\item [\colorbox{tagtype}{\color{white} \textbf{\textsf{RETURN}}}] \textbf{UNICODE} --- 
\end{description}




\rule{\linewidth}{0.5pt}
\subsection*{\textsf{\colorbox{headtoc}{\color{white} FUNCTION}
LocaleToLowerCase}}

\hypertarget{ecldoc:uni.localetolowercase}{}
\hspace{0pt} \hyperlink{ecldoc:Uni}{Uni} \textbackslash 

{\renewcommand{\arraystretch}{1.5}
\begin{tabularx}{\textwidth}{|>{\raggedright\arraybackslash}l|X|}
\hline
\hspace{0pt}\mytexttt{\color{red} unicode} & \textbf{LocaleToLowerCase} \\
\hline
\multicolumn{2}{|>{\raggedright\arraybackslash}X|}{\hspace{0pt}\mytexttt{\color{param} (unicode src, varstring locale\_name)}} \\
\hline
\end{tabularx}
}

\par





Returns the lower case variant of the string using the rules for a particular locale.






\par
\begin{description}
\item [\colorbox{tagtype}{\color{white} \textbf{\textsf{PARAMETER}}}] \textbf{\underline{src}} ||| UNICODE --- The string that is being converted.
\item [\colorbox{tagtype}{\color{white} \textbf{\textsf{PARAMETER}}}] \textbf{\underline{locale\_name}} ||| VARSTRING --- The locale to use for the comparison
\end{description}







\par
\begin{description}
\item [\colorbox{tagtype}{\color{white} \textbf{\textsf{RETURN}}}] \textbf{UNICODE} --- 
\end{description}




\rule{\linewidth}{0.5pt}
\subsection*{\textsf{\colorbox{headtoc}{\color{white} FUNCTION}
LocaleToUpperCase}}

\hypertarget{ecldoc:uni.localetouppercase}{}
\hspace{0pt} \hyperlink{ecldoc:Uni}{Uni} \textbackslash 

{\renewcommand{\arraystretch}{1.5}
\begin{tabularx}{\textwidth}{|>{\raggedright\arraybackslash}l|X|}
\hline
\hspace{0pt}\mytexttt{\color{red} unicode} & \textbf{LocaleToUpperCase} \\
\hline
\multicolumn{2}{|>{\raggedright\arraybackslash}X|}{\hspace{0pt}\mytexttt{\color{param} (unicode src, varstring locale\_name)}} \\
\hline
\end{tabularx}
}

\par





Returns the upper case variant of the string using the rules for a particular locale.






\par
\begin{description}
\item [\colorbox{tagtype}{\color{white} \textbf{\textsf{PARAMETER}}}] \textbf{\underline{src}} ||| UNICODE --- The string that is being converted.
\item [\colorbox{tagtype}{\color{white} \textbf{\textsf{PARAMETER}}}] \textbf{\underline{locale\_name}} ||| VARSTRING --- The locale to use for the comparison
\end{description}







\par
\begin{description}
\item [\colorbox{tagtype}{\color{white} \textbf{\textsf{RETURN}}}] \textbf{UNICODE} --- 
\end{description}




\rule{\linewidth}{0.5pt}
\subsection*{\textsf{\colorbox{headtoc}{\color{white} FUNCTION}
LocaleToTitleCase}}

\hypertarget{ecldoc:uni.localetotitlecase}{}
\hspace{0pt} \hyperlink{ecldoc:Uni}{Uni} \textbackslash 

{\renewcommand{\arraystretch}{1.5}
\begin{tabularx}{\textwidth}{|>{\raggedright\arraybackslash}l|X|}
\hline
\hspace{0pt}\mytexttt{\color{red} unicode} & \textbf{LocaleToTitleCase} \\
\hline
\multicolumn{2}{|>{\raggedright\arraybackslash}X|}{\hspace{0pt}\mytexttt{\color{param} (unicode src, varstring locale\_name)}} \\
\hline
\end{tabularx}
}

\par





Returns the upper case variant of the string using the rules for a particular locale.






\par
\begin{description}
\item [\colorbox{tagtype}{\color{white} \textbf{\textsf{PARAMETER}}}] \textbf{\underline{src}} ||| UNICODE --- The string that is being converted.
\item [\colorbox{tagtype}{\color{white} \textbf{\textsf{PARAMETER}}}] \textbf{\underline{locale\_name}} ||| VARSTRING --- The locale to use for the comparison
\end{description}







\par
\begin{description}
\item [\colorbox{tagtype}{\color{white} \textbf{\textsf{RETURN}}}] \textbf{UNICODE} --- 
\end{description}




\rule{\linewidth}{0.5pt}
\subsection*{\textsf{\colorbox{headtoc}{\color{white} FUNCTION}
CompareIgnoreCase}}

\hypertarget{ecldoc:uni.compareignorecase}{}
\hspace{0pt} \hyperlink{ecldoc:Uni}{Uni} \textbackslash 

{\renewcommand{\arraystretch}{1.5}
\begin{tabularx}{\textwidth}{|>{\raggedright\arraybackslash}l|X|}
\hline
\hspace{0pt}\mytexttt{\color{red} integer4} & \textbf{CompareIgnoreCase} \\
\hline
\multicolumn{2}{|>{\raggedright\arraybackslash}X|}{\hspace{0pt}\mytexttt{\color{param} (unicode src1, unicode src2)}} \\
\hline
\end{tabularx}
}

\par





Compares the two strings case insensitively. Equivalent to comparing at strength 2.






\par
\begin{description}
\item [\colorbox{tagtype}{\color{white} \textbf{\textsf{PARAMETER}}}] \textbf{\underline{src2}} ||| UNICODE --- The second string to be compared.
\item [\colorbox{tagtype}{\color{white} \textbf{\textsf{PARAMETER}}}] \textbf{\underline{src1}} ||| UNICODE --- The first string to be compared.
\end{description}







\par
\begin{description}
\item [\colorbox{tagtype}{\color{white} \textbf{\textsf{RETURN}}}] \textbf{INTEGER4} --- 
\end{description}






\par
\begin{description}
\item [\colorbox{tagtype}{\color{white} \textbf{\textsf{SEE}}}] Std.Uni.CompareAtStrength
\end{description}




\rule{\linewidth}{0.5pt}
\subsection*{\textsf{\colorbox{headtoc}{\color{white} FUNCTION}
CompareAtStrength}}

\hypertarget{ecldoc:uni.compareatstrength}{}
\hspace{0pt} \hyperlink{ecldoc:Uni}{Uni} \textbackslash 

{\renewcommand{\arraystretch}{1.5}
\begin{tabularx}{\textwidth}{|>{\raggedright\arraybackslash}l|X|}
\hline
\hspace{0pt}\mytexttt{\color{red} integer4} & \textbf{CompareAtStrength} \\
\hline
\multicolumn{2}{|>{\raggedright\arraybackslash}X|}{\hspace{0pt}\mytexttt{\color{param} (unicode src1, unicode src2, integer1 strength)}} \\
\hline
\end{tabularx}
}

\par





Compares the two strings case insensitively. Equivalent to comparing at strength 2.






\par
\begin{description}
\item [\colorbox{tagtype}{\color{white} \textbf{\textsf{PARAMETER}}}] \textbf{\underline{src2}} ||| UNICODE --- The second string to be compared.
\item [\colorbox{tagtype}{\color{white} \textbf{\textsf{PARAMETER}}}] \textbf{\underline{strength}} ||| INTEGER1 --- The strength of the comparison 1 ignores accents and case, differentiating only between letters 2 ignores case but differentiates between accents. 3 differentiates between accents and case but ignores e.g. differences between Hiragana and Katakana 4 differentiates between accents and case and e.g. Hiragana/Katakana, but ignores e.g. Hebrew cantellation marks 5 differentiates between all strings whose canonically decomposed forms (NFDNormalization Form D) are non-identical
\item [\colorbox{tagtype}{\color{white} \textbf{\textsf{PARAMETER}}}] \textbf{\underline{src1}} ||| UNICODE --- The first string to be compared.
\end{description}







\par
\begin{description}
\item [\colorbox{tagtype}{\color{white} \textbf{\textsf{RETURN}}}] \textbf{INTEGER4} --- 
\end{description}






\par
\begin{description}
\item [\colorbox{tagtype}{\color{white} \textbf{\textsf{SEE}}}] Std.Uni.CompareAtStrength
\end{description}




\rule{\linewidth}{0.5pt}
\subsection*{\textsf{\colorbox{headtoc}{\color{white} FUNCTION}
LocaleCompareIgnoreCase}}

\hypertarget{ecldoc:uni.localecompareignorecase}{}
\hspace{0pt} \hyperlink{ecldoc:Uni}{Uni} \textbackslash 

{\renewcommand{\arraystretch}{1.5}
\begin{tabularx}{\textwidth}{|>{\raggedright\arraybackslash}l|X|}
\hline
\hspace{0pt}\mytexttt{\color{red} integer4} & \textbf{LocaleCompareIgnoreCase} \\
\hline
\multicolumn{2}{|>{\raggedright\arraybackslash}X|}{\hspace{0pt}\mytexttt{\color{param} (unicode src1, unicode src2, varstring locale\_name)}} \\
\hline
\end{tabularx}
}

\par





Compares the two strings case insensitively. Equivalent to comparing at strength 2.






\par
\begin{description}
\item [\colorbox{tagtype}{\color{white} \textbf{\textsf{PARAMETER}}}] \textbf{\underline{src2}} ||| UNICODE --- The second string to be compared.
\item [\colorbox{tagtype}{\color{white} \textbf{\textsf{PARAMETER}}}] \textbf{\underline{src1}} ||| UNICODE --- The first string to be compared.
\item [\colorbox{tagtype}{\color{white} \textbf{\textsf{PARAMETER}}}] \textbf{\underline{locale\_name}} ||| VARSTRING --- The locale to use for the comparison
\end{description}







\par
\begin{description}
\item [\colorbox{tagtype}{\color{white} \textbf{\textsf{RETURN}}}] \textbf{INTEGER4} --- 
\end{description}






\par
\begin{description}
\item [\colorbox{tagtype}{\color{white} \textbf{\textsf{SEE}}}] Std.Uni.CompareAtStrength
\end{description}




\rule{\linewidth}{0.5pt}
\subsection*{\textsf{\colorbox{headtoc}{\color{white} FUNCTION}
LocaleCompareAtStrength}}

\hypertarget{ecldoc:uni.localecompareatstrength}{}
\hspace{0pt} \hyperlink{ecldoc:Uni}{Uni} \textbackslash 

{\renewcommand{\arraystretch}{1.5}
\begin{tabularx}{\textwidth}{|>{\raggedright\arraybackslash}l|X|}
\hline
\hspace{0pt}\mytexttt{\color{red} integer4} & \textbf{LocaleCompareAtStrength} \\
\hline
\multicolumn{2}{|>{\raggedright\arraybackslash}X|}{\hspace{0pt}\mytexttt{\color{param} (unicode src1, unicode src2, varstring locale\_name, integer1 strength)}} \\
\hline
\end{tabularx}
}

\par





Compares the two strings case insensitively. Equivalent to comparing at strength 2.






\par
\begin{description}
\item [\colorbox{tagtype}{\color{white} \textbf{\textsf{PARAMETER}}}] \textbf{\underline{src2}} ||| UNICODE --- The second string to be compared.
\item [\colorbox{tagtype}{\color{white} \textbf{\textsf{PARAMETER}}}] \textbf{\underline{strength}} ||| INTEGER1 --- The strength of the comparison 1 ignores accents and case, differentiating only between letters 2 ignores case but differentiates between accents. 3 differentiates between accents and case but ignores e.g. differences between Hiragana and Katakana 4 differentiates between accents and case and e.g. Hiragana/Katakana, but ignores e.g. Hebrew cantellation marks 5 differentiates between all strings whose canonically decomposed forms (NFDNormalization Form D) are non-identical
\item [\colorbox{tagtype}{\color{white} \textbf{\textsf{PARAMETER}}}] \textbf{\underline{src1}} ||| UNICODE --- The first string to be compared.
\item [\colorbox{tagtype}{\color{white} \textbf{\textsf{PARAMETER}}}] \textbf{\underline{locale\_name}} ||| VARSTRING --- The locale to use for the comparison
\end{description}







\par
\begin{description}
\item [\colorbox{tagtype}{\color{white} \textbf{\textsf{RETURN}}}] \textbf{INTEGER4} --- 
\end{description}




\rule{\linewidth}{0.5pt}
\subsection*{\textsf{\colorbox{headtoc}{\color{white} FUNCTION}
Reverse}}

\hypertarget{ecldoc:uni.reverse}{}
\hspace{0pt} \hyperlink{ecldoc:Uni}{Uni} \textbackslash 

{\renewcommand{\arraystretch}{1.5}
\begin{tabularx}{\textwidth}{|>{\raggedright\arraybackslash}l|X|}
\hline
\hspace{0pt}\mytexttt{\color{red} unicode} & \textbf{Reverse} \\
\hline
\multicolumn{2}{|>{\raggedright\arraybackslash}X|}{\hspace{0pt}\mytexttt{\color{param} (unicode src)}} \\
\hline
\end{tabularx}
}

\par





Returns the argument string with all characters in reverse order. Note the argument is not TRIMMED before it is reversed.






\par
\begin{description}
\item [\colorbox{tagtype}{\color{white} \textbf{\textsf{PARAMETER}}}] \textbf{\underline{src}} ||| UNICODE --- The string that is being reversed.
\end{description}







\par
\begin{description}
\item [\colorbox{tagtype}{\color{white} \textbf{\textsf{RETURN}}}] \textbf{UNICODE} --- 
\end{description}




\rule{\linewidth}{0.5pt}
\subsection*{\textsf{\colorbox{headtoc}{\color{white} FUNCTION}
FindReplace}}

\hypertarget{ecldoc:uni.findreplace}{}
\hspace{0pt} \hyperlink{ecldoc:Uni}{Uni} \textbackslash 

{\renewcommand{\arraystretch}{1.5}
\begin{tabularx}{\textwidth}{|>{\raggedright\arraybackslash}l|X|}
\hline
\hspace{0pt}\mytexttt{\color{red} unicode} & \textbf{FindReplace} \\
\hline
\multicolumn{2}{|>{\raggedright\arraybackslash}X|}{\hspace{0pt}\mytexttt{\color{param} (unicode src, unicode sought, unicode replacement)}} \\
\hline
\end{tabularx}
}

\par





Returns the source string with the replacement string substituted for all instances of the search string.






\par
\begin{description}
\item [\colorbox{tagtype}{\color{white} \textbf{\textsf{PARAMETER}}}] \textbf{\underline{src}} ||| UNICODE --- The string that is being transformed.
\item [\colorbox{tagtype}{\color{white} \textbf{\textsf{PARAMETER}}}] \textbf{\underline{replacement}} ||| UNICODE --- The string to be substituted into the result.
\item [\colorbox{tagtype}{\color{white} \textbf{\textsf{PARAMETER}}}] \textbf{\underline{sought}} ||| UNICODE --- The string to be replaced.
\end{description}







\par
\begin{description}
\item [\colorbox{tagtype}{\color{white} \textbf{\textsf{RETURN}}}] \textbf{UNICODE} --- 
\end{description}




\rule{\linewidth}{0.5pt}
\subsection*{\textsf{\colorbox{headtoc}{\color{white} FUNCTION}
LocaleFindReplace}}

\hypertarget{ecldoc:uni.localefindreplace}{}
\hspace{0pt} \hyperlink{ecldoc:Uni}{Uni} \textbackslash 

{\renewcommand{\arraystretch}{1.5}
\begin{tabularx}{\textwidth}{|>{\raggedright\arraybackslash}l|X|}
\hline
\hspace{0pt}\mytexttt{\color{red} unicode} & \textbf{LocaleFindReplace} \\
\hline
\multicolumn{2}{|>{\raggedright\arraybackslash}X|}{\hspace{0pt}\mytexttt{\color{param} (unicode src, unicode sought, unicode replacement, varstring locale\_name)}} \\
\hline
\end{tabularx}
}

\par





Returns the source string with the replacement string substituted for all instances of the search string.






\par
\begin{description}
\item [\colorbox{tagtype}{\color{white} \textbf{\textsf{PARAMETER}}}] \textbf{\underline{src}} ||| UNICODE --- The string that is being transformed.
\item [\colorbox{tagtype}{\color{white} \textbf{\textsf{PARAMETER}}}] \textbf{\underline{replacement}} ||| UNICODE --- The string to be substituted into the result.
\item [\colorbox{tagtype}{\color{white} \textbf{\textsf{PARAMETER}}}] \textbf{\underline{sought}} ||| UNICODE --- The string to be replaced.
\item [\colorbox{tagtype}{\color{white} \textbf{\textsf{PARAMETER}}}] \textbf{\underline{locale\_name}} ||| VARSTRING --- The locale to use for the comparison
\end{description}







\par
\begin{description}
\item [\colorbox{tagtype}{\color{white} \textbf{\textsf{RETURN}}}] \textbf{UNICODE} --- 
\end{description}




\rule{\linewidth}{0.5pt}
\subsection*{\textsf{\colorbox{headtoc}{\color{white} FUNCTION}
LocaleFindAtStrengthReplace}}

\hypertarget{ecldoc:uni.localefindatstrengthreplace}{}
\hspace{0pt} \hyperlink{ecldoc:Uni}{Uni} \textbackslash 

{\renewcommand{\arraystretch}{1.5}
\begin{tabularx}{\textwidth}{|>{\raggedright\arraybackslash}l|X|}
\hline
\hspace{0pt}\mytexttt{\color{red} unicode} & \textbf{LocaleFindAtStrengthReplace} \\
\hline
\multicolumn{2}{|>{\raggedright\arraybackslash}X|}{\hspace{0pt}\mytexttt{\color{param} (unicode src, unicode sought, unicode replacement, varstring locale\_name, integer1 strength)}} \\
\hline
\end{tabularx}
}

\par





Returns the source string with the replacement string substituted for all instances of the search string.






\par
\begin{description}
\item [\colorbox{tagtype}{\color{white} \textbf{\textsf{PARAMETER}}}] \textbf{\underline{strength}} ||| INTEGER1 --- The strength of the comparison
\item [\colorbox{tagtype}{\color{white} \textbf{\textsf{PARAMETER}}}] \textbf{\underline{src}} ||| UNICODE --- The string that is being transformed.
\item [\colorbox{tagtype}{\color{white} \textbf{\textsf{PARAMETER}}}] \textbf{\underline{replacement}} ||| UNICODE --- The string to be substituted into the result.
\item [\colorbox{tagtype}{\color{white} \textbf{\textsf{PARAMETER}}}] \textbf{\underline{sought}} ||| UNICODE --- The string to be replaced.
\item [\colorbox{tagtype}{\color{white} \textbf{\textsf{PARAMETER}}}] \textbf{\underline{locale\_name}} ||| VARSTRING --- The locale to use for the comparison
\end{description}







\par
\begin{description}
\item [\colorbox{tagtype}{\color{white} \textbf{\textsf{RETURN}}}] \textbf{UNICODE} --- 
\end{description}




\rule{\linewidth}{0.5pt}
\subsection*{\textsf{\colorbox{headtoc}{\color{white} FUNCTION}
CleanAccents}}

\hypertarget{ecldoc:uni.cleanaccents}{}
\hspace{0pt} \hyperlink{ecldoc:Uni}{Uni} \textbackslash 

{\renewcommand{\arraystretch}{1.5}
\begin{tabularx}{\textwidth}{|>{\raggedright\arraybackslash}l|X|}
\hline
\hspace{0pt}\mytexttt{\color{red} unicode} & \textbf{CleanAccents} \\
\hline
\multicolumn{2}{|>{\raggedright\arraybackslash}X|}{\hspace{0pt}\mytexttt{\color{param} (unicode src)}} \\
\hline
\end{tabularx}
}

\par





Returns the source string with all accented characters replaced with unaccented.






\par
\begin{description}
\item [\colorbox{tagtype}{\color{white} \textbf{\textsf{PARAMETER}}}] \textbf{\underline{src}} ||| UNICODE --- The string that is being transformed.
\end{description}







\par
\begin{description}
\item [\colorbox{tagtype}{\color{white} \textbf{\textsf{RETURN}}}] \textbf{UNICODE} --- 
\end{description}




\rule{\linewidth}{0.5pt}
\subsection*{\textsf{\colorbox{headtoc}{\color{white} FUNCTION}
CleanSpaces}}

\hypertarget{ecldoc:uni.cleanspaces}{}
\hspace{0pt} \hyperlink{ecldoc:Uni}{Uni} \textbackslash 

{\renewcommand{\arraystretch}{1.5}
\begin{tabularx}{\textwidth}{|>{\raggedright\arraybackslash}l|X|}
\hline
\hspace{0pt}\mytexttt{\color{red} unicode} & \textbf{CleanSpaces} \\
\hline
\multicolumn{2}{|>{\raggedright\arraybackslash}X|}{\hspace{0pt}\mytexttt{\color{param} (unicode src)}} \\
\hline
\end{tabularx}
}

\par





Returns the source string with all instances of multiple adjacent space characters (2 or more spaces together) reduced to a single space character. Leading and trailing spaces are removed, and tab characters are converted to spaces.






\par
\begin{description}
\item [\colorbox{tagtype}{\color{white} \textbf{\textsf{PARAMETER}}}] \textbf{\underline{src}} ||| UNICODE --- The string to be cleaned.
\end{description}







\par
\begin{description}
\item [\colorbox{tagtype}{\color{white} \textbf{\textsf{RETURN}}}] \textbf{UNICODE} --- 
\end{description}




\rule{\linewidth}{0.5pt}
\subsection*{\textsf{\colorbox{headtoc}{\color{white} FUNCTION}
WildMatch}}

\hypertarget{ecldoc:uni.wildmatch}{}
\hspace{0pt} \hyperlink{ecldoc:Uni}{Uni} \textbackslash 

{\renewcommand{\arraystretch}{1.5}
\begin{tabularx}{\textwidth}{|>{\raggedright\arraybackslash}l|X|}
\hline
\hspace{0pt}\mytexttt{\color{red} boolean} & \textbf{WildMatch} \\
\hline
\multicolumn{2}{|>{\raggedright\arraybackslash}X|}{\hspace{0pt}\mytexttt{\color{param} (unicode src, unicode \_pattern, boolean \_noCase)}} \\
\hline
\end{tabularx}
}

\par





Tests if the search string matches the pattern. The pattern can contain wildcards '?' (single character) and '*' (multiple character).






\par
\begin{description}
\item [\colorbox{tagtype}{\color{white} \textbf{\textsf{PARAMETER}}}] \textbf{\underline{pattern}} |||  --- The pattern to match against.
\item [\colorbox{tagtype}{\color{white} \textbf{\textsf{PARAMETER}}}] \textbf{\underline{src}} ||| UNICODE --- The string that is being tested.
\item [\colorbox{tagtype}{\color{white} \textbf{\textsf{PARAMETER}}}] \textbf{\underline{ignore\_case}} |||  --- Whether to ignore differences in case between characters
\item [\colorbox{tagtype}{\color{white} \textbf{\textsf{PARAMETER}}}] \textbf{\underline{\_nocase}} ||| BOOLEAN --- No Doc
\item [\colorbox{tagtype}{\color{white} \textbf{\textsf{PARAMETER}}}] \textbf{\underline{\_pattern}} ||| UNICODE --- No Doc
\end{description}







\par
\begin{description}
\item [\colorbox{tagtype}{\color{white} \textbf{\textsf{RETURN}}}] \textbf{BOOLEAN} --- 
\end{description}




\rule{\linewidth}{0.5pt}
\subsection*{\textsf{\colorbox{headtoc}{\color{white} FUNCTION}
Contains}}

\hypertarget{ecldoc:uni.contains}{}
\hspace{0pt} \hyperlink{ecldoc:Uni}{Uni} \textbackslash 

{\renewcommand{\arraystretch}{1.5}
\begin{tabularx}{\textwidth}{|>{\raggedright\arraybackslash}l|X|}
\hline
\hspace{0pt}\mytexttt{\color{red} BOOLEAN} & \textbf{Contains} \\
\hline
\multicolumn{2}{|>{\raggedright\arraybackslash}X|}{\hspace{0pt}\mytexttt{\color{param} (unicode src, unicode \_pattern, boolean \_noCase)}} \\
\hline
\end{tabularx}
}

\par





Tests if the search string contains each of the characters in the pattern. If the pattern contains duplicate characters those characters will match once for each occurence in the pattern.






\par
\begin{description}
\item [\colorbox{tagtype}{\color{white} \textbf{\textsf{PARAMETER}}}] \textbf{\underline{pattern}} |||  --- The pattern to match against.
\item [\colorbox{tagtype}{\color{white} \textbf{\textsf{PARAMETER}}}] \textbf{\underline{src}} ||| UNICODE --- The string that is being tested.
\item [\colorbox{tagtype}{\color{white} \textbf{\textsf{PARAMETER}}}] \textbf{\underline{ignore\_case}} |||  --- Whether to ignore differences in case between characters
\item [\colorbox{tagtype}{\color{white} \textbf{\textsf{PARAMETER}}}] \textbf{\underline{\_nocase}} ||| BOOLEAN --- No Doc
\item [\colorbox{tagtype}{\color{white} \textbf{\textsf{PARAMETER}}}] \textbf{\underline{\_pattern}} ||| UNICODE --- No Doc
\end{description}







\par
\begin{description}
\item [\colorbox{tagtype}{\color{white} \textbf{\textsf{RETURN}}}] \textbf{BOOLEAN} --- 
\end{description}




\rule{\linewidth}{0.5pt}
\subsection*{\textsf{\colorbox{headtoc}{\color{white} FUNCTION}
EditDistance}}

\hypertarget{ecldoc:uni.editdistance}{}
\hspace{0pt} \hyperlink{ecldoc:Uni}{Uni} \textbackslash 

{\renewcommand{\arraystretch}{1.5}
\begin{tabularx}{\textwidth}{|>{\raggedright\arraybackslash}l|X|}
\hline
\hspace{0pt}\mytexttt{\color{red} UNSIGNED4} & \textbf{EditDistance} \\
\hline
\multicolumn{2}{|>{\raggedright\arraybackslash}X|}{\hspace{0pt}\mytexttt{\color{param} (unicode \_left, unicode \_right, varstring localename = '')}} \\
\hline
\end{tabularx}
}

\par





Returns the minimum edit distance between the two strings. An insert change or delete counts as a single edit. The two strings are trimmed before comparing.






\par
\begin{description}
\item [\colorbox{tagtype}{\color{white} \textbf{\textsf{PARAMETER}}}] \textbf{\underline{\_left}} ||| UNICODE --- The first string to be compared.
\item [\colorbox{tagtype}{\color{white} \textbf{\textsf{PARAMETER}}}] \textbf{\underline{localname}} |||  --- The locale to use for the comparison. Defaults to ''.
\item [\colorbox{tagtype}{\color{white} \textbf{\textsf{PARAMETER}}}] \textbf{\underline{\_right}} ||| UNICODE --- The second string to be compared.
\item [\colorbox{tagtype}{\color{white} \textbf{\textsf{PARAMETER}}}] \textbf{\underline{localename}} ||| VARSTRING --- No Doc
\end{description}







\par
\begin{description}
\item [\colorbox{tagtype}{\color{white} \textbf{\textsf{RETURN}}}] \textbf{UNSIGNED4} --- The minimum edit distance between the two strings.
\end{description}




\rule{\linewidth}{0.5pt}
\subsection*{\textsf{\colorbox{headtoc}{\color{white} FUNCTION}
EditDistanceWithinRadius}}

\hypertarget{ecldoc:uni.editdistancewithinradius}{}
\hspace{0pt} \hyperlink{ecldoc:Uni}{Uni} \textbackslash 

{\renewcommand{\arraystretch}{1.5}
\begin{tabularx}{\textwidth}{|>{\raggedright\arraybackslash}l|X|}
\hline
\hspace{0pt}\mytexttt{\color{red} BOOLEAN} & \textbf{EditDistanceWithinRadius} \\
\hline
\multicolumn{2}{|>{\raggedright\arraybackslash}X|}{\hspace{0pt}\mytexttt{\color{param} (unicode \_left, unicode \_right, unsigned4 radius, varstring localename = '')}} \\
\hline
\end{tabularx}
}

\par





Returns true if the minimum edit distance between the two strings is with a specific range. The two strings are trimmed before comparing.






\par
\begin{description}
\item [\colorbox{tagtype}{\color{white} \textbf{\textsf{PARAMETER}}}] \textbf{\underline{\_left}} ||| UNICODE --- The first string to be compared.
\item [\colorbox{tagtype}{\color{white} \textbf{\textsf{PARAMETER}}}] \textbf{\underline{localname}} |||  --- The locale to use for the comparison. Defaults to ''.
\item [\colorbox{tagtype}{\color{white} \textbf{\textsf{PARAMETER}}}] \textbf{\underline{\_right}} ||| UNICODE --- The second string to be compared.
\item [\colorbox{tagtype}{\color{white} \textbf{\textsf{PARAMETER}}}] \textbf{\underline{radius}} ||| UNSIGNED4 --- The maximum edit distance that is accepable.
\item [\colorbox{tagtype}{\color{white} \textbf{\textsf{PARAMETER}}}] \textbf{\underline{localename}} ||| VARSTRING --- No Doc
\end{description}







\par
\begin{description}
\item [\colorbox{tagtype}{\color{white} \textbf{\textsf{RETURN}}}] \textbf{BOOLEAN} --- Whether or not the two strings are within the given specified edit distance.
\end{description}




\rule{\linewidth}{0.5pt}
\subsection*{\textsf{\colorbox{headtoc}{\color{white} FUNCTION}
WordCount}}

\hypertarget{ecldoc:uni.wordcount}{}
\hspace{0pt} \hyperlink{ecldoc:Uni}{Uni} \textbackslash 

{\renewcommand{\arraystretch}{1.5}
\begin{tabularx}{\textwidth}{|>{\raggedright\arraybackslash}l|X|}
\hline
\hspace{0pt}\mytexttt{\color{red} unsigned4} & \textbf{WordCount} \\
\hline
\multicolumn{2}{|>{\raggedright\arraybackslash}X|}{\hspace{0pt}\mytexttt{\color{param} (unicode text, varstring localename = '')}} \\
\hline
\end{tabularx}
}

\par





Returns the number of words in the string. Word boundaries are marked by the unicode break semantics.






\par
\begin{description}
\item [\colorbox{tagtype}{\color{white} \textbf{\textsf{PARAMETER}}}] \textbf{\underline{localname}} |||  --- The locale to use for the break semantics. Defaults to ''.
\item [\colorbox{tagtype}{\color{white} \textbf{\textsf{PARAMETER}}}] \textbf{\underline{text}} ||| UNICODE --- The string to be broken into words.
\item [\colorbox{tagtype}{\color{white} \textbf{\textsf{PARAMETER}}}] \textbf{\underline{localename}} ||| VARSTRING --- No Doc
\end{description}







\par
\begin{description}
\item [\colorbox{tagtype}{\color{white} \textbf{\textsf{RETURN}}}] \textbf{UNSIGNED4} --- The number of words in the string.
\end{description}




\rule{\linewidth}{0.5pt}
\subsection*{\textsf{\colorbox{headtoc}{\color{white} FUNCTION}
GetNthWord}}

\hypertarget{ecldoc:uni.getnthword}{}
\hspace{0pt} \hyperlink{ecldoc:Uni}{Uni} \textbackslash 

{\renewcommand{\arraystretch}{1.5}
\begin{tabularx}{\textwidth}{|>{\raggedright\arraybackslash}l|X|}
\hline
\hspace{0pt}\mytexttt{\color{red} unicode} & \textbf{GetNthWord} \\
\hline
\multicolumn{2}{|>{\raggedright\arraybackslash}X|}{\hspace{0pt}\mytexttt{\color{param} (unicode text, unsigned4 n, varstring localename = '')}} \\
\hline
\end{tabularx}
}

\par





Returns the n-th word from the string. Word boundaries are marked by the unicode break semantics.






\par
\begin{description}
\item [\colorbox{tagtype}{\color{white} \textbf{\textsf{PARAMETER}}}] \textbf{\underline{localname}} |||  --- The locale to use for the break semantics. Defaults to ''.
\item [\colorbox{tagtype}{\color{white} \textbf{\textsf{PARAMETER}}}] \textbf{\underline{n}} ||| UNSIGNED4 --- Which word should be returned from the function.
\item [\colorbox{tagtype}{\color{white} \textbf{\textsf{PARAMETER}}}] \textbf{\underline{text}} ||| UNICODE --- The string to be broken into words.
\item [\colorbox{tagtype}{\color{white} \textbf{\textsf{PARAMETER}}}] \textbf{\underline{localename}} ||| VARSTRING --- No Doc
\end{description}







\par
\begin{description}
\item [\colorbox{tagtype}{\color{white} \textbf{\textsf{RETURN}}}] \textbf{UNICODE} --- The number of words in the string.
\end{description}




\rule{\linewidth}{0.5pt}



\chapter*{\color{headtoc} root}
\hypertarget{ecldoc:toc:root}{}
\hyperlink{ecldoc:toc:}{Go Up}


\section*{Table of Contents}
{\renewcommand{\arraystretch}{1.5}
\begin{longtable}{|p{\textwidth}|}
\hline
\hyperlink{ecldoc:toc:BLAS}{BLAS.ecl} \\
\hline
\hyperlink{ecldoc:toc:BundleBase}{BundleBase.ecl} \\
\hline
\hyperlink{ecldoc:toc:Date}{Date.ecl} \\
\hline
\hyperlink{ecldoc:toc:File}{File.ecl} \\
\hline
\hyperlink{ecldoc:toc:math}{math.ecl} \\
\hline
\hyperlink{ecldoc:toc:Metaphone}{Metaphone.ecl} \\
\hline
\hyperlink{ecldoc:toc:str}{str.ecl} \\
\hline
\hyperlink{ecldoc:toc:Uni}{Uni.ecl} \\
\hline
\hyperlink{ecldoc:toc:root/system}{system} \\
\hline
\end{longtable}
}

\input{root/BLAS.ecl}
\input{root/BundleBase.ecl}
\input{root/Date.ecl}
\input{root/File.ecl}
\input{root/math.ecl}
\input{root/Metaphone.ecl}
\input{root/str.ecl}
\input{root/Uni.ecl}
\input{root/system/pkg.toc}



