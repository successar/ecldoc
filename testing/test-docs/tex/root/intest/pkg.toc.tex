\chapter*{\color{headtoc} intest}
\hypertarget{ecldoc:toc:root/intest}{}
\hyperlink{ecldoc:toc:root}{Go Up}


\section*{Table of Contents}
{\renewcommand{\arraystretch}{1.5}
\begin{longtable}{|p{\textwidth}|}
\hline
\hyperlink{ecldoc:toc:intest.example_11}{example\_11.ecl} \\
\hline
\hyperlink{ecldoc:toc:intest.example_2}{example\_2.ecl} \\
Basic Inheritance documentation : mod\_3 inherits both mod\_1 and mod\_2 \\
\hline
\hyperlink{ecldoc:toc:intest.example_3}{example\_3.ecl} \\
Example : Inheritance across files \\
\hline
\hyperlink{ecldoc:toc:intest.example_4}{example\_4.ecl} \\
Example : Inheritance across files \\
\hline
\hyperlink{ecldoc:toc:intest.example_5}{example\_5.ecl} \\
\hline
\hyperlink{ecldoc:toc:intest.example_7}{example\_7.ecl} \\
Basic Type Example \\
\hline
\hyperlink{ecldoc:toc:intest.example_9}{example\_9.ecl} \\
\hline
\hyperlink{ecldoc:toc:root/intest/in1intest}{in1intest} \\
\hline
\hyperlink{ecldoc:toc:root/intest/inintest}{inintest} \\
\hline
\end{longtable}
}

\chapter*{example\_11}
\hypertarget{ecldoc:toc:example_11}{}

\section*{\underline{IMPORTS}}
\begin{itemize}
\item Inintest
\item Example\_3
\item intest.Example\_3
\item intest.inintest.Example\_3
\item Inintest.Example\_3
\end{itemize}

\section*{\underline{DESCRIPTIONS}}
\subsection*{MODULE : example\_11}
\hypertarget{ecldoc:example_11}{}
\par
\begin{minipage}[t]{\textwidth}
\begin{flushleft}
  
\end{flushleft}
\end{minipage}
\hyperlink{ecldoc:toc:root}{Up} \\
\par
\par
\begin{enumerate}
\item \hyperlink{ecldoc:Inintest.Example_3}{Example\_3}
\end{enumerate}
\subsection*{MODULE : Example\_3}
\hypertarget{ecldoc:Inintest.Example_3}{}
\par
\begin{minipage}[t]{\textwidth}
\begin{flushleft}
  
\end{flushleft}
\end{minipage}
\hyperlink{ecldoc:example_11}{Up} \\
\par
\par
\textbf{OVERRIDE} : True \\
\begin{enumerate}
\item \hyperlink{ecldoc:Inintest.Example_3.mod_1}{mod\_1}
\end{enumerate}
\subsection*{MODULE : mod\_1}
\hypertarget{ecldoc:Inintest.Example_3.mod_1}{}
\par
\begin{minipage}[t]{\textwidth}
\begin{flushleft}
  
\end{flushleft}
\end{minipage}
\hyperlink{ecldoc:Inintest.Example_3}{Up} \\
\par
\par
\begin{enumerate}
\item \hyperlink{ecldoc:inintest.example_3.mod_1.v2_m1_ex3}{v2\_m1\_ex3}
\end{enumerate}
\subsection*{ATTRIBUTE : v2\_m1\_ex3}
\hypertarget{ecldoc:inintest.example_3.mod_1.v2_m1_ex3}{}
\par
\begin{minipage}[t]{\textwidth}
\begin{flushleft}
  
\end{flushleft}
\end{minipage}
\hyperlink{ecldoc:Inintest.Example_3.mod_1}{Up} \\
\par
\par




\chapter*{intest.in1intest.example\_2}
\hypertarget{ecldoc:toc:intest.in1intest.example_2}{}

\section*{\underline{IMPORTS}}

\section*{\underline{DESCRIPTIONS}}
\subsection*{MODULE : example\_2}
\hypertarget{ecldoc:intest.in1intest.example_2}{}
\hyperlink{ecldoc:toc:intest/in1intest}{Up} :

{\renewcommand{\arraystretch}{1.5}
\begin{tabularx}{\textwidth}{|>{\raggedright\arraybackslash}l|X|}
\hline
\hspace{0pt} & example\_2 \\
\hline
\end{tabularx}
}

\par
Basic Inheritance documentation : mod\_3 inherits both mod\_1 and mod\_2 . Inherits v2\_m1, v2\_m2, Overrides v1\_m1, new locals v2\_m3 . Interface Inheritance : mod\_4 inherits interface iface\_1, overrides v1\_i1


\hyperlink{ecldoc:intest.in1intest.example_2.rec_1}{rec\_1}  |
\hyperlink{ecldoc:intest.in1intest.example_2.rec_2}{rec\_2}  |
\hyperlink{ecldoc:intest.in1intest.example_2.rec_3}{rec\_3}  |
\hyperlink{ecldoc:intest.in1intest.example_2.mod_1}{mod\_1}  |
\hyperlink{ecldoc:intest.in1intest.example_2.mod_2}{mod\_2}  |
\hyperlink{ecldoc:intest.in1intest.example_2.mod_3}{mod\_3}  |
\hyperlink{ecldoc:intest.in1intest.example_2.iface_1}{iface\_1}  |
\hyperlink{ecldoc:intest.in1intest.example_2.mod_4}{mod\_4}  |

\rule{\linewidth}{0.5pt}

\subsection*{RECORD : rec\_1}
\hypertarget{ecldoc:intest.in1intest.example_2.rec_1}{}
\hyperlink{ecldoc:intest.in1intest.example_2}{Up} :
\hspace{0pt} \hyperlink{ecldoc:intest.in1intest.example_2}{example_2} \textbackslash 

{\renewcommand{\arraystretch}{1.5}
\begin{tabularx}{\textwidth}{|>{\raggedright\arraybackslash}l|X|}
\hline
\hspace{0pt} & rec\_1 \\
\hline
\end{tabularx}
}

\par


\rule{\linewidth}{0.5pt}
\subsection*{RECORD : rec\_2}
\hypertarget{ecldoc:intest.in1intest.example_2.rec_2}{}
\hyperlink{ecldoc:intest.in1intest.example_2}{Up} :
\hspace{0pt} \hyperlink{ecldoc:intest.in1intest.example_2}{example_2} \textbackslash 

{\renewcommand{\arraystretch}{1.5}
\begin{tabularx}{\textwidth}{|>{\raggedright\arraybackslash}l|X|}
\hline
\hspace{0pt} & rec\_2 \\
\hline
\end{tabularx}
}

\par


\rule{\linewidth}{0.5pt}
\subsection*{RECORD : rec\_3}
\hypertarget{ecldoc:intest.in1intest.example_2.rec_3}{}
\hyperlink{ecldoc:intest.in1intest.example_2}{Up} :
\hspace{0pt} \hyperlink{ecldoc:intest.in1intest.example_2}{example_2} \textbackslash 

{\renewcommand{\arraystretch}{1.5}
\begin{tabularx}{\textwidth}{|>{\raggedright\arraybackslash}l|X|}
\hline
\hspace{0pt} & rec\_3 \\
\hline
\end{tabularx}
}

\par


\rule{\linewidth}{0.5pt}
\subsection*{MODULE : mod\_1}
\hypertarget{ecldoc:intest.in1intest.example_2.mod_1}{}
\hyperlink{ecldoc:intest.in1intest.example_2}{Up} :
\hspace{0pt} \hyperlink{ecldoc:intest.in1intest.example_2}{example_2} \textbackslash 

{\renewcommand{\arraystretch}{1.5}
\begin{tabularx}{\textwidth}{|>{\raggedright\arraybackslash}l|X|}
\hline
\hspace{0pt} & mod\_1 \\
\hline
\end{tabularx}
}

\par


\hyperlink{ecldoc:intest.in1intest.example_2.mod_1.v1_m1}{v1\_m1}  |
\hyperlink{ecldoc:intest.in1intest.example_2.mod_1.v2_m1}{v2\_m1}  |

\rule{\linewidth}{0.5pt}

\subsection*{ATTRIBUTE : v1\_m1}
\hypertarget{ecldoc:intest.in1intest.example_2.mod_1.v1_m1}{}
\hyperlink{ecldoc:intest.in1intest.example_2.mod_1}{Up} :
\hspace{0pt} \hyperlink{ecldoc:intest.in1intest.example_2}{example_2} \textbackslash 
\hspace{0pt} \hyperlink{ecldoc:intest.in1intest.example_2.mod_1}{mod_1} \textbackslash 

{\renewcommand{\arraystretch}{1.5}
\begin{tabularx}{\textwidth}{|>{\raggedright\arraybackslash}l|X|}
\hline
\hspace{0pt}real8 & v1\_m1 \\
\hline
\end{tabularx}
}

\par


\rule{\linewidth}{0.5pt}
\subsection*{ATTRIBUTE : v2\_m1}
\hypertarget{ecldoc:intest.in1intest.example_2.mod_1.v2_m1}{}
\hyperlink{ecldoc:intest.in1intest.example_2.mod_1}{Up} :
\hspace{0pt} \hyperlink{ecldoc:intest.in1intest.example_2}{example_2} \textbackslash 
\hspace{0pt} \hyperlink{ecldoc:intest.in1intest.example_2.mod_1}{mod_1} \textbackslash 

{\renewcommand{\arraystretch}{1.5}
\begin{tabularx}{\textwidth}{|>{\raggedright\arraybackslash}l|X|}
\hline
\hspace{0pt} & v2\_m1 \\
\hline
\end{tabularx}
}

\par


\rule{\linewidth}{0.5pt}


\subsection*{MODULE : mod\_2}
\hypertarget{ecldoc:intest.in1intest.example_2.mod_2}{}
\hyperlink{ecldoc:intest.in1intest.example_2}{Up} :
\hspace{0pt} \hyperlink{ecldoc:intest.in1intest.example_2}{example_2} \textbackslash 

{\renewcommand{\arraystretch}{1.5}
\begin{tabularx}{\textwidth}{|>{\raggedright\arraybackslash}l|X|}
\hline
\hspace{0pt} & mod\_2 \\
\hline
\end{tabularx}
}

\par


\hyperlink{ecldoc:intest.in1intest.example_2.mod_2.v1_m1}{v1\_m1}  |
\hyperlink{ecldoc:intest.in1intest.example_2.mod_2.v2_m2}{v2\_m2}  |

\rule{\linewidth}{0.5pt}

\subsection*{ATTRIBUTE : v1\_m1}
\hypertarget{ecldoc:intest.in1intest.example_2.mod_2.v1_m1}{}
\hyperlink{ecldoc:intest.in1intest.example_2.mod_2}{Up} :
\hspace{0pt} \hyperlink{ecldoc:intest.in1intest.example_2}{example_2} \textbackslash 
\hspace{0pt} \hyperlink{ecldoc:intest.in1intest.example_2.mod_2}{mod_2} \textbackslash 

{\renewcommand{\arraystretch}{1.5}
\begin{tabularx}{\textwidth}{|>{\raggedright\arraybackslash}l|X|}
\hline
\hspace{0pt} & v1\_m1 \\
\hline
\end{tabularx}
}

\par


\rule{\linewidth}{0.5pt}
\subsection*{ATTRIBUTE : v2\_m2}
\hypertarget{ecldoc:intest.in1intest.example_2.mod_2.v2_m2}{}
\hyperlink{ecldoc:intest.in1intest.example_2.mod_2}{Up} :
\hspace{0pt} \hyperlink{ecldoc:intest.in1intest.example_2}{example_2} \textbackslash 
\hspace{0pt} \hyperlink{ecldoc:intest.in1intest.example_2.mod_2}{mod_2} \textbackslash 

{\renewcommand{\arraystretch}{1.5}
\begin{tabularx}{\textwidth}{|>{\raggedright\arraybackslash}l|X|}
\hline
\hspace{0pt} & v2\_m2 \\
\hline
\end{tabularx}
}

\par


\rule{\linewidth}{0.5pt}


\subsection*{MODULE : mod\_3}
\hypertarget{ecldoc:intest.in1intest.example_2.mod_3}{}
\hyperlink{ecldoc:intest.in1intest.example_2}{Up} :
\hspace{0pt} \hyperlink{ecldoc:intest.in1intest.example_2}{example_2} \textbackslash 

{\renewcommand{\arraystretch}{1.5}
\begin{tabularx}{\textwidth}{|>{\raggedright\arraybackslash}l|X|}
\hline
\hspace{0pt} & mod\_3 \\
\hline
\end{tabularx}
}

\par


\hyperlink{ecldoc:intest.in1intest.example_2.mod_1.v2_m1}{v2\_m1}  |
\hyperlink{ecldoc:intest.in1intest.example_2.mod_2.v2_m2}{v2\_m2}  |
\hyperlink{ecldoc:intest.in1intest.example_2.mod_3.v1_m1}{v1\_m1}  |
\hyperlink{ecldoc:intest.in1intest.example_2.mod_3.v2_m3}{v2\_m3}  |

\rule{\linewidth}{0.5pt}

\subsection*{ATTRIBUTE : v2\_m1}
\hypertarget{ecldoc:intest.in1intest.example_2.mod_1.v2_m1}{}
\hyperlink{ecldoc:intest.in1intest.example_2.mod_3}{Up} :
\hspace{0pt} \hyperlink{ecldoc:intest.in1intest.example_2}{example_2} \textbackslash 
\hspace{0pt} \hyperlink{ecldoc:intest.in1intest.example_2.mod_3}{mod_3} \textbackslash 

{\renewcommand{\arraystretch}{1.5}
\begin{tabularx}{\textwidth}{|>{\raggedright\arraybackslash}l|X|}
\hline
\hspace{0pt} & v2\_m1 \\
\hline
\end{tabularx}
}

\par

\par
\begin{description}
\item [\textbf{INHERITED}] True
\end{description}

\rule{\linewidth}{0.5pt}
\subsection*{ATTRIBUTE : v2\_m2}
\hypertarget{ecldoc:intest.in1intest.example_2.mod_2.v2_m2}{}
\hyperlink{ecldoc:intest.in1intest.example_2.mod_3}{Up} :
\hspace{0pt} \hyperlink{ecldoc:intest.in1intest.example_2}{example_2} \textbackslash 
\hspace{0pt} \hyperlink{ecldoc:intest.in1intest.example_2.mod_3}{mod_3} \textbackslash 

{\renewcommand{\arraystretch}{1.5}
\begin{tabularx}{\textwidth}{|>{\raggedright\arraybackslash}l|X|}
\hline
\hspace{0pt} & v2\_m2 \\
\hline
\end{tabularx}
}

\par

\par
\begin{description}
\item [\textbf{INHERITED}] True
\end{description}

\rule{\linewidth}{0.5pt}
\subsection*{ATTRIBUTE : v1\_m1}
\hypertarget{ecldoc:intest.in1intest.example_2.mod_3.v1_m1}{}
\hyperlink{ecldoc:intest.in1intest.example_2.mod_3}{Up} :
\hspace{0pt} \hyperlink{ecldoc:intest.in1intest.example_2}{example_2} \textbackslash 
\hspace{0pt} \hyperlink{ecldoc:intest.in1intest.example_2.mod_3}{mod_3} \textbackslash 

{\renewcommand{\arraystretch}{1.5}
\begin{tabularx}{\textwidth}{|>{\raggedright\arraybackslash}l|X|}
\hline
\hspace{0pt} & v1\_m1 \\
\hline
\end{tabularx}
}

\par

\par
\begin{description}
\item [\textbf{OVERRIDE}] True
\end{description}

\rule{\linewidth}{0.5pt}
\subsection*{ATTRIBUTE : v2\_m3}
\hypertarget{ecldoc:intest.in1intest.example_2.mod_3.v2_m3}{}
\hyperlink{ecldoc:intest.in1intest.example_2.mod_3}{Up} :
\hspace{0pt} \hyperlink{ecldoc:intest.in1intest.example_2}{example_2} \textbackslash 
\hspace{0pt} \hyperlink{ecldoc:intest.in1intest.example_2.mod_3}{mod_3} \textbackslash 

{\renewcommand{\arraystretch}{1.5}
\begin{tabularx}{\textwidth}{|>{\raggedright\arraybackslash}l|X|}
\hline
\hspace{0pt} & v2\_m3 \\
\hline
\end{tabularx}
}

\par


\rule{\linewidth}{0.5pt}


\subsection*{INTERFACE : iface\_1}
\hypertarget{ecldoc:intest.in1intest.example_2.iface_1}{}
\hyperlink{ecldoc:intest.in1intest.example_2}{Up} :
\hspace{0pt} \hyperlink{ecldoc:intest.in1intest.example_2}{example_2} \textbackslash 

{\renewcommand{\arraystretch}{1.5}
\begin{tabularx}{\textwidth}{|>{\raggedright\arraybackslash}l|X|}
\hline
\hspace{0pt} & iface\_1 \\
\hline
\end{tabularx}
}

\par


\hyperlink{ecldoc:intest.in1intest.example_2.iface_1.v1_i1}{v1\_i1}  |

\rule{\linewidth}{0.5pt}

\subsection*{ATTRIBUTE : v1\_i1}
\hypertarget{ecldoc:intest.in1intest.example_2.iface_1.v1_i1}{}
\hyperlink{ecldoc:intest.in1intest.example_2.iface_1}{Up} :
\hspace{0pt} \hyperlink{ecldoc:intest.in1intest.example_2}{example_2} \textbackslash 
\hspace{0pt} \hyperlink{ecldoc:intest.in1intest.example_2.iface_1}{iface_1} \textbackslash 

{\renewcommand{\arraystretch}{1.5}
\begin{tabularx}{\textwidth}{|>{\raggedright\arraybackslash}l|X|}
\hline
\hspace{0pt}real8 & v1\_i1 \\
\hline
\end{tabularx}
}

\par


\rule{\linewidth}{0.5pt}


\subsection*{MODULE : mod\_4}
\hypertarget{ecldoc:intest.in1intest.example_2.mod_4}{}
\hyperlink{ecldoc:intest.in1intest.example_2}{Up} :
\hspace{0pt} \hyperlink{ecldoc:intest.in1intest.example_2}{example_2} \textbackslash 

{\renewcommand{\arraystretch}{1.5}
\begin{tabularx}{\textwidth}{|>{\raggedright\arraybackslash}l|X|}
\hline
\hspace{0pt} & mod\_4 \\
\hline
\end{tabularx}
}

\par


\hyperlink{ecldoc:intest.in1intest.example_2.mod_4.v1_i1}{v1\_i1}  |
\hyperlink{ecldoc:intest.in1intest.example_2.mod_4.v2_m4}{v2\_m4}  |

\rule{\linewidth}{0.5pt}

\subsection*{ATTRIBUTE : v1\_i1}
\hypertarget{ecldoc:intest.in1intest.example_2.mod_4.v1_i1}{}
\hyperlink{ecldoc:intest.in1intest.example_2.mod_4}{Up} :
\hspace{0pt} \hyperlink{ecldoc:intest.in1intest.example_2}{example_2} \textbackslash 
\hspace{0pt} \hyperlink{ecldoc:intest.in1intest.example_2.mod_4}{mod_4} \textbackslash 

{\renewcommand{\arraystretch}{1.5}
\begin{tabularx}{\textwidth}{|>{\raggedright\arraybackslash}l|X|}
\hline
\hspace{0pt} & v1\_i1 \\
\hline
\end{tabularx}
}

\par

\par
\begin{description}
\item [\textbf{OVERRIDE}] True
\end{description}

\rule{\linewidth}{0.5pt}
\subsection*{ATTRIBUTE : v2\_m4}
\hypertarget{ecldoc:intest.in1intest.example_2.mod_4.v2_m4}{}
\hyperlink{ecldoc:intest.in1intest.example_2.mod_4}{Up} :
\hspace{0pt} \hyperlink{ecldoc:intest.in1intest.example_2}{example_2} \textbackslash 
\hspace{0pt} \hyperlink{ecldoc:intest.in1intest.example_2.mod_4}{mod_4} \textbackslash 

{\renewcommand{\arraystretch}{1.5}
\begin{tabularx}{\textwidth}{|>{\raggedright\arraybackslash}l|X|}
\hline
\hspace{0pt}STRING20 & v2\_m4 \\
\hline
\end{tabularx}
}

\par


\rule{\linewidth}{0.5pt}





\chapter*{intest.inintest.example\_3}

\section*{\underline{IMPORTS}}
\begin{itemize}
\item std.Str
\end{itemize}

\section*{\underline{DESCRIPTIONS}}
\subsection*{MODULE : Example\_3}
\hypertarget{ecldoc:intest.inintest.example_3_intest.inintest.Example_3}{}
Example : Inheritance across files mod\_1 in Example\_4 inherits mod\_1 in Example\_3 \\
\begin{enumerate}
\item \hyperlink{ecldoc:intest.inintest.example_3_intest.inintest.Example_3.mod_1}{mod\_1}
\end{enumerate}
\subsection*{MODULE : mod\_1}
\hypertarget{ecldoc:intest.inintest.example_3_intest.inintest.Example_3.mod_1}{}
\begin{enumerate}
\item \hyperlink{ecldoc:intest.inintest.example_3_intest.inintest.example_3.mod_1.v1_m1}{v1\_m1}
\item \hyperlink{ecldoc:intest.inintest.example_3_intest.inintest.example_3.mod_1.v2_m1_ex3}{v2\_m1\_ex3}
\end{enumerate}
\subsection*{ATTRIBUTE : v1\_m1}
\hypertarget{ecldoc:intest.inintest.example_3_intest.inintest.example_3.mod_1.v1_m1}{}
\subsection*{ATTRIBUTE : v2\_m1\_ex3}
\hypertarget{ecldoc:intest.inintest.example_3_intest.inintest.example_3.mod_1.v2_m1_ex3}{}



\chapter*{example\_4}
\hypertarget{ecldoc:toc:example_4}{}

\section*{\underline{IMPORTS}}
\begin{itemize}
\item Inintest.Example\_3.mod\_1
\end{itemize}

\section*{\underline{DESCRIPTIONS}}
\subsection*{MODULE : example\_4}
\hypertarget{ecldoc:example_4}{}
\hyperlink{ecldoc:toc:root}{Up} :

{\renewcommand{\arraystretch}{1.5}
\begin{tabularx}{\textwidth}{|>{\raggedright\arraybackslash}l|X|}
\hline
\hspace{0pt} & example\_4 \\
\hline
\end{tabularx}
}

\par
Example : Inheritance across files mod\_1 in Example\_4 inherits mod\_1 in Example\_3


\hyperlink{ecldoc:example_4.mod_1}{mod\_1}  |

\rule{\linewidth}{0.5pt}

\subsection*{MODULE : mod\_1}
\hypertarget{ecldoc:example_4.mod_1}{}
\hyperlink{ecldoc:example_4}{Up} :
\hspace{0pt} \hyperlink{ecldoc:example_4}{example_4} \textbackslash 

{\renewcommand{\arraystretch}{1.5}
\begin{tabularx}{\textwidth}{|>{\raggedright\arraybackslash}l|X|}
\hline
\hspace{0pt} & mod\_1 \\
\hline
\end{tabularx}
}

\par


\hyperlink{ecldoc:inintest.example_3.mod_1.v2_m1_ex3}{v2\_m1\_ex3}  |
\hyperlink{ecldoc:example_4.mod_1.v2_m1_ex4}{v2\_m1\_ex4}  |

\rule{\linewidth}{0.5pt}

\subsection*{ATTRIBUTE : v2\_m1\_ex3}
\hypertarget{ecldoc:inintest.example_3.mod_1.v2_m1_ex3}{}
\hyperlink{ecldoc:example_4.mod_1}{Up} :
\hspace{0pt} \hyperlink{ecldoc:example_4}{example_4} \textbackslash 
\hspace{0pt} \hyperlink{ecldoc:example_4.mod_1}{mod_1} \textbackslash 

{\renewcommand{\arraystretch}{1.5}
\begin{tabularx}{\textwidth}{|>{\raggedright\arraybackslash}l|X|}
\hline
\hspace{0pt} & v2\_m1\_ex3 \\
\hline
\end{tabularx}
}

\par

\par
\begin{description}
\item [\textbf{INHERITED}] True
\end{description}

\rule{\linewidth}{0.5pt}
\subsection*{ATTRIBUTE : v2\_m1\_ex4}
\hypertarget{ecldoc:example_4.mod_1.v2_m1_ex4}{}
\hyperlink{ecldoc:example_4.mod_1}{Up} :
\hspace{0pt} \hyperlink{ecldoc:example_4}{example_4} \textbackslash 
\hspace{0pt} \hyperlink{ecldoc:example_4.mod_1}{mod_1} \textbackslash 

{\renewcommand{\arraystretch}{1.5}
\begin{tabularx}{\textwidth}{|>{\raggedright\arraybackslash}l|X|}
\hline
\hspace{0pt} & v2\_m1\_ex4 \\
\hline
\end{tabularx}
}

\par


\rule{\linewidth}{0.5pt}





\chapter*{intest.inintest.example\_5}
\hypertarget{ecldoc:toc:intest.inintest.example_5}{}

\section*{\underline{IMPORTS}}

\section*{\underline{DESCRIPTIONS}}

\chapter*{intest.in1intest.example\_7}

\section*{\underline{IMPORTS}}

\section*{\underline{DESCRIPTIONS}}
\subsection*{MODULE : example\_7}
\hypertarget{ecldoc:intest.in1intest.example_7_intest.in1intest.example_7}{}
Basic Type Example Source Code copied from ECL Documentation \\
\begin{enumerate}
\item \hyperlink{ecldoc:intest.in1intest.example_7_intest.in1intest.example_7.r}{R}
\end{enumerate}
\subsection*{RECORD : R}
\hypertarget{ecldoc:intest.in1intest.example_7_intest.in1intest.example_7.r}{}


\chapter*{intest.in1intest.example\_9}
\hypertarget{ecldoc:toc:intest.in1intest.example_9}{}

\section*{\underline{IMPORTS}}
\begin{itemize}
\item example\_8
\item example\_8.mod\_1
\end{itemize}

\section*{\underline{DESCRIPTIONS}}

\chapter*{\color{headtoc} root}
\hypertarget{ecldoc:toc:root}{}
\hyperlink{ecldoc:toc:}{Go Up}


\section*{Table of Contents}
{\renewcommand{\arraystretch}{1.5}
\begin{longtable}{|p{\textwidth}|}
\hline
\hyperlink{ecldoc:toc:import_test}{import\_test.ecl} \\
This module exists to turn a dataset of numberfields into a dataset of DiscreteFields \\
\hline
\hyperlink{ecldoc:toc:import_test_2}{import\_test\_2.ecl} \\
\hline
\hyperlink{ecldoc:toc:mod_1}{mod\_1.ecl} \\
\hline
\hyperlink{ecldoc:toc:mod_2}{mod\_2.ecl} \\
\hline
\end{longtable}
}

\chapter*{\color{headfile}
import_test
}
\hypertarget{ecldoc:toc:import_test}{}
\hyperlink{ecldoc:toc:root}{Go Up}

\section*{\underline{\textsf{IMPORTS}}}
\begin{doublespace}
{\large
Generate |
ML\_Core.Constants |
std.Str |
ML\_Core |
LogisticRegression |
PBblas.MatUtils |
}
\end{doublespace}

\section*{\underline{\textsf{DESCRIPTIONS}}}
\subsection*{\textsf{\colorbox{headtoc}{\color{white} MODULE}
import\_test}}

\hypertarget{ecldoc:ML_Core.Discretize}{}

{\renewcommand{\arraystretch}{1.5}
\begin{tabularx}{\textwidth}{|>{\raggedright\arraybackslash}l|X|}
\hline
\hspace{0pt}\mytexttt{\color{red} } & \textbf{import\_test} \\
\hline
\end{tabularx}
}

\par
This module exists to turn a dataset of numberfields into a dataset of DiscreteFields. This is not quite as trivial as it seems as there are a number of different ways to make the underlying data discrete; and even within one method there may be different parameters. Further - it is quite probable that different methods are going to be desired for each field.


\textbf{Children}
\begin{enumerate}
\item \hyperlink{ecldoc:ecldoc-c_Method}{c\_Method}
\item \hyperlink{ecldoc:ml_core.discretize.r_method}{r\_Method}
\item \hyperlink{ecldoc:ml_core.discretize.i_byrounding}{i\_ByRounding}
\item \hyperlink{ecldoc:ml_core.discretize.byrounding}{ByRounding}
\item \hyperlink{ecldoc:ml_core.discretize.i_bybucketing}{i\_ByBucketing}
\item \hyperlink{ecldoc:ml_core.discretize.bybucketing}{ByBucketing}
\item \hyperlink{ecldoc:ml_core.discretize.i_bytiling}{i\_ByTiling}
\item \hyperlink{ecldoc:ml_core.discretize.bytiling}{ByTiling}
\item \hyperlink{ecldoc:ml_core.discretize.do}{Do}
\end{enumerate}

\rule{\linewidth}{0.5pt}

\subsection*{\textsf{\colorbox{headtoc}{\color{white} ATTRIBUTE}
c\_Method}}

\hypertarget{ecldoc:ecldoc-c_Method}{}
\hspace{0pt} \hyperlink{ecldoc:ML_Core.Discretize}{import_test} \textbackslash 

{\renewcommand{\arraystretch}{1.5}
\begin{tabularx}{\textwidth}{|>{\raggedright\arraybackslash}l|X|}
\hline
\hspace{0pt}\mytexttt{\color{red} } & \textbf{c\_Method} \\
\hline
\end{tabularx}
}

\par


\rule{\linewidth}{0.5pt}
\subsection*{\textsf{\colorbox{headtoc}{\color{white} RECORD}
r\_Method}}

\hypertarget{ecldoc:ml_core.discretize.r_method}{}
\hspace{0pt} \hyperlink{ecldoc:ML_Core.Discretize}{import_test} \textbackslash 

{\renewcommand{\arraystretch}{1.5}
\begin{tabularx}{\textwidth}{|>{\raggedright\arraybackslash}l|X|}
\hline
\hspace{0pt}\mytexttt{\color{red} } & \textbf{r\_Method} \\
\hline
\end{tabularx}
}

\par


\rule{\linewidth}{0.5pt}
\subsection*{\textsf{\colorbox{headtoc}{\color{white} FUNCTION}
i\_ByRounding}}

\hypertarget{ecldoc:ml_core.discretize.i_byrounding}{}
\hspace{0pt} \hyperlink{ecldoc:ML_Core.Discretize}{import_test} \textbackslash 

{\renewcommand{\arraystretch}{1.5}
\begin{tabularx}{\textwidth}{|>{\raggedright\arraybackslash}l|X|}
\hline
\hspace{0pt}\mytexttt{\color{red} } & \textbf{i\_ByRounding} \\
\hline
\multicolumn{2}{|>{\raggedright\arraybackslash}X|}{\hspace{0pt}\mytexttt{\color{param} (SET OF Types.t\_FieldNumber f, REAL Scale=1.0,REAL Delta=0.0)}} \\
\hline
\end{tabularx}
}

\par


\rule{\linewidth}{0.5pt}
\subsection*{\textsf{\colorbox{headtoc}{\color{white} FUNCTION}
ByRounding}}

\hypertarget{ecldoc:ml_core.discretize.byrounding}{}
\hspace{0pt} \hyperlink{ecldoc:ML_Core.Discretize}{import_test} \textbackslash 

{\renewcommand{\arraystretch}{1.5}
\begin{tabularx}{\textwidth}{|>{\raggedright\arraybackslash}l|X|}
\hline
\hspace{0pt}\mytexttt{\color{red} } & \textbf{ByRounding} \\
\hline
\multicolumn{2}{|>{\raggedright\arraybackslash}X|}{\hspace{0pt}\mytexttt{\color{param} (DATASET(Types.NumericField) d,REAL Scale=1.0, REAL Delta=0.0)}} \\
\hline
\end{tabularx}
}

\par


\rule{\linewidth}{0.5pt}
\subsection*{\textsf{\colorbox{headtoc}{\color{white} FUNCTION}
i\_ByBucketing}}

\hypertarget{ecldoc:ml_core.discretize.i_bybucketing}{}
\hspace{0pt} \hyperlink{ecldoc:ML_Core.Discretize}{import_test} \textbackslash 

{\renewcommand{\arraystretch}{1.5}
\begin{tabularx}{\textwidth}{|>{\raggedright\arraybackslash}l|X|}
\hline
\hspace{0pt}\mytexttt{\color{red} } & \textbf{i\_ByBucketing} \\
\hline
\multicolumn{2}{|>{\raggedright\arraybackslash}X|}{\hspace{0pt}\mytexttt{\color{param} (SET OF Types.t\_FieldNumber f, Types.t\_Discrete N=ML\_Core.Config.Discrete)}} \\
\hline
\end{tabularx}
}

\par


\rule{\linewidth}{0.5pt}
\subsection*{\textsf{\colorbox{headtoc}{\color{white} FUNCTION}
ByBucketing}}

\hypertarget{ecldoc:ml_core.discretize.bybucketing}{}
\hspace{0pt} \hyperlink{ecldoc:ML_Core.Discretize}{import_test} \textbackslash 

{\renewcommand{\arraystretch}{1.5}
\begin{tabularx}{\textwidth}{|>{\raggedright\arraybackslash}l|X|}
\hline
\hspace{0pt}\mytexttt{\color{red} } & \textbf{ByBucketing} \\
\hline
\multicolumn{2}{|>{\raggedright\arraybackslash}X|}{\hspace{0pt}\mytexttt{\color{param} (DATASET(Types.NumericField) d, Types.t\_Discrete N=ML\_Core.Config.Discrete)}} \\
\hline
\end{tabularx}
}

\par


\rule{\linewidth}{0.5pt}
\subsection*{\textsf{\colorbox{headtoc}{\color{white} FUNCTION}
i\_ByTiling}}

\hypertarget{ecldoc:ml_core.discretize.i_bytiling}{}
\hspace{0pt} \hyperlink{ecldoc:ML_Core.Discretize}{import_test} \textbackslash 

{\renewcommand{\arraystretch}{1.5}
\begin{tabularx}{\textwidth}{|>{\raggedright\arraybackslash}l|X|}
\hline
\hspace{0pt}\mytexttt{\color{red} } & \textbf{i\_ByTiling} \\
\hline
\multicolumn{2}{|>{\raggedright\arraybackslash}X|}{\hspace{0pt}\mytexttt{\color{param} (SET OF Types.t\_FieldNumber f, Types.t\_Discrete N=ML\_Core.Config.Discrete)}} \\
\hline
\end{tabularx}
}

\par


\rule{\linewidth}{0.5pt}
\subsection*{\textsf{\colorbox{headtoc}{\color{white} FUNCTION}
ByTiling}}

\hypertarget{ecldoc:ml_core.discretize.bytiling}{}
\hspace{0pt} \hyperlink{ecldoc:ML_Core.Discretize}{import_test} \textbackslash 

{\renewcommand{\arraystretch}{1.5}
\begin{tabularx}{\textwidth}{|>{\raggedright\arraybackslash}l|X|}
\hline
\hspace{0pt}\mytexttt{\color{red} } & \textbf{ByTiling} \\
\hline
\multicolumn{2}{|>{\raggedright\arraybackslash}X|}{\hspace{0pt}\mytexttt{\color{param} (DATASET(Types.NumericField) d, Types.t\_Discrete N=ML\_Core.Config.Discrete)}} \\
\hline
\end{tabularx}
}

\par


\rule{\linewidth}{0.5pt}
\subsection*{\textsf{\colorbox{headtoc}{\color{white} FUNCTION}
Do}}

\hypertarget{ecldoc:ml_core.discretize.do}{}
\hspace{0pt} \hyperlink{ecldoc:ML_Core.Discretize}{import_test} \textbackslash 

{\renewcommand{\arraystretch}{1.5}
\begin{tabularx}{\textwidth}{|>{\raggedright\arraybackslash}l|X|}
\hline
\hspace{0pt}\mytexttt{\color{red} } & \textbf{Do} \\
\hline
\multicolumn{2}{|>{\raggedright\arraybackslash}X|}{\hspace{0pt}\mytexttt{\color{param} (DATASET(Types.NumericField) d, DATASET(r\_Method) to\_do)}} \\
\hline
\end{tabularx}
}

\par


\rule{\linewidth}{0.5pt}



\chapter*{\color{headfile}
import_test_2
}
\hypertarget{ecldoc:toc:import_test_2}{}
\hyperlink{ecldoc:toc:root}{Go Up}

\section*{\underline{\textsf{IMPORTS}}}
\begin{doublespace}
{\large
ML\_Core.Constants |
PBblas |
Example\_3 |
}
\end{doublespace}

\section*{\underline{\textsf{DESCRIPTIONS}}}
\subsection*{\textsf{\colorbox{headtoc}{\color{white} MODULE}
import\_test\_2}}

\hypertarget{ecldoc:import_test_2}{}

{\renewcommand{\arraystretch}{1.5}
\begin{tabularx}{\textwidth}{|>{\raggedright\arraybackslash}l|X|}
\hline
\hspace{0pt}\mytexttt{\color{red} } & \textbf{import\_test\_2} \\
\hline
\end{tabularx}
}

\par





No Documentation Found







\textbf{Children}
\begin{enumerate}
\item \hyperlink{ecldoc:ML_Core.Constants}{nod\_1}
: Useful constants
\item \hyperlink{ecldoc:PBblas}{mod\_1}
: No Documentation Found
\end{enumerate}

\rule{\linewidth}{0.5pt}

\subsection*{\textsf{\colorbox{headtoc}{\color{white} MODULE}
nod\_1}}

\hypertarget{ecldoc:ML_Core.Constants}{}
\hspace{0pt} \hyperlink{ecldoc:import_test_2}{import_test_2} \textbackslash 

{\renewcommand{\arraystretch}{1.5}
\begin{tabularx}{\textwidth}{|>{\raggedright\arraybackslash}l|X|}
\hline
\hspace{0pt}\mytexttt{\color{red} } & \textbf{nod\_1} \\
\hline
\end{tabularx}
}

\par





Useful constants







\textbf{Children}
\begin{enumerate}
\item \hyperlink{ecldoc:ml_core.constants.pi}{Pi}
: Constant PI
\item \hyperlink{ecldoc:ml_core.constants.root_2}{Root\_2}
: Constant square root of 2
\end{enumerate}

\rule{\linewidth}{0.5pt}

\subsection*{\textsf{\colorbox{headtoc}{\color{white} ATTRIBUTE}
Pi}}

\hypertarget{ecldoc:ml_core.constants.pi}{}
\hspace{0pt} \hyperlink{ecldoc:import_test_2}{import_test_2} \textbackslash 
\hspace{0pt} \hyperlink{ecldoc:ML_Core.Constants}{nod_1} \textbackslash 

{\renewcommand{\arraystretch}{1.5}
\begin{tabularx}{\textwidth}{|>{\raggedright\arraybackslash}l|X|}
\hline
\hspace{0pt}\mytexttt{\color{red} } & \textbf{Pi} \\
\hline
\end{tabularx}
}

\par





Constant PI








\par
\begin{description}
\item [\colorbox{tagtype}{\color{white} \textbf{\textsf{RETURN}}}] \textbf{REAL8} --- 
\end{description}




\rule{\linewidth}{0.5pt}
\subsection*{\textsf{\colorbox{headtoc}{\color{white} ATTRIBUTE}
Root\_2}}

\hypertarget{ecldoc:ml_core.constants.root_2}{}
\hspace{0pt} \hyperlink{ecldoc:import_test_2}{import_test_2} \textbackslash 
\hspace{0pt} \hyperlink{ecldoc:ML_Core.Constants}{nod_1} \textbackslash 

{\renewcommand{\arraystretch}{1.5}
\begin{tabularx}{\textwidth}{|>{\raggedright\arraybackslash}l|X|}
\hline
\hspace{0pt}\mytexttt{\color{red} } & \textbf{Root\_2} \\
\hline
\end{tabularx}
}

\par





Constant square root of 2








\par
\begin{description}
\item [\colorbox{tagtype}{\color{white} \textbf{\textsf{RETURN}}}] \textbf{REAL8} --- 
\end{description}




\rule{\linewidth}{0.5pt}


\subsection*{\textsf{\colorbox{headtoc}{\color{white} MODULE}
mod\_1}}

\hypertarget{ecldoc:PBblas}{}
\hspace{0pt} \hyperlink{ecldoc:import_test_2}{import_test_2} \textbackslash 

{\renewcommand{\arraystretch}{1.5}
\begin{tabularx}{\textwidth}{|>{\raggedright\arraybackslash}l|X|}
\hline
\hspace{0pt}\mytexttt{\color{red} } & \textbf{mod\_1} \\
\hline
\end{tabularx}
}

\par





No Documentation Found







\rule{\linewidth}{0.5pt}



\chapter*{\color{headfile}
mod_1
}
\hypertarget{ecldoc:toc:mod_1}{}
\hyperlink{ecldoc:toc:root}{Go Up}


\section*{\underline{\textsf{DESCRIPTIONS}}}
\subsection*{\textsf{\colorbox{headtoc}{\color{white} MODULE}
mod\_1}}

\hypertarget{ecldoc:mod_1}{}

{\renewcommand{\arraystretch}{1.5}
\begin{tabularx}{\textwidth}{|>{\raggedright\arraybackslash}l|X|}
\hline
\hspace{0pt}\mytexttt{\color{red} } & \textbf{mod\_1} \\
\hline
\end{tabularx}
}

\par





No Documentation Found







\textbf{Children}
\begin{enumerate}
\item \hyperlink{ecldoc:mod_1.v1}{v1}
: No Documentation Found
\item \hyperlink{ecldoc:mod_1.m1v4}{m1v4}
: No Documentation Found
\item \hyperlink{ecldoc:mod_1.m1v6}{m1v6}
: No Documentation Found
\end{enumerate}

\rule{\linewidth}{0.5pt}

\subsection*{\textsf{\colorbox{headtoc}{\color{white} ATTRIBUTE}
v1}}

\hypertarget{ecldoc:mod_1.v1}{}
\hspace{0pt} \hyperlink{ecldoc:mod_1}{mod_1} \textbackslash 

{\renewcommand{\arraystretch}{1.5}
\begin{tabularx}{\textwidth}{|>{\raggedright\arraybackslash}l|X|}
\hline
\hspace{0pt}\mytexttt{\color{red} } & \textbf{v1} \\
\hline
\end{tabularx}
}

\par





No Documentation Found








\par
\begin{description}
\item [\colorbox{tagtype}{\color{white} \textbf{\textsf{RETURN}}}] \textbf{REAL8} --- 
\end{description}




\rule{\linewidth}{0.5pt}
\subsection*{\textsf{\colorbox{headtoc}{\color{white} MODULE}
m1v4}}

\hypertarget{ecldoc:mod_1.m1v4}{}
\hspace{0pt} \hyperlink{ecldoc:mod_1}{mod_1} \textbackslash 

{\renewcommand{\arraystretch}{1.5}
\begin{tabularx}{\textwidth}{|>{\raggedright\arraybackslash}l|X|}
\hline
\hspace{0pt}\mytexttt{\color{red} } & \textbf{m1v4} \\
\hline
\multicolumn{2}{|>{\raggedright\arraybackslash}X|}{\hspace{0pt}\mytexttt{\color{param} (REAL8 a1)}} \\
\hline
\end{tabularx}
}

\par





No Documentation Found






\par
\begin{description}
\item [\colorbox{tagtype}{\color{white} \textbf{\textsf{PARAMETER}}}] \textbf{\underline{a1}} ||| REAL8 --- No Doc
\end{description}






\textbf{Children}
\begin{enumerate}
\item \hyperlink{ecldoc:mod_1.m1v4.m1v5}{m1v5}
: No Documentation Found
\end{enumerate}

\rule{\linewidth}{0.5pt}

\subsection*{\textsf{\colorbox{headtoc}{\color{white} ATTRIBUTE}
m1v5}}

\hypertarget{ecldoc:mod_1.m1v4.m1v5}{}
\hspace{0pt} \hyperlink{ecldoc:mod_1}{mod_1} \textbackslash 
\hspace{0pt} \hyperlink{ecldoc:mod_1.m1v4}{m1v4} \textbackslash 

{\renewcommand{\arraystretch}{1.5}
\begin{tabularx}{\textwidth}{|>{\raggedright\arraybackslash}l|X|}
\hline
\hspace{0pt}\mytexttt{\color{red} } & \textbf{m1v5} \\
\hline
\end{tabularx}
}

\par





No Documentation Found








\par
\begin{description}
\item [\colorbox{tagtype}{\color{white} \textbf{\textsf{RETURN}}}] \textbf{REAL8} --- 
\end{description}




\rule{\linewidth}{0.5pt}


\subsection*{\textsf{\colorbox{headtoc}{\color{white} ATTRIBUTE}
m1v6}}

\hypertarget{ecldoc:mod_1.m1v6}{}
\hspace{0pt} \hyperlink{ecldoc:mod_1}{mod_1} \textbackslash 

{\renewcommand{\arraystretch}{1.5}
\begin{tabularx}{\textwidth}{|>{\raggedright\arraybackslash}l|X|}
\hline
\hspace{0pt}\mytexttt{\color{red} } & \textbf{m1v6} \\
\hline
\end{tabularx}
}

\par





No Documentation Found








\par
\begin{description}
\item [\colorbox{tagtype}{\color{white} \textbf{\textsf{RETURN}}}] \textbf{REAL8} --- 
\end{description}




\rule{\linewidth}{0.5pt}



\chapter*{\color{headfile}
mod_2
}
\hypertarget{ecldoc:toc:mod_2}{}
\hyperlink{ecldoc:toc:root}{Go Up}

\section*{\underline{\textsf{IMPORTS}}}
\begin{doublespace}
{\large
mod\_1 |
}
\end{doublespace}

\section*{\underline{\textsf{DESCRIPTIONS}}}
\subsection*{\textsf{\colorbox{headtoc}{\color{white} MODULE}
mod\_2}}

\hypertarget{ecldoc:mod_2}{}

{\renewcommand{\arraystretch}{1.5}
\begin{tabularx}{\textwidth}{|>{\raggedright\arraybackslash}l|X|}
\hline
\hspace{0pt}\mytexttt{\color{red} } & \textbf{mod\_2} \\
\hline
\end{tabularx}
}

\par





No Documentation Found







\textbf{Children}
\begin{enumerate}
\item \hyperlink{ecldoc:mod_1}{v2}
: No Documentation Found
\item \hyperlink{ecldoc:mod_2.v3}{v3}
: No Documentation Found
\item \hyperlink{ecldoc:mod_2.v4}{v4}
: No Documentation Found
\item \hyperlink{ecldoc:mod_2.v5}{v5}
: No Documentation Found
\item \hyperlink{ecldoc:mod_2.v6}{v6}
: No Documentation Found
\end{enumerate}

\rule{\linewidth}{0.5pt}

\subsection*{\textsf{\colorbox{headtoc}{\color{white} MODULE}
v2}}

\hypertarget{ecldoc:mod_1}{}
\hspace{0pt} \hyperlink{ecldoc:mod_2}{mod_2} \textbackslash 

{\renewcommand{\arraystretch}{1.5}
\begin{tabularx}{\textwidth}{|>{\raggedright\arraybackslash}l|X|}
\hline
\hspace{0pt}\mytexttt{\color{red} } & \textbf{v2} \\
\hline
\end{tabularx}
}

\par





No Documentation Found







\textbf{Children}
\begin{enumerate}
\item \hyperlink{ecldoc:mod_1.v1}{v1}
: No Documentation Found
\item \hyperlink{ecldoc:mod_1.m1v4}{m1v4}
: No Documentation Found
\item \hyperlink{ecldoc:mod_1.m1v6}{m1v6}
: No Documentation Found
\end{enumerate}

\rule{\linewidth}{0.5pt}

\subsection*{\textsf{\colorbox{headtoc}{\color{white} ATTRIBUTE}
v1}}

\hypertarget{ecldoc:mod_1.v1}{}
\hspace{0pt} \hyperlink{ecldoc:mod_2}{mod_2} \textbackslash 
\hspace{0pt} \hyperlink{ecldoc:mod_1}{v2} \textbackslash 

{\renewcommand{\arraystretch}{1.5}
\begin{tabularx}{\textwidth}{|>{\raggedright\arraybackslash}l|X|}
\hline
\hspace{0pt}\mytexttt{\color{red} } & \textbf{v1} \\
\hline
\end{tabularx}
}

\par





No Documentation Found








\par
\begin{description}
\item [\colorbox{tagtype}{\color{white} \textbf{\textsf{RETURN}}}] \textbf{REAL8} --- 
\end{description}




\rule{\linewidth}{0.5pt}
\subsection*{\textsf{\colorbox{headtoc}{\color{white} MODULE}
m1v4}}

\hypertarget{ecldoc:mod_1.m1v4}{}
\hspace{0pt} \hyperlink{ecldoc:mod_2}{mod_2} \textbackslash 
\hspace{0pt} \hyperlink{ecldoc:mod_1}{v2} \textbackslash 

{\renewcommand{\arraystretch}{1.5}
\begin{tabularx}{\textwidth}{|>{\raggedright\arraybackslash}l|X|}
\hline
\hspace{0pt}\mytexttt{\color{red} } & \textbf{m1v4} \\
\hline
\multicolumn{2}{|>{\raggedright\arraybackslash}X|}{\hspace{0pt}\mytexttt{\color{param} (REAL8 a1)}} \\
\hline
\end{tabularx}
}

\par





No Documentation Found






\par
\begin{description}
\item [\colorbox{tagtype}{\color{white} \textbf{\textsf{PARAMETER}}}] \textbf{\underline{a1}} ||| REAL8 --- No Doc
\end{description}






\textbf{Children}
\begin{enumerate}
\item \hyperlink{ecldoc:mod_1.m1v4.m1v5}{m1v5}
: No Documentation Found
\end{enumerate}

\rule{\linewidth}{0.5pt}

\subsection*{\textsf{\colorbox{headtoc}{\color{white} ATTRIBUTE}
m1v5}}

\hypertarget{ecldoc:mod_1.m1v4.m1v5}{}
\hspace{0pt} \hyperlink{ecldoc:mod_2}{mod_2} \textbackslash 
\hspace{0pt} \hyperlink{ecldoc:mod_1}{v2} \textbackslash 
\hspace{0pt} \hyperlink{ecldoc:mod_1.m1v4}{m1v4} \textbackslash 

{\renewcommand{\arraystretch}{1.5}
\begin{tabularx}{\textwidth}{|>{\raggedright\arraybackslash}l|X|}
\hline
\hspace{0pt}\mytexttt{\color{red} } & \textbf{m1v5} \\
\hline
\end{tabularx}
}

\par





No Documentation Found








\par
\begin{description}
\item [\colorbox{tagtype}{\color{white} \textbf{\textsf{RETURN}}}] \textbf{REAL8} --- 
\end{description}




\rule{\linewidth}{0.5pt}


\subsection*{\textsf{\colorbox{headtoc}{\color{white} ATTRIBUTE}
m1v6}}

\hypertarget{ecldoc:mod_1.m1v6}{}
\hspace{0pt} \hyperlink{ecldoc:mod_2}{mod_2} \textbackslash 
\hspace{0pt} \hyperlink{ecldoc:mod_1}{v2} \textbackslash 

{\renewcommand{\arraystretch}{1.5}
\begin{tabularx}{\textwidth}{|>{\raggedright\arraybackslash}l|X|}
\hline
\hspace{0pt}\mytexttt{\color{red} } & \textbf{m1v6} \\
\hline
\end{tabularx}
}

\par





No Documentation Found








\par
\begin{description}
\item [\colorbox{tagtype}{\color{white} \textbf{\textsf{RETURN}}}] \textbf{REAL8} --- 
\end{description}




\rule{\linewidth}{0.5pt}


\subsection*{\textsf{\colorbox{headtoc}{\color{white} ATTRIBUTE}
v3}}

\hypertarget{ecldoc:mod_2.v3}{}
\hspace{0pt} \hyperlink{ecldoc:mod_2}{mod_2} \textbackslash 

{\renewcommand{\arraystretch}{1.5}
\begin{tabularx}{\textwidth}{|>{\raggedright\arraybackslash}l|X|}
\hline
\hspace{0pt}\mytexttt{\color{red} } & \textbf{v3} \\
\hline
\end{tabularx}
}

\par





No Documentation Found








\par
\begin{description}
\item [\colorbox{tagtype}{\color{white} \textbf{\textsf{RETURN}}}] \textbf{REAL8} --- 
\end{description}




\rule{\linewidth}{0.5pt}
\subsection*{\textsf{\colorbox{headtoc}{\color{white} MODULE}
v4}}

\hypertarget{ecldoc:mod_2.v4}{}
\hspace{0pt} \hyperlink{ecldoc:mod_2}{mod_2} \textbackslash 

{\renewcommand{\arraystretch}{1.5}
\begin{tabularx}{\textwidth}{|>{\raggedright\arraybackslash}l|X|}
\hline
\hspace{0pt}\mytexttt{\color{red} } & \textbf{v4} \\
\hline
\multicolumn{2}{|>{\raggedright\arraybackslash}X|}{\hspace{0pt}\mytexttt{\color{param} (REAL8 a2)}} \\
\hline
\end{tabularx}
}

\par





No Documentation Found






\par
\begin{description}
\item [\colorbox{tagtype}{\color{white} \textbf{\textsf{PARAMETER}}}] \textbf{\underline{a2}} ||| REAL8 --- No Doc
\end{description}






\textbf{Children}
\begin{enumerate}
\item \hyperlink{ecldoc:mod_1.m1v4.m1v5}{m1v5}
: No Documentation Found
\end{enumerate}

\rule{\linewidth}{0.5pt}

\subsection*{\textsf{\colorbox{headtoc}{\color{white} ATTRIBUTE}
m1v5}}

\hypertarget{ecldoc:mod_1.m1v4.m1v5}{}
\hspace{0pt} \hyperlink{ecldoc:mod_2}{mod_2} \textbackslash 
\hspace{0pt} \hyperlink{ecldoc:mod_2.v4}{v4} \textbackslash 

{\renewcommand{\arraystretch}{1.5}
\begin{tabularx}{\textwidth}{|>{\raggedright\arraybackslash}l|X|}
\hline
\hspace{0pt}\mytexttt{\color{red} } & \textbf{m1v5} \\
\hline
\end{tabularx}
}

\par





No Documentation Found








\par
\begin{description}
\item [\colorbox{tagtype}{\color{white} \textbf{\textsf{RETURN}}}] \textbf{REAL8} --- 
\end{description}




\rule{\linewidth}{0.5pt}


\subsection*{\textsf{\colorbox{headtoc}{\color{white} ATTRIBUTE}
v5}}

\hypertarget{ecldoc:mod_2.v5}{}
\hspace{0pt} \hyperlink{ecldoc:mod_2}{mod_2} \textbackslash 

{\renewcommand{\arraystretch}{1.5}
\begin{tabularx}{\textwidth}{|>{\raggedright\arraybackslash}l|X|}
\hline
\hspace{0pt}\mytexttt{\color{red} } & \textbf{v5} \\
\hline
\end{tabularx}
}

\par





No Documentation Found








\par
\begin{description}
\item [\colorbox{tagtype}{\color{white} \textbf{\textsf{RETURN}}}] \textbf{REAL8} --- 
\end{description}




\rule{\linewidth}{0.5pt}
\subsection*{\textsf{\colorbox{headtoc}{\color{white} ATTRIBUTE}
v6}}

\hypertarget{ecldoc:mod_2.v6}{}
\hspace{0pt} \hyperlink{ecldoc:mod_2}{mod_2} \textbackslash 

{\renewcommand{\arraystretch}{1.5}
\begin{tabularx}{\textwidth}{|>{\raggedright\arraybackslash}l|X|}
\hline
\hspace{0pt}\mytexttt{\color{red} } & \textbf{v6} \\
\hline
\end{tabularx}
}

\par





No Documentation Found








\par
\begin{description}
\item [\colorbox{tagtype}{\color{white} \textbf{\textsf{RETURN}}}] \textbf{REAL8} --- 
\end{description}




\rule{\linewidth}{0.5pt}




\chapter*{\color{headtoc} root}
\hypertarget{ecldoc:toc:root}{}
\hyperlink{ecldoc:toc:}{Go Up}


\section*{Table of Contents}
{\renewcommand{\arraystretch}{1.5}
\begin{longtable}{|p{\textwidth}|}
\hline
\hyperlink{ecldoc:toc:import_test}{import\_test.ecl} \\
This module exists to turn a dataset of numberfields into a dataset of DiscreteFields \\
\hline
\hyperlink{ecldoc:toc:import_test_2}{import\_test\_2.ecl} \\
\hline
\hyperlink{ecldoc:toc:mod_1}{mod\_1.ecl} \\
\hline
\hyperlink{ecldoc:toc:mod_2}{mod\_2.ecl} \\
\hline
\end{longtable}
}

\chapter*{\color{headfile}
import_test
}
\hypertarget{ecldoc:toc:import_test}{}
\hyperlink{ecldoc:toc:root}{Go Up}

\section*{\underline{\textsf{IMPORTS}}}
\begin{doublespace}
{\large
Generate |
ML\_Core.Constants |
std.Str |
ML\_Core |
LogisticRegression |
PBblas.MatUtils |
}
\end{doublespace}

\section*{\underline{\textsf{DESCRIPTIONS}}}
\subsection*{\textsf{\colorbox{headtoc}{\color{white} MODULE}
import\_test}}

\hypertarget{ecldoc:ML_Core.Discretize}{}

{\renewcommand{\arraystretch}{1.5}
\begin{tabularx}{\textwidth}{|>{\raggedright\arraybackslash}l|X|}
\hline
\hspace{0pt}\mytexttt{\color{red} } & \textbf{import\_test} \\
\hline
\end{tabularx}
}

\par
This module exists to turn a dataset of numberfields into a dataset of DiscreteFields. This is not quite as trivial as it seems as there are a number of different ways to make the underlying data discrete; and even within one method there may be different parameters. Further - it is quite probable that different methods are going to be desired for each field.


\textbf{Children}
\begin{enumerate}
\item \hyperlink{ecldoc:ecldoc-c_Method}{c\_Method}
\item \hyperlink{ecldoc:ml_core.discretize.r_method}{r\_Method}
\item \hyperlink{ecldoc:ml_core.discretize.i_byrounding}{i\_ByRounding}
\item \hyperlink{ecldoc:ml_core.discretize.byrounding}{ByRounding}
\item \hyperlink{ecldoc:ml_core.discretize.i_bybucketing}{i\_ByBucketing}
\item \hyperlink{ecldoc:ml_core.discretize.bybucketing}{ByBucketing}
\item \hyperlink{ecldoc:ml_core.discretize.i_bytiling}{i\_ByTiling}
\item \hyperlink{ecldoc:ml_core.discretize.bytiling}{ByTiling}
\item \hyperlink{ecldoc:ml_core.discretize.do}{Do}
\end{enumerate}

\rule{\linewidth}{0.5pt}

\subsection*{\textsf{\colorbox{headtoc}{\color{white} ATTRIBUTE}
c\_Method}}

\hypertarget{ecldoc:ecldoc-c_Method}{}
\hspace{0pt} \hyperlink{ecldoc:ML_Core.Discretize}{import_test} \textbackslash 

{\renewcommand{\arraystretch}{1.5}
\begin{tabularx}{\textwidth}{|>{\raggedright\arraybackslash}l|X|}
\hline
\hspace{0pt}\mytexttt{\color{red} } & \textbf{c\_Method} \\
\hline
\end{tabularx}
}

\par


\rule{\linewidth}{0.5pt}
\subsection*{\textsf{\colorbox{headtoc}{\color{white} RECORD}
r\_Method}}

\hypertarget{ecldoc:ml_core.discretize.r_method}{}
\hspace{0pt} \hyperlink{ecldoc:ML_Core.Discretize}{import_test} \textbackslash 

{\renewcommand{\arraystretch}{1.5}
\begin{tabularx}{\textwidth}{|>{\raggedright\arraybackslash}l|X|}
\hline
\hspace{0pt}\mytexttt{\color{red} } & \textbf{r\_Method} \\
\hline
\end{tabularx}
}

\par


\rule{\linewidth}{0.5pt}
\subsection*{\textsf{\colorbox{headtoc}{\color{white} FUNCTION}
i\_ByRounding}}

\hypertarget{ecldoc:ml_core.discretize.i_byrounding}{}
\hspace{0pt} \hyperlink{ecldoc:ML_Core.Discretize}{import_test} \textbackslash 

{\renewcommand{\arraystretch}{1.5}
\begin{tabularx}{\textwidth}{|>{\raggedright\arraybackslash}l|X|}
\hline
\hspace{0pt}\mytexttt{\color{red} } & \textbf{i\_ByRounding} \\
\hline
\multicolumn{2}{|>{\raggedright\arraybackslash}X|}{\hspace{0pt}\mytexttt{\color{param} (SET OF Types.t\_FieldNumber f, REAL Scale=1.0,REAL Delta=0.0)}} \\
\hline
\end{tabularx}
}

\par


\rule{\linewidth}{0.5pt}
\subsection*{\textsf{\colorbox{headtoc}{\color{white} FUNCTION}
ByRounding}}

\hypertarget{ecldoc:ml_core.discretize.byrounding}{}
\hspace{0pt} \hyperlink{ecldoc:ML_Core.Discretize}{import_test} \textbackslash 

{\renewcommand{\arraystretch}{1.5}
\begin{tabularx}{\textwidth}{|>{\raggedright\arraybackslash}l|X|}
\hline
\hspace{0pt}\mytexttt{\color{red} } & \textbf{ByRounding} \\
\hline
\multicolumn{2}{|>{\raggedright\arraybackslash}X|}{\hspace{0pt}\mytexttt{\color{param} (DATASET(Types.NumericField) d,REAL Scale=1.0, REAL Delta=0.0)}} \\
\hline
\end{tabularx}
}

\par


\rule{\linewidth}{0.5pt}
\subsection*{\textsf{\colorbox{headtoc}{\color{white} FUNCTION}
i\_ByBucketing}}

\hypertarget{ecldoc:ml_core.discretize.i_bybucketing}{}
\hspace{0pt} \hyperlink{ecldoc:ML_Core.Discretize}{import_test} \textbackslash 

{\renewcommand{\arraystretch}{1.5}
\begin{tabularx}{\textwidth}{|>{\raggedright\arraybackslash}l|X|}
\hline
\hspace{0pt}\mytexttt{\color{red} } & \textbf{i\_ByBucketing} \\
\hline
\multicolumn{2}{|>{\raggedright\arraybackslash}X|}{\hspace{0pt}\mytexttt{\color{param} (SET OF Types.t\_FieldNumber f, Types.t\_Discrete N=ML\_Core.Config.Discrete)}} \\
\hline
\end{tabularx}
}

\par


\rule{\linewidth}{0.5pt}
\subsection*{\textsf{\colorbox{headtoc}{\color{white} FUNCTION}
ByBucketing}}

\hypertarget{ecldoc:ml_core.discretize.bybucketing}{}
\hspace{0pt} \hyperlink{ecldoc:ML_Core.Discretize}{import_test} \textbackslash 

{\renewcommand{\arraystretch}{1.5}
\begin{tabularx}{\textwidth}{|>{\raggedright\arraybackslash}l|X|}
\hline
\hspace{0pt}\mytexttt{\color{red} } & \textbf{ByBucketing} \\
\hline
\multicolumn{2}{|>{\raggedright\arraybackslash}X|}{\hspace{0pt}\mytexttt{\color{param} (DATASET(Types.NumericField) d, Types.t\_Discrete N=ML\_Core.Config.Discrete)}} \\
\hline
\end{tabularx}
}

\par


\rule{\linewidth}{0.5pt}
\subsection*{\textsf{\colorbox{headtoc}{\color{white} FUNCTION}
i\_ByTiling}}

\hypertarget{ecldoc:ml_core.discretize.i_bytiling}{}
\hspace{0pt} \hyperlink{ecldoc:ML_Core.Discretize}{import_test} \textbackslash 

{\renewcommand{\arraystretch}{1.5}
\begin{tabularx}{\textwidth}{|>{\raggedright\arraybackslash}l|X|}
\hline
\hspace{0pt}\mytexttt{\color{red} } & \textbf{i\_ByTiling} \\
\hline
\multicolumn{2}{|>{\raggedright\arraybackslash}X|}{\hspace{0pt}\mytexttt{\color{param} (SET OF Types.t\_FieldNumber f, Types.t\_Discrete N=ML\_Core.Config.Discrete)}} \\
\hline
\end{tabularx}
}

\par


\rule{\linewidth}{0.5pt}
\subsection*{\textsf{\colorbox{headtoc}{\color{white} FUNCTION}
ByTiling}}

\hypertarget{ecldoc:ml_core.discretize.bytiling}{}
\hspace{0pt} \hyperlink{ecldoc:ML_Core.Discretize}{import_test} \textbackslash 

{\renewcommand{\arraystretch}{1.5}
\begin{tabularx}{\textwidth}{|>{\raggedright\arraybackslash}l|X|}
\hline
\hspace{0pt}\mytexttt{\color{red} } & \textbf{ByTiling} \\
\hline
\multicolumn{2}{|>{\raggedright\arraybackslash}X|}{\hspace{0pt}\mytexttt{\color{param} (DATASET(Types.NumericField) d, Types.t\_Discrete N=ML\_Core.Config.Discrete)}} \\
\hline
\end{tabularx}
}

\par


\rule{\linewidth}{0.5pt}
\subsection*{\textsf{\colorbox{headtoc}{\color{white} FUNCTION}
Do}}

\hypertarget{ecldoc:ml_core.discretize.do}{}
\hspace{0pt} \hyperlink{ecldoc:ML_Core.Discretize}{import_test} \textbackslash 

{\renewcommand{\arraystretch}{1.5}
\begin{tabularx}{\textwidth}{|>{\raggedright\arraybackslash}l|X|}
\hline
\hspace{0pt}\mytexttt{\color{red} } & \textbf{Do} \\
\hline
\multicolumn{2}{|>{\raggedright\arraybackslash}X|}{\hspace{0pt}\mytexttt{\color{param} (DATASET(Types.NumericField) d, DATASET(r\_Method) to\_do)}} \\
\hline
\end{tabularx}
}

\par


\rule{\linewidth}{0.5pt}



\chapter*{\color{headfile}
import_test_2
}
\hypertarget{ecldoc:toc:import_test_2}{}
\hyperlink{ecldoc:toc:root}{Go Up}

\section*{\underline{\textsf{IMPORTS}}}
\begin{doublespace}
{\large
ML\_Core.Constants |
PBblas |
Example\_3 |
}
\end{doublespace}

\section*{\underline{\textsf{DESCRIPTIONS}}}
\subsection*{\textsf{\colorbox{headtoc}{\color{white} MODULE}
import\_test\_2}}

\hypertarget{ecldoc:import_test_2}{}

{\renewcommand{\arraystretch}{1.5}
\begin{tabularx}{\textwidth}{|>{\raggedright\arraybackslash}l|X|}
\hline
\hspace{0pt}\mytexttt{\color{red} } & \textbf{import\_test\_2} \\
\hline
\end{tabularx}
}

\par





No Documentation Found







\textbf{Children}
\begin{enumerate}
\item \hyperlink{ecldoc:ML_Core.Constants}{nod\_1}
: Useful constants
\item \hyperlink{ecldoc:PBblas}{mod\_1}
: No Documentation Found
\end{enumerate}

\rule{\linewidth}{0.5pt}

\subsection*{\textsf{\colorbox{headtoc}{\color{white} MODULE}
nod\_1}}

\hypertarget{ecldoc:ML_Core.Constants}{}
\hspace{0pt} \hyperlink{ecldoc:import_test_2}{import_test_2} \textbackslash 

{\renewcommand{\arraystretch}{1.5}
\begin{tabularx}{\textwidth}{|>{\raggedright\arraybackslash}l|X|}
\hline
\hspace{0pt}\mytexttt{\color{red} } & \textbf{nod\_1} \\
\hline
\end{tabularx}
}

\par





Useful constants







\textbf{Children}
\begin{enumerate}
\item \hyperlink{ecldoc:ml_core.constants.pi}{Pi}
: Constant PI
\item \hyperlink{ecldoc:ml_core.constants.root_2}{Root\_2}
: Constant square root of 2
\end{enumerate}

\rule{\linewidth}{0.5pt}

\subsection*{\textsf{\colorbox{headtoc}{\color{white} ATTRIBUTE}
Pi}}

\hypertarget{ecldoc:ml_core.constants.pi}{}
\hspace{0pt} \hyperlink{ecldoc:import_test_2}{import_test_2} \textbackslash 
\hspace{0pt} \hyperlink{ecldoc:ML_Core.Constants}{nod_1} \textbackslash 

{\renewcommand{\arraystretch}{1.5}
\begin{tabularx}{\textwidth}{|>{\raggedright\arraybackslash}l|X|}
\hline
\hspace{0pt}\mytexttt{\color{red} } & \textbf{Pi} \\
\hline
\end{tabularx}
}

\par





Constant PI








\par
\begin{description}
\item [\colorbox{tagtype}{\color{white} \textbf{\textsf{RETURN}}}] \textbf{REAL8} --- 
\end{description}




\rule{\linewidth}{0.5pt}
\subsection*{\textsf{\colorbox{headtoc}{\color{white} ATTRIBUTE}
Root\_2}}

\hypertarget{ecldoc:ml_core.constants.root_2}{}
\hspace{0pt} \hyperlink{ecldoc:import_test_2}{import_test_2} \textbackslash 
\hspace{0pt} \hyperlink{ecldoc:ML_Core.Constants}{nod_1} \textbackslash 

{\renewcommand{\arraystretch}{1.5}
\begin{tabularx}{\textwidth}{|>{\raggedright\arraybackslash}l|X|}
\hline
\hspace{0pt}\mytexttt{\color{red} } & \textbf{Root\_2} \\
\hline
\end{tabularx}
}

\par





Constant square root of 2








\par
\begin{description}
\item [\colorbox{tagtype}{\color{white} \textbf{\textsf{RETURN}}}] \textbf{REAL8} --- 
\end{description}




\rule{\linewidth}{0.5pt}


\subsection*{\textsf{\colorbox{headtoc}{\color{white} MODULE}
mod\_1}}

\hypertarget{ecldoc:PBblas}{}
\hspace{0pt} \hyperlink{ecldoc:import_test_2}{import_test_2} \textbackslash 

{\renewcommand{\arraystretch}{1.5}
\begin{tabularx}{\textwidth}{|>{\raggedright\arraybackslash}l|X|}
\hline
\hspace{0pt}\mytexttt{\color{red} } & \textbf{mod\_1} \\
\hline
\end{tabularx}
}

\par





No Documentation Found







\rule{\linewidth}{0.5pt}



\chapter*{\color{headfile}
mod_1
}
\hypertarget{ecldoc:toc:mod_1}{}
\hyperlink{ecldoc:toc:root}{Go Up}


\section*{\underline{\textsf{DESCRIPTIONS}}}
\subsection*{\textsf{\colorbox{headtoc}{\color{white} MODULE}
mod\_1}}

\hypertarget{ecldoc:mod_1}{}

{\renewcommand{\arraystretch}{1.5}
\begin{tabularx}{\textwidth}{|>{\raggedright\arraybackslash}l|X|}
\hline
\hspace{0pt}\mytexttt{\color{red} } & \textbf{mod\_1} \\
\hline
\end{tabularx}
}

\par





No Documentation Found







\textbf{Children}
\begin{enumerate}
\item \hyperlink{ecldoc:mod_1.v1}{v1}
: No Documentation Found
\item \hyperlink{ecldoc:mod_1.m1v4}{m1v4}
: No Documentation Found
\item \hyperlink{ecldoc:mod_1.m1v6}{m1v6}
: No Documentation Found
\end{enumerate}

\rule{\linewidth}{0.5pt}

\subsection*{\textsf{\colorbox{headtoc}{\color{white} ATTRIBUTE}
v1}}

\hypertarget{ecldoc:mod_1.v1}{}
\hspace{0pt} \hyperlink{ecldoc:mod_1}{mod_1} \textbackslash 

{\renewcommand{\arraystretch}{1.5}
\begin{tabularx}{\textwidth}{|>{\raggedright\arraybackslash}l|X|}
\hline
\hspace{0pt}\mytexttt{\color{red} } & \textbf{v1} \\
\hline
\end{tabularx}
}

\par





No Documentation Found








\par
\begin{description}
\item [\colorbox{tagtype}{\color{white} \textbf{\textsf{RETURN}}}] \textbf{REAL8} --- 
\end{description}




\rule{\linewidth}{0.5pt}
\subsection*{\textsf{\colorbox{headtoc}{\color{white} MODULE}
m1v4}}

\hypertarget{ecldoc:mod_1.m1v4}{}
\hspace{0pt} \hyperlink{ecldoc:mod_1}{mod_1} \textbackslash 

{\renewcommand{\arraystretch}{1.5}
\begin{tabularx}{\textwidth}{|>{\raggedright\arraybackslash}l|X|}
\hline
\hspace{0pt}\mytexttt{\color{red} } & \textbf{m1v4} \\
\hline
\multicolumn{2}{|>{\raggedright\arraybackslash}X|}{\hspace{0pt}\mytexttt{\color{param} (REAL8 a1)}} \\
\hline
\end{tabularx}
}

\par





No Documentation Found






\par
\begin{description}
\item [\colorbox{tagtype}{\color{white} \textbf{\textsf{PARAMETER}}}] \textbf{\underline{a1}} ||| REAL8 --- No Doc
\end{description}






\textbf{Children}
\begin{enumerate}
\item \hyperlink{ecldoc:mod_1.m1v4.m1v5}{m1v5}
: No Documentation Found
\end{enumerate}

\rule{\linewidth}{0.5pt}

\subsection*{\textsf{\colorbox{headtoc}{\color{white} ATTRIBUTE}
m1v5}}

\hypertarget{ecldoc:mod_1.m1v4.m1v5}{}
\hspace{0pt} \hyperlink{ecldoc:mod_1}{mod_1} \textbackslash 
\hspace{0pt} \hyperlink{ecldoc:mod_1.m1v4}{m1v4} \textbackslash 

{\renewcommand{\arraystretch}{1.5}
\begin{tabularx}{\textwidth}{|>{\raggedright\arraybackslash}l|X|}
\hline
\hspace{0pt}\mytexttt{\color{red} } & \textbf{m1v5} \\
\hline
\end{tabularx}
}

\par





No Documentation Found








\par
\begin{description}
\item [\colorbox{tagtype}{\color{white} \textbf{\textsf{RETURN}}}] \textbf{REAL8} --- 
\end{description}




\rule{\linewidth}{0.5pt}


\subsection*{\textsf{\colorbox{headtoc}{\color{white} ATTRIBUTE}
m1v6}}

\hypertarget{ecldoc:mod_1.m1v6}{}
\hspace{0pt} \hyperlink{ecldoc:mod_1}{mod_1} \textbackslash 

{\renewcommand{\arraystretch}{1.5}
\begin{tabularx}{\textwidth}{|>{\raggedright\arraybackslash}l|X|}
\hline
\hspace{0pt}\mytexttt{\color{red} } & \textbf{m1v6} \\
\hline
\end{tabularx}
}

\par





No Documentation Found








\par
\begin{description}
\item [\colorbox{tagtype}{\color{white} \textbf{\textsf{RETURN}}}] \textbf{REAL8} --- 
\end{description}




\rule{\linewidth}{0.5pt}



\chapter*{\color{headfile}
mod_2
}
\hypertarget{ecldoc:toc:mod_2}{}
\hyperlink{ecldoc:toc:root}{Go Up}

\section*{\underline{\textsf{IMPORTS}}}
\begin{doublespace}
{\large
mod\_1 |
}
\end{doublespace}

\section*{\underline{\textsf{DESCRIPTIONS}}}
\subsection*{\textsf{\colorbox{headtoc}{\color{white} MODULE}
mod\_2}}

\hypertarget{ecldoc:mod_2}{}

{\renewcommand{\arraystretch}{1.5}
\begin{tabularx}{\textwidth}{|>{\raggedright\arraybackslash}l|X|}
\hline
\hspace{0pt}\mytexttt{\color{red} } & \textbf{mod\_2} \\
\hline
\end{tabularx}
}

\par





No Documentation Found







\textbf{Children}
\begin{enumerate}
\item \hyperlink{ecldoc:mod_1}{v2}
: No Documentation Found
\item \hyperlink{ecldoc:mod_2.v3}{v3}
: No Documentation Found
\item \hyperlink{ecldoc:mod_2.v4}{v4}
: No Documentation Found
\item \hyperlink{ecldoc:mod_2.v5}{v5}
: No Documentation Found
\item \hyperlink{ecldoc:mod_2.v6}{v6}
: No Documentation Found
\end{enumerate}

\rule{\linewidth}{0.5pt}

\subsection*{\textsf{\colorbox{headtoc}{\color{white} MODULE}
v2}}

\hypertarget{ecldoc:mod_1}{}
\hspace{0pt} \hyperlink{ecldoc:mod_2}{mod_2} \textbackslash 

{\renewcommand{\arraystretch}{1.5}
\begin{tabularx}{\textwidth}{|>{\raggedright\arraybackslash}l|X|}
\hline
\hspace{0pt}\mytexttt{\color{red} } & \textbf{v2} \\
\hline
\end{tabularx}
}

\par





No Documentation Found







\textbf{Children}
\begin{enumerate}
\item \hyperlink{ecldoc:mod_1.v1}{v1}
: No Documentation Found
\item \hyperlink{ecldoc:mod_1.m1v4}{m1v4}
: No Documentation Found
\item \hyperlink{ecldoc:mod_1.m1v6}{m1v6}
: No Documentation Found
\end{enumerate}

\rule{\linewidth}{0.5pt}

\subsection*{\textsf{\colorbox{headtoc}{\color{white} ATTRIBUTE}
v1}}

\hypertarget{ecldoc:mod_1.v1}{}
\hspace{0pt} \hyperlink{ecldoc:mod_2}{mod_2} \textbackslash 
\hspace{0pt} \hyperlink{ecldoc:mod_1}{v2} \textbackslash 

{\renewcommand{\arraystretch}{1.5}
\begin{tabularx}{\textwidth}{|>{\raggedright\arraybackslash}l|X|}
\hline
\hspace{0pt}\mytexttt{\color{red} } & \textbf{v1} \\
\hline
\end{tabularx}
}

\par





No Documentation Found








\par
\begin{description}
\item [\colorbox{tagtype}{\color{white} \textbf{\textsf{RETURN}}}] \textbf{REAL8} --- 
\end{description}




\rule{\linewidth}{0.5pt}
\subsection*{\textsf{\colorbox{headtoc}{\color{white} MODULE}
m1v4}}

\hypertarget{ecldoc:mod_1.m1v4}{}
\hspace{0pt} \hyperlink{ecldoc:mod_2}{mod_2} \textbackslash 
\hspace{0pt} \hyperlink{ecldoc:mod_1}{v2} \textbackslash 

{\renewcommand{\arraystretch}{1.5}
\begin{tabularx}{\textwidth}{|>{\raggedright\arraybackslash}l|X|}
\hline
\hspace{0pt}\mytexttt{\color{red} } & \textbf{m1v4} \\
\hline
\multicolumn{2}{|>{\raggedright\arraybackslash}X|}{\hspace{0pt}\mytexttt{\color{param} (REAL8 a1)}} \\
\hline
\end{tabularx}
}

\par





No Documentation Found






\par
\begin{description}
\item [\colorbox{tagtype}{\color{white} \textbf{\textsf{PARAMETER}}}] \textbf{\underline{a1}} ||| REAL8 --- No Doc
\end{description}






\textbf{Children}
\begin{enumerate}
\item \hyperlink{ecldoc:mod_1.m1v4.m1v5}{m1v5}
: No Documentation Found
\end{enumerate}

\rule{\linewidth}{0.5pt}

\subsection*{\textsf{\colorbox{headtoc}{\color{white} ATTRIBUTE}
m1v5}}

\hypertarget{ecldoc:mod_1.m1v4.m1v5}{}
\hspace{0pt} \hyperlink{ecldoc:mod_2}{mod_2} \textbackslash 
\hspace{0pt} \hyperlink{ecldoc:mod_1}{v2} \textbackslash 
\hspace{0pt} \hyperlink{ecldoc:mod_1.m1v4}{m1v4} \textbackslash 

{\renewcommand{\arraystretch}{1.5}
\begin{tabularx}{\textwidth}{|>{\raggedright\arraybackslash}l|X|}
\hline
\hspace{0pt}\mytexttt{\color{red} } & \textbf{m1v5} \\
\hline
\end{tabularx}
}

\par





No Documentation Found








\par
\begin{description}
\item [\colorbox{tagtype}{\color{white} \textbf{\textsf{RETURN}}}] \textbf{REAL8} --- 
\end{description}




\rule{\linewidth}{0.5pt}


\subsection*{\textsf{\colorbox{headtoc}{\color{white} ATTRIBUTE}
m1v6}}

\hypertarget{ecldoc:mod_1.m1v6}{}
\hspace{0pt} \hyperlink{ecldoc:mod_2}{mod_2} \textbackslash 
\hspace{0pt} \hyperlink{ecldoc:mod_1}{v2} \textbackslash 

{\renewcommand{\arraystretch}{1.5}
\begin{tabularx}{\textwidth}{|>{\raggedright\arraybackslash}l|X|}
\hline
\hspace{0pt}\mytexttt{\color{red} } & \textbf{m1v6} \\
\hline
\end{tabularx}
}

\par





No Documentation Found








\par
\begin{description}
\item [\colorbox{tagtype}{\color{white} \textbf{\textsf{RETURN}}}] \textbf{REAL8} --- 
\end{description}




\rule{\linewidth}{0.5pt}


\subsection*{\textsf{\colorbox{headtoc}{\color{white} ATTRIBUTE}
v3}}

\hypertarget{ecldoc:mod_2.v3}{}
\hspace{0pt} \hyperlink{ecldoc:mod_2}{mod_2} \textbackslash 

{\renewcommand{\arraystretch}{1.5}
\begin{tabularx}{\textwidth}{|>{\raggedright\arraybackslash}l|X|}
\hline
\hspace{0pt}\mytexttt{\color{red} } & \textbf{v3} \\
\hline
\end{tabularx}
}

\par





No Documentation Found








\par
\begin{description}
\item [\colorbox{tagtype}{\color{white} \textbf{\textsf{RETURN}}}] \textbf{REAL8} --- 
\end{description}




\rule{\linewidth}{0.5pt}
\subsection*{\textsf{\colorbox{headtoc}{\color{white} MODULE}
v4}}

\hypertarget{ecldoc:mod_2.v4}{}
\hspace{0pt} \hyperlink{ecldoc:mod_2}{mod_2} \textbackslash 

{\renewcommand{\arraystretch}{1.5}
\begin{tabularx}{\textwidth}{|>{\raggedright\arraybackslash}l|X|}
\hline
\hspace{0pt}\mytexttt{\color{red} } & \textbf{v4} \\
\hline
\multicolumn{2}{|>{\raggedright\arraybackslash}X|}{\hspace{0pt}\mytexttt{\color{param} (REAL8 a2)}} \\
\hline
\end{tabularx}
}

\par





No Documentation Found






\par
\begin{description}
\item [\colorbox{tagtype}{\color{white} \textbf{\textsf{PARAMETER}}}] \textbf{\underline{a2}} ||| REAL8 --- No Doc
\end{description}






\textbf{Children}
\begin{enumerate}
\item \hyperlink{ecldoc:mod_1.m1v4.m1v5}{m1v5}
: No Documentation Found
\end{enumerate}

\rule{\linewidth}{0.5pt}

\subsection*{\textsf{\colorbox{headtoc}{\color{white} ATTRIBUTE}
m1v5}}

\hypertarget{ecldoc:mod_1.m1v4.m1v5}{}
\hspace{0pt} \hyperlink{ecldoc:mod_2}{mod_2} \textbackslash 
\hspace{0pt} \hyperlink{ecldoc:mod_2.v4}{v4} \textbackslash 

{\renewcommand{\arraystretch}{1.5}
\begin{tabularx}{\textwidth}{|>{\raggedright\arraybackslash}l|X|}
\hline
\hspace{0pt}\mytexttt{\color{red} } & \textbf{m1v5} \\
\hline
\end{tabularx}
}

\par





No Documentation Found








\par
\begin{description}
\item [\colorbox{tagtype}{\color{white} \textbf{\textsf{RETURN}}}] \textbf{REAL8} --- 
\end{description}




\rule{\linewidth}{0.5pt}


\subsection*{\textsf{\colorbox{headtoc}{\color{white} ATTRIBUTE}
v5}}

\hypertarget{ecldoc:mod_2.v5}{}
\hspace{0pt} \hyperlink{ecldoc:mod_2}{mod_2} \textbackslash 

{\renewcommand{\arraystretch}{1.5}
\begin{tabularx}{\textwidth}{|>{\raggedright\arraybackslash}l|X|}
\hline
\hspace{0pt}\mytexttt{\color{red} } & \textbf{v5} \\
\hline
\end{tabularx}
}

\par





No Documentation Found








\par
\begin{description}
\item [\colorbox{tagtype}{\color{white} \textbf{\textsf{RETURN}}}] \textbf{REAL8} --- 
\end{description}




\rule{\linewidth}{0.5pt}
\subsection*{\textsf{\colorbox{headtoc}{\color{white} ATTRIBUTE}
v6}}

\hypertarget{ecldoc:mod_2.v6}{}
\hspace{0pt} \hyperlink{ecldoc:mod_2}{mod_2} \textbackslash 

{\renewcommand{\arraystretch}{1.5}
\begin{tabularx}{\textwidth}{|>{\raggedright\arraybackslash}l|X|}
\hline
\hspace{0pt}\mytexttt{\color{red} } & \textbf{v6} \\
\hline
\end{tabularx}
}

\par





No Documentation Found








\par
\begin{description}
\item [\colorbox{tagtype}{\color{white} \textbf{\textsf{RETURN}}}] \textbf{REAL8} --- 
\end{description}




\rule{\linewidth}{0.5pt}




