\chapter*{\color{headtoc} in1intest}
\hypertarget{ecldoc:toc:root/intest/in1intest}{}
\hyperlink{ecldoc:toc:}{Go Up}


\section*{Table of Contents}
{\renewcommand{\arraystretch}{1.5}
\begin{longtable}{|p{\textwidth}|}
\hline
\hyperlink{ecldoc:toc:intest.in1intest.example_2}{example\_2.ecl} \\
Basic Inheritance documentation : mod\_3 inherits both mod\_1 and mod\_2 \\
\hline
\hyperlink{ecldoc:toc:intest.in1intest.example_3}{example\_3.ecl} \\
Example : Inheritance across files \\
\hline
\hyperlink{ecldoc:toc:intest.in1intest.example_4}{example\_4.ecl} \\
Example : Inheritance across files \\
\hline
\hyperlink{ecldoc:toc:intest.in1intest.example_6}{example\_6.ecl} \\
Module Hierarchy Example : mod\_1 -\&gt; mod\_11 -\&gt; mod\_111 \\
\hline
\hyperlink{ecldoc:toc:intest.in1intest.example_7}{example\_7.ecl} \\
Basic Type Example \\
\hline
\hyperlink{ecldoc:toc:intest.in1intest.example_8}{example\_8.ecl} \\
Three level Hierarchy Example \\
\hline
\end{longtable}
}

\chapter*{\color{headfile}
example_2
}
\hypertarget{ecldoc:toc:example_2}{}
\hyperlink{ecldoc:toc:root}{Go Up}


\section*{\underline{\textsf{DESCRIPTIONS}}}
\subsection*{\textsf{\colorbox{headtoc}{\color{white} MODULE}
example\_2}}

\hypertarget{ecldoc:example_2}{}

{\renewcommand{\arraystretch}{1.5}
\begin{tabularx}{\textwidth}{|>{\raggedright\arraybackslash}l|X|}
\hline
\hspace{0pt}\mytexttt{\color{red} } & \textbf{example\_2} \\
\hline
\end{tabularx}
}

\par
Basic Inheritance documentation : mod\_3 inherits both mod\_1 and mod\_2 . Inherits v2\_m1, v2\_m2, Overrides v1\_m1, new locals v2\_m3 . Interface Inheritance : mod\_4 inherits interface iface\_1, overrides v1\_i1


\textbf{Children}
\begin{enumerate}
\item \hyperlink{ecldoc:example_2.rec_1}{rec\_1}
\item \hyperlink{ecldoc:example_2.rec_2}{rec\_2}
\item \hyperlink{ecldoc:example_2.rec_3}{rec\_3}
\item \hyperlink{ecldoc:example_2.mod_1}{mod\_1}
\item \hyperlink{ecldoc:example_2.mod_2}{mod\_2}
\item \hyperlink{ecldoc:example_2.mod_3}{mod\_3}
\item \hyperlink{ecldoc:example_2.iface_1}{iface\_1}
\item \hyperlink{ecldoc:example_2.mod_4}{mod\_4}
\end{enumerate}

\rule{\linewidth}{0.5pt}

\subsection*{\textsf{\colorbox{headtoc}{\color{white} RECORD}
rec\_1}}

\hypertarget{ecldoc:example_2.rec_1}{}
\hspace{0pt} \hyperlink{ecldoc:example_2}{example_2} \textbackslash 

{\renewcommand{\arraystretch}{1.5}
\begin{tabularx}{\textwidth}{|>{\raggedright\arraybackslash}l|X|}
\hline
\hspace{0pt}\mytexttt{\color{red} } & \textbf{rec\_1} \\
\hline
\end{tabularx}
}

\par


\rule{\linewidth}{0.5pt}
\subsection*{\textsf{\colorbox{headtoc}{\color{white} RECORD}
rec\_2}}

\hypertarget{ecldoc:example_2.rec_2}{}
\hspace{0pt} \hyperlink{ecldoc:example_2}{example_2} \textbackslash 

{\renewcommand{\arraystretch}{1.5}
\begin{tabularx}{\textwidth}{|>{\raggedright\arraybackslash}l|X|}
\hline
\hspace{0pt}\mytexttt{\color{red} } & \textbf{rec\_2} \\
\hline
\end{tabularx}
}

\par


\rule{\linewidth}{0.5pt}
\subsection*{\textsf{\colorbox{headtoc}{\color{white} RECORD}
rec\_3}}

\hypertarget{ecldoc:example_2.rec_3}{}
\hspace{0pt} \hyperlink{ecldoc:example_2}{example_2} \textbackslash 

{\renewcommand{\arraystretch}{1.5}
\begin{tabularx}{\textwidth}{|>{\raggedright\arraybackslash}l|X|}
\hline
\hspace{0pt}\mytexttt{\color{red} } & \textbf{rec\_3} \\
\hline
\end{tabularx}
}

\par


\rule{\linewidth}{0.5pt}
\subsection*{\textsf{\colorbox{headtoc}{\color{white} MODULE}
mod\_1}}

\hypertarget{ecldoc:example_2.mod_1}{}
\hspace{0pt} \hyperlink{ecldoc:example_2}{example_2} \textbackslash 

{\renewcommand{\arraystretch}{1.5}
\begin{tabularx}{\textwidth}{|>{\raggedright\arraybackslash}l|X|}
\hline
\hspace{0pt}\mytexttt{\color{red} } & \textbf{mod\_1} \\
\hline
\end{tabularx}
}

\par


\textbf{Children}
\begin{enumerate}
\item \hyperlink{ecldoc:example_2.mod_1.v1_m1}{v1\_m1}
\item \hyperlink{ecldoc:example_2.mod_1.v2_m1}{v2\_m1}
\end{enumerate}

\rule{\linewidth}{0.5pt}

\subsection*{\textsf{\colorbox{headtoc}{\color{white} ATTRIBUTE}
v1\_m1}}

\hypertarget{ecldoc:example_2.mod_1.v1_m1}{}
\hspace{0pt} \hyperlink{ecldoc:example_2}{example_2} \textbackslash 
\hspace{0pt} \hyperlink{ecldoc:example_2.mod_1}{mod_1} \textbackslash 

{\renewcommand{\arraystretch}{1.5}
\begin{tabularx}{\textwidth}{|>{\raggedright\arraybackslash}l|X|}
\hline
\hspace{0pt}\mytexttt{\color{red} real8} & \textbf{v1\_m1} \\
\hline
\end{tabularx}
}

\par


\rule{\linewidth}{0.5pt}
\subsection*{\textsf{\colorbox{headtoc}{\color{white} ATTRIBUTE}
v2\_m1}}

\hypertarget{ecldoc:example_2.mod_1.v2_m1}{}
\hspace{0pt} \hyperlink{ecldoc:example_2}{example_2} \textbackslash 
\hspace{0pt} \hyperlink{ecldoc:example_2.mod_1}{mod_1} \textbackslash 

{\renewcommand{\arraystretch}{1.5}
\begin{tabularx}{\textwidth}{|>{\raggedright\arraybackslash}l|X|}
\hline
\hspace{0pt}\mytexttt{\color{red} } & \textbf{v2\_m1} \\
\hline
\end{tabularx}
}

\par


\rule{\linewidth}{0.5pt}


\subsection*{\textsf{\colorbox{headtoc}{\color{white} MODULE}
mod\_2}}

\hypertarget{ecldoc:example_2.mod_2}{}
\hspace{0pt} \hyperlink{ecldoc:example_2}{example_2} \textbackslash 

{\renewcommand{\arraystretch}{1.5}
\begin{tabularx}{\textwidth}{|>{\raggedright\arraybackslash}l|X|}
\hline
\hspace{0pt}\mytexttt{\color{red} } & \textbf{mod\_2} \\
\hline
\end{tabularx}
}

\par


\textbf{Children}
\begin{enumerate}
\item \hyperlink{ecldoc:example_2.mod_2.v1_m1}{v1\_m1}
\item \hyperlink{ecldoc:example_2.mod_2.v2_m2}{v2\_m2}
\end{enumerate}

\rule{\linewidth}{0.5pt}

\subsection*{\textsf{\colorbox{headtoc}{\color{white} ATTRIBUTE}
v1\_m1}}

\hypertarget{ecldoc:example_2.mod_2.v1_m1}{}
\hspace{0pt} \hyperlink{ecldoc:example_2}{example_2} \textbackslash 
\hspace{0pt} \hyperlink{ecldoc:example_2.mod_2}{mod_2} \textbackslash 

{\renewcommand{\arraystretch}{1.5}
\begin{tabularx}{\textwidth}{|>{\raggedright\arraybackslash}l|X|}
\hline
\hspace{0pt}\mytexttt{\color{red} } & \textbf{v1\_m1} \\
\hline
\end{tabularx}
}

\par


\rule{\linewidth}{0.5pt}
\subsection*{\textsf{\colorbox{headtoc}{\color{white} ATTRIBUTE}
v2\_m2}}

\hypertarget{ecldoc:example_2.mod_2.v2_m2}{}
\hspace{0pt} \hyperlink{ecldoc:example_2}{example_2} \textbackslash 
\hspace{0pt} \hyperlink{ecldoc:example_2.mod_2}{mod_2} \textbackslash 

{\renewcommand{\arraystretch}{1.5}
\begin{tabularx}{\textwidth}{|>{\raggedright\arraybackslash}l|X|}
\hline
\hspace{0pt}\mytexttt{\color{red} } & \textbf{v2\_m2} \\
\hline
\end{tabularx}
}

\par


\rule{\linewidth}{0.5pt}


\subsection*{\textsf{\colorbox{headtoc}{\color{white} MODULE}
mod\_3}}

\hypertarget{ecldoc:example_2.mod_3}{}
\hspace{0pt} \hyperlink{ecldoc:example_2}{example_2} \textbackslash 

{\renewcommand{\arraystretch}{1.5}
\begin{tabularx}{\textwidth}{|>{\raggedright\arraybackslash}l|X|}
\hline
\hspace{0pt}\mytexttt{\color{red} } & \textbf{mod\_3} \\
\hline
\end{tabularx}
}

\par


\textbf{Children}
\begin{enumerate}
\item \hyperlink{ecldoc:example_2.mod_1.v2_m1}{v2\_m1}
\item \hyperlink{ecldoc:example_2.mod_2.v2_m2}{v2\_m2}
\item \hyperlink{ecldoc:example_2.mod_3.v1_m1}{v1\_m1}
\item \hyperlink{ecldoc:example_2.mod_3.v2_m3}{v2\_m3}
\end{enumerate}

\rule{\linewidth}{0.5pt}

\subsection*{\textsf{\colorbox{headtoc}{\color{white} ATTRIBUTE}
v2\_m1}}

\hypertarget{ecldoc:example_2.mod_1.v2_m1}{}
\hspace{0pt} \hyperlink{ecldoc:example_2}{example_2} \textbackslash 
\hspace{0pt} \hyperlink{ecldoc:example_2.mod_3}{mod_3} \textbackslash 

{\renewcommand{\arraystretch}{1.5}
\begin{tabularx}{\textwidth}{|>{\raggedright\arraybackslash}l|X|}
\hline
\hspace{0pt}\mytexttt{\color{red} } & \textbf{v2\_m1} \\
\hline
\end{tabularx}
}

\par

\par
\begin{description}
\item [\colorbox{tagtype}{\color{white} \textbf{\textsf{INHERITED}}}] \textbf{\underline{}} True
\end{description}

\rule{\linewidth}{0.5pt}
\subsection*{\textsf{\colorbox{headtoc}{\color{white} ATTRIBUTE}
v2\_m2}}

\hypertarget{ecldoc:example_2.mod_2.v2_m2}{}
\hspace{0pt} \hyperlink{ecldoc:example_2}{example_2} \textbackslash 
\hspace{0pt} \hyperlink{ecldoc:example_2.mod_3}{mod_3} \textbackslash 

{\renewcommand{\arraystretch}{1.5}
\begin{tabularx}{\textwidth}{|>{\raggedright\arraybackslash}l|X|}
\hline
\hspace{0pt}\mytexttt{\color{red} } & \textbf{v2\_m2} \\
\hline
\end{tabularx}
}

\par

\par
\begin{description}
\item [\colorbox{tagtype}{\color{white} \textbf{\textsf{INHERITED}}}] \textbf{\underline{}} True
\end{description}

\rule{\linewidth}{0.5pt}
\subsection*{\textsf{\colorbox{headtoc}{\color{white} ATTRIBUTE}
v1\_m1}}

\hypertarget{ecldoc:example_2.mod_3.v1_m1}{}
\hspace{0pt} \hyperlink{ecldoc:example_2}{example_2} \textbackslash 
\hspace{0pt} \hyperlink{ecldoc:example_2.mod_3}{mod_3} \textbackslash 

{\renewcommand{\arraystretch}{1.5}
\begin{tabularx}{\textwidth}{|>{\raggedright\arraybackslash}l|X|}
\hline
\hspace{0pt}\mytexttt{\color{red} } & \textbf{v1\_m1} \\
\hline
\end{tabularx}
}

\par

\par
\begin{description}
\item [\colorbox{tagtype}{\color{white} \textbf{\textsf{OVERRIDE}}}] \textbf{\underline{}} True
\end{description}

\rule{\linewidth}{0.5pt}
\subsection*{\textsf{\colorbox{headtoc}{\color{white} ATTRIBUTE}
v2\_m3}}

\hypertarget{ecldoc:example_2.mod_3.v2_m3}{}
\hspace{0pt} \hyperlink{ecldoc:example_2}{example_2} \textbackslash 
\hspace{0pt} \hyperlink{ecldoc:example_2.mod_3}{mod_3} \textbackslash 

{\renewcommand{\arraystretch}{1.5}
\begin{tabularx}{\textwidth}{|>{\raggedright\arraybackslash}l|X|}
\hline
\hspace{0pt}\mytexttt{\color{red} } & \textbf{v2\_m3} \\
\hline
\end{tabularx}
}

\par


\rule{\linewidth}{0.5pt}


\subsection*{\textsf{\colorbox{headtoc}{\color{white} INTERFACE}
iface\_1}}

\hypertarget{ecldoc:example_2.iface_1}{}
\hspace{0pt} \hyperlink{ecldoc:example_2}{example_2} \textbackslash 

{\renewcommand{\arraystretch}{1.5}
\begin{tabularx}{\textwidth}{|>{\raggedright\arraybackslash}l|X|}
\hline
\hspace{0pt}\mytexttt{\color{red} } & \textbf{iface\_1} \\
\hline
\end{tabularx}
}

\par


\textbf{Children}
\begin{enumerate}
\item \hyperlink{ecldoc:example_2.iface_1.v1_i1}{v1\_i1}
\end{enumerate}

\rule{\linewidth}{0.5pt}

\subsection*{\textsf{\colorbox{headtoc}{\color{white} ATTRIBUTE}
v1\_i1}}

\hypertarget{ecldoc:example_2.iface_1.v1_i1}{}
\hspace{0pt} \hyperlink{ecldoc:example_2}{example_2} \textbackslash 
\hspace{0pt} \hyperlink{ecldoc:example_2.iface_1}{iface_1} \textbackslash 

{\renewcommand{\arraystretch}{1.5}
\begin{tabularx}{\textwidth}{|>{\raggedright\arraybackslash}l|X|}
\hline
\hspace{0pt}\mytexttt{\color{red} real8} & \textbf{v1\_i1} \\
\hline
\end{tabularx}
}

\par


\rule{\linewidth}{0.5pt}


\subsection*{\textsf{\colorbox{headtoc}{\color{white} MODULE}
mod\_4}}

\hypertarget{ecldoc:example_2.mod_4}{}
\hspace{0pt} \hyperlink{ecldoc:example_2}{example_2} \textbackslash 

{\renewcommand{\arraystretch}{1.5}
\begin{tabularx}{\textwidth}{|>{\raggedright\arraybackslash}l|X|}
\hline
\hspace{0pt}\mytexttt{\color{red} } & \textbf{mod\_4} \\
\hline
\end{tabularx}
}

\par


\textbf{Children}
\begin{enumerate}
\item \hyperlink{ecldoc:example_2.mod_4.v1_i1}{v1\_i1}
\item \hyperlink{ecldoc:example_2.mod_4.v2_m4}{v2\_m4}
\end{enumerate}

\rule{\linewidth}{0.5pt}

\subsection*{\textsf{\colorbox{headtoc}{\color{white} ATTRIBUTE}
v1\_i1}}

\hypertarget{ecldoc:example_2.mod_4.v1_i1}{}
\hspace{0pt} \hyperlink{ecldoc:example_2}{example_2} \textbackslash 
\hspace{0pt} \hyperlink{ecldoc:example_2.mod_4}{mod_4} \textbackslash 

{\renewcommand{\arraystretch}{1.5}
\begin{tabularx}{\textwidth}{|>{\raggedright\arraybackslash}l|X|}
\hline
\hspace{0pt}\mytexttt{\color{red} } & \textbf{v1\_i1} \\
\hline
\end{tabularx}
}

\par

\par
\begin{description}
\item [\colorbox{tagtype}{\color{white} \textbf{\textsf{OVERRIDE}}}] \textbf{\underline{}} True
\end{description}

\rule{\linewidth}{0.5pt}
\subsection*{\textsf{\colorbox{headtoc}{\color{white} ATTRIBUTE}
v2\_m4}}

\hypertarget{ecldoc:example_2.mod_4.v2_m4}{}
\hspace{0pt} \hyperlink{ecldoc:example_2}{example_2} \textbackslash 
\hspace{0pt} \hyperlink{ecldoc:example_2.mod_4}{mod_4} \textbackslash 

{\renewcommand{\arraystretch}{1.5}
\begin{tabularx}{\textwidth}{|>{\raggedright\arraybackslash}l|X|}
\hline
\hspace{0pt}\mytexttt{\color{red} STRING20} & \textbf{v2\_m4} \\
\hline
\end{tabularx}
}

\par


\rule{\linewidth}{0.5pt}





\chapter*{\color{headfile}
{\large intest\slash\hspace{0pt}}
{\large in1intest\slash\hspace{0pt}}
 \\
example_3
}
\hypertarget{ecldoc:toc:intest.in1intest.example_3}{}
\hyperlink{ecldoc:toc:root/intest/in1intest}{Go Up}


\section*{\underline{\textsf{DESCRIPTIONS}}}
\subsection*{\textsf{\colorbox{headtoc}{\color{white} MODULE}
Example\_3}}

\hypertarget{ecldoc:intest.in1intest.Example_3}{}

{\renewcommand{\arraystretch}{1.5}
\begin{tabularx}{\textwidth}{|>{\raggedright\arraybackslash}l|X|}
\hline
\hspace{0pt}\mytexttt{\color{red} } & \textbf{Example\_3} \\
\hline
\end{tabularx}
}

\par
Example : Inheritance across files mod\_1 in Example\_4 inherits mod\_1 in Example\_3


\textbf{Children}
\begin{enumerate}
\item \hyperlink{ecldoc:intest.in1intest.Example_3.mod_1}{mod\_1}
\end{enumerate}

\rule{\linewidth}{0.5pt}

\subsection*{\textsf{\colorbox{headtoc}{\color{white} MODULE}
mod\_1}}

\hypertarget{ecldoc:intest.in1intest.Example_3.mod_1}{}
\hspace{0pt} \hyperlink{ecldoc:intest.in1intest.Example_3}{Example_3} \textbackslash 

{\renewcommand{\arraystretch}{1.5}
\begin{tabularx}{\textwidth}{|>{\raggedright\arraybackslash}l|X|}
\hline
\hspace{0pt}\mytexttt{\color{red} } & \textbf{mod\_1} \\
\hline
\end{tabularx}
}

\par


\textbf{Children}
\begin{enumerate}
\item \hyperlink{ecldoc:intest.in1intest.example_3.mod_1.v1_m1}{v1\_m1}
\item \hyperlink{ecldoc:intest.in1intest.example_3.mod_1.v2_m1_ex3}{v2\_m1\_ex3}
\end{enumerate}

\rule{\linewidth}{0.5pt}

\subsection*{\textsf{\colorbox{headtoc}{\color{white} ATTRIBUTE}
v1\_m1}}

\hypertarget{ecldoc:intest.in1intest.example_3.mod_1.v1_m1}{}
\hspace{0pt} \hyperlink{ecldoc:intest.in1intest.Example_3}{Example_3} \textbackslash 
\hspace{0pt} \hyperlink{ecldoc:intest.in1intest.Example_3.mod_1}{mod_1} \textbackslash 

{\renewcommand{\arraystretch}{1.5}
\begin{tabularx}{\textwidth}{|>{\raggedright\arraybackslash}l|X|}
\hline
\hspace{0pt}\mytexttt{\color{red} } & \textbf{v1\_m1} \\
\hline
\end{tabularx}
}

\par


\rule{\linewidth}{0.5pt}
\subsection*{\textsf{\colorbox{headtoc}{\color{white} ATTRIBUTE}
v2\_m1\_ex3}}

\hypertarget{ecldoc:intest.in1intest.example_3.mod_1.v2_m1_ex3}{}
\hspace{0pt} \hyperlink{ecldoc:intest.in1intest.Example_3}{Example_3} \textbackslash 
\hspace{0pt} \hyperlink{ecldoc:intest.in1intest.Example_3.mod_1}{mod_1} \textbackslash 

{\renewcommand{\arraystretch}{1.5}
\begin{tabularx}{\textwidth}{|>{\raggedright\arraybackslash}l|X|}
\hline
\hspace{0pt}\mytexttt{\color{red} } & \textbf{v2\_m1\_ex3} \\
\hline
\end{tabularx}
}

\par


\rule{\linewidth}{0.5pt}





\chapter*{\color{headfile}
{\large intest\slash\hspace{0pt}}
 \\
example_4
}
\hypertarget{ecldoc:toc:intest.example_4}{}
\hyperlink{ecldoc:toc:root/intest}{Go Up}

\section*{\underline{\textsf{IMPORTS}}}
\begin{doublespace}
{\large
Example\_3.mod\_1 |
}
\end{doublespace}

\section*{\underline{\textsf{DESCRIPTIONS}}}
\subsection*{\textsf{\colorbox{headtoc}{\color{white} MODULE}
example\_4}}

\hypertarget{ecldoc:intest.example_4}{}

{\renewcommand{\arraystretch}{1.5}
\begin{tabularx}{\textwidth}{|>{\raggedright\arraybackslash}l|X|}
\hline
\hspace{0pt}\mytexttt{\color{red} } & \textbf{example\_4} \\
\hline
\end{tabularx}
}

\par





Example : Inheritance across files mod\_1 in Example\_4 inherits mod\_1 in Example\_3







\textbf{Children}
\begin{enumerate}
\item \hyperlink{ecldoc:intest.example_4.mod_1}{mod\_1}
: No Documentation Found
\end{enumerate}

\rule{\linewidth}{0.5pt}

\subsection*{\textsf{\colorbox{headtoc}{\color{white} MODULE}
mod\_1}}

\hypertarget{ecldoc:intest.example_4.mod_1}{}
\hspace{0pt} \hyperlink{ecldoc:intest.example_4}{example_4} \textbackslash 

{\renewcommand{\arraystretch}{1.5}
\begin{tabularx}{\textwidth}{|>{\raggedright\arraybackslash}l|X|}
\hline
\hspace{0pt}\mytexttt{\color{red} } & \textbf{mod\_1} \\
\hline
\end{tabularx}
}

\par





No Documentation Found










\par
\begin{description}
\item [\colorbox{tagtype}{\color{white} \textbf{\textsf{PARENT}}}] \textbf{Example\_3.mod\_1} <../example\_3.ecl.tex>
\end{description}


\textbf{Children}
\begin{enumerate}
\item \hyperlink{ecldoc:intest.example_4.mod_1.v2_m1_ex4}{v2\_m1\_ex4}
: No Documentation Found
\item \hyperlink{ecldoc:example_3.mod_1.v1_m1}{v1\_m1}
: Doc test 2
\item \hyperlink{ecldoc:example_3.mod_1.v2_m1_ex3}{v2\_m1\_ex3}
: DOC Test 3
\item \hyperlink{ecldoc:example_3.mod_1.abc}{abc}
: No Documentation Found
\item \hyperlink{ecldoc:example_3.mod_1.long_name}{long\_name}
: No Documentation Found
\end{enumerate}

\rule{\linewidth}{0.5pt}

\subsection*{\textsf{\colorbox{headtoc}{\color{white} ATTRIBUTE}
v2\_m1\_ex4}}

\hypertarget{ecldoc:intest.example_4.mod_1.v2_m1_ex4}{}
\hspace{0pt} \hyperlink{ecldoc:intest.example_4}{example_4} \textbackslash 
\hspace{0pt} \hyperlink{ecldoc:intest.example_4.mod_1}{mod_1} \textbackslash 

{\renewcommand{\arraystretch}{1.5}
\begin{tabularx}{\textwidth}{|>{\raggedright\arraybackslash}l|X|}
\hline
\hspace{0pt}\mytexttt{\color{red} } & \textbf{v2\_m1\_ex4} \\
\hline
\end{tabularx}
}

\par





No Documentation Found








\par
\begin{description}
\item [\colorbox{tagtype}{\color{white} \textbf{\textsf{RETURN}}}] \textbf{REAL8} --- 
\end{description}




\rule{\linewidth}{0.5pt}
\subsection*{\textsf{\colorbox{headtoc}{\color{white} ATTRIBUTE}
v1\_m1}}

\hypertarget{ecldoc:example_3.mod_1.v1_m1}{}
\hspace{0pt} \hyperlink{ecldoc:intest.example_4}{example_4} \textbackslash 
\hspace{0pt} \hyperlink{ecldoc:intest.example_4.mod_1}{mod_1} \textbackslash 

{\renewcommand{\arraystretch}{1.5}
\begin{tabularx}{\textwidth}{|>{\raggedright\arraybackslash}l|X|}
\hline
\hspace{0pt}\mytexttt{\color{red} } & \textbf{v1\_m1} \\
\hline
\end{tabularx}
}

\par





Doc test 2. Title end by period not newline 
\begin{verbatim}

 ABCD ||||
 CDEF ||||\end{verbatim}










\par
\begin{description}
\item [\colorbox{tagtype}{\color{white} \textbf{\textsf{RETURN}}}] \textbf{REAL8} --- 
\end{description}






\par
\begin{description}
\item [\colorbox{tagtype}{\color{white} \textbf{\textsf{INHERITED}}}] 
\end{description}



\rule{\linewidth}{0.5pt}
\subsection*{\textsf{\colorbox{headtoc}{\color{white} ATTRIBUTE}
v2\_m1\_ex3}}

\hypertarget{ecldoc:example_3.mod_1.v2_m1_ex3}{}
\hspace{0pt} \hyperlink{ecldoc:intest.example_4}{example_4} \textbackslash 
\hspace{0pt} \hyperlink{ecldoc:intest.example_4.mod_1}{mod_1} \textbackslash 

{\renewcommand{\arraystretch}{1.5}
\begin{tabularx}{\textwidth}{|>{\raggedright\arraybackslash}l|X|}
\hline
\hspace{0pt}\mytexttt{\color{red} } & \textbf{v2\_m1\_ex3} \\
\hline
\end{tabularx}
}

\par





DOC Test 3 No Period title








\par
\begin{description}
\item [\colorbox{tagtype}{\color{white} \textbf{\textsf{RETURN}}}] \textbf{REAL8} --- 
\end{description}






\par
\begin{description}
\item [\colorbox{tagtype}{\color{white} \textbf{\textsf{INHERITED}}}] 
\end{description}



\rule{\linewidth}{0.5pt}
\subsection*{\textsf{\colorbox{headtoc}{\color{white} FUNCTION}
abc}}

\hypertarget{ecldoc:example_3.mod_1.abc}{}
\hspace{0pt} \hyperlink{ecldoc:intest.example_4}{example_4} \textbackslash 
\hspace{0pt} \hyperlink{ecldoc:intest.example_4.mod_1}{mod_1} \textbackslash 

{\renewcommand{\arraystretch}{1.5}
\begin{tabularx}{\textwidth}{|>{\raggedright\arraybackslash}l|X|}
\hline
\hspace{0pt}\mytexttt{\color{red} REAL8} & \textbf{abc} \\
\hline
\multicolumn{2}{|>{\raggedright\arraybackslash}X|}{\hspace{0pt}\mytexttt{\color{param} (REAL8 x)}} \\
\hline
\end{tabularx}
}

\par





No Documentation Found






\par
\begin{description}
\item [\colorbox{tagtype}{\color{white} \textbf{\textsf{PARAMETER}}}] \textbf{\underline{x}} ||| REAL8 --- No Doc
\end{description}







\par
\begin{description}
\item [\colorbox{tagtype}{\color{white} \textbf{\textsf{RETURN}}}] \textbf{REAL8} --- 
\end{description}






\par
\begin{description}
\item [\colorbox{tagtype}{\color{white} \textbf{\textsf{INHERITED}}}] 
\end{description}



\rule{\linewidth}{0.5pt}
\subsection*{\textsf{\colorbox{headtoc}{\color{white} FUNCTION}
long\_name}}

\hypertarget{ecldoc:example_3.mod_1.long_name}{}
\hspace{0pt} \hyperlink{ecldoc:intest.example_4}{example_4} \textbackslash 
\hspace{0pt} \hyperlink{ecldoc:intest.example_4.mod_1}{mod_1} \textbackslash 

{\renewcommand{\arraystretch}{1.5}
\begin{tabularx}{\textwidth}{|>{\raggedright\arraybackslash}l|X|}
\hline
\hspace{0pt}\mytexttt{\color{red} } & \textbf{long\_name} \\
\hline
\multicolumn{2}{|>{\raggedright\arraybackslash}X|}{\hspace{0pt}\mytexttt{\color{param} (DATASET(\{REAL8 u\}) X, DATASET(\{REAL8 u\}) IntW, DATASET(\{REAL8 u\}) Intb, REAL8 BETA=0.1, REAL8 sparsityParam=0.1 , REAL8 LAMBDA=0.001, REAL8 ALPHA=0.1, UNSIGNED2 MaxIter=100)}} \\
\hline
\end{tabularx}
}

\par





No Documentation Found






\par
\begin{description}
\item [\colorbox{tagtype}{\color{white} \textbf{\textsf{PARAMETER}}}] \textbf{\underline{sparsityparam}} ||| REAL8 --- No Doc
\item [\colorbox{tagtype}{\color{white} \textbf{\textsf{PARAMETER}}}] \textbf{\underline{alpha}} ||| REAL8 --- No Doc
\item [\colorbox{tagtype}{\color{white} \textbf{\textsf{PARAMETER}}}] \textbf{\underline{lambda}} ||| REAL8 --- No Doc
\item [\colorbox{tagtype}{\color{white} \textbf{\textsf{PARAMETER}}}] \textbf{\underline{maxiter}} ||| UNSIGNED2 --- No Doc
\item [\colorbox{tagtype}{\color{white} \textbf{\textsf{PARAMETER}}}] \textbf{\underline{intb}} ||| TABLE ( \{ REAL8 u \} ) --- No Doc
\item [\colorbox{tagtype}{\color{white} \textbf{\textsf{PARAMETER}}}] \textbf{\underline{beta}} ||| REAL8 --- No Doc
\item [\colorbox{tagtype}{\color{white} \textbf{\textsf{PARAMETER}}}] \textbf{\underline{x}} ||| TABLE ( \{ REAL8 u \} ) --- No Doc
\item [\colorbox{tagtype}{\color{white} \textbf{\textsf{PARAMETER}}}] \textbf{\underline{intw}} ||| TABLE ( \{ REAL8 u \} ) --- No Doc
\end{description}







\par
\begin{description}
\item [\colorbox{tagtype}{\color{white} \textbf{\textsf{RETURN}}}] \textbf{REAL8} --- 
\end{description}






\par
\begin{description}
\item [\colorbox{tagtype}{\color{white} \textbf{\textsf{INHERITED}}}] 
\end{description}



\rule{\linewidth}{0.5pt}





\chapter*{\color{headfile}
{\large intest\slash\hspace{0pt}}
 \\
example_6
}
\hypertarget{ecldoc:toc:intest.example_6}{}
\hyperlink{ecldoc:toc:root/intest}{Go Up}


\section*{\underline{\textsf{DESCRIPTIONS}}}
\subsection*{\textsf{\colorbox{headtoc}{\color{white} MODULE}
example\_6}}

\hypertarget{ecldoc:intest.example_6}{}

{\renewcommand{\arraystretch}{1.5}
\begin{tabularx}{\textwidth}{|>{\raggedright\arraybackslash}l|X|}
\hline
\hspace{0pt}\mytexttt{\color{red} } & \textbf{example\_6} \\
\hline
\end{tabularx}
}

\par





Module Hierarchy Example : mod\_1 -> mod\_11 -> mod\_111 . Inheritance across Hierarchy : mod\_2 inherits mod\_1.mod\_11 , mod\_3.mod\_31 inherits mod\_1.mod\_11 , mod\_4 inherits mod\_3.mod\_31, mod\_2 , mod\_5 inherits mod\_1 and mod\_1.mod\_11







\textbf{Children}
\begin{enumerate}
\item \hyperlink{ecldoc:intest.example_6.mod_1}{mod\_1}
: No Documentation Found
\item \hyperlink{ecldoc:intest.example_6.mod_2}{mod\_2}
: No Documentation Found
\item \hyperlink{ecldoc:intest.example_6.mod_3}{mod\_3}
: No Documentation Found
\item \hyperlink{ecldoc:intest.example_6.mod_4}{mod\_4}
: No Documentation Found
\item \hyperlink{ecldoc:intest.example_6.mod_5}{mod\_5}
: No Documentation Found
\end{enumerate}

\rule{\linewidth}{0.5pt}

\subsection*{\textsf{\colorbox{headtoc}{\color{white} MODULE}
mod\_1}}

\hypertarget{ecldoc:intest.example_6.mod_1}{}
\hspace{0pt} \hyperlink{ecldoc:intest.example_6}{example_6} \textbackslash 

{\renewcommand{\arraystretch}{1.5}
\begin{tabularx}{\textwidth}{|>{\raggedright\arraybackslash}l|X|}
\hline
\hspace{0pt}\mytexttt{\color{red} } & \textbf{mod\_1} \\
\hline
\end{tabularx}
}

\par





No Documentation Found







\textbf{Children}
\begin{enumerate}
\item \hyperlink{ecldoc:intest.example_6.mod_1.v1_m1}{v1\_m1}
: No Documentation Found
\item \hyperlink{ecldoc:intest.example_6.mod_1.mod_11}{mod\_11}
: No Documentation Found
\end{enumerate}

\rule{\linewidth}{0.5pt}

\subsection*{\textsf{\colorbox{headtoc}{\color{white} ATTRIBUTE}
v1\_m1}}

\hypertarget{ecldoc:intest.example_6.mod_1.v1_m1}{}
\hspace{0pt} \hyperlink{ecldoc:intest.example_6}{example_6} \textbackslash 
\hspace{0pt} \hyperlink{ecldoc:intest.example_6.mod_1}{mod_1} \textbackslash 

{\renewcommand{\arraystretch}{1.5}
\begin{tabularx}{\textwidth}{|>{\raggedright\arraybackslash}l|X|}
\hline
\hspace{0pt}\mytexttt{\color{red} } & \textbf{v1\_m1} \\
\hline
\end{tabularx}
}

\par





No Documentation Found








\par
\begin{description}
\item [\colorbox{tagtype}{\color{white} \textbf{\textsf{RETURN}}}] \textbf{REAL8} --- 
\end{description}




\rule{\linewidth}{0.5pt}
\subsection*{\textsf{\colorbox{headtoc}{\color{white} MODULE}
mod\_11}}

\hypertarget{ecldoc:intest.example_6.mod_1.mod_11}{}
\hspace{0pt} \hyperlink{ecldoc:intest.example_6}{example_6} \textbackslash 
\hspace{0pt} \hyperlink{ecldoc:intest.example_6.mod_1}{mod_1} \textbackslash 

{\renewcommand{\arraystretch}{1.5}
\begin{tabularx}{\textwidth}{|>{\raggedright\arraybackslash}l|X|}
\hline
\hspace{0pt}\mytexttt{\color{red} } & \textbf{mod\_11} \\
\hline
\multicolumn{2}{|>{\raggedright\arraybackslash}X|}{\hspace{0pt}\mytexttt{\color{param} (real8 a\_11)}} \\
\hline
\end{tabularx}
}

\par





No Documentation Found






\par
\begin{description}
\item [\colorbox{tagtype}{\color{white} \textbf{\textsf{PARAMETER}}}] \textbf{\underline{a\_11}} ||| REAL8 --- No Doc
\end{description}






\textbf{Children}
\begin{enumerate}
\item \hyperlink{ecldoc:intest.example_6.mod_1.mod_11.v1_m11}{v1\_m11}
: No Documentation Found
\item \hyperlink{ecldoc:intest.example_6.mod_1.mod_11.mod_111}{mod\_111}
: No Documentation Found
\end{enumerate}

\rule{\linewidth}{0.5pt}

\subsection*{\textsf{\colorbox{headtoc}{\color{white} ATTRIBUTE}
v1\_m11}}

\hypertarget{ecldoc:intest.example_6.mod_1.mod_11.v1_m11}{}
\hspace{0pt} \hyperlink{ecldoc:intest.example_6}{example_6} \textbackslash 
\hspace{0pt} \hyperlink{ecldoc:intest.example_6.mod_1}{mod_1} \textbackslash 
\hspace{0pt} \hyperlink{ecldoc:intest.example_6.mod_1.mod_11}{mod_11} \textbackslash 

{\renewcommand{\arraystretch}{1.5}
\begin{tabularx}{\textwidth}{|>{\raggedright\arraybackslash}l|X|}
\hline
\hspace{0pt}\mytexttt{\color{red} } & \textbf{v1\_m11} \\
\hline
\end{tabularx}
}

\par





No Documentation Found








\par
\begin{description}
\item [\colorbox{tagtype}{\color{white} \textbf{\textsf{RETURN}}}] \textbf{REAL8} --- 
\end{description}




\rule{\linewidth}{0.5pt}
\subsection*{\textsf{\colorbox{headtoc}{\color{white} MODULE}
mod\_111}}

\hypertarget{ecldoc:intest.example_6.mod_1.mod_11.mod_111}{}
\hspace{0pt} \hyperlink{ecldoc:intest.example_6}{example_6} \textbackslash 
\hspace{0pt} \hyperlink{ecldoc:intest.example_6.mod_1}{mod_1} \textbackslash 
\hspace{0pt} \hyperlink{ecldoc:intest.example_6.mod_1.mod_11}{mod_11} \textbackslash 

{\renewcommand{\arraystretch}{1.5}
\begin{tabularx}{\textwidth}{|>{\raggedright\arraybackslash}l|X|}
\hline
\hspace{0pt}\mytexttt{\color{red} } & \textbf{mod\_111} \\
\hline
\multicolumn{2}{|>{\raggedright\arraybackslash}X|}{\hspace{0pt}\mytexttt{\color{param} (real8 a\_111)}} \\
\hline
\end{tabularx}
}

\par





No Documentation Found






\par
\begin{description}
\item [\colorbox{tagtype}{\color{white} \textbf{\textsf{PARAMETER}}}] \textbf{\underline{a\_111}} ||| REAL8 --- No Doc
\end{description}






\textbf{Children}
\begin{enumerate}
\item \hyperlink{ecldoc:intest.example_6.mod_1.mod_11.mod_111.v1_m111}{v1\_m111}
: No Documentation Found
\end{enumerate}

\rule{\linewidth}{0.5pt}

\subsection*{\textsf{\colorbox{headtoc}{\color{white} ATTRIBUTE}
v1\_m111}}

\hypertarget{ecldoc:intest.example_6.mod_1.mod_11.mod_111.v1_m111}{}
\hspace{0pt} \hyperlink{ecldoc:intest.example_6}{example_6} \textbackslash 
\hspace{0pt} \hyperlink{ecldoc:intest.example_6.mod_1}{mod_1} \textbackslash 
\hspace{0pt} \hyperlink{ecldoc:intest.example_6.mod_1.mod_11}{mod_11} \textbackslash 
\hspace{0pt} \hyperlink{ecldoc:intest.example_6.mod_1.mod_11.mod_111}{mod_111} \textbackslash 

{\renewcommand{\arraystretch}{1.5}
\begin{tabularx}{\textwidth}{|>{\raggedright\arraybackslash}l|X|}
\hline
\hspace{0pt}\mytexttt{\color{red} } & \textbf{v1\_m111} \\
\hline
\end{tabularx}
}

\par





No Documentation Found








\par
\begin{description}
\item [\colorbox{tagtype}{\color{white} \textbf{\textsf{RETURN}}}] \textbf{REAL8} --- 
\end{description}




\rule{\linewidth}{0.5pt}






\subsection*{\textsf{\colorbox{headtoc}{\color{white} MODULE}
mod\_2}}

\hypertarget{ecldoc:intest.example_6.mod_2}{}
\hspace{0pt} \hyperlink{ecldoc:intest.example_6}{example_6} \textbackslash 

{\renewcommand{\arraystretch}{1.5}
\begin{tabularx}{\textwidth}{|>{\raggedright\arraybackslash}l|X|}
\hline
\hspace{0pt}\mytexttt{\color{red} } & \textbf{mod\_2} \\
\hline
\end{tabularx}
}

\par





No Documentation Found










\par
\begin{description}
\item [\colorbox{tagtype}{\color{white} \textbf{\textsf{PARENT}}}] \textbf{intest.example\_6.mod\_1.mod\_11} <example\_6.ecl.tex>
\end{description}


\textbf{Children}
\begin{enumerate}
\item \hyperlink{ecldoc:intest.example_6.mod_1.mod_11.v1_m11}{v1\_m11}
: No Documentation Found
\item \hyperlink{ecldoc:intest.example_6.mod_1.mod_11.mod_111}{mod\_111}
: No Documentation Found
\item \hyperlink{ecldoc:intest.example_6.mod_2.v1_m2}{v1\_m2}
: No Documentation Found
\end{enumerate}

\rule{\linewidth}{0.5pt}

\subsection*{\textsf{\colorbox{headtoc}{\color{white} ATTRIBUTE}
v1\_m11}}

\hypertarget{ecldoc:intest.example_6.mod_1.mod_11.v1_m11}{}
\hspace{0pt} \hyperlink{ecldoc:intest.example_6}{example_6} \textbackslash 
\hspace{0pt} \hyperlink{ecldoc:intest.example_6.mod_2}{mod_2} \textbackslash 

{\renewcommand{\arraystretch}{1.5}
\begin{tabularx}{\textwidth}{|>{\raggedright\arraybackslash}l|X|}
\hline
\hspace{0pt}\mytexttt{\color{red} } & \textbf{v1\_m11} \\
\hline
\end{tabularx}
}

\par





No Documentation Found








\par
\begin{description}
\item [\colorbox{tagtype}{\color{white} \textbf{\textsf{RETURN}}}] \textbf{REAL8} --- 
\end{description}






\par
\begin{description}
\item [\colorbox{tagtype}{\color{white} \textbf{\textsf{INHERITED}}}] 
\end{description}



\rule{\linewidth}{0.5pt}
\subsection*{\textsf{\colorbox{headtoc}{\color{white} MODULE}
mod\_111}}

\hypertarget{ecldoc:intest.example_6.mod_1.mod_11.mod_111}{}
\hspace{0pt} \hyperlink{ecldoc:intest.example_6}{example_6} \textbackslash 
\hspace{0pt} \hyperlink{ecldoc:intest.example_6.mod_2}{mod_2} \textbackslash 

{\renewcommand{\arraystretch}{1.5}
\begin{tabularx}{\textwidth}{|>{\raggedright\arraybackslash}l|X|}
\hline
\hspace{0pt}\mytexttt{\color{red} } & \textbf{mod\_111} \\
\hline
\multicolumn{2}{|>{\raggedright\arraybackslash}X|}{\hspace{0pt}\mytexttt{\color{param} (real8 a\_111)}} \\
\hline
\end{tabularx}
}

\par





No Documentation Found






\par
\begin{description}
\item [\colorbox{tagtype}{\color{white} \textbf{\textsf{PARAMETER}}}] \textbf{\underline{a\_111}} ||| REAL8 --- No Doc
\end{description}








\par
\begin{description}
\item [\colorbox{tagtype}{\color{white} \textbf{\textsf{INHERITED}}}] 
\end{description}



\rule{\linewidth}{0.5pt}
\subsection*{\textsf{\colorbox{headtoc}{\color{white} ATTRIBUTE}
v1\_m2}}

\hypertarget{ecldoc:intest.example_6.mod_2.v1_m2}{}
\hspace{0pt} \hyperlink{ecldoc:intest.example_6}{example_6} \textbackslash 
\hspace{0pt} \hyperlink{ecldoc:intest.example_6.mod_2}{mod_2} \textbackslash 

{\renewcommand{\arraystretch}{1.5}
\begin{tabularx}{\textwidth}{|>{\raggedright\arraybackslash}l|X|}
\hline
\hspace{0pt}\mytexttt{\color{red} } & \textbf{v1\_m2} \\
\hline
\end{tabularx}
}

\par





No Documentation Found








\par
\begin{description}
\item [\colorbox{tagtype}{\color{white} \textbf{\textsf{RETURN}}}] \textbf{REAL8} --- 
\end{description}




\rule{\linewidth}{0.5pt}


\subsection*{\textsf{\colorbox{headtoc}{\color{white} MODULE}
mod\_3}}

\hypertarget{ecldoc:intest.example_6.mod_3}{}
\hspace{0pt} \hyperlink{ecldoc:intest.example_6}{example_6} \textbackslash 

{\renewcommand{\arraystretch}{1.5}
\begin{tabularx}{\textwidth}{|>{\raggedright\arraybackslash}l|X|}
\hline
\hspace{0pt}\mytexttt{\color{red} } & \textbf{mod\_3} \\
\hline
\end{tabularx}
}

\par





No Documentation Found







\textbf{Children}
\begin{enumerate}
\item \hyperlink{ecldoc:intest.example_6.mod_3.v1_m3}{v1\_m3}
: No Documentation Found
\item \hyperlink{ecldoc:intest.example_6.mod_3.mod_31}{mod\_31}
: No Documentation Found
\end{enumerate}

\rule{\linewidth}{0.5pt}

\subsection*{\textsf{\colorbox{headtoc}{\color{white} ATTRIBUTE}
v1\_m3}}

\hypertarget{ecldoc:intest.example_6.mod_3.v1_m3}{}
\hspace{0pt} \hyperlink{ecldoc:intest.example_6}{example_6} \textbackslash 
\hspace{0pt} \hyperlink{ecldoc:intest.example_6.mod_3}{mod_3} \textbackslash 

{\renewcommand{\arraystretch}{1.5}
\begin{tabularx}{\textwidth}{|>{\raggedright\arraybackslash}l|X|}
\hline
\hspace{0pt}\mytexttt{\color{red} } & \textbf{v1\_m3} \\
\hline
\end{tabularx}
}

\par





No Documentation Found








\par
\begin{description}
\item [\colorbox{tagtype}{\color{white} \textbf{\textsf{RETURN}}}] \textbf{REAL8} --- 
\end{description}




\rule{\linewidth}{0.5pt}
\subsection*{\textsf{\colorbox{headtoc}{\color{white} MODULE}
mod\_31}}

\hypertarget{ecldoc:intest.example_6.mod_3.mod_31}{}
\hspace{0pt} \hyperlink{ecldoc:intest.example_6}{example_6} \textbackslash 
\hspace{0pt} \hyperlink{ecldoc:intest.example_6.mod_3}{mod_3} \textbackslash 

{\renewcommand{\arraystretch}{1.5}
\begin{tabularx}{\textwidth}{|>{\raggedright\arraybackslash}l|X|}
\hline
\hspace{0pt}\mytexttt{\color{red} } & \textbf{mod\_31} \\
\hline
\end{tabularx}
}

\par





No Documentation Found










\par
\begin{description}
\item [\colorbox{tagtype}{\color{white} \textbf{\textsf{PARENT}}}] \textbf{intest.example\_6.mod\_1.mod\_11} <example\_6.ecl.tex>
\end{description}


\textbf{Children}
\begin{enumerate}
\item \hyperlink{ecldoc:intest.example_6.mod_1.mod_11.v1_m11}{v1\_m11}
: No Documentation Found
\item \hyperlink{ecldoc:intest.example_6.mod_1.mod_11.mod_111}{mod\_111}
: No Documentation Found
\item \hyperlink{ecldoc:intest.example_6.mod_3.mod_31.v1_m31}{v1\_m31}
: No Documentation Found
\end{enumerate}

\rule{\linewidth}{0.5pt}

\subsection*{\textsf{\colorbox{headtoc}{\color{white} ATTRIBUTE}
v1\_m11}}

\hypertarget{ecldoc:intest.example_6.mod_1.mod_11.v1_m11}{}
\hspace{0pt} \hyperlink{ecldoc:intest.example_6}{example_6} \textbackslash 
\hspace{0pt} \hyperlink{ecldoc:intest.example_6.mod_3}{mod_3} \textbackslash 
\hspace{0pt} \hyperlink{ecldoc:intest.example_6.mod_3.mod_31}{mod_31} \textbackslash 

{\renewcommand{\arraystretch}{1.5}
\begin{tabularx}{\textwidth}{|>{\raggedright\arraybackslash}l|X|}
\hline
\hspace{0pt}\mytexttt{\color{red} } & \textbf{v1\_m11} \\
\hline
\end{tabularx}
}

\par





No Documentation Found








\par
\begin{description}
\item [\colorbox{tagtype}{\color{white} \textbf{\textsf{RETURN}}}] \textbf{REAL8} --- 
\end{description}






\par
\begin{description}
\item [\colorbox{tagtype}{\color{white} \textbf{\textsf{INHERITED}}}] 
\end{description}



\rule{\linewidth}{0.5pt}
\subsection*{\textsf{\colorbox{headtoc}{\color{white} MODULE}
mod\_111}}

\hypertarget{ecldoc:intest.example_6.mod_1.mod_11.mod_111}{}
\hspace{0pt} \hyperlink{ecldoc:intest.example_6}{example_6} \textbackslash 
\hspace{0pt} \hyperlink{ecldoc:intest.example_6.mod_3}{mod_3} \textbackslash 
\hspace{0pt} \hyperlink{ecldoc:intest.example_6.mod_3.mod_31}{mod_31} \textbackslash 

{\renewcommand{\arraystretch}{1.5}
\begin{tabularx}{\textwidth}{|>{\raggedright\arraybackslash}l|X|}
\hline
\hspace{0pt}\mytexttt{\color{red} } & \textbf{mod\_111} \\
\hline
\multicolumn{2}{|>{\raggedright\arraybackslash}X|}{\hspace{0pt}\mytexttt{\color{param} (real8 a\_111)}} \\
\hline
\end{tabularx}
}

\par





No Documentation Found






\par
\begin{description}
\item [\colorbox{tagtype}{\color{white} \textbf{\textsf{PARAMETER}}}] \textbf{\underline{a\_111}} ||| REAL8 --- No Doc
\end{description}








\par
\begin{description}
\item [\colorbox{tagtype}{\color{white} \textbf{\textsf{INHERITED}}}] 
\end{description}



\rule{\linewidth}{0.5pt}
\subsection*{\textsf{\colorbox{headtoc}{\color{white} ATTRIBUTE}
v1\_m31}}

\hypertarget{ecldoc:intest.example_6.mod_3.mod_31.v1_m31}{}
\hspace{0pt} \hyperlink{ecldoc:intest.example_6}{example_6} \textbackslash 
\hspace{0pt} \hyperlink{ecldoc:intest.example_6.mod_3}{mod_3} \textbackslash 
\hspace{0pt} \hyperlink{ecldoc:intest.example_6.mod_3.mod_31}{mod_31} \textbackslash 

{\renewcommand{\arraystretch}{1.5}
\begin{tabularx}{\textwidth}{|>{\raggedright\arraybackslash}l|X|}
\hline
\hspace{0pt}\mytexttt{\color{red} } & \textbf{v1\_m31} \\
\hline
\end{tabularx}
}

\par





No Documentation Found








\par
\begin{description}
\item [\colorbox{tagtype}{\color{white} \textbf{\textsf{RETURN}}}] \textbf{REAL8} --- 
\end{description}




\rule{\linewidth}{0.5pt}




\subsection*{\textsf{\colorbox{headtoc}{\color{white} MODULE}
mod\_4}}

\hypertarget{ecldoc:intest.example_6.mod_4}{}
\hspace{0pt} \hyperlink{ecldoc:intest.example_6}{example_6} \textbackslash 

{\renewcommand{\arraystretch}{1.5}
\begin{tabularx}{\textwidth}{|>{\raggedright\arraybackslash}l|X|}
\hline
\hspace{0pt}\mytexttt{\color{red} } & \textbf{mod\_4} \\
\hline
\end{tabularx}
}

\par





No Documentation Found










\par
\begin{description}
\item [\colorbox{tagtype}{\color{white} \textbf{\textsf{PARENT}}}] \textbf{intest.example\_6.mod\_3.mod\_31} <example\_6.ecl.tex>
\item [\colorbox{tagtype}{\color{white} \textbf{\textsf{PARENT}}}] \textbf{intest.example\_6.mod\_2} <example\_6.ecl.tex>
\end{description}


\textbf{Children}
\begin{enumerate}
\item \hyperlink{ecldoc:intest.example_6.mod_1.mod_11.v1_m11}{v1\_m11}
: No Documentation Found
\item \hyperlink{ecldoc:intest.example_6.mod_1.mod_11.mod_111}{mod\_111}
: No Documentation Found
\item \hyperlink{ecldoc:intest.example_6.mod_2.v1_m2}{v1\_m2}
: No Documentation Found
\item \hyperlink{ecldoc:intest.example_6.mod_3.mod_31.v1_m31}{v1\_m31}
: No Documentation Found
\item \hyperlink{ecldoc:intest.example_6.mod_4.v1_m4}{v1\_m4}
: No Documentation Found
\end{enumerate}

\rule{\linewidth}{0.5pt}

\subsection*{\textsf{\colorbox{headtoc}{\color{white} ATTRIBUTE}
v1\_m11}}

\hypertarget{ecldoc:intest.example_6.mod_1.mod_11.v1_m11}{}
\hspace{0pt} \hyperlink{ecldoc:intest.example_6}{example_6} \textbackslash 
\hspace{0pt} \hyperlink{ecldoc:intest.example_6.mod_4}{mod_4} \textbackslash 

{\renewcommand{\arraystretch}{1.5}
\begin{tabularx}{\textwidth}{|>{\raggedright\arraybackslash}l|X|}
\hline
\hspace{0pt}\mytexttt{\color{red} } & \textbf{v1\_m11} \\
\hline
\end{tabularx}
}

\par





No Documentation Found








\par
\begin{description}
\item [\colorbox{tagtype}{\color{white} \textbf{\textsf{RETURN}}}] \textbf{REAL8} --- 
\end{description}






\par
\begin{description}
\item [\colorbox{tagtype}{\color{white} \textbf{\textsf{INHERITED}}}] 
\end{description}



\rule{\linewidth}{0.5pt}
\subsection*{\textsf{\colorbox{headtoc}{\color{white} MODULE}
mod\_111}}

\hypertarget{ecldoc:intest.example_6.mod_1.mod_11.mod_111}{}
\hspace{0pt} \hyperlink{ecldoc:intest.example_6}{example_6} \textbackslash 
\hspace{0pt} \hyperlink{ecldoc:intest.example_6.mod_4}{mod_4} \textbackslash 

{\renewcommand{\arraystretch}{1.5}
\begin{tabularx}{\textwidth}{|>{\raggedright\arraybackslash}l|X|}
\hline
\hspace{0pt}\mytexttt{\color{red} } & \textbf{mod\_111} \\
\hline
\multicolumn{2}{|>{\raggedright\arraybackslash}X|}{\hspace{0pt}\mytexttt{\color{param} (real8 a\_111)}} \\
\hline
\end{tabularx}
}

\par





No Documentation Found






\par
\begin{description}
\item [\colorbox{tagtype}{\color{white} \textbf{\textsf{PARAMETER}}}] \textbf{\underline{a\_111}} ||| REAL8 --- No Doc
\end{description}








\par
\begin{description}
\item [\colorbox{tagtype}{\color{white} \textbf{\textsf{INHERITED}}}] 
\end{description}



\rule{\linewidth}{0.5pt}
\subsection*{\textsf{\colorbox{headtoc}{\color{white} ATTRIBUTE}
v1\_m2}}

\hypertarget{ecldoc:intest.example_6.mod_2.v1_m2}{}
\hspace{0pt} \hyperlink{ecldoc:intest.example_6}{example_6} \textbackslash 
\hspace{0pt} \hyperlink{ecldoc:intest.example_6.mod_4}{mod_4} \textbackslash 

{\renewcommand{\arraystretch}{1.5}
\begin{tabularx}{\textwidth}{|>{\raggedright\arraybackslash}l|X|}
\hline
\hspace{0pt}\mytexttt{\color{red} } & \textbf{v1\_m2} \\
\hline
\end{tabularx}
}

\par





No Documentation Found








\par
\begin{description}
\item [\colorbox{tagtype}{\color{white} \textbf{\textsf{RETURN}}}] \textbf{REAL8} --- 
\end{description}






\par
\begin{description}
\item [\colorbox{tagtype}{\color{white} \textbf{\textsf{INHERITED}}}] 
\end{description}



\rule{\linewidth}{0.5pt}
\subsection*{\textsf{\colorbox{headtoc}{\color{white} ATTRIBUTE}
v1\_m31}}

\hypertarget{ecldoc:intest.example_6.mod_3.mod_31.v1_m31}{}
\hspace{0pt} \hyperlink{ecldoc:intest.example_6}{example_6} \textbackslash 
\hspace{0pt} \hyperlink{ecldoc:intest.example_6.mod_4}{mod_4} \textbackslash 

{\renewcommand{\arraystretch}{1.5}
\begin{tabularx}{\textwidth}{|>{\raggedright\arraybackslash}l|X|}
\hline
\hspace{0pt}\mytexttt{\color{red} } & \textbf{v1\_m31} \\
\hline
\end{tabularx}
}

\par





No Documentation Found








\par
\begin{description}
\item [\colorbox{tagtype}{\color{white} \textbf{\textsf{RETURN}}}] \textbf{REAL8} --- 
\end{description}






\par
\begin{description}
\item [\colorbox{tagtype}{\color{white} \textbf{\textsf{INHERITED}}}] 
\end{description}



\rule{\linewidth}{0.5pt}
\subsection*{\textsf{\colorbox{headtoc}{\color{white} ATTRIBUTE}
v1\_m4}}

\hypertarget{ecldoc:intest.example_6.mod_4.v1_m4}{}
\hspace{0pt} \hyperlink{ecldoc:intest.example_6}{example_6} \textbackslash 
\hspace{0pt} \hyperlink{ecldoc:intest.example_6.mod_4}{mod_4} \textbackslash 

{\renewcommand{\arraystretch}{1.5}
\begin{tabularx}{\textwidth}{|>{\raggedright\arraybackslash}l|X|}
\hline
\hspace{0pt}\mytexttt{\color{red} } & \textbf{v1\_m4} \\
\hline
\end{tabularx}
}

\par





No Documentation Found








\par
\begin{description}
\item [\colorbox{tagtype}{\color{white} \textbf{\textsf{RETURN}}}] \textbf{REAL8} --- 
\end{description}




\rule{\linewidth}{0.5pt}


\subsection*{\textsf{\colorbox{headtoc}{\color{white} MODULE}
mod\_5}}

\hypertarget{ecldoc:intest.example_6.mod_5}{}
\hspace{0pt} \hyperlink{ecldoc:intest.example_6}{example_6} \textbackslash 

{\renewcommand{\arraystretch}{1.5}
\begin{tabularx}{\textwidth}{|>{\raggedright\arraybackslash}l|X|}
\hline
\hspace{0pt}\mytexttt{\color{red} } & \textbf{mod\_5} \\
\hline
\end{tabularx}
}

\par





No Documentation Found










\par
\begin{description}
\item [\colorbox{tagtype}{\color{white} \textbf{\textsf{PARENT}}}] \textbf{intest.example\_6.mod\_1} <example\_6.ecl.tex>
\item [\colorbox{tagtype}{\color{white} \textbf{\textsf{PARENT}}}] \textbf{intest.example\_6.mod\_1.mod\_11} <example\_6.ecl.tex>
\end{description}


\textbf{Children}
\begin{enumerate}
\item \hyperlink{ecldoc:intest.example_6.mod_1.v1_m1}{v1\_m1}
: No Documentation Found
\item \hyperlink{ecldoc:intest.example_6.mod_1.mod_11}{mod\_11}
: No Documentation Found
\item \hyperlink{ecldoc:intest.example_6.mod_1.mod_11.v1_m11}{v1\_m11}
: No Documentation Found
\item \hyperlink{ecldoc:intest.example_6.mod_1.mod_11.mod_111}{mod\_111}
: No Documentation Found
\item \hyperlink{ecldoc:intest.example_6.mod_5.v1_m5}{v1\_m5}
: No Documentation Found
\end{enumerate}

\rule{\linewidth}{0.5pt}

\subsection*{\textsf{\colorbox{headtoc}{\color{white} ATTRIBUTE}
v1\_m1}}

\hypertarget{ecldoc:intest.example_6.mod_1.v1_m1}{}
\hspace{0pt} \hyperlink{ecldoc:intest.example_6}{example_6} \textbackslash 
\hspace{0pt} \hyperlink{ecldoc:intest.example_6.mod_5}{mod_5} \textbackslash 

{\renewcommand{\arraystretch}{1.5}
\begin{tabularx}{\textwidth}{|>{\raggedright\arraybackslash}l|X|}
\hline
\hspace{0pt}\mytexttt{\color{red} } & \textbf{v1\_m1} \\
\hline
\end{tabularx}
}

\par





No Documentation Found








\par
\begin{description}
\item [\colorbox{tagtype}{\color{white} \textbf{\textsf{RETURN}}}] \textbf{REAL8} --- 
\end{description}






\par
\begin{description}
\item [\colorbox{tagtype}{\color{white} \textbf{\textsf{INHERITED}}}] 
\end{description}



\rule{\linewidth}{0.5pt}
\subsection*{\textsf{\colorbox{headtoc}{\color{white} MODULE}
mod\_11}}

\hypertarget{ecldoc:intest.example_6.mod_1.mod_11}{}
\hspace{0pt} \hyperlink{ecldoc:intest.example_6}{example_6} \textbackslash 
\hspace{0pt} \hyperlink{ecldoc:intest.example_6.mod_5}{mod_5} \textbackslash 

{\renewcommand{\arraystretch}{1.5}
\begin{tabularx}{\textwidth}{|>{\raggedright\arraybackslash}l|X|}
\hline
\hspace{0pt}\mytexttt{\color{red} } & \textbf{mod\_11} \\
\hline
\multicolumn{2}{|>{\raggedright\arraybackslash}X|}{\hspace{0pt}\mytexttt{\color{param} (real8 a\_11)}} \\
\hline
\end{tabularx}
}

\par





No Documentation Found






\par
\begin{description}
\item [\colorbox{tagtype}{\color{white} \textbf{\textsf{PARAMETER}}}] \textbf{\underline{a\_11}} ||| REAL8 --- No Doc
\end{description}








\par
\begin{description}
\item [\colorbox{tagtype}{\color{white} \textbf{\textsf{INHERITED}}}] 
\end{description}



\rule{\linewidth}{0.5pt}
\subsection*{\textsf{\colorbox{headtoc}{\color{white} ATTRIBUTE}
v1\_m11}}

\hypertarget{ecldoc:intest.example_6.mod_1.mod_11.v1_m11}{}
\hspace{0pt} \hyperlink{ecldoc:intest.example_6}{example_6} \textbackslash 
\hspace{0pt} \hyperlink{ecldoc:intest.example_6.mod_5}{mod_5} \textbackslash 

{\renewcommand{\arraystretch}{1.5}
\begin{tabularx}{\textwidth}{|>{\raggedright\arraybackslash}l|X|}
\hline
\hspace{0pt}\mytexttt{\color{red} } & \textbf{v1\_m11} \\
\hline
\end{tabularx}
}

\par





No Documentation Found








\par
\begin{description}
\item [\colorbox{tagtype}{\color{white} \textbf{\textsf{RETURN}}}] \textbf{REAL8} --- 
\end{description}






\par
\begin{description}
\item [\colorbox{tagtype}{\color{white} \textbf{\textsf{INHERITED}}}] 
\end{description}



\rule{\linewidth}{0.5pt}
\subsection*{\textsf{\colorbox{headtoc}{\color{white} MODULE}
mod\_111}}

\hypertarget{ecldoc:intest.example_6.mod_1.mod_11.mod_111}{}
\hspace{0pt} \hyperlink{ecldoc:intest.example_6}{example_6} \textbackslash 
\hspace{0pt} \hyperlink{ecldoc:intest.example_6.mod_5}{mod_5} \textbackslash 

{\renewcommand{\arraystretch}{1.5}
\begin{tabularx}{\textwidth}{|>{\raggedright\arraybackslash}l|X|}
\hline
\hspace{0pt}\mytexttt{\color{red} } & \textbf{mod\_111} \\
\hline
\multicolumn{2}{|>{\raggedright\arraybackslash}X|}{\hspace{0pt}\mytexttt{\color{param} (real8 a\_111)}} \\
\hline
\end{tabularx}
}

\par





No Documentation Found






\par
\begin{description}
\item [\colorbox{tagtype}{\color{white} \textbf{\textsf{PARAMETER}}}] \textbf{\underline{a\_111}} ||| REAL8 --- No Doc
\end{description}








\par
\begin{description}
\item [\colorbox{tagtype}{\color{white} \textbf{\textsf{INHERITED}}}] 
\end{description}



\rule{\linewidth}{0.5pt}
\subsection*{\textsf{\colorbox{headtoc}{\color{white} ATTRIBUTE}
v1\_m5}}

\hypertarget{ecldoc:intest.example_6.mod_5.v1_m5}{}
\hspace{0pt} \hyperlink{ecldoc:intest.example_6}{example_6} \textbackslash 
\hspace{0pt} \hyperlink{ecldoc:intest.example_6.mod_5}{mod_5} \textbackslash 

{\renewcommand{\arraystretch}{1.5}
\begin{tabularx}{\textwidth}{|>{\raggedright\arraybackslash}l|X|}
\hline
\hspace{0pt}\mytexttt{\color{red} } & \textbf{v1\_m5} \\
\hline
\end{tabularx}
}

\par





No Documentation Found








\par
\begin{description}
\item [\colorbox{tagtype}{\color{white} \textbf{\textsf{RETURN}}}] \textbf{REAL8} --- 
\end{description}




\rule{\linewidth}{0.5pt}





\chapter*{\color{headfile}
example_7
}
\hypertarget{ecldoc:toc:example_7}{}
\hyperlink{ecldoc:toc:root}{Go Up}


\section*{\underline{\textsf{DESCRIPTIONS}}}
\subsection*{\textsf{\colorbox{headtoc}{\color{white} MODULE}
example\_7}}

\hypertarget{ecldoc:example_7}{}

{\renewcommand{\arraystretch}{1.5}
\begin{tabularx}{\textwidth}{|>{\raggedright\arraybackslash}l|X|}
\hline
\hspace{0pt}\mytexttt{\color{red} } & \textbf{example\_7} \\
\hline
\end{tabularx}
}

\par





Basic Type Example Source Code copied from ECL Documentation







\textbf{Children}
\begin{enumerate}
\item \hyperlink{ecldoc:example_7.r}{R}
: No Documentation Found
\end{enumerate}

\rule{\linewidth}{0.5pt}

\subsection*{\textsf{\colorbox{headtoc}{\color{white} RECORD}
R}}

\hypertarget{ecldoc:example_7.r}{}
\hspace{0pt} \hyperlink{ecldoc:example_7}{example_7} \textbackslash 

{\renewcommand{\arraystretch}{1.5}
\begin{tabularx}{\textwidth}{|>{\raggedright\arraybackslash}l|X|}
\hline
\hspace{0pt}\mytexttt{\color{red} } & \textbf{R} \\
\hline
\end{tabularx}
}

\par





No Documentation Found







\par
\begin{description}
\item [\colorbox{tagtype}{\color{white} \textbf{\textsf{FIELD}}}] \textbf{\underline{f3}} ||| SCALEINT --- No Doc
\item [\colorbox{tagtype}{\color{white} \textbf{\textsf{FIELD}}}] \textbf{\underline{f1}} ||| REVERSESTRING4 --- No Doc
\item [\colorbox{tagtype}{\color{white} \textbf{\textsf{FIELD}}}] \textbf{\underline{f2}} ||| NEEDC --- No Doc
\end{description}





\rule{\linewidth}{0.5pt}



\chapter*{\color{headfile}
{\large intest\slash\hspace{0pt}}
{\large inintest\slash\hspace{0pt}}
 \\
example_8
}
\hypertarget{ecldoc:toc:intest.inintest.example_8}{}
\hyperlink{ecldoc:toc:root/intest/inintest}{Go Up}


\section*{\underline{\textsf{DESCRIPTIONS}}}
\subsection*{\textsf{\colorbox{headtoc}{\color{white} MODULE}
example\_8}}

\hypertarget{ecldoc:intest.inintest.example_8}{}

{\renewcommand{\arraystretch}{1.5}
\begin{tabularx}{\textwidth}{|>{\raggedright\arraybackslash}l|X|}
\hline
\hspace{0pt}\mytexttt{\color{red} } & \textbf{example\_8} \\
\hline
\end{tabularx}
}

\par





Three level Hierarchy Example . Inheritance across Hierarchy . Problems with Type System -- PROJECT Expression does not maintain record typename (rec\_2) but do maintain record structure . IE mod\_2.v1\_m2 should be  but shown  .  has same structure as record  .







\textbf{Children}
\begin{enumerate}
\item \hyperlink{ecldoc:intest.inintest.example_8.mod_1}{mod\_1}
: No Documentation Found
\item \hyperlink{ecldoc:intest.inintest.example_8.mod_2}{mod\_2}
: No Documentation Found
\end{enumerate}

\rule{\linewidth}{0.5pt}

\subsection*{\textsf{\colorbox{headtoc}{\color{white} MODULE}
mod\_1}}

\hypertarget{ecldoc:intest.inintest.example_8.mod_1}{}
\hspace{0pt} \hyperlink{ecldoc:intest.inintest.example_8}{example_8} \textbackslash 

{\renewcommand{\arraystretch}{1.5}
\begin{tabularx}{\textwidth}{|>{\raggedright\arraybackslash}l|X|}
\hline
\hspace{0pt}\mytexttt{\color{red} } & \textbf{mod\_1} \\
\hline
\end{tabularx}
}

\par





No Documentation Found







\textbf{Children}
\begin{enumerate}
\item \hyperlink{ecldoc:intest.inintest.example_8.mod_1.rec_1}{rec\_1}
: No Documentation Found
\item \hyperlink{ecldoc:intest.inintest.example_8.mod_1.mod_11}{mod\_11}
: No Documentation Found
\end{enumerate}

\rule{\linewidth}{0.5pt}

\subsection*{\textsf{\colorbox{headtoc}{\color{white} RECORD}
rec\_1}}

\hypertarget{ecldoc:intest.inintest.example_8.mod_1.rec_1}{}
\hspace{0pt} \hyperlink{ecldoc:intest.inintest.example_8}{example_8} \textbackslash 
\hspace{0pt} \hyperlink{ecldoc:intest.inintest.example_8.mod_1}{mod_1} \textbackslash 

{\renewcommand{\arraystretch}{1.5}
\begin{tabularx}{\textwidth}{|>{\raggedright\arraybackslash}l|X|}
\hline
\hspace{0pt}\mytexttt{\color{red} } & \textbf{rec\_1} \\
\hline
\end{tabularx}
}

\par





No Documentation Found







\par
\begin{description}
\item [\colorbox{tagtype}{\color{white} \textbf{\textsf{FIELD}}}] \textbf{\underline{a}} ||| REAL8 --- No Doc
\end{description}





\rule{\linewidth}{0.5pt}
\subsection*{\textsf{\colorbox{headtoc}{\color{white} MODULE}
mod\_11}}

\hypertarget{ecldoc:intest.inintest.example_8.mod_1.mod_11}{}
\hspace{0pt} \hyperlink{ecldoc:intest.inintest.example_8}{example_8} \textbackslash 
\hspace{0pt} \hyperlink{ecldoc:intest.inintest.example_8.mod_1}{mod_1} \textbackslash 

{\renewcommand{\arraystretch}{1.5}
\begin{tabularx}{\textwidth}{|>{\raggedright\arraybackslash}l|X|}
\hline
\hspace{0pt}\mytexttt{\color{red} } & \textbf{mod\_11} \\
\hline
\end{tabularx}
}

\par





No Documentation Found







\textbf{Children}
\begin{enumerate}
\item \hyperlink{ecldoc:intest.inintest.example_8.mod_1.mod_11.v1_m11}{v1\_m11}
: No Documentation Found
\end{enumerate}

\rule{\linewidth}{0.5pt}

\subsection*{\textsf{\colorbox{headtoc}{\color{white} ATTRIBUTE}
v1\_m11}}

\hypertarget{ecldoc:intest.inintest.example_8.mod_1.mod_11.v1_m11}{}
\hspace{0pt} \hyperlink{ecldoc:intest.inintest.example_8}{example_8} \textbackslash 
\hspace{0pt} \hyperlink{ecldoc:intest.inintest.example_8.mod_1}{mod_1} \textbackslash 
\hspace{0pt} \hyperlink{ecldoc:intest.inintest.example_8.mod_1.mod_11}{mod_11} \textbackslash 

{\renewcommand{\arraystretch}{1.5}
\begin{tabularx}{\textwidth}{|>{\raggedright\arraybackslash}l|X|}
\hline
\hspace{0pt}\mytexttt{\color{red} } & \textbf{v1\_m11} \\
\hline
\end{tabularx}
}

\par





No Documentation Found








\par
\begin{description}
\item [\colorbox{tagtype}{\color{white} \textbf{\textsf{RETURN}}}] \textbf{TABLE ( rec\_1 )} --- 
\end{description}




\rule{\linewidth}{0.5pt}




\subsection*{\textsf{\colorbox{headtoc}{\color{white} MODULE}
mod\_2}}

\hypertarget{ecldoc:intest.inintest.example_8.mod_2}{}
\hspace{0pt} \hyperlink{ecldoc:intest.inintest.example_8}{example_8} \textbackslash 

{\renewcommand{\arraystretch}{1.5}
\begin{tabularx}{\textwidth}{|>{\raggedright\arraybackslash}l|X|}
\hline
\hspace{0pt}\mytexttt{\color{red} } & \textbf{mod\_2} \\
\hline
\end{tabularx}
}

\par





No Documentation Found










\par
\begin{description}
\item [\colorbox{tagtype}{\color{white} \textbf{\textsf{PARENT}}}] \textbf{intest.inintest.example\_8.mod\_1.mod\_11} <example\_8.ecl.tex>
\end{description}


\textbf{Children}
\begin{enumerate}
\item \hyperlink{ecldoc:intest.inintest.example_8.mod_1.mod_11.v1_m11}{v1\_m11}
: No Documentation Found
\item \hyperlink{ecldoc:intest.inintest.example_8.mod_2.rec_2}{rec\_2}
: No Documentation Found
\item \hyperlink{ecldoc:intest.inintest.example_8.mod_2.v1_m2}{v1\_m2}
: No Documentation Found
\end{enumerate}

\rule{\linewidth}{0.5pt}

\subsection*{\textsf{\colorbox{headtoc}{\color{white} ATTRIBUTE}
v1\_m11}}

\hypertarget{ecldoc:intest.inintest.example_8.mod_1.mod_11.v1_m11}{}
\hspace{0pt} \hyperlink{ecldoc:intest.inintest.example_8}{example_8} \textbackslash 
\hspace{0pt} \hyperlink{ecldoc:intest.inintest.example_8.mod_2}{mod_2} \textbackslash 

{\renewcommand{\arraystretch}{1.5}
\begin{tabularx}{\textwidth}{|>{\raggedright\arraybackslash}l|X|}
\hline
\hspace{0pt}\mytexttt{\color{red} } & \textbf{v1\_m11} \\
\hline
\end{tabularx}
}

\par





No Documentation Found








\par
\begin{description}
\item [\colorbox{tagtype}{\color{white} \textbf{\textsf{RETURN}}}] \textbf{TABLE ( rec\_1 )} --- 
\end{description}






\par
\begin{description}
\item [\colorbox{tagtype}{\color{white} \textbf{\textsf{INHERITED}}}] 
\end{description}



\rule{\linewidth}{0.5pt}
\subsection*{\textsf{\colorbox{headtoc}{\color{white} RECORD}
rec\_2}}

\hypertarget{ecldoc:intest.inintest.example_8.mod_2.rec_2}{}
\hspace{0pt} \hyperlink{ecldoc:intest.inintest.example_8}{example_8} \textbackslash 
\hspace{0pt} \hyperlink{ecldoc:intest.inintest.example_8.mod_2}{mod_2} \textbackslash 

{\renewcommand{\arraystretch}{1.5}
\begin{tabularx}{\textwidth}{|>{\raggedright\arraybackslash}l|X|}
\hline
\hspace{0pt}\mytexttt{\color{red} } & \textbf{rec\_2} \\
\hline
\end{tabularx}
}

\par





No Documentation Found







\par
\begin{description}
\item [\colorbox{tagtype}{\color{white} \textbf{\textsf{FIELD}}}] \textbf{\underline{b}} ||| REAL8 --- No Doc
\end{description}





\rule{\linewidth}{0.5pt}
\subsection*{\textsf{\colorbox{headtoc}{\color{white} FUNCTION}
v1\_m2}}

\hypertarget{ecldoc:intest.inintest.example_8.mod_2.v1_m2}{}
\hspace{0pt} \hyperlink{ecldoc:intest.inintest.example_8}{example_8} \textbackslash 
\hspace{0pt} \hyperlink{ecldoc:intest.inintest.example_8.mod_2}{mod_2} \textbackslash 

{\renewcommand{\arraystretch}{1.5}
\begin{tabularx}{\textwidth}{|>{\raggedright\arraybackslash}l|X|}
\hline
\hspace{0pt}\mytexttt{\color{red} } & \textbf{v1\_m2} \\
\hline
\multicolumn{2}{|>{\raggedright\arraybackslash}X|}{\hspace{0pt}\mytexttt{\color{param} (REAL8 ag\_1)}} \\
\hline
\end{tabularx}
}

\par





No Documentation Found






\par
\begin{description}
\item [\colorbox{tagtype}{\color{white} \textbf{\textsf{PARAMETER}}}] \textbf{\underline{ag\_1}} ||| REAL8 --- No Doc
\end{description}







\par
\begin{description}
\item [\colorbox{tagtype}{\color{white} \textbf{\textsf{RETURN}}}] \textbf{TABLE ( \{ REAL8 b \} )} --- 
\end{description}




\rule{\linewidth}{0.5pt}





