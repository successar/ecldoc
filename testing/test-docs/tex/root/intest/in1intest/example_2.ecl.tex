\chapter*{\color{headfile}
{\large intest\slash\hspace{0pt}}
{\large in1intest\slash\hspace{0pt}}
 \\
example_2
}
\hypertarget{ecldoc:toc:intest.in1intest.example_2}{}
\hyperlink{ecldoc:toc:root/intest/in1intest}{Go Up}


\section*{\underline{\textsf{DESCRIPTIONS}}}
\subsection*{\textsf{\colorbox{headtoc}{\color{white} MODULE}
example\_2}}

\hypertarget{ecldoc:intest.in1intest.example_2}{}

{\renewcommand{\arraystretch}{1.5}
\begin{tabularx}{\textwidth}{|>{\raggedright\arraybackslash}l|X|}
\hline
\hspace{0pt}\mytexttt{\color{red} } & \textbf{example\_2} \\
\hline
\end{tabularx}
}

\par
Basic Inheritance documentation : mod\_3 inherits both mod\_1 and mod\_2 . Inherits v2\_m1, v2\_m2, Overrides v1\_m1, new locals v2\_m3 . Interface Inheritance : mod\_4 inherits interface iface\_1, overrides v1\_i1


\textbf{Children}
\begin{enumerate}
\item \hyperlink{ecldoc:intest.in1intest.example_2.rec_1}{rec\_1}
\item \hyperlink{ecldoc:intest.in1intest.example_2.rec_2}{rec\_2}
\item \hyperlink{ecldoc:intest.in1intest.example_2.rec_3}{rec\_3}
\item \hyperlink{ecldoc:intest.in1intest.example_2.mod_1}{mod\_1}
\item \hyperlink{ecldoc:intest.in1intest.example_2.mod_2}{mod\_2}
\item \hyperlink{ecldoc:intest.in1intest.example_2.mod_3}{mod\_3}
\item \hyperlink{ecldoc:intest.in1intest.example_2.iface_1}{iface\_1}
\item \hyperlink{ecldoc:intest.in1intest.example_2.mod_4}{mod\_4}
\end{enumerate}

\rule{\linewidth}{0.5pt}

\subsection*{\textsf{\colorbox{headtoc}{\color{white} RECORD}
rec\_1}}

\hypertarget{ecldoc:intest.in1intest.example_2.rec_1}{}
\hspace{0pt} \hyperlink{ecldoc:intest.in1intest.example_2}{example_2} \textbackslash 

{\renewcommand{\arraystretch}{1.5}
\begin{tabularx}{\textwidth}{|>{\raggedright\arraybackslash}l|X|}
\hline
\hspace{0pt}\mytexttt{\color{red} } & \textbf{rec\_1} \\
\hline
\end{tabularx}
}

\par


\rule{\linewidth}{0.5pt}
\subsection*{\textsf{\colorbox{headtoc}{\color{white} RECORD}
rec\_2}}

\hypertarget{ecldoc:intest.in1intest.example_2.rec_2}{}
\hspace{0pt} \hyperlink{ecldoc:intest.in1intest.example_2}{example_2} \textbackslash 

{\renewcommand{\arraystretch}{1.5}
\begin{tabularx}{\textwidth}{|>{\raggedright\arraybackslash}l|X|}
\hline
\hspace{0pt}\mytexttt{\color{red} } & \textbf{rec\_2} \\
\hline
\end{tabularx}
}

\par


\rule{\linewidth}{0.5pt}
\subsection*{\textsf{\colorbox{headtoc}{\color{white} RECORD}
rec\_3}}

\hypertarget{ecldoc:intest.in1intest.example_2.rec_3}{}
\hspace{0pt} \hyperlink{ecldoc:intest.in1intest.example_2}{example_2} \textbackslash 

{\renewcommand{\arraystretch}{1.5}
\begin{tabularx}{\textwidth}{|>{\raggedright\arraybackslash}l|X|}
\hline
\hspace{0pt}\mytexttt{\color{red} } & \textbf{rec\_3} \\
\hline
\end{tabularx}
}

\par


\rule{\linewidth}{0.5pt}
\subsection*{\textsf{\colorbox{headtoc}{\color{white} MODULE}
mod\_1}}

\hypertarget{ecldoc:intest.in1intest.example_2.mod_1}{}
\hspace{0pt} \hyperlink{ecldoc:intest.in1intest.example_2}{example_2} \textbackslash 

{\renewcommand{\arraystretch}{1.5}
\begin{tabularx}{\textwidth}{|>{\raggedright\arraybackslash}l|X|}
\hline
\hspace{0pt}\mytexttt{\color{red} } & \textbf{mod\_1} \\
\hline
\end{tabularx}
}

\par


\textbf{Children}
\begin{enumerate}
\item \hyperlink{ecldoc:intest.in1intest.example_2.mod_1.v1_m1}{v1\_m1}
\item \hyperlink{ecldoc:intest.in1intest.example_2.mod_1.v2_m1}{v2\_m1}
\end{enumerate}

\rule{\linewidth}{0.5pt}

\subsection*{\textsf{\colorbox{headtoc}{\color{white} ATTRIBUTE}
v1\_m1}}

\hypertarget{ecldoc:intest.in1intest.example_2.mod_1.v1_m1}{}
\hspace{0pt} \hyperlink{ecldoc:intest.in1intest.example_2}{example_2} \textbackslash 
\hspace{0pt} \hyperlink{ecldoc:intest.in1intest.example_2.mod_1}{mod_1} \textbackslash 

{\renewcommand{\arraystretch}{1.5}
\begin{tabularx}{\textwidth}{|>{\raggedright\arraybackslash}l|X|}
\hline
\hspace{0pt}\mytexttt{\color{red} real8} & \textbf{v1\_m1} \\
\hline
\end{tabularx}
}

\par


\rule{\linewidth}{0.5pt}
\subsection*{\textsf{\colorbox{headtoc}{\color{white} ATTRIBUTE}
v2\_m1}}

\hypertarget{ecldoc:intest.in1intest.example_2.mod_1.v2_m1}{}
\hspace{0pt} \hyperlink{ecldoc:intest.in1intest.example_2}{example_2} \textbackslash 
\hspace{0pt} \hyperlink{ecldoc:intest.in1intest.example_2.mod_1}{mod_1} \textbackslash 

{\renewcommand{\arraystretch}{1.5}
\begin{tabularx}{\textwidth}{|>{\raggedright\arraybackslash}l|X|}
\hline
\hspace{0pt}\mytexttt{\color{red} } & \textbf{v2\_m1} \\
\hline
\end{tabularx}
}

\par


\rule{\linewidth}{0.5pt}


\subsection*{\textsf{\colorbox{headtoc}{\color{white} MODULE}
mod\_2}}

\hypertarget{ecldoc:intest.in1intest.example_2.mod_2}{}
\hspace{0pt} \hyperlink{ecldoc:intest.in1intest.example_2}{example_2} \textbackslash 

{\renewcommand{\arraystretch}{1.5}
\begin{tabularx}{\textwidth}{|>{\raggedright\arraybackslash}l|X|}
\hline
\hspace{0pt}\mytexttt{\color{red} } & \textbf{mod\_2} \\
\hline
\end{tabularx}
}

\par


\textbf{Children}
\begin{enumerate}
\item \hyperlink{ecldoc:intest.in1intest.example_2.mod_2.v1_m1}{v1\_m1}
\item \hyperlink{ecldoc:intest.in1intest.example_2.mod_2.v2_m2}{v2\_m2}
\end{enumerate}

\rule{\linewidth}{0.5pt}

\subsection*{\textsf{\colorbox{headtoc}{\color{white} ATTRIBUTE}
v1\_m1}}

\hypertarget{ecldoc:intest.in1intest.example_2.mod_2.v1_m1}{}
\hspace{0pt} \hyperlink{ecldoc:intest.in1intest.example_2}{example_2} \textbackslash 
\hspace{0pt} \hyperlink{ecldoc:intest.in1intest.example_2.mod_2}{mod_2} \textbackslash 

{\renewcommand{\arraystretch}{1.5}
\begin{tabularx}{\textwidth}{|>{\raggedright\arraybackslash}l|X|}
\hline
\hspace{0pt}\mytexttt{\color{red} } & \textbf{v1\_m1} \\
\hline
\end{tabularx}
}

\par


\rule{\linewidth}{0.5pt}
\subsection*{\textsf{\colorbox{headtoc}{\color{white} ATTRIBUTE}
v2\_m2}}

\hypertarget{ecldoc:intest.in1intest.example_2.mod_2.v2_m2}{}
\hspace{0pt} \hyperlink{ecldoc:intest.in1intest.example_2}{example_2} \textbackslash 
\hspace{0pt} \hyperlink{ecldoc:intest.in1intest.example_2.mod_2}{mod_2} \textbackslash 

{\renewcommand{\arraystretch}{1.5}
\begin{tabularx}{\textwidth}{|>{\raggedright\arraybackslash}l|X|}
\hline
\hspace{0pt}\mytexttt{\color{red} } & \textbf{v2\_m2} \\
\hline
\end{tabularx}
}

\par


\rule{\linewidth}{0.5pt}


\subsection*{\textsf{\colorbox{headtoc}{\color{white} MODULE}
mod\_3}}

\hypertarget{ecldoc:intest.in1intest.example_2.mod_3}{}
\hspace{0pt} \hyperlink{ecldoc:intest.in1intest.example_2}{example_2} \textbackslash 

{\renewcommand{\arraystretch}{1.5}
\begin{tabularx}{\textwidth}{|>{\raggedright\arraybackslash}l|X|}
\hline
\hspace{0pt}\mytexttt{\color{red} } & \textbf{mod\_3} \\
\hline
\end{tabularx}
}

\par


\textbf{Children}
\begin{enumerate}
\item \hyperlink{ecldoc:intest.in1intest.example_2.mod_1.v2_m1}{v2\_m1}
\item \hyperlink{ecldoc:intest.in1intest.example_2.mod_2.v2_m2}{v2\_m2}
\item \hyperlink{ecldoc:intest.in1intest.example_2.mod_3.v1_m1}{v1\_m1}
\item \hyperlink{ecldoc:intest.in1intest.example_2.mod_3.v2_m3}{v2\_m3}
\end{enumerate}

\rule{\linewidth}{0.5pt}

\subsection*{\textsf{\colorbox{headtoc}{\color{white} ATTRIBUTE}
v2\_m1}}

\hypertarget{ecldoc:intest.in1intest.example_2.mod_1.v2_m1}{}
\hspace{0pt} \hyperlink{ecldoc:intest.in1intest.example_2}{example_2} \textbackslash 
\hspace{0pt} \hyperlink{ecldoc:intest.in1intest.example_2.mod_3}{mod_3} \textbackslash 

{\renewcommand{\arraystretch}{1.5}
\begin{tabularx}{\textwidth}{|>{\raggedright\arraybackslash}l|X|}
\hline
\hspace{0pt}\mytexttt{\color{red} } & \textbf{v2\_m1} \\
\hline
\end{tabularx}
}

\par

\par
\begin{description}
\item [\colorbox{tagtype}{\color{white} \textbf{\textsf{INHERITED}}}] \textbf{\underline{}} True
\end{description}

\rule{\linewidth}{0.5pt}
\subsection*{\textsf{\colorbox{headtoc}{\color{white} ATTRIBUTE}
v2\_m2}}

\hypertarget{ecldoc:intest.in1intest.example_2.mod_2.v2_m2}{}
\hspace{0pt} \hyperlink{ecldoc:intest.in1intest.example_2}{example_2} \textbackslash 
\hspace{0pt} \hyperlink{ecldoc:intest.in1intest.example_2.mod_3}{mod_3} \textbackslash 

{\renewcommand{\arraystretch}{1.5}
\begin{tabularx}{\textwidth}{|>{\raggedright\arraybackslash}l|X|}
\hline
\hspace{0pt}\mytexttt{\color{red} } & \textbf{v2\_m2} \\
\hline
\end{tabularx}
}

\par

\par
\begin{description}
\item [\colorbox{tagtype}{\color{white} \textbf{\textsf{INHERITED}}}] \textbf{\underline{}} True
\end{description}

\rule{\linewidth}{0.5pt}
\subsection*{\textsf{\colorbox{headtoc}{\color{white} ATTRIBUTE}
v1\_m1}}

\hypertarget{ecldoc:intest.in1intest.example_2.mod_3.v1_m1}{}
\hspace{0pt} \hyperlink{ecldoc:intest.in1intest.example_2}{example_2} \textbackslash 
\hspace{0pt} \hyperlink{ecldoc:intest.in1intest.example_2.mod_3}{mod_3} \textbackslash 

{\renewcommand{\arraystretch}{1.5}
\begin{tabularx}{\textwidth}{|>{\raggedright\arraybackslash}l|X|}
\hline
\hspace{0pt}\mytexttt{\color{red} } & \textbf{v1\_m1} \\
\hline
\end{tabularx}
}

\par

\par
\begin{description}
\item [\colorbox{tagtype}{\color{white} \textbf{\textsf{OVERRIDE}}}] \textbf{\underline{}} True
\end{description}

\rule{\linewidth}{0.5pt}
\subsection*{\textsf{\colorbox{headtoc}{\color{white} ATTRIBUTE}
v2\_m3}}

\hypertarget{ecldoc:intest.in1intest.example_2.mod_3.v2_m3}{}
\hspace{0pt} \hyperlink{ecldoc:intest.in1intest.example_2}{example_2} \textbackslash 
\hspace{0pt} \hyperlink{ecldoc:intest.in1intest.example_2.mod_3}{mod_3} \textbackslash 

{\renewcommand{\arraystretch}{1.5}
\begin{tabularx}{\textwidth}{|>{\raggedright\arraybackslash}l|X|}
\hline
\hspace{0pt}\mytexttt{\color{red} } & \textbf{v2\_m3} \\
\hline
\end{tabularx}
}

\par


\rule{\linewidth}{0.5pt}


\subsection*{\textsf{\colorbox{headtoc}{\color{white} INTERFACE}
iface\_1}}

\hypertarget{ecldoc:intest.in1intest.example_2.iface_1}{}
\hspace{0pt} \hyperlink{ecldoc:intest.in1intest.example_2}{example_2} \textbackslash 

{\renewcommand{\arraystretch}{1.5}
\begin{tabularx}{\textwidth}{|>{\raggedright\arraybackslash}l|X|}
\hline
\hspace{0pt}\mytexttt{\color{red} } & \textbf{iface\_1} \\
\hline
\end{tabularx}
}

\par


\textbf{Children}
\begin{enumerate}
\item \hyperlink{ecldoc:intest.in1intest.example_2.iface_1.v1_i1}{v1\_i1}
\end{enumerate}

\rule{\linewidth}{0.5pt}

\subsection*{\textsf{\colorbox{headtoc}{\color{white} ATTRIBUTE}
v1\_i1}}

\hypertarget{ecldoc:intest.in1intest.example_2.iface_1.v1_i1}{}
\hspace{0pt} \hyperlink{ecldoc:intest.in1intest.example_2}{example_2} \textbackslash 
\hspace{0pt} \hyperlink{ecldoc:intest.in1intest.example_2.iface_1}{iface_1} \textbackslash 

{\renewcommand{\arraystretch}{1.5}
\begin{tabularx}{\textwidth}{|>{\raggedright\arraybackslash}l|X|}
\hline
\hspace{0pt}\mytexttt{\color{red} real8} & \textbf{v1\_i1} \\
\hline
\end{tabularx}
}

\par


\rule{\linewidth}{0.5pt}


\subsection*{\textsf{\colorbox{headtoc}{\color{white} MODULE}
mod\_4}}

\hypertarget{ecldoc:intest.in1intest.example_2.mod_4}{}
\hspace{0pt} \hyperlink{ecldoc:intest.in1intest.example_2}{example_2} \textbackslash 

{\renewcommand{\arraystretch}{1.5}
\begin{tabularx}{\textwidth}{|>{\raggedright\arraybackslash}l|X|}
\hline
\hspace{0pt}\mytexttt{\color{red} } & \textbf{mod\_4} \\
\hline
\end{tabularx}
}

\par


\textbf{Children}
\begin{enumerate}
\item \hyperlink{ecldoc:intest.in1intest.example_2.mod_4.v1_i1}{v1\_i1}
\item \hyperlink{ecldoc:intest.in1intest.example_2.mod_4.v2_m4}{v2\_m4}
\end{enumerate}

\rule{\linewidth}{0.5pt}

\subsection*{\textsf{\colorbox{headtoc}{\color{white} ATTRIBUTE}
v1\_i1}}

\hypertarget{ecldoc:intest.in1intest.example_2.mod_4.v1_i1}{}
\hspace{0pt} \hyperlink{ecldoc:intest.in1intest.example_2}{example_2} \textbackslash 
\hspace{0pt} \hyperlink{ecldoc:intest.in1intest.example_2.mod_4}{mod_4} \textbackslash 

{\renewcommand{\arraystretch}{1.5}
\begin{tabularx}{\textwidth}{|>{\raggedright\arraybackslash}l|X|}
\hline
\hspace{0pt}\mytexttt{\color{red} } & \textbf{v1\_i1} \\
\hline
\end{tabularx}
}

\par

\par
\begin{description}
\item [\colorbox{tagtype}{\color{white} \textbf{\textsf{OVERRIDE}}}] \textbf{\underline{}} True
\end{description}

\rule{\linewidth}{0.5pt}
\subsection*{\textsf{\colorbox{headtoc}{\color{white} ATTRIBUTE}
v2\_m4}}

\hypertarget{ecldoc:intest.in1intest.example_2.mod_4.v2_m4}{}
\hspace{0pt} \hyperlink{ecldoc:intest.in1intest.example_2}{example_2} \textbackslash 
\hspace{0pt} \hyperlink{ecldoc:intest.in1intest.example_2.mod_4}{mod_4} \textbackslash 

{\renewcommand{\arraystretch}{1.5}
\begin{tabularx}{\textwidth}{|>{\raggedright\arraybackslash}l|X|}
\hline
\hspace{0pt}\mytexttt{\color{red} STRING20} & \textbf{v2\_m4} \\
\hline
\end{tabularx}
}

\par


\rule{\linewidth}{0.5pt}




