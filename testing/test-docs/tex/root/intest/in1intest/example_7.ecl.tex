\chapter*{\color{headfile}
{\large intest\slash\hspace{0pt}}
{\large in1intest\slash\hspace{0pt}}
 \\
example_7
}
\hypertarget{ecldoc:toc:intest.in1intest.example_7}{}
\hyperlink{ecldoc:toc:root/intest/in1intest}{Go Up}


\section*{\underline{\textsf{DESCRIPTIONS}}}
\subsection*{\textsf{\colorbox{headtoc}{\color{white} MODULE}
example\_7}}

\hypertarget{ecldoc:intest.in1intest.example_7}{}

{\renewcommand{\arraystretch}{1.5}
\begin{tabularx}{\textwidth}{|>{\raggedright\arraybackslash}l|X|}
\hline
\hspace{0pt}\mytexttt{\color{red} } & \textbf{example\_7} \\
\hline
\end{tabularx}
}

\par
Basic Type Example Source Code copied from ECL Documentation


\textbf{Children}
\begin{enumerate}
\item \hyperlink{ecldoc:intest.in1intest.example_7.r}{R}
\end{enumerate}

\rule{\linewidth}{0.5pt}

\subsection*{\textsf{\colorbox{headtoc}{\color{white} RECORD}
R}}

\hypertarget{ecldoc:intest.in1intest.example_7.r}{}
\hspace{0pt} \hyperlink{ecldoc:intest.in1intest.example_7}{example_7} \textbackslash 

{\renewcommand{\arraystretch}{1.5}
\begin{tabularx}{\textwidth}{|>{\raggedright\arraybackslash}l|X|}
\hline
\hspace{0pt}\mytexttt{\color{red} } & \textbf{R} \\
\hline
\end{tabularx}
}

\par


\rule{\linewidth}{0.5pt}


