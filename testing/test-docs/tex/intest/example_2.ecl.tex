\chapter*{intest.example\_2}
\hypertarget{ecldoc:toc:intest.example_2}{}

\section*{\underline{IMPORTS}}

\section*{\underline{DESCRIPTIONS}}
\subsection*{MODULE : example\_2}
\hypertarget{ecldoc:intest.example_2}{}
\hyperlink{ecldoc:toc:intest}{Up} :

{\renewcommand{\arraystretch}{1.5}
\begin{tabularx}{\textwidth}{|>{\raggedright\arraybackslash}l|X|}
\hline
\hspace{0pt} & example\_2 \\
\hline
\end{tabularx}
}

\par
Basic Inheritance documentation : mod\_3 inherits both mod\_1 and mod\_2 . Inherits v2\_m1, v2\_m2, Overrides v1\_m1, new locals v2\_m3 . Interface Inheritance : mod\_4 inherits interface iface\_1, overrides v1\_i1


\hyperlink{ecldoc:intest.example_2.rec_1}{rec\_1}  |
\hyperlink{ecldoc:intest.example_2.rec_2}{rec\_2}  |
\hyperlink{ecldoc:intest.example_2.rec_3}{rec\_3}  |
\hyperlink{ecldoc:intest.example_2.mod_1}{mod\_1}  |
\hyperlink{ecldoc:intest.example_2.mod_2}{mod\_2}  |
\hyperlink{ecldoc:intest.example_2.mod_3}{mod\_3}  |
\hyperlink{ecldoc:intest.example_2.iface_1}{iface\_1}  |
\hyperlink{ecldoc:intest.example_2.mod_4}{mod\_4}  |

\rule{\linewidth}{0.5pt}

\subsection*{RECORD : rec\_1}
\hypertarget{ecldoc:intest.example_2.rec_1}{}
\hyperlink{ecldoc:intest.example_2}{Up} :
\hspace{0pt} \hyperlink{ecldoc:intest.example_2}{example_2} \textbackslash 

{\renewcommand{\arraystretch}{1.5}
\begin{tabularx}{\textwidth}{|>{\raggedright\arraybackslash}l|X|}
\hline
\hspace{0pt} & rec\_1 \\
\hline
\end{tabularx}
}

\par


\rule{\linewidth}{0.5pt}
\subsection*{RECORD : rec\_2}
\hypertarget{ecldoc:intest.example_2.rec_2}{}
\hyperlink{ecldoc:intest.example_2}{Up} :
\hspace{0pt} \hyperlink{ecldoc:intest.example_2}{example_2} \textbackslash 

{\renewcommand{\arraystretch}{1.5}
\begin{tabularx}{\textwidth}{|>{\raggedright\arraybackslash}l|X|}
\hline
\hspace{0pt} & rec\_2 \\
\hline
\end{tabularx}
}

\par


\rule{\linewidth}{0.5pt}
\subsection*{RECORD : rec\_3}
\hypertarget{ecldoc:intest.example_2.rec_3}{}
\hyperlink{ecldoc:intest.example_2}{Up} :
\hspace{0pt} \hyperlink{ecldoc:intest.example_2}{example_2} \textbackslash 

{\renewcommand{\arraystretch}{1.5}
\begin{tabularx}{\textwidth}{|>{\raggedright\arraybackslash}l|X|}
\hline
\hspace{0pt} & rec\_3 \\
\hline
\end{tabularx}
}

\par


\rule{\linewidth}{0.5pt}
\subsection*{MODULE : mod\_1}
\hypertarget{ecldoc:intest.example_2.mod_1}{}
\hyperlink{ecldoc:intest.example_2}{Up} :
\hspace{0pt} \hyperlink{ecldoc:intest.example_2}{example_2} \textbackslash 

{\renewcommand{\arraystretch}{1.5}
\begin{tabularx}{\textwidth}{|>{\raggedright\arraybackslash}l|X|}
\hline
\hspace{0pt} & mod\_1 \\
\hline
\end{tabularx}
}

\par


\hyperlink{ecldoc:intest.example_2.mod_1.v1_m1}{v1\_m1}  |
\hyperlink{ecldoc:intest.example_2.mod_1.v2_m1}{v2\_m1}  |

\rule{\linewidth}{0.5pt}

\subsection*{ATTRIBUTE : v1\_m1}
\hypertarget{ecldoc:intest.example_2.mod_1.v1_m1}{}
\hyperlink{ecldoc:intest.example_2.mod_1}{Up} :
\hspace{0pt} \hyperlink{ecldoc:intest.example_2}{example_2} \textbackslash 
\hspace{0pt} \hyperlink{ecldoc:intest.example_2.mod_1}{mod_1} \textbackslash 

{\renewcommand{\arraystretch}{1.5}
\begin{tabularx}{\textwidth}{|>{\raggedright\arraybackslash}l|X|}
\hline
\hspace{0pt}real8 & v1\_m1 \\
\hline
\end{tabularx}
}

\par


\rule{\linewidth}{0.5pt}
\subsection*{ATTRIBUTE : v2\_m1}
\hypertarget{ecldoc:intest.example_2.mod_1.v2_m1}{}
\hyperlink{ecldoc:intest.example_2.mod_1}{Up} :
\hspace{0pt} \hyperlink{ecldoc:intest.example_2}{example_2} \textbackslash 
\hspace{0pt} \hyperlink{ecldoc:intest.example_2.mod_1}{mod_1} \textbackslash 

{\renewcommand{\arraystretch}{1.5}
\begin{tabularx}{\textwidth}{|>{\raggedright\arraybackslash}l|X|}
\hline
\hspace{0pt} & v2\_m1 \\
\hline
\end{tabularx}
}

\par


\rule{\linewidth}{0.5pt}


\subsection*{MODULE : mod\_2}
\hypertarget{ecldoc:intest.example_2.mod_2}{}
\hyperlink{ecldoc:intest.example_2}{Up} :
\hspace{0pt} \hyperlink{ecldoc:intest.example_2}{example_2} \textbackslash 

{\renewcommand{\arraystretch}{1.5}
\begin{tabularx}{\textwidth}{|>{\raggedright\arraybackslash}l|X|}
\hline
\hspace{0pt} & mod\_2 \\
\hline
\end{tabularx}
}

\par


\hyperlink{ecldoc:intest.example_2.mod_2.v1_m1}{v1\_m1}  |
\hyperlink{ecldoc:intest.example_2.mod_2.v2_m2}{v2\_m2}  |

\rule{\linewidth}{0.5pt}

\subsection*{ATTRIBUTE : v1\_m1}
\hypertarget{ecldoc:intest.example_2.mod_2.v1_m1}{}
\hyperlink{ecldoc:intest.example_2.mod_2}{Up} :
\hspace{0pt} \hyperlink{ecldoc:intest.example_2}{example_2} \textbackslash 
\hspace{0pt} \hyperlink{ecldoc:intest.example_2.mod_2}{mod_2} \textbackslash 

{\renewcommand{\arraystretch}{1.5}
\begin{tabularx}{\textwidth}{|>{\raggedright\arraybackslash}l|X|}
\hline
\hspace{0pt} & v1\_m1 \\
\hline
\end{tabularx}
}

\par


\rule{\linewidth}{0.5pt}
\subsection*{ATTRIBUTE : v2\_m2}
\hypertarget{ecldoc:intest.example_2.mod_2.v2_m2}{}
\hyperlink{ecldoc:intest.example_2.mod_2}{Up} :
\hspace{0pt} \hyperlink{ecldoc:intest.example_2}{example_2} \textbackslash 
\hspace{0pt} \hyperlink{ecldoc:intest.example_2.mod_2}{mod_2} \textbackslash 

{\renewcommand{\arraystretch}{1.5}
\begin{tabularx}{\textwidth}{|>{\raggedright\arraybackslash}l|X|}
\hline
\hspace{0pt} & v2\_m2 \\
\hline
\end{tabularx}
}

\par


\rule{\linewidth}{0.5pt}


\subsection*{MODULE : mod\_3}
\hypertarget{ecldoc:intest.example_2.mod_3}{}
\hyperlink{ecldoc:intest.example_2}{Up} :
\hspace{0pt} \hyperlink{ecldoc:intest.example_2}{example_2} \textbackslash 

{\renewcommand{\arraystretch}{1.5}
\begin{tabularx}{\textwidth}{|>{\raggedright\arraybackslash}l|X|}
\hline
\hspace{0pt} & mod\_3 \\
\hline
\end{tabularx}
}

\par


\hyperlink{ecldoc:intest.example_2.mod_1.v2_m1}{v2\_m1}  |
\hyperlink{ecldoc:intest.example_2.mod_2.v2_m2}{v2\_m2}  |
\hyperlink{ecldoc:intest.example_2.mod_3.v1_m1}{v1\_m1}  |
\hyperlink{ecldoc:intest.example_2.mod_3.v2_m3}{v2\_m3}  |

\rule{\linewidth}{0.5pt}

\subsection*{ATTRIBUTE : v2\_m1}
\hypertarget{ecldoc:intest.example_2.mod_1.v2_m1}{}
\hyperlink{ecldoc:intest.example_2.mod_3}{Up} :
\hspace{0pt} \hyperlink{ecldoc:intest.example_2}{example_2} \textbackslash 
\hspace{0pt} \hyperlink{ecldoc:intest.example_2.mod_3}{mod_3} \textbackslash 

{\renewcommand{\arraystretch}{1.5}
\begin{tabularx}{\textwidth}{|>{\raggedright\arraybackslash}l|X|}
\hline
\hspace{0pt} & v2\_m1 \\
\hline
\end{tabularx}
}

\par

\par
\begin{description}
\item [\textbf{INHERITED}] True
\end{description}

\rule{\linewidth}{0.5pt}
\subsection*{ATTRIBUTE : v2\_m2}
\hypertarget{ecldoc:intest.example_2.mod_2.v2_m2}{}
\hyperlink{ecldoc:intest.example_2.mod_3}{Up} :
\hspace{0pt} \hyperlink{ecldoc:intest.example_2}{example_2} \textbackslash 
\hspace{0pt} \hyperlink{ecldoc:intest.example_2.mod_3}{mod_3} \textbackslash 

{\renewcommand{\arraystretch}{1.5}
\begin{tabularx}{\textwidth}{|>{\raggedright\arraybackslash}l|X|}
\hline
\hspace{0pt} & v2\_m2 \\
\hline
\end{tabularx}
}

\par

\par
\begin{description}
\item [\textbf{INHERITED}] True
\end{description}

\rule{\linewidth}{0.5pt}
\subsection*{ATTRIBUTE : v1\_m1}
\hypertarget{ecldoc:intest.example_2.mod_3.v1_m1}{}
\hyperlink{ecldoc:intest.example_2.mod_3}{Up} :
\hspace{0pt} \hyperlink{ecldoc:intest.example_2}{example_2} \textbackslash 
\hspace{0pt} \hyperlink{ecldoc:intest.example_2.mod_3}{mod_3} \textbackslash 

{\renewcommand{\arraystretch}{1.5}
\begin{tabularx}{\textwidth}{|>{\raggedright\arraybackslash}l|X|}
\hline
\hspace{0pt} & v1\_m1 \\
\hline
\end{tabularx}
}

\par

\par
\begin{description}
\item [\textbf{OVERRIDE}] True
\end{description}

\rule{\linewidth}{0.5pt}
\subsection*{ATTRIBUTE : v2\_m3}
\hypertarget{ecldoc:intest.example_2.mod_3.v2_m3}{}
\hyperlink{ecldoc:intest.example_2.mod_3}{Up} :
\hspace{0pt} \hyperlink{ecldoc:intest.example_2}{example_2} \textbackslash 
\hspace{0pt} \hyperlink{ecldoc:intest.example_2.mod_3}{mod_3} \textbackslash 

{\renewcommand{\arraystretch}{1.5}
\begin{tabularx}{\textwidth}{|>{\raggedright\arraybackslash}l|X|}
\hline
\hspace{0pt} & v2\_m3 \\
\hline
\end{tabularx}
}

\par


\rule{\linewidth}{0.5pt}


\subsection*{INTERFACE : iface\_1}
\hypertarget{ecldoc:intest.example_2.iface_1}{}
\hyperlink{ecldoc:intest.example_2}{Up} :
\hspace{0pt} \hyperlink{ecldoc:intest.example_2}{example_2} \textbackslash 

{\renewcommand{\arraystretch}{1.5}
\begin{tabularx}{\textwidth}{|>{\raggedright\arraybackslash}l|X|}
\hline
\hspace{0pt} & iface\_1 \\
\hline
\end{tabularx}
}

\par


\hyperlink{ecldoc:intest.example_2.iface_1.v1_i1}{v1\_i1}  |

\rule{\linewidth}{0.5pt}

\subsection*{ATTRIBUTE : v1\_i1}
\hypertarget{ecldoc:intest.example_2.iface_1.v1_i1}{}
\hyperlink{ecldoc:intest.example_2.iface_1}{Up} :
\hspace{0pt} \hyperlink{ecldoc:intest.example_2}{example_2} \textbackslash 
\hspace{0pt} \hyperlink{ecldoc:intest.example_2.iface_1}{iface_1} \textbackslash 

{\renewcommand{\arraystretch}{1.5}
\begin{tabularx}{\textwidth}{|>{\raggedright\arraybackslash}l|X|}
\hline
\hspace{0pt}real8 & v1\_i1 \\
\hline
\end{tabularx}
}

\par


\rule{\linewidth}{0.5pt}


\subsection*{MODULE : mod\_4}
\hypertarget{ecldoc:intest.example_2.mod_4}{}
\hyperlink{ecldoc:intest.example_2}{Up} :
\hspace{0pt} \hyperlink{ecldoc:intest.example_2}{example_2} \textbackslash 

{\renewcommand{\arraystretch}{1.5}
\begin{tabularx}{\textwidth}{|>{\raggedright\arraybackslash}l|X|}
\hline
\hspace{0pt} & mod\_4 \\
\hline
\end{tabularx}
}

\par


\hyperlink{ecldoc:intest.example_2.mod_4.v1_i1}{v1\_i1}  |
\hyperlink{ecldoc:intest.example_2.mod_4.v2_m4}{v2\_m4}  |

\rule{\linewidth}{0.5pt}

\subsection*{ATTRIBUTE : v1\_i1}
\hypertarget{ecldoc:intest.example_2.mod_4.v1_i1}{}
\hyperlink{ecldoc:intest.example_2.mod_4}{Up} :
\hspace{0pt} \hyperlink{ecldoc:intest.example_2}{example_2} \textbackslash 
\hspace{0pt} \hyperlink{ecldoc:intest.example_2.mod_4}{mod_4} \textbackslash 

{\renewcommand{\arraystretch}{1.5}
\begin{tabularx}{\textwidth}{|>{\raggedright\arraybackslash}l|X|}
\hline
\hspace{0pt} & v1\_i1 \\
\hline
\end{tabularx}
}

\par

\par
\begin{description}
\item [\textbf{OVERRIDE}] True
\end{description}

\rule{\linewidth}{0.5pt}
\subsection*{ATTRIBUTE : v2\_m4}
\hypertarget{ecldoc:intest.example_2.mod_4.v2_m4}{}
\hyperlink{ecldoc:intest.example_2.mod_4}{Up} :
\hspace{0pt} \hyperlink{ecldoc:intest.example_2}{example_2} \textbackslash 
\hspace{0pt} \hyperlink{ecldoc:intest.example_2.mod_4}{mod_4} \textbackslash 

{\renewcommand{\arraystretch}{1.5}
\begin{tabularx}{\textwidth}{|>{\raggedright\arraybackslash}l|X|}
\hline
\hspace{0pt}STRING20 & v2\_m4 \\
\hline
\end{tabularx}
}

\par


\rule{\linewidth}{0.5pt}




