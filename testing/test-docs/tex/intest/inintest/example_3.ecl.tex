\chapter*{intest.inintest.example\_3}
\hypertarget{ecldoc:toc:intest.inintest.example_3}{}

\section*{\underline{IMPORTS}}
\begin{itemize}
\item std.Str
\end{itemize}

\section*{\underline{DESCRIPTIONS}}
\subsection*{MODULE : Example\_3}
\hypertarget{ecldoc:intest.inintest.Example_3}{}
\hyperlink{ecldoc:toc:intest/inintest}{Up} :

{\renewcommand{\arraystretch}{1.5}
\begin{tabularx}{\textwidth}{|>{\raggedright\arraybackslash}l|X|}
\hline
\hspace{0pt} & Example\_3 \\
\hline
\end{tabularx}
}

\par
Example : Inheritance across files mod\_1 in Example\_4 inherits mod\_1 in Example\_3


\hyperlink{ecldoc:intest.inintest.Example_3.mod_1}{mod\_1}  |

\rule{\linewidth}{0.5pt}

\subsection*{MODULE : mod\_1}
\hypertarget{ecldoc:intest.inintest.Example_3.mod_1}{}
\hyperlink{ecldoc:intest.inintest.Example_3}{Up} :
\hspace{0pt} \hyperlink{ecldoc:intest.inintest.Example_3}{Example_3} \textbackslash 

{\renewcommand{\arraystretch}{1.5}
\begin{tabularx}{\textwidth}{|>{\raggedright\arraybackslash}l|X|}
\hline
\hspace{0pt} & mod\_1 \\
\hline
\end{tabularx}
}

\par


\hyperlink{ecldoc:intest.inintest.example_3.mod_1.v1_m1}{v1\_m1}  |
\hyperlink{ecldoc:intest.inintest.example_3.mod_1.v2_m1_ex3}{v2\_m1\_ex3}  |

\rule{\linewidth}{0.5pt}

\subsection*{ATTRIBUTE : v1\_m1}
\hypertarget{ecldoc:intest.inintest.example_3.mod_1.v1_m1}{}
\hyperlink{ecldoc:intest.inintest.Example_3.mod_1}{Up} :
\hspace{0pt} \hyperlink{ecldoc:intest.inintest.Example_3}{Example_3} \textbackslash 
\hspace{0pt} \hyperlink{ecldoc:intest.inintest.Example_3.mod_1}{mod_1} \textbackslash 

{\renewcommand{\arraystretch}{1.5}
\begin{tabularx}{\textwidth}{|>{\raggedright\arraybackslash}l|X|}
\hline
\hspace{0pt} & v1\_m1 \\
\hline
\end{tabularx}
}

\par


\rule{\linewidth}{0.5pt}
\subsection*{ATTRIBUTE : v2\_m1\_ex3}
\hypertarget{ecldoc:intest.inintest.example_3.mod_1.v2_m1_ex3}{}
\hyperlink{ecldoc:intest.inintest.Example_3.mod_1}{Up} :
\hspace{0pt} \hyperlink{ecldoc:intest.inintest.Example_3}{Example_3} \textbackslash 
\hspace{0pt} \hyperlink{ecldoc:intest.inintest.Example_3.mod_1}{mod_1} \textbackslash 

{\renewcommand{\arraystretch}{1.5}
\begin{tabularx}{\textwidth}{|>{\raggedright\arraybackslash}l|X|}
\hline
\hspace{0pt} & v2\_m1\_ex3 \\
\hline
\end{tabularx}
}

\par


\rule{\linewidth}{0.5pt}




