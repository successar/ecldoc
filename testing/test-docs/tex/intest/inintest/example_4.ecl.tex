\chapter*{intest.inintest.example\_4}
\hypertarget{ecldoc:toc:intest.inintest.example_4}{}

\section*{\underline{IMPORTS}}
\begin{itemize}
\item Example\_3.mod\_1
\end{itemize}

\section*{\underline{DESCRIPTIONS}}
\subsection*{MODULE : example\_4}
\hypertarget{ecldoc:intest.inintest.example_4}{}

{\renewcommand{\arraystretch}{1.5}
\begin{tabularx}{\textwidth}{|>{\raggedright\arraybackslash}l|X|}
\hline
\hspace{0pt} & example\_4 \\
\hline
\end{tabularx}
}

\hyperlink{ecldoc:toc:intest/inintest}{Up}

\par
Example : Inheritance across files mod\_1 in Example\_4 inherits mod\_1 in Example\_3


\hyperlink{ecldoc:intest.inintest.example_4.mod_1}{mod\_1}  |

\rule{\textwidth}{0.4pt}

\subsection*{MODULE : mod\_1}
\hypertarget{ecldoc:intest.inintest.example_4.mod_1}{}

{\renewcommand{\arraystretch}{1.5}
\begin{tabularx}{\textwidth}{|>{\raggedright\arraybackslash}l|X|}
\hline
\hspace{0pt} & mod\_1 \\
\hline
\end{tabularx}
}

\hyperlink{ecldoc:intest.inintest.example_4}{Up}

\par


\hyperlink{ecldoc:intest.inintest.example_4.mod_1.v2_m1_ex4}{v2\_m1\_ex4}  |
\hyperlink{ecldoc:example_3.mod_1.v1_m1}{v1\_m1}  |
\hyperlink{ecldoc:example_3.mod_1.v2_m1_ex3}{v2\_m1\_ex3}  |
\hyperlink{ecldoc:example_3.mod_1.long_name}{long\_name}  |

\rule{\textwidth}{0.4pt}

\subsection*{ATTRIBUTE : v2\_m1\_ex4}
\hypertarget{ecldoc:intest.inintest.example_4.mod_1.v2_m1_ex4}{}

{\renewcommand{\arraystretch}{1.5}
\begin{tabularx}{\textwidth}{|>{\raggedright\arraybackslash}l|X|}
\hline
\hspace{0pt} & v2\_m1\_ex4 \\
\hline
\end{tabularx}
}

\hyperlink{ecldoc:intest.inintest.example_4.mod_1}{Up}

\par


\rule{\textwidth}{0.4pt}
\subsection*{ATTRIBUTE : v1\_m1}
\hypertarget{ecldoc:example_3.mod_1.v1_m1}{}

{\renewcommand{\arraystretch}{1.5}
\begin{tabularx}{\textwidth}{|>{\raggedright\arraybackslash}l|X|}
\hline
\hspace{0pt} & v1\_m1 \\
\hline
\end{tabularx}
}

\hyperlink{ecldoc:intest.inintest.example_4.mod_1}{Up}

\par
Doc test 2. Title end by period not newline 
\begin{verbatim}

 ABCD ||||
 CDEF ||||\end{verbatim}



\par
\begin{description}
\item [\textbf{INHERITED}] True
\end{description}

\rule{\textwidth}{0.4pt}
\subsection*{ATTRIBUTE : v2\_m1\_ex3}
\hypertarget{ecldoc:example_3.mod_1.v2_m1_ex3}{}

{\renewcommand{\arraystretch}{1.5}
\begin{tabularx}{\textwidth}{|>{\raggedright\arraybackslash}l|X|}
\hline
\hspace{0pt} & v2\_m1\_ex3 \\
\hline
\end{tabularx}
}

\hyperlink{ecldoc:intest.inintest.example_4.mod_1}{Up}

\par
DOC Test 3 No Period title

\par
\begin{description}
\item [\textbf{INHERITED}] True
\end{description}

\rule{\textwidth}{0.4pt}
\subsection*{FUNCTION : long\_name}
\hypertarget{ecldoc:example_3.mod_1.long_name}{}

{\renewcommand{\arraystretch}{1.5}
\begin{tabularx}{\textwidth}{|>{\raggedright\arraybackslash}l|X|}
\hline
\hspace{0pt} & long\_name \\
\hline
\multicolumn{2}{|>{\raggedright\arraybackslash}X|}{\hspace{0pt}(DATASET(\{REAL8 u\}) X, DATASET(\{REAL8 u\}) IntW, DATASET(\{REAL8 u\}) Intb, REAL8 BETA=0.1, REAL8 sparsityParam=0.1 , REAL8 LAMBDA=0.001, REAL8 ALPHA=0.1, UNSIGNED2 MaxIter=100)} \\
\hline
\end{tabularx}
}

\hyperlink{ecldoc:intest.inintest.example_4.mod_1}{Up}

\par

\par
\begin{description}
\item [\textbf{INHERITED}] True
\end{description}

\rule{\textwidth}{0.4pt}




