\chapter*{Math}
\hypertarget{ecldoc:toc:Math}{}

\section*{\underline{IMPORTS}}

\section*{\underline{DESCRIPTIONS}}
\subsection*{MODULE : Math}
\hypertarget{ecldoc:Math}{}

{\renewcommand{\arraystretch}{1.5}
\begin{tabularx}{\textwidth}{|>{\raggedright\arraybackslash}l|X|}
\hline
\hspace{0pt} & Math \\
\hline
\end{tabularx}
}

\hyperlink{ecldoc:toc:root}{Up}

\par


\hyperlink{ecldoc:math.infinity}{Infinity}  |
\hyperlink{ecldoc:math.nan}{NaN}  |
\hyperlink{ecldoc:math.isinfinite}{isInfinite}  |
\hyperlink{ecldoc:math.isnan}{isNaN}  |
\hyperlink{ecldoc:math.isfinite}{isFinite}  |
\hyperlink{ecldoc:math.fmod}{FMod}  |
\hyperlink{ecldoc:math.fmatch}{FMatch}  |

\rule{\textwidth}{0.4pt}

\subsection*{ATTRIBUTE : Infinity}
\hypertarget{ecldoc:math.infinity}{}

{\renewcommand{\arraystretch}{1.5}
\begin{tabularx}{\textwidth}{|>{\raggedright\arraybackslash}l|X|}
\hline
\hspace{0pt}REAL8 & Infinity \\
\hline
\end{tabularx}
}

\hyperlink{ecldoc:Math}{Up}

\par
Return a real ''infinity'' value.


\rule{\textwidth}{0.4pt}
\subsection*{ATTRIBUTE : NaN}
\hypertarget{ecldoc:math.nan}{}

{\renewcommand{\arraystretch}{1.5}
\begin{tabularx}{\textwidth}{|>{\raggedright\arraybackslash}l|X|}
\hline
\hspace{0pt}REAL8 & NaN \\
\hline
\end{tabularx}
}

\hyperlink{ecldoc:Math}{Up}

\par
Return a non-signalling NaN (Not a Number)value.


\rule{\textwidth}{0.4pt}
\subsection*{FUNCTION : isInfinite}
\hypertarget{ecldoc:math.isinfinite}{}

{\renewcommand{\arraystretch}{1.5}
\begin{tabularx}{\textwidth}{|>{\raggedright\arraybackslash}l|X|}
\hline
\hspace{0pt}BOOLEAN & isInfinite \\
\hline
\multicolumn{2}{|>{\raggedright\arraybackslash}X|}{\hspace{0pt}(REAL8 val)} \\
\hline
\end{tabularx}
}

\hyperlink{ecldoc:Math}{Up}

\par
Return whether a real value is infinite (positive or negative).

\par
\begin{description}
\item [\textbf{Parameter}] val ||| The value to test.
\end{description}

\rule{\textwidth}{0.4pt}
\subsection*{FUNCTION : isNaN}
\hypertarget{ecldoc:math.isnan}{}

{\renewcommand{\arraystretch}{1.5}
\begin{tabularx}{\textwidth}{|>{\raggedright\arraybackslash}l|X|}
\hline
\hspace{0pt}BOOLEAN & isNaN \\
\hline
\multicolumn{2}{|>{\raggedright\arraybackslash}X|}{\hspace{0pt}(REAL8 val)} \\
\hline
\end{tabularx}
}

\hyperlink{ecldoc:Math}{Up}

\par
Return whether a real value is a NaN (not a number) value.

\par
\begin{description}
\item [\textbf{Parameter}] val ||| The value to test.
\end{description}

\rule{\textwidth}{0.4pt}
\subsection*{FUNCTION : isFinite}
\hypertarget{ecldoc:math.isfinite}{}

{\renewcommand{\arraystretch}{1.5}
\begin{tabularx}{\textwidth}{|>{\raggedright\arraybackslash}l|X|}
\hline
\hspace{0pt}BOOLEAN & isFinite \\
\hline
\multicolumn{2}{|>{\raggedright\arraybackslash}X|}{\hspace{0pt}(REAL8 val)} \\
\hline
\end{tabularx}
}

\hyperlink{ecldoc:Math}{Up}

\par
Return whether a real value is a valid value (neither infinite not NaN).

\par
\begin{description}
\item [\textbf{Parameter}] val ||| The value to test.
\end{description}

\rule{\textwidth}{0.4pt}
\subsection*{FUNCTION : FMod}
\hypertarget{ecldoc:math.fmod}{}

{\renewcommand{\arraystretch}{1.5}
\begin{tabularx}{\textwidth}{|>{\raggedright\arraybackslash}l|X|}
\hline
\hspace{0pt}REAL8 & FMod \\
\hline
\multicolumn{2}{|>{\raggedright\arraybackslash}X|}{\hspace{0pt}(REAL8 numer, REAL8 denom)} \\
\hline
\end{tabularx}
}

\hyperlink{ecldoc:Math}{Up}

\par
Returns the floating-point remainder of numer/denom (rounded towards zero). If denom is zero, the result depends on the -fdivideByZero flag: 'zero' or unset: return zero. 'nan': return a non-signalling NaN value 'fail': throw an exception

\par
\begin{description}
\item [\textbf{Parameter}] numer ||| The numerator.
\item [\textbf{Parameter}] denom ||| The numerator.
\end{description}

\rule{\textwidth}{0.4pt}
\subsection*{FUNCTION : FMatch}
\hypertarget{ecldoc:math.fmatch}{}

{\renewcommand{\arraystretch}{1.5}
\begin{tabularx}{\textwidth}{|>{\raggedright\arraybackslash}l|X|}
\hline
\hspace{0pt}BOOLEAN & FMatch \\
\hline
\multicolumn{2}{|>{\raggedright\arraybackslash}X|}{\hspace{0pt}(REAL8 a, REAL8 b, REAL8 epsilon=0.0)} \\
\hline
\end{tabularx}
}

\hyperlink{ecldoc:Math}{Up}

\par
Returns whether two floating point values are the same, within margin of error epsilon.

\par
\begin{description}
\item [\textbf{Parameter}] a ||| The first value.
\item [\textbf{Parameter}] b ||| The second value.
\item [\textbf{Parameter}] epsilon ||| The allowable margin of error.
\end{description}

\rule{\textwidth}{0.4pt}


