\chapter*{\color{headfile}
File
}
\hypertarget{ecldoc:toc:File}{}
\hyperlink{ecldoc:toc:root}{Go Up}

\section*{\underline{\textsf{IMPORTS}}}
\begin{doublespace}
{\large
lib\_fileservices |
}
\end{doublespace}

\section*{\underline{\textsf{DESCRIPTIONS}}}
\subsection*{\textsf{\colorbox{headtoc}{\color{white} MODULE}
File}}

\hypertarget{ecldoc:File}{}

{\renewcommand{\arraystretch}{1.5}
\begin{tabularx}{\textwidth}{|>{\raggedright\arraybackslash}l|X|}
\hline
\hspace{0pt}\mytexttt{\color{red} } & \textbf{File} \\
\hline
\end{tabularx}
}

\par





No Documentation Found







\textbf{Children}
\begin{enumerate}
\item \hyperlink{ecldoc:file.fsfilenamerecord}{FsFilenameRecord}
: A record containing information about filename
\item \hyperlink{ecldoc:file.fslogicalfilename}{FsLogicalFileName}
: An alias for a logical filename that is stored in a row
\item \hyperlink{ecldoc:file.fslogicalfilenamerecord}{FsLogicalFileNameRecord}
: A record containing a logical filename
\item \hyperlink{ecldoc:file.fslogicalfileinforecord}{FsLogicalFileInfoRecord}
: A record containing information about a logical file
\item \hyperlink{ecldoc:file.fslogicalsupersubrecord}{FsLogicalSuperSubRecord}
: A record containing information about a superfile and its contents
\item \hyperlink{ecldoc:file.fsfilerelationshiprecord}{FsFileRelationshipRecord}
: A record containing information about the relationship between two files
\item \hyperlink{ecldoc:file.recfmv_recsize}{RECFMV\_RECSIZE}
: Constant that indicates IBM RECFM V format file
\item \hyperlink{ecldoc:file.recfmvb_recsize}{RECFMVB\_RECSIZE}
: Constant that indicates IBM RECFM VB format file
\item \hyperlink{ecldoc:file.prefix_variable_recsize}{PREFIX\_VARIABLE\_RECSIZE}
: Constant that indicates a variable little endian 4 byte length prefixed file
\item \hyperlink{ecldoc:file.prefix_variable_bigendian_recsize}{PREFIX\_VARIABLE\_BIGENDIAN\_RECSIZE}
: Constant that indicates a variable big endian 4 byte length prefixed file
\item \hyperlink{ecldoc:file.fileexists}{FileExists}
: Returns whether the file exists
\item \hyperlink{ecldoc:file.deletelogicalfile}{DeleteLogicalFile}
: Removes the logical file from the system, and deletes from the disk
\item \hyperlink{ecldoc:file.setreadonly}{SetReadOnly}
: Changes whether access to a file is read only or not
\item \hyperlink{ecldoc:file.renamelogicalfile}{RenameLogicalFile}
: Changes the name of a logical file
\item \hyperlink{ecldoc:file.foreignlogicalfilename}{ForeignLogicalFileName}
: Returns a logical filename that can be used to refer to a logical file in a local or remote dali
\item \hyperlink{ecldoc:file.externallogicalfilename}{ExternalLogicalFileName}
: Returns an encoded logical filename that can be used to refer to a external file
\item \hyperlink{ecldoc:file.getfiledescription}{GetFileDescription}
: Returns a string containing the description information associated with the specified filename
\item \hyperlink{ecldoc:file.setfiledescription}{SetFileDescription}
: Sets the description associated with the specified filename
\item \hyperlink{ecldoc:file.remotedirectory}{RemoteDirectory}
: Returns a dataset containing a list of files from the specified machineIP and directory
\item \hyperlink{ecldoc:file.logicalfilelist}{LogicalFileList}
: Returns a dataset of information about the logical files known to the system
\item \hyperlink{ecldoc:file.comparefiles}{CompareFiles}
: Compares two files, and returns a result indicating how well they match
\item \hyperlink{ecldoc:file.verifyfile}{VerifyFile}
: Checks the system datastore (Dali) information for the file against the physical parts on disk
\item \hyperlink{ecldoc:file.addfilerelationship}{AddFileRelationship}
: Defines the relationship between two files
\item \hyperlink{ecldoc:file.filerelationshiplist}{FileRelationshipList}
: Returns a dataset of relationships
\item \hyperlink{ecldoc:file.removefilerelationship}{RemoveFileRelationship}
: Removes a relationship between two files
\item \hyperlink{ecldoc:file.getcolumnmapping}{GetColumnMapping}
: Returns the field mappings for the file, in the same format specified for the SetColumnMapping function
\item \hyperlink{ecldoc:file.setcolumnmapping}{SetColumnMapping}
: Defines how the data in the fields of the file mist be transformed between the actual data storage format and the input format used to query that data
\item \hyperlink{ecldoc:file.encoderfsquery}{EncodeRfsQuery}
: Returns a string that can be used in a DATASET declaration to read data from an RFS (Remote File Server) instance (e.g
\item \hyperlink{ecldoc:file.rfsaction}{RfsAction}
: Sends the query to the rfs server
\item \hyperlink{ecldoc:file.moveexternalfile}{MoveExternalFile}
: Moves the single physical file between two locations on the same remote machine
\item \hyperlink{ecldoc:file.deleteexternalfile}{DeleteExternalFile}
: Removes a single physical file from a remote machine
\item \hyperlink{ecldoc:file.createexternaldirectory}{CreateExternalDirectory}
: Creates the path on the location (if it does not already exist)
\item \hyperlink{ecldoc:file.getlogicalfileattribute}{GetLogicalFileAttribute}
: Returns the value of the given attribute for the specified logicalfilename
\item \hyperlink{ecldoc:file.protectlogicalfile}{ProtectLogicalFile}
: Toggles protection on and off for the specified logicalfilename
\item \hyperlink{ecldoc:file.dfuplusexec}{DfuPlusExec}
: The DfuPlusExec action executes the specified command line just as the DfuPLus.exe program would do
\item \hyperlink{ecldoc:file.fsprayfixed}{fSprayFixed}
: Sprays a file of fixed length records from a single machine and distributes it across the nodes of the destination group
\item \hyperlink{ecldoc:file.sprayfixed}{SprayFixed}
: Same as fSprayFixed, but does not return the DFU Workunit ID
\item \hyperlink{ecldoc:file.fsprayvariable}{fSprayVariable}
: No Documentation Found
\item \hyperlink{ecldoc:file.sprayvariable}{SprayVariable}
: No Documentation Found
\item \hyperlink{ecldoc:file.fspraydelimited}{fSprayDelimited}
: Sprays a file of fixed delimited records from a single machine and distributes it across the nodes of the destination group
\item \hyperlink{ecldoc:file.spraydelimited}{SprayDelimited}
: Same as fSprayDelimited, but does not return the DFU Workunit ID
\item \hyperlink{ecldoc:file.fsprayxml}{fSprayXml}
: Sprays an xml file from a single machine and distributes it across the nodes of the destination group
\item \hyperlink{ecldoc:file.sprayxml}{SprayXml}
: Same as fSprayXml, but does not return the DFU Workunit ID
\item \hyperlink{ecldoc:file.fdespray}{fDespray}
: Copies a distributed file from multiple machines, and desprays it to a single file on a single machine
\item \hyperlink{ecldoc:file.despray}{Despray}
: Same as fDespray, but does not return the DFU Workunit ID
\item \hyperlink{ecldoc:file.fcopy}{fCopy}
: Copies a distributed file to another distributed file
\item \hyperlink{ecldoc:file.copy}{Copy}
: Same as fCopy, but does not return the DFU Workunit ID
\item \hyperlink{ecldoc:file.freplicate}{fReplicate}
: Ensures the specified file is replicated to its mirror copies
\item \hyperlink{ecldoc:file.replicate}{Replicate}
: Same as fReplicated, but does not return the DFU Workunit ID
\item \hyperlink{ecldoc:file.fremotepull}{fRemotePull}
: Copies a distributed file to a distributed file on remote system
\item \hyperlink{ecldoc:file.remotepull}{RemotePull}
: Same as fRemotePull, but does not return the DFU Workunit ID
\item \hyperlink{ecldoc:file.fmonitorlogicalfilename}{fMonitorLogicalFileName}
: Creates a file monitor job in the DFU Server
\item \hyperlink{ecldoc:file.monitorlogicalfilename}{MonitorLogicalFileName}
: Same as fMonitorLogicalFileName, but does not return the DFU Workunit ID
\item \hyperlink{ecldoc:file.fmonitorfile}{fMonitorFile}
: Creates a file monitor job in the DFU Server
\item \hyperlink{ecldoc:file.monitorfile}{MonitorFile}
: Same as fMonitorFile, but does not return the DFU Workunit ID
\item \hyperlink{ecldoc:file.waitdfuworkunit}{WaitDfuWorkunit}
: Waits for the specified DFU workunit to finish
\item \hyperlink{ecldoc:file.abortdfuworkunit}{AbortDfuWorkunit}
: Aborts the specified DFU workunit
\item \hyperlink{ecldoc:file.createsuperfile}{CreateSuperFile}
: Creates an empty superfile
\item \hyperlink{ecldoc:file.superfileexists}{SuperFileExists}
: Checks if the specified filename is present in the Distributed File Utility (DFU) and is a SuperFile
\item \hyperlink{ecldoc:file.deletesuperfile}{DeleteSuperFile}
: Deletes the superfile
\item \hyperlink{ecldoc:file.getsuperfilesubcount}{GetSuperFileSubCount}
: Returns the number of sub-files contained within a superfile
\item \hyperlink{ecldoc:file.getsuperfilesubname}{GetSuperFileSubName}
: Returns the name of the Nth sub-file within a superfile
\item \hyperlink{ecldoc:file.findsuperfilesubname}{FindSuperFileSubName}
: Returns the position of a file within a superfile
\item \hyperlink{ecldoc:file.startsuperfiletransaction}{StartSuperFileTransaction}
: Starts a superfile transaction
\item \hyperlink{ecldoc:file.addsuperfile}{AddSuperFile}
: Adds a file to a superfile
\item \hyperlink{ecldoc:file.removesuperfile}{RemoveSuperFile}
: Removes a sub-file from a superfile
\item \hyperlink{ecldoc:file.clearsuperfile}{ClearSuperFile}
: Removes all sub-files from a superfile
\item \hyperlink{ecldoc:file.removeownedsubfiles}{RemoveOwnedSubFiles}
: Removes all soley-owned sub-files from a superfile
\item \hyperlink{ecldoc:file.deleteownedsubfiles}{DeleteOwnedSubFiles}
: Legacy version of RemoveOwnedSubFiles which was incorrectly named in a previous version
\item \hyperlink{ecldoc:file.swapsuperfile}{SwapSuperFile}
: Swap the contents of two superfiles
\item \hyperlink{ecldoc:file.replacesuperfile}{ReplaceSuperFile}
: Removes a sub-file from a superfile and replaces it with another
\item \hyperlink{ecldoc:file.finishsuperfiletransaction}{FinishSuperFileTransaction}
: Finishes a superfile transaction
\item \hyperlink{ecldoc:file.superfilecontents}{SuperFileContents}
: Returns the list of sub-files contained within a superfile
\item \hyperlink{ecldoc:file.logicalfilesuperowners}{LogicalFileSuperOwners}
: Returns the list of superfiles that a logical file is contained within
\item \hyperlink{ecldoc:file.logicalfilesupersublist}{LogicalFileSuperSubList}
: Returns the list of all the superfiles in the system and their component sub-files
\item \hyperlink{ecldoc:file.fpromotesuperfilelist}{fPromoteSuperFileList}
: Moves the sub-files from the first entry in the list of superfiles to the next in the list, repeating the process through the list of superfiles
\item \hyperlink{ecldoc:file.promotesuperfilelist}{PromoteSuperFileList}
: Same as fPromoteSuperFileList, but does not return the DFU Workunit ID
\end{enumerate}

\rule{\linewidth}{0.5pt}

\subsection*{\textsf{\colorbox{headtoc}{\color{white} RECORD}
FsFilenameRecord}}

\hypertarget{ecldoc:file.fsfilenamerecord}{}
\hspace{0pt} \hyperlink{ecldoc:File}{File} \textbackslash 

{\renewcommand{\arraystretch}{1.5}
\begin{tabularx}{\textwidth}{|>{\raggedright\arraybackslash}l|X|}
\hline
\hspace{0pt}\mytexttt{\color{red} } & \textbf{FsFilenameRecord} \\
\hline
\end{tabularx}
}

\par





A record containing information about filename. Includes name, size and when last modified. export FsFilenameRecord := RECORD string name; integer8 size; string19 modified; END;







\par
\begin{description}
\item [\colorbox{tagtype}{\color{white} \textbf{\textsf{FIELD}}}] \textbf{\underline{size}} ||| INTEGER8 --- No Doc
\item [\colorbox{tagtype}{\color{white} \textbf{\textsf{FIELD}}}] \textbf{\underline{modified}} ||| STRING19 --- No Doc
\item [\colorbox{tagtype}{\color{white} \textbf{\textsf{FIELD}}}] \textbf{\underline{name}} ||| STRING --- No Doc
\end{description}





\rule{\linewidth}{0.5pt}
\subsection*{\textsf{\colorbox{headtoc}{\color{white} ATTRIBUTE}
FsLogicalFileName}}

\hypertarget{ecldoc:file.fslogicalfilename}{}
\hspace{0pt} \hyperlink{ecldoc:File}{File} \textbackslash 

{\renewcommand{\arraystretch}{1.5}
\begin{tabularx}{\textwidth}{|>{\raggedright\arraybackslash}l|X|}
\hline
\hspace{0pt}\mytexttt{\color{red} } & \textbf{FsLogicalFileName} \\
\hline
\end{tabularx}
}

\par





An alias for a logical filename that is stored in a row.








\par
\begin{description}
\item [\colorbox{tagtype}{\color{white} \textbf{\textsf{RETURN}}}] \textbf{STRING} --- 
\end{description}




\rule{\linewidth}{0.5pt}
\subsection*{\textsf{\colorbox{headtoc}{\color{white} RECORD}
FsLogicalFileNameRecord}}

\hypertarget{ecldoc:file.fslogicalfilenamerecord}{}
\hspace{0pt} \hyperlink{ecldoc:File}{File} \textbackslash 

{\renewcommand{\arraystretch}{1.5}
\begin{tabularx}{\textwidth}{|>{\raggedright\arraybackslash}l|X|}
\hline
\hspace{0pt}\mytexttt{\color{red} } & \textbf{FsLogicalFileNameRecord} \\
\hline
\end{tabularx}
}

\par





A record containing a logical filename. It contains the following fields:







\par
\begin{description}
\item [\colorbox{tagtype}{\color{white} \textbf{\textsf{FIELD}}}] \textbf{\underline{name}} ||| STRING --- The logical name of the file;
\end{description}





\rule{\linewidth}{0.5pt}
\subsection*{\textsf{\colorbox{headtoc}{\color{white} RECORD}
FsLogicalFileInfoRecord}}

\hypertarget{ecldoc:file.fslogicalfileinforecord}{}
\hspace{0pt} \hyperlink{ecldoc:File}{File} \textbackslash 

{\renewcommand{\arraystretch}{1.5}
\begin{tabularx}{\textwidth}{|>{\raggedright\arraybackslash}l|X|}
\hline
\hspace{0pt}\mytexttt{\color{red} } & \textbf{FsLogicalFileInfoRecord} \\
\hline
\end{tabularx}
}

\par





A record containing information about a logical file.







\par
\begin{description}
\item [\colorbox{tagtype}{\color{white} \textbf{\textsf{FIELD}}}] \textbf{\underline{size}} ||| INTEGER8 --- Number of bytes in the file (before compression)
\item [\colorbox{tagtype}{\color{white} \textbf{\textsf{FIELD}}}] \textbf{\underline{rowcount}} ||| INTEGER8 --- Number of rows in the file.
\item [\colorbox{tagtype}{\color{white} \textbf{\textsf{FIELD}}}] \textbf{\underline{superfile}} ||| BOOLEAN --- Is this a superfile?
\item [\colorbox{tagtype}{\color{white} \textbf{\textsf{FIELD}}}] \textbf{\underline{cluster}} ||| STRING --- No Doc
\item [\colorbox{tagtype}{\color{white} \textbf{\textsf{FIELD}}}] \textbf{\underline{modified}} ||| STRING19 --- No Doc
\item [\colorbox{tagtype}{\color{white} \textbf{\textsf{FIELD}}}] \textbf{\underline{owner}} ||| STRING --- No Doc
\item [\colorbox{tagtype}{\color{white} \textbf{\textsf{FIELD}}}] \textbf{\underline{name}} ||| STRING --- No Doc
\end{description}







\par
\begin{description}
\item [\colorbox{tagtype}{\color{white} \textbf{\textsf{CLUSTER}}}] The cluster that this file was created on.
\end{description}






\par
\begin{description}
\item [\colorbox{tagtype}{\color{white} \textbf{\textsf{MODIFIED}}}] Timestamp when the file was last modified;
\end{description}






\par
\begin{description}
\item [\colorbox{tagtype}{\color{white} \textbf{\textsf{OWNER}}}] The username of the owner who ran the job to create this file.
\end{description}







\par
\begin{description}
\item [\colorbox{tagtype}{\color{white} \textbf{\textsf{INHERITS}}}] Contains all the fields in FsLogicalFileNameRecord)
\end{description}



\rule{\linewidth}{0.5pt}
\subsection*{\textsf{\colorbox{headtoc}{\color{white} RECORD}
FsLogicalSuperSubRecord}}

\hypertarget{ecldoc:file.fslogicalsupersubrecord}{}
\hspace{0pt} \hyperlink{ecldoc:File}{File} \textbackslash 

{\renewcommand{\arraystretch}{1.5}
\begin{tabularx}{\textwidth}{|>{\raggedright\arraybackslash}l|X|}
\hline
\hspace{0pt}\mytexttt{\color{red} } & \textbf{FsLogicalSuperSubRecord} \\
\hline
\end{tabularx}
}

\par





A record containing information about a superfile and its contents.







\par
\begin{description}
\item [\colorbox{tagtype}{\color{white} \textbf{\textsf{FIELD}}}] \textbf{\underline{supername}} ||| STRING --- The name of the superfile
\item [\colorbox{tagtype}{\color{white} \textbf{\textsf{FIELD}}}] \textbf{\underline{subname}} ||| STRING --- The name of the sub-file
\end{description}





\rule{\linewidth}{0.5pt}
\subsection*{\textsf{\colorbox{headtoc}{\color{white} RECORD}
FsFileRelationshipRecord}}

\hypertarget{ecldoc:file.fsfilerelationshiprecord}{}
\hspace{0pt} \hyperlink{ecldoc:File}{File} \textbackslash 

{\renewcommand{\arraystretch}{1.5}
\begin{tabularx}{\textwidth}{|>{\raggedright\arraybackslash}l|X|}
\hline
\hspace{0pt}\mytexttt{\color{red} } & \textbf{FsFileRelationshipRecord} \\
\hline
\end{tabularx}
}

\par





A record containing information about the relationship between two files.







\par
\begin{description}
\item [\colorbox{tagtype}{\color{white} \textbf{\textsf{FIELD}}}] \textbf{\underline{secondaryfile}} ||| STRING --- The logical filename of the secondary file.
\item [\colorbox{tagtype}{\color{white} \textbf{\textsf{FIELD}}}] \textbf{\underline{secondaryflds}} ||| STRING --- The name of the foreign key field relating to the primary file.
\item [\colorbox{tagtype}{\color{white} \textbf{\textsf{FIELD}}}] \textbf{\underline{payload}} ||| BOOLEAN --- Indicates whether the primary or secondary are payload INDEXes.
\item [\colorbox{tagtype}{\color{white} \textbf{\textsf{FIELD}}}] \textbf{\underline{primaryflds}} ||| STRING --- The name of the primary key field for the primary file. The value ''\_\_fileposition\_\_'' indicates the secondary is an INDEX that must use FETCH to access non-keyed fields.
\item [\colorbox{tagtype}{\color{white} \textbf{\textsf{FIELD}}}] \textbf{\underline{kind}} ||| STRING --- The type of relationship between the primary and secondary files. Containing either 'link' or 'view'.
\item [\colorbox{tagtype}{\color{white} \textbf{\textsf{FIELD}}}] \textbf{\underline{primaryfile}} ||| STRING --- The logical filename of the primary file
\item [\colorbox{tagtype}{\color{white} \textbf{\textsf{FIELD}}}] \textbf{\underline{cardinality}} ||| STRING --- The cardinality of the relationship. The format is <primary>:<secondary>. Valid values are ''1'' or ''M''.</secondary></primary>
\item [\colorbox{tagtype}{\color{white} \textbf{\textsf{FIELD}}}] \textbf{\underline{description}} ||| STRING --- The description of the relationship.
\end{description}





\rule{\linewidth}{0.5pt}
\subsection*{\textsf{\colorbox{headtoc}{\color{white} ATTRIBUTE}
RECFMV\_RECSIZE}}

\hypertarget{ecldoc:file.recfmv_recsize}{}
\hspace{0pt} \hyperlink{ecldoc:File}{File} \textbackslash 

{\renewcommand{\arraystretch}{1.5}
\begin{tabularx}{\textwidth}{|>{\raggedright\arraybackslash}l|X|}
\hline
\hspace{0pt}\mytexttt{\color{red} } & \textbf{RECFMV\_RECSIZE} \\
\hline
\end{tabularx}
}

\par





Constant that indicates IBM RECFM V format file. Can be passed to SprayFixed for the record size.








\par
\begin{description}
\item [\colorbox{tagtype}{\color{white} \textbf{\textsf{RETURN}}}] \textbf{INTEGER4} --- 
\end{description}




\rule{\linewidth}{0.5pt}
\subsection*{\textsf{\colorbox{headtoc}{\color{white} ATTRIBUTE}
RECFMVB\_RECSIZE}}

\hypertarget{ecldoc:file.recfmvb_recsize}{}
\hspace{0pt} \hyperlink{ecldoc:File}{File} \textbackslash 

{\renewcommand{\arraystretch}{1.5}
\begin{tabularx}{\textwidth}{|>{\raggedright\arraybackslash}l|X|}
\hline
\hspace{0pt}\mytexttt{\color{red} } & \textbf{RECFMVB\_RECSIZE} \\
\hline
\end{tabularx}
}

\par





Constant that indicates IBM RECFM VB format file. Can be passed to SprayFixed for the record size.








\par
\begin{description}
\item [\colorbox{tagtype}{\color{white} \textbf{\textsf{RETURN}}}] \textbf{INTEGER4} --- 
\end{description}




\rule{\linewidth}{0.5pt}
\subsection*{\textsf{\colorbox{headtoc}{\color{white} ATTRIBUTE}
PREFIX\_VARIABLE\_RECSIZE}}

\hypertarget{ecldoc:file.prefix_variable_recsize}{}
\hspace{0pt} \hyperlink{ecldoc:File}{File} \textbackslash 

{\renewcommand{\arraystretch}{1.5}
\begin{tabularx}{\textwidth}{|>{\raggedright\arraybackslash}l|X|}
\hline
\hspace{0pt}\mytexttt{\color{red} INTEGER4} & \textbf{PREFIX\_VARIABLE\_RECSIZE} \\
\hline
\end{tabularx}
}

\par





Constant that indicates a variable little endian 4 byte length prefixed file. Can be passed to SprayFixed for the record size.








\par
\begin{description}
\item [\colorbox{tagtype}{\color{white} \textbf{\textsf{RETURN}}}] \textbf{INTEGER4} --- 
\end{description}




\rule{\linewidth}{0.5pt}
\subsection*{\textsf{\colorbox{headtoc}{\color{white} ATTRIBUTE}
PREFIX\_VARIABLE\_BIGENDIAN\_RECSIZE}}

\hypertarget{ecldoc:file.prefix_variable_bigendian_recsize}{}
\hspace{0pt} \hyperlink{ecldoc:File}{File} \textbackslash 

{\renewcommand{\arraystretch}{1.5}
\begin{tabularx}{\textwidth}{|>{\raggedright\arraybackslash}l|X|}
\hline
\hspace{0pt}\mytexttt{\color{red} INTEGER4} & \textbf{PREFIX\_VARIABLE\_BIGENDIAN\_RECSIZE} \\
\hline
\end{tabularx}
}

\par





Constant that indicates a variable big endian 4 byte length prefixed file. Can be passed to SprayFixed for the record size.








\par
\begin{description}
\item [\colorbox{tagtype}{\color{white} \textbf{\textsf{RETURN}}}] \textbf{INTEGER4} --- 
\end{description}




\rule{\linewidth}{0.5pt}
\subsection*{\textsf{\colorbox{headtoc}{\color{white} FUNCTION}
FileExists}}

\hypertarget{ecldoc:file.fileexists}{}
\hspace{0pt} \hyperlink{ecldoc:File}{File} \textbackslash 

{\renewcommand{\arraystretch}{1.5}
\begin{tabularx}{\textwidth}{|>{\raggedright\arraybackslash}l|X|}
\hline
\hspace{0pt}\mytexttt{\color{red} boolean} & \textbf{FileExists} \\
\hline
\multicolumn{2}{|>{\raggedright\arraybackslash}X|}{\hspace{0pt}\mytexttt{\color{param} (varstring lfn, boolean physical=FALSE)}} \\
\hline
\end{tabularx}
}

\par





Returns whether the file exists.






\par
\begin{description}
\item [\colorbox{tagtype}{\color{white} \textbf{\textsf{PARAMETER}}}] \textbf{\underline{lfn}} ||| VARSTRING --- The logical name of the file.
\item [\colorbox{tagtype}{\color{white} \textbf{\textsf{PARAMETER}}}] \textbf{\underline{physical}} ||| BOOLEAN --- Whether to also check for the physical existence on disk. Defaults to FALSE.
\end{description}







\par
\begin{description}
\item [\colorbox{tagtype}{\color{white} \textbf{\textsf{RETURN}}}] \textbf{BOOLEAN} --- Whether the file exists.
\end{description}




\rule{\linewidth}{0.5pt}
\subsection*{\textsf{\colorbox{headtoc}{\color{white} FUNCTION}
DeleteLogicalFile}}

\hypertarget{ecldoc:file.deletelogicalfile}{}
\hspace{0pt} \hyperlink{ecldoc:File}{File} \textbackslash 

{\renewcommand{\arraystretch}{1.5}
\begin{tabularx}{\textwidth}{|>{\raggedright\arraybackslash}l|X|}
\hline
\hspace{0pt}\mytexttt{\color{red} } & \textbf{DeleteLogicalFile} \\
\hline
\multicolumn{2}{|>{\raggedright\arraybackslash}X|}{\hspace{0pt}\mytexttt{\color{param} (varstring lfn, boolean allowMissing=FALSE)}} \\
\hline
\end{tabularx}
}

\par





Removes the logical file from the system, and deletes from the disk.






\par
\begin{description}
\item [\colorbox{tagtype}{\color{white} \textbf{\textsf{PARAMETER}}}] \textbf{\underline{lfn}} ||| VARSTRING --- The logical name of the file.
\item [\colorbox{tagtype}{\color{white} \textbf{\textsf{PARAMETER}}}] \textbf{\underline{allowMissing}} ||| BOOLEAN --- Whether to suppress an error if the filename does not exist. Defaults to FALSE.
\end{description}







\par
\begin{description}
\item [\colorbox{tagtype}{\color{white} \textbf{\textsf{RETURN}}}] \textbf{} --- 
\end{description}




\rule{\linewidth}{0.5pt}
\subsection*{\textsf{\colorbox{headtoc}{\color{white} FUNCTION}
SetReadOnly}}

\hypertarget{ecldoc:file.setreadonly}{}
\hspace{0pt} \hyperlink{ecldoc:File}{File} \textbackslash 

{\renewcommand{\arraystretch}{1.5}
\begin{tabularx}{\textwidth}{|>{\raggedright\arraybackslash}l|X|}
\hline
\hspace{0pt}\mytexttt{\color{red} } & \textbf{SetReadOnly} \\
\hline
\multicolumn{2}{|>{\raggedright\arraybackslash}X|}{\hspace{0pt}\mytexttt{\color{param} (varstring lfn, boolean ro=TRUE)}} \\
\hline
\end{tabularx}
}

\par





Changes whether access to a file is read only or not.






\par
\begin{description}
\item [\colorbox{tagtype}{\color{white} \textbf{\textsf{PARAMETER}}}] \textbf{\underline{lfn}} ||| VARSTRING --- The logical name of the file.
\item [\colorbox{tagtype}{\color{white} \textbf{\textsf{PARAMETER}}}] \textbf{\underline{ro}} ||| BOOLEAN --- Whether updates to the file are disallowed. Defaults to TRUE.
\end{description}







\par
\begin{description}
\item [\colorbox{tagtype}{\color{white} \textbf{\textsf{RETURN}}}] \textbf{} --- 
\end{description}




\rule{\linewidth}{0.5pt}
\subsection*{\textsf{\colorbox{headtoc}{\color{white} FUNCTION}
RenameLogicalFile}}

\hypertarget{ecldoc:file.renamelogicalfile}{}
\hspace{0pt} \hyperlink{ecldoc:File}{File} \textbackslash 

{\renewcommand{\arraystretch}{1.5}
\begin{tabularx}{\textwidth}{|>{\raggedright\arraybackslash}l|X|}
\hline
\hspace{0pt}\mytexttt{\color{red} } & \textbf{RenameLogicalFile} \\
\hline
\multicolumn{2}{|>{\raggedright\arraybackslash}X|}{\hspace{0pt}\mytexttt{\color{param} (varstring oldname, varstring newname)}} \\
\hline
\end{tabularx}
}

\par





Changes the name of a logical file.






\par
\begin{description}
\item [\colorbox{tagtype}{\color{white} \textbf{\textsf{PARAMETER}}}] \textbf{\underline{newname}} ||| VARSTRING --- The new logical name of the file.
\item [\colorbox{tagtype}{\color{white} \textbf{\textsf{PARAMETER}}}] \textbf{\underline{oldname}} ||| VARSTRING --- The current name of the file to be renamed.
\end{description}







\par
\begin{description}
\item [\colorbox{tagtype}{\color{white} \textbf{\textsf{RETURN}}}] \textbf{} --- 
\end{description}




\rule{\linewidth}{0.5pt}
\subsection*{\textsf{\colorbox{headtoc}{\color{white} FUNCTION}
ForeignLogicalFileName}}

\hypertarget{ecldoc:file.foreignlogicalfilename}{}
\hspace{0pt} \hyperlink{ecldoc:File}{File} \textbackslash 

{\renewcommand{\arraystretch}{1.5}
\begin{tabularx}{\textwidth}{|>{\raggedright\arraybackslash}l|X|}
\hline
\hspace{0pt}\mytexttt{\color{red} varstring} & \textbf{ForeignLogicalFileName} \\
\hline
\multicolumn{2}{|>{\raggedright\arraybackslash}X|}{\hspace{0pt}\mytexttt{\color{param} (varstring name, varstring foreigndali='', boolean abspath=FALSE)}} \\
\hline
\end{tabularx}
}

\par





Returns a logical filename that can be used to refer to a logical file in a local or remote dali.






\par
\begin{description}
\item [\colorbox{tagtype}{\color{white} \textbf{\textsf{PARAMETER}}}] \textbf{\underline{abspath}} ||| BOOLEAN --- Should a tilde (\~{}) be prepended to the resulting logical file name. Defaults to FALSE.
\item [\colorbox{tagtype}{\color{white} \textbf{\textsf{PARAMETER}}}] \textbf{\underline{foreigndali}} ||| VARSTRING --- The IP address of the foreign dali used to resolve the file. If blank then the file is resolved locally. Defaults to blank.
\item [\colorbox{tagtype}{\color{white} \textbf{\textsf{PARAMETER}}}] \textbf{\underline{name}} ||| VARSTRING --- The logical name of the file.
\end{description}







\par
\begin{description}
\item [\colorbox{tagtype}{\color{white} \textbf{\textsf{RETURN}}}] \textbf{VARSTRING} --- 
\end{description}




\rule{\linewidth}{0.5pt}
\subsection*{\textsf{\colorbox{headtoc}{\color{white} FUNCTION}
ExternalLogicalFileName}}

\hypertarget{ecldoc:file.externallogicalfilename}{}
\hspace{0pt} \hyperlink{ecldoc:File}{File} \textbackslash 

{\renewcommand{\arraystretch}{1.5}
\begin{tabularx}{\textwidth}{|>{\raggedright\arraybackslash}l|X|}
\hline
\hspace{0pt}\mytexttt{\color{red} varstring} & \textbf{ExternalLogicalFileName} \\
\hline
\multicolumn{2}{|>{\raggedright\arraybackslash}X|}{\hspace{0pt}\mytexttt{\color{param} (varstring location, varstring path, boolean abspath=TRUE)}} \\
\hline
\end{tabularx}
}

\par





Returns an encoded logical filename that can be used to refer to a external file. Examples include directly reading from a landing zone. Upper case characters and other details are escaped.






\par
\begin{description}
\item [\colorbox{tagtype}{\color{white} \textbf{\textsf{PARAMETER}}}] \textbf{\underline{abspath}} ||| BOOLEAN --- Should a tilde (\~{}) be prepended to the resulting logical file name. Defaults to TRUE.
\item [\colorbox{tagtype}{\color{white} \textbf{\textsf{PARAMETER}}}] \textbf{\underline{path}} ||| VARSTRING --- The path/name of the file on the remote machine.
\item [\colorbox{tagtype}{\color{white} \textbf{\textsf{PARAMETER}}}] \textbf{\underline{location}} ||| VARSTRING --- The IP address of the remote machine. '.' can be used for the local machine.
\end{description}







\par
\begin{description}
\item [\colorbox{tagtype}{\color{white} \textbf{\textsf{RETURN}}}] \textbf{VARSTRING} --- The encoded logical filename.
\end{description}




\rule{\linewidth}{0.5pt}
\subsection*{\textsf{\colorbox{headtoc}{\color{white} FUNCTION}
GetFileDescription}}

\hypertarget{ecldoc:file.getfiledescription}{}
\hspace{0pt} \hyperlink{ecldoc:File}{File} \textbackslash 

{\renewcommand{\arraystretch}{1.5}
\begin{tabularx}{\textwidth}{|>{\raggedright\arraybackslash}l|X|}
\hline
\hspace{0pt}\mytexttt{\color{red} varstring} & \textbf{GetFileDescription} \\
\hline
\multicolumn{2}{|>{\raggedright\arraybackslash}X|}{\hspace{0pt}\mytexttt{\color{param} (varstring lfn)}} \\
\hline
\end{tabularx}
}

\par





Returns a string containing the description information associated with the specified filename. This description is set either through ECL watch or by using the FileServices.SetFileDescription function.






\par
\begin{description}
\item [\colorbox{tagtype}{\color{white} \textbf{\textsf{PARAMETER}}}] \textbf{\underline{lfn}} ||| VARSTRING --- The logical name of the file.
\end{description}







\par
\begin{description}
\item [\colorbox{tagtype}{\color{white} \textbf{\textsf{RETURN}}}] \textbf{VARSTRING} --- 
\end{description}




\rule{\linewidth}{0.5pt}
\subsection*{\textsf{\colorbox{headtoc}{\color{white} FUNCTION}
SetFileDescription}}

\hypertarget{ecldoc:file.setfiledescription}{}
\hspace{0pt} \hyperlink{ecldoc:File}{File} \textbackslash 

{\renewcommand{\arraystretch}{1.5}
\begin{tabularx}{\textwidth}{|>{\raggedright\arraybackslash}l|X|}
\hline
\hspace{0pt}\mytexttt{\color{red} } & \textbf{SetFileDescription} \\
\hline
\multicolumn{2}{|>{\raggedright\arraybackslash}X|}{\hspace{0pt}\mytexttt{\color{param} (varstring lfn, varstring val)}} \\
\hline
\end{tabularx}
}

\par





Sets the description associated with the specified filename.






\par
\begin{description}
\item [\colorbox{tagtype}{\color{white} \textbf{\textsf{PARAMETER}}}] \textbf{\underline{lfn}} ||| VARSTRING --- The logical name of the file.
\item [\colorbox{tagtype}{\color{white} \textbf{\textsf{PARAMETER}}}] \textbf{\underline{val}} ||| VARSTRING --- The description to be associated with the file.
\end{description}







\par
\begin{description}
\item [\colorbox{tagtype}{\color{white} \textbf{\textsf{RETURN}}}] \textbf{} --- 
\end{description}




\rule{\linewidth}{0.5pt}
\subsection*{\textsf{\colorbox{headtoc}{\color{white} FUNCTION}
RemoteDirectory}}

\hypertarget{ecldoc:file.remotedirectory}{}
\hspace{0pt} \hyperlink{ecldoc:File}{File} \textbackslash 

{\renewcommand{\arraystretch}{1.5}
\begin{tabularx}{\textwidth}{|>{\raggedright\arraybackslash}l|X|}
\hline
\hspace{0pt}\mytexttt{\color{red} dataset(FsFilenameRecord)} & \textbf{RemoteDirectory} \\
\hline
\multicolumn{2}{|>{\raggedright\arraybackslash}X|}{\hspace{0pt}\mytexttt{\color{param} (varstring machineIP, varstring dir, varstring mask='*', boolean recurse=FALSE)}} \\
\hline
\end{tabularx}
}

\par





Returns a dataset containing a list of files from the specified machineIP and directory.






\par
\begin{description}
\item [\colorbox{tagtype}{\color{white} \textbf{\textsf{PARAMETER}}}] \textbf{\underline{directory}} |||  --- The path to the directory to read. This must be in the appropriate format for the operating system running on the remote machine.
\item [\colorbox{tagtype}{\color{white} \textbf{\textsf{PARAMETER}}}] \textbf{\underline{recurse}} ||| BOOLEAN --- Whether to include files from subdirectories under the directory. Defaults to FALSE.
\item [\colorbox{tagtype}{\color{white} \textbf{\textsf{PARAMETER}}}] \textbf{\underline{machineIP}} ||| VARSTRING --- The IP address of the remote machine.
\item [\colorbox{tagtype}{\color{white} \textbf{\textsf{PARAMETER}}}] \textbf{\underline{mask}} ||| VARSTRING --- The filemask specifying which files to include in the result. Defaults to '*' (all files).
\item [\colorbox{tagtype}{\color{white} \textbf{\textsf{PARAMETER}}}] \textbf{\underline{dir}} ||| VARSTRING --- No Doc
\end{description}







\par
\begin{description}
\item [\colorbox{tagtype}{\color{white} \textbf{\textsf{RETURN}}}] \textbf{TABLE ( FsFilenameRecord )} --- 
\end{description}




\rule{\linewidth}{0.5pt}
\subsection*{\textsf{\colorbox{headtoc}{\color{white} FUNCTION}
LogicalFileList}}

\hypertarget{ecldoc:file.logicalfilelist}{}
\hspace{0pt} \hyperlink{ecldoc:File}{File} \textbackslash 

{\renewcommand{\arraystretch}{1.5}
\begin{tabularx}{\textwidth}{|>{\raggedright\arraybackslash}l|X|}
\hline
\hspace{0pt}\mytexttt{\color{red} dataset(FsLogicalFileInfoRecord)} & \textbf{LogicalFileList} \\
\hline
\multicolumn{2}{|>{\raggedright\arraybackslash}X|}{\hspace{0pt}\mytexttt{\color{param} (varstring namepattern='*', boolean includenormal=TRUE, boolean includesuper=FALSE, boolean unknownszero=FALSE, varstring foreigndali='')}} \\
\hline
\end{tabularx}
}

\par





Returns a dataset of information about the logical files known to the system.






\par
\begin{description}
\item [\colorbox{tagtype}{\color{white} \textbf{\textsf{PARAMETER}}}] \textbf{\underline{namepattern}} ||| VARSTRING --- The mask of the files to list. Defaults to '*' (all files).
\item [\colorbox{tagtype}{\color{white} \textbf{\textsf{PARAMETER}}}] \textbf{\underline{includesuper}} ||| BOOLEAN --- Whether to include SuperFiles. Defaults to FALSE.
\item [\colorbox{tagtype}{\color{white} \textbf{\textsf{PARAMETER}}}] \textbf{\underline{includenormal}} ||| BOOLEAN --- Whether to include 'normal' files. Defaults to TRUE.
\item [\colorbox{tagtype}{\color{white} \textbf{\textsf{PARAMETER}}}] \textbf{\underline{foreigndali}} ||| VARSTRING --- The IP address of the foreign dali used to resolve the file. If blank then the file is resolved locally. Defaults to blank.
\item [\colorbox{tagtype}{\color{white} \textbf{\textsf{PARAMETER}}}] \textbf{\underline{unknownszero}} ||| BOOLEAN --- Whether to set file sizes that are unknown to zero(0) instead of minus-one (-1). Defaults to FALSE.
\end{description}







\par
\begin{description}
\item [\colorbox{tagtype}{\color{white} \textbf{\textsf{RETURN}}}] \textbf{TABLE ( FsLogicalFileInfoRecord )} --- 
\end{description}




\rule{\linewidth}{0.5pt}
\subsection*{\textsf{\colorbox{headtoc}{\color{white} FUNCTION}
CompareFiles}}

\hypertarget{ecldoc:file.comparefiles}{}
\hspace{0pt} \hyperlink{ecldoc:File}{File} \textbackslash 

{\renewcommand{\arraystretch}{1.5}
\begin{tabularx}{\textwidth}{|>{\raggedright\arraybackslash}l|X|}
\hline
\hspace{0pt}\mytexttt{\color{red} INTEGER4} & \textbf{CompareFiles} \\
\hline
\multicolumn{2}{|>{\raggedright\arraybackslash}X|}{\hspace{0pt}\mytexttt{\color{param} (varstring lfn1, varstring lfn2, boolean logical\_only=TRUE, boolean use\_crcs=FALSE)}} \\
\hline
\end{tabularx}
}

\par





Compares two files, and returns a result indicating how well they match.






\par
\begin{description}
\item [\colorbox{tagtype}{\color{white} \textbf{\textsf{PARAMETER}}}] \textbf{\underline{logical\_only}} ||| BOOLEAN --- Whether to only compare logical information in the system datastore (Dali), and ignore physical information on disk. [Default TRUE]
\item [\colorbox{tagtype}{\color{white} \textbf{\textsf{PARAMETER}}}] \textbf{\underline{use\_crcs}} ||| BOOLEAN --- Whether to compare physical CRCs of all the parts on disk. This may be slow on large files. Defaults to FALSE.
\item [\colorbox{tagtype}{\color{white} \textbf{\textsf{PARAMETER}}}] \textbf{\underline{file2}} |||  --- The logical name of the second file.
\item [\colorbox{tagtype}{\color{white} \textbf{\textsf{PARAMETER}}}] \textbf{\underline{file1}} |||  --- The logical name of the first file.
\item [\colorbox{tagtype}{\color{white} \textbf{\textsf{PARAMETER}}}] \textbf{\underline{lfn2}} ||| VARSTRING --- No Doc
\item [\colorbox{tagtype}{\color{white} \textbf{\textsf{PARAMETER}}}] \textbf{\underline{lfn1}} ||| VARSTRING --- No Doc
\end{description}







\par
\begin{description}
\item [\colorbox{tagtype}{\color{white} \textbf{\textsf{RETURN}}}] \textbf{INTEGER4} --- 0 if file1 and file2 match exactly 1 if file1 and file2 contents match, but file1 is newer than file2 -1 if file1 and file2 contents match, but file2 is newer than file1 2 if file1 and file2 contents do not match and file1 is newer than file2 -2 if file1 and file2 contents do not match and file2 is newer than file1
\end{description}




\rule{\linewidth}{0.5pt}
\subsection*{\textsf{\colorbox{headtoc}{\color{white} FUNCTION}
VerifyFile}}

\hypertarget{ecldoc:file.verifyfile}{}
\hspace{0pt} \hyperlink{ecldoc:File}{File} \textbackslash 

{\renewcommand{\arraystretch}{1.5}
\begin{tabularx}{\textwidth}{|>{\raggedright\arraybackslash}l|X|}
\hline
\hspace{0pt}\mytexttt{\color{red} varstring} & \textbf{VerifyFile} \\
\hline
\multicolumn{2}{|>{\raggedright\arraybackslash}X|}{\hspace{0pt}\mytexttt{\color{param} (varstring lfn, boolean usecrcs)}} \\
\hline
\end{tabularx}
}

\par





Checks the system datastore (Dali) information for the file against the physical parts on disk.






\par
\begin{description}
\item [\colorbox{tagtype}{\color{white} \textbf{\textsf{PARAMETER}}}] \textbf{\underline{use\_crcs}} |||  --- Whether to compare physical CRCs of all the parts on disk. This may be slow on large files.
\item [\colorbox{tagtype}{\color{white} \textbf{\textsf{PARAMETER}}}] \textbf{\underline{lfn}} ||| VARSTRING --- The name of the file to check.
\item [\colorbox{tagtype}{\color{white} \textbf{\textsf{PARAMETER}}}] \textbf{\underline{usecrcs}} ||| BOOLEAN --- No Doc
\end{description}







\par
\begin{description}
\item [\colorbox{tagtype}{\color{white} \textbf{\textsf{RETURN}}}] \textbf{VARSTRING} --- 'OK' - The file parts match the datastore information 'Could not find file: <filename>' - The logical filename was not found 'Could not find part file: <partname>' - The partname was not found 'Modified time differs for: <partname>' - The partname has a different timestamp 'File size differs for: <partname>' - The partname has a file size 'File CRC differs for: <partname>' - The partname has a different CRC</partname></partname></partname></partname></filename>
\end{description}




\rule{\linewidth}{0.5pt}
\subsection*{\textsf{\colorbox{headtoc}{\color{white} FUNCTION}
AddFileRelationship}}

\hypertarget{ecldoc:file.addfilerelationship}{}
\hspace{0pt} \hyperlink{ecldoc:File}{File} \textbackslash 

{\renewcommand{\arraystretch}{1.5}
\begin{tabularx}{\textwidth}{|>{\raggedright\arraybackslash}l|X|}
\hline
\hspace{0pt}\mytexttt{\color{red} } & \textbf{AddFileRelationship} \\
\hline
\multicolumn{2}{|>{\raggedright\arraybackslash}X|}{\hspace{0pt}\mytexttt{\color{param} (varstring primary, varstring secondary, varstring primaryflds, varstring secondaryflds, varstring kind='link', varstring cardinality, boolean payload, varstring description='')}} \\
\hline
\end{tabularx}
}

\par





Defines the relationship between two files. These may be DATASETs or INDEXes. Each record in the primary file should be uniquely defined by the primaryfields (ideally), preferably efficiently. This information is used by the roxie browser to link files together.






\par
\begin{description}
\item [\colorbox{tagtype}{\color{white} \textbf{\textsf{PARAMETER}}}] \textbf{\underline{primaryfields}} |||  --- The name of the primary key field for the primary file. The value ''\_\_fileposition\_\_'' indicates the secondary is an INDEX that must use FETCH to access non-keyed fields.
\item [\colorbox{tagtype}{\color{white} \textbf{\textsf{PARAMETER}}}] \textbf{\underline{secondaryfields}} |||  --- The name of the foreign key field relating to the primary file.
\item [\colorbox{tagtype}{\color{white} \textbf{\textsf{PARAMETER}}}] \textbf{\underline{secondary}} ||| VARSTRING --- The logical filename of the secondary file.
\item [\colorbox{tagtype}{\color{white} \textbf{\textsf{PARAMETER}}}] \textbf{\underline{cardinality}} ||| VARSTRING --- The cardinality of the relationship. The format is <primary>:<secondary>. Valid values are ''1'' or ''M''.</secondary></primary>
\item [\colorbox{tagtype}{\color{white} \textbf{\textsf{PARAMETER}}}] \textbf{\underline{primary}} ||| VARSTRING --- The logical filename of the primary file.
\item [\colorbox{tagtype}{\color{white} \textbf{\textsf{PARAMETER}}}] \textbf{\underline{relationship}} |||  --- The type of relationship between the primary and secondary files. Containing either 'link' or 'view'. Default is ''link''.
\item [\colorbox{tagtype}{\color{white} \textbf{\textsf{PARAMETER}}}] \textbf{\underline{payload}} ||| BOOLEAN --- Indicates whether the primary or secondary are payload INDEXes.
\item [\colorbox{tagtype}{\color{white} \textbf{\textsf{PARAMETER}}}] \textbf{\underline{description}} ||| VARSTRING --- The description of the relationship.
\item [\colorbox{tagtype}{\color{white} \textbf{\textsf{PARAMETER}}}] \textbf{\underline{secondaryflds}} ||| VARSTRING --- No Doc
\item [\colorbox{tagtype}{\color{white} \textbf{\textsf{PARAMETER}}}] \textbf{\underline{kind}} ||| VARSTRING --- No Doc
\item [\colorbox{tagtype}{\color{white} \textbf{\textsf{PARAMETER}}}] \textbf{\underline{primaryflds}} ||| VARSTRING --- No Doc
\end{description}







\par
\begin{description}
\item [\colorbox{tagtype}{\color{white} \textbf{\textsf{RETURN}}}] \textbf{} --- 
\end{description}




\rule{\linewidth}{0.5pt}
\subsection*{\textsf{\colorbox{headtoc}{\color{white} FUNCTION}
FileRelationshipList}}

\hypertarget{ecldoc:file.filerelationshiplist}{}
\hspace{0pt} \hyperlink{ecldoc:File}{File} \textbackslash 

{\renewcommand{\arraystretch}{1.5}
\begin{tabularx}{\textwidth}{|>{\raggedright\arraybackslash}l|X|}
\hline
\hspace{0pt}\mytexttt{\color{red} dataset(FsFileRelationshipRecord)} & \textbf{FileRelationshipList} \\
\hline
\multicolumn{2}{|>{\raggedright\arraybackslash}X|}{\hspace{0pt}\mytexttt{\color{param} (varstring primary, varstring secondary, varstring primflds='', varstring secondaryflds='', varstring kind='link')}} \\
\hline
\end{tabularx}
}

\par





Returns a dataset of relationships. The return records are structured in the FsFileRelationshipRecord format.






\par
\begin{description}
\item [\colorbox{tagtype}{\color{white} \textbf{\textsf{PARAMETER}}}] \textbf{\underline{primary}} ||| VARSTRING --- The logical filename of the primary file.
\item [\colorbox{tagtype}{\color{white} \textbf{\textsf{PARAMETER}}}] \textbf{\underline{secondaryfields}} |||  --- The name of the foreign key field relating to the primary file.
\item [\colorbox{tagtype}{\color{white} \textbf{\textsf{PARAMETER}}}] \textbf{\underline{relationship}} |||  --- The type of relationship between the primary and secondary files. Containing either 'link' or 'view'. Default is ''link''.
\item [\colorbox{tagtype}{\color{white} \textbf{\textsf{PARAMETER}}}] \textbf{\underline{secondary}} ||| VARSTRING --- The logical filename of the secondary file.
\item [\colorbox{tagtype}{\color{white} \textbf{\textsf{PARAMETER}}}] \textbf{\underline{primaryfields}} |||  --- The name of the primary key field for the primary file.
\item [\colorbox{tagtype}{\color{white} \textbf{\textsf{PARAMETER}}}] \textbf{\underline{secondaryflds}} ||| VARSTRING --- No Doc
\item [\colorbox{tagtype}{\color{white} \textbf{\textsf{PARAMETER}}}] \textbf{\underline{kind}} ||| VARSTRING --- No Doc
\item [\colorbox{tagtype}{\color{white} \textbf{\textsf{PARAMETER}}}] \textbf{\underline{primflds}} ||| VARSTRING --- No Doc
\end{description}







\par
\begin{description}
\item [\colorbox{tagtype}{\color{white} \textbf{\textsf{RETURN}}}] \textbf{TABLE ( FsFileRelationshipRecord )} --- 
\end{description}




\rule{\linewidth}{0.5pt}
\subsection*{\textsf{\colorbox{headtoc}{\color{white} FUNCTION}
RemoveFileRelationship}}

\hypertarget{ecldoc:file.removefilerelationship}{}
\hspace{0pt} \hyperlink{ecldoc:File}{File} \textbackslash 

{\renewcommand{\arraystretch}{1.5}
\begin{tabularx}{\textwidth}{|>{\raggedright\arraybackslash}l|X|}
\hline
\hspace{0pt}\mytexttt{\color{red} } & \textbf{RemoveFileRelationship} \\
\hline
\multicolumn{2}{|>{\raggedright\arraybackslash}X|}{\hspace{0pt}\mytexttt{\color{param} (varstring primary, varstring secondary, varstring primaryflds='', varstring secondaryflds='', varstring kind='link')}} \\
\hline
\end{tabularx}
}

\par





Removes a relationship between two files.






\par
\begin{description}
\item [\colorbox{tagtype}{\color{white} \textbf{\textsf{PARAMETER}}}] \textbf{\underline{primary}} ||| VARSTRING --- The logical filename of the primary file.
\item [\colorbox{tagtype}{\color{white} \textbf{\textsf{PARAMETER}}}] \textbf{\underline{secondaryfields}} |||  --- The name of the foreign key field relating to the primary file.
\item [\colorbox{tagtype}{\color{white} \textbf{\textsf{PARAMETER}}}] \textbf{\underline{relationship}} |||  --- The type of relationship between the primary and secondary files. Containing either 'link' or 'view'. Default is ''link''.
\item [\colorbox{tagtype}{\color{white} \textbf{\textsf{PARAMETER}}}] \textbf{\underline{secondary}} ||| VARSTRING --- The logical filename of the secondary file.
\item [\colorbox{tagtype}{\color{white} \textbf{\textsf{PARAMETER}}}] \textbf{\underline{primaryfields}} |||  --- The name of the primary key field for the primary file.
\item [\colorbox{tagtype}{\color{white} \textbf{\textsf{PARAMETER}}}] \textbf{\underline{secondaryflds}} ||| VARSTRING --- No Doc
\item [\colorbox{tagtype}{\color{white} \textbf{\textsf{PARAMETER}}}] \textbf{\underline{kind}} ||| VARSTRING --- No Doc
\item [\colorbox{tagtype}{\color{white} \textbf{\textsf{PARAMETER}}}] \textbf{\underline{primaryflds}} ||| VARSTRING --- No Doc
\end{description}







\par
\begin{description}
\item [\colorbox{tagtype}{\color{white} \textbf{\textsf{RETURN}}}] \textbf{} --- 
\end{description}




\rule{\linewidth}{0.5pt}
\subsection*{\textsf{\colorbox{headtoc}{\color{white} FUNCTION}
GetColumnMapping}}

\hypertarget{ecldoc:file.getcolumnmapping}{}
\hspace{0pt} \hyperlink{ecldoc:File}{File} \textbackslash 

{\renewcommand{\arraystretch}{1.5}
\begin{tabularx}{\textwidth}{|>{\raggedright\arraybackslash}l|X|}
\hline
\hspace{0pt}\mytexttt{\color{red} varstring} & \textbf{GetColumnMapping} \\
\hline
\multicolumn{2}{|>{\raggedright\arraybackslash}X|}{\hspace{0pt}\mytexttt{\color{param} (varstring lfn)}} \\
\hline
\end{tabularx}
}

\par





Returns the field mappings for the file, in the same format specified for the SetColumnMapping function.






\par
\begin{description}
\item [\colorbox{tagtype}{\color{white} \textbf{\textsf{PARAMETER}}}] \textbf{\underline{lfn}} ||| VARSTRING --- The logical filename of the primary file.
\end{description}







\par
\begin{description}
\item [\colorbox{tagtype}{\color{white} \textbf{\textsf{RETURN}}}] \textbf{VARSTRING} --- 
\end{description}




\rule{\linewidth}{0.5pt}
\subsection*{\textsf{\colorbox{headtoc}{\color{white} FUNCTION}
SetColumnMapping}}

\hypertarget{ecldoc:file.setcolumnmapping}{}
\hspace{0pt} \hyperlink{ecldoc:File}{File} \textbackslash 

{\renewcommand{\arraystretch}{1.5}
\begin{tabularx}{\textwidth}{|>{\raggedright\arraybackslash}l|X|}
\hline
\hspace{0pt}\mytexttt{\color{red} } & \textbf{SetColumnMapping} \\
\hline
\multicolumn{2}{|>{\raggedright\arraybackslash}X|}{\hspace{0pt}\mytexttt{\color{param} (varstring lfn, varstring mapping)}} \\
\hline
\end{tabularx}
}

\par





Defines how the data in the fields of the file mist be transformed between the actual data storage format and the input format used to query that data. This is used by the user interface of the roxie browser.






\par
\begin{description}
\item [\colorbox{tagtype}{\color{white} \textbf{\textsf{PARAMETER}}}] \textbf{\underline{lfn}} ||| VARSTRING --- The logical filename of the primary file.
\item [\colorbox{tagtype}{\color{white} \textbf{\textsf{PARAMETER}}}] \textbf{\underline{mapping}} ||| VARSTRING --- A string containing a comma separated list of field mappings.
\end{description}







\par
\begin{description}
\item [\colorbox{tagtype}{\color{white} \textbf{\textsf{RETURN}}}] \textbf{} --- 
\end{description}




\rule{\linewidth}{0.5pt}
\subsection*{\textsf{\colorbox{headtoc}{\color{white} FUNCTION}
EncodeRfsQuery}}

\hypertarget{ecldoc:file.encoderfsquery}{}
\hspace{0pt} \hyperlink{ecldoc:File}{File} \textbackslash 

{\renewcommand{\arraystretch}{1.5}
\begin{tabularx}{\textwidth}{|>{\raggedright\arraybackslash}l|X|}
\hline
\hspace{0pt}\mytexttt{\color{red} varstring} & \textbf{EncodeRfsQuery} \\
\hline
\multicolumn{2}{|>{\raggedright\arraybackslash}X|}{\hspace{0pt}\mytexttt{\color{param} (varstring server, varstring query)}} \\
\hline
\end{tabularx}
}

\par





Returns a string that can be used in a DATASET declaration to read data from an RFS (Remote File Server) instance (e.g. rfsmysql) on another node.






\par
\begin{description}
\item [\colorbox{tagtype}{\color{white} \textbf{\textsf{PARAMETER}}}] \textbf{\underline{query}} ||| VARSTRING --- The text of the query to send to the server
\item [\colorbox{tagtype}{\color{white} \textbf{\textsf{PARAMETER}}}] \textbf{\underline{server}} ||| VARSTRING --- A string containing the ip:port address for the remote file server.
\end{description}







\par
\begin{description}
\item [\colorbox{tagtype}{\color{white} \textbf{\textsf{RETURN}}}] \textbf{VARSTRING} --- 
\end{description}




\rule{\linewidth}{0.5pt}
\subsection*{\textsf{\colorbox{headtoc}{\color{white} FUNCTION}
RfsAction}}

\hypertarget{ecldoc:file.rfsaction}{}
\hspace{0pt} \hyperlink{ecldoc:File}{File} \textbackslash 

{\renewcommand{\arraystretch}{1.5}
\begin{tabularx}{\textwidth}{|>{\raggedright\arraybackslash}l|X|}
\hline
\hspace{0pt}\mytexttt{\color{red} } & \textbf{RfsAction} \\
\hline
\multicolumn{2}{|>{\raggedright\arraybackslash}X|}{\hspace{0pt}\mytexttt{\color{param} (varstring server, varstring query)}} \\
\hline
\end{tabularx}
}

\par





Sends the query to the rfs server.






\par
\begin{description}
\item [\colorbox{tagtype}{\color{white} \textbf{\textsf{PARAMETER}}}] \textbf{\underline{query}} ||| VARSTRING --- The text of the query to send to the server
\item [\colorbox{tagtype}{\color{white} \textbf{\textsf{PARAMETER}}}] \textbf{\underline{server}} ||| VARSTRING --- A string containing the ip:port address for the remote file server.
\end{description}







\par
\begin{description}
\item [\colorbox{tagtype}{\color{white} \textbf{\textsf{RETURN}}}] \textbf{} --- 
\end{description}




\rule{\linewidth}{0.5pt}
\subsection*{\textsf{\colorbox{headtoc}{\color{white} FUNCTION}
MoveExternalFile}}

\hypertarget{ecldoc:file.moveexternalfile}{}
\hspace{0pt} \hyperlink{ecldoc:File}{File} \textbackslash 

{\renewcommand{\arraystretch}{1.5}
\begin{tabularx}{\textwidth}{|>{\raggedright\arraybackslash}l|X|}
\hline
\hspace{0pt}\mytexttt{\color{red} } & \textbf{MoveExternalFile} \\
\hline
\multicolumn{2}{|>{\raggedright\arraybackslash}X|}{\hspace{0pt}\mytexttt{\color{param} (varstring location, varstring frompath, varstring topath)}} \\
\hline
\end{tabularx}
}

\par





Moves the single physical file between two locations on the same remote machine. The dafileserv utility program must be running on the location machine.






\par
\begin{description}
\item [\colorbox{tagtype}{\color{white} \textbf{\textsf{PARAMETER}}}] \textbf{\underline{frompath}} ||| VARSTRING --- The path/name of the file to move.
\item [\colorbox{tagtype}{\color{white} \textbf{\textsf{PARAMETER}}}] \textbf{\underline{topath}} ||| VARSTRING --- The path/name of the target file.
\item [\colorbox{tagtype}{\color{white} \textbf{\textsf{PARAMETER}}}] \textbf{\underline{location}} ||| VARSTRING --- The IP address of the remote machine.
\end{description}







\par
\begin{description}
\item [\colorbox{tagtype}{\color{white} \textbf{\textsf{RETURN}}}] \textbf{} --- 
\end{description}




\rule{\linewidth}{0.5pt}
\subsection*{\textsf{\colorbox{headtoc}{\color{white} FUNCTION}
DeleteExternalFile}}

\hypertarget{ecldoc:file.deleteexternalfile}{}
\hspace{0pt} \hyperlink{ecldoc:File}{File} \textbackslash 

{\renewcommand{\arraystretch}{1.5}
\begin{tabularx}{\textwidth}{|>{\raggedright\arraybackslash}l|X|}
\hline
\hspace{0pt}\mytexttt{\color{red} } & \textbf{DeleteExternalFile} \\
\hline
\multicolumn{2}{|>{\raggedright\arraybackslash}X|}{\hspace{0pt}\mytexttt{\color{param} (varstring location, varstring path)}} \\
\hline
\end{tabularx}
}

\par





Removes a single physical file from a remote machine. The dafileserv utility program must be running on the location machine.






\par
\begin{description}
\item [\colorbox{tagtype}{\color{white} \textbf{\textsf{PARAMETER}}}] \textbf{\underline{path}} ||| VARSTRING --- The path/name of the file to remove.
\item [\colorbox{tagtype}{\color{white} \textbf{\textsf{PARAMETER}}}] \textbf{\underline{location}} ||| VARSTRING --- The IP address of the remote machine.
\end{description}







\par
\begin{description}
\item [\colorbox{tagtype}{\color{white} \textbf{\textsf{RETURN}}}] \textbf{} --- 
\end{description}




\rule{\linewidth}{0.5pt}
\subsection*{\textsf{\colorbox{headtoc}{\color{white} FUNCTION}
CreateExternalDirectory}}

\hypertarget{ecldoc:file.createexternaldirectory}{}
\hspace{0pt} \hyperlink{ecldoc:File}{File} \textbackslash 

{\renewcommand{\arraystretch}{1.5}
\begin{tabularx}{\textwidth}{|>{\raggedright\arraybackslash}l|X|}
\hline
\hspace{0pt}\mytexttt{\color{red} } & \textbf{CreateExternalDirectory} \\
\hline
\multicolumn{2}{|>{\raggedright\arraybackslash}X|}{\hspace{0pt}\mytexttt{\color{param} (varstring location, varstring path)}} \\
\hline
\end{tabularx}
}

\par





Creates the path on the location (if it does not already exist). The dafileserv utility program must be running on the location machine.






\par
\begin{description}
\item [\colorbox{tagtype}{\color{white} \textbf{\textsf{PARAMETER}}}] \textbf{\underline{path}} ||| VARSTRING --- The path/name of the file to remove.
\item [\colorbox{tagtype}{\color{white} \textbf{\textsf{PARAMETER}}}] \textbf{\underline{location}} ||| VARSTRING --- The IP address of the remote machine.
\end{description}







\par
\begin{description}
\item [\colorbox{tagtype}{\color{white} \textbf{\textsf{RETURN}}}] \textbf{} --- 
\end{description}




\rule{\linewidth}{0.5pt}
\subsection*{\textsf{\colorbox{headtoc}{\color{white} FUNCTION}
GetLogicalFileAttribute}}

\hypertarget{ecldoc:file.getlogicalfileattribute}{}
\hspace{0pt} \hyperlink{ecldoc:File}{File} \textbackslash 

{\renewcommand{\arraystretch}{1.5}
\begin{tabularx}{\textwidth}{|>{\raggedright\arraybackslash}l|X|}
\hline
\hspace{0pt}\mytexttt{\color{red} varstring} & \textbf{GetLogicalFileAttribute} \\
\hline
\multicolumn{2}{|>{\raggedright\arraybackslash}X|}{\hspace{0pt}\mytexttt{\color{param} (varstring lfn, varstring attrname)}} \\
\hline
\end{tabularx}
}

\par





Returns the value of the given attribute for the specified logicalfilename.






\par
\begin{description}
\item [\colorbox{tagtype}{\color{white} \textbf{\textsf{PARAMETER}}}] \textbf{\underline{attrname}} ||| VARSTRING --- The name of the file attribute to return.
\item [\colorbox{tagtype}{\color{white} \textbf{\textsf{PARAMETER}}}] \textbf{\underline{lfn}} ||| VARSTRING --- The name of the logical file.
\end{description}







\par
\begin{description}
\item [\colorbox{tagtype}{\color{white} \textbf{\textsf{RETURN}}}] \textbf{VARSTRING} --- 
\end{description}




\rule{\linewidth}{0.5pt}
\subsection*{\textsf{\colorbox{headtoc}{\color{white} FUNCTION}
ProtectLogicalFile}}

\hypertarget{ecldoc:file.protectlogicalfile}{}
\hspace{0pt} \hyperlink{ecldoc:File}{File} \textbackslash 

{\renewcommand{\arraystretch}{1.5}
\begin{tabularx}{\textwidth}{|>{\raggedright\arraybackslash}l|X|}
\hline
\hspace{0pt}\mytexttt{\color{red} } & \textbf{ProtectLogicalFile} \\
\hline
\multicolumn{2}{|>{\raggedright\arraybackslash}X|}{\hspace{0pt}\mytexttt{\color{param} (varstring lfn, boolean value=TRUE)}} \\
\hline
\end{tabularx}
}

\par





Toggles protection on and off for the specified logicalfilename.






\par
\begin{description}
\item [\colorbox{tagtype}{\color{white} \textbf{\textsf{PARAMETER}}}] \textbf{\underline{value}} ||| BOOLEAN --- TRUE to enable protection, FALSE to disable.
\item [\colorbox{tagtype}{\color{white} \textbf{\textsf{PARAMETER}}}] \textbf{\underline{lfn}} ||| VARSTRING --- The name of the logical file.
\end{description}







\par
\begin{description}
\item [\colorbox{tagtype}{\color{white} \textbf{\textsf{RETURN}}}] \textbf{} --- 
\end{description}




\rule{\linewidth}{0.5pt}
\subsection*{\textsf{\colorbox{headtoc}{\color{white} FUNCTION}
DfuPlusExec}}

\hypertarget{ecldoc:file.dfuplusexec}{}
\hspace{0pt} \hyperlink{ecldoc:File}{File} \textbackslash 

{\renewcommand{\arraystretch}{1.5}
\begin{tabularx}{\textwidth}{|>{\raggedright\arraybackslash}l|X|}
\hline
\hspace{0pt}\mytexttt{\color{red} } & \textbf{DfuPlusExec} \\
\hline
\multicolumn{2}{|>{\raggedright\arraybackslash}X|}{\hspace{0pt}\mytexttt{\color{param} (varstring cmdline)}} \\
\hline
\end{tabularx}
}

\par





The DfuPlusExec action executes the specified command line just as the DfuPLus.exe program would do. This allows you to have all the functionality of the DfuPLus.exe program available within your ECL code. param cmdline The DFUPlus.exe command line to execute. The valid arguments are documented in the Client Tools manual, in the section describing the DfuPlus.exe program.






\par
\begin{description}
\item [\colorbox{tagtype}{\color{white} \textbf{\textsf{PARAMETER}}}] \textbf{\underline{cmdline}} ||| VARSTRING --- No Doc
\end{description}







\par
\begin{description}
\item [\colorbox{tagtype}{\color{white} \textbf{\textsf{RETURN}}}] \textbf{} --- 
\end{description}




\rule{\linewidth}{0.5pt}
\subsection*{\textsf{\colorbox{headtoc}{\color{white} FUNCTION}
fSprayFixed}}

\hypertarget{ecldoc:file.fsprayfixed}{}
\hspace{0pt} \hyperlink{ecldoc:File}{File} \textbackslash 

{\renewcommand{\arraystretch}{1.5}
\begin{tabularx}{\textwidth}{|>{\raggedright\arraybackslash}l|X|}
\hline
\hspace{0pt}\mytexttt{\color{red} varstring} & \textbf{fSprayFixed} \\
\hline
\multicolumn{2}{|>{\raggedright\arraybackslash}X|}{\hspace{0pt}\mytexttt{\color{param} (varstring sourceIP, varstring sourcePath, integer4 recordSize, varstring destinationGroup, varstring destinationLogicalName, integer4 timeOut=-1, varstring espServerIpPort=GETENV('ws\_fs\_server'), integer4 maxConnections=-1, boolean allowOverwrite=FALSE, boolean replicate=FALSE, boolean compress=FALSE, boolean failIfNoSourceFile=FALSE, integer4 expireDays=-1)}} \\
\hline
\end{tabularx}
}

\par





Sprays a file of fixed length records from a single machine and distributes it across the nodes of the destination group.






\par
\begin{description}
\item [\colorbox{tagtype}{\color{white} \textbf{\textsf{PARAMETER}}}] \textbf{\underline{expireDays}} ||| INTEGER4 --- Number of days to auto-remove file. Default is -1, not expire.
\item [\colorbox{tagtype}{\color{white} \textbf{\textsf{PARAMETER}}}] \textbf{\underline{failIfNoSourceFile}} ||| BOOLEAN --- If TRUE it causes a missing source file to trigger a failure. Defaults to FALSE.
\item [\colorbox{tagtype}{\color{white} \textbf{\textsf{PARAMETER}}}] \textbf{\underline{recordsize}} ||| INTEGER4 --- The size (in bytes) of the records in the file.
\item [\colorbox{tagtype}{\color{white} \textbf{\textsf{PARAMETER}}}] \textbf{\underline{compress}} ||| BOOLEAN --- Whether to compress the new file. Defaults to FALSE.
\item [\colorbox{tagtype}{\color{white} \textbf{\textsf{PARAMETER}}}] \textbf{\underline{destinationGroup}} ||| VARSTRING --- The name of the group to distribute the file across.
\item [\colorbox{tagtype}{\color{white} \textbf{\textsf{PARAMETER}}}] \textbf{\underline{timeOut}} ||| INTEGER4 --- The time in ms to wait for the operation to complete. A value of 0 causes the call to return immediately. Defaults to no timeout (-1).
\item [\colorbox{tagtype}{\color{white} \textbf{\textsf{PARAMETER}}}] \textbf{\underline{replicate}} ||| BOOLEAN --- Whether to replicate the new file. Defaults to FALSE.
\item [\colorbox{tagtype}{\color{white} \textbf{\textsf{PARAMETER}}}] \textbf{\underline{sourceIP}} ||| VARSTRING --- The IP address of the file.
\item [\colorbox{tagtype}{\color{white} \textbf{\textsf{PARAMETER}}}] \textbf{\underline{sourcePath}} ||| VARSTRING --- The path and name of the file.
\item [\colorbox{tagtype}{\color{white} \textbf{\textsf{PARAMETER}}}] \textbf{\underline{destinationLogicalName}} ||| VARSTRING --- The logical name of the file to create.
\item [\colorbox{tagtype}{\color{white} \textbf{\textsf{PARAMETER}}}] \textbf{\underline{maxConnections}} ||| INTEGER4 --- The maximum number of target nodes to write to concurrently. Defaults to 1.
\item [\colorbox{tagtype}{\color{white} \textbf{\textsf{PARAMETER}}}] \textbf{\underline{espServerIpPort}} ||| VARSTRING --- The url of the ESP file copying service. Defaults to the value of ws\_fs\_server in the environment.
\item [\colorbox{tagtype}{\color{white} \textbf{\textsf{PARAMETER}}}] \textbf{\underline{allowOverwrite}} ||| BOOLEAN --- Is it valid to overwrite an existing file of the same name? Defaults to FALSE
\end{description}







\par
\begin{description}
\item [\colorbox{tagtype}{\color{white} \textbf{\textsf{RETURN}}}] \textbf{VARSTRING} --- The DFU workunit id for the job.
\end{description}




\rule{\linewidth}{0.5pt}
\subsection*{\textsf{\colorbox{headtoc}{\color{white} FUNCTION}
SprayFixed}}

\hypertarget{ecldoc:file.sprayfixed}{}
\hspace{0pt} \hyperlink{ecldoc:File}{File} \textbackslash 

{\renewcommand{\arraystretch}{1.5}
\begin{tabularx}{\textwidth}{|>{\raggedright\arraybackslash}l|X|}
\hline
\hspace{0pt}\mytexttt{\color{red} } & \textbf{SprayFixed} \\
\hline
\multicolumn{2}{|>{\raggedright\arraybackslash}X|}{\hspace{0pt}\mytexttt{\color{param} (varstring sourceIP, varstring sourcePath, integer4 recordSize, varstring destinationGroup, varstring destinationLogicalName, integer4 timeOut=-1, varstring espServerIpPort=GETENV('ws\_fs\_server'), integer4 maxConnections=-1, boolean allowOverwrite=FALSE, boolean replicate=FALSE, boolean compress=FALSE, boolean failIfNoSourceFile=FALSE, integer4 expireDays=-1)}} \\
\hline
\end{tabularx}
}

\par





Same as fSprayFixed, but does not return the DFU Workunit ID.






\par
\begin{description}
\item [\colorbox{tagtype}{\color{white} \textbf{\textsf{PARAMETER}}}] \textbf{\underline{espserveripport}} ||| VARSTRING --- No Doc
\item [\colorbox{tagtype}{\color{white} \textbf{\textsf{PARAMETER}}}] \textbf{\underline{recordsize}} ||| INTEGER4 --- No Doc
\item [\colorbox{tagtype}{\color{white} \textbf{\textsf{PARAMETER}}}] \textbf{\underline{compress}} ||| BOOLEAN --- No Doc
\item [\colorbox{tagtype}{\color{white} \textbf{\textsf{PARAMETER}}}] \textbf{\underline{replicate}} ||| BOOLEAN --- No Doc
\item [\colorbox{tagtype}{\color{white} \textbf{\textsf{PARAMETER}}}] \textbf{\underline{failifnosourcefile}} ||| BOOLEAN --- No Doc
\item [\colorbox{tagtype}{\color{white} \textbf{\textsf{PARAMETER}}}] \textbf{\underline{allowoverwrite}} ||| BOOLEAN --- No Doc
\item [\colorbox{tagtype}{\color{white} \textbf{\textsf{PARAMETER}}}] \textbf{\underline{sourceip}} ||| VARSTRING --- No Doc
\item [\colorbox{tagtype}{\color{white} \textbf{\textsf{PARAMETER}}}] \textbf{\underline{timeout}} ||| INTEGER4 --- No Doc
\item [\colorbox{tagtype}{\color{white} \textbf{\textsf{PARAMETER}}}] \textbf{\underline{maxconnections}} ||| INTEGER4 --- No Doc
\item [\colorbox{tagtype}{\color{white} \textbf{\textsf{PARAMETER}}}] \textbf{\underline{expiredays}} ||| INTEGER4 --- No Doc
\item [\colorbox{tagtype}{\color{white} \textbf{\textsf{PARAMETER}}}] \textbf{\underline{destinationgroup}} ||| VARSTRING --- No Doc
\item [\colorbox{tagtype}{\color{white} \textbf{\textsf{PARAMETER}}}] \textbf{\underline{destinationlogicalname}} ||| VARSTRING --- No Doc
\item [\colorbox{tagtype}{\color{white} \textbf{\textsf{PARAMETER}}}] \textbf{\underline{sourcepath}} ||| VARSTRING --- No Doc
\end{description}







\par
\begin{description}
\item [\colorbox{tagtype}{\color{white} \textbf{\textsf{RETURN}}}] \textbf{} --- 
\end{description}






\par
\begin{description}
\item [\colorbox{tagtype}{\color{white} \textbf{\textsf{SEE}}}] fSprayFixed
\end{description}




\rule{\linewidth}{0.5pt}
\subsection*{\textsf{\colorbox{headtoc}{\color{white} FUNCTION}
fSprayVariable}}

\hypertarget{ecldoc:file.fsprayvariable}{}
\hspace{0pt} \hyperlink{ecldoc:File}{File} \textbackslash 

{\renewcommand{\arraystretch}{1.5}
\begin{tabularx}{\textwidth}{|>{\raggedright\arraybackslash}l|X|}
\hline
\hspace{0pt}\mytexttt{\color{red} varstring} & \textbf{fSprayVariable} \\
\hline
\multicolumn{2}{|>{\raggedright\arraybackslash}X|}{\hspace{0pt}\mytexttt{\color{param} (varstring sourceIP, varstring sourcePath, integer4 sourceMaxRecordSize=8192, varstring sourceCsvSeparate='\textbackslash \textbackslash ,', varstring sourceCsvTerminate='\textbackslash \textbackslash n,\textbackslash \textbackslash r\textbackslash \textbackslash n', varstring sourceCsvQuote='\textbackslash ''', varstring destinationGroup, varstring destinationLogicalName, integer4 timeOut=-1, varstring espServerIpPort=GETENV('ws\_fs\_server'), integer4 maxConnections=-1, boolean allowOverwrite=FALSE, boolean replicate=FALSE, boolean compress=FALSE, varstring sourceCsvEscape='', boolean failIfNoSourceFile=FALSE, boolean recordStructurePresent=FALSE, boolean quotedTerminator=TRUE, varstring encoding='ascii', integer4 expireDays=-1)}} \\
\hline
\end{tabularx}
}

\par





No Documentation Found






\par
\begin{description}
\item [\colorbox{tagtype}{\color{white} \textbf{\textsf{PARAMETER}}}] \textbf{\underline{espserveripport}} ||| VARSTRING --- No Doc
\item [\colorbox{tagtype}{\color{white} \textbf{\textsf{PARAMETER}}}] \textbf{\underline{sourcecsvquote}} ||| VARSTRING --- No Doc
\item [\colorbox{tagtype}{\color{white} \textbf{\textsf{PARAMETER}}}] \textbf{\underline{timeout}} ||| INTEGER4 --- No Doc
\item [\colorbox{tagtype}{\color{white} \textbf{\textsf{PARAMETER}}}] \textbf{\underline{destinationgroup}} ||| VARSTRING --- No Doc
\item [\colorbox{tagtype}{\color{white} \textbf{\textsf{PARAMETER}}}] \textbf{\underline{replicate}} ||| BOOLEAN --- No Doc
\item [\colorbox{tagtype}{\color{white} \textbf{\textsf{PARAMETER}}}] \textbf{\underline{failifnosourcefile}} ||| BOOLEAN --- No Doc
\item [\colorbox{tagtype}{\color{white} \textbf{\textsf{PARAMETER}}}] \textbf{\underline{recordstructurepresent}} ||| BOOLEAN --- No Doc
\item [\colorbox{tagtype}{\color{white} \textbf{\textsf{PARAMETER}}}] \textbf{\underline{allowoverwrite}} ||| BOOLEAN --- No Doc
\item [\colorbox{tagtype}{\color{white} \textbf{\textsf{PARAMETER}}}] \textbf{\underline{sourceip}} ||| VARSTRING --- No Doc
\item [\colorbox{tagtype}{\color{white} \textbf{\textsf{PARAMETER}}}] \textbf{\underline{sourcemaxrecordsize}} ||| INTEGER4 --- No Doc
\item [\colorbox{tagtype}{\color{white} \textbf{\textsf{PARAMETER}}}] \textbf{\underline{sourcecsvterminate}} ||| VARSTRING --- No Doc
\item [\colorbox{tagtype}{\color{white} \textbf{\textsf{PARAMETER}}}] \textbf{\underline{quotedterminator}} ||| BOOLEAN --- No Doc
\item [\colorbox{tagtype}{\color{white} \textbf{\textsf{PARAMETER}}}] \textbf{\underline{maxconnections}} ||| INTEGER4 --- No Doc
\item [\colorbox{tagtype}{\color{white} \textbf{\textsf{PARAMETER}}}] \textbf{\underline{sourcecsvescape}} ||| VARSTRING --- No Doc
\item [\colorbox{tagtype}{\color{white} \textbf{\textsf{PARAMETER}}}] \textbf{\underline{expiredays}} ||| INTEGER4 --- No Doc
\item [\colorbox{tagtype}{\color{white} \textbf{\textsf{PARAMETER}}}] \textbf{\underline{compress}} ||| BOOLEAN --- No Doc
\item [\colorbox{tagtype}{\color{white} \textbf{\textsf{PARAMETER}}}] \textbf{\underline{sourcecsvseparate}} ||| VARSTRING --- No Doc
\item [\colorbox{tagtype}{\color{white} \textbf{\textsf{PARAMETER}}}] \textbf{\underline{destinationlogicalname}} ||| VARSTRING --- No Doc
\item [\colorbox{tagtype}{\color{white} \textbf{\textsf{PARAMETER}}}] \textbf{\underline{sourcepath}} ||| VARSTRING --- No Doc
\item [\colorbox{tagtype}{\color{white} \textbf{\textsf{PARAMETER}}}] \textbf{\underline{encoding}} ||| VARSTRING --- No Doc
\end{description}







\par
\begin{description}
\item [\colorbox{tagtype}{\color{white} \textbf{\textsf{RETURN}}}] \textbf{VARSTRING} --- 
\end{description}




\rule{\linewidth}{0.5pt}
\subsection*{\textsf{\colorbox{headtoc}{\color{white} FUNCTION}
SprayVariable}}

\hypertarget{ecldoc:file.sprayvariable}{}
\hspace{0pt} \hyperlink{ecldoc:File}{File} \textbackslash 

{\renewcommand{\arraystretch}{1.5}
\begin{tabularx}{\textwidth}{|>{\raggedright\arraybackslash}l|X|}
\hline
\hspace{0pt}\mytexttt{\color{red} } & \textbf{SprayVariable} \\
\hline
\multicolumn{2}{|>{\raggedright\arraybackslash}X|}{\hspace{0pt}\mytexttt{\color{param} (varstring sourceIP, varstring sourcePath, integer4 sourceMaxRecordSize=8192, varstring sourceCsvSeparate='\textbackslash \textbackslash ,', varstring sourceCsvTerminate='\textbackslash \textbackslash n,\textbackslash \textbackslash r\textbackslash \textbackslash n', varstring sourceCsvQuote='\textbackslash ''', varstring destinationGroup, varstring destinationLogicalName, integer4 timeOut=-1, varstring espServerIpPort=GETENV('ws\_fs\_server'), integer4 maxConnections=-1, boolean allowOverwrite=FALSE, boolean replicate=FALSE, boolean compress=FALSE, varstring sourceCsvEscape='', boolean failIfNoSourceFile=FALSE, boolean recordStructurePresent=FALSE, boolean quotedTerminator=TRUE, varstring encoding='ascii', integer4 expireDays=-1)}} \\
\hline
\end{tabularx}
}

\par





No Documentation Found






\par
\begin{description}
\item [\colorbox{tagtype}{\color{white} \textbf{\textsf{PARAMETER}}}] \textbf{\underline{espserveripport}} ||| VARSTRING --- No Doc
\item [\colorbox{tagtype}{\color{white} \textbf{\textsf{PARAMETER}}}] \textbf{\underline{sourcecsvquote}} ||| VARSTRING --- No Doc
\item [\colorbox{tagtype}{\color{white} \textbf{\textsf{PARAMETER}}}] \textbf{\underline{timeout}} ||| INTEGER4 --- No Doc
\item [\colorbox{tagtype}{\color{white} \textbf{\textsf{PARAMETER}}}] \textbf{\underline{destinationgroup}} ||| VARSTRING --- No Doc
\item [\colorbox{tagtype}{\color{white} \textbf{\textsf{PARAMETER}}}] \textbf{\underline{replicate}} ||| BOOLEAN --- No Doc
\item [\colorbox{tagtype}{\color{white} \textbf{\textsf{PARAMETER}}}] \textbf{\underline{failifnosourcefile}} ||| BOOLEAN --- No Doc
\item [\colorbox{tagtype}{\color{white} \textbf{\textsf{PARAMETER}}}] \textbf{\underline{recordstructurepresent}} ||| BOOLEAN --- No Doc
\item [\colorbox{tagtype}{\color{white} \textbf{\textsf{PARAMETER}}}] \textbf{\underline{allowoverwrite}} ||| BOOLEAN --- No Doc
\item [\colorbox{tagtype}{\color{white} \textbf{\textsf{PARAMETER}}}] \textbf{\underline{sourceip}} ||| VARSTRING --- No Doc
\item [\colorbox{tagtype}{\color{white} \textbf{\textsf{PARAMETER}}}] \textbf{\underline{sourcemaxrecordsize}} ||| INTEGER4 --- No Doc
\item [\colorbox{tagtype}{\color{white} \textbf{\textsf{PARAMETER}}}] \textbf{\underline{sourcecsvterminate}} ||| VARSTRING --- No Doc
\item [\colorbox{tagtype}{\color{white} \textbf{\textsf{PARAMETER}}}] \textbf{\underline{quotedterminator}} ||| BOOLEAN --- No Doc
\item [\colorbox{tagtype}{\color{white} \textbf{\textsf{PARAMETER}}}] \textbf{\underline{maxconnections}} ||| INTEGER4 --- No Doc
\item [\colorbox{tagtype}{\color{white} \textbf{\textsf{PARAMETER}}}] \textbf{\underline{sourcecsvescape}} ||| VARSTRING --- No Doc
\item [\colorbox{tagtype}{\color{white} \textbf{\textsf{PARAMETER}}}] \textbf{\underline{expiredays}} ||| INTEGER4 --- No Doc
\item [\colorbox{tagtype}{\color{white} \textbf{\textsf{PARAMETER}}}] \textbf{\underline{compress}} ||| BOOLEAN --- No Doc
\item [\colorbox{tagtype}{\color{white} \textbf{\textsf{PARAMETER}}}] \textbf{\underline{sourcecsvseparate}} ||| VARSTRING --- No Doc
\item [\colorbox{tagtype}{\color{white} \textbf{\textsf{PARAMETER}}}] \textbf{\underline{destinationlogicalname}} ||| VARSTRING --- No Doc
\item [\colorbox{tagtype}{\color{white} \textbf{\textsf{PARAMETER}}}] \textbf{\underline{sourcepath}} ||| VARSTRING --- No Doc
\item [\colorbox{tagtype}{\color{white} \textbf{\textsf{PARAMETER}}}] \textbf{\underline{encoding}} ||| VARSTRING --- No Doc
\end{description}







\par
\begin{description}
\item [\colorbox{tagtype}{\color{white} \textbf{\textsf{RETURN}}}] \textbf{} --- 
\end{description}




\rule{\linewidth}{0.5pt}
\subsection*{\textsf{\colorbox{headtoc}{\color{white} FUNCTION}
fSprayDelimited}}

\hypertarget{ecldoc:file.fspraydelimited}{}
\hspace{0pt} \hyperlink{ecldoc:File}{File} \textbackslash 

{\renewcommand{\arraystretch}{1.5}
\begin{tabularx}{\textwidth}{|>{\raggedright\arraybackslash}l|X|}
\hline
\hspace{0pt}\mytexttt{\color{red} varstring} & \textbf{fSprayDelimited} \\
\hline
\multicolumn{2}{|>{\raggedright\arraybackslash}X|}{\hspace{0pt}\mytexttt{\color{param} (varstring sourceIP, varstring sourcePath, integer4 sourceMaxRecordSize=8192, varstring sourceCsvSeparate='\textbackslash \textbackslash ,', varstring sourceCsvTerminate='\textbackslash \textbackslash n,\textbackslash \textbackslash r\textbackslash \textbackslash n', varstring sourceCsvQuote='\textbackslash ''', varstring destinationGroup, varstring destinationLogicalName, integer4 timeOut=-1, varstring espServerIpPort=GETENV('ws\_fs\_server'), integer4 maxConnections=-1, boolean allowOverwrite=FALSE, boolean replicate=FALSE, boolean compress=FALSE, varstring sourceCsvEscape='', boolean failIfNoSourceFile=FALSE, boolean recordStructurePresent=FALSE, boolean quotedTerminator=TRUE, varstring encoding='ascii', integer4 expireDays=-1)}} \\
\hline
\end{tabularx}
}

\par





Sprays a file of fixed delimited records from a single machine and distributes it across the nodes of the destination group.






\par
\begin{description}
\item [\colorbox{tagtype}{\color{white} \textbf{\textsf{PARAMETER}}}] \textbf{\underline{expireDays}} ||| INTEGER4 --- Number of days to auto-remove file. Default is -1, not expire.
\item [\colorbox{tagtype}{\color{white} \textbf{\textsf{PARAMETER}}}] \textbf{\underline{failIfNoSourceFile}} ||| BOOLEAN --- If TRUE it causes a missing source file to trigger a failure. Defaults to FALSE.
\item [\colorbox{tagtype}{\color{white} \textbf{\textsf{PARAMETER}}}] \textbf{\underline{sourceCsvEscape}} ||| VARSTRING --- A character that is used to escape quote characters. Defaults to none.
\item [\colorbox{tagtype}{\color{white} \textbf{\textsf{PARAMETER}}}] \textbf{\underline{sourceCsvSeparate}} ||| VARSTRING --- The character sequence which separates fields in the file.
\item [\colorbox{tagtype}{\color{white} \textbf{\textsf{PARAMETER}}}] \textbf{\underline{quotedTerminator}} ||| BOOLEAN --- Can the terminator character be included in a quoted field. Defaults to TRUE. If FALSE it allows quicker partitioning of the file (avoiding a complete file scan).
\item [\colorbox{tagtype}{\color{white} \textbf{\textsf{PARAMETER}}}] \textbf{\underline{compress}} ||| BOOLEAN --- Whether to compress the new file. Defaults to FALSE.
\item [\colorbox{tagtype}{\color{white} \textbf{\textsf{PARAMETER}}}] \textbf{\underline{sourceCsvTerminate}} ||| VARSTRING --- The character sequence which separates records in the file.
\item [\colorbox{tagtype}{\color{white} \textbf{\textsf{PARAMETER}}}] \textbf{\underline{timeOut}} ||| INTEGER4 --- The time in ms to wait for the operation to complete. A value of 0 causes the call to return immediately. Defaults to no timeout (-1).
\item [\colorbox{tagtype}{\color{white} \textbf{\textsf{PARAMETER}}}] \textbf{\underline{allowOverwrite}} ||| BOOLEAN --- Is it valid to overwrite an existing file of the same name? Defaults to FALSE
\item [\colorbox{tagtype}{\color{white} \textbf{\textsf{PARAMETER}}}] \textbf{\underline{recordStructurePresent}} ||| BOOLEAN --- If TRUE derives the record structure from the header of the file.
\item [\colorbox{tagtype}{\color{white} \textbf{\textsf{PARAMETER}}}] \textbf{\underline{sourceCsvQuote}} ||| VARSTRING --- A string which can be used to delimit fields in the file.
\item [\colorbox{tagtype}{\color{white} \textbf{\textsf{PARAMETER}}}] \textbf{\underline{replicate}} ||| BOOLEAN --- Whether to replicate the new file. Defaults to FALSE.
\item [\colorbox{tagtype}{\color{white} \textbf{\textsf{PARAMETER}}}] \textbf{\underline{sourceIP}} ||| VARSTRING --- The IP address of the file.
\item [\colorbox{tagtype}{\color{white} \textbf{\textsf{PARAMETER}}}] \textbf{\underline{sourcePath}} ||| VARSTRING --- The path and name of the file.
\item [\colorbox{tagtype}{\color{white} \textbf{\textsf{PARAMETER}}}] \textbf{\underline{maxConnections}} ||| INTEGER4 --- The maximum number of target nodes to write to concurrently. Defaults to 1.
\item [\colorbox{tagtype}{\color{white} \textbf{\textsf{PARAMETER}}}] \textbf{\underline{destinationLogicalName}} ||| VARSTRING --- The logical name of the file to create.
\item [\colorbox{tagtype}{\color{white} \textbf{\textsf{PARAMETER}}}] \textbf{\underline{sourceMaxRecordSize}} ||| INTEGER4 --- The maximum size (in bytes) of the records in the file.
\item [\colorbox{tagtype}{\color{white} \textbf{\textsf{PARAMETER}}}] \textbf{\underline{espServerIpPort}} ||| VARSTRING --- The url of the ESP file copying service. Defaults to the value of ws\_fs\_server in the environment.
\item [\colorbox{tagtype}{\color{white} \textbf{\textsf{PARAMETER}}}] \textbf{\underline{destinationGroup}} ||| VARSTRING --- The name of the group to distribute the file across.
\item [\colorbox{tagtype}{\color{white} \textbf{\textsf{PARAMETER}}}] \textbf{\underline{encoding}} ||| VARSTRING --- No Doc
\end{description}







\par
\begin{description}
\item [\colorbox{tagtype}{\color{white} \textbf{\textsf{RETURN}}}] \textbf{VARSTRING} --- The DFU workunit id for the job.
\end{description}




\rule{\linewidth}{0.5pt}
\subsection*{\textsf{\colorbox{headtoc}{\color{white} FUNCTION}
SprayDelimited}}

\hypertarget{ecldoc:file.spraydelimited}{}
\hspace{0pt} \hyperlink{ecldoc:File}{File} \textbackslash 

{\renewcommand{\arraystretch}{1.5}
\begin{tabularx}{\textwidth}{|>{\raggedright\arraybackslash}l|X|}
\hline
\hspace{0pt}\mytexttt{\color{red} } & \textbf{SprayDelimited} \\
\hline
\multicolumn{2}{|>{\raggedright\arraybackslash}X|}{\hspace{0pt}\mytexttt{\color{param} (varstring sourceIP, varstring sourcePath, integer4 sourceMaxRecordSize=8192, varstring sourceCsvSeparate='\textbackslash \textbackslash ,', varstring sourceCsvTerminate='\textbackslash \textbackslash n,\textbackslash \textbackslash r\textbackslash \textbackslash n', varstring sourceCsvQuote='\textbackslash ''', varstring destinationGroup, varstring destinationLogicalName, integer4 timeOut=-1, varstring espServerIpPort=GETENV('ws\_fs\_server'), integer4 maxConnections=-1, boolean allowOverwrite=FALSE, boolean replicate=FALSE, boolean compress=FALSE, varstring sourceCsvEscape='', boolean failIfNoSourceFile=FALSE, boolean recordStructurePresent=FALSE, boolean quotedTerminator=TRUE, const varstring encoding='ascii', integer4 expireDays=-1)}} \\
\hline
\end{tabularx}
}

\par





Same as fSprayDelimited, but does not return the DFU Workunit ID.






\par
\begin{description}
\item [\colorbox{tagtype}{\color{white} \textbf{\textsf{PARAMETER}}}] \textbf{\underline{espserveripport}} ||| VARSTRING --- No Doc
\item [\colorbox{tagtype}{\color{white} \textbf{\textsf{PARAMETER}}}] \textbf{\underline{sourcecsvquote}} ||| VARSTRING --- No Doc
\item [\colorbox{tagtype}{\color{white} \textbf{\textsf{PARAMETER}}}] \textbf{\underline{timeout}} ||| INTEGER4 --- No Doc
\item [\colorbox{tagtype}{\color{white} \textbf{\textsf{PARAMETER}}}] \textbf{\underline{destinationgroup}} ||| VARSTRING --- No Doc
\item [\colorbox{tagtype}{\color{white} \textbf{\textsf{PARAMETER}}}] \textbf{\underline{replicate}} ||| BOOLEAN --- No Doc
\item [\colorbox{tagtype}{\color{white} \textbf{\textsf{PARAMETER}}}] \textbf{\underline{failifnosourcefile}} ||| BOOLEAN --- No Doc
\item [\colorbox{tagtype}{\color{white} \textbf{\textsf{PARAMETER}}}] \textbf{\underline{recordstructurepresent}} ||| BOOLEAN --- No Doc
\item [\colorbox{tagtype}{\color{white} \textbf{\textsf{PARAMETER}}}] \textbf{\underline{allowoverwrite}} ||| BOOLEAN --- No Doc
\item [\colorbox{tagtype}{\color{white} \textbf{\textsf{PARAMETER}}}] \textbf{\underline{sourceip}} ||| VARSTRING --- No Doc
\item [\colorbox{tagtype}{\color{white} \textbf{\textsf{PARAMETER}}}] \textbf{\underline{sourcemaxrecordsize}} ||| INTEGER4 --- No Doc
\item [\colorbox{tagtype}{\color{white} \textbf{\textsf{PARAMETER}}}] \textbf{\underline{sourcecsvterminate}} ||| VARSTRING --- No Doc
\item [\colorbox{tagtype}{\color{white} \textbf{\textsf{PARAMETER}}}] \textbf{\underline{quotedterminator}} ||| BOOLEAN --- No Doc
\item [\colorbox{tagtype}{\color{white} \textbf{\textsf{PARAMETER}}}] \textbf{\underline{maxconnections}} ||| INTEGER4 --- No Doc
\item [\colorbox{tagtype}{\color{white} \textbf{\textsf{PARAMETER}}}] \textbf{\underline{sourcecsvescape}} ||| VARSTRING --- No Doc
\item [\colorbox{tagtype}{\color{white} \textbf{\textsf{PARAMETER}}}] \textbf{\underline{expiredays}} ||| INTEGER4 --- No Doc
\item [\colorbox{tagtype}{\color{white} \textbf{\textsf{PARAMETER}}}] \textbf{\underline{compress}} ||| BOOLEAN --- No Doc
\item [\colorbox{tagtype}{\color{white} \textbf{\textsf{PARAMETER}}}] \textbf{\underline{sourcecsvseparate}} ||| VARSTRING --- No Doc
\item [\colorbox{tagtype}{\color{white} \textbf{\textsf{PARAMETER}}}] \textbf{\underline{destinationlogicalname}} ||| VARSTRING --- No Doc
\item [\colorbox{tagtype}{\color{white} \textbf{\textsf{PARAMETER}}}] \textbf{\underline{sourcepath}} ||| VARSTRING --- No Doc
\item [\colorbox{tagtype}{\color{white} \textbf{\textsf{PARAMETER}}}] \textbf{\underline{encoding}} ||| VARSTRING --- No Doc
\end{description}







\par
\begin{description}
\item [\colorbox{tagtype}{\color{white} \textbf{\textsf{RETURN}}}] \textbf{} --- 
\end{description}






\par
\begin{description}
\item [\colorbox{tagtype}{\color{white} \textbf{\textsf{SEE}}}] fSprayDelimited
\end{description}




\rule{\linewidth}{0.5pt}
\subsection*{\textsf{\colorbox{headtoc}{\color{white} FUNCTION}
fSprayXml}}

\hypertarget{ecldoc:file.fsprayxml}{}
\hspace{0pt} \hyperlink{ecldoc:File}{File} \textbackslash 

{\renewcommand{\arraystretch}{1.5}
\begin{tabularx}{\textwidth}{|>{\raggedright\arraybackslash}l|X|}
\hline
\hspace{0pt}\mytexttt{\color{red} varstring} & \textbf{fSprayXml} \\
\hline
\multicolumn{2}{|>{\raggedright\arraybackslash}X|}{\hspace{0pt}\mytexttt{\color{param} (varstring sourceIP, varstring sourcePath, integer4 sourceMaxRecordSize=8192, varstring sourceRowTag, varstring sourceEncoding='utf8', varstring destinationGroup, varstring destinationLogicalName, integer4 timeOut=-1, varstring espServerIpPort=GETENV('ws\_fs\_server'), integer4 maxConnections=-1, boolean allowOverwrite=FALSE, boolean replicate=FALSE, boolean compress=FALSE, boolean failIfNoSourceFile=FALSE, integer4 expireDays=-1)}} \\
\hline
\end{tabularx}
}

\par





Sprays an xml file from a single machine and distributes it across the nodes of the destination group.






\par
\begin{description}
\item [\colorbox{tagtype}{\color{white} \textbf{\textsf{PARAMETER}}}] \textbf{\underline{expireDays}} ||| INTEGER4 --- Number of days to auto-remove file. Default is -1, not expire.
\item [\colorbox{tagtype}{\color{white} \textbf{\textsf{PARAMETER}}}] \textbf{\underline{failIfNoSourceFile}} ||| BOOLEAN --- If TRUE it causes a missing source file to trigger a failure. Defaults to FALSE.
\item [\colorbox{tagtype}{\color{white} \textbf{\textsf{PARAMETER}}}] \textbf{\underline{sourceMaxRecordSize}} ||| INTEGER4 --- The maximum size (in bytes) of the records in the file.
\item [\colorbox{tagtype}{\color{white} \textbf{\textsf{PARAMETER}}}] \textbf{\underline{compress}} ||| BOOLEAN --- Whether to compress the new file. Defaults to FALSE.
\item [\colorbox{tagtype}{\color{white} \textbf{\textsf{PARAMETER}}}] \textbf{\underline{destinationGroup}} ||| VARSTRING --- The name of the group to distribute the file across.
\item [\colorbox{tagtype}{\color{white} \textbf{\textsf{PARAMETER}}}] \textbf{\underline{sourceRowTag}} ||| VARSTRING --- The xml tag that is used to delimit records in the source file. (This tag cannot recursivly nest.)
\item [\colorbox{tagtype}{\color{white} \textbf{\textsf{PARAMETER}}}] \textbf{\underline{sourceEncoding}} ||| VARSTRING --- The unicode encoding of the file. (utf8,utf8n,utf16be,utf16le,utf32be,utf32le)
\item [\colorbox{tagtype}{\color{white} \textbf{\textsf{PARAMETER}}}] \textbf{\underline{replicate}} ||| BOOLEAN --- Whether to replicate the new file. Defaults to FALSE.
\item [\colorbox{tagtype}{\color{white} \textbf{\textsf{PARAMETER}}}] \textbf{\underline{timeOut}} ||| INTEGER4 --- The time in ms to wait for the operation to complete. A value of 0 causes the call to return immediately. Defaults to no timeout (-1).
\item [\colorbox{tagtype}{\color{white} \textbf{\textsf{PARAMETER}}}] \textbf{\underline{sourcePath}} ||| VARSTRING --- The path and name of the file.
\item [\colorbox{tagtype}{\color{white} \textbf{\textsf{PARAMETER}}}] \textbf{\underline{destinationLogicalName}} ||| VARSTRING --- The logical name of the file to create.
\item [\colorbox{tagtype}{\color{white} \textbf{\textsf{PARAMETER}}}] \textbf{\underline{maxConnections}} ||| INTEGER4 --- The maximum number of target nodes to write to concurrently. Defaults to 1.
\item [\colorbox{tagtype}{\color{white} \textbf{\textsf{PARAMETER}}}] \textbf{\underline{espServerIpPort}} ||| VARSTRING --- The url of the ESP file copying service. Defaults to the value of ws\_fs\_server in the environment.
\item [\colorbox{tagtype}{\color{white} \textbf{\textsf{PARAMETER}}}] \textbf{\underline{allowOverwrite}} ||| BOOLEAN --- Is it valid to overwrite an existing file of the same name? Defaults to FALSE
\item [\colorbox{tagtype}{\color{white} \textbf{\textsf{PARAMETER}}}] \textbf{\underline{sourceIP}} ||| VARSTRING --- The IP address of the file.
\end{description}







\par
\begin{description}
\item [\colorbox{tagtype}{\color{white} \textbf{\textsf{RETURN}}}] \textbf{VARSTRING} --- The DFU workunit id for the job.
\end{description}




\rule{\linewidth}{0.5pt}
\subsection*{\textsf{\colorbox{headtoc}{\color{white} FUNCTION}
SprayXml}}

\hypertarget{ecldoc:file.sprayxml}{}
\hspace{0pt} \hyperlink{ecldoc:File}{File} \textbackslash 

{\renewcommand{\arraystretch}{1.5}
\begin{tabularx}{\textwidth}{|>{\raggedright\arraybackslash}l|X|}
\hline
\hspace{0pt}\mytexttt{\color{red} } & \textbf{SprayXml} \\
\hline
\multicolumn{2}{|>{\raggedright\arraybackslash}X|}{\hspace{0pt}\mytexttt{\color{param} (varstring sourceIP, varstring sourcePath, integer4 sourceMaxRecordSize=8192, varstring sourceRowTag, varstring sourceEncoding='utf8', varstring destinationGroup, varstring destinationLogicalName, integer4 timeOut=-1, varstring espServerIpPort=GETENV('ws\_fs\_server'), integer4 maxConnections=-1, boolean allowOverwrite=FALSE, boolean replicate=FALSE, boolean compress=FALSE, boolean failIfNoSourceFile=FALSE, integer4 expireDays=-1)}} \\
\hline
\end{tabularx}
}

\par





Same as fSprayXml, but does not return the DFU Workunit ID.






\par
\begin{description}
\item [\colorbox{tagtype}{\color{white} \textbf{\textsf{PARAMETER}}}] \textbf{\underline{timeout}} ||| INTEGER4 --- No Doc
\item [\colorbox{tagtype}{\color{white} \textbf{\textsf{PARAMETER}}}] \textbf{\underline{replicate}} ||| BOOLEAN --- No Doc
\item [\colorbox{tagtype}{\color{white} \textbf{\textsf{PARAMETER}}}] \textbf{\underline{espserveripport}} ||| VARSTRING --- No Doc
\item [\colorbox{tagtype}{\color{white} \textbf{\textsf{PARAMETER}}}] \textbf{\underline{failifnosourcefile}} ||| BOOLEAN --- No Doc
\item [\colorbox{tagtype}{\color{white} \textbf{\textsf{PARAMETER}}}] \textbf{\underline{allowoverwrite}} ||| BOOLEAN --- No Doc
\item [\colorbox{tagtype}{\color{white} \textbf{\textsf{PARAMETER}}}] \textbf{\underline{sourceip}} ||| VARSTRING --- No Doc
\item [\colorbox{tagtype}{\color{white} \textbf{\textsf{PARAMETER}}}] \textbf{\underline{sourcemaxrecordsize}} ||| INTEGER4 --- No Doc
\item [\colorbox{tagtype}{\color{white} \textbf{\textsf{PARAMETER}}}] \textbf{\underline{compress}} ||| BOOLEAN --- No Doc
\item [\colorbox{tagtype}{\color{white} \textbf{\textsf{PARAMETER}}}] \textbf{\underline{maxconnections}} ||| INTEGER4 --- No Doc
\item [\colorbox{tagtype}{\color{white} \textbf{\textsf{PARAMETER}}}] \textbf{\underline{sourceencoding}} ||| VARSTRING --- No Doc
\item [\colorbox{tagtype}{\color{white} \textbf{\textsf{PARAMETER}}}] \textbf{\underline{expiredays}} ||| INTEGER4 --- No Doc
\item [\colorbox{tagtype}{\color{white} \textbf{\textsf{PARAMETER}}}] \textbf{\underline{sourcerowtag}} ||| VARSTRING --- No Doc
\item [\colorbox{tagtype}{\color{white} \textbf{\textsf{PARAMETER}}}] \textbf{\underline{destinationgroup}} ||| VARSTRING --- No Doc
\item [\colorbox{tagtype}{\color{white} \textbf{\textsf{PARAMETER}}}] \textbf{\underline{destinationlogicalname}} ||| VARSTRING --- No Doc
\item [\colorbox{tagtype}{\color{white} \textbf{\textsf{PARAMETER}}}] \textbf{\underline{sourcepath}} ||| VARSTRING --- No Doc
\end{description}







\par
\begin{description}
\item [\colorbox{tagtype}{\color{white} \textbf{\textsf{RETURN}}}] \textbf{} --- 
\end{description}






\par
\begin{description}
\item [\colorbox{tagtype}{\color{white} \textbf{\textsf{SEE}}}] fSprayXml
\end{description}




\rule{\linewidth}{0.5pt}
\subsection*{\textsf{\colorbox{headtoc}{\color{white} FUNCTION}
fDespray}}

\hypertarget{ecldoc:file.fdespray}{}
\hspace{0pt} \hyperlink{ecldoc:File}{File} \textbackslash 

{\renewcommand{\arraystretch}{1.5}
\begin{tabularx}{\textwidth}{|>{\raggedright\arraybackslash}l|X|}
\hline
\hspace{0pt}\mytexttt{\color{red} varstring} & \textbf{fDespray} \\
\hline
\multicolumn{2}{|>{\raggedright\arraybackslash}X|}{\hspace{0pt}\mytexttt{\color{param} (varstring logicalName, varstring destinationIP, varstring destinationPath, integer4 timeOut=-1, varstring espServerIpPort=GETENV('ws\_fs\_server'), integer4 maxConnections=-1, boolean allowOverwrite=FALSE)}} \\
\hline
\end{tabularx}
}

\par





Copies a distributed file from multiple machines, and desprays it to a single file on a single machine.






\par
\begin{description}
\item [\colorbox{tagtype}{\color{white} \textbf{\textsf{PARAMETER}}}] \textbf{\underline{logicalName}} ||| VARSTRING --- The name of the file to despray.
\item [\colorbox{tagtype}{\color{white} \textbf{\textsf{PARAMETER}}}] \textbf{\underline{destinationPath}} ||| VARSTRING --- The path of the file to create on the destination machine.
\item [\colorbox{tagtype}{\color{white} \textbf{\textsf{PARAMETER}}}] \textbf{\underline{maxConnections}} ||| INTEGER4 --- The maximum number of target nodes to write to concurrently. Defaults to 1.
\item [\colorbox{tagtype}{\color{white} \textbf{\textsf{PARAMETER}}}] \textbf{\underline{espServerIpPort}} ||| VARSTRING --- The url of the ESP file copying service. Defaults to the value of ws\_fs\_server in the environment.
\item [\colorbox{tagtype}{\color{white} \textbf{\textsf{PARAMETER}}}] \textbf{\underline{destinationIP}} ||| VARSTRING --- The IP of the target machine.
\item [\colorbox{tagtype}{\color{white} \textbf{\textsf{PARAMETER}}}] \textbf{\underline{allowOverwrite}} ||| BOOLEAN --- Is it valid to overwrite an existing file of the same name? Defaults to FALSE
\item [\colorbox{tagtype}{\color{white} \textbf{\textsf{PARAMETER}}}] \textbf{\underline{timeOut}} ||| INTEGER4 --- The time in ms to wait for the operation to complete. A value of 0 causes the call to return immediately. Defaults to no timeout (-1).
\end{description}







\par
\begin{description}
\item [\colorbox{tagtype}{\color{white} \textbf{\textsf{RETURN}}}] \textbf{VARSTRING} --- The DFU workunit id for the job.
\end{description}




\rule{\linewidth}{0.5pt}
\subsection*{\textsf{\colorbox{headtoc}{\color{white} FUNCTION}
Despray}}

\hypertarget{ecldoc:file.despray}{}
\hspace{0pt} \hyperlink{ecldoc:File}{File} \textbackslash 

{\renewcommand{\arraystretch}{1.5}
\begin{tabularx}{\textwidth}{|>{\raggedright\arraybackslash}l|X|}
\hline
\hspace{0pt}\mytexttt{\color{red} } & \textbf{Despray} \\
\hline
\multicolumn{2}{|>{\raggedright\arraybackslash}X|}{\hspace{0pt}\mytexttt{\color{param} (varstring logicalName, varstring destinationIP, varstring destinationPath, integer4 timeOut=-1, varstring espServerIpPort=GETENV('ws\_fs\_server'), integer4 maxConnections=-1, boolean allowOverwrite=FALSE)}} \\
\hline
\end{tabularx}
}

\par





Same as fDespray, but does not return the DFU Workunit ID.






\par
\begin{description}
\item [\colorbox{tagtype}{\color{white} \textbf{\textsf{PARAMETER}}}] \textbf{\underline{timeout}} ||| INTEGER4 --- No Doc
\item [\colorbox{tagtype}{\color{white} \textbf{\textsf{PARAMETER}}}] \textbf{\underline{espserveripport}} ||| VARSTRING --- No Doc
\item [\colorbox{tagtype}{\color{white} \textbf{\textsf{PARAMETER}}}] \textbf{\underline{maxconnections}} ||| INTEGER4 --- No Doc
\item [\colorbox{tagtype}{\color{white} \textbf{\textsf{PARAMETER}}}] \textbf{\underline{destinationpath}} ||| VARSTRING --- No Doc
\item [\colorbox{tagtype}{\color{white} \textbf{\textsf{PARAMETER}}}] \textbf{\underline{destinationip}} ||| VARSTRING --- No Doc
\item [\colorbox{tagtype}{\color{white} \textbf{\textsf{PARAMETER}}}] \textbf{\underline{logicalname}} ||| VARSTRING --- No Doc
\item [\colorbox{tagtype}{\color{white} \textbf{\textsf{PARAMETER}}}] \textbf{\underline{allowoverwrite}} ||| BOOLEAN --- No Doc
\end{description}







\par
\begin{description}
\item [\colorbox{tagtype}{\color{white} \textbf{\textsf{RETURN}}}] \textbf{} --- 
\end{description}






\par
\begin{description}
\item [\colorbox{tagtype}{\color{white} \textbf{\textsf{SEE}}}] fDespray
\end{description}




\rule{\linewidth}{0.5pt}
\subsection*{\textsf{\colorbox{headtoc}{\color{white} FUNCTION}
fCopy}}

\hypertarget{ecldoc:file.fcopy}{}
\hspace{0pt} \hyperlink{ecldoc:File}{File} \textbackslash 

{\renewcommand{\arraystretch}{1.5}
\begin{tabularx}{\textwidth}{|>{\raggedright\arraybackslash}l|X|}
\hline
\hspace{0pt}\mytexttt{\color{red} varstring} & \textbf{fCopy} \\
\hline
\multicolumn{2}{|>{\raggedright\arraybackslash}X|}{\hspace{0pt}\mytexttt{\color{param} (varstring sourceLogicalName, varstring destinationGroup, varstring destinationLogicalName, varstring sourceDali='', integer4 timeOut=-1, varstring espServerIpPort=GETENV('ws\_fs\_server'), integer4 maxConnections=-1, boolean allowOverwrite=FALSE, boolean replicate=FALSE, boolean asSuperfile=FALSE, boolean compress=FALSE, boolean forcePush=FALSE, integer4 transferBufferSize=0, boolean preserveCompression=TRUE)}} \\
\hline
\end{tabularx}
}

\par





Copies a distributed file to another distributed file.






\par
\begin{description}
\item [\colorbox{tagtype}{\color{white} \textbf{\textsf{PARAMETER}}}] \textbf{\underline{replicate}} ||| BOOLEAN --- Should the copied file also be replicated on the destination? Defaults to FALSE
\item [\colorbox{tagtype}{\color{white} \textbf{\textsf{PARAMETER}}}] \textbf{\underline{transferBufferSize}} ||| INTEGER4 --- Overrides the size (in bytes) of the internal buffer used to copy the file. Default is 64k.
\item [\colorbox{tagtype}{\color{white} \textbf{\textsf{PARAMETER}}}] \textbf{\underline{compress}} ||| BOOLEAN --- Whether to compress the new file. Defaults to FALSE.
\item [\colorbox{tagtype}{\color{white} \textbf{\textsf{PARAMETER}}}] \textbf{\underline{destinationGroup}} ||| VARSTRING --- The name of the group to distribute the file across.
\item [\colorbox{tagtype}{\color{white} \textbf{\textsf{PARAMETER}}}] \textbf{\underline{timeOut}} ||| INTEGER4 --- The time in ms to wait for the operation to complete. A value of 0 causes the call to return immediately. Defaults to no timeout (-1).
\item [\colorbox{tagtype}{\color{white} \textbf{\textsf{PARAMETER}}}] \textbf{\underline{allowOverwrite}} ||| BOOLEAN --- Is it valid to overwrite an existing file of the same name? Defaults to FALSE
\item [\colorbox{tagtype}{\color{white} \textbf{\textsf{PARAMETER}}}] \textbf{\underline{asSuperfile}} ||| BOOLEAN --- Should the file be copied as a superfile? If TRUE and source is a superfile, then the operation creates a superfile on the target, creating sub-files as needed and only overwriting existing sub-files whose content has changed. If FALSE, a single file is created. Defaults to FALSE.
\item [\colorbox{tagtype}{\color{white} \textbf{\textsf{PARAMETER}}}] \textbf{\underline{sourceDali}} ||| VARSTRING --- The dali that contains the source file (blank implies same dali). Defaults to same dali.
\item [\colorbox{tagtype}{\color{white} \textbf{\textsf{PARAMETER}}}] \textbf{\underline{destinationLogicalName}} ||| VARSTRING --- The logical name of the file to create.
\item [\colorbox{tagtype}{\color{white} \textbf{\textsf{PARAMETER}}}] \textbf{\underline{maxConnections}} ||| INTEGER4 --- The maximum number of target nodes to write to concurrently. Defaults to 1.
\item [\colorbox{tagtype}{\color{white} \textbf{\textsf{PARAMETER}}}] \textbf{\underline{sourceLogicalName}} ||| VARSTRING --- The name of the file to despray.
\item [\colorbox{tagtype}{\color{white} \textbf{\textsf{PARAMETER}}}] \textbf{\underline{espServerIpPort}} ||| VARSTRING --- The url of the ESP file copying service. Defaults to the value of ws\_fs\_server in the environment.
\item [\colorbox{tagtype}{\color{white} \textbf{\textsf{PARAMETER}}}] \textbf{\underline{forcePush}} ||| BOOLEAN --- Should the copy process be executed on the source nodes (push) or on the destination nodes (pull)? Default is to pull.
\item [\colorbox{tagtype}{\color{white} \textbf{\textsf{PARAMETER}}}] \textbf{\underline{preservecompression}} ||| BOOLEAN --- No Doc
\end{description}







\par
\begin{description}
\item [\colorbox{tagtype}{\color{white} \textbf{\textsf{RETURN}}}] \textbf{VARSTRING} --- The DFU workunit id for the job.
\end{description}




\rule{\linewidth}{0.5pt}
\subsection*{\textsf{\colorbox{headtoc}{\color{white} FUNCTION}
Copy}}

\hypertarget{ecldoc:file.copy}{}
\hspace{0pt} \hyperlink{ecldoc:File}{File} \textbackslash 

{\renewcommand{\arraystretch}{1.5}
\begin{tabularx}{\textwidth}{|>{\raggedright\arraybackslash}l|X|}
\hline
\hspace{0pt}\mytexttt{\color{red} } & \textbf{Copy} \\
\hline
\multicolumn{2}{|>{\raggedright\arraybackslash}X|}{\hspace{0pt}\mytexttt{\color{param} (varstring sourceLogicalName, varstring destinationGroup, varstring destinationLogicalName, varstring sourceDali='', integer4 timeOut=-1, varstring espServerIpPort=GETENV('ws\_fs\_server'), integer4 maxConnections=-1, boolean allowOverwrite=FALSE, boolean replicate=FALSE, boolean asSuperfile=FALSE, boolean compress=FALSE, boolean forcePush=FALSE, integer4 transferBufferSize=0, boolean preserveCompression=TRUE)}} \\
\hline
\end{tabularx}
}

\par





Same as fCopy, but does not return the DFU Workunit ID.






\par
\begin{description}
\item [\colorbox{tagtype}{\color{white} \textbf{\textsf{PARAMETER}}}] \textbf{\underline{espserveripport}} ||| VARSTRING --- No Doc
\item [\colorbox{tagtype}{\color{white} \textbf{\textsf{PARAMETER}}}] \textbf{\underline{replicate}} ||| BOOLEAN --- No Doc
\item [\colorbox{tagtype}{\color{white} \textbf{\textsf{PARAMETER}}}] \textbf{\underline{forcepush}} ||| BOOLEAN --- No Doc
\item [\colorbox{tagtype}{\color{white} \textbf{\textsf{PARAMETER}}}] \textbf{\underline{compress}} ||| BOOLEAN --- No Doc
\item [\colorbox{tagtype}{\color{white} \textbf{\textsf{PARAMETER}}}] \textbf{\underline{sourcelogicalname}} ||| VARSTRING --- No Doc
\item [\colorbox{tagtype}{\color{white} \textbf{\textsf{PARAMETER}}}] \textbf{\underline{preservecompression}} ||| BOOLEAN --- No Doc
\item [\colorbox{tagtype}{\color{white} \textbf{\textsf{PARAMETER}}}] \textbf{\underline{allowoverwrite}} ||| BOOLEAN --- No Doc
\item [\colorbox{tagtype}{\color{white} \textbf{\textsf{PARAMETER}}}] \textbf{\underline{timeout}} ||| INTEGER4 --- No Doc
\item [\colorbox{tagtype}{\color{white} \textbf{\textsf{PARAMETER}}}] \textbf{\underline{assuperfile}} ||| BOOLEAN --- No Doc
\item [\colorbox{tagtype}{\color{white} \textbf{\textsf{PARAMETER}}}] \textbf{\underline{maxconnections}} ||| INTEGER4 --- No Doc
\item [\colorbox{tagtype}{\color{white} \textbf{\textsf{PARAMETER}}}] \textbf{\underline{destinationgroup}} ||| VARSTRING --- No Doc
\item [\colorbox{tagtype}{\color{white} \textbf{\textsf{PARAMETER}}}] \textbf{\underline{destinationlogicalname}} ||| VARSTRING --- No Doc
\item [\colorbox{tagtype}{\color{white} \textbf{\textsf{PARAMETER}}}] \textbf{\underline{sourcedali}} ||| VARSTRING --- No Doc
\item [\colorbox{tagtype}{\color{white} \textbf{\textsf{PARAMETER}}}] \textbf{\underline{transferbuffersize}} ||| INTEGER4 --- No Doc
\end{description}







\par
\begin{description}
\item [\colorbox{tagtype}{\color{white} \textbf{\textsf{RETURN}}}] \textbf{} --- 
\end{description}






\par
\begin{description}
\item [\colorbox{tagtype}{\color{white} \textbf{\textsf{SEE}}}] fCopy
\end{description}




\rule{\linewidth}{0.5pt}
\subsection*{\textsf{\colorbox{headtoc}{\color{white} FUNCTION}
fReplicate}}

\hypertarget{ecldoc:file.freplicate}{}
\hspace{0pt} \hyperlink{ecldoc:File}{File} \textbackslash 

{\renewcommand{\arraystretch}{1.5}
\begin{tabularx}{\textwidth}{|>{\raggedright\arraybackslash}l|X|}
\hline
\hspace{0pt}\mytexttt{\color{red} varstring} & \textbf{fReplicate} \\
\hline
\multicolumn{2}{|>{\raggedright\arraybackslash}X|}{\hspace{0pt}\mytexttt{\color{param} (varstring logicalName, integer4 timeOut=-1, varstring espServerIpPort=GETENV('ws\_fs\_server'))}} \\
\hline
\end{tabularx}
}

\par





Ensures the specified file is replicated to its mirror copies.






\par
\begin{description}
\item [\colorbox{tagtype}{\color{white} \textbf{\textsf{PARAMETER}}}] \textbf{\underline{espServerIpPort}} ||| VARSTRING --- The url of the ESP file copying service. Defaults to the value of ws\_fs\_server in the environment.
\item [\colorbox{tagtype}{\color{white} \textbf{\textsf{PARAMETER}}}] \textbf{\underline{timeOut}} ||| INTEGER4 --- The time in ms to wait for the operation to complete. A value of 0 causes the call to return immediately. Defaults to no timeout (-1).
\item [\colorbox{tagtype}{\color{white} \textbf{\textsf{PARAMETER}}}] \textbf{\underline{logicalName}} ||| VARSTRING --- The name of the file to replicate.
\end{description}







\par
\begin{description}
\item [\colorbox{tagtype}{\color{white} \textbf{\textsf{RETURN}}}] \textbf{VARSTRING} --- The DFU workunit id for the job.
\end{description}




\rule{\linewidth}{0.5pt}
\subsection*{\textsf{\colorbox{headtoc}{\color{white} FUNCTION}
Replicate}}

\hypertarget{ecldoc:file.replicate}{}
\hspace{0pt} \hyperlink{ecldoc:File}{File} \textbackslash 

{\renewcommand{\arraystretch}{1.5}
\begin{tabularx}{\textwidth}{|>{\raggedright\arraybackslash}l|X|}
\hline
\hspace{0pt}\mytexttt{\color{red} } & \textbf{Replicate} \\
\hline
\multicolumn{2}{|>{\raggedright\arraybackslash}X|}{\hspace{0pt}\mytexttt{\color{param} (varstring logicalName, integer4 timeOut=-1, varstring espServerIpPort=GETENV('ws\_fs\_server'))}} \\
\hline
\end{tabularx}
}

\par





Same as fReplicated, but does not return the DFU Workunit ID.






\par
\begin{description}
\item [\colorbox{tagtype}{\color{white} \textbf{\textsf{PARAMETER}}}] \textbf{\underline{timeout}} ||| INTEGER4 --- No Doc
\item [\colorbox{tagtype}{\color{white} \textbf{\textsf{PARAMETER}}}] \textbf{\underline{espserveripport}} ||| VARSTRING --- No Doc
\item [\colorbox{tagtype}{\color{white} \textbf{\textsf{PARAMETER}}}] \textbf{\underline{logicalname}} ||| VARSTRING --- No Doc
\end{description}







\par
\begin{description}
\item [\colorbox{tagtype}{\color{white} \textbf{\textsf{RETURN}}}] \textbf{} --- 
\end{description}






\par
\begin{description}
\item [\colorbox{tagtype}{\color{white} \textbf{\textsf{SEE}}}] fReplicate
\end{description}




\rule{\linewidth}{0.5pt}
\subsection*{\textsf{\colorbox{headtoc}{\color{white} FUNCTION}
fRemotePull}}

\hypertarget{ecldoc:file.fremotepull}{}
\hspace{0pt} \hyperlink{ecldoc:File}{File} \textbackslash 

{\renewcommand{\arraystretch}{1.5}
\begin{tabularx}{\textwidth}{|>{\raggedright\arraybackslash}l|X|}
\hline
\hspace{0pt}\mytexttt{\color{red} varstring} & \textbf{fRemotePull} \\
\hline
\multicolumn{2}{|>{\raggedright\arraybackslash}X|}{\hspace{0pt}\mytexttt{\color{param} (varstring remoteEspFsURL, varstring sourceLogicalName, varstring destinationGroup, varstring destinationLogicalName, integer4 timeOut=-1, integer4 maxConnections=-1, boolean allowOverwrite=FALSE, boolean replicate=FALSE, boolean asSuperfile=FALSE, boolean forcePush=FALSE, integer4 transferBufferSize=0, boolean wrap=FALSE, boolean compress=FALSE)}} \\
\hline
\end{tabularx}
}

\par





Copies a distributed file to a distributed file on remote system. Similar to fCopy, except the copy executes remotely. Since the DFU workunit executes on the remote DFU server, the user name authentication must be the same on both systems, and the user must have rights to copy files on both systems.






\par
\begin{description}
\item [\colorbox{tagtype}{\color{white} \textbf{\textsf{PARAMETER}}}] \textbf{\underline{remoteEspFsURL}} ||| VARSTRING --- The url of the remote ESP file copying service.
\item [\colorbox{tagtype}{\color{white} \textbf{\textsf{PARAMETER}}}] \textbf{\underline{replicate}} ||| BOOLEAN --- Should the copied file also be replicated on the destination? Defaults to FALSE
\item [\colorbox{tagtype}{\color{white} \textbf{\textsf{PARAMETER}}}] \textbf{\underline{transferBufferSize}} ||| INTEGER4 --- Overrides the size (in bytes) of the internal buffer used to copy the file. Default is 64k.
\item [\colorbox{tagtype}{\color{white} \textbf{\textsf{PARAMETER}}}] \textbf{\underline{compress}} ||| BOOLEAN --- Whether to compress the new file. Defaults to FALSE.
\item [\colorbox{tagtype}{\color{white} \textbf{\textsf{PARAMETER}}}] \textbf{\underline{destinationGroup}} ||| VARSTRING --- The name of the group to distribute the file across.
\item [\colorbox{tagtype}{\color{white} \textbf{\textsf{PARAMETER}}}] \textbf{\underline{wrap}} ||| BOOLEAN --- Should the fileparts be wrapped when copying to a smaller sized cluster? The default is FALSE.
\item [\colorbox{tagtype}{\color{white} \textbf{\textsf{PARAMETER}}}] \textbf{\underline{timeOut}} ||| INTEGER4 --- The time in ms to wait for the operation to complete. A value of 0 causes the call to return immediately. Defaults to no timeout (-1).
\item [\colorbox{tagtype}{\color{white} \textbf{\textsf{PARAMETER}}}] \textbf{\underline{asSuperfile}} ||| BOOLEAN --- Should the file be copied as a superfile? If TRUE and source is a superfile, then the operation creates a superfile on the target, creating sub-files as needed and only overwriting existing sub-files whose content has changed. If FALSE a single file is created. Defaults to FALSE.
\item [\colorbox{tagtype}{\color{white} \textbf{\textsf{PARAMETER}}}] \textbf{\underline{destinationLogicalName}} ||| VARSTRING --- The logical name of the file to create.
\item [\colorbox{tagtype}{\color{white} \textbf{\textsf{PARAMETER}}}] \textbf{\underline{maxConnections}} ||| INTEGER4 --- The maximum number of target nodes to write to concurrently. Defaults to 1.
\item [\colorbox{tagtype}{\color{white} \textbf{\textsf{PARAMETER}}}] \textbf{\underline{sourceLogicalName}} ||| VARSTRING --- The name of the file to despray.
\item [\colorbox{tagtype}{\color{white} \textbf{\textsf{PARAMETER}}}] \textbf{\underline{allowOverwrite}} ||| BOOLEAN --- Is it valid to overwrite an existing file of the same name? Defaults to FALSE
\item [\colorbox{tagtype}{\color{white} \textbf{\textsf{PARAMETER}}}] \textbf{\underline{forcePush}} ||| BOOLEAN --- Should the copy process should be executed on the source nodes (push) or on the destination nodes (pull)? Default is to pull.
\end{description}







\par
\begin{description}
\item [\colorbox{tagtype}{\color{white} \textbf{\textsf{RETURN}}}] \textbf{VARSTRING} --- The DFU workunit id for the job.
\end{description}




\rule{\linewidth}{0.5pt}
\subsection*{\textsf{\colorbox{headtoc}{\color{white} FUNCTION}
RemotePull}}

\hypertarget{ecldoc:file.remotepull}{}
\hspace{0pt} \hyperlink{ecldoc:File}{File} \textbackslash 

{\renewcommand{\arraystretch}{1.5}
\begin{tabularx}{\textwidth}{|>{\raggedright\arraybackslash}l|X|}
\hline
\hspace{0pt}\mytexttt{\color{red} } & \textbf{RemotePull} \\
\hline
\multicolumn{2}{|>{\raggedright\arraybackslash}X|}{\hspace{0pt}\mytexttt{\color{param} (varstring remoteEspFsURL, varstring sourceLogicalName, varstring destinationGroup, varstring destinationLogicalName, integer4 timeOut=-1, integer4 maxConnections=-1, boolean allowOverwrite=FALSE, boolean replicate=FALSE, boolean asSuperfile=FALSE, boolean forcePush=FALSE, integer4 transferBufferSize=0, boolean wrap=FALSE, boolean compress=FALSE)}} \\
\hline
\end{tabularx}
}

\par





Same as fRemotePull, but does not return the DFU Workunit ID.






\par
\begin{description}
\item [\colorbox{tagtype}{\color{white} \textbf{\textsf{PARAMETER}}}] \textbf{\underline{forcepush}} ||| BOOLEAN --- No Doc
\item [\colorbox{tagtype}{\color{white} \textbf{\textsf{PARAMETER}}}] \textbf{\underline{replicate}} ||| BOOLEAN --- No Doc
\item [\colorbox{tagtype}{\color{white} \textbf{\textsf{PARAMETER}}}] \textbf{\underline{transferbuffersize}} ||| INTEGER4 --- No Doc
\item [\colorbox{tagtype}{\color{white} \textbf{\textsf{PARAMETER}}}] \textbf{\underline{destinationgroup}} ||| VARSTRING --- No Doc
\item [\colorbox{tagtype}{\color{white} \textbf{\textsf{PARAMETER}}}] \textbf{\underline{sourcelogicalname}} ||| VARSTRING --- No Doc
\item [\colorbox{tagtype}{\color{white} \textbf{\textsf{PARAMETER}}}] \textbf{\underline{wrap}} ||| BOOLEAN --- No Doc
\item [\colorbox{tagtype}{\color{white} \textbf{\textsf{PARAMETER}}}] \textbf{\underline{allowoverwrite}} ||| BOOLEAN --- No Doc
\item [\colorbox{tagtype}{\color{white} \textbf{\textsf{PARAMETER}}}] \textbf{\underline{timeout}} ||| INTEGER4 --- No Doc
\item [\colorbox{tagtype}{\color{white} \textbf{\textsf{PARAMETER}}}] \textbf{\underline{assuperfile}} ||| BOOLEAN --- No Doc
\item [\colorbox{tagtype}{\color{white} \textbf{\textsf{PARAMETER}}}] \textbf{\underline{maxconnections}} ||| INTEGER4 --- No Doc
\item [\colorbox{tagtype}{\color{white} \textbf{\textsf{PARAMETER}}}] \textbf{\underline{compress}} ||| BOOLEAN --- No Doc
\item [\colorbox{tagtype}{\color{white} \textbf{\textsf{PARAMETER}}}] \textbf{\underline{destinationlogicalname}} ||| VARSTRING --- No Doc
\item [\colorbox{tagtype}{\color{white} \textbf{\textsf{PARAMETER}}}] \textbf{\underline{remoteespfsurl}} ||| VARSTRING --- No Doc
\end{description}







\par
\begin{description}
\item [\colorbox{tagtype}{\color{white} \textbf{\textsf{RETURN}}}] \textbf{} --- 
\end{description}






\par
\begin{description}
\item [\colorbox{tagtype}{\color{white} \textbf{\textsf{SEE}}}] fRemotePull
\end{description}




\rule{\linewidth}{0.5pt}
\subsection*{\textsf{\colorbox{headtoc}{\color{white} FUNCTION}
fMonitorLogicalFileName}}

\hypertarget{ecldoc:file.fmonitorlogicalfilename}{}
\hspace{0pt} \hyperlink{ecldoc:File}{File} \textbackslash 

{\renewcommand{\arraystretch}{1.5}
\begin{tabularx}{\textwidth}{|>{\raggedright\arraybackslash}l|X|}
\hline
\hspace{0pt}\mytexttt{\color{red} varstring} & \textbf{fMonitorLogicalFileName} \\
\hline
\multicolumn{2}{|>{\raggedright\arraybackslash}X|}{\hspace{0pt}\mytexttt{\color{param} (varstring eventToFire, varstring name, integer4 shotCount=1, varstring espServerIpPort=GETENV('ws\_fs\_server'))}} \\
\hline
\end{tabularx}
}

\par





Creates a file monitor job in the DFU Server. If an appropriately named file arrives in this interval it will fire the event with the name of the triggering object as the event subtype (see the EVENT function).






\par
\begin{description}
\item [\colorbox{tagtype}{\color{white} \textbf{\textsf{PARAMETER}}}] \textbf{\underline{espServerIpPort}} ||| VARSTRING --- The url of the ESP file copying service. Defaults to the value of ws\_fs\_server in the environment.
\item [\colorbox{tagtype}{\color{white} \textbf{\textsf{PARAMETER}}}] \textbf{\underline{shotCount}} ||| INTEGER4 --- The number of times to generate the event before the monitoring job completes. A value of -1 indicates the monitoring job continues until manually aborted. The default is 1.
\item [\colorbox{tagtype}{\color{white} \textbf{\textsf{PARAMETER}}}] \textbf{\underline{eventToFire}} ||| VARSTRING --- The user-defined name of the event to fire when the filename appears. This value is used as the first parameter to the EVENT function.
\item [\colorbox{tagtype}{\color{white} \textbf{\textsf{PARAMETER}}}] \textbf{\underline{name}} ||| VARSTRING --- The name of the logical file to monitor. This may contain wildcard characters ( * and ?)
\end{description}







\par
\begin{description}
\item [\colorbox{tagtype}{\color{white} \textbf{\textsf{RETURN}}}] \textbf{VARSTRING} --- The DFU workunit id for the job.
\end{description}




\rule{\linewidth}{0.5pt}
\subsection*{\textsf{\colorbox{headtoc}{\color{white} FUNCTION}
MonitorLogicalFileName}}

\hypertarget{ecldoc:file.monitorlogicalfilename}{}
\hspace{0pt} \hyperlink{ecldoc:File}{File} \textbackslash 

{\renewcommand{\arraystretch}{1.5}
\begin{tabularx}{\textwidth}{|>{\raggedright\arraybackslash}l|X|}
\hline
\hspace{0pt}\mytexttt{\color{red} } & \textbf{MonitorLogicalFileName} \\
\hline
\multicolumn{2}{|>{\raggedright\arraybackslash}X|}{\hspace{0pt}\mytexttt{\color{param} (varstring eventToFire, varstring name, integer4 shotCount=1, varstring espServerIpPort=GETENV('ws\_fs\_server'))}} \\
\hline
\end{tabularx}
}

\par





Same as fMonitorLogicalFileName, but does not return the DFU Workunit ID.






\par
\begin{description}
\item [\colorbox{tagtype}{\color{white} \textbf{\textsf{PARAMETER}}}] \textbf{\underline{espserveripport}} ||| VARSTRING --- No Doc
\item [\colorbox{tagtype}{\color{white} \textbf{\textsf{PARAMETER}}}] \textbf{\underline{shotcount}} ||| INTEGER4 --- No Doc
\item [\colorbox{tagtype}{\color{white} \textbf{\textsf{PARAMETER}}}] \textbf{\underline{eventtofire}} ||| VARSTRING --- No Doc
\item [\colorbox{tagtype}{\color{white} \textbf{\textsf{PARAMETER}}}] \textbf{\underline{name}} ||| VARSTRING --- No Doc
\end{description}







\par
\begin{description}
\item [\colorbox{tagtype}{\color{white} \textbf{\textsf{RETURN}}}] \textbf{} --- 
\end{description}






\par
\begin{description}
\item [\colorbox{tagtype}{\color{white} \textbf{\textsf{SEE}}}] fMonitorLogicalFileName
\end{description}




\rule{\linewidth}{0.5pt}
\subsection*{\textsf{\colorbox{headtoc}{\color{white} FUNCTION}
fMonitorFile}}

\hypertarget{ecldoc:file.fmonitorfile}{}
\hspace{0pt} \hyperlink{ecldoc:File}{File} \textbackslash 

{\renewcommand{\arraystretch}{1.5}
\begin{tabularx}{\textwidth}{|>{\raggedright\arraybackslash}l|X|}
\hline
\hspace{0pt}\mytexttt{\color{red} varstring} & \textbf{fMonitorFile} \\
\hline
\multicolumn{2}{|>{\raggedright\arraybackslash}X|}{\hspace{0pt}\mytexttt{\color{param} (varstring eventToFire, varstring ip, varstring filename, boolean subDirs=FALSE, integer4 shotCount=1, varstring espServerIpPort=GETENV('ws\_fs\_server'))}} \\
\hline
\end{tabularx}
}

\par





Creates a file monitor job in the DFU Server. If an appropriately named file arrives in this interval it will fire the event with the name of the triggering object as the event subtype (see the EVENT function).






\par
\begin{description}
\item [\colorbox{tagtype}{\color{white} \textbf{\textsf{PARAMETER}}}] \textbf{\underline{filename}} ||| VARSTRING --- The full path of the file(s) to monitor. This may contain wildcard characters ( * and ?)
\item [\colorbox{tagtype}{\color{white} \textbf{\textsf{PARAMETER}}}] \textbf{\underline{ip}} ||| VARSTRING --- The the IP address for the file to monitor. This may be omitted if the filename parameter contains a complete URL.
\item [\colorbox{tagtype}{\color{white} \textbf{\textsf{PARAMETER}}}] \textbf{\underline{espServerIpPort}} ||| VARSTRING --- The url of the ESP file copying service. Defaults to the value of ws\_fs\_server in the environment.
\item [\colorbox{tagtype}{\color{white} \textbf{\textsf{PARAMETER}}}] \textbf{\underline{subDirs}} ||| BOOLEAN --- Whether to include files in sub-directories (when the filename contains wildcards). Defaults to FALSE.
\item [\colorbox{tagtype}{\color{white} \textbf{\textsf{PARAMETER}}}] \textbf{\underline{shotCount}} ||| INTEGER4 --- The number of times to generate the event before the monitoring job completes. A value of -1 indicates the monitoring job continues until manually aborted. The default is 1.
\item [\colorbox{tagtype}{\color{white} \textbf{\textsf{PARAMETER}}}] \textbf{\underline{eventToFire}} ||| VARSTRING --- The user-defined name of the event to fire when the filename appears. This value is used as the first parameter to the EVENT function.
\end{description}







\par
\begin{description}
\item [\colorbox{tagtype}{\color{white} \textbf{\textsf{RETURN}}}] \textbf{VARSTRING} --- The DFU workunit id for the job.
\end{description}




\rule{\linewidth}{0.5pt}
\subsection*{\textsf{\colorbox{headtoc}{\color{white} FUNCTION}
MonitorFile}}

\hypertarget{ecldoc:file.monitorfile}{}
\hspace{0pt} \hyperlink{ecldoc:File}{File} \textbackslash 

{\renewcommand{\arraystretch}{1.5}
\begin{tabularx}{\textwidth}{|>{\raggedright\arraybackslash}l|X|}
\hline
\hspace{0pt}\mytexttt{\color{red} } & \textbf{MonitorFile} \\
\hline
\multicolumn{2}{|>{\raggedright\arraybackslash}X|}{\hspace{0pt}\mytexttt{\color{param} (varstring eventToFire, varstring ip, varstring filename, boolean subdirs=FALSE, integer4 shotCount=1, varstring espServerIpPort=GETENV('ws\_fs\_server'))}} \\
\hline
\end{tabularx}
}

\par





Same as fMonitorFile, but does not return the DFU Workunit ID.






\par
\begin{description}
\item [\colorbox{tagtype}{\color{white} \textbf{\textsf{PARAMETER}}}] \textbf{\underline{filename}} ||| VARSTRING --- No Doc
\item [\colorbox{tagtype}{\color{white} \textbf{\textsf{PARAMETER}}}] \textbf{\underline{shotcount}} ||| INTEGER4 --- No Doc
\item [\colorbox{tagtype}{\color{white} \textbf{\textsf{PARAMETER}}}] \textbf{\underline{ip}} ||| VARSTRING --- No Doc
\item [\colorbox{tagtype}{\color{white} \textbf{\textsf{PARAMETER}}}] \textbf{\underline{espserveripport}} ||| VARSTRING --- No Doc
\item [\colorbox{tagtype}{\color{white} \textbf{\textsf{PARAMETER}}}] \textbf{\underline{eventtofire}} ||| VARSTRING --- No Doc
\item [\colorbox{tagtype}{\color{white} \textbf{\textsf{PARAMETER}}}] \textbf{\underline{subdirs}} ||| BOOLEAN --- No Doc
\end{description}







\par
\begin{description}
\item [\colorbox{tagtype}{\color{white} \textbf{\textsf{RETURN}}}] \textbf{} --- 
\end{description}






\par
\begin{description}
\item [\colorbox{tagtype}{\color{white} \textbf{\textsf{SEE}}}] fMonitorFile
\end{description}




\rule{\linewidth}{0.5pt}
\subsection*{\textsf{\colorbox{headtoc}{\color{white} FUNCTION}
WaitDfuWorkunit}}

\hypertarget{ecldoc:file.waitdfuworkunit}{}
\hspace{0pt} \hyperlink{ecldoc:File}{File} \textbackslash 

{\renewcommand{\arraystretch}{1.5}
\begin{tabularx}{\textwidth}{|>{\raggedright\arraybackslash}l|X|}
\hline
\hspace{0pt}\mytexttt{\color{red} varstring} & \textbf{WaitDfuWorkunit} \\
\hline
\multicolumn{2}{|>{\raggedright\arraybackslash}X|}{\hspace{0pt}\mytexttt{\color{param} (varstring wuid, integer4 timeOut=-1, varstring espServerIpPort=GETENV('ws\_fs\_server'))}} \\
\hline
\end{tabularx}
}

\par





Waits for the specified DFU workunit to finish.






\par
\begin{description}
\item [\colorbox{tagtype}{\color{white} \textbf{\textsf{PARAMETER}}}] \textbf{\underline{wuid}} ||| VARSTRING --- The dfu wfid to wait for.
\item [\colorbox{tagtype}{\color{white} \textbf{\textsf{PARAMETER}}}] \textbf{\underline{espServerIpPort}} ||| VARSTRING --- The url of the ESP file copying service. Defaults to the value of ws\_fs\_server in the environment.
\item [\colorbox{tagtype}{\color{white} \textbf{\textsf{PARAMETER}}}] \textbf{\underline{timeOut}} ||| INTEGER4 --- The time in ms to wait for the operation to complete. A value of 0 causes the call to return immediately. Defaults to no timeout (-1).
\end{description}







\par
\begin{description}
\item [\colorbox{tagtype}{\color{white} \textbf{\textsf{RETURN}}}] \textbf{VARSTRING} --- A string containing the final status string of the DFU workunit.
\end{description}




\rule{\linewidth}{0.5pt}
\subsection*{\textsf{\colorbox{headtoc}{\color{white} FUNCTION}
AbortDfuWorkunit}}

\hypertarget{ecldoc:file.abortdfuworkunit}{}
\hspace{0pt} \hyperlink{ecldoc:File}{File} \textbackslash 

{\renewcommand{\arraystretch}{1.5}
\begin{tabularx}{\textwidth}{|>{\raggedright\arraybackslash}l|X|}
\hline
\hspace{0pt}\mytexttt{\color{red} } & \textbf{AbortDfuWorkunit} \\
\hline
\multicolumn{2}{|>{\raggedright\arraybackslash}X|}{\hspace{0pt}\mytexttt{\color{param} (varstring wuid, varstring espServerIpPort=GETENV('ws\_fs\_server'))}} \\
\hline
\end{tabularx}
}

\par





Aborts the specified DFU workunit.






\par
\begin{description}
\item [\colorbox{tagtype}{\color{white} \textbf{\textsf{PARAMETER}}}] \textbf{\underline{wuid}} ||| VARSTRING --- The dfu wfid to abort.
\item [\colorbox{tagtype}{\color{white} \textbf{\textsf{PARAMETER}}}] \textbf{\underline{espServerIpPort}} ||| VARSTRING --- The url of the ESP file copying service. Defaults to the value of ws\_fs\_server in the environment.
\end{description}







\par
\begin{description}
\item [\colorbox{tagtype}{\color{white} \textbf{\textsf{RETURN}}}] \textbf{} --- 
\end{description}




\rule{\linewidth}{0.5pt}
\subsection*{\textsf{\colorbox{headtoc}{\color{white} FUNCTION}
CreateSuperFile}}

\hypertarget{ecldoc:file.createsuperfile}{}
\hspace{0pt} \hyperlink{ecldoc:File}{File} \textbackslash 

{\renewcommand{\arraystretch}{1.5}
\begin{tabularx}{\textwidth}{|>{\raggedright\arraybackslash}l|X|}
\hline
\hspace{0pt}\mytexttt{\color{red} } & \textbf{CreateSuperFile} \\
\hline
\multicolumn{2}{|>{\raggedright\arraybackslash}X|}{\hspace{0pt}\mytexttt{\color{param} (varstring superName, boolean sequentialParts=FALSE, boolean allowExist=FALSE)}} \\
\hline
\end{tabularx}
}

\par





Creates an empty superfile. This function is not included in a superfile transaction.






\par
\begin{description}
\item [\colorbox{tagtype}{\color{white} \textbf{\textsf{PARAMETER}}}] \textbf{\underline{sequentialParts}} ||| BOOLEAN --- Whether the sub-files must be sequentially ordered. Default to FALSE.
\item [\colorbox{tagtype}{\color{white} \textbf{\textsf{PARAMETER}}}] \textbf{\underline{allowExist}} ||| BOOLEAN --- Indicating whether to post an error if the superfile already exists. If TRUE, no error is posted. Defaults to FALSE.
\item [\colorbox{tagtype}{\color{white} \textbf{\textsf{PARAMETER}}}] \textbf{\underline{superName}} ||| VARSTRING --- The logical name of the superfile.
\end{description}







\par
\begin{description}
\item [\colorbox{tagtype}{\color{white} \textbf{\textsf{RETURN}}}] \textbf{} --- 
\end{description}




\rule{\linewidth}{0.5pt}
\subsection*{\textsf{\colorbox{headtoc}{\color{white} FUNCTION}
SuperFileExists}}

\hypertarget{ecldoc:file.superfileexists}{}
\hspace{0pt} \hyperlink{ecldoc:File}{File} \textbackslash 

{\renewcommand{\arraystretch}{1.5}
\begin{tabularx}{\textwidth}{|>{\raggedright\arraybackslash}l|X|}
\hline
\hspace{0pt}\mytexttt{\color{red} boolean} & \textbf{SuperFileExists} \\
\hline
\multicolumn{2}{|>{\raggedright\arraybackslash}X|}{\hspace{0pt}\mytexttt{\color{param} (varstring superName)}} \\
\hline
\end{tabularx}
}

\par





Checks if the specified filename is present in the Distributed File Utility (DFU) and is a SuperFile.






\par
\begin{description}
\item [\colorbox{tagtype}{\color{white} \textbf{\textsf{PARAMETER}}}] \textbf{\underline{superName}} ||| VARSTRING --- The logical name of the superfile.
\end{description}







\par
\begin{description}
\item [\colorbox{tagtype}{\color{white} \textbf{\textsf{RETURN}}}] \textbf{BOOLEAN} --- Whether the file exists.
\end{description}






\par
\begin{description}
\item [\colorbox{tagtype}{\color{white} \textbf{\textsf{SEE}}}] FileExists
\end{description}




\rule{\linewidth}{0.5pt}
\subsection*{\textsf{\colorbox{headtoc}{\color{white} FUNCTION}
DeleteSuperFile}}

\hypertarget{ecldoc:file.deletesuperfile}{}
\hspace{0pt} \hyperlink{ecldoc:File}{File} \textbackslash 

{\renewcommand{\arraystretch}{1.5}
\begin{tabularx}{\textwidth}{|>{\raggedright\arraybackslash}l|X|}
\hline
\hspace{0pt}\mytexttt{\color{red} } & \textbf{DeleteSuperFile} \\
\hline
\multicolumn{2}{|>{\raggedright\arraybackslash}X|}{\hspace{0pt}\mytexttt{\color{param} (varstring superName, boolean deletesub=FALSE)}} \\
\hline
\end{tabularx}
}

\par





Deletes the superfile.






\par
\begin{description}
\item [\colorbox{tagtype}{\color{white} \textbf{\textsf{PARAMETER}}}] \textbf{\underline{superName}} ||| VARSTRING --- The logical name of the superfile.
\item [\colorbox{tagtype}{\color{white} \textbf{\textsf{PARAMETER}}}] \textbf{\underline{deletesub}} ||| BOOLEAN --- No Doc
\end{description}







\par
\begin{description}
\item [\colorbox{tagtype}{\color{white} \textbf{\textsf{RETURN}}}] \textbf{} --- 
\end{description}






\par
\begin{description}
\item [\colorbox{tagtype}{\color{white} \textbf{\textsf{SEE}}}] FileExists
\end{description}




\rule{\linewidth}{0.5pt}
\subsection*{\textsf{\colorbox{headtoc}{\color{white} FUNCTION}
GetSuperFileSubCount}}

\hypertarget{ecldoc:file.getsuperfilesubcount}{}
\hspace{0pt} \hyperlink{ecldoc:File}{File} \textbackslash 

{\renewcommand{\arraystretch}{1.5}
\begin{tabularx}{\textwidth}{|>{\raggedright\arraybackslash}l|X|}
\hline
\hspace{0pt}\mytexttt{\color{red} unsigned4} & \textbf{GetSuperFileSubCount} \\
\hline
\multicolumn{2}{|>{\raggedright\arraybackslash}X|}{\hspace{0pt}\mytexttt{\color{param} (varstring superName)}} \\
\hline
\end{tabularx}
}

\par





Returns the number of sub-files contained within a superfile.






\par
\begin{description}
\item [\colorbox{tagtype}{\color{white} \textbf{\textsf{PARAMETER}}}] \textbf{\underline{superName}} ||| VARSTRING --- The logical name of the superfile.
\end{description}







\par
\begin{description}
\item [\colorbox{tagtype}{\color{white} \textbf{\textsf{RETURN}}}] \textbf{UNSIGNED4} --- The number of sub-files within the superfile.
\end{description}




\rule{\linewidth}{0.5pt}
\subsection*{\textsf{\colorbox{headtoc}{\color{white} FUNCTION}
GetSuperFileSubName}}

\hypertarget{ecldoc:file.getsuperfilesubname}{}
\hspace{0pt} \hyperlink{ecldoc:File}{File} \textbackslash 

{\renewcommand{\arraystretch}{1.5}
\begin{tabularx}{\textwidth}{|>{\raggedright\arraybackslash}l|X|}
\hline
\hspace{0pt}\mytexttt{\color{red} varstring} & \textbf{GetSuperFileSubName} \\
\hline
\multicolumn{2}{|>{\raggedright\arraybackslash}X|}{\hspace{0pt}\mytexttt{\color{param} (varstring superName, unsigned4 fileNum, boolean absPath=FALSE)}} \\
\hline
\end{tabularx}
}

\par





Returns the name of the Nth sub-file within a superfile.






\par
\begin{description}
\item [\colorbox{tagtype}{\color{white} \textbf{\textsf{PARAMETER}}}] \textbf{\underline{fileNum}} ||| UNSIGNED4 --- The 1-based position of the sub-file to return the name of.
\item [\colorbox{tagtype}{\color{white} \textbf{\textsf{PARAMETER}}}] \textbf{\underline{superName}} ||| VARSTRING --- The logical name of the superfile.
\item [\colorbox{tagtype}{\color{white} \textbf{\textsf{PARAMETER}}}] \textbf{\underline{absPath}} ||| BOOLEAN --- Whether to prepend '\~{}' to the name of the resulting logical file name.
\end{description}







\par
\begin{description}
\item [\colorbox{tagtype}{\color{white} \textbf{\textsf{RETURN}}}] \textbf{VARSTRING} --- The logical name of the selected sub-file.
\end{description}




\rule{\linewidth}{0.5pt}
\subsection*{\textsf{\colorbox{headtoc}{\color{white} FUNCTION}
FindSuperFileSubName}}

\hypertarget{ecldoc:file.findsuperfilesubname}{}
\hspace{0pt} \hyperlink{ecldoc:File}{File} \textbackslash 

{\renewcommand{\arraystretch}{1.5}
\begin{tabularx}{\textwidth}{|>{\raggedright\arraybackslash}l|X|}
\hline
\hspace{0pt}\mytexttt{\color{red} unsigned4} & \textbf{FindSuperFileSubName} \\
\hline
\multicolumn{2}{|>{\raggedright\arraybackslash}X|}{\hspace{0pt}\mytexttt{\color{param} (varstring superName, varstring subName)}} \\
\hline
\end{tabularx}
}

\par





Returns the position of a file within a superfile.






\par
\begin{description}
\item [\colorbox{tagtype}{\color{white} \textbf{\textsf{PARAMETER}}}] \textbf{\underline{superName}} ||| VARSTRING --- The logical name of the superfile.
\item [\colorbox{tagtype}{\color{white} \textbf{\textsf{PARAMETER}}}] \textbf{\underline{subName}} ||| VARSTRING --- The logical name of the sub-file.
\end{description}







\par
\begin{description}
\item [\colorbox{tagtype}{\color{white} \textbf{\textsf{RETURN}}}] \textbf{UNSIGNED4} --- The 1-based position of the sub-file within the superfile.
\end{description}




\rule{\linewidth}{0.5pt}
\subsection*{\textsf{\colorbox{headtoc}{\color{white} FUNCTION}
StartSuperFileTransaction}}

\hypertarget{ecldoc:file.startsuperfiletransaction}{}
\hspace{0pt} \hyperlink{ecldoc:File}{File} \textbackslash 

{\renewcommand{\arraystretch}{1.5}
\begin{tabularx}{\textwidth}{|>{\raggedright\arraybackslash}l|X|}
\hline
\hspace{0pt}\mytexttt{\color{red} } & \textbf{StartSuperFileTransaction} \\
\hline
\multicolumn{2}{|>{\raggedright\arraybackslash}X|}{\hspace{0pt}\mytexttt{\color{param} ()}} \\
\hline
\end{tabularx}
}

\par





Starts a superfile transaction. All superfile operations within the transaction will either be executed atomically or rolled back when the transaction is finished.








\par
\begin{description}
\item [\colorbox{tagtype}{\color{white} \textbf{\textsf{RETURN}}}] \textbf{} --- 
\end{description}




\rule{\linewidth}{0.5pt}
\subsection*{\textsf{\colorbox{headtoc}{\color{white} FUNCTION}
AddSuperFile}}

\hypertarget{ecldoc:file.addsuperfile}{}
\hspace{0pt} \hyperlink{ecldoc:File}{File} \textbackslash 

{\renewcommand{\arraystretch}{1.5}
\begin{tabularx}{\textwidth}{|>{\raggedright\arraybackslash}l|X|}
\hline
\hspace{0pt}\mytexttt{\color{red} } & \textbf{AddSuperFile} \\
\hline
\multicolumn{2}{|>{\raggedright\arraybackslash}X|}{\hspace{0pt}\mytexttt{\color{param} (varstring superName, varstring subName, unsigned4 atPos=0, boolean addContents=FALSE, boolean strict=FALSE)}} \\
\hline
\end{tabularx}
}

\par





Adds a file to a superfile.






\par
\begin{description}
\item [\colorbox{tagtype}{\color{white} \textbf{\textsf{PARAMETER}}}] \textbf{\underline{addContents}} ||| BOOLEAN --- Controls whether adding a superfile adds the superfile, or its contents. Defaults to FALSE (do not expand).
\item [\colorbox{tagtype}{\color{white} \textbf{\textsf{PARAMETER}}}] \textbf{\underline{strict}} ||| BOOLEAN --- Check addContents only if subName is a superfile, and ensure superfiles exist.
\item [\colorbox{tagtype}{\color{white} \textbf{\textsf{PARAMETER}}}] \textbf{\underline{superName}} ||| VARSTRING --- The logical name of the superfile.
\item [\colorbox{tagtype}{\color{white} \textbf{\textsf{PARAMETER}}}] \textbf{\underline{atPos}} ||| UNSIGNED4 --- The position to add the sub-file, or 0 to append. Defaults to 0.
\item [\colorbox{tagtype}{\color{white} \textbf{\textsf{PARAMETER}}}] \textbf{\underline{subName}} ||| VARSTRING --- The name of the logical file to add.
\end{description}







\par
\begin{description}
\item [\colorbox{tagtype}{\color{white} \textbf{\textsf{RETURN}}}] \textbf{} --- 
\end{description}




\rule{\linewidth}{0.5pt}
\subsection*{\textsf{\colorbox{headtoc}{\color{white} FUNCTION}
RemoveSuperFile}}

\hypertarget{ecldoc:file.removesuperfile}{}
\hspace{0pt} \hyperlink{ecldoc:File}{File} \textbackslash 

{\renewcommand{\arraystretch}{1.5}
\begin{tabularx}{\textwidth}{|>{\raggedright\arraybackslash}l|X|}
\hline
\hspace{0pt}\mytexttt{\color{red} } & \textbf{RemoveSuperFile} \\
\hline
\multicolumn{2}{|>{\raggedright\arraybackslash}X|}{\hspace{0pt}\mytexttt{\color{param} (varstring superName, varstring subName, boolean del=FALSE, boolean removeContents=FALSE)}} \\
\hline
\end{tabularx}
}

\par





Removes a sub-file from a superfile.






\par
\begin{description}
\item [\colorbox{tagtype}{\color{white} \textbf{\textsf{PARAMETER}}}] \textbf{\underline{del}} ||| BOOLEAN --- Indicates whether the sub-file should also be removed from the disk. Defaults to FALSE.
\item [\colorbox{tagtype}{\color{white} \textbf{\textsf{PARAMETER}}}] \textbf{\underline{superName}} ||| VARSTRING --- The logical name of the superfile.
\item [\colorbox{tagtype}{\color{white} \textbf{\textsf{PARAMETER}}}] \textbf{\underline{removeContents}} ||| BOOLEAN --- Controls whether the contents of a sub-file which is a superfile should be recursively removed. Defaults to FALSE.
\item [\colorbox{tagtype}{\color{white} \textbf{\textsf{PARAMETER}}}] \textbf{\underline{subName}} ||| VARSTRING --- The name of the sub-file to remove.
\end{description}







\par
\begin{description}
\item [\colorbox{tagtype}{\color{white} \textbf{\textsf{RETURN}}}] \textbf{} --- 
\end{description}




\rule{\linewidth}{0.5pt}
\subsection*{\textsf{\colorbox{headtoc}{\color{white} FUNCTION}
ClearSuperFile}}

\hypertarget{ecldoc:file.clearsuperfile}{}
\hspace{0pt} \hyperlink{ecldoc:File}{File} \textbackslash 

{\renewcommand{\arraystretch}{1.5}
\begin{tabularx}{\textwidth}{|>{\raggedright\arraybackslash}l|X|}
\hline
\hspace{0pt}\mytexttt{\color{red} } & \textbf{ClearSuperFile} \\
\hline
\multicolumn{2}{|>{\raggedright\arraybackslash}X|}{\hspace{0pt}\mytexttt{\color{param} (varstring superName, boolean del=FALSE)}} \\
\hline
\end{tabularx}
}

\par





Removes all sub-files from a superfile.






\par
\begin{description}
\item [\colorbox{tagtype}{\color{white} \textbf{\textsf{PARAMETER}}}] \textbf{\underline{del}} ||| BOOLEAN --- Indicates whether the sub-files should also be removed from the disk. Defaults to FALSE.
\item [\colorbox{tagtype}{\color{white} \textbf{\textsf{PARAMETER}}}] \textbf{\underline{superName}} ||| VARSTRING --- The logical name of the superfile.
\end{description}







\par
\begin{description}
\item [\colorbox{tagtype}{\color{white} \textbf{\textsf{RETURN}}}] \textbf{} --- 
\end{description}




\rule{\linewidth}{0.5pt}
\subsection*{\textsf{\colorbox{headtoc}{\color{white} FUNCTION}
RemoveOwnedSubFiles}}

\hypertarget{ecldoc:file.removeownedsubfiles}{}
\hspace{0pt} \hyperlink{ecldoc:File}{File} \textbackslash 

{\renewcommand{\arraystretch}{1.5}
\begin{tabularx}{\textwidth}{|>{\raggedright\arraybackslash}l|X|}
\hline
\hspace{0pt}\mytexttt{\color{red} } & \textbf{RemoveOwnedSubFiles} \\
\hline
\multicolumn{2}{|>{\raggedright\arraybackslash}X|}{\hspace{0pt}\mytexttt{\color{param} (varstring superName, boolean del=FALSE)}} \\
\hline
\end{tabularx}
}

\par





Removes all soley-owned sub-files from a superfile. If a sub-file is also contained within another superfile then it is retained.






\par
\begin{description}
\item [\colorbox{tagtype}{\color{white} \textbf{\textsf{PARAMETER}}}] \textbf{\underline{superName}} ||| VARSTRING --- The logical name of the superfile.
\item [\colorbox{tagtype}{\color{white} \textbf{\textsf{PARAMETER}}}] \textbf{\underline{del}} ||| BOOLEAN --- No Doc
\end{description}







\par
\begin{description}
\item [\colorbox{tagtype}{\color{white} \textbf{\textsf{RETURN}}}] \textbf{} --- 
\end{description}




\rule{\linewidth}{0.5pt}
\subsection*{\textsf{\colorbox{headtoc}{\color{white} FUNCTION}
DeleteOwnedSubFiles}}

\hypertarget{ecldoc:file.deleteownedsubfiles}{}
\hspace{0pt} \hyperlink{ecldoc:File}{File} \textbackslash 

{\renewcommand{\arraystretch}{1.5}
\begin{tabularx}{\textwidth}{|>{\raggedright\arraybackslash}l|X|}
\hline
\hspace{0pt}\mytexttt{\color{red} } & \textbf{DeleteOwnedSubFiles} \\
\hline
\multicolumn{2}{|>{\raggedright\arraybackslash}X|}{\hspace{0pt}\mytexttt{\color{param} (varstring superName)}} \\
\hline
\end{tabularx}
}

\par





Legacy version of RemoveOwnedSubFiles which was incorrectly named in a previous version.






\par
\begin{description}
\item [\colorbox{tagtype}{\color{white} \textbf{\textsf{PARAMETER}}}] \textbf{\underline{supername}} ||| VARSTRING --- No Doc
\end{description}







\par
\begin{description}
\item [\colorbox{tagtype}{\color{white} \textbf{\textsf{RETURN}}}] \textbf{} --- 
\end{description}






\par
\begin{description}
\item [\colorbox{tagtype}{\color{white} \textbf{\textsf{SEE}}}] RemoveOwnedSubFIles
\end{description}




\rule{\linewidth}{0.5pt}
\subsection*{\textsf{\colorbox{headtoc}{\color{white} FUNCTION}
SwapSuperFile}}

\hypertarget{ecldoc:file.swapsuperfile}{}
\hspace{0pt} \hyperlink{ecldoc:File}{File} \textbackslash 

{\renewcommand{\arraystretch}{1.5}
\begin{tabularx}{\textwidth}{|>{\raggedright\arraybackslash}l|X|}
\hline
\hspace{0pt}\mytexttt{\color{red} } & \textbf{SwapSuperFile} \\
\hline
\multicolumn{2}{|>{\raggedright\arraybackslash}X|}{\hspace{0pt}\mytexttt{\color{param} (varstring superName1, varstring superName2)}} \\
\hline
\end{tabularx}
}

\par





Swap the contents of two superfiles.






\par
\begin{description}
\item [\colorbox{tagtype}{\color{white} \textbf{\textsf{PARAMETER}}}] \textbf{\underline{superName1}} ||| VARSTRING --- The logical name of the first superfile.
\item [\colorbox{tagtype}{\color{white} \textbf{\textsf{PARAMETER}}}] \textbf{\underline{superName2}} ||| VARSTRING --- The logical name of the second superfile.
\end{description}







\par
\begin{description}
\item [\colorbox{tagtype}{\color{white} \textbf{\textsf{RETURN}}}] \textbf{} --- 
\end{description}




\rule{\linewidth}{0.5pt}
\subsection*{\textsf{\colorbox{headtoc}{\color{white} FUNCTION}
ReplaceSuperFile}}

\hypertarget{ecldoc:file.replacesuperfile}{}
\hspace{0pt} \hyperlink{ecldoc:File}{File} \textbackslash 

{\renewcommand{\arraystretch}{1.5}
\begin{tabularx}{\textwidth}{|>{\raggedright\arraybackslash}l|X|}
\hline
\hspace{0pt}\mytexttt{\color{red} } & \textbf{ReplaceSuperFile} \\
\hline
\multicolumn{2}{|>{\raggedright\arraybackslash}X|}{\hspace{0pt}\mytexttt{\color{param} (varstring superName, varstring oldSubFile, varstring newSubFile)}} \\
\hline
\end{tabularx}
}

\par





Removes a sub-file from a superfile and replaces it with another.






\par
\begin{description}
\item [\colorbox{tagtype}{\color{white} \textbf{\textsf{PARAMETER}}}] \textbf{\underline{superName}} ||| VARSTRING --- The logical name of the superfile.
\item [\colorbox{tagtype}{\color{white} \textbf{\textsf{PARAMETER}}}] \textbf{\underline{oldSubFile}} ||| VARSTRING --- The logical name of the sub-file to remove.
\item [\colorbox{tagtype}{\color{white} \textbf{\textsf{PARAMETER}}}] \textbf{\underline{newSubFile}} ||| VARSTRING --- The logical name of the sub-file to replace within the superfile.
\end{description}







\par
\begin{description}
\item [\colorbox{tagtype}{\color{white} \textbf{\textsf{RETURN}}}] \textbf{} --- 
\end{description}




\rule{\linewidth}{0.5pt}
\subsection*{\textsf{\colorbox{headtoc}{\color{white} FUNCTION}
FinishSuperFileTransaction}}

\hypertarget{ecldoc:file.finishsuperfiletransaction}{}
\hspace{0pt} \hyperlink{ecldoc:File}{File} \textbackslash 

{\renewcommand{\arraystretch}{1.5}
\begin{tabularx}{\textwidth}{|>{\raggedright\arraybackslash}l|X|}
\hline
\hspace{0pt}\mytexttt{\color{red} } & \textbf{FinishSuperFileTransaction} \\
\hline
\multicolumn{2}{|>{\raggedright\arraybackslash}X|}{\hspace{0pt}\mytexttt{\color{param} (boolean rollback=FALSE)}} \\
\hline
\end{tabularx}
}

\par





Finishes a superfile transaction. This executes all the operations since the matching StartSuperFileTransaction(). If there are any errors, then all of the operations are rolled back.






\par
\begin{description}
\item [\colorbox{tagtype}{\color{white} \textbf{\textsf{PARAMETER}}}] \textbf{\underline{rollback}} ||| BOOLEAN --- No Doc
\end{description}







\par
\begin{description}
\item [\colorbox{tagtype}{\color{white} \textbf{\textsf{RETURN}}}] \textbf{} --- 
\end{description}




\rule{\linewidth}{0.5pt}
\subsection*{\textsf{\colorbox{headtoc}{\color{white} FUNCTION}
SuperFileContents}}

\hypertarget{ecldoc:file.superfilecontents}{}
\hspace{0pt} \hyperlink{ecldoc:File}{File} \textbackslash 

{\renewcommand{\arraystretch}{1.5}
\begin{tabularx}{\textwidth}{|>{\raggedright\arraybackslash}l|X|}
\hline
\hspace{0pt}\mytexttt{\color{red} dataset(FsLogicalFileNameRecord)} & \textbf{SuperFileContents} \\
\hline
\multicolumn{2}{|>{\raggedright\arraybackslash}X|}{\hspace{0pt}\mytexttt{\color{param} (varstring superName, boolean recurse=FALSE)}} \\
\hline
\end{tabularx}
}

\par





Returns the list of sub-files contained within a superfile.






\par
\begin{description}
\item [\colorbox{tagtype}{\color{white} \textbf{\textsf{PARAMETER}}}] \textbf{\underline{recurse}} ||| BOOLEAN --- Should the contents of child-superfiles be expanded. Default is FALSE.
\item [\colorbox{tagtype}{\color{white} \textbf{\textsf{PARAMETER}}}] \textbf{\underline{superName}} ||| VARSTRING --- The logical name of the superfile.
\end{description}







\par
\begin{description}
\item [\colorbox{tagtype}{\color{white} \textbf{\textsf{RETURN}}}] \textbf{TABLE ( FsLogicalFileNameRecord )} --- A dataset containing the names of the sub-files.
\end{description}




\rule{\linewidth}{0.5pt}
\subsection*{\textsf{\colorbox{headtoc}{\color{white} FUNCTION}
LogicalFileSuperOwners}}

\hypertarget{ecldoc:file.logicalfilesuperowners}{}
\hspace{0pt} \hyperlink{ecldoc:File}{File} \textbackslash 

{\renewcommand{\arraystretch}{1.5}
\begin{tabularx}{\textwidth}{|>{\raggedright\arraybackslash}l|X|}
\hline
\hspace{0pt}\mytexttt{\color{red} dataset(FsLogicalFileNameRecord)} & \textbf{LogicalFileSuperOwners} \\
\hline
\multicolumn{2}{|>{\raggedright\arraybackslash}X|}{\hspace{0pt}\mytexttt{\color{param} (varstring name)}} \\
\hline
\end{tabularx}
}

\par





Returns the list of superfiles that a logical file is contained within.






\par
\begin{description}
\item [\colorbox{tagtype}{\color{white} \textbf{\textsf{PARAMETER}}}] \textbf{\underline{name}} ||| VARSTRING --- The name of the logical file.
\end{description}







\par
\begin{description}
\item [\colorbox{tagtype}{\color{white} \textbf{\textsf{RETURN}}}] \textbf{TABLE ( FsLogicalFileNameRecord )} --- A dataset containing the names of the superfiles.
\end{description}




\rule{\linewidth}{0.5pt}
\subsection*{\textsf{\colorbox{headtoc}{\color{white} FUNCTION}
LogicalFileSuperSubList}}

\hypertarget{ecldoc:file.logicalfilesupersublist}{}
\hspace{0pt} \hyperlink{ecldoc:File}{File} \textbackslash 

{\renewcommand{\arraystretch}{1.5}
\begin{tabularx}{\textwidth}{|>{\raggedright\arraybackslash}l|X|}
\hline
\hspace{0pt}\mytexttt{\color{red} dataset(FsLogicalSuperSubRecord)} & \textbf{LogicalFileSuperSubList} \\
\hline
\multicolumn{2}{|>{\raggedright\arraybackslash}X|}{\hspace{0pt}\mytexttt{\color{param} ()}} \\
\hline
\end{tabularx}
}

\par





Returns the list of all the superfiles in the system and their component sub-files.








\par
\begin{description}
\item [\colorbox{tagtype}{\color{white} \textbf{\textsf{RETURN}}}] \textbf{TABLE ( FsLogicalSuperSubRecord )} --- A dataset containing pairs of superName,subName for each component file.
\end{description}




\rule{\linewidth}{0.5pt}
\subsection*{\textsf{\colorbox{headtoc}{\color{white} FUNCTION}
fPromoteSuperFileList}}

\hypertarget{ecldoc:file.fpromotesuperfilelist}{}
\hspace{0pt} \hyperlink{ecldoc:File}{File} \textbackslash 

{\renewcommand{\arraystretch}{1.5}
\begin{tabularx}{\textwidth}{|>{\raggedright\arraybackslash}l|X|}
\hline
\hspace{0pt}\mytexttt{\color{red} varstring} & \textbf{fPromoteSuperFileList} \\
\hline
\multicolumn{2}{|>{\raggedright\arraybackslash}X|}{\hspace{0pt}\mytexttt{\color{param} (set of varstring superNames, varstring addHead='', boolean delTail=FALSE, boolean createOnlyOne=FALSE, boolean reverse=FALSE)}} \\
\hline
\end{tabularx}
}

\par





Moves the sub-files from the first entry in the list of superfiles to the next in the list, repeating the process through the list of superfiles.






\par
\begin{description}
\item [\colorbox{tagtype}{\color{white} \textbf{\textsf{PARAMETER}}}] \textbf{\underline{reverse}} ||| BOOLEAN --- Reverse the order of processing the superfiles list, effectively 'demoting' instead of 'promoting' the sub-files. The default is FALSE.
\item [\colorbox{tagtype}{\color{white} \textbf{\textsf{PARAMETER}}}] \textbf{\underline{createOnlyOne}} ||| BOOLEAN --- Specifies whether to only create a single superfile (truncate the list at the first non-existent superfile). The default is FALSE.
\item [\colorbox{tagtype}{\color{white} \textbf{\textsf{PARAMETER}}}] \textbf{\underline{addHead}} ||| VARSTRING --- A string containing a comma-delimited list of logical file names to add to the first superfile after the promotion process is complete. Defaults to ''.
\item [\colorbox{tagtype}{\color{white} \textbf{\textsf{PARAMETER}}}] \textbf{\underline{superNames}} ||| SET ( VARSTRING ) --- A set of the names of the superfiles to act on. Any that do not exist will be created. The contents of each superfile will be moved to the next in the list.
\item [\colorbox{tagtype}{\color{white} \textbf{\textsf{PARAMETER}}}] \textbf{\underline{delTail}} ||| BOOLEAN --- Indicates whether to physically delete the contents moved out of the last superfile. The default is FALSE.
\end{description}







\par
\begin{description}
\item [\colorbox{tagtype}{\color{white} \textbf{\textsf{RETURN}}}] \textbf{VARSTRING} --- A string containing a comma separated list of the previous sub-file contents of the emptied superfile.
\end{description}




\rule{\linewidth}{0.5pt}
\subsection*{\textsf{\colorbox{headtoc}{\color{white} FUNCTION}
PromoteSuperFileList}}

\hypertarget{ecldoc:file.promotesuperfilelist}{}
\hspace{0pt} \hyperlink{ecldoc:File}{File} \textbackslash 

{\renewcommand{\arraystretch}{1.5}
\begin{tabularx}{\textwidth}{|>{\raggedright\arraybackslash}l|X|}
\hline
\hspace{0pt}\mytexttt{\color{red} } & \textbf{PromoteSuperFileList} \\
\hline
\multicolumn{2}{|>{\raggedright\arraybackslash}X|}{\hspace{0pt}\mytexttt{\color{param} (set of varstring superNames, varstring addHead='', boolean delTail=FALSE, boolean createOnlyOne=FALSE, boolean reverse=FALSE)}} \\
\hline
\end{tabularx}
}

\par





Same as fPromoteSuperFileList, but does not return the DFU Workunit ID.






\par
\begin{description}
\item [\colorbox{tagtype}{\color{white} \textbf{\textsf{PARAMETER}}}] \textbf{\underline{addhead}} ||| VARSTRING --- No Doc
\item [\colorbox{tagtype}{\color{white} \textbf{\textsf{PARAMETER}}}] \textbf{\underline{createonlyone}} ||| BOOLEAN --- No Doc
\item [\colorbox{tagtype}{\color{white} \textbf{\textsf{PARAMETER}}}] \textbf{\underline{supernames}} ||| SET ( VARSTRING ) --- No Doc
\item [\colorbox{tagtype}{\color{white} \textbf{\textsf{PARAMETER}}}] \textbf{\underline{reverse}} ||| BOOLEAN --- No Doc
\item [\colorbox{tagtype}{\color{white} \textbf{\textsf{PARAMETER}}}] \textbf{\underline{deltail}} ||| BOOLEAN --- No Doc
\end{description}







\par
\begin{description}
\item [\colorbox{tagtype}{\color{white} \textbf{\textsf{RETURN}}}] \textbf{} --- 
\end{description}






\par
\begin{description}
\item [\colorbox{tagtype}{\color{white} \textbf{\textsf{SEE}}}] fPromoteSuperFileList
\end{description}




\rule{\linewidth}{0.5pt}


