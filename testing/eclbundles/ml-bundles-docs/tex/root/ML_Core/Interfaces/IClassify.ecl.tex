\chapter*{\color{headfile}
{\large ML\_Core\slash\hspace{0pt}}
{\large Interfaces\slash\hspace{0pt}}
 \\
IClassify
}
\hypertarget{ecldoc:toc:ML_Core.Interfaces.IClassify}{}
\hyperlink{ecldoc:toc:root/ML_Core/Interfaces}{Go Up}

\section*{\underline{\textsf{IMPORTS}}}
\begin{doublespace}
{\large
ML\_Core |
ML\_Core.Types |
}
\end{doublespace}

\section*{\underline{\textsf{DESCRIPTIONS}}}
\subsection*{\textsf{\colorbox{headtoc}{\color{white} MODULE}
IClassify}}

\hypertarget{ecldoc:ML_Core.Interfaces.IClassify}{}

{\renewcommand{\arraystretch}{1.5}
\begin{tabularx}{\textwidth}{|>{\raggedright\arraybackslash}l|X|}
\hline
\hspace{0pt}\mytexttt{\color{red} } & \textbf{IClassify} \\
\hline
\end{tabularx}
}

\par





Interface definition for Classification. Actual implementation modules will probably take parameters.







\textbf{Children}
\begin{enumerate}
\item \hyperlink{ecldoc:ml_core.interfaces.iclassify.getmodel}{GetModel}
: Calculate the model to fit the observation data to the observed classes
\item \hyperlink{ecldoc:ml_core.interfaces.iclassify.classify}{Classify}
: Classify the observations using a model
\item \hyperlink{ecldoc:ml_core.interfaces.iclassify.report}{Report}
: Report the confusion matrix for the classifier and training data
\end{enumerate}

\rule{\linewidth}{0.5pt}

\subsection*{\textsf{\colorbox{headtoc}{\color{white} FUNCTION}
GetModel}}

\hypertarget{ecldoc:ml_core.interfaces.iclassify.getmodel}{}
\hspace{0pt} \hyperlink{ecldoc:ML_Core.Interfaces.IClassify}{IClassify} \textbackslash 

{\renewcommand{\arraystretch}{1.5}
\begin{tabularx}{\textwidth}{|>{\raggedright\arraybackslash}l|X|}
\hline
\hspace{0pt}\mytexttt{\color{red} DATASET(Types.Layout\_Model)} & \textbf{GetModel} \\
\hline
\multicolumn{2}{|>{\raggedright\arraybackslash}X|}{\hspace{0pt}\mytexttt{\color{param} (DATASET(Types.NumericField) observations, DATASET(Types.DiscreteField) classifications)}} \\
\hline
\end{tabularx}
}

\par





Calculate the model to fit the observation data to the observed classes.






\par
\begin{description}
\item [\colorbox{tagtype}{\color{white} \textbf{\textsf{PARAMETER}}}] \textbf{\underline{classifications}} ||| TABLE ( DiscreteField ) --- the observed classification used to build the model
\item [\colorbox{tagtype}{\color{white} \textbf{\textsf{PARAMETER}}}] \textbf{\underline{observations}} ||| TABLE ( NumericField ) --- the observed explanatory values
\end{description}







\par
\begin{description}
\item [\colorbox{tagtype}{\color{white} \textbf{\textsf{RETURN}}}] \textbf{TABLE ( \{ UNSIGNED2 wi , UNSIGNED8 id , UNSIGNED4 number , REAL8 value \} )} --- the encoded model
\end{description}




\rule{\linewidth}{0.5pt}
\subsection*{\textsf{\colorbox{headtoc}{\color{white} FUNCTION}
Classify}}

\hypertarget{ecldoc:ml_core.interfaces.iclassify.classify}{}
\hspace{0pt} \hyperlink{ecldoc:ML_Core.Interfaces.IClassify}{IClassify} \textbackslash 

{\renewcommand{\arraystretch}{1.5}
\begin{tabularx}{\textwidth}{|>{\raggedright\arraybackslash}l|X|}
\hline
\hspace{0pt}\mytexttt{\color{red} DATASET(Types.Classify\_Result)} & \textbf{Classify} \\
\hline
\multicolumn{2}{|>{\raggedright\arraybackslash}X|}{\hspace{0pt}\mytexttt{\color{param} (DATASET(Types.Layout\_Model) model, DATASET(Types.NumericField) new\_observations)}} \\
\hline
\end{tabularx}
}

\par





Classify the observations using a model.






\par
\begin{description}
\item [\colorbox{tagtype}{\color{white} \textbf{\textsf{PARAMETER}}}] \textbf{\underline{new\_observations}} ||| TABLE ( NumericField ) --- observations to be classified
\item [\colorbox{tagtype}{\color{white} \textbf{\textsf{PARAMETER}}}] \textbf{\underline{model}} ||| TABLE ( Layout\_Model ) --- The model, which must be produced by a corresponding getModel function.
\end{description}







\par
\begin{description}
\item [\colorbox{tagtype}{\color{white} \textbf{\textsf{RETURN}}}] \textbf{TABLE ( \{ UNSIGNED2 wi , UNSIGNED8 id , UNSIGNED4 number , INTEGER4 value , REAL8 conf \} )} --- Classification with a confidence value
\end{description}




\rule{\linewidth}{0.5pt}
\subsection*{\textsf{\colorbox{headtoc}{\color{white} FUNCTION}
Report}}

\hypertarget{ecldoc:ml_core.interfaces.iclassify.report}{}
\hspace{0pt} \hyperlink{ecldoc:ML_Core.Interfaces.IClassify}{IClassify} \textbackslash 

{\renewcommand{\arraystretch}{1.5}
\begin{tabularx}{\textwidth}{|>{\raggedright\arraybackslash}l|X|}
\hline
\hspace{0pt}\mytexttt{\color{red} DATASET(Types.Confusion\_Detail)} & \textbf{Report} \\
\hline
\multicolumn{2}{|>{\raggedright\arraybackslash}X|}{\hspace{0pt}\mytexttt{\color{param} (DATASET(Types.Layout\_Model) model, DATASET(Types.NumericField) observations, DATASET(Types.DiscreteField) classifications)}} \\
\hline
\end{tabularx}
}

\par





Report the confusion matrix for the classifier and training data.






\par
\begin{description}
\item [\colorbox{tagtype}{\color{white} \textbf{\textsf{PARAMETER}}}] \textbf{\underline{classifications}} ||| TABLE ( DiscreteField ) --- the classifications associated with the observations
\item [\colorbox{tagtype}{\color{white} \textbf{\textsf{PARAMETER}}}] \textbf{\underline{observations}} ||| TABLE ( NumericField ) --- the explanatory values.
\item [\colorbox{tagtype}{\color{white} \textbf{\textsf{PARAMETER}}}] \textbf{\underline{model}} ||| TABLE ( Layout\_Model ) --- the encoded model
\end{description}







\par
\begin{description}
\item [\colorbox{tagtype}{\color{white} \textbf{\textsf{RETURN}}}] \textbf{TABLE ( \{ UNSIGNED2 wi , UNSIGNED4 classifier , INTEGER4 actual\_class , INTEGER4 predict\_class , UNSIGNED4 occurs , BOOLEAN correct \} )} --- the confusion matrix showing correct and incorrect results
\end{description}




\rule{\linewidth}{0.5pt}


