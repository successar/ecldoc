\chapter*{\color{headfile}
{\large ML\_Core\slash\hspace{0pt}}
{\large Math\slash\hspace{0pt}}
 \\
Poly
}
\hypertarget{ecldoc:toc:ML_Core.Math.Poly}{}
\hyperlink{ecldoc:toc:root/ML_Core/Math}{Go Up}


\section*{\underline{\textsf{DESCRIPTIONS}}}
\subsection*{\textsf{\colorbox{headtoc}{\color{white} EMBED}
Poly}}

\hypertarget{ecldoc:ml_core.math.poly}{}

{\renewcommand{\arraystretch}{1.5}
\begin{tabularx}{\textwidth}{|>{\raggedright\arraybackslash}l|X|}
\hline
\hspace{0pt}\mytexttt{\color{red} REAL8} & \textbf{Poly} \\
\hline
\multicolumn{2}{|>{\raggedright\arraybackslash}X|}{\hspace{0pt}\mytexttt{\color{param} (REAL8 x, SET OF REAL8 Coeffs)}} \\
\hline
\end{tabularx}
}

\par





Evaluate a polynomial from a set of co-effs. Co-effs 1 is assumed to be the HIGH order of the equation. Thus for ax\^{}2+bx+c - the set would need to be Coef := [a,b,c];






\par
\begin{description}
\item [\colorbox{tagtype}{\color{white} \textbf{\textsf{PARAMETER}}}] \textbf{\underline{x}} ||| REAL8 --- the value of x in the polynomial
\item [\colorbox{tagtype}{\color{white} \textbf{\textsf{PARAMETER}}}] \textbf{\underline{Coeffs}} ||| SET ( REAL8 ) --- a set of coefficients forthe polynomial. The ALL set is considered to be all zero values
\end{description}







\par
\begin{description}
\item [\colorbox{tagtype}{\color{white} \textbf{\textsf{RETURN}}}] \textbf{REAL8} --- value of the polynomial at x
\end{description}




\rule{\linewidth}{0.5pt}
