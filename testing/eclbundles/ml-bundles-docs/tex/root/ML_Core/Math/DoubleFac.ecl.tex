\chapter*{\color{headfile}
{\large ML\_Core\slash\hspace{0pt}}
{\large Math\slash\hspace{0pt}}
 \\
DoubleFac
}
\hypertarget{ecldoc:toc:ML_Core.Math.DoubleFac}{}
\hyperlink{ecldoc:toc:root/ML_Core/Math}{Go Up}


\section*{\underline{\textsf{DESCRIPTIONS}}}
\subsection*{\textsf{\colorbox{headtoc}{\color{white} EMBED}
DoubleFac}}

\hypertarget{ecldoc:ml_core.math.doublefac}{}

{\renewcommand{\arraystretch}{1.5}
\begin{tabularx}{\textwidth}{|>{\raggedright\arraybackslash}l|X|}
\hline
\hspace{0pt}\mytexttt{\color{red} REAL8} & \textbf{DoubleFac} \\
\hline
\multicolumn{2}{|>{\raggedright\arraybackslash}X|}{\hspace{0pt}\mytexttt{\color{param} (INTEGER2 i)}} \\
\hline
\end{tabularx}
}

\par





The 'double' factorial is defined for ODD n and is the product of all the odd numbers up to and including that number. We are extending the meaning to even numbers to mean the product of the even numbers up to and including that number. Thus DoubleFac(8) = 8*6*4*2 We also defend against i < 2 (returning 1.0)






\par
\begin{description}
\item [\colorbox{tagtype}{\color{white} \textbf{\textsf{PARAMETER}}}] \textbf{\underline{i}} ||| INTEGER2 --- the value used in the calculation
\end{description}







\par
\begin{description}
\item [\colorbox{tagtype}{\color{white} \textbf{\textsf{RETURN}}}] \textbf{REAL8} --- the factorial of the sequence, declining by 2
\end{description}




\rule{\linewidth}{0.5pt}
