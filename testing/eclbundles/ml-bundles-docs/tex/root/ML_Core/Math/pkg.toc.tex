\chapter*{\color{headtoc} Math}
\hypertarget{ecldoc:toc:root/ML_Core/Math}{}
\hyperlink{ecldoc:toc:root/ML_Core}{Go Up}


\section*{Table of Contents}
{\renewcommand{\arraystretch}{1.5}
\begin{longtable}{|p{\textwidth}|}
\hline
\hyperlink{ecldoc:toc:ML_Core.Math.Beta}{Beta.ecl} \\
Return the beta value of two positive real numbers, x and y \\
\hline
\hyperlink{ecldoc:toc:ML_Core.Math.Distributions}{Distributions.ecl} \\
\hline
\hyperlink{ecldoc:toc:ML_Core.Math.DoubleFac}{DoubleFac.ecl} \\
The 'double' factorial is defined for ODD n and is the product of all the odd numbers up to and including that number \\
\hline
\hyperlink{ecldoc:toc:ML_Core.Math.Fac}{Fac.ecl} \\
Factorial function \\
\hline
\hyperlink{ecldoc:toc:ML_Core.Math.gamma}{gamma.ecl} \\
Return the value of gamma function of real number x A wrapper for the standard C tgamma function \\
\hline
\hyperlink{ecldoc:toc:ML_Core.Math.log_gamma}{log\_gamma.ecl} \\
Return the value of the log gamma function of the absolute value of X \\
\hline
\hyperlink{ecldoc:toc:ML_Core.Math.lowerGamma}{lowerGamma.ecl} \\
Return the lower incomplete gamma value of two real numbers, \\
\hline
\hyperlink{ecldoc:toc:ML_Core.Math.NCK}{NCK.ecl} \\
\hline
\hyperlink{ecldoc:toc:ML_Core.Math.Poly}{Poly.ecl} \\
Evaluate a polynomial from a set of co-effs \\
\hline
\hyperlink{ecldoc:toc:ML_Core.Math.StirlingFormula}{StirlingFormula.ecl} \\
Stirling's formula \\
\hline
\hyperlink{ecldoc:toc:ML_Core.Math.upperGamma}{upperGamma.ecl} \\
Return the upper incomplete gamma value of two real numbers, x and y \\
\hline
\end{longtable}
}

\chapter*{ML\_Core.Math.Beta}
\hypertarget{ecldoc:toc:ML_Core.Math.Beta}{}

\section*{\underline{IMPORTS}}
\begin{itemize}
\item ML\_Core.Math
\end{itemize}

\section*{\underline{DESCRIPTIONS}}
\subsection*{FUNCTION : Beta}
\hypertarget{ecldoc:ml_core.math.beta}{}
\begin{minipage}[t]{\textwidth}
\begin{flushleft}
 Beta (REAL8 x, REAL8 y)
\end{flushleft}
\end{minipage}
\hyperlink{ecldoc:toc:ML_Core/Math}{Up}

\par
Return the beta value of two positive real numbers, x and y
\par
\textbf{Parameter} : x ||| the value of the first number \\
\textbf{Parameter} : y ||| the value of the second number \\
\textbf{Return} : the beta value \\

\chapter*{ML\_Core.Math.Distributions}
\hypertarget{ML_Core.Math.Distributions}{}

\section*{\underline{IMPORTS}}
\begin{itemize}
\item ML\_Core.Constants
\item ML\_Core.Math
\end{itemize}

\section*{\underline{DESCRIPTIONS}}
\subsection*{module : Distributions}
\hypertarget{ecldoc:ML_Core.Math.Distributions}{MODULE : Distributions} \\
\hyperlink{ecldoc:}{Up} \\
\par
\begin{enumerate}
\item \hyperlink{ecldoc:ml_core.math.distributions.normal_cdf}{Normal\_CDF}
\item \hyperlink{ecldoc:ml_core.math.distributions.normal_ppf}{Normal\_PPF}
\item \hyperlink{ecldoc:ml_core.math.distributions.t_cdf}{T\_CDF}
\item \hyperlink{ecldoc:ml_core.math.distributions.t_ppf}{T\_PPF}
\item \hyperlink{ecldoc:ml_core.math.distributions.chi2_cdf}{Chi2\_CDF}
\item \hyperlink{ecldoc:ml_core.math.distributions.chi2_ppf}{Chi2\_PPF}
\end{enumerate}
\subsection*{function : Normal\_CDF}
\hypertarget{ecldoc:ml_core.math.distributions.normal_cdf}{FUNCTION : REAL8 Normal\_CDF(REAL8 x)} \\
\hyperlink{ecldoc:ML_Core.Math.Distributions}{Up} \\
\par
Cumulative Distribution of the standard normal distribution, the probability that a normal random variable will be smaller than x standard deviations above or below the mean. Taken from C/C++ Mathematical Algorithms for Scientists and Engineers, n. Shammas, McGraw-Hill, 1995 \\
\textbf{Parameter} : x ||| the number of standard deviations \\
\subsection*{function : Normal\_PPF}
\hypertarget{ecldoc:ml_core.math.distributions.normal_ppf}{FUNCTION : REAL8 Normal\_PPF(REAL8 x)} \\
\hyperlink{ecldoc:ML_Core.Math.Distributions}{Up} \\
\par
Normal Distribution Percentage Point Function. Translated from C/C++ Mathematical Algorithms for Scientists and Engineers, N. Shammas, McGraw-Hill, 1995 \\
\textbf{Parameter} : x ||| probability \\
\subsection*{function : T\_CDF}
\hypertarget{ecldoc:ml_core.math.distributions.t_cdf}{FUNCTION : REAL8 T\_CDF(REAL8 x, REAL8 df)} \\
\hyperlink{ecldoc:ML_Core.Math.Distributions}{Up} \\
\par
Students t distribution integral evaluated between negative infinity and x. Translated from NIST SEL DATAPAC Fortran TCDF.f source \\
\textbf{Parameter} : x ||| value of the evaluation \\
\textbf{Parameter} : df ||| degrees of freedom \\
\subsection*{function : T\_PPF}
\hypertarget{ecldoc:ml_core.math.distributions.t_ppf}{FUNCTION : REAL8 T\_PPF(REAL8 x, REAL8 df)} \\
\hyperlink{ecldoc:ML_Core.Math.Distributions}{Up} \\
\par
Percentage point function for the T distribution. Translated from NIST SEL DATAPAC Fortran TPPF.f source \\
\subsection*{function : Chi2\_CDF}
\hypertarget{ecldoc:ml_core.math.distributions.chi2_cdf}{FUNCTION : REAL8 Chi2\_CDF(REAL8 x, REAL8 df)} \\
\hyperlink{ecldoc:ML_Core.Math.Distributions}{Up} \\
\par
The cumulative distribution function for the Chi Square distribution. the CDF for the specfied degrees of freedom. Translated from the NIST SEL DATAPAC Fortran subroutine CHSCDF. \\
\subsection*{function : Chi2\_PPF}
\hypertarget{ecldoc:ml_core.math.distributions.chi2_ppf}{FUNCTION : REAL8 Chi2\_PPF(REAL8 x, REAL8 df)} \\
\hyperlink{ecldoc:ML_Core.Math.Distributions}{Up} \\
\par
The Chi Squared PPF function. Translated from the NIST SEL DATAPAC Fortran subroutine CHSPPF. \\


\chapter*{\color{headfile}
{\large ML\_Core\slash\hspace{0pt}}
{\large Math\slash\hspace{0pt}}
 \\
DoubleFac
}
\hypertarget{ecldoc:toc:ML_Core.Math.DoubleFac}{}
\hyperlink{ecldoc:toc:root/ML_Core/Math}{Go Up}


\section*{\underline{\textsf{DESCRIPTIONS}}}
\subsection*{\textsf{\colorbox{headtoc}{\color{white} EMBED}
DoubleFac}}

\hypertarget{ecldoc:ml_core.math.doublefac}{}

{\renewcommand{\arraystretch}{1.5}
\begin{tabularx}{\textwidth}{|>{\raggedright\arraybackslash}l|X|}
\hline
\hspace{0pt}\mytexttt{\color{red} REAL8} & \textbf{DoubleFac} \\
\hline
\multicolumn{2}{|>{\raggedright\arraybackslash}X|}{\hspace{0pt}\mytexttt{\color{param} (INTEGER2 i)}} \\
\hline
\end{tabularx}
}

\par
The 'double' factorial is defined for ODD n and is the product of all the odd numbers up to and including that number. We are extending the meaning to even numbers to mean the product of the even numbers up to and including that number. Thus DoubleFac(8) = 8*6*4*2 We also defend against i < 2 (returning 1.0)

\par
\begin{description}
\item [\colorbox{tagtype}{\color{white} \textbf{\textsf{PARAMETER}}}] \textbf{\underline{i}} the value used in the calculation
\item [\colorbox{tagtype}{\color{white} \textbf{\textsf{RETURN}}}] \textbf{\underline{}} the factorial of the sequence, declining by 2
\end{description}

\rule{\linewidth}{0.5pt}

\chapter*{ML\_Core.Math.Fac}
\hypertarget{ML_Core.Math.Fac}{}

\section*{\underline{IMPORTS}}

\section*{\underline{DESCRIPTIONS}}
\subsection*{embed : Fac}
\hypertarget{ecldoc:ml_core.math.fac}{EMBED : REAL8 Fac(UNSIGNED2 i)} \\
\hyperlink{ecldoc:}{Up} \\
\par
Factorial function \\
\textbf{Parameter} : i ||| the value used, (i)(i-1)(i-2)\ldots(2) \\
\textbf{Return} : the factorial i! \\

\chapter*{\color{headfile}
{\large ML\_Core\slash\hspace{0pt}}
{\large Math\slash\hspace{0pt}}
 \\
gamma
}
\hypertarget{ecldoc:toc:ML_Core.Math.gamma}{}
\hyperlink{ecldoc:toc:root/ML_Core/Math}{Go Up}


\section*{\underline{\textsf{DESCRIPTIONS}}}
\subsection*{\textsf{\colorbox{headtoc}{\color{white} EMBED}
gamma}}

\hypertarget{ecldoc:ml_core.math.gamma}{}

{\renewcommand{\arraystretch}{1.5}
\begin{tabularx}{\textwidth}{|>{\raggedright\arraybackslash}l|X|}
\hline
\hspace{0pt}\mytexttt{\color{red} REAL8} & \textbf{gamma} \\
\hline
\multicolumn{2}{|>{\raggedright\arraybackslash}X|}{\hspace{0pt}\mytexttt{\color{param} (REAL8 x)}} \\
\hline
\end{tabularx}
}

\par
Return the value of gamma function of real number x A wrapper for the standard C tgamma function.

\par
\begin{description}
\item [\colorbox{tagtype}{\color{white} \textbf{\textsf{PARAMETER}}}] \textbf{\underline{x}} the input x
\item [\colorbox{tagtype}{\color{white} \textbf{\textsf{RETURN}}}] \textbf{\underline{}} the value of GAMMA evaluated at x
\end{description}

\rule{\linewidth}{0.5pt}

\chapter*{\color{headfile}
{\large ML\_Core\slash\hspace{0pt}}
{\large Math\slash\hspace{0pt}}
 \\
log_gamma
}
\hypertarget{ecldoc:toc:ML_Core.Math.log_gamma}{}
\hyperlink{ecldoc:toc:root/ML_Core/Math}{Go Up}


\section*{\underline{\textsf{DESCRIPTIONS}}}
\subsection*{\textsf{\colorbox{headtoc}{\color{white} EMBED}
log\_gamma}}

\hypertarget{ecldoc:ml_core.math.log_gamma}{}

{\renewcommand{\arraystretch}{1.5}
\begin{tabularx}{\textwidth}{|>{\raggedright\arraybackslash}l|X|}
\hline
\hspace{0pt}\mytexttt{\color{red} REAL8} & \textbf{log\_gamma} \\
\hline
\multicolumn{2}{|>{\raggedright\arraybackslash}X|}{\hspace{0pt}\mytexttt{\color{param} (REAL8 x)}} \\
\hline
\end{tabularx}
}

\par





Return the value of the log gamma function of the absolute value of X. A wrapper for the standard C lgamma function. Avoids the race condition found on some platforms by taking the absolute value of the of the input argument.






\par
\begin{description}
\item [\colorbox{tagtype}{\color{white} \textbf{\textsf{PARAMETER}}}] \textbf{\underline{x}} ||| REAL8 --- the input x
\end{description}







\par
\begin{description}
\item [\colorbox{tagtype}{\color{white} \textbf{\textsf{RETURN}}}] \textbf{REAL8} --- the value of the log of the GAMMA evaluated at ABS(x)
\end{description}




\rule{\linewidth}{0.5pt}

\chapter*{\color{headfile}
{\large ML\_Core\slash\hspace{0pt}}
{\large Math\slash\hspace{0pt}}
 \\
lowerGamma
}
\hypertarget{ecldoc:toc:ML_Core.Math.lowerGamma}{}
\hyperlink{ecldoc:toc:root/ML_Core/Math}{Go Up}


\section*{\underline{\textsf{DESCRIPTIONS}}}
\subsection*{\textsf{\colorbox{headtoc}{\color{white} EMBED}
lowerGamma}}

\hypertarget{ecldoc:ml_core.math.lowergamma}{}

{\renewcommand{\arraystretch}{1.5}
\begin{tabularx}{\textwidth}{|>{\raggedright\arraybackslash}l|X|}
\hline
\hspace{0pt}\mytexttt{\color{red} REAL8} & \textbf{lowerGamma} \\
\hline
\multicolumn{2}{|>{\raggedright\arraybackslash}X|}{\hspace{0pt}\mytexttt{\color{param} (REAL8 x, REAL8 y)}} \\
\hline
\end{tabularx}
}

\par





Return the lower incomplete gamma value of two real numbers, x and y






\par
\begin{description}
\item [\colorbox{tagtype}{\color{white} \textbf{\textsf{PARAMETER}}}] \textbf{\underline{x}} ||| REAL8 --- the value of the first number
\item [\colorbox{tagtype}{\color{white} \textbf{\textsf{PARAMETER}}}] \textbf{\underline{y}} ||| REAL8 --- the value of the second number
\end{description}







\par
\begin{description}
\item [\colorbox{tagtype}{\color{white} \textbf{\textsf{RETURN}}}] \textbf{REAL8} --- the lower incomplete gamma value
\end{description}




\rule{\linewidth}{0.5pt}

\chapter*{ML\_Core.Math.NCK}
\hypertarget{ecldoc:toc:ML_Core.Math.NCK}{}

\section*{\underline{IMPORTS}}
\begin{itemize}
\item ML\_Core.Math
\end{itemize}

\section*{\underline{DESCRIPTIONS}}
\subsection*{FUNCTION : NCK}
\hypertarget{ecldoc:ml_core.math.nck}{}

{\renewcommand{\arraystretch}{1.5}
\begin{tabularx}{\textwidth}{|>{\raggedright\arraybackslash}l|X|}
\hline
\hspace{0pt}REAL8 & NCK \\
\hline
\multicolumn{2}{|>{\raggedright\arraybackslash}X|}{\hspace{0pt}(INTEGER2 N, INTEGER2 K)} \\
\hline
\end{tabularx}
}

\hyperlink{ecldoc:toc:ML_Core/Math}{Up}

\par


\rule{\textwidth}{0.4pt}

\chapter*{\color{headfile}
{\large ML\_Core\slash\hspace{0pt}}
{\large Math\slash\hspace{0pt}}
 \\
Poly
}
\hypertarget{ecldoc:toc:ML_Core.Math.Poly}{}
\hyperlink{ecldoc:toc:root/ML_Core/Math}{Go Up}


\section*{\underline{\textsf{DESCRIPTIONS}}}
\subsection*{\textsf{\colorbox{headtoc}{\color{white} EMBED}
Poly}}

\hypertarget{ecldoc:ml_core.math.poly}{}

{\renewcommand{\arraystretch}{1.5}
\begin{tabularx}{\textwidth}{|>{\raggedright\arraybackslash}l|X|}
\hline
\hspace{0pt}\mytexttt{\color{red} REAL8} & \textbf{Poly} \\
\hline
\multicolumn{2}{|>{\raggedright\arraybackslash}X|}{\hspace{0pt}\mytexttt{\color{param} (REAL8 x, SET OF REAL8 Coeffs)}} \\
\hline
\end{tabularx}
}

\par





Evaluate a polynomial from a set of co-effs. Co-effs 1 is assumed to be the HIGH order of the equation. Thus for ax\^{}2+bx+c - the set would need to be Coef := [a,b,c];






\par
\begin{description}
\item [\colorbox{tagtype}{\color{white} \textbf{\textsf{PARAMETER}}}] \textbf{\underline{Coeffs}} ||| SET ( REAL8 ) --- a set of coefficients forthe polynomial. The ALL set is considered to be all zero values
\item [\colorbox{tagtype}{\color{white} \textbf{\textsf{PARAMETER}}}] \textbf{\underline{x}} ||| REAL8 --- the value of x in the polynomial
\end{description}







\par
\begin{description}
\item [\colorbox{tagtype}{\color{white} \textbf{\textsf{RETURN}}}] \textbf{REAL8} --- value of the polynomial at x
\end{description}




\rule{\linewidth}{0.5pt}

\chapter*{ML\_Core.Math.StirlingFormula}
\hypertarget{ecldoc:toc:ML_Core.Math.StirlingFormula}{}

\section*{\underline{IMPORTS}}
\begin{itemize}
\item ML\_Core.Math
\item ML\_Core.Constants
\end{itemize}

\section*{\underline{DESCRIPTIONS}}
\subsection*{FUNCTION : StirlingFormula}
\hypertarget{ecldoc:ml_core.math.stirlingformula}{}

{\renewcommand{\arraystretch}{1.5}
\begin{tabularx}{\textwidth}{|>{\raggedright\arraybackslash}l|X|}
\hline
\hspace{0pt} & StirlingFormula \\
\hline
\multicolumn{2}{|>{\raggedright\arraybackslash}X|}{\hspace{0pt}(REAL x)} \\
\hline
\end{tabularx}
}

\hyperlink{ecldoc:toc:ML_Core/Math}{Up}

\par
Stirling's formula

\par
\begin{description}
\item [\textbf{Parameter}] x ||| the point of evaluation
\item [\textbf{Return}] evaluation result
\end{description}

\rule{\textwidth}{0.4pt}

\chapter*{\color{headfile}
{\large ML\_Core\slash\hspace{0pt}}
{\large Math\slash\hspace{0pt}}
 \\
upperGamma
}
\hypertarget{ecldoc:toc:ML_Core.Math.upperGamma}{}
\hyperlink{ecldoc:toc:root/ML_Core/Math}{Go Up}


\section*{\underline{\textsf{DESCRIPTIONS}}}
\subsection*{\textsf{\colorbox{headtoc}{\color{white} EMBED}
upperGamma}}

\hypertarget{ecldoc:ml_core.math.uppergamma}{}

{\renewcommand{\arraystretch}{1.5}
\begin{tabularx}{\textwidth}{|>{\raggedright\arraybackslash}l|X|}
\hline
\hspace{0pt}\mytexttt{\color{red} REAL8} & \textbf{upperGamma} \\
\hline
\multicolumn{2}{|>{\raggedright\arraybackslash}X|}{\hspace{0pt}\mytexttt{\color{param} (REAL8 x, REAL8 y)}} \\
\hline
\end{tabularx}
}

\par
Return the upper incomplete gamma value of two real numbers, x and y.

\par
\begin{description}
\item [\colorbox{tagtype}{\color{white} \textbf{\textsf{PARAMETER}}}] \textbf{\underline{x}} the value of the first number
\item [\colorbox{tagtype}{\color{white} \textbf{\textsf{PARAMETER}}}] \textbf{\underline{y}} the value of the second number
\item [\colorbox{tagtype}{\color{white} \textbf{\textsf{RETURN}}}] \textbf{\underline{}} the upper incomplete gamma value
\end{description}

\rule{\linewidth}{0.5pt}

