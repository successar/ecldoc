\chapter*{\color{headfile}
{\large PBblas\slash\hspace{0pt}}
 \\
Vector2Diag
}
\hypertarget{ecldoc:toc:PBblas.Vector2Diag}{}
\hyperlink{ecldoc:toc:root/PBblas}{Go Up}

\section*{\underline{\textsf{IMPORTS}}}
\begin{doublespace}
{\large
PBblas |
PBblas.internal |
PBblas.internal.MatDims |
PBblas.Types |
PBblas.internal.Types |
PBblas.Constants |
}
\end{doublespace}

\section*{\underline{\textsf{DESCRIPTIONS}}}
\subsection*{\textsf{\colorbox{headtoc}{\color{white} FUNCTION}
Vector2Diag}}

\hypertarget{ecldoc:pbblas.vector2diag}{}

{\renewcommand{\arraystretch}{1.5}
\begin{tabularx}{\textwidth}{|>{\raggedright\arraybackslash}l|X|}
\hline
\hspace{0pt}\mytexttt{\color{red} DATASET(Layout\_Cell)} & \textbf{Vector2Diag} \\
\hline
\multicolumn{2}{|>{\raggedright\arraybackslash}X|}{\hspace{0pt}\mytexttt{\color{param} (DATASET(Layout\_Cell) X)}} \\
\hline
\end{tabularx}
}

\par
Convert a vector into a diagonal matrix. The typical notation is D = diag(V). The input X must be a 1 x N column vector or an N x 1 row vector. The resulting matrix, in either case will be N x N, with zero everywhere except the diagonal.

\par
\begin{description}
\item [\colorbox{tagtype}{\color{white} \textbf{\textsf{PARAMETER}}}] \textbf{\underline{X}} A row or column vector (i.e. N x 1 or 1 x N) in Layout\_Cell format
\item [\colorbox{tagtype}{\color{white} \textbf{\textsf{RETURN}}}] \textbf{\underline{}} An N x N matrix in Layout\_Cell format
\item [\colorbox{tagtype}{\color{white} \textbf{\textsf{SEE}}}] \textbf{\underline{}} PBblas/Types.Layout\_cell
\end{description}

\rule{\linewidth}{0.5pt}
