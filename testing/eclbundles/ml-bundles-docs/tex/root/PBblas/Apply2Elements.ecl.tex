\chapter*{\color{headfile}
{\large PBblas\slash\hspace{0pt}}
 \\
Apply2Elements
}
\hypertarget{ecldoc:toc:PBblas.Apply2Elements}{}
\hyperlink{ecldoc:toc:root/PBblas}{Go Up}

\section*{\underline{\textsf{IMPORTS}}}
\begin{doublespace}
{\large
PBblas |
PBblas.Types |
std.blas |
}
\end{doublespace}

\section*{\underline{\textsf{DESCRIPTIONS}}}
\subsection*{\textsf{\colorbox{headtoc}{\color{white} FUNCTION}
Apply2Elements}}

\hypertarget{ecldoc:pbblas.apply2elements}{}

{\renewcommand{\arraystretch}{1.5}
\begin{tabularx}{\textwidth}{|>{\raggedright\arraybackslash}l|X|}
\hline
\hspace{0pt}\mytexttt{\color{red} DATASET(Layout\_Cell)} & \textbf{Apply2Elements} \\
\hline
\multicolumn{2}{|>{\raggedright\arraybackslash}X|}{\hspace{0pt}\mytexttt{\color{param} (DATASET(Layout\_Cell) X, IElementFunc f)}} \\
\hline
\end{tabularx}
}

\par





Apply a function to each element of the matrix Use PBblas.IElementFunc as the prototype function. Input and ouput may be a single matrix, or myriad matrixes with different work item ids.






\par
\begin{description}
\item [\colorbox{tagtype}{\color{white} \textbf{\textsf{PARAMETER}}}] \textbf{\underline{X}} ||| TABLE ( Layout\_Cell ) --- A matrix (or multiple matrices) in Layout\_Cell form
\item [\colorbox{tagtype}{\color{white} \textbf{\textsf{PARAMETER}}}] \textbf{\underline{f}} ||| FUNCTION [ REAL8 , UNSIGNED4 , UNSIGNED4 ] ( REAL8 ) --- A function based on the IElementFunc prototype
\end{description}







\par
\begin{description}
\item [\colorbox{tagtype}{\color{white} \textbf{\textsf{RETURN}}}] \textbf{TABLE ( \{ UNSIGNED2 wi\_id , UNSIGNED4 x , UNSIGNED4 y , REAL8 v \} )} --- A matrix (or multiple matrices) in Layout\_Cell form
\end{description}






\par
\begin{description}
\item [\colorbox{tagtype}{\color{white} \textbf{\textsf{SEE}}}] PBblas/IElementFunc
\item [\colorbox{tagtype}{\color{white} \textbf{\textsf{SEE}}}] PBblas/Types.Layout\_Cell
\end{description}




\rule{\linewidth}{0.5pt}
