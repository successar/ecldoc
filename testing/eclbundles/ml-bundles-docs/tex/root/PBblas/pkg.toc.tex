\chapter*{\color{headtoc} PBblas}
\hypertarget{ecldoc:toc:root/PBblas}{}
\hyperlink{ecldoc:toc:root}{Go Up}

\begin{tabularx}{\textwidth}{|l|X|}
\hline
Name &
PBblas
 \\
\hline
Version &
3.0.1
 \\
\hline
Description &
Parallel Block Basic Linear Algebra Subsystem
 \\
\hline
License &
\url{http://www.apache.org/licenses/LICENSE-2.0}
 \\
\hline
Copyright &
Copyright (C) 2016, 2017 HPCC Systems
 \\
\hline
Authors &
HPCCSystems
 \\
\hline
DependsOn &
ML\_Core
 \\
\hline
Platform &
6.2.0
 \\
\hline
\end{tabularx}

\section*{Table of Contents}
{\renewcommand{\arraystretch}{1.5}
\begin{longtable}{|p{\textwidth}|}
\hline
\hyperlink{ecldoc:toc:PBblas.Apply2Elements}{Apply2Elements.ecl} \\
Apply a function to each element of the matrix Use PBblas.IElementFunc as the prototype function \\
\hline
\hyperlink{ecldoc:toc:PBblas.asum}{asum.ecl} \\
Absolute sum -- the ''Entrywise'' 1-norm \\
\hline
\hyperlink{ecldoc:toc:PBblas.axpy}{axpy.ecl} \\
Implements alpha*X + Y \\
\hline
\hyperlink{ecldoc:toc:PBblas.Constants}{Constants.ecl} \\
\hline
\hyperlink{ecldoc:toc:PBblas.Converted}{Converted.ecl} \\
Module to convert between ML\_Core/Types Field layouts (i.e \\
\hline
\hyperlink{ecldoc:toc:PBblas.ExtractTri}{ExtractTri.ecl} \\
Extract the upper or lower triangle from the composite output from getrf (LU Factorization) \\
\hline
\hyperlink{ecldoc:toc:PBblas.gemm}{gemm.ecl} \\
Extended Parallel Block Matrix Multiplication Module Implements: Result = alpha * op(A)op(B) + beta * C \\
\hline
\hyperlink{ecldoc:toc:PBblas.getrf}{getrf.ecl} \\
LU Factorization Splits a matrix into Lower and Upper triangular factors Produces composite LU matrix for the diagonal blocks \\
\hline
\hyperlink{ecldoc:toc:PBblas.HadamardProduct}{HadamardProduct.ecl} \\
Element-wise multiplication of X * Y \\
\hline
\hyperlink{ecldoc:toc:PBblas.IElementFunc}{IElementFunc.ecl} \\
Function prototype for a function to apply to each element of the \\
\hline
\hyperlink{ecldoc:toc:PBblas.MatUtils}{MatUtils.ecl} \\
Provides various utility attributes for manipulating cell-based matrixes \\
\hline
\hyperlink{ecldoc:toc:PBblas.potrf}{potrf.ecl} \\
Implements Cholesky factorization of A = U**T * U if Triangular.Upper requested or A = L * L**T if Triangualr.Lower is requested \\
\hline
\hyperlink{ecldoc:toc:PBblas.scal}{scal.ecl} \\
Scale a matrix by a constant Result is alpha * X This supports a ''myriad'' style interface in that X may be a set of independent matrices separated by different work-item ids \\
\hline
\hyperlink{ecldoc:toc:PBblas.tran}{tran.ecl} \\
Transpose a matrix and sum into base matrix \\
\hline
\hyperlink{ecldoc:toc:PBblas.trsm}{trsm.ecl} \\
Partitioned block parallel triangular matrix solver \\
\hline
\hyperlink{ecldoc:toc:PBblas.Types}{Types.ecl} \\
Types for the Parallel Block Basic Linear Algebra Sub-programs support WARNING: attributes marked with WARNING can not be changed without making corresponding changes to the C++ attributes \\
\hline
\hyperlink{ecldoc:toc:PBblas.Vector2Diag}{Vector2Diag.ecl} \\
Convert a vector into a diagonal matrix \\
\hline
\end{longtable}
}

\chapter*{PBblas.Apply2Elements}
\hypertarget{PBblas.Apply2Elements}{}

\section*{\underline{IMPORTS}}
\begin{itemize}
\item PBblas
\item PBblas.Types
\item std.BLAS
\end{itemize}

\section*{\underline{DESCRIPTIONS}}
\subsection*{function : Apply2Elements}
\hypertarget{ecldoc:pbblas.apply2elements}{FUNCTION : DATASET(Layout\_Cell) Apply2Elements(DATASET(Layout\_Cell) X, IElementFunc f)} \\
\hyperlink{ecldoc:}{Up} \\
\par
Apply a function to each element of the matrix Use PBblas.IElementFunc as the prototype function. Input and ouput may be a single matrix, or myriad matrixes with different work item ids. \\
\textbf{Parameter} : X ||| A matrix (or multiple matrices) in Layout\_Cell form \\
\textbf{Parameter} : f ||| A function based on the IElementFunc prototype \\
\textbf{Return} : A matrix (or multiple matrices) in Layout\_Cell form \\
\textbf{See} : PBblas/IElementFunc \\
\textbf{See} : PBblas/Types.Layout\_Cell \\

\chapter*{PBblas.asum}
\hypertarget{ecldoc:toc:PBblas.asum}{}

\section*{\underline{IMPORTS}}
\begin{itemize}
\item PBblas
\item PBblas.Types
\item PBblas.internal
\item PBblas.internal.Types
\item PBblas.internal.MatDims
\item PBblas.internal.Converted
\item std.BLAS
\end{itemize}

\section*{\underline{DESCRIPTIONS}}
\subsection*{FUNCTION : asum}
\hypertarget{ecldoc:pbblas.asum}{}
\begin{minipage}[t]{\textwidth}
\begin{flushleft}
DATASET(Layout\_Norm) asum (DATASET(Layout\_Cell) X)
\end{flushleft}
\end{minipage}
\hyperlink{ecldoc:toc:PBblas}{Up}

\par
Absolute sum -- the ''Entrywise'' 1-norm Compute SUM(ABS(X))
\par
\textbf{Parameter} : X ||| Matrix or set of matrices in Layout\_Cell format \\
\textbf{Return} : DATASET(Layout\_Norm) with one record per work item \\
\textbf{See} : PBblas/Types.Layout\_Cell \\

\chapter*{\color{headfile}
{\large PBblas\slash\hspace{0pt}}
 \\
axpy
}
\hypertarget{ecldoc:toc:PBblas.axpy}{}
\hyperlink{ecldoc:toc:root/PBblas}{Go Up}

\section*{\underline{\textsf{IMPORTS}}}
\begin{doublespace}
{\large
PBblas |
PBblas.Types |
}
\end{doublespace}

\section*{\underline{\textsf{DESCRIPTIONS}}}
\subsection*{\textsf{\colorbox{headtoc}{\color{white} FUNCTION}
axpy}}

\hypertarget{ecldoc:pbblas.axpy}{}

{\renewcommand{\arraystretch}{1.5}
\begin{tabularx}{\textwidth}{|>{\raggedright\arraybackslash}l|X|}
\hline
\hspace{0pt}\mytexttt{\color{red} DATASET(Layout\_Cell)} & \textbf{axpy} \\
\hline
\multicolumn{2}{|>{\raggedright\arraybackslash}X|}{\hspace{0pt}\mytexttt{\color{param} (value\_t alpha, DATASET(Layout\_Cell) X, DATASET(Layout\_Cell) Y)}} \\
\hline
\end{tabularx}
}

\par
Implements alpha*X + Y X and Y must have same shape

\par
\begin{description}
\item [\colorbox{tagtype}{\color{white} \textbf{\textsf{PARAMETER}}}] \textbf{\underline{alpha}} Scalar multiplier for the X matrix
\item [\colorbox{tagtype}{\color{white} \textbf{\textsf{PARAMETER}}}] \textbf{\underline{X}} X matrix in DATASET(Layout\_Cell) form
\item [\colorbox{tagtype}{\color{white} \textbf{\textsf{PARAMETER}}}] \textbf{\underline{Y}} Y matrix in DATASET(Layout\_Call) form
\item [\colorbox{tagtype}{\color{white} \textbf{\textsf{RETURN}}}] \textbf{\underline{}} Matrix in DATASET(Layout\_Cell) form
\item [\colorbox{tagtype}{\color{white} \textbf{\textsf{SEE}}}] \textbf{\underline{}} PBblas/Types.layout\_cell
\end{description}

\rule{\linewidth}{0.5pt}

\chapter*{ML\_Core.Constants}
\hypertarget{ecldoc:toc:ML_Core.Constants}{}

\section*{\underline{IMPORTS}}

\section*{\underline{DESCRIPTIONS}}
\subsection*{MODULE : Constants}
\hypertarget{ecldoc:ML_Core.Constants}{}

{\renewcommand{\arraystretch}{1.5}
\begin{tabularx}{\textwidth}{|>{\raggedright\arraybackslash}l|X|}
\hline
\hspace{0pt} & Constants \\
\hline
\end{tabularx}
}

\hyperlink{ecldoc:toc:ML_Core}{Up}

\par
Useful constants


\hyperlink{ecldoc:ml_core.constants.pi}{Pi}  |
\hyperlink{ecldoc:ml_core.constants.root_2}{Root\_2}  |

\rule{\textwidth}{0.4pt}

\subsection*{ATTRIBUTE : Pi}
\hypertarget{ecldoc:ml_core.constants.pi}{}

{\renewcommand{\arraystretch}{1.5}
\begin{tabularx}{\textwidth}{|>{\raggedright\arraybackslash}l|X|}
\hline
\hspace{0pt} & Pi \\
\hline
\end{tabularx}
}

\hyperlink{ecldoc:ML_Core.Constants}{Up}

\par
Constant PI


\rule{\textwidth}{0.4pt}
\subsection*{ATTRIBUTE : Root\_2}
\hypertarget{ecldoc:ml_core.constants.root_2}{}

{\renewcommand{\arraystretch}{1.5}
\begin{tabularx}{\textwidth}{|>{\raggedright\arraybackslash}l|X|}
\hline
\hspace{0pt} & Root\_2 \\
\hline
\end{tabularx}
}

\hyperlink{ecldoc:ML_Core.Constants}{Up}

\par
Constant square root of 2


\rule{\textwidth}{0.4pt}



\chapter*{\color{headfile}
{\large PBblas\slash\hspace{0pt}}
 \\
Converted
}
\hypertarget{ecldoc:toc:PBblas.Converted}{}
\hyperlink{ecldoc:toc:root/PBblas}{Go Up}

\section*{\underline{\textsf{IMPORTS}}}
\begin{doublespace}
{\large
PBblas |
PBblas.Types |
ML\_Core.Types |
}
\end{doublespace}

\section*{\underline{\textsf{DESCRIPTIONS}}}
\subsection*{\textsf{\colorbox{headtoc}{\color{white} MODULE}
Converted}}

\hypertarget{ecldoc:PBblas.Converted}{}

{\renewcommand{\arraystretch}{1.5}
\begin{tabularx}{\textwidth}{|>{\raggedright\arraybackslash}l|X|}
\hline
\hspace{0pt}\mytexttt{\color{red} } & \textbf{Converted} \\
\hline
\end{tabularx}
}

\par





Module to convert between ML\_Core/Types Field layouts (i.e. NumericField and DiscreteField) and PBblas matrix layout (i.e. Layout\_Cell)







\textbf{Children}
\begin{enumerate}
\item \hyperlink{ecldoc:pbblas.converted.nftomatrix}{NFToMatrix}
: Convert NumericField dataset to Matrix
\item \hyperlink{ecldoc:pbblas.converted.dftomatrix}{DFToMatrix}
: Convert DiscreteField dataset to Matrix
\item \hyperlink{ecldoc:pbblas.converted.matrixtonf}{MatrixToNF}
: Convert Matrix to NumericField dataset
\item \hyperlink{ecldoc:pbblas.converted.matrixtodf}{MatrixToDF}
: Convert Matrix to DiscreteField dataset
\end{enumerate}

\rule{\linewidth}{0.5pt}

\subsection*{\textsf{\colorbox{headtoc}{\color{white} FUNCTION}
NFToMatrix}}

\hypertarget{ecldoc:pbblas.converted.nftomatrix}{}
\hspace{0pt} \hyperlink{ecldoc:PBblas.Converted}{Converted} \textbackslash 

{\renewcommand{\arraystretch}{1.5}
\begin{tabularx}{\textwidth}{|>{\raggedright\arraybackslash}l|X|}
\hline
\hspace{0pt}\mytexttt{\color{red} DATASET(Layout\_Cell)} & \textbf{NFToMatrix} \\
\hline
\multicolumn{2}{|>{\raggedright\arraybackslash}X|}{\hspace{0pt}\mytexttt{\color{param} (DATASET(NumericField) recs)}} \\
\hline
\end{tabularx}
}

\par





Convert NumericField dataset to Matrix






\par
\begin{description}
\item [\colorbox{tagtype}{\color{white} \textbf{\textsf{PARAMETER}}}] \textbf{\underline{recs}} ||| TABLE ( NumericField ) --- Record Dataset in DATASET(NumericField) format
\end{description}







\par
\begin{description}
\item [\colorbox{tagtype}{\color{white} \textbf{\textsf{RETURN}}}] \textbf{TABLE ( \{ UNSIGNED2 wi\_id , UNSIGNED4 x , UNSIGNED4 y , REAL8 v \} )} --- Matrix in DATASET(Layout\_Cell) format
\end{description}







\par
\begin{description}
\item [\colorbox{tagtype}{\color{white} \textbf{\textsf{SEE}}}] PBblas/Types.Layout\_Cell
\item [\colorbox{tagtype}{\color{white} \textbf{\textsf{SEE}}}] ML\_Core/Types.NumericField
\end{description}



\rule{\linewidth}{0.5pt}
\subsection*{\textsf{\colorbox{headtoc}{\color{white} FUNCTION}
DFToMatrix}}

\hypertarget{ecldoc:pbblas.converted.dftomatrix}{}
\hspace{0pt} \hyperlink{ecldoc:PBblas.Converted}{Converted} \textbackslash 

{\renewcommand{\arraystretch}{1.5}
\begin{tabularx}{\textwidth}{|>{\raggedright\arraybackslash}l|X|}
\hline
\hspace{0pt}\mytexttt{\color{red} DATASET(Layout\_Cell)} & \textbf{DFToMatrix} \\
\hline
\multicolumn{2}{|>{\raggedright\arraybackslash}X|}{\hspace{0pt}\mytexttt{\color{param} (DATASET(DiscreteField) recs)}} \\
\hline
\end{tabularx}
}

\par





Convert DiscreteField dataset to Matrix






\par
\begin{description}
\item [\colorbox{tagtype}{\color{white} \textbf{\textsf{PARAMETER}}}] \textbf{\underline{recs}} ||| TABLE ( DiscreteField ) --- Record Dataset in DATASET(DiscreteField) format
\end{description}







\par
\begin{description}
\item [\colorbox{tagtype}{\color{white} \textbf{\textsf{RETURN}}}] \textbf{TABLE ( \{ UNSIGNED2 wi\_id , UNSIGNED4 x , UNSIGNED4 y , REAL8 v \} )} --- Matrix in DATASET(Layout\_Cell) format
\end{description}







\par
\begin{description}
\item [\colorbox{tagtype}{\color{white} \textbf{\textsf{SEE}}}] PBblas/Types.Layout\_Cell
\item [\colorbox{tagtype}{\color{white} \textbf{\textsf{SEE}}}] ML\_Core/Types.DiscreteField
\end{description}



\rule{\linewidth}{0.5pt}
\subsection*{\textsf{\colorbox{headtoc}{\color{white} FUNCTION}
MatrixToNF}}

\hypertarget{ecldoc:pbblas.converted.matrixtonf}{}
\hspace{0pt} \hyperlink{ecldoc:PBblas.Converted}{Converted} \textbackslash 

{\renewcommand{\arraystretch}{1.5}
\begin{tabularx}{\textwidth}{|>{\raggedright\arraybackslash}l|X|}
\hline
\hspace{0pt}\mytexttt{\color{red} DATASET(NumericField)} & \textbf{MatrixToNF} \\
\hline
\multicolumn{2}{|>{\raggedright\arraybackslash}X|}{\hspace{0pt}\mytexttt{\color{param} (DATASET(Layout\_Cell) mat)}} \\
\hline
\end{tabularx}
}

\par





Convert Matrix to NumericField dataset






\par
\begin{description}
\item [\colorbox{tagtype}{\color{white} \textbf{\textsf{PARAMETER}}}] \textbf{\underline{mat}} ||| TABLE ( Layout\_Cell ) --- Matrix in DATASET(Layout\_Cell) format
\end{description}







\par
\begin{description}
\item [\colorbox{tagtype}{\color{white} \textbf{\textsf{RETURN}}}] \textbf{TABLE ( \{ UNSIGNED2 wi , UNSIGNED8 id , UNSIGNED4 number , REAL8 value \} )} --- NumericField Dataset
\end{description}







\par
\begin{description}
\item [\colorbox{tagtype}{\color{white} \textbf{\textsf{SEE}}}] PBblas/Types.Layout\_Cell
\item [\colorbox{tagtype}{\color{white} \textbf{\textsf{SEE}}}] ML\_Core/Types.NumericField
\end{description}



\rule{\linewidth}{0.5pt}
\subsection*{\textsf{\colorbox{headtoc}{\color{white} FUNCTION}
MatrixToDF}}

\hypertarget{ecldoc:pbblas.converted.matrixtodf}{}
\hspace{0pt} \hyperlink{ecldoc:PBblas.Converted}{Converted} \textbackslash 

{\renewcommand{\arraystretch}{1.5}
\begin{tabularx}{\textwidth}{|>{\raggedright\arraybackslash}l|X|}
\hline
\hspace{0pt}\mytexttt{\color{red} DATASET(DiscreteField)} & \textbf{MatrixToDF} \\
\hline
\multicolumn{2}{|>{\raggedright\arraybackslash}X|}{\hspace{0pt}\mytexttt{\color{param} (DATASET(Layout\_Cell) mat)}} \\
\hline
\end{tabularx}
}

\par





Convert Matrix to DiscreteField dataset






\par
\begin{description}
\item [\colorbox{tagtype}{\color{white} \textbf{\textsf{PARAMETER}}}] \textbf{\underline{mat}} ||| TABLE ( Layout\_Cell ) --- Matrix in DATASET(Layout\_Cell) format
\end{description}







\par
\begin{description}
\item [\colorbox{tagtype}{\color{white} \textbf{\textsf{RETURN}}}] \textbf{TABLE ( \{ UNSIGNED2 wi , UNSIGNED8 id , UNSIGNED4 number , INTEGER4 value \} )} --- DiscreteField Dataset
\end{description}







\par
\begin{description}
\item [\colorbox{tagtype}{\color{white} \textbf{\textsf{SEE}}}] PBblas/Types.Layout\_Cell
\item [\colorbox{tagtype}{\color{white} \textbf{\textsf{SEE}}}] ML\_Core/Types.DiscreteField
\end{description}



\rule{\linewidth}{0.5pt}



\chapter*{\color{headfile}
{\large PBblas\slash\hspace{0pt}}
 \\
ExtractTri
}
\hypertarget{ecldoc:toc:PBblas.ExtractTri}{}
\hyperlink{ecldoc:toc:root/PBblas}{Go Up}

\section*{\underline{\textsf{IMPORTS}}}
\begin{doublespace}
{\large
PBblas |
std.blas |
PBblas.Types |
PBblas.internal |
PBblas.internal.Types |
PBblas.internal.MatDims |
PBblas.internal.Converted |
}
\end{doublespace}

\section*{\underline{\textsf{DESCRIPTIONS}}}
\subsection*{\textsf{\colorbox{headtoc}{\color{white} FUNCTION}
ExtractTri}}

\hypertarget{ecldoc:pbblas.extracttri}{}

{\renewcommand{\arraystretch}{1.5}
\begin{tabularx}{\textwidth}{|>{\raggedright\arraybackslash}l|X|}
\hline
\hspace{0pt}\mytexttt{\color{red} DATASET(Layout\_Cell)} & \textbf{ExtractTri} \\
\hline
\multicolumn{2}{|>{\raggedright\arraybackslash}X|}{\hspace{0pt}\mytexttt{\color{param} (Triangle tri, Diagonal dt, DATASET(Layout\_Cell) A)}} \\
\hline
\end{tabularx}
}

\par
Extract the upper or lower triangle from the composite output from getrf (LU Factorization).

\par
\begin{description}
\item [\colorbox{tagtype}{\color{white} \textbf{\textsf{PARAMETER}}}] \textbf{\underline{tri}} Triangle type: Upper or Lower (see Types.Triangle)
\item [\colorbox{tagtype}{\color{white} \textbf{\textsf{PARAMETER}}}] \textbf{\underline{dt}} Diagonal type: Unit or non unit (see Types.Diagonal)
\item [\colorbox{tagtype}{\color{white} \textbf{\textsf{PARAMETER}}}] \textbf{\underline{A}} Matrix of cells. See Types.Layout\_Cell
\item [\colorbox{tagtype}{\color{white} \textbf{\textsf{RETURN}}}] \textbf{\underline{}} Matrix of cells in Layout\_Cell format representing a triangular matrix (upper or lower)
\item [\colorbox{tagtype}{\color{white} \textbf{\textsf{SEE}}}] \textbf{\underline{}} Std.PBblas.Types
\end{description}

\rule{\linewidth}{0.5pt}

\chapter*{\color{headfile}
{\large PBblas\slash\hspace{0pt}}
 \\
gemm
}
\hypertarget{ecldoc:toc:PBblas.gemm}{}
\hyperlink{ecldoc:toc:root/PBblas}{Go Up}

\section*{\underline{\textsf{IMPORTS}}}
\begin{doublespace}
{\large
PBblas |
PBblas.Types |
PBblas.internal |
PBblas.internal.Types |
std.blas |
PBblas.internal.MatDims |
std.system.Thorlib |
}
\end{doublespace}

\section*{\underline{\textsf{DESCRIPTIONS}}}
\subsection*{\textsf{\colorbox{headtoc}{\color{white} FUNCTION}
gemm}}

\hypertarget{ecldoc:pbblas.gemm}{}

{\renewcommand{\arraystretch}{1.5}
\begin{tabularx}{\textwidth}{|>{\raggedright\arraybackslash}l|X|}
\hline
\hspace{0pt}\mytexttt{\color{red} DATASET(Layout\_Cell)} & \textbf{gemm} \\
\hline
\multicolumn{2}{|>{\raggedright\arraybackslash}X|}{\hspace{0pt}\mytexttt{\color{param} (BOOLEAN transposeA, BOOLEAN transposeB, value\_t alpha, DATASET(Layout\_Cell) A\_in, DATASET(Layout\_Cell) B\_in, DATASET(Layout\_Cell) C\_in=emptyC, value\_t beta=0.0)}} \\
\hline
\end{tabularx}
}

\par





Extended Parallel Block Matrix Multiplication Module Implements: Result = alpha * op(A)op(B) + beta * C. op is No Transpose or Transpose. Multiplies two matrixes A and B, with an optional pre-multiply transpose for each Optionally scales the product by the scalar ''alpha''. Then adds an optional C matrix to the product after scaling C by the scalar ''beta''. A, B, and C are specified as DATASET(Layout\_Cell), as is the Resulting matrix. Layout\_Cell describes a sparse matrix stored as a list of x, y, and value. This interface also provides a ''Myriad'' capability allowing multiple similar operations to be performed on independent sets of matrixes in parallel. This is done by use of the work-item id (wi\_id) in each cell of the matrixes. Cells with the same wi\_id are considered part of the same matrix. In the myriad form, each input matrix A, B, and (optionally) C can contain many independent matrixes. The wi\_ids are matched up such that each operation involves the A, B, and C with the same wi\_id. A and B must therefore contain the same set of wi\_ids, while C is optional for any wi\_id. The same parameters: alpha, beta, transposeA, and transposeB are used for all work-items. The result will contain cells from all provided work-items. Result has same shape as C if provided. Note that matrixes are not explicitly dimensioned. The shape is determined by the highest value of x and y for each work-item.






\par
\begin{description}
\item [\colorbox{tagtype}{\color{white} \textbf{\textsf{PARAMETER}}}] \textbf{\underline{alpha}} ||| REAL8 --- Scalar multiplier for alpha * A * B
\item [\colorbox{tagtype}{\color{white} \textbf{\textsf{PARAMETER}}}] \textbf{\underline{transposeA}} ||| BOOLEAN --- Boolean indicating whether matrix A should be transposed before multiplying
\item [\colorbox{tagtype}{\color{white} \textbf{\textsf{PARAMETER}}}] \textbf{\underline{C\_in}} ||| TABLE ( Layout\_Cell ) --- Same as above for the 'C' matrix (addend). May be omitted.
\item [\colorbox{tagtype}{\color{white} \textbf{\textsf{PARAMETER}}}] \textbf{\underline{beta}} ||| REAL8 --- A scalar multiplier for beta * C, scales the C matrix before addition. May be omitted.
\item [\colorbox{tagtype}{\color{white} \textbf{\textsf{PARAMETER}}}] \textbf{\underline{B\_in}} ||| TABLE ( Layout\_Cell ) --- Same as above for the 'B' matrix (multiplicand)
\item [\colorbox{tagtype}{\color{white} \textbf{\textsf{PARAMETER}}}] \textbf{\underline{transposeB}} ||| BOOLEAN --- Same as above but for matrix B
\item [\colorbox{tagtype}{\color{white} \textbf{\textsf{PARAMETER}}}] \textbf{\underline{A\_in}} ||| TABLE ( Layout\_Cell ) --- 'A' matrix (multiplier) in Layout\_Cell format
\end{description}







\par
\begin{description}
\item [\colorbox{tagtype}{\color{white} \textbf{\textsf{RETURN}}}] \textbf{TABLE ( \{ UNSIGNED2 wi\_id , UNSIGNED4 x , UNSIGNED4 y , REAL8 v \} )} --- Result matrix in Layout\_Cell format.
\end{description}







\par
\begin{description}
\item [\colorbox{tagtype}{\color{white} \textbf{\textsf{SEE}}}] PBblas/Types.Layout\_Cell
\end{description}



\rule{\linewidth}{0.5pt}

\chapter*{\color{headfile}
{\large PBblas\slash\hspace{0pt}}
 \\
getrf
}
\hypertarget{ecldoc:toc:PBblas.getrf}{}
\hyperlink{ecldoc:toc:root/PBblas}{Go Up}

\section*{\underline{\textsf{IMPORTS}}}
\begin{doublespace}
{\large
PBblas |
PBblas.Types |
PBblas.internal |
PBblas.internal.Types |
std.blas |
PBblas.internal.MatDims |
std.system.Thorlib |
}
\end{doublespace}

\section*{\underline{\textsf{DESCRIPTIONS}}}
\subsection*{\textsf{\colorbox{headtoc}{\color{white} FUNCTION}
getrf}}

\hypertarget{ecldoc:pbblas.getrf}{}

{\renewcommand{\arraystretch}{1.5}
\begin{tabularx}{\textwidth}{|>{\raggedright\arraybackslash}l|X|}
\hline
\hspace{0pt}\mytexttt{\color{red} DATASET(Layout\_Cell)} & \textbf{getrf} \\
\hline
\multicolumn{2}{|>{\raggedright\arraybackslash}X|}{\hspace{0pt}\mytexttt{\color{param} (DATASET(Layout\_Cell) A)}} \\
\hline
\end{tabularx}
}

\par





LU Factorization Splits a matrix into Lower and Upper triangular factors Produces composite LU matrix for the diagonal blocks. Iterates through the matrix a row of blocks and column of blocks at a time. Partition A into M block rows and N block columns. The A11 cell is a single block. A12 is a single row of blocks with N-1 columns. A21 is a single column of blocks with M-1 rows. A22 is a sub-matrix of M-1 x N-1 blocks. | A11 A12 | | L11 0 | | U11 U12 | | A21 A22 | == | L21 L22 | * | 0 U22 | | L11*U11 L11*U12 | == | L21*U11 L21*U12 + L22*U22 | Based upon PB-BLAS: A set of parallel block basic linear algebra subprograms by Choi and Dongarra This module supports the ''Myriad'' style interface, allowing many independent problems to be worked on at once. The A matrix can contain multiple matrixes to be factored, indicated by different values for work-item id (wi\_id). Note: The returned matrix includes both the upper and lower factors. This matrix can be used directly by trsm which will only use the part indicated by trsm's 'triangle' parameter (i.e. upper or lower). To extract the upper or lower triangle explicitly for other purposes, use the ExtractTri function. When passing the Lower matrix to the triangle solver (trsm), set the ''Diagonal'' parameter to ''UnitTri''. This is necessary because both triangular matrixes returned from this function are packed into a square matrix with only one diagonal. By convention, The Lower triangle is assumed to be a Unit Triangle (diagonal all ones), so the diagonal contained in the returned matrix is for the Upper factor and must be ignored (i.e. assumed to be all ones) when referencing the Lower triangle.






\par
\begin{description}
\item [\colorbox{tagtype}{\color{white} \textbf{\textsf{PARAMETER}}}] \textbf{\underline{A}} ||| TABLE ( Layout\_Cell ) --- The input matrix in Types.Layout\_Cell format
\end{description}







\par
\begin{description}
\item [\colorbox{tagtype}{\color{white} \textbf{\textsf{RETURN}}}] \textbf{TABLE ( \{ UNSIGNED2 wi\_id , UNSIGNED4 x , UNSIGNED4 y , REAL8 v \} )} --- Resulting factored matrix in Layout\_Cell format
\end{description}







\par
\begin{description}
\item [\colorbox{tagtype}{\color{white} \textbf{\textsf{SEE}}}] Types.Layout\_Cell
\item [\colorbox{tagtype}{\color{white} \textbf{\textsf{SEE}}}] ExtractTri
\end{description}



\rule{\linewidth}{0.5pt}

\chapter*{\color{headfile}
{\large PBblas\slash\hspace{0pt}}
 \\
HadamardProduct
}
\hypertarget{ecldoc:toc:PBblas.HadamardProduct}{}
\hyperlink{ecldoc:toc:root/PBblas}{Go Up}

\section*{\underline{\textsf{IMPORTS}}}
\begin{doublespace}
{\large
PBblas |
PBblas.internal |
PBblas.internal.MatDims |
PBblas.Types |
PBblas.internal.Types |
PBblas.internal.Converted |
std.blas |
std.system.Thorlib |
}
\end{doublespace}

\section*{\underline{\textsf{DESCRIPTIONS}}}
\subsection*{\textsf{\colorbox{headtoc}{\color{white} FUNCTION}
HadamardProduct}}

\hypertarget{ecldoc:pbblas.hadamardproduct}{}

{\renewcommand{\arraystretch}{1.5}
\begin{tabularx}{\textwidth}{|>{\raggedright\arraybackslash}l|X|}
\hline
\hspace{0pt}\mytexttt{\color{red} DATASET(Layout\_Cell)} & \textbf{HadamardProduct} \\
\hline
\multicolumn{2}{|>{\raggedright\arraybackslash}X|}{\hspace{0pt}\mytexttt{\color{param} (DATASET(Layout\_Cell) X, DATASET(Layout\_Cell) Y)}} \\
\hline
\end{tabularx}
}

\par





Element-wise multiplication of X * Y. Supports the ''myriad'' style interface -- X and Y may contain multiple separate matrixes. Each X will be multiplied by the Y with the same work-item id. Note: This performs element-wise multiplication. For dot-product matrix multiplication, use PBblas.gemm.






\par
\begin{description}
\item [\colorbox{tagtype}{\color{white} \textbf{\textsf{PARAMETER}}}] \textbf{\underline{Y}} ||| TABLE ( Layout\_Cell ) --- A matrix (or multiple matrices) in Layout\_Cell form
\item [\colorbox{tagtype}{\color{white} \textbf{\textsf{PARAMETER}}}] \textbf{\underline{X}} ||| TABLE ( Layout\_Cell ) --- A matrix (or multiple matrices) in Layout\_Cell form
\end{description}







\par
\begin{description}
\item [\colorbox{tagtype}{\color{white} \textbf{\textsf{RETURN}}}] \textbf{TABLE ( \{ UNSIGNED2 wi\_id , UNSIGNED4 x , UNSIGNED4 y , REAL8 v \} )} --- A matrix (or multiple matrices) in Layout\_Cell form
\end{description}







\par
\begin{description}
\item [\colorbox{tagtype}{\color{white} \textbf{\textsf{SEE}}}] PBblas/Types.Layout\_Cell
\end{description}



\rule{\linewidth}{0.5pt}

\chapter*{PBblas.IElementFunc}
\hypertarget{ecldoc:toc:PBblas.IElementFunc}{}

\section*{\underline{IMPORTS}}
\begin{itemize}
\item PBblas
\end{itemize}

\section*{\underline{DESCRIPTIONS}}
\subsection*{FUNCTION : IElementFunc}
\hypertarget{ecldoc:pbblas.ielementfunc}{}
\begin{minipage}[t]{\textwidth}
\begin{flushleft}
value\_t IElementFunc (value\_t v, dimension\_t r, dimension\_t c)
\end{flushleft}
\end{minipage}
\hyperlink{ecldoc:toc:PBblas}{Up}

\par
Function prototype for a function to apply to each element of the distributed matrix Base your function on this prototype:
\par
\textbf{Parameter} : v ||| Input value \\
\textbf{Parameter} : r ||| Row number (1 based) \\
\textbf{Parameter} : c ||| Column number (1 based) \\
\textbf{Return} : Output value \\
\textbf{See} : PBblas/Apply2Elements \\

\chapter*{PBblas.MatUtils}
\hypertarget{ecldoc:toc:PBblas.MatUtils}{}

\section*{\underline{IMPORTS}}
\begin{itemize}
\item PBblas
\item PBblas.Types
\item PBblas.internal
\item PBblas.internal.Types
\item PBblas.internal.MatDims
\end{itemize}

\section*{\underline{DESCRIPTIONS}}
\subsection*{MODULE : MatUtils}
\hypertarget{ecldoc:PBblas.MatUtils}{}

{\renewcommand{\arraystretch}{1.5}
\begin{tabularx}{\textwidth}{|>{\raggedright\arraybackslash}l|X|}
\hline
\hspace{0pt} & MatUtils \\
\hline
\end{tabularx}
}

\hyperlink{ecldoc:toc:PBblas}{Up}

\par
Provides various utility attributes for manipulating cell-based matrixes

\par
\begin{description}
\item [\textbf{See}] Std/PBblas/Types.Layout\_Cell
\end{description}

\begin{enumerate}
\item \hyperlink{ecldoc:pbblas.matutils.getworkitems}{GetWorkItems}
\item \hyperlink{ecldoc:pbblas.matutils.insertcols}{InsertCols}
\item \hyperlink{ecldoc:pbblas.matutils.transpose}{Transpose}
\end{enumerate}

\rule{\textwidth}{0.4pt}

\subsection*{FUNCTION : GetWorkItems}
\hypertarget{ecldoc:pbblas.matutils.getworkitems}{}

{\renewcommand{\arraystretch}{1.5}
\begin{tabularx}{\textwidth}{|>{\raggedright\arraybackslash}l|X|}
\hline
\hspace{0pt}DATASET(Layout\_WI\_ID) & GetWorkItems \\
\hline
\multicolumn{2}{|>{\raggedright\arraybackslash}X|}{\hspace{0pt}(DATASET(Layout\_Cell) cells)} \\
\hline
\end{tabularx}
}

\hyperlink{ecldoc:PBblas.MatUtils}{Up}

\par
Get a list of work-item ids from a matrix containing one or more work items

\par
\begin{description}
\item [\textbf{Parameter}] cells ||| A matrix in Layout\_Cell format
\item [\textbf{Return}] DATASET(Layout\_WI\_ID), one record per work-item
\item [\textbf{See}] PBblas/Types.Layout\_Cell
\item [\textbf{See}] PBblas/Types.Layout\_WI\_ID
\end{description}

\rule{\textwidth}{0.4pt}
\subsection*{FUNCTION : InsertCols}
\hypertarget{ecldoc:pbblas.matutils.insertcols}{}

{\renewcommand{\arraystretch}{1.5}
\begin{tabularx}{\textwidth}{|>{\raggedright\arraybackslash}l|X|}
\hline
\hspace{0pt}DATASET(Layout\_Cell) & InsertCols \\
\hline
\multicolumn{2}{|>{\raggedright\arraybackslash}X|}{\hspace{0pt}(DATASET(Layout\_Cell) M, UNSIGNED cols\_to\_insert=1, value\_t insert\_val=1)} \\
\hline
\end{tabularx}
}

\hyperlink{ecldoc:PBblas.MatUtils}{Up}

\par
Insert one or more columns of a fixed value into a matrix. Columns are inserted before the first original column. This attribute supports the myriad interface. Multiple independent matrixes can be represented by M.

\par
\begin{description}
\item [\textbf{Parameter}] M ||| the input matrix
\item [\textbf{Parameter}] cols\_to\_insert ||| the number of columns to insert, default 1
\item [\textbf{Parameter}] insert\_val ||| the value for each cell of the new column(s), default 0
\item [\textbf{Return}] matrix in Layout\_Cell format with additional column(s)
\end{description}

\rule{\textwidth}{0.4pt}
\subsection*{FUNCTION : Transpose}
\hypertarget{ecldoc:pbblas.matutils.transpose}{}

{\renewcommand{\arraystretch}{1.5}
\begin{tabularx}{\textwidth}{|>{\raggedright\arraybackslash}l|X|}
\hline
\hspace{0pt}DATASET(Layout\_Cell) & Transpose \\
\hline
\multicolumn{2}{|>{\raggedright\arraybackslash}X|}{\hspace{0pt}(DATASET(Layout\_Cell) M)} \\
\hline
\end{tabularx}
}

\hyperlink{ecldoc:PBblas.MatUtils}{Up}

\par
Transpose a matrix This attribute supports the myriad interface. Multiple independent matrixes can be represented by M.

\par
\begin{description}
\item [\textbf{Parameter}] M ||| A matrix represented as DATASET(Layout\_Cell)
\item [\textbf{Return}] Transposed matrix in Layout\_Cell format
\item [\textbf{See}] PBblas/Types.Layout\_Cell
\end{description}

\rule{\textwidth}{0.4pt}



\chapter*{\color{headfile}
{\large PBblas\slash\hspace{0pt}}
 \\
potrf
}
\hypertarget{ecldoc:toc:PBblas.potrf}{}
\hyperlink{ecldoc:toc:root/PBblas}{Go Up}

\section*{\underline{\textsf{IMPORTS}}}
\begin{doublespace}
{\large
PBblas |
PBblas.Types |
std.blas |
PBblas.internal |
PBblas.internal.Types |
PBblas.internal.MatDims |
PBblas.internal.Converted |
std.system.Thorlib |
}
\end{doublespace}

\section*{\underline{\textsf{DESCRIPTIONS}}}
\subsection*{\textsf{\colorbox{headtoc}{\color{white} FUNCTION}
potrf}}

\hypertarget{ecldoc:pbblas.potrf}{}

{\renewcommand{\arraystretch}{1.5}
\begin{tabularx}{\textwidth}{|>{\raggedright\arraybackslash}l|X|}
\hline
\hspace{0pt}\mytexttt{\color{red} DATASET(Layout\_Cell)} & \textbf{potrf} \\
\hline
\multicolumn{2}{|>{\raggedright\arraybackslash}X|}{\hspace{0pt}\mytexttt{\color{param} (Triangle tri, DATASET(Layout\_Cell) A\_in)}} \\
\hline
\end{tabularx}
}

\par
Implements Cholesky factorization of A = U**T * U if Triangular.Upper requested or A = L * L**T if Triangualr.Lower is requested. The matrix A must be symmetric positive definite. 
\begin{verbatim}

  | A11   A12 |      |  L11   0   |    | L11**T     L21**T |
  | A21   A22 |  ==  |  L21   L22 | *  |  0           L22  |

                     | L11*L11**T          L11*L21**T      |
                 ==  | L21*L11**T  L21*L21**T + L22*L22**T |
 \end{verbatim}

 So, use Cholesky on the first block to get L11. L21 = A21*L11**T**-1 which can be found by dtrsm on each column block A22' is A22 - L21*L21**T 
\par
 Based upon PB-BLAS: A set of parallel block basic linear algebra subprograms by Choi and Dongarra 


\par
 This module supports the ''Myriad'' style interface, allowing many independent problems to be worked on at once. The A matrix can contain multiple matrixes to be factored, indicated by different values for work-item id (wi\_id).



\par
\begin{description}
\item [\colorbox{tagtype}{\color{white} \textbf{\textsf{PARAMETER}}}] \textbf{\underline{tri}} Types.Triangle enumeration indicating whether we are looking for the Upper or the Lower factor
\item [\colorbox{tagtype}{\color{white} \textbf{\textsf{PARAMETER}}}] \textbf{\underline{A\_in}} The matrix or matrixes to be factored in Types.Layout\_Cell format
\item [\colorbox{tagtype}{\color{white} \textbf{\textsf{RETURN}}}] \textbf{\underline{}} Triangular matrix in Layout\_Cell format
\item [\colorbox{tagtype}{\color{white} \textbf{\textsf{SEE}}}] \textbf{\underline{}} Std.PBblas.Types.Layout\_Cell
\item [\colorbox{tagtype}{\color{white} \textbf{\textsf{SEE}}}] \textbf{\underline{}} Std.PBblas.Types.Triangle
\end{description}

\rule{\linewidth}{0.5pt}

\chapter*{PBblas.scal}
\hypertarget{ecldoc:toc:PBblas.scal}{}

\section*{\underline{IMPORTS}}
\begin{itemize}
\item PBblas
\item PBblas.Types
\end{itemize}

\section*{\underline{DESCRIPTIONS}}
\subsection*{FUNCTION : scal}
\hypertarget{ecldoc:pbblas.scal}{}
\hyperlink{ecldoc:toc:PBblas}{Up} :

{\renewcommand{\arraystretch}{1.5}
\begin{tabularx}{\textwidth}{|>{\raggedright\arraybackslash}l|X|}
\hline
\hspace{0pt}DATASET(Layout\_Cell) & scal \\
\hline
\multicolumn{2}{|>{\raggedright\arraybackslash}X|}{\hspace{0pt}(value\_t alpha, DATASET(Layout\_Cell) X)} \\
\hline
\end{tabularx}
}

\par
Scale a matrix by a constant Result is alpha * X This supports a ''myriad'' style interface in that X may be a set of independent matrices separated by different work-item ids.

\par
\begin{description}
\item [\textbf{Parameter}] alpha ||| A scalar multiplier
\item [\textbf{Parameter}] X ||| The matrix(es) to be scaled in Layout\_Cell format
\item [\textbf{Return}] Matrix in Layout\_Cell form, of the same shape as X
\item [\textbf{See}] PBblas/Types.Layout\_Cell
\end{description}

\rule{\linewidth}{0.5pt}

\chapter*{\color{headfile}
{\large PBblas\slash\hspace{0pt}}
 \\
tran
}
\hypertarget{ecldoc:toc:PBblas.tran}{}
\hyperlink{ecldoc:toc:root/PBblas}{Go Up}

\section*{\underline{\textsf{IMPORTS}}}
\begin{doublespace}
{\large
PBblas |
PBblas.Types |
PBblas.internal |
PBblas.internal.Types |
PBblas.internal.MatDims |
PBblas.internal.Converted |
std.blas |
std.system.Thorlib |
}
\end{doublespace}

\section*{\underline{\textsf{DESCRIPTIONS}}}
\subsection*{\textsf{\colorbox{headtoc}{\color{white} FUNCTION}
tran}}

\hypertarget{ecldoc:pbblas.tran}{}

{\renewcommand{\arraystretch}{1.5}
\begin{tabularx}{\textwidth}{|>{\raggedright\arraybackslash}l|X|}
\hline
\hspace{0pt}\mytexttt{\color{red} DATASET(Layout\_Cell)} & \textbf{tran} \\
\hline
\multicolumn{2}{|>{\raggedright\arraybackslash}X|}{\hspace{0pt}\mytexttt{\color{param} (value\_t alpha, DATASET(Layout\_Cell) A, value\_t beta=0, DATASET(Layout\_Cell) C=empty\_c)}} \\
\hline
\end{tabularx}
}

\par
Transpose a matrix and sum into base matrix result <== alpha * A**t + beta * C, A is n by m, C is m by n A**T (A Transpose) and C must have same shape

\par
\begin{description}
\item [\colorbox{tagtype}{\color{white} \textbf{\textsf{PARAMETER}}}] \textbf{\underline{alpha}} Scalar multiplier for the A**T matrix
\item [\colorbox{tagtype}{\color{white} \textbf{\textsf{PARAMETER}}}] \textbf{\underline{A}} A matrix in DATASET(Layout\_Cell) form
\item [\colorbox{tagtype}{\color{white} \textbf{\textsf{PARAMETER}}}] \textbf{\underline{beta}} Scalar multiplier for the C matrix
\item [\colorbox{tagtype}{\color{white} \textbf{\textsf{PARAMETER}}}] \textbf{\underline{C}} C matrix in DATASET(Layout\_Call) form
\item [\colorbox{tagtype}{\color{white} \textbf{\textsf{RETURN}}}] \textbf{\underline{}} Matrix in DATASET(Layout\_Cell) form alpha * A**T + beta * C
\item [\colorbox{tagtype}{\color{white} \textbf{\textsf{SEE}}}] \textbf{\underline{}} PBblas/Types.layout\_cell
\end{description}

\rule{\linewidth}{0.5pt}

\chapter*{PBblas.trsm}
\hypertarget{ecldoc:toc:PBblas.trsm}{}

\section*{\underline{IMPORTS}}
\begin{itemize}
\item PBblas
\item PBblas.Types
\item std.BLAS
\item PBblas.internal
\item PBblas.internal.Types
\item PBblas.internal.MatDims
\item PBblas.internal.Converted
\item std.system.Thorlib
\end{itemize}

\section*{\underline{DESCRIPTIONS}}
\subsection*{FUNCTION : trsm}
\hypertarget{ecldoc:pbblas.trsm}{}
\begin{minipage}[t]{\textwidth}
\begin{flushleft}
DATASET(Layout\_Cell) trsm (Side s, Triangle tri, BOOLEAN transposeA, Diagonal diag, value\_t alpha, DATASET(Layout\_Cell) A\_in, DATASET(Layout\_Cell) B\_in)
\end{flushleft}
\end{minipage}
\hyperlink{ecldoc:toc:PBblas}{Up}

\par
Partitioned block parallel triangular matrix solver. Solves for X using: AX = B or XA = B A is is a square triangular matrix, X and B have the same dimensions. A may be an upper triangular matrix (UX = B or XU = B), or a lower triangular matrix (LX = B or XL = B). Allows optional transposing and scaling of A. Partially based upon an approach discussed by MJ DAYDE, IS DUFF, AP CERFACS. A Parallel Block implementation of Level-3 BLAS for MIMD Vector Processors ACM Tran. Mathematical Software, Vol 20, No 2, June 1994 pp 178-193 and other papers about PB-BLAS by Choi and Dongarra This module supports the ''Myriad'' style interface, allowing many independent problems to be worked on at once. Corresponding A and B matrixes are related by a common work-item identifier (wi\_id) within each cell of the matrix. The returned X matrix will contain cells for the same set of work-items as specified for the A and B matrices.
\par
\textbf{Parameter} : s ||| Types.Side enumeration indicating whether we are solving AX = B or XA = B \\
\textbf{Parameter} : tri ||| Types.Triangle enumeration indicating whether we are solving an Upper or Lower triangle. \\
\textbf{Parameter} : transposeA ||| Boolean indicating whether or not to transpose the A matrix before solving \\
\textbf{Parameter} : diag ||| Types.Diagonal enumeration indicating whether A is a unit matrix or not. This is primarily used after factoring matrixes using getrf (LU factorization). That module produces a factored matrix stored within the same space as the original matrix. Since the diagonal is used by both factors, by convention, the Lower triangle has a unit matrix (diagonal all 1's) while the Upper triangle uses the diagonal cells. Setting this to UnitTri, causes the contents of the diagonal to be ignored, and assumed to be 1. NotUnitTri should be used for most other cases. \\
\textbf{Parameter} : alpha ||| Multiplier to scale A \\
\textbf{Parameter} : A\_in ||| The A matrix in Layout\_Cell format \\
\textbf{Parameter} : B\_in ||| The B matrix in Layout\_Cell format \\
\textbf{Return} : X solution matrix in Layout\_Cell format \\
\textbf{See} : Types.Layout\_Cell \\
\textbf{See} : Types.Triangle \\
\textbf{See} : Types.Side \\

\chapter*{\color{headfile}
{\large ML\_Core\slash\hspace{0pt}}
 \\
Types
}
\hypertarget{ecldoc:toc:ML_Core.Types}{}
\hyperlink{ecldoc:toc:root/ML_Core}{Go Up}


\section*{\underline{\textsf{DESCRIPTIONS}}}
\subsection*{\textsf{\colorbox{headtoc}{\color{white} MODULE}
Types}}

\hypertarget{ecldoc:ML_Core.Types}{}

{\renewcommand{\arraystretch}{1.5}
\begin{tabularx}{\textwidth}{|>{\raggedright\arraybackslash}l|X|}
\hline
\hspace{0pt}\mytexttt{\color{red} } & \textbf{Types} \\
\hline
\end{tabularx}
}

\par


\textbf{Children}
\begin{enumerate}
\item \hyperlink{ecldoc:ml_core.types.t_recordid}{t\_RecordID}
\item \hyperlink{ecldoc:ml_core.types.t_fieldnumber}{t\_FieldNumber}
\item \hyperlink{ecldoc:ml_core.types.t_fieldreal}{t\_FieldReal}
\item \hyperlink{ecldoc:ml_core.types.t_fieldsign}{t\_FieldSign}
\item \hyperlink{ecldoc:ml_core.types.t_discrete}{t\_Discrete}
\item \hyperlink{ecldoc:ml_core.types.t_item}{t\_Item}
\item \hyperlink{ecldoc:ml_core.types.t_count}{t\_Count}
\item \hyperlink{ecldoc:ml_core.types.t_work_item}{t\_Work\_Item}
\item \hyperlink{ecldoc:ml_core.types.anyfield}{AnyField}
\item \hyperlink{ecldoc:ml_core.types.numericfield}{NumericField}
\item \hyperlink{ecldoc:ml_core.types.discretefield}{DiscreteField}
\item \hyperlink{ecldoc:ml_core.types.layout_model}{Layout\_Model}
\item \hyperlink{ecldoc:ml_core.types.classify_result}{Classify\_Result}
\item \hyperlink{ecldoc:ml_core.types.l_result}{l\_result}
\item \hyperlink{ecldoc:ml_core.types.confusion_detail}{Confusion\_Detail}
\item \hyperlink{ecldoc:ml_core.types.itemelement}{ItemElement}
\item \hyperlink{ecldoc:ml_core.types.t_node}{t\_node}
\item \hyperlink{ecldoc:ml_core.types.t_level}{t\_level}
\item \hyperlink{ecldoc:ml_core.types.nodeid}{NodeID}
\end{enumerate}

\rule{\linewidth}{0.5pt}

\subsection*{\textsf{\colorbox{headtoc}{\color{white} ATTRIBUTE}
t\_RecordID}}

\hypertarget{ecldoc:ml_core.types.t_recordid}{}
\hspace{0pt} \hyperlink{ecldoc:ML_Core.Types}{Types} \textbackslash 

{\renewcommand{\arraystretch}{1.5}
\begin{tabularx}{\textwidth}{|>{\raggedright\arraybackslash}l|X|}
\hline
\hspace{0pt}\mytexttt{\color{red} } & \textbf{t\_RecordID} \\
\hline
\end{tabularx}
}

\par


\rule{\linewidth}{0.5pt}
\subsection*{\textsf{\colorbox{headtoc}{\color{white} ATTRIBUTE}
t\_FieldNumber}}

\hypertarget{ecldoc:ml_core.types.t_fieldnumber}{}
\hspace{0pt} \hyperlink{ecldoc:ML_Core.Types}{Types} \textbackslash 

{\renewcommand{\arraystretch}{1.5}
\begin{tabularx}{\textwidth}{|>{\raggedright\arraybackslash}l|X|}
\hline
\hspace{0pt}\mytexttt{\color{red} } & \textbf{t\_FieldNumber} \\
\hline
\end{tabularx}
}

\par


\rule{\linewidth}{0.5pt}
\subsection*{\textsf{\colorbox{headtoc}{\color{white} ATTRIBUTE}
t\_FieldReal}}

\hypertarget{ecldoc:ml_core.types.t_fieldreal}{}
\hspace{0pt} \hyperlink{ecldoc:ML_Core.Types}{Types} \textbackslash 

{\renewcommand{\arraystretch}{1.5}
\begin{tabularx}{\textwidth}{|>{\raggedright\arraybackslash}l|X|}
\hline
\hspace{0pt}\mytexttt{\color{red} } & \textbf{t\_FieldReal} \\
\hline
\end{tabularx}
}

\par


\rule{\linewidth}{0.5pt}
\subsection*{\textsf{\colorbox{headtoc}{\color{white} ATTRIBUTE}
t\_FieldSign}}

\hypertarget{ecldoc:ml_core.types.t_fieldsign}{}
\hspace{0pt} \hyperlink{ecldoc:ML_Core.Types}{Types} \textbackslash 

{\renewcommand{\arraystretch}{1.5}
\begin{tabularx}{\textwidth}{|>{\raggedright\arraybackslash}l|X|}
\hline
\hspace{0pt}\mytexttt{\color{red} } & \textbf{t\_FieldSign} \\
\hline
\end{tabularx}
}

\par


\rule{\linewidth}{0.5pt}
\subsection*{\textsf{\colorbox{headtoc}{\color{white} ATTRIBUTE}
t\_Discrete}}

\hypertarget{ecldoc:ml_core.types.t_discrete}{}
\hspace{0pt} \hyperlink{ecldoc:ML_Core.Types}{Types} \textbackslash 

{\renewcommand{\arraystretch}{1.5}
\begin{tabularx}{\textwidth}{|>{\raggedright\arraybackslash}l|X|}
\hline
\hspace{0pt}\mytexttt{\color{red} } & \textbf{t\_Discrete} \\
\hline
\end{tabularx}
}

\par


\rule{\linewidth}{0.5pt}
\subsection*{\textsf{\colorbox{headtoc}{\color{white} ATTRIBUTE}
t\_Item}}

\hypertarget{ecldoc:ml_core.types.t_item}{}
\hspace{0pt} \hyperlink{ecldoc:ML_Core.Types}{Types} \textbackslash 

{\renewcommand{\arraystretch}{1.5}
\begin{tabularx}{\textwidth}{|>{\raggedright\arraybackslash}l|X|}
\hline
\hspace{0pt}\mytexttt{\color{red} } & \textbf{t\_Item} \\
\hline
\end{tabularx}
}

\par


\rule{\linewidth}{0.5pt}
\subsection*{\textsf{\colorbox{headtoc}{\color{white} ATTRIBUTE}
t\_Count}}

\hypertarget{ecldoc:ml_core.types.t_count}{}
\hspace{0pt} \hyperlink{ecldoc:ML_Core.Types}{Types} \textbackslash 

{\renewcommand{\arraystretch}{1.5}
\begin{tabularx}{\textwidth}{|>{\raggedright\arraybackslash}l|X|}
\hline
\hspace{0pt}\mytexttt{\color{red} } & \textbf{t\_Count} \\
\hline
\end{tabularx}
}

\par


\rule{\linewidth}{0.5pt}
\subsection*{\textsf{\colorbox{headtoc}{\color{white} ATTRIBUTE}
t\_Work\_Item}}

\hypertarget{ecldoc:ml_core.types.t_work_item}{}
\hspace{0pt} \hyperlink{ecldoc:ML_Core.Types}{Types} \textbackslash 

{\renewcommand{\arraystretch}{1.5}
\begin{tabularx}{\textwidth}{|>{\raggedright\arraybackslash}l|X|}
\hline
\hspace{0pt}\mytexttt{\color{red} } & \textbf{t\_Work\_Item} \\
\hline
\end{tabularx}
}

\par


\rule{\linewidth}{0.5pt}
\subsection*{\textsf{\colorbox{headtoc}{\color{white} RECORD}
AnyField}}

\hypertarget{ecldoc:ml_core.types.anyfield}{}
\hspace{0pt} \hyperlink{ecldoc:ML_Core.Types}{Types} \textbackslash 

{\renewcommand{\arraystretch}{1.5}
\begin{tabularx}{\textwidth}{|>{\raggedright\arraybackslash}l|X|}
\hline
\hspace{0pt}\mytexttt{\color{red} } & \textbf{AnyField} \\
\hline
\end{tabularx}
}

\par


\rule{\linewidth}{0.5pt}
\subsection*{\textsf{\colorbox{headtoc}{\color{white} RECORD}
NumericField}}

\hypertarget{ecldoc:ml_core.types.numericfield}{}
\hspace{0pt} \hyperlink{ecldoc:ML_Core.Types}{Types} \textbackslash 

{\renewcommand{\arraystretch}{1.5}
\begin{tabularx}{\textwidth}{|>{\raggedright\arraybackslash}l|X|}
\hline
\hspace{0pt}\mytexttt{\color{red} } & \textbf{NumericField} \\
\hline
\end{tabularx}
}

\par


\rule{\linewidth}{0.5pt}
\subsection*{\textsf{\colorbox{headtoc}{\color{white} RECORD}
DiscreteField}}

\hypertarget{ecldoc:ml_core.types.discretefield}{}
\hspace{0pt} \hyperlink{ecldoc:ML_Core.Types}{Types} \textbackslash 

{\renewcommand{\arraystretch}{1.5}
\begin{tabularx}{\textwidth}{|>{\raggedright\arraybackslash}l|X|}
\hline
\hspace{0pt}\mytexttt{\color{red} } & \textbf{DiscreteField} \\
\hline
\end{tabularx}
}

\par


\rule{\linewidth}{0.5pt}
\subsection*{\textsf{\colorbox{headtoc}{\color{white} RECORD}
Layout\_Model}}

\hypertarget{ecldoc:ml_core.types.layout_model}{}
\hspace{0pt} \hyperlink{ecldoc:ML_Core.Types}{Types} \textbackslash 

{\renewcommand{\arraystretch}{1.5}
\begin{tabularx}{\textwidth}{|>{\raggedright\arraybackslash}l|X|}
\hline
\hspace{0pt}\mytexttt{\color{red} } & \textbf{Layout\_Model} \\
\hline
\end{tabularx}
}

\par


\rule{\linewidth}{0.5pt}
\subsection*{\textsf{\colorbox{headtoc}{\color{white} RECORD}
Classify\_Result}}

\hypertarget{ecldoc:ml_core.types.classify_result}{}
\hspace{0pt} \hyperlink{ecldoc:ML_Core.Types}{Types} \textbackslash 

{\renewcommand{\arraystretch}{1.5}
\begin{tabularx}{\textwidth}{|>{\raggedright\arraybackslash}l|X|}
\hline
\hspace{0pt}\mytexttt{\color{red} } & \textbf{Classify\_Result} \\
\hline
\end{tabularx}
}

\par


\rule{\linewidth}{0.5pt}
\subsection*{\textsf{\colorbox{headtoc}{\color{white} RECORD}
l\_result}}

\hypertarget{ecldoc:ml_core.types.l_result}{}
\hspace{0pt} \hyperlink{ecldoc:ML_Core.Types}{Types} \textbackslash 

{\renewcommand{\arraystretch}{1.5}
\begin{tabularx}{\textwidth}{|>{\raggedright\arraybackslash}l|X|}
\hline
\hspace{0pt}\mytexttt{\color{red} } & \textbf{l\_result} \\
\hline
\end{tabularx}
}

\par


\rule{\linewidth}{0.5pt}
\subsection*{\textsf{\colorbox{headtoc}{\color{white} RECORD}
Confusion\_Detail}}

\hypertarget{ecldoc:ml_core.types.confusion_detail}{}
\hspace{0pt} \hyperlink{ecldoc:ML_Core.Types}{Types} \textbackslash 

{\renewcommand{\arraystretch}{1.5}
\begin{tabularx}{\textwidth}{|>{\raggedright\arraybackslash}l|X|}
\hline
\hspace{0pt}\mytexttt{\color{red} } & \textbf{Confusion\_Detail} \\
\hline
\end{tabularx}
}

\par


\rule{\linewidth}{0.5pt}
\subsection*{\textsf{\colorbox{headtoc}{\color{white} RECORD}
ItemElement}}

\hypertarget{ecldoc:ml_core.types.itemelement}{}
\hspace{0pt} \hyperlink{ecldoc:ML_Core.Types}{Types} \textbackslash 

{\renewcommand{\arraystretch}{1.5}
\begin{tabularx}{\textwidth}{|>{\raggedright\arraybackslash}l|X|}
\hline
\hspace{0pt}\mytexttt{\color{red} } & \textbf{ItemElement} \\
\hline
\end{tabularx}
}

\par


\rule{\linewidth}{0.5pt}
\subsection*{\textsf{\colorbox{headtoc}{\color{white} ATTRIBUTE}
t\_node}}

\hypertarget{ecldoc:ml_core.types.t_node}{}
\hspace{0pt} \hyperlink{ecldoc:ML_Core.Types}{Types} \textbackslash 

{\renewcommand{\arraystretch}{1.5}
\begin{tabularx}{\textwidth}{|>{\raggedright\arraybackslash}l|X|}
\hline
\hspace{0pt}\mytexttt{\color{red} } & \textbf{t\_node} \\
\hline
\end{tabularx}
}

\par


\rule{\linewidth}{0.5pt}
\subsection*{\textsf{\colorbox{headtoc}{\color{white} ATTRIBUTE}
t\_level}}

\hypertarget{ecldoc:ml_core.types.t_level}{}
\hspace{0pt} \hyperlink{ecldoc:ML_Core.Types}{Types} \textbackslash 

{\renewcommand{\arraystretch}{1.5}
\begin{tabularx}{\textwidth}{|>{\raggedright\arraybackslash}l|X|}
\hline
\hspace{0pt}\mytexttt{\color{red} } & \textbf{t\_level} \\
\hline
\end{tabularx}
}

\par


\rule{\linewidth}{0.5pt}
\subsection*{\textsf{\colorbox{headtoc}{\color{white} RECORD}
NodeID}}

\hypertarget{ecldoc:ml_core.types.nodeid}{}
\hspace{0pt} \hyperlink{ecldoc:ML_Core.Types}{Types} \textbackslash 

{\renewcommand{\arraystretch}{1.5}
\begin{tabularx}{\textwidth}{|>{\raggedright\arraybackslash}l|X|}
\hline
\hspace{0pt}\mytexttt{\color{red} } & \textbf{NodeID} \\
\hline
\end{tabularx}
}

\par


\rule{\linewidth}{0.5pt}



\chapter*{\color{headfile}
{\large PBblas\slash\hspace{0pt}}
 \\
Vector2Diag
}
\hypertarget{ecldoc:toc:PBblas.Vector2Diag}{}
\hyperlink{ecldoc:toc:root/PBblas}{Go Up}

\section*{\underline{\textsf{IMPORTS}}}
\begin{doublespace}
{\large
PBblas |
PBblas.internal |
PBblas.internal.MatDims |
PBblas.Types |
PBblas.internal.Types |
PBblas.Constants |
}
\end{doublespace}

\section*{\underline{\textsf{DESCRIPTIONS}}}
\subsection*{\textsf{\colorbox{headtoc}{\color{white} FUNCTION}
Vector2Diag}}

\hypertarget{ecldoc:pbblas.vector2diag}{}

{\renewcommand{\arraystretch}{1.5}
\begin{tabularx}{\textwidth}{|>{\raggedright\arraybackslash}l|X|}
\hline
\hspace{0pt}\mytexttt{\color{red} DATASET(Layout\_Cell)} & \textbf{Vector2Diag} \\
\hline
\multicolumn{2}{|>{\raggedright\arraybackslash}X|}{\hspace{0pt}\mytexttt{\color{param} (DATASET(Layout\_Cell) X)}} \\
\hline
\end{tabularx}
}

\par
Convert a vector into a diagonal matrix. The typical notation is D = diag(V). The input X must be a 1 x N column vector or an N x 1 row vector. The resulting matrix, in either case will be N x N, with zero everywhere except the diagonal.

\par
\begin{description}
\item [\colorbox{tagtype}{\color{white} \textbf{\textsf{PARAMETER}}}] \textbf{\underline{X}} A row or column vector (i.e. N x 1 or 1 x N) in Layout\_Cell format
\item [\colorbox{tagtype}{\color{white} \textbf{\textsf{RETURN}}}] \textbf{\underline{}} An N x N matrix in Layout\_Cell format
\item [\colorbox{tagtype}{\color{white} \textbf{\textsf{SEE}}}] \textbf{\underline{}} PBblas/Types.Layout\_cell
\end{description}

\rule{\linewidth}{0.5pt}

