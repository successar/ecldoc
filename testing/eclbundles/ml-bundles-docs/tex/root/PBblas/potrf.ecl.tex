\chapter*{\color{headfile}
{\large PBblas\slash\hspace{0pt}}
 \\
potrf
}
\hypertarget{ecldoc:toc:PBblas.potrf}{}
\hyperlink{ecldoc:toc:root/PBblas}{Go Up}

\section*{\underline{\textsf{IMPORTS}}}
\begin{doublespace}
{\large
PBblas |
PBblas.Types |
std.blas |
PBblas.internal |
PBblas.internal.Types |
PBblas.internal.MatDims |
PBblas.internal.Converted |
std.system.Thorlib |
}
\end{doublespace}

\section*{\underline{\textsf{DESCRIPTIONS}}}
\subsection*{\textsf{\colorbox{headtoc}{\color{white} FUNCTION}
potrf}}

\hypertarget{ecldoc:pbblas.potrf}{}

{\renewcommand{\arraystretch}{1.5}
\begin{tabularx}{\textwidth}{|>{\raggedright\arraybackslash}l|X|}
\hline
\hspace{0pt}\mytexttt{\color{red} DATASET(Layout\_Cell)} & \textbf{potrf} \\
\hline
\multicolumn{2}{|>{\raggedright\arraybackslash}X|}{\hspace{0pt}\mytexttt{\color{param} (Triangle tri, DATASET(Layout\_Cell) A\_in)}} \\
\hline
\end{tabularx}
}

\par





Implements Cholesky factorization of A = U**T * U if Triangular.Upper requested or A = L * L**T if Triangualr.Lower is requested. The matrix A must be symmetric positive definite. 
\begin{verbatim}

  | A11   A12 |      |  L11   0   |    | L11**T     L21**T |
  | A21   A22 |  ==  |  L21   L22 | *  |  0           L22  |

                     | L11*L11**T          L11*L21**T      |
                 ==  | L21*L11**T  L21*L21**T + L22*L22**T |
 \end{verbatim}

 So, use Cholesky on the first block to get L11. L21 = A21*L11**T**-1 which can be found by dtrsm on each column block A22' is A22 - L21*L21**T 
\par
 Based upon PB-BLAS: A set of parallel block basic linear algebra subprograms by Choi and Dongarra 


\par
 This module supports the ''Myriad'' style interface, allowing many independent problems to be worked on at once. The A matrix can contain multiple matrixes to be factored, indicated by different values for work-item id (wi\_id).








\par
\begin{description}
\item [\colorbox{tagtype}{\color{white} \textbf{\textsf{PARAMETER}}}] \textbf{\underline{tri}} ||| UNSIGNED1 --- Types.Triangle enumeration indicating whether we are looking for the Upper or the Lower factor
\item [\colorbox{tagtype}{\color{white} \textbf{\textsf{PARAMETER}}}] \textbf{\underline{A\_in}} ||| TABLE ( Layout\_Cell ) --- The matrix or matrixes to be factored in Types.Layout\_Cell format
\end{description}







\par
\begin{description}
\item [\colorbox{tagtype}{\color{white} \textbf{\textsf{RETURN}}}] \textbf{TABLE ( \{ UNSIGNED2 wi\_id , UNSIGNED4 x , UNSIGNED4 y , REAL8 v \} )} --- Triangular matrix in Layout\_Cell format
\end{description}







\par
\begin{description}
\item [\colorbox{tagtype}{\color{white} \textbf{\textsf{SEE}}}] Std.PBblas.Types.Layout\_Cell
\item [\colorbox{tagtype}{\color{white} \textbf{\textsf{SEE}}}] Std.PBblas.Types.Triangle
\end{description}



\rule{\linewidth}{0.5pt}
