\chapter*{\color{headfile}
{\large PBblas\slash\hspace{0pt}}
 \\
ExtractTri
}
\hypertarget{ecldoc:toc:PBblas.ExtractTri}{}
\hyperlink{ecldoc:toc:root/PBblas}{Go Up}

\section*{\underline{\textsf{IMPORTS}}}
\begin{doublespace}
{\large
PBblas |
std.blas |
PBblas.Types |
PBblas.internal |
PBblas.internal.Types |
PBblas.internal.MatDims |
PBblas.internal.Converted |
}
\end{doublespace}

\section*{\underline{\textsf{DESCRIPTIONS}}}
\subsection*{\textsf{\colorbox{headtoc}{\color{white} FUNCTION}
ExtractTri}}

\hypertarget{ecldoc:pbblas.extracttri}{}

{\renewcommand{\arraystretch}{1.5}
\begin{tabularx}{\textwidth}{|>{\raggedright\arraybackslash}l|X|}
\hline
\hspace{0pt}\mytexttt{\color{red} DATASET(Layout\_Cell)} & \textbf{ExtractTri} \\
\hline
\multicolumn{2}{|>{\raggedright\arraybackslash}X|}{\hspace{0pt}\mytexttt{\color{param} (Triangle tri, Diagonal dt, DATASET(Layout\_Cell) A)}} \\
\hline
\end{tabularx}
}

\par





Extract the upper or lower triangle from the composite output from getrf (LU Factorization).






\par
\begin{description}
\item [\colorbox{tagtype}{\color{white} \textbf{\textsf{PARAMETER}}}] \textbf{\underline{tri}} ||| UNSIGNED1 --- Triangle type: Upper or Lower (see Types.Triangle)
\item [\colorbox{tagtype}{\color{white} \textbf{\textsf{PARAMETER}}}] \textbf{\underline{A}} ||| TABLE ( Layout\_Cell ) --- Matrix of cells. See Types.Layout\_Cell
\item [\colorbox{tagtype}{\color{white} \textbf{\textsf{PARAMETER}}}] \textbf{\underline{dt}} ||| UNSIGNED1 --- Diagonal type: Unit or non unit (see Types.Diagonal)
\end{description}







\par
\begin{description}
\item [\colorbox{tagtype}{\color{white} \textbf{\textsf{RETURN}}}] \textbf{TABLE ( \{ UNSIGNED2 wi\_id , UNSIGNED4 x , UNSIGNED4 y , REAL8 v \} )} --- Matrix of cells in Layout\_Cell format representing a triangular matrix (upper or lower)
\end{description}







\par
\begin{description}
\item [\colorbox{tagtype}{\color{white} \textbf{\textsf{SEE}}}] Std.PBblas.Types
\end{description}



\rule{\linewidth}{0.5pt}
