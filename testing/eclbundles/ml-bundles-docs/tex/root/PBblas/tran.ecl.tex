\chapter*{\color{headfile}
{\large PBblas\slash\hspace{0pt}}
 \\
tran
}
\hypertarget{ecldoc:toc:PBblas.tran}{}
\hyperlink{ecldoc:toc:root/PBblas}{Go Up}

\section*{\underline{\textsf{IMPORTS}}}
\begin{doublespace}
{\large
PBblas |
PBblas.Types |
PBblas.internal |
PBblas.internal.Types |
PBblas.internal.MatDims |
PBblas.internal.Converted |
std.blas |
std.system.Thorlib |
}
\end{doublespace}

\section*{\underline{\textsf{DESCRIPTIONS}}}
\subsection*{\textsf{\colorbox{headtoc}{\color{white} FUNCTION}
tran}}

\hypertarget{ecldoc:pbblas.tran}{}

{\renewcommand{\arraystretch}{1.5}
\begin{tabularx}{\textwidth}{|>{\raggedright\arraybackslash}l|X|}
\hline
\hspace{0pt}\mytexttt{\color{red} DATASET(Layout\_Cell)} & \textbf{tran} \\
\hline
\multicolumn{2}{|>{\raggedright\arraybackslash}X|}{\hspace{0pt}\mytexttt{\color{param} (value\_t alpha, DATASET(Layout\_Cell) A, value\_t beta=0, DATASET(Layout\_Cell) C=empty\_c)}} \\
\hline
\end{tabularx}
}

\par
Transpose a matrix and sum into base matrix result <== alpha * A**t + beta * C, A is n by m, C is m by n A**T (A Transpose) and C must have same shape

\par
\begin{description}
\item [\colorbox{tagtype}{\color{white} \textbf{\textsf{PARAMETER}}}] \textbf{\underline{alpha}} Scalar multiplier for the A**T matrix
\item [\colorbox{tagtype}{\color{white} \textbf{\textsf{PARAMETER}}}] \textbf{\underline{A}} A matrix in DATASET(Layout\_Cell) form
\item [\colorbox{tagtype}{\color{white} \textbf{\textsf{PARAMETER}}}] \textbf{\underline{beta}} Scalar multiplier for the C matrix
\item [\colorbox{tagtype}{\color{white} \textbf{\textsf{PARAMETER}}}] \textbf{\underline{C}} C matrix in DATASET(Layout\_Call) form
\item [\colorbox{tagtype}{\color{white} \textbf{\textsf{RETURN}}}] \textbf{\underline{}} Matrix in DATASET(Layout\_Cell) form alpha * A**T + beta * C
\item [\colorbox{tagtype}{\color{white} \textbf{\textsf{SEE}}}] \textbf{\underline{}} PBblas/Types.layout\_cell
\end{description}

\rule{\linewidth}{0.5pt}
