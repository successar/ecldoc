\chapter*{\color{headfile}
{\large PBblas\slash\hspace{0pt}}
 \\
scal
}
\hypertarget{ecldoc:toc:PBblas.scal}{}
\hyperlink{ecldoc:toc:root/PBblas}{Go Up}

\section*{\underline{\textsf{IMPORTS}}}
\begin{doublespace}
{\large
PBblas |
PBblas.Types |
}
\end{doublespace}

\section*{\underline{\textsf{DESCRIPTIONS}}}
\subsection*{\textsf{\colorbox{headtoc}{\color{white} FUNCTION}
scal}}

\hypertarget{ecldoc:pbblas.scal}{}

{\renewcommand{\arraystretch}{1.5}
\begin{tabularx}{\textwidth}{|>{\raggedright\arraybackslash}l|X|}
\hline
\hspace{0pt}\mytexttt{\color{red} DATASET(Layout\_Cell)} & \textbf{scal} \\
\hline
\multicolumn{2}{|>{\raggedright\arraybackslash}X|}{\hspace{0pt}\mytexttt{\color{param} (value\_t alpha, DATASET(Layout\_Cell) X)}} \\
\hline
\end{tabularx}
}

\par





Scale a matrix by a constant Result is alpha * X This supports a ''myriad'' style interface in that X may be a set of independent matrices separated by different work-item ids.






\par
\begin{description}
\item [\colorbox{tagtype}{\color{white} \textbf{\textsf{PARAMETER}}}] \textbf{\underline{alpha}} ||| REAL8 --- A scalar multiplier
\item [\colorbox{tagtype}{\color{white} \textbf{\textsf{PARAMETER}}}] \textbf{\underline{X}} ||| TABLE ( Layout\_Cell ) --- The matrix(es) to be scaled in Layout\_Cell format
\end{description}







\par
\begin{description}
\item [\colorbox{tagtype}{\color{white} \textbf{\textsf{RETURN}}}] \textbf{TABLE ( \{ UNSIGNED2 wi\_id , UNSIGNED4 x , UNSIGNED4 y , REAL8 v \} )} --- Matrix in Layout\_Cell form, of the same shape as X
\end{description}







\par
\begin{description}
\item [\colorbox{tagtype}{\color{white} \textbf{\textsf{SEE}}}] PBblas/Types.Layout\_Cell
\end{description}



\rule{\linewidth}{0.5pt}
