\chapter*{\color{headfile}
{\large LogisticRegression\slash\hspace{0pt}}
 \\
Null_Deviance
}
\hypertarget{ecldoc:toc:LogisticRegression.Null_Deviance}{}
\hyperlink{ecldoc:toc:root/LogisticRegression}{Go Up}

\section*{\underline{\textsf{IMPORTS}}}
\begin{doublespace}
{\large
LogisticRegression |
LogisticRegression.Types |
}
\end{doublespace}

\section*{\underline{\textsf{DESCRIPTIONS}}}
\subsection*{\textsf{\colorbox{headtoc}{\color{white} FUNCTION}
Null\_Deviance}}

\hypertarget{ecldoc:logisticregression.null_deviance}{}

{\renewcommand{\arraystretch}{1.5}
\begin{tabularx}{\textwidth}{|>{\raggedright\arraybackslash}l|X|}
\hline
\hspace{0pt}\mytexttt{\color{red} DATASET(Types.Deviance\_Record)} & \textbf{Null\_Deviance} \\
\hline
\multicolumn{2}{|>{\raggedright\arraybackslash}X|}{\hspace{0pt}\mytexttt{\color{param} (DATASET(Types.Observation\_Deviance) od)}} \\
\hline
\end{tabularx}
}

\par
Deviance for the null model, that is, a model with only an intercept.

\par
\begin{description}
\item [\colorbox{tagtype}{\color{white} \textbf{\textsf{PARAMETER}}}] \textbf{\underline{od}} Observation Deviance record set.
\item [\colorbox{tagtype}{\color{white} \textbf{\textsf{RETURN}}}] \textbf{\underline{}} a data set of the null model deviances for each work item and classifier.
\end{description}

\rule{\linewidth}{0.5pt}
