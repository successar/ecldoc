\chapter*{\color{headfile}
{\large LogisticRegression\slash\hspace{0pt}}
{\large IRLS\slash\hspace{0pt}}
 \\
GetModel
}
\hypertarget{ecldoc:toc:LogisticRegression.IRLS.GetModel}{}
\hyperlink{ecldoc:toc:root/LogisticRegression/IRLS}{Go Up}

\section*{\underline{\textsf{IMPORTS}}}
\begin{doublespace}
{\large
ML\_Core |
ML\_Core.Types |
LogisticRegression |
LogisticRegression.Constants |
LogisticRegression.Types |
LogisticRegression.IRLS |
}
\end{doublespace}

\section*{\underline{\textsf{DESCRIPTIONS}}}
\subsection*{\textsf{\colorbox{headtoc}{\color{white} FUNCTION}
GetModel}}

\hypertarget{ecldoc:logisticregression.irls.getmodel}{}

{\renewcommand{\arraystretch}{1.5}
\begin{tabularx}{\textwidth}{|>{\raggedright\arraybackslash}l|X|}
\hline
\hspace{0pt}\mytexttt{\color{red} DATASET(Layout\_Model)} & \textbf{GetModel} \\
\hline
\multicolumn{2}{|>{\raggedright\arraybackslash}X|}{\hspace{0pt}\mytexttt{\color{param} (DATASET(NumericField) independents, DATASET(DiscreteField) dependents, UNSIGNED max\_iter=200, REAL8 epsilon=Constants.default\_epsilon, REAL8 ridge=Constants.default\_ridge)}} \\
\hline
\end{tabularx}
}

\par
Generate logistic regression model from training data. The size of the inputs is used to determin which work items are processed with purely local operations (the data is moved once as necessary) or with global operations supporting a work item to use multiple nodes.

\par
\begin{description}
\item [\colorbox{tagtype}{\color{white} \textbf{\textsf{PARAMETER}}}] \textbf{\underline{independents}} the independent values
\item [\colorbox{tagtype}{\color{white} \textbf{\textsf{PARAMETER}}}] \textbf{\underline{dependents}} the dependent values.
\item [\colorbox{tagtype}{\color{white} \textbf{\textsf{PARAMETER}}}] \textbf{\underline{max\_iter}} maximum number of iterations to try
\item [\colorbox{tagtype}{\color{white} \textbf{\textsf{PARAMETER}}}] \textbf{\underline{epsilon}} the minimum change in the Beta value estimate to continue
\item [\colorbox{tagtype}{\color{white} \textbf{\textsf{PARAMETER}}}] \textbf{\underline{ridge}} a value to pupulate a diagonal matrix that is added to a matrix help assure that the matrix is invertible.
\item [\colorbox{tagtype}{\color{white} \textbf{\textsf{RETURN}}}] \textbf{\underline{}} coefficient matrix plus model building stats
\end{description}

\rule{\linewidth}{0.5pt}
