\chapter*{LogisticRegression.Null\_Deviance}
\hypertarget{ecldoc:toc:LogisticRegression.Null_Deviance}{}

\section*{\underline{IMPORTS}}
\begin{itemize}
\item LogisticRegression
\item LogisticRegression.Types
\end{itemize}

\section*{\underline{DESCRIPTIONS}}
\subsection*{FUNCTION : Null\_Deviance}
\hypertarget{ecldoc:logisticregression.null_deviance}{}

{\renewcommand{\arraystretch}{1.5}
\begin{tabularx}{\textwidth}{|>{\raggedright\arraybackslash}l|X|}
\hline
\hspace{0pt}DATASET(Types.Deviance\_Record) & Null\_Deviance \\
\hline
\multicolumn{2}{|>{\raggedright\arraybackslash}X|}{\hspace{0pt}(DATASET(Types.Observation\_Deviance) od)} \\
\hline
\end{tabularx}
}

\hyperlink{ecldoc:toc:LogisticRegression}{Up}

\par
Deviance for the null model, that is, a model with only an intercept.

\par
\begin{description}
\item [\textbf{Parameter}] od ||| Observation Deviance record set.
\item [\textbf{Return}] a data set of the null model deviances for each work item and classifier.
\end{description}

\rule{\textwidth}{0.4pt}
