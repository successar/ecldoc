\chapter*{LogisticRegression}
\hypertarget{ecldoc:toc:LogisticRegression}{}

\begin{tabularx}{\textwidth}{|l|X|}
\hline
Name & LogisticRegression \\
\hline
Version & 1.0.0 \\
\hline
Description & Logistic Regression implementation \\
\hline
License & http://www.apache.org/licenses/LICENSE-2.0 \\
\hline
Copyright & Copyright (C) 2017 HPCC Systems \\
\hline
Authors & HPCCSystems \\
\hline
DependsOn & ML\_Core, PBblas \\
\hline
Platform & 6.2.0 \\
\hline
\end{tabularx}

\section*{Table of Contents}
{\renewcommand{\arraystretch}{1.5}
\begin{longtable}{|p{\textwidth}|}
\hline
\hyperlink{ecldoc:toc:LogisticRegression.BinomialConfusion}{BinomialConfusion.ecl} \\
Binomial confusion matrix \\
\hline
\hyperlink{ecldoc:toc:LogisticRegression.BinomialLogisticRegression}{BinomialLogisticRegression.ecl} \\
Binomial logistic regression using iteratively re-weighted least squares \\
\hline
\hyperlink{ecldoc:toc:LogisticRegression.Confusion}{Confusion.ecl} \\
Detail confusion records to compare actual versus predicted response variable values \\
\hline
\hyperlink{ecldoc:toc:LogisticRegression.Constants}{Constants.ecl} \\
\hline
\hyperlink{ecldoc:toc:LogisticRegression.DataStats}{DataStats.ecl} \\
Information about the datasets \\
\hline
\hyperlink{ecldoc:toc:LogisticRegression.Deviance_Analysis}{Deviance\_Analysis.ecl} \\
Compare deviance information for an analysis of deviance \\
\hline
\hyperlink{ecldoc:toc:LogisticRegression.Deviance_Detail}{Deviance\_Detail.ecl} \\
Detail deviance for each observation \\
\hline
\hyperlink{ecldoc:toc:LogisticRegression.dimm}{dimm.ecl} \\
Matrix multiply when either A or B is a diagonal and is passed as a vector \\
\hline
\hyperlink{ecldoc:toc:LogisticRegression.Distributions}{Distributions.ecl} \\
\hline
\hyperlink{ecldoc:toc:LogisticRegression.ExtractBeta}{ExtractBeta.ecl} \\
Extract the beta values form the model dataset \\
\hline
\hyperlink{ecldoc:toc:LogisticRegression.ExtractBeta_CI}{ExtractBeta\_CI.ecl} \\
Extract the beta values form the model dataset \\
\hline
\hyperlink{ecldoc:toc:LogisticRegression.ExtractBeta_pval}{ExtractBeta\_pval.ecl} \\
Extract the beta values form the model dataset \\
\hline
\hyperlink{ecldoc:toc:LogisticRegression.ExtractReport}{ExtractReport.ecl} \\
Extract Report records from model \\
\hline
\hyperlink{ecldoc:toc:LogisticRegression.LogitPredict}{LogitPredict.ecl} \\
Predict the category values with the logit function and the the supplied beta coefficients \\
\hline
\hyperlink{ecldoc:toc:LogisticRegression.LogitScore}{LogitScore.ecl} \\
Calculate the score using the logit function and the the supplied beta coefficients \\
\hline
\hyperlink{ecldoc:toc:LogisticRegression.Model_Deviance}{Model\_Deviance.ecl} \\
Model Deviance \\
\hline
\hyperlink{ecldoc:toc:LogisticRegression.Null_Deviance}{Null\_Deviance.ecl} \\
Deviance for the null model, that is, a model with only an intercept \\
\hline
\hyperlink{ecldoc:toc:LogisticRegression.Types}{Types.ecl} \\
\hline
\end{longtable}
}

\chapter*{\color{headfile}
{\large LogisticRegression\slash\hspace{0pt}}
 \\
BinomialConfusion
}
\hypertarget{ecldoc:toc:LogisticRegression.BinomialConfusion}{}
\hyperlink{ecldoc:toc:root/LogisticRegression}{Go Up}

\section*{\underline{\textsf{IMPORTS}}}
\begin{doublespace}
{\large
ML\_Core.Types |
LogisticRegression |
LogisticRegression.Types |
}
\end{doublespace}

\section*{\underline{\textsf{DESCRIPTIONS}}}
\subsection*{\textsf{\colorbox{headtoc}{\color{white} FUNCTION}
BinomialConfusion}}

\hypertarget{ecldoc:logisticregression.binomialconfusion}{}

{\renewcommand{\arraystretch}{1.5}
\begin{tabularx}{\textwidth}{|>{\raggedright\arraybackslash}l|X|}
\hline
\hspace{0pt}\mytexttt{\color{red} DATASET(Types.Binomial\_Confusion\_Summary)} & \textbf{BinomialConfusion} \\
\hline
\multicolumn{2}{|>{\raggedright\arraybackslash}X|}{\hspace{0pt}\mytexttt{\color{param} (DATASET(Core\_Types.Confusion\_Detail) d)}} \\
\hline
\end{tabularx}
}

\par
Binomial confusion matrix. Work items with multinomial responses are ignored by this function. The higher value lexically is considered to be the positive indication.

\par
\begin{description}
\item [\colorbox{tagtype}{\color{white} \textbf{\textsf{PARAMETER}}}] \textbf{\underline{d}} confusion detail for the work item and classifier
\item [\colorbox{tagtype}{\color{white} \textbf{\textsf{RETURN}}}] \textbf{\underline{}} confusion matrix for a binomial classifier
\end{description}

\rule{\linewidth}{0.5pt}

\chapter*{\color{headfile}
BinomialLogisticRegression
}
\hypertarget{ecldoc:toc:BinomialLogisticRegression}{}
\hyperlink{ecldoc:toc:root}{Go Up}

\section*{\underline{\textsf{IMPORTS}}}
\begin{doublespace}
{\large
Constants |
ML\_Core.Interfaces |
ML\_Core.Types |
}
\end{doublespace}

\section*{\underline{\textsf{DESCRIPTIONS}}}
\subsection*{\textsf{\colorbox{headtoc}{\color{white} MODULE}
BinomialLogisticRegression}}

\hypertarget{ecldoc:binomiallogisticregression}{}

{\renewcommand{\arraystretch}{1.5}
\begin{tabularx}{\textwidth}{|>{\raggedright\arraybackslash}l|X|}
\hline
\hspace{0pt}\mytexttt{\color{red} } & \textbf{BinomialLogisticRegression} \\
\hline
\multicolumn{2}{|>{\raggedright\arraybackslash}X|}{\hspace{0pt}\mytexttt{\color{param} (UNSIGNED max\_iter=200, REAL8 epsilon=Constants.default\_epsilon, REAL8 ridge=Constants.default\_ridge)}} \\
\hline
\end{tabularx}
}

\par
Binomial logistic regression using iteratively re-weighted least squares.

\par
\begin{description}
\item [\colorbox{tagtype}{\color{white} \textbf{\textsf{PARAMETER}}}] \textbf{\underline{max\_iter}} maximum number of iterations to try
\item [\colorbox{tagtype}{\color{white} \textbf{\textsf{PARAMETER}}}] \textbf{\underline{epsilon}} the minimum change in the Beta value estimate to continue
\item [\colorbox{tagtype}{\color{white} \textbf{\textsf{PARAMETER}}}] \textbf{\underline{ridge}} a value to populate a diagonal matrix that is added to a matrix help assure that the matrix is invertible.
\end{description}

\textbf{Children}
\begin{enumerate}
\item \hyperlink{ecldoc:binomiallogisticregression.getmodel}{GetModel}
: Calculate the model to fit the observation data to the observed classes
\item \hyperlink{ecldoc:binomiallogisticregression.classify}{Classify}
: Classify the observations using a model
\item \hyperlink{ecldoc:binomiallogisticregression.report}{Report}
: Report the confusion matrix for the classifier and training data
\end{enumerate}

\rule{\linewidth}{0.5pt}

\subsection*{\textsf{\colorbox{headtoc}{\color{white} FUNCTION}
GetModel}}

\hypertarget{ecldoc:binomiallogisticregression.getmodel}{}
\hspace{0pt} \hyperlink{ecldoc:binomiallogisticregression}{BinomialLogisticRegression} \textbackslash 

{\renewcommand{\arraystretch}{1.5}
\begin{tabularx}{\textwidth}{|>{\raggedright\arraybackslash}l|X|}
\hline
\hspace{0pt}\mytexttt{\color{red} DATASET(Types.Layout\_Model)} & \textbf{GetModel} \\
\hline
\multicolumn{2}{|>{\raggedright\arraybackslash}X|}{\hspace{0pt}\mytexttt{\color{param} (DATASET(Types.NumericField) observations, DATASET(Types.DiscreteField) classifications)}} \\
\hline
\end{tabularx}
}

\par
Calculate the model to fit the observation data to the observed classes.

\par
\begin{description}
\item [\colorbox{tagtype}{\color{white} \textbf{\textsf{PARAMETER}}}] \textbf{\underline{observations}} the observed explanatory values
\item [\colorbox{tagtype}{\color{white} \textbf{\textsf{PARAMETER}}}] \textbf{\underline{classifications}} the observed classification used to build the model
\item [\colorbox{tagtype}{\color{white} \textbf{\textsf{RETURN}}}] \textbf{\underline{}} the encoded model
\item [\colorbox{tagtype}{\color{white} \textbf{\textsf{OVERRIDE}}}] \textbf{\underline{}} True
\end{description}

\rule{\linewidth}{0.5pt}
\subsection*{\textsf{\colorbox{headtoc}{\color{white} FUNCTION}
Classify}}

\hypertarget{ecldoc:binomiallogisticregression.classify}{}
\hspace{0pt} \hyperlink{ecldoc:binomiallogisticregression}{BinomialLogisticRegression} \textbackslash 

{\renewcommand{\arraystretch}{1.5}
\begin{tabularx}{\textwidth}{|>{\raggedright\arraybackslash}l|X|}
\hline
\hspace{0pt}\mytexttt{\color{red} DATASET(Types.Classify\_Result)} & \textbf{Classify} \\
\hline
\multicolumn{2}{|>{\raggedright\arraybackslash}X|}{\hspace{0pt}\mytexttt{\color{param} (DATASET(Types.Layout\_Model) model, DATASET(Types.NumericField) new\_observations)}} \\
\hline
\end{tabularx}
}

\par
Classify the observations using a model.

\par
\begin{description}
\item [\colorbox{tagtype}{\color{white} \textbf{\textsf{PARAMETER}}}] \textbf{\underline{model}} The model, which must be produced by a corresponding getModel function.
\item [\colorbox{tagtype}{\color{white} \textbf{\textsf{PARAMETER}}}] \textbf{\underline{new\_observations}} observations to be classified
\item [\colorbox{tagtype}{\color{white} \textbf{\textsf{RETURN}}}] \textbf{\underline{}} Classification with a confidence value
\item [\colorbox{tagtype}{\color{white} \textbf{\textsf{OVERRIDE}}}] \textbf{\underline{}} True
\end{description}

\rule{\linewidth}{0.5pt}
\subsection*{\textsf{\colorbox{headtoc}{\color{white} FUNCTION}
Report}}

\hypertarget{ecldoc:binomiallogisticregression.report}{}
\hspace{0pt} \hyperlink{ecldoc:binomiallogisticregression}{BinomialLogisticRegression} \textbackslash 

{\renewcommand{\arraystretch}{1.5}
\begin{tabularx}{\textwidth}{|>{\raggedright\arraybackslash}l|X|}
\hline
\hspace{0pt}\mytexttt{\color{red} DATASET(Types.Confusion\_Detail)} & \textbf{Report} \\
\hline
\multicolumn{2}{|>{\raggedright\arraybackslash}X|}{\hspace{0pt}\mytexttt{\color{param} (DATASET(Types.Layout\_Model) model, DATASET(Types.NumericField) observations, DATASET(Types.DiscreteField) classifications)}} \\
\hline
\end{tabularx}
}

\par
Report the confusion matrix for the classifier and training data.

\par
\begin{description}
\item [\colorbox{tagtype}{\color{white} \textbf{\textsf{PARAMETER}}}] \textbf{\underline{model}} the encoded model
\item [\colorbox{tagtype}{\color{white} \textbf{\textsf{PARAMETER}}}] \textbf{\underline{observations}} the explanatory values.
\item [\colorbox{tagtype}{\color{white} \textbf{\textsf{PARAMETER}}}] \textbf{\underline{classifications}} the classifications associated with the observations
\item [\colorbox{tagtype}{\color{white} \textbf{\textsf{RETURN}}}] \textbf{\underline{}} the confusion matrix showing correct and incorrect results
\item [\colorbox{tagtype}{\color{white} \textbf{\textsf{OVERRIDE}}}] \textbf{\underline{}} True
\end{description}

\rule{\linewidth}{0.5pt}



\chapter*{\color{headfile}
{\large LogisticRegression\slash\hspace{0pt}}
 \\
Confusion
}
\hypertarget{ecldoc:toc:LogisticRegression.Confusion}{}
\hyperlink{ecldoc:toc:root/LogisticRegression}{Go Up}

\section*{\underline{\textsf{IMPORTS}}}
\begin{doublespace}
{\large
ML\_Core |
ML\_Core.Types |
LogisticRegression |
LogisticRegression.Types |
}
\end{doublespace}

\section*{\underline{\textsf{DESCRIPTIONS}}}
\subsection*{\textsf{\colorbox{headtoc}{\color{white} FUNCTION}
Confusion}}

\hypertarget{ecldoc:logisticregression.confusion}{}

{\renewcommand{\arraystretch}{1.5}
\begin{tabularx}{\textwidth}{|>{\raggedright\arraybackslash}l|X|}
\hline
\hspace{0pt}\mytexttt{\color{red} DATASET(Confusion\_Detail)} & \textbf{Confusion} \\
\hline
\multicolumn{2}{|>{\raggedright\arraybackslash}X|}{\hspace{0pt}\mytexttt{\color{param} (DATASET(DiscreteField) dependents, DATASET(DiscreteField) predicts)}} \\
\hline
\end{tabularx}
}

\par
Detail confusion records to compare actual versus predicted response variable values.

\par
\begin{description}
\item [\colorbox{tagtype}{\color{white} \textbf{\textsf{PARAMETER}}}] \textbf{\underline{dependents}} the original response values
\item [\colorbox{tagtype}{\color{white} \textbf{\textsf{PARAMETER}}}] \textbf{\underline{predicts}} the predicted responses
\item [\colorbox{tagtype}{\color{white} \textbf{\textsf{RETURN}}}] \textbf{\underline{}} confusion counts by predicted and actual response values.
\end{description}

\rule{\linewidth}{0.5pt}

\chapter*{ML\_Core.Constants}
\hypertarget{ecldoc:toc:ML_Core.Constants}{}

\section*{\underline{IMPORTS}}

\section*{\underline{DESCRIPTIONS}}
\subsection*{MODULE : Constants}
\hypertarget{ecldoc:ML_Core.Constants}{}

{\renewcommand{\arraystretch}{1.5}
\begin{tabularx}{\textwidth}{|>{\raggedright\arraybackslash}l|X|}
\hline
\hspace{0pt} & Constants \\
\hline
\end{tabularx}
}

\hyperlink{ecldoc:toc:ML_Core}{Up}

\par
Useful constants


\hyperlink{ecldoc:ml_core.constants.pi}{Pi}  |
\hyperlink{ecldoc:ml_core.constants.root_2}{Root\_2}  |

\rule{\textwidth}{0.4pt}

\subsection*{ATTRIBUTE : Pi}
\hypertarget{ecldoc:ml_core.constants.pi}{}

{\renewcommand{\arraystretch}{1.5}
\begin{tabularx}{\textwidth}{|>{\raggedright\arraybackslash}l|X|}
\hline
\hspace{0pt} & Pi \\
\hline
\end{tabularx}
}

\hyperlink{ecldoc:ML_Core.Constants}{Up}

\par
Constant PI


\rule{\textwidth}{0.4pt}
\subsection*{ATTRIBUTE : Root\_2}
\hypertarget{ecldoc:ml_core.constants.root_2}{}

{\renewcommand{\arraystretch}{1.5}
\begin{tabularx}{\textwidth}{|>{\raggedright\arraybackslash}l|X|}
\hline
\hspace{0pt} & Root\_2 \\
\hline
\end{tabularx}
}

\hyperlink{ecldoc:ML_Core.Constants}{Up}

\par
Constant square root of 2


\rule{\textwidth}{0.4pt}



\chapter*{\color{headfile}
{\large LogisticRegression\slash\hspace{0pt}}
 \\
DataStats
}
\hypertarget{ecldoc:toc:LogisticRegression.DataStats}{}
\hyperlink{ecldoc:toc:root/LogisticRegression}{Go Up}

\section*{\underline{\textsf{IMPORTS}}}
\begin{doublespace}
{\large
ML\_Core.Types |
LogisticRegression |
LogisticRegression.Types |
LogisticRegression.Constants |
}
\end{doublespace}

\section*{\underline{\textsf{DESCRIPTIONS}}}
\subsection*{\textsf{\colorbox{headtoc}{\color{white} FUNCTION}
DataStats}}

\hypertarget{ecldoc:logisticregression.datastats}{}

{\renewcommand{\arraystretch}{1.5}
\begin{tabularx}{\textwidth}{|>{\raggedright\arraybackslash}l|X|}
\hline
\hspace{0pt}\mytexttt{\color{red} DATASET(Types.Data\_Info)} & \textbf{DataStats} \\
\hline
\multicolumn{2}{|>{\raggedright\arraybackslash}X|}{\hspace{0pt}\mytexttt{\color{param} (DATASET(Core\_Types.NumericField) indep, DATASET(Core\_Types.DiscreteField) dep, BOOLEAN field\_details=FALSE)}} \\
\hline
\end{tabularx}
}

\par





Information about the datasets. Without details the range for the x and y (independent and dependent) columns. Note that a column of all zero values cannot be distinguished from a missing column. When details are requested, the cardinality, minimum, and maximum values are returned. A zero cardinality is returned when the field cardinality exceeds the Constants.limit\_card value.






\par
\begin{description}
\item [\colorbox{tagtype}{\color{white} \textbf{\textsf{PARAMETER}}}] \textbf{\underline{indep}} ||| TABLE ( NumericField ) --- data set of independent variables
\item [\colorbox{tagtype}{\color{white} \textbf{\textsf{PARAMETER}}}] \textbf{\underline{dep}} ||| TABLE ( DiscreteField ) --- data set of dependent variables
\item [\colorbox{tagtype}{\color{white} \textbf{\textsf{PARAMETER}}}] \textbf{\underline{field\_details}} ||| BOOLEAN --- Boolean directive to provide field level info
\end{description}







\par
\begin{description}
\item [\colorbox{tagtype}{\color{white} \textbf{\textsf{RETURN}}}] \textbf{TABLE ( \{ UNSIGNED2 wi , UNSIGNED4 dependent\_fields , UNSIGNED4 dependent\_records , UNSIGNED4 independent\_fields , UNSIGNED4 independent\_records , UNSIGNED4 dependent\_count , UNSIGNED4 independent\_count , TABLE ( Field\_Desc ) dependent\_stats , TABLE ( Field\_Desc ) independent\_stats \} )} --- 
\end{description}




\rule{\linewidth}{0.5pt}

\chapter*{\color{headfile}
{\large LogisticRegression\slash\hspace{0pt}}
 \\
Deviance_Analysis
}
\hypertarget{ecldoc:toc:LogisticRegression.Deviance_Analysis}{}
\hyperlink{ecldoc:toc:root/LogisticRegression}{Go Up}

\section*{\underline{\textsf{IMPORTS}}}
\begin{doublespace}
{\large
LogisticRegression |
LogisticRegression.Types |
}
\end{doublespace}

\section*{\underline{\textsf{DESCRIPTIONS}}}
\subsection*{\textsf{\colorbox{headtoc}{\color{white} FUNCTION}
Deviance\_Analysis}}

\hypertarget{ecldoc:logisticregression.deviance_analysis}{}

{\renewcommand{\arraystretch}{1.5}
\begin{tabularx}{\textwidth}{|>{\raggedright\arraybackslash}l|X|}
\hline
\hspace{0pt}\mytexttt{\color{red} DATASET(Types.AOD\_Record)} & \textbf{Deviance\_Analysis} \\
\hline
\multicolumn{2}{|>{\raggedright\arraybackslash}X|}{\hspace{0pt}\mytexttt{\color{param} (DATASET(Types.Deviance\_Record) proposed, DATASET(Types.Deviance\_Record) base)}} \\
\hline
\end{tabularx}
}

\par
Compare deviance information for an analysis of deviance.

\par
\begin{description}
\item [\colorbox{tagtype}{\color{white} \textbf{\textsf{PARAMETER}}}] \textbf{\underline{proposed}} the proposed model
\item [\colorbox{tagtype}{\color{white} \textbf{\textsf{PARAMETER}}}] \textbf{\underline{base}} the base model for comparison
\item [\colorbox{tagtype}{\color{white} \textbf{\textsf{RETURN}}}] \textbf{\underline{}} the comparison of the deviance between the models
\end{description}

\rule{\linewidth}{0.5pt}

\chapter*{\color{headfile}
{\large LogisticRegression\slash\hspace{0pt}}
 \\
Deviance_Detail
}
\hypertarget{ecldoc:toc:LogisticRegression.Deviance_Detail}{}
\hyperlink{ecldoc:toc:root/LogisticRegression}{Go Up}

\section*{\underline{\textsf{IMPORTS}}}
\begin{doublespace}
{\large
ML\_Core |
ML\_Core.Types |
LogisticRegression |
LogisticRegression.Types |
}
\end{doublespace}

\section*{\underline{\textsf{DESCRIPTIONS}}}
\subsection*{\textsf{\colorbox{headtoc}{\color{white} FUNCTION}
Deviance\_Detail}}

\hypertarget{ecldoc:logisticregression.deviance_detail}{}

{\renewcommand{\arraystretch}{1.5}
\begin{tabularx}{\textwidth}{|>{\raggedright\arraybackslash}l|X|}
\hline
\hspace{0pt}\mytexttt{\color{red} DATASET(Types.Observation\_Deviance)} & \textbf{Deviance\_Detail} \\
\hline
\multicolumn{2}{|>{\raggedright\arraybackslash}X|}{\hspace{0pt}\mytexttt{\color{param} (DATASET(Core\_Types.DiscreteField) dependents, DATASET(Types.Raw\_Prediction) predicts)}} \\
\hline
\end{tabularx}
}

\par
Detail deviance for each observation.

\par
\begin{description}
\item [\colorbox{tagtype}{\color{white} \textbf{\textsf{PARAMETER}}}] \textbf{\underline{dependents}} original dependent records for the model
\item [\colorbox{tagtype}{\color{white} \textbf{\textsf{PARAMETER}}}] \textbf{\underline{predicts}} the predicted values of the response variable
\item [\colorbox{tagtype}{\color{white} \textbf{\textsf{RETURN}}}] \textbf{\underline{}} the deviance information by observation and the log likelihood of the predicted result.
\end{description}

\rule{\linewidth}{0.5pt}

\chapter*{LogisticRegression.dimm}
\hypertarget{ecldoc:toc:LogisticRegression.dimm}{}

\section*{\underline{IMPORTS}}
\begin{itemize}
\item std.BLAS
\item std.BLAS.Types
\end{itemize}

\section*{\underline{DESCRIPTIONS}}
\subsection*{EMBED : dimm}
\hypertarget{ecldoc:logisticregression.dimm}{}
\begin{minipage}[t]{\textwidth}
\begin{flushleft}
Types.matrix\_t dimm (BOOLEAN transposeA, BOOLEAN transposeB, BOOLEAN diagonalA, BOOLEAN diagonalB, Types.dimension\_t m, Types.dimension\_t n, Types.dimension\_t k, Types.value\_t alpha, Types.matrix\_t A, Types.matrix\_t B, Types.value\_t beta=0.0, Types.matrix\_t C=[])
\end{flushleft}
\end{minipage}
\hyperlink{ecldoc:toc:LogisticRegression}{Up}

\par
Matrix multiply when either A or B is a diagonal and is passed as a vector. alpha*op(A) op(B) + beta*C where op() is transpose
\par
\textbf{Parameter} : transposeA ||| true when transpose of A is used \\
\textbf{Parameter} : transposeB ||| true when transpose of B is used \\
\textbf{Parameter} : diagonalA ||| true when A is the diagonal matrix \\
\textbf{Parameter} : diagonalB ||| true when B is the diagonal matrix \\
\textbf{Parameter} : m ||| number of rows in product \\
\textbf{Parameter} : n ||| number of columns in product \\
\textbf{Parameter} : k ||| number of columns/rows for the multiplier/multiplicand \\
\textbf{Parameter} : alpha ||| scalar used on A \\
\textbf{Parameter} : A ||| matrix A \\
\textbf{Parameter} : B ||| matrix B \\
\textbf{Parameter} : beta ||| scalar for matrix C \\
\textbf{Parameter} : C ||| matrix C or empty \\

\chapter*{ML\_Core.Math.Distributions}
\hypertarget{ML_Core.Math.Distributions}{}

\section*{\underline{IMPORTS}}
\begin{itemize}
\item ML\_Core.Constants
\item ML\_Core.Math
\end{itemize}

\section*{\underline{DESCRIPTIONS}}
\subsection*{module : Distributions}
\hypertarget{ecldoc:ML_Core.Math.Distributions}{MODULE : Distributions} \\
\hyperlink{ecldoc:}{Up} \\
\par
\begin{enumerate}
\item \hyperlink{ecldoc:ml_core.math.distributions.normal_cdf}{Normal\_CDF}
\item \hyperlink{ecldoc:ml_core.math.distributions.normal_ppf}{Normal\_PPF}
\item \hyperlink{ecldoc:ml_core.math.distributions.t_cdf}{T\_CDF}
\item \hyperlink{ecldoc:ml_core.math.distributions.t_ppf}{T\_PPF}
\item \hyperlink{ecldoc:ml_core.math.distributions.chi2_cdf}{Chi2\_CDF}
\item \hyperlink{ecldoc:ml_core.math.distributions.chi2_ppf}{Chi2\_PPF}
\end{enumerate}
\subsection*{function : Normal\_CDF}
\hypertarget{ecldoc:ml_core.math.distributions.normal_cdf}{FUNCTION : REAL8 Normal\_CDF(REAL8 x)} \\
\hyperlink{ecldoc:ML_Core.Math.Distributions}{Up} \\
\par
Cumulative Distribution of the standard normal distribution, the probability that a normal random variable will be smaller than x standard deviations above or below the mean. Taken from C/C++ Mathematical Algorithms for Scientists and Engineers, n. Shammas, McGraw-Hill, 1995 \\
\textbf{Parameter} : x ||| the number of standard deviations \\
\subsection*{function : Normal\_PPF}
\hypertarget{ecldoc:ml_core.math.distributions.normal_ppf}{FUNCTION : REAL8 Normal\_PPF(REAL8 x)} \\
\hyperlink{ecldoc:ML_Core.Math.Distributions}{Up} \\
\par
Normal Distribution Percentage Point Function. Translated from C/C++ Mathematical Algorithms for Scientists and Engineers, N. Shammas, McGraw-Hill, 1995 \\
\textbf{Parameter} : x ||| probability \\
\subsection*{function : T\_CDF}
\hypertarget{ecldoc:ml_core.math.distributions.t_cdf}{FUNCTION : REAL8 T\_CDF(REAL8 x, REAL8 df)} \\
\hyperlink{ecldoc:ML_Core.Math.Distributions}{Up} \\
\par
Students t distribution integral evaluated between negative infinity and x. Translated from NIST SEL DATAPAC Fortran TCDF.f source \\
\textbf{Parameter} : x ||| value of the evaluation \\
\textbf{Parameter} : df ||| degrees of freedom \\
\subsection*{function : T\_PPF}
\hypertarget{ecldoc:ml_core.math.distributions.t_ppf}{FUNCTION : REAL8 T\_PPF(REAL8 x, REAL8 df)} \\
\hyperlink{ecldoc:ML_Core.Math.Distributions}{Up} \\
\par
Percentage point function for the T distribution. Translated from NIST SEL DATAPAC Fortran TPPF.f source \\
\subsection*{function : Chi2\_CDF}
\hypertarget{ecldoc:ml_core.math.distributions.chi2_cdf}{FUNCTION : REAL8 Chi2\_CDF(REAL8 x, REAL8 df)} \\
\hyperlink{ecldoc:ML_Core.Math.Distributions}{Up} \\
\par
The cumulative distribution function for the Chi Square distribution. the CDF for the specfied degrees of freedom. Translated from the NIST SEL DATAPAC Fortran subroutine CHSCDF. \\
\subsection*{function : Chi2\_PPF}
\hypertarget{ecldoc:ml_core.math.distributions.chi2_ppf}{FUNCTION : REAL8 Chi2\_PPF(REAL8 x, REAL8 df)} \\
\hyperlink{ecldoc:ML_Core.Math.Distributions}{Up} \\
\par
The Chi Squared PPF function. Translated from the NIST SEL DATAPAC Fortran subroutine CHSPPF. \\


\chapter*{LogisticRegression.ExtractBeta}
\hypertarget{LogisticRegression.ExtractBeta}{}

\section*{\underline{IMPORTS}}
\begin{itemize}
\item LogisticRegression
\item LogisticRegression.Types
\item ML\_Core.Types
\end{itemize}

\section*{\underline{DESCRIPTIONS}}
\subsection*{function : ExtractBeta}
\hypertarget{ecldoc:logisticregression.extractbeta}{FUNCTION : ExtractBeta(DATASET(Core\_Types.Layout\_Model) mod\_ds)} \\
\hyperlink{ecldoc:}{Up} \\
\par
Extract the beta values form the model dataset. \\
\textbf{Parameter} : mod\_ds ||| the model dataset \\
\textbf{Return} : a beta values as Model Coefficient records, zero as the constant term. \\

\chapter*{\color{headfile}
{\large LogisticRegression\slash\hspace{0pt}}
 \\
ExtractBeta_CI
}
\hypertarget{ecldoc:toc:LogisticRegression.ExtractBeta_CI}{}
\hyperlink{ecldoc:toc:root/LogisticRegression}{Go Up}

\section*{\underline{\textsf{IMPORTS}}}
\begin{doublespace}
{\large
ML\_Core.Types |
LogisticRegression |
LogisticRegression.Types |
}
\end{doublespace}

\section*{\underline{\textsf{DESCRIPTIONS}}}
\subsection*{\textsf{\colorbox{headtoc}{\color{white} FUNCTION}
ExtractBeta\_CI}}

\hypertarget{ecldoc:logisticregression.extractbeta_ci}{}

{\renewcommand{\arraystretch}{1.5}
\begin{tabularx}{\textwidth}{|>{\raggedright\arraybackslash}l|X|}
\hline
\hspace{0pt}\mytexttt{\color{red} DATASET(Types.Confidence\_Model\_Coef)} & \textbf{ExtractBeta\_CI} \\
\hline
\multicolumn{2}{|>{\raggedright\arraybackslash}X|}{\hspace{0pt}\mytexttt{\color{param} (DATASET(Core\_Types.Layout\_Model) mod\_ds, REAL8 level)}} \\
\hline
\end{tabularx}
}

\par





Extract the beta values form the model dataset.






\par
\begin{description}
\item [\colorbox{tagtype}{\color{white} \textbf{\textsf{PARAMETER}}}] \textbf{\underline{level}} ||| REAL8 --- the significance value for the intervals
\item [\colorbox{tagtype}{\color{white} \textbf{\textsf{PARAMETER}}}] \textbf{\underline{mod\_ds}} ||| TABLE ( Layout\_Model ) --- the model dataset
\end{description}







\par
\begin{description}
\item [\colorbox{tagtype}{\color{white} \textbf{\textsf{RETURN}}}] \textbf{TABLE ( \{ UNSIGNED2 wi , UNSIGNED4 ind\_col , UNSIGNED4 dep\_nom , REAL8 w , REAL8 SE , REAL8 upper , REAL8 lower \} )} --- the beta values with confidence intervals term.
\end{description}




\rule{\linewidth}{0.5pt}

\chapter*{\color{headfile}
{\large LogisticRegression\slash\hspace{0pt}}
 \\
ExtractBeta_pval
}
\hypertarget{ecldoc:toc:LogisticRegression.ExtractBeta_pval}{}
\hyperlink{ecldoc:toc:root/LogisticRegression}{Go Up}

\section*{\underline{\textsf{IMPORTS}}}
\begin{doublespace}
{\large
LogisticRegression |
LogisticRegression.Types |
ML\_Core.Types |
}
\end{doublespace}

\section*{\underline{\textsf{DESCRIPTIONS}}}
\subsection*{\textsf{\colorbox{headtoc}{\color{white} FUNCTION}
ExtractBeta\_pval}}

\hypertarget{ecldoc:logisticregression.extractbeta_pval}{}

{\renewcommand{\arraystretch}{1.5}
\begin{tabularx}{\textwidth}{|>{\raggedright\arraybackslash}l|X|}
\hline
\hspace{0pt}\mytexttt{\color{red} DATASET(Types.pval\_Model\_Coef)} & \textbf{ExtractBeta\_pval} \\
\hline
\multicolumn{2}{|>{\raggedright\arraybackslash}X|}{\hspace{0pt}\mytexttt{\color{param} (DATASET(Core\_Types.Layout\_Model) mod\_ds)}} \\
\hline
\end{tabularx}
}

\par
Extract the beta values form the model dataset.

\par
\begin{description}
\item [\colorbox{tagtype}{\color{white} \textbf{\textsf{PARAMETER}}}] \textbf{\underline{mod\_ds}} the model dataset
\item [\colorbox{tagtype}{\color{white} \textbf{\textsf{RETURN}}}] \textbf{\underline{}} the beta values with p-values as Model Coefficient records, zero as the constant term.
\end{description}

\rule{\linewidth}{0.5pt}

\chapter*{\color{headfile}
{\large LogisticRegression\slash\hspace{0pt}}
 \\
ExtractReport
}
\hypertarget{ecldoc:toc:LogisticRegression.ExtractReport}{}
\hyperlink{ecldoc:toc:root/LogisticRegression}{Go Up}

\section*{\underline{\textsf{IMPORTS}}}
\begin{doublespace}
{\large
LogisticRegression |
LogisticRegression.Types |
LogisticRegression.Constants |
ML\_Core.Types |
}
\end{doublespace}

\section*{\underline{\textsf{DESCRIPTIONS}}}
\subsection*{\textsf{\colorbox{headtoc}{\color{white} FUNCTION}
ExtractReport}}

\hypertarget{ecldoc:logisticregression.extractreport}{}

{\renewcommand{\arraystretch}{1.5}
\begin{tabularx}{\textwidth}{|>{\raggedright\arraybackslash}l|X|}
\hline
\hspace{0pt}\mytexttt{\color{red} DATASET(Types.Model\_Report)} & \textbf{ExtractReport} \\
\hline
\multicolumn{2}{|>{\raggedright\arraybackslash}X|}{\hspace{0pt}\mytexttt{\color{param} (DATASET(Core\_Types.Layout\_Model) mod\_ds)}} \\
\hline
\end{tabularx}
}

\par
Extract Report records from model

\par
\begin{description}
\item [\colorbox{tagtype}{\color{white} \textbf{\textsf{PARAMETER}}}] \textbf{\underline{mod\_ds}} the model dataset
\item [\colorbox{tagtype}{\color{white} \textbf{\textsf{RETURN}}}] \textbf{\underline{}} the model report dataset
\end{description}

\rule{\linewidth}{0.5pt}

\chapter*{\color{headfile}
LogitPredict
}
\hypertarget{ecldoc:toc:LogitPredict}{}
\hyperlink{ecldoc:toc:root}{Go Up}

\section*{\underline{\textsf{IMPORTS}}}
\begin{doublespace}
{\large
ML\_Core.Types |
Types |
}
\end{doublespace}

\section*{\underline{\textsf{DESCRIPTIONS}}}
\subsection*{\textsf{\colorbox{headtoc}{\color{white} FUNCTION}
LogitPredict}}

\hypertarget{ecldoc:logitpredict}{}

{\renewcommand{\arraystretch}{1.5}
\begin{tabularx}{\textwidth}{|>{\raggedright\arraybackslash}l|X|}
\hline
\hspace{0pt}\mytexttt{\color{red} DATASET(Classify\_Result)} & \textbf{LogitPredict} \\
\hline
\multicolumn{2}{|>{\raggedright\arraybackslash}X|}{\hspace{0pt}\mytexttt{\color{param} (DATASET(Model\_Coef) coef, DATASET(NumericField) independents)}} \\
\hline
\end{tabularx}
}

\par
Predict the category values with the logit function and the the supplied beta coefficients.

\par
\begin{description}
\item [\colorbox{tagtype}{\color{white} \textbf{\textsf{PARAMETER}}}] \textbf{\underline{coef}} the model beta coefficients
\item [\colorbox{tagtype}{\color{white} \textbf{\textsf{PARAMETER}}}] \textbf{\underline{independents}} the observations
\item [\colorbox{tagtype}{\color{white} \textbf{\textsf{RETURN}}}] \textbf{\underline{}} the predicted category values and a confidence score
\end{description}

\rule{\linewidth}{0.5pt}

\chapter*{LogisticRegression.LogitScore}
\hypertarget{ecldoc:toc:LogisticRegression.LogitScore}{}

\section*{\underline{IMPORTS}}
\begin{itemize}
\item LogisticRegression
\item LogisticRegression.Types
\item ML\_Core.Types
\end{itemize}

\section*{\underline{DESCRIPTIONS}}
\subsection*{FUNCTION : LogitScore}
\hypertarget{ecldoc:logisticregression.logitscore}{}
\hyperlink{ecldoc:toc:LogisticRegression}{Up} :

{\renewcommand{\arraystretch}{1.5}
\begin{tabularx}{\textwidth}{|>{\raggedright\arraybackslash}l|X|}
\hline
\hspace{0pt}DATASET(Raw\_Prediction) & LogitScore \\
\hline
\multicolumn{2}{|>{\raggedright\arraybackslash}X|}{\hspace{0pt}(DATASET(Model\_Coef) coef, DATASET(NumericField) independents)} \\
\hline
\end{tabularx}
}

\par
Calculate the score using the logit function and the the supplied beta coefficients.

\par
\begin{description}
\item [\textbf{Parameter}] coef ||| the model beta coefficients
\item [\textbf{Parameter}] independents ||| the observations
\item [\textbf{Return}] the raw prediction value
\end{description}

\rule{\linewidth}{0.5pt}

\chapter*{LogisticRegression.Model\_Deviance}
\hypertarget{ecldoc:toc:LogisticRegression.Model_Deviance}{}

\section*{\underline{IMPORTS}}
\begin{itemize}
\item LogisticRegression
\item LogisticRegression.Types
\end{itemize}

\section*{\underline{DESCRIPTIONS}}
\subsection*{FUNCTION : Model\_Deviance}
\hypertarget{ecldoc:logisticregression.model_deviance}{}
\hyperlink{ecldoc:toc:LogisticRegression}{Up} :

{\renewcommand{\arraystretch}{1.5}
\begin{tabularx}{\textwidth}{|>{\raggedright\arraybackslash}l|X|}
\hline
\hspace{0pt}DATASET(Types.Deviance\_Record) & Model\_Deviance \\
\hline
\multicolumn{2}{|>{\raggedright\arraybackslash}X|}{\hspace{0pt}(DATASET(Types.Observation\_Deviance) od, DATASET(Types.Model\_Coef) mod)} \\
\hline
\end{tabularx}
}

\par
Model Deviance.

\par
\begin{description}
\item [\textbf{Parameter}] od ||| observation deviance record
\item [\textbf{Parameter}] mod ||| model co-efficients
\item [\textbf{Return}] model deviance
\end{description}

\rule{\linewidth}{0.5pt}

\chapter*{LogisticRegression.Null\_Deviance}
\hypertarget{ecldoc:toc:LogisticRegression.Null_Deviance}{}

\section*{\underline{IMPORTS}}
\begin{itemize}
\item LogisticRegression
\item LogisticRegression.Types
\end{itemize}

\section*{\underline{DESCRIPTIONS}}
\subsection*{FUNCTION : Null\_Deviance}
\hypertarget{ecldoc:logisticregression.null_deviance}{}

{\renewcommand{\arraystretch}{1.5}
\begin{tabularx}{\textwidth}{|>{\raggedright\arraybackslash}l|X|}
\hline
\hspace{0pt}DATASET(Types.Deviance\_Record) & Null\_Deviance \\
\hline
\multicolumn{2}{|>{\raggedright\arraybackslash}X|}{\hspace{0pt}(DATASET(Types.Observation\_Deviance) od)} \\
\hline
\end{tabularx}
}

\hyperlink{ecldoc:toc:LogisticRegression}{Up}

\par
Deviance for the null model, that is, a model with only an intercept.

\par
\begin{description}
\item [\textbf{Parameter}] od ||| Observation Deviance record set.
\item [\textbf{Return}] a data set of the null model deviances for each work item and classifier.
\end{description}

\rule{\textwidth}{0.4pt}

\chapter*{\color{headfile}
{\large ML\_Core\slash\hspace{0pt}}
 \\
Types
}
\hypertarget{ecldoc:toc:ML_Core.Types}{}
\hyperlink{ecldoc:toc:root/ML_Core}{Go Up}


\section*{\underline{\textsf{DESCRIPTIONS}}}
\subsection*{\textsf{\colorbox{headtoc}{\color{white} MODULE}
Types}}

\hypertarget{ecldoc:ML_Core.Types}{}

{\renewcommand{\arraystretch}{1.5}
\begin{tabularx}{\textwidth}{|>{\raggedright\arraybackslash}l|X|}
\hline
\hspace{0pt}\mytexttt{\color{red} } & \textbf{Types} \\
\hline
\end{tabularx}
}

\par


\textbf{Children}
\begin{enumerate}
\item \hyperlink{ecldoc:ml_core.types.t_recordid}{t\_RecordID}
\item \hyperlink{ecldoc:ml_core.types.t_fieldnumber}{t\_FieldNumber}
\item \hyperlink{ecldoc:ml_core.types.t_fieldreal}{t\_FieldReal}
\item \hyperlink{ecldoc:ml_core.types.t_fieldsign}{t\_FieldSign}
\item \hyperlink{ecldoc:ml_core.types.t_discrete}{t\_Discrete}
\item \hyperlink{ecldoc:ml_core.types.t_item}{t\_Item}
\item \hyperlink{ecldoc:ml_core.types.t_count}{t\_Count}
\item \hyperlink{ecldoc:ml_core.types.t_work_item}{t\_Work\_Item}
\item \hyperlink{ecldoc:ml_core.types.anyfield}{AnyField}
\item \hyperlink{ecldoc:ml_core.types.numericfield}{NumericField}
\item \hyperlink{ecldoc:ml_core.types.discretefield}{DiscreteField}
\item \hyperlink{ecldoc:ml_core.types.layout_model}{Layout\_Model}
\item \hyperlink{ecldoc:ml_core.types.classify_result}{Classify\_Result}
\item \hyperlink{ecldoc:ml_core.types.l_result}{l\_result}
\item \hyperlink{ecldoc:ml_core.types.confusion_detail}{Confusion\_Detail}
\item \hyperlink{ecldoc:ml_core.types.itemelement}{ItemElement}
\item \hyperlink{ecldoc:ml_core.types.t_node}{t\_node}
\item \hyperlink{ecldoc:ml_core.types.t_level}{t\_level}
\item \hyperlink{ecldoc:ml_core.types.nodeid}{NodeID}
\end{enumerate}

\rule{\linewidth}{0.5pt}

\subsection*{\textsf{\colorbox{headtoc}{\color{white} ATTRIBUTE}
t\_RecordID}}

\hypertarget{ecldoc:ml_core.types.t_recordid}{}
\hspace{0pt} \hyperlink{ecldoc:ML_Core.Types}{Types} \textbackslash 

{\renewcommand{\arraystretch}{1.5}
\begin{tabularx}{\textwidth}{|>{\raggedright\arraybackslash}l|X|}
\hline
\hspace{0pt}\mytexttt{\color{red} } & \textbf{t\_RecordID} \\
\hline
\end{tabularx}
}

\par


\rule{\linewidth}{0.5pt}
\subsection*{\textsf{\colorbox{headtoc}{\color{white} ATTRIBUTE}
t\_FieldNumber}}

\hypertarget{ecldoc:ml_core.types.t_fieldnumber}{}
\hspace{0pt} \hyperlink{ecldoc:ML_Core.Types}{Types} \textbackslash 

{\renewcommand{\arraystretch}{1.5}
\begin{tabularx}{\textwidth}{|>{\raggedright\arraybackslash}l|X|}
\hline
\hspace{0pt}\mytexttt{\color{red} } & \textbf{t\_FieldNumber} \\
\hline
\end{tabularx}
}

\par


\rule{\linewidth}{0.5pt}
\subsection*{\textsf{\colorbox{headtoc}{\color{white} ATTRIBUTE}
t\_FieldReal}}

\hypertarget{ecldoc:ml_core.types.t_fieldreal}{}
\hspace{0pt} \hyperlink{ecldoc:ML_Core.Types}{Types} \textbackslash 

{\renewcommand{\arraystretch}{1.5}
\begin{tabularx}{\textwidth}{|>{\raggedright\arraybackslash}l|X|}
\hline
\hspace{0pt}\mytexttt{\color{red} } & \textbf{t\_FieldReal} \\
\hline
\end{tabularx}
}

\par


\rule{\linewidth}{0.5pt}
\subsection*{\textsf{\colorbox{headtoc}{\color{white} ATTRIBUTE}
t\_FieldSign}}

\hypertarget{ecldoc:ml_core.types.t_fieldsign}{}
\hspace{0pt} \hyperlink{ecldoc:ML_Core.Types}{Types} \textbackslash 

{\renewcommand{\arraystretch}{1.5}
\begin{tabularx}{\textwidth}{|>{\raggedright\arraybackslash}l|X|}
\hline
\hspace{0pt}\mytexttt{\color{red} } & \textbf{t\_FieldSign} \\
\hline
\end{tabularx}
}

\par


\rule{\linewidth}{0.5pt}
\subsection*{\textsf{\colorbox{headtoc}{\color{white} ATTRIBUTE}
t\_Discrete}}

\hypertarget{ecldoc:ml_core.types.t_discrete}{}
\hspace{0pt} \hyperlink{ecldoc:ML_Core.Types}{Types} \textbackslash 

{\renewcommand{\arraystretch}{1.5}
\begin{tabularx}{\textwidth}{|>{\raggedright\arraybackslash}l|X|}
\hline
\hspace{0pt}\mytexttt{\color{red} } & \textbf{t\_Discrete} \\
\hline
\end{tabularx}
}

\par


\rule{\linewidth}{0.5pt}
\subsection*{\textsf{\colorbox{headtoc}{\color{white} ATTRIBUTE}
t\_Item}}

\hypertarget{ecldoc:ml_core.types.t_item}{}
\hspace{0pt} \hyperlink{ecldoc:ML_Core.Types}{Types} \textbackslash 

{\renewcommand{\arraystretch}{1.5}
\begin{tabularx}{\textwidth}{|>{\raggedright\arraybackslash}l|X|}
\hline
\hspace{0pt}\mytexttt{\color{red} } & \textbf{t\_Item} \\
\hline
\end{tabularx}
}

\par


\rule{\linewidth}{0.5pt}
\subsection*{\textsf{\colorbox{headtoc}{\color{white} ATTRIBUTE}
t\_Count}}

\hypertarget{ecldoc:ml_core.types.t_count}{}
\hspace{0pt} \hyperlink{ecldoc:ML_Core.Types}{Types} \textbackslash 

{\renewcommand{\arraystretch}{1.5}
\begin{tabularx}{\textwidth}{|>{\raggedright\arraybackslash}l|X|}
\hline
\hspace{0pt}\mytexttt{\color{red} } & \textbf{t\_Count} \\
\hline
\end{tabularx}
}

\par


\rule{\linewidth}{0.5pt}
\subsection*{\textsf{\colorbox{headtoc}{\color{white} ATTRIBUTE}
t\_Work\_Item}}

\hypertarget{ecldoc:ml_core.types.t_work_item}{}
\hspace{0pt} \hyperlink{ecldoc:ML_Core.Types}{Types} \textbackslash 

{\renewcommand{\arraystretch}{1.5}
\begin{tabularx}{\textwidth}{|>{\raggedright\arraybackslash}l|X|}
\hline
\hspace{0pt}\mytexttt{\color{red} } & \textbf{t\_Work\_Item} \\
\hline
\end{tabularx}
}

\par


\rule{\linewidth}{0.5pt}
\subsection*{\textsf{\colorbox{headtoc}{\color{white} RECORD}
AnyField}}

\hypertarget{ecldoc:ml_core.types.anyfield}{}
\hspace{0pt} \hyperlink{ecldoc:ML_Core.Types}{Types} \textbackslash 

{\renewcommand{\arraystretch}{1.5}
\begin{tabularx}{\textwidth}{|>{\raggedright\arraybackslash}l|X|}
\hline
\hspace{0pt}\mytexttt{\color{red} } & \textbf{AnyField} \\
\hline
\end{tabularx}
}

\par


\rule{\linewidth}{0.5pt}
\subsection*{\textsf{\colorbox{headtoc}{\color{white} RECORD}
NumericField}}

\hypertarget{ecldoc:ml_core.types.numericfield}{}
\hspace{0pt} \hyperlink{ecldoc:ML_Core.Types}{Types} \textbackslash 

{\renewcommand{\arraystretch}{1.5}
\begin{tabularx}{\textwidth}{|>{\raggedright\arraybackslash}l|X|}
\hline
\hspace{0pt}\mytexttt{\color{red} } & \textbf{NumericField} \\
\hline
\end{tabularx}
}

\par


\rule{\linewidth}{0.5pt}
\subsection*{\textsf{\colorbox{headtoc}{\color{white} RECORD}
DiscreteField}}

\hypertarget{ecldoc:ml_core.types.discretefield}{}
\hspace{0pt} \hyperlink{ecldoc:ML_Core.Types}{Types} \textbackslash 

{\renewcommand{\arraystretch}{1.5}
\begin{tabularx}{\textwidth}{|>{\raggedright\arraybackslash}l|X|}
\hline
\hspace{0pt}\mytexttt{\color{red} } & \textbf{DiscreteField} \\
\hline
\end{tabularx}
}

\par


\rule{\linewidth}{0.5pt}
\subsection*{\textsf{\colorbox{headtoc}{\color{white} RECORD}
Layout\_Model}}

\hypertarget{ecldoc:ml_core.types.layout_model}{}
\hspace{0pt} \hyperlink{ecldoc:ML_Core.Types}{Types} \textbackslash 

{\renewcommand{\arraystretch}{1.5}
\begin{tabularx}{\textwidth}{|>{\raggedright\arraybackslash}l|X|}
\hline
\hspace{0pt}\mytexttt{\color{red} } & \textbf{Layout\_Model} \\
\hline
\end{tabularx}
}

\par


\rule{\linewidth}{0.5pt}
\subsection*{\textsf{\colorbox{headtoc}{\color{white} RECORD}
Classify\_Result}}

\hypertarget{ecldoc:ml_core.types.classify_result}{}
\hspace{0pt} \hyperlink{ecldoc:ML_Core.Types}{Types} \textbackslash 

{\renewcommand{\arraystretch}{1.5}
\begin{tabularx}{\textwidth}{|>{\raggedright\arraybackslash}l|X|}
\hline
\hspace{0pt}\mytexttt{\color{red} } & \textbf{Classify\_Result} \\
\hline
\end{tabularx}
}

\par


\rule{\linewidth}{0.5pt}
\subsection*{\textsf{\colorbox{headtoc}{\color{white} RECORD}
l\_result}}

\hypertarget{ecldoc:ml_core.types.l_result}{}
\hspace{0pt} \hyperlink{ecldoc:ML_Core.Types}{Types} \textbackslash 

{\renewcommand{\arraystretch}{1.5}
\begin{tabularx}{\textwidth}{|>{\raggedright\arraybackslash}l|X|}
\hline
\hspace{0pt}\mytexttt{\color{red} } & \textbf{l\_result} \\
\hline
\end{tabularx}
}

\par


\rule{\linewidth}{0.5pt}
\subsection*{\textsf{\colorbox{headtoc}{\color{white} RECORD}
Confusion\_Detail}}

\hypertarget{ecldoc:ml_core.types.confusion_detail}{}
\hspace{0pt} \hyperlink{ecldoc:ML_Core.Types}{Types} \textbackslash 

{\renewcommand{\arraystretch}{1.5}
\begin{tabularx}{\textwidth}{|>{\raggedright\arraybackslash}l|X|}
\hline
\hspace{0pt}\mytexttt{\color{red} } & \textbf{Confusion\_Detail} \\
\hline
\end{tabularx}
}

\par


\rule{\linewidth}{0.5pt}
\subsection*{\textsf{\colorbox{headtoc}{\color{white} RECORD}
ItemElement}}

\hypertarget{ecldoc:ml_core.types.itemelement}{}
\hspace{0pt} \hyperlink{ecldoc:ML_Core.Types}{Types} \textbackslash 

{\renewcommand{\arraystretch}{1.5}
\begin{tabularx}{\textwidth}{|>{\raggedright\arraybackslash}l|X|}
\hline
\hspace{0pt}\mytexttt{\color{red} } & \textbf{ItemElement} \\
\hline
\end{tabularx}
}

\par


\rule{\linewidth}{0.5pt}
\subsection*{\textsf{\colorbox{headtoc}{\color{white} ATTRIBUTE}
t\_node}}

\hypertarget{ecldoc:ml_core.types.t_node}{}
\hspace{0pt} \hyperlink{ecldoc:ML_Core.Types}{Types} \textbackslash 

{\renewcommand{\arraystretch}{1.5}
\begin{tabularx}{\textwidth}{|>{\raggedright\arraybackslash}l|X|}
\hline
\hspace{0pt}\mytexttt{\color{red} } & \textbf{t\_node} \\
\hline
\end{tabularx}
}

\par


\rule{\linewidth}{0.5pt}
\subsection*{\textsf{\colorbox{headtoc}{\color{white} ATTRIBUTE}
t\_level}}

\hypertarget{ecldoc:ml_core.types.t_level}{}
\hspace{0pt} \hyperlink{ecldoc:ML_Core.Types}{Types} \textbackslash 

{\renewcommand{\arraystretch}{1.5}
\begin{tabularx}{\textwidth}{|>{\raggedright\arraybackslash}l|X|}
\hline
\hspace{0pt}\mytexttt{\color{red} } & \textbf{t\_level} \\
\hline
\end{tabularx}
}

\par


\rule{\linewidth}{0.5pt}
\subsection*{\textsf{\colorbox{headtoc}{\color{white} RECORD}
NodeID}}

\hypertarget{ecldoc:ml_core.types.nodeid}{}
\hspace{0pt} \hyperlink{ecldoc:ML_Core.Types}{Types} \textbackslash 

{\renewcommand{\arraystretch}{1.5}
\begin{tabularx}{\textwidth}{|>{\raggedright\arraybackslash}l|X|}
\hline
\hspace{0pt}\mytexttt{\color{red} } & \textbf{NodeID} \\
\hline
\end{tabularx}
}

\par


\rule{\linewidth}{0.5pt}



