\chapter*{LogisticRegression.Confusion}
\hypertarget{ecldoc:toc:LogisticRegression.Confusion}{}

\section*{\underline{IMPORTS}}
\begin{itemize}
\item ML\_Core
\item ML\_Core.Types
\item LogisticRegression
\item LogisticRegression.Types
\end{itemize}

\section*{\underline{DESCRIPTIONS}}
\subsection*{FUNCTION : Confusion}
\hypertarget{ecldoc:logisticregression.confusion}{}
\hyperlink{ecldoc:toc:LogisticRegression}{Up} :

{\renewcommand{\arraystretch}{1.5}
\begin{tabularx}{\textwidth}{|>{\raggedright\arraybackslash}l|X|}
\hline
\hspace{0pt}DATASET(Confusion\_Detail) & Confusion \\
\hline
\multicolumn{2}{|>{\raggedright\arraybackslash}X|}{\hspace{0pt}(DATASET(DiscreteField) dependents, DATASET(DiscreteField) predicts)} \\
\hline
\end{tabularx}
}

\par
Detail confusion records to compare actual versus predicted response variable values.

\par
\begin{description}
\item [\textbf{Parameter}] dependents ||| the original response values
\item [\textbf{Parameter}] predicts ||| the predicted responses
\item [\textbf{Return}] confusion counts by predicted and actual response values.
\end{description}

\rule{\linewidth}{0.5pt}
