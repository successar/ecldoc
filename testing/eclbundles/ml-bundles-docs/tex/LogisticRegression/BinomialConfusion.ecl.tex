\chapter*{LogisticRegression.BinomialConfusion}
\hypertarget{ecldoc:toc:LogisticRegression.BinomialConfusion}{}

\section*{\underline{IMPORTS}}
\begin{itemize}
\item LogisticRegression
\item LogisticRegression.Types
\item ML\_Core.Types
\end{itemize}

\section*{\underline{DESCRIPTIONS}}
\subsection*{FUNCTION : BinomialConfusion}
\hypertarget{ecldoc:logisticregression.binomialconfusion}{}
\begin{minipage}[t]{\textwidth}
\begin{flushleft}
DATASET(Types.Binomial\_Confusion\_Summary) BinomialConfusion (DATASET(Core\_Types.Confusion\_Detail) d)
\end{flushleft}
\end{minipage}
\hyperlink{ecldoc:toc:LogisticRegression}{Up}

\par
Binomial confusion matrix. Work items with multinomial responses are ignored by this function. The higher value lexically is considered to be the positive indication.
\par
\textbf{Parameter} : d ||| confusion detail for the work item and classifier \\
\textbf{Return} : confusion matrix for a binomial classifier \\
