\chapter*{LogisticRegression.BinomialConfusion}
\hypertarget{ecldoc:toc:LogisticRegression.BinomialConfusion}{}

\section*{\underline{IMPORTS}}
\begin{itemize}
\item LogisticRegression
\item LogisticRegression.Types
\item ML\_Core.Types
\end{itemize}

\section*{\underline{DESCRIPTIONS}}
\subsection*{FUNCTION : BinomialConfusion}
\hypertarget{ecldoc:logisticregression.binomialconfusion}{}
\hyperlink{ecldoc:toc:LogisticRegression}{Up} :

{\renewcommand{\arraystretch}{1.5}
\begin{tabularx}{\textwidth}{|>{\raggedright\arraybackslash}l|X|}
\hline
\hspace{0pt}DATASET(Types.Binomial\_Confusion\_Summary) & BinomialConfusion \\
\hline
\multicolumn{2}{|>{\raggedright\arraybackslash}X|}{\hspace{0pt}(DATASET(Core\_Types.Confusion\_Detail) d)} \\
\hline
\end{tabularx}
}

\par
Binomial confusion matrix. Work items with multinomial responses are ignored by this function. The higher value lexically is considered to be the positive indication.

\par
\begin{description}
\item [\textbf{Parameter}] d ||| confusion detail for the work item and classifier
\item [\textbf{Return}] confusion matrix for a binomial classifier
\end{description}

\rule{\linewidth}{0.5pt}
