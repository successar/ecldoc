\chapter*{LogisticRegression.BinomialLogisticRegression}
\hypertarget{ecldoc:toc:LogisticRegression.BinomialLogisticRegression}{}

\section*{\underline{IMPORTS}}
\begin{itemize}
\item LogisticRegression
\item LogisticRegression.Constants
\item ML\_Core.Interfaces
\item ML\_Core.Types
\end{itemize}

\section*{\underline{DESCRIPTIONS}}
\subsection*{MODULE : BinomialLogisticRegression}
\hypertarget{ecldoc:logisticregression.binomiallogisticregression}{}
\begin{minipage}[t]{\textwidth}
\begin{flushleft}
 BinomialLogisticRegression (UNSIGNED max\_iter=200, REAL8 epsilon=Constants.default\_epsilon, REAL8 ridge=Constants.default\_ridge)
\end{flushleft}
\end{minipage}
\hyperlink{ecldoc:toc:LogisticRegression}{Up}

\par
Binomial logistic regression using iteratively re-weighted least squares.
\par
\textbf{Parameter} : max\_iter ||| maximum number of iterations to try \\
\textbf{Parameter} : epsilon ||| the minimum change in the Beta value estimate to continue \\
\textbf{Parameter} : ridge ||| a value to populate a diagonal matrix that is added to a matrix help assure that the matrix is invertible. \\
\begin{enumerate}
\item \hyperlink{ecldoc:logisticregression.binomiallogisticregression.getmodel}{GetModel}
\item \hyperlink{ecldoc:logisticregression.binomiallogisticregression.classify}{Classify}
\item \hyperlink{ecldoc:logisticregression.binomiallogisticregression.report}{Report}
\end{enumerate}
\subsection*{FUNCTION : GetModel}
\hypertarget{ecldoc:logisticregression.binomiallogisticregression.getmodel}{}
\begin{minipage}[t]{\textwidth}
\begin{flushleft}
DATASET(Types.Layout\_Model) GetModel (DATASET(Types.NumericField) observations, DATASET(Types.DiscreteField) classifications)
\end{flushleft}
\end{minipage}
\hyperlink{ecldoc:logisticregression.binomiallogisticregression}{Up}

\par
Calculate the model to fit the observation data to the observed classes.
\par
\textbf{Parameter} : observations ||| the observed explanatory values \\
\textbf{Parameter} : classifications ||| the observed classification used to build the model \\
\textbf{Return} : the encoded model \\
\textbf{OVERRIDE} : True \\
\subsection*{FUNCTION : Classify}
\hypertarget{ecldoc:logisticregression.binomiallogisticregression.classify}{}
\begin{minipage}[t]{\textwidth}
\begin{flushleft}
DATASET(Types.Classify\_Result) Classify (DATASET(Types.Layout\_Model) model, DATASET(Types.NumericField) new\_observations)
\end{flushleft}
\end{minipage}
\hyperlink{ecldoc:logisticregression.binomiallogisticregression}{Up}

\par
Classify the observations using a model.
\par
\textbf{Parameter} : model ||| The model, which must be produced by a corresponding getModel function. \\
\textbf{Parameter} : new\_observations ||| observations to be classified \\
\textbf{Return} : Classification with a confidence value \\
\textbf{OVERRIDE} : True \\
\subsection*{FUNCTION : Report}
\hypertarget{ecldoc:logisticregression.binomiallogisticregression.report}{}
\begin{minipage}[t]{\textwidth}
\begin{flushleft}
DATASET(Types.Confusion\_Detail) Report (DATASET(Types.Layout\_Model) model, DATASET(Types.NumericField) observations, DATASET(Types.DiscreteField) classifications)
\end{flushleft}
\end{minipage}
\hyperlink{ecldoc:logisticregression.binomiallogisticregression}{Up}

\par
Report the confusion matrix for the classifier and training data.
\par
\textbf{Parameter} : model ||| the encoded model \\
\textbf{Parameter} : observations ||| the explanatory values. \\
\textbf{Parameter} : classifications ||| the classifications associated with the observations \\
\textbf{Return} : the confusion matrix showing correct and incorrect results \\
\textbf{OVERRIDE} : True \\

