\chapter*{LogisticRegression.BinomialLogisticRegression}
\hypertarget{ecldoc:toc:LogisticRegression.BinomialLogisticRegression}{}

\section*{\underline{IMPORTS}}
\begin{itemize}
\item LogisticRegression
\item LogisticRegression.Constants
\item ML\_Core.Interfaces
\item ML\_Core.Types
\end{itemize}

\section*{\underline{DESCRIPTIONS}}
\subsection*{MODULE : BinomialLogisticRegression}
\hypertarget{ecldoc:logisticregression.binomiallogisticregression}{}

{\renewcommand{\arraystretch}{1.5}
\begin{tabularx}{\textwidth}{|>{\raggedright\arraybackslash}l|X|}
\hline
\hspace{0pt} & BinomialLogisticRegression \\
\hline
\multicolumn{2}{|>{\raggedright\arraybackslash}X|}{\hspace{0pt}(UNSIGNED max\_iter=200, REAL8 epsilon=Constants.default\_epsilon, REAL8 ridge=Constants.default\_ridge)} \\
\hline
\end{tabularx}
}

\hyperlink{ecldoc:toc:LogisticRegression}{Up}

\par
Binomial logistic regression using iteratively re-weighted least squares.

\par
\begin{description}
\item [\textbf{Parameter}] max\_iter ||| maximum number of iterations to try
\item [\textbf{Parameter}] epsilon ||| the minimum change in the Beta value estimate to continue
\item [\textbf{Parameter}] ridge ||| a value to populate a diagonal matrix that is added to a matrix help assure that the matrix is invertible.
\end{description}

\begin{enumerate}
\item \hyperlink{ecldoc:logisticregression.binomiallogisticregression.getmodel}{GetModel}
\item \hyperlink{ecldoc:logisticregression.binomiallogisticregression.classify}{Classify}
\item \hyperlink{ecldoc:logisticregression.binomiallogisticregression.report}{Report}
\end{enumerate}

\rule{\textwidth}{0.4pt}

\subsection*{FUNCTION : GetModel}
\hypertarget{ecldoc:logisticregression.binomiallogisticregression.getmodel}{}

{\renewcommand{\arraystretch}{1.5}
\begin{tabularx}{\textwidth}{|>{\raggedright\arraybackslash}l|X|}
\hline
\hspace{0pt}DATASET(Types.Layout\_Model) & GetModel \\
\hline
\multicolumn{2}{|>{\raggedright\arraybackslash}X|}{\hspace{0pt}(DATASET(Types.NumericField) observations, DATASET(Types.DiscreteField) classifications)} \\
\hline
\end{tabularx}
}

\hyperlink{ecldoc:logisticregression.binomiallogisticregression}{Up}

\par
Calculate the model to fit the observation data to the observed classes.

\par
\begin{description}
\item [\textbf{Parameter}] observations ||| the observed explanatory values
\item [\textbf{Parameter}] classifications ||| the observed classification used to build the model
\item [\textbf{Return}] the encoded model
\item [\textbf{OVERRIDE}] True
\end{description}

\rule{\textwidth}{0.4pt}
\subsection*{FUNCTION : Classify}
\hypertarget{ecldoc:logisticregression.binomiallogisticregression.classify}{}

{\renewcommand{\arraystretch}{1.5}
\begin{tabularx}{\textwidth}{|>{\raggedright\arraybackslash}l|X|}
\hline
\hspace{0pt}DATASET(Types.Classify\_Result) & Classify \\
\hline
\multicolumn{2}{|>{\raggedright\arraybackslash}X|}{\hspace{0pt}(DATASET(Types.Layout\_Model) model, DATASET(Types.NumericField) new\_observations)} \\
\hline
\end{tabularx}
}

\hyperlink{ecldoc:logisticregression.binomiallogisticregression}{Up}

\par
Classify the observations using a model.

\par
\begin{description}
\item [\textbf{Parameter}] model ||| The model, which must be produced by a corresponding getModel function.
\item [\textbf{Parameter}] new\_observations ||| observations to be classified
\item [\textbf{Return}] Classification with a confidence value
\item [\textbf{OVERRIDE}] True
\end{description}

\rule{\textwidth}{0.4pt}
\subsection*{FUNCTION : Report}
\hypertarget{ecldoc:logisticregression.binomiallogisticregression.report}{}

{\renewcommand{\arraystretch}{1.5}
\begin{tabularx}{\textwidth}{|>{\raggedright\arraybackslash}l|X|}
\hline
\hspace{0pt}DATASET(Types.Confusion\_Detail) & Report \\
\hline
\multicolumn{2}{|>{\raggedright\arraybackslash}X|}{\hspace{0pt}(DATASET(Types.Layout\_Model) model, DATASET(Types.NumericField) observations, DATASET(Types.DiscreteField) classifications)} \\
\hline
\end{tabularx}
}

\hyperlink{ecldoc:logisticregression.binomiallogisticregression}{Up}

\par
Report the confusion matrix for the classifier and training data.

\par
\begin{description}
\item [\textbf{Parameter}] model ||| the encoded model
\item [\textbf{Parameter}] observations ||| the explanatory values.
\item [\textbf{Parameter}] classifications ||| the classifications associated with the observations
\item [\textbf{Return}] the confusion matrix showing correct and incorrect results
\item [\textbf{OVERRIDE}] True
\end{description}

\rule{\textwidth}{0.4pt}


