\chapter*{ML\_Core.Interfaces.IClassify}
\hypertarget{ecldoc:toc:ML_Core.Interfaces.IClassify}{}

\section*{\underline{IMPORTS}}
\begin{itemize}
\item ML\_Core
\item ML\_Core.Types
\end{itemize}

\section*{\underline{DESCRIPTIONS}}
\subsection*{MODULE : IClassify}
\hypertarget{ecldoc:ML_Core.Interfaces.IClassify}{}
\hyperlink{ecldoc:toc:ML_Core/Interfaces}{Up} :

{\renewcommand{\arraystretch}{1.5}
\begin{tabularx}{\textwidth}{|>{\raggedright\arraybackslash}l|X|}
\hline
\hspace{0pt} & IClassify \\
\hline
\end{tabularx}
}

\par
Interface definition for Classification. Actual implementation modules will probably take parameters.


\hyperlink{ecldoc:ml_core.interfaces.iclassify.getmodel}{GetModel}  |
\hyperlink{ecldoc:ml_core.interfaces.iclassify.classify}{Classify}  |
\hyperlink{ecldoc:ml_core.interfaces.iclassify.report}{Report}  |

\rule{\linewidth}{0.5pt}

\subsection*{FUNCTION : GetModel}
\hypertarget{ecldoc:ml_core.interfaces.iclassify.getmodel}{}
\hyperlink{ecldoc:ML_Core.Interfaces.IClassify}{Up} :
\hspace{0pt} \hyperlink{ecldoc:ML_Core.Interfaces.IClassify}{IClassify} \textbackslash 

{\renewcommand{\arraystretch}{1.5}
\begin{tabularx}{\textwidth}{|>{\raggedright\arraybackslash}l|X|}
\hline
\hspace{0pt}DATASET(Types.Layout\_Model) & GetModel \\
\hline
\multicolumn{2}{|>{\raggedright\arraybackslash}X|}{\hspace{0pt}(DATASET(Types.NumericField) observations, DATASET(Types.DiscreteField) classifications)} \\
\hline
\end{tabularx}
}

\par
Calculate the model to fit the observation data to the observed classes.

\par
\begin{description}
\item [\textbf{Parameter}] observations ||| the observed explanatory values
\item [\textbf{Parameter}] classifications ||| the observed classification used to build the model
\item [\textbf{Return}] the encoded model
\end{description}

\rule{\linewidth}{0.5pt}
\subsection*{FUNCTION : Classify}
\hypertarget{ecldoc:ml_core.interfaces.iclassify.classify}{}
\hyperlink{ecldoc:ML_Core.Interfaces.IClassify}{Up} :
\hspace{0pt} \hyperlink{ecldoc:ML_Core.Interfaces.IClassify}{IClassify} \textbackslash 

{\renewcommand{\arraystretch}{1.5}
\begin{tabularx}{\textwidth}{|>{\raggedright\arraybackslash}l|X|}
\hline
\hspace{0pt}DATASET(Types.Classify\_Result) & Classify \\
\hline
\multicolumn{2}{|>{\raggedright\arraybackslash}X|}{\hspace{0pt}(DATASET(Types.Layout\_Model) model, DATASET(Types.NumericField) new\_observations)} \\
\hline
\end{tabularx}
}

\par
Classify the observations using a model.

\par
\begin{description}
\item [\textbf{Parameter}] model ||| The model, which must be produced by a corresponding getModel function.
\item [\textbf{Parameter}] new\_observations ||| observations to be classified
\item [\textbf{Return}] Classification with a confidence value
\end{description}

\rule{\linewidth}{0.5pt}
\subsection*{FUNCTION : Report}
\hypertarget{ecldoc:ml_core.interfaces.iclassify.report}{}
\hyperlink{ecldoc:ML_Core.Interfaces.IClassify}{Up} :
\hspace{0pt} \hyperlink{ecldoc:ML_Core.Interfaces.IClassify}{IClassify} \textbackslash 

{\renewcommand{\arraystretch}{1.5}
\begin{tabularx}{\textwidth}{|>{\raggedright\arraybackslash}l|X|}
\hline
\hspace{0pt}DATASET(Types.Confusion\_Detail) & Report \\
\hline
\multicolumn{2}{|>{\raggedright\arraybackslash}X|}{\hspace{0pt}(DATASET(Types.Layout\_Model) model, DATASET(Types.NumericField) observations, DATASET(Types.DiscreteField) classifications)} \\
\hline
\end{tabularx}
}

\par
Report the confusion matrix for the classifier and training data.

\par
\begin{description}
\item [\textbf{Parameter}] model ||| the encoded model
\item [\textbf{Parameter}] observations ||| the explanatory values.
\item [\textbf{Parameter}] classifications ||| the classifications associated with the observations
\item [\textbf{Return}] the confusion matrix showing correct and incorrect results
\end{description}

\rule{\linewidth}{0.5pt}


