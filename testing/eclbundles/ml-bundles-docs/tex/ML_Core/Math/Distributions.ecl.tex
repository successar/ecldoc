\chapter*{ML\_Core.Math.Distributions}
\hypertarget{ecldoc:toc:ML_Core.Math.Distributions}{}

\section*{\underline{IMPORTS}}
\begin{itemize}
\item ML\_Core.Constants
\item ML\_Core.Math
\end{itemize}

\section*{\underline{DESCRIPTIONS}}
\subsection*{MODULE : Distributions}
\hypertarget{ecldoc:ML_Core.Math.Distributions}{}
\hyperlink{ecldoc:toc:ML_Core/Math}{Up} :

{\renewcommand{\arraystretch}{1.5}
\begin{tabularx}{\textwidth}{|>{\raggedright\arraybackslash}l|X|}
\hline
\hspace{0pt} & Distributions \\
\hline
\end{tabularx}
}

\par


\hyperlink{ecldoc:ml_core.math.distributions.normal_cdf}{Normal\_CDF}  |
\hyperlink{ecldoc:ml_core.math.distributions.normal_ppf}{Normal\_PPF}  |
\hyperlink{ecldoc:ml_core.math.distributions.t_cdf}{T\_CDF}  |
\hyperlink{ecldoc:ml_core.math.distributions.t_ppf}{T\_PPF}  |
\hyperlink{ecldoc:ml_core.math.distributions.chi2_cdf}{Chi2\_CDF}  |
\hyperlink{ecldoc:ml_core.math.distributions.chi2_ppf}{Chi2\_PPF}  |

\rule{\linewidth}{0.5pt}

\subsection*{FUNCTION : Normal\_CDF}
\hypertarget{ecldoc:ml_core.math.distributions.normal_cdf}{}
\hyperlink{ecldoc:ML_Core.Math.Distributions}{Up} :
\hspace{0pt} \hyperlink{ecldoc:ML_Core.Math.Distributions}{Distributions} \textbackslash 

{\renewcommand{\arraystretch}{1.5}
\begin{tabularx}{\textwidth}{|>{\raggedright\arraybackslash}l|X|}
\hline
\hspace{0pt}REAL8 & Normal\_CDF \\
\hline
\multicolumn{2}{|>{\raggedright\arraybackslash}X|}{\hspace{0pt}(REAL8 x)} \\
\hline
\end{tabularx}
}

\par
Cumulative Distribution of the standard normal distribution, the probability that a normal random variable will be smaller than x standard deviations above or below the mean. Taken from C/C++ Mathematical Algorithms for Scientists and Engineers, n. Shammas, McGraw-Hill, 1995

\par
\begin{description}
\item [\textbf{Parameter}] x ||| the number of standard deviations
\end{description}

\rule{\linewidth}{0.5pt}
\subsection*{FUNCTION : Normal\_PPF}
\hypertarget{ecldoc:ml_core.math.distributions.normal_ppf}{}
\hyperlink{ecldoc:ML_Core.Math.Distributions}{Up} :
\hspace{0pt} \hyperlink{ecldoc:ML_Core.Math.Distributions}{Distributions} \textbackslash 

{\renewcommand{\arraystretch}{1.5}
\begin{tabularx}{\textwidth}{|>{\raggedright\arraybackslash}l|X|}
\hline
\hspace{0pt}REAL8 & Normal\_PPF \\
\hline
\multicolumn{2}{|>{\raggedright\arraybackslash}X|}{\hspace{0pt}(REAL8 x)} \\
\hline
\end{tabularx}
}

\par
Normal Distribution Percentage Point Function. Translated from C/C++ Mathematical Algorithms for Scientists and Engineers, N. Shammas, McGraw-Hill, 1995

\par
\begin{description}
\item [\textbf{Parameter}] x ||| probability
\end{description}

\rule{\linewidth}{0.5pt}
\subsection*{FUNCTION : T\_CDF}
\hypertarget{ecldoc:ml_core.math.distributions.t_cdf}{}
\hyperlink{ecldoc:ML_Core.Math.Distributions}{Up} :
\hspace{0pt} \hyperlink{ecldoc:ML_Core.Math.Distributions}{Distributions} \textbackslash 

{\renewcommand{\arraystretch}{1.5}
\begin{tabularx}{\textwidth}{|>{\raggedright\arraybackslash}l|X|}
\hline
\hspace{0pt}REAL8 & T\_CDF \\
\hline
\multicolumn{2}{|>{\raggedright\arraybackslash}X|}{\hspace{0pt}(REAL8 x, REAL8 df)} \\
\hline
\end{tabularx}
}

\par
Students t distribution integral evaluated between negative infinity and x. Translated from NIST SEL DATAPAC Fortran TCDF.f source

\par
\begin{description}
\item [\textbf{Parameter}] x ||| value of the evaluation
\item [\textbf{Parameter}] df ||| degrees of freedom
\end{description}

\rule{\linewidth}{0.5pt}
\subsection*{FUNCTION : T\_PPF}
\hypertarget{ecldoc:ml_core.math.distributions.t_ppf}{}
\hyperlink{ecldoc:ML_Core.Math.Distributions}{Up} :
\hspace{0pt} \hyperlink{ecldoc:ML_Core.Math.Distributions}{Distributions} \textbackslash 

{\renewcommand{\arraystretch}{1.5}
\begin{tabularx}{\textwidth}{|>{\raggedright\arraybackslash}l|X|}
\hline
\hspace{0pt}REAL8 & T\_PPF \\
\hline
\multicolumn{2}{|>{\raggedright\arraybackslash}X|}{\hspace{0pt}(REAL8 x, REAL8 df)} \\
\hline
\end{tabularx}
}

\par
Percentage point function for the T distribution. Translated from NIST SEL DATAPAC Fortran TPPF.f source


\rule{\linewidth}{0.5pt}
\subsection*{FUNCTION : Chi2\_CDF}
\hypertarget{ecldoc:ml_core.math.distributions.chi2_cdf}{}
\hyperlink{ecldoc:ML_Core.Math.Distributions}{Up} :
\hspace{0pt} \hyperlink{ecldoc:ML_Core.Math.Distributions}{Distributions} \textbackslash 

{\renewcommand{\arraystretch}{1.5}
\begin{tabularx}{\textwidth}{|>{\raggedright\arraybackslash}l|X|}
\hline
\hspace{0pt}REAL8 & Chi2\_CDF \\
\hline
\multicolumn{2}{|>{\raggedright\arraybackslash}X|}{\hspace{0pt}(REAL8 x, REAL8 df)} \\
\hline
\end{tabularx}
}

\par
The cumulative distribution function for the Chi Square distribution. the CDF for the specfied degrees of freedom. Translated from the NIST SEL DATAPAC Fortran subroutine CHSCDF.


\rule{\linewidth}{0.5pt}
\subsection*{FUNCTION : Chi2\_PPF}
\hypertarget{ecldoc:ml_core.math.distributions.chi2_ppf}{}
\hyperlink{ecldoc:ML_Core.Math.Distributions}{Up} :
\hspace{0pt} \hyperlink{ecldoc:ML_Core.Math.Distributions}{Distributions} \textbackslash 

{\renewcommand{\arraystretch}{1.5}
\begin{tabularx}{\textwidth}{|>{\raggedright\arraybackslash}l|X|}
\hline
\hspace{0pt}REAL8 & Chi2\_PPF \\
\hline
\multicolumn{2}{|>{\raggedright\arraybackslash}X|}{\hspace{0pt}(REAL8 x, REAL8 df)} \\
\hline
\end{tabularx}
}

\par
The Chi Squared PPF function. Translated from the NIST SEL DATAPAC Fortran subroutine CHSPPF.


\rule{\linewidth}{0.5pt}


