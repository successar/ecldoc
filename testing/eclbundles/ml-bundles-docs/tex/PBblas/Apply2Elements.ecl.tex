\chapter*{PBblas.Apply2Elements}
\hypertarget{ecldoc:toc:PBblas.Apply2Elements}{}

\section*{\underline{IMPORTS}}
\begin{itemize}
\item PBblas
\item PBblas.Types
\item std.BLAS
\end{itemize}

\section*{\underline{DESCRIPTIONS}}
\subsection*{FUNCTION : Apply2Elements}
\hypertarget{ecldoc:pbblas.apply2elements}{}
\hyperlink{ecldoc:toc:PBblas}{Up} :

{\renewcommand{\arraystretch}{1.5}
\begin{tabularx}{\textwidth}{|>{\raggedright\arraybackslash}l|X|}
\hline
\hspace{0pt}DATASET(Layout\_Cell) & Apply2Elements \\
\hline
\multicolumn{2}{|>{\raggedright\arraybackslash}X|}{\hspace{0pt}(DATASET(Layout\_Cell) X, IElementFunc f)} \\
\hline
\end{tabularx}
}

\par
Apply a function to each element of the matrix Use PBblas.IElementFunc as the prototype function. Input and ouput may be a single matrix, or myriad matrixes with different work item ids.

\par
\begin{description}
\item [\textbf{Parameter}] X ||| A matrix (or multiple matrices) in Layout\_Cell form
\item [\textbf{Parameter}] f ||| A function based on the IElementFunc prototype
\item [\textbf{Return}] A matrix (or multiple matrices) in Layout\_Cell form
\item [\textbf{See}] PBblas/IElementFunc
\item [\textbf{See}] PBblas/Types.Layout\_Cell
\end{description}

\rule{\linewidth}{0.5pt}
