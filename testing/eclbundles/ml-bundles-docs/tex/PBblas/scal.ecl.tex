\chapter*{PBblas.scal}
\hypertarget{ecldoc:toc:PBblas.scal}{}

\section*{\underline{IMPORTS}}
\begin{itemize}
\item PBblas
\item PBblas.Types
\end{itemize}

\section*{\underline{DESCRIPTIONS}}
\subsection*{FUNCTION : scal}
\hypertarget{ecldoc:pbblas.scal}{}
\hyperlink{ecldoc:toc:PBblas}{Up} :

{\renewcommand{\arraystretch}{1.5}
\begin{tabularx}{\textwidth}{|>{\raggedright\arraybackslash}l|X|}
\hline
\hspace{0pt}DATASET(Layout\_Cell) & scal \\
\hline
\multicolumn{2}{|>{\raggedright\arraybackslash}X|}{\hspace{0pt}(value\_t alpha, DATASET(Layout\_Cell) X)} \\
\hline
\end{tabularx}
}

\par
Scale a matrix by a constant Result is alpha * X This supports a ''myriad'' style interface in that X may be a set of independent matrices separated by different work-item ids.

\par
\begin{description}
\item [\textbf{Parameter}] alpha ||| A scalar multiplier
\item [\textbf{Parameter}] X ||| The matrix(es) to be scaled in Layout\_Cell format
\item [\textbf{Return}] Matrix in Layout\_Cell form, of the same shape as X
\item [\textbf{See}] PBblas/Types.Layout\_Cell
\end{description}

\rule{\linewidth}{0.5pt}
