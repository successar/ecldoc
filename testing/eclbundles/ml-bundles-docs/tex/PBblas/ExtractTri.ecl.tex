\chapter*{PBblas.ExtractTri}
\hypertarget{ecldoc:toc:PBblas.ExtractTri}{}

\section*{\underline{IMPORTS}}
\begin{itemize}
\item PBblas
\item std.BLAS
\item PBblas.Types
\item PBblas.internal
\item PBblas.internal.Types
\item PBblas.internal.MatDims
\item PBblas.internal.Converted
\end{itemize}

\section*{\underline{DESCRIPTIONS}}
\subsection*{FUNCTION : ExtractTri}
\hypertarget{ecldoc:pbblas.extracttri}{}

{\renewcommand{\arraystretch}{1.5}
\begin{tabularx}{\textwidth}{|>{\raggedright\arraybackslash}l|X|}
\hline
\hspace{0pt}DATASET(Layout\_Cell) & ExtractTri \\
\hline
\multicolumn{2}{|>{\raggedright\arraybackslash}X|}{\hspace{0pt}(Triangle tri, Diagonal dt, DATASET(Layout\_Cell) A)} \\
\hline
\end{tabularx}
}

\hyperlink{ecldoc:toc:PBblas}{Up}

\par
Extract the upper or lower triangle from the composite output from getrf (LU Factorization).

\par
\begin{description}
\item [\textbf{Parameter}] tri ||| Triangle type: Upper or Lower (see Types.Triangle)
\item [\textbf{Parameter}] dt ||| Diagonal type: Unit or non unit (see Types.Diagonal)
\item [\textbf{Parameter}] A ||| Matrix of cells. See Types.Layout\_Cell
\item [\textbf{Return}] Matrix of cells in Layout\_Cell format representing a triangular matrix (upper or lower)
\item [\textbf{See}] Std.PBblas.Types
\end{description}

\rule{\textwidth}{0.4pt}
