\chapter*{PBblas.ExtractTri}
\hypertarget{ecldoc:toc:PBblas.ExtractTri}{}

\section*{\underline{IMPORTS}}
\begin{itemize}
\item PBblas
\item std.BLAS
\item PBblas.Types
\item PBblas.internal
\item PBblas.internal.Types
\item PBblas.internal.MatDims
\item PBblas.internal.Converted
\end{itemize}

\section*{\underline{DESCRIPTIONS}}
\subsection*{FUNCTION : ExtractTri}
\hypertarget{ecldoc:pbblas.extracttri}{}
\begin{minipage}[t]{\textwidth}
\begin{flushleft}
DATASET(Layout\_Cell) ExtractTri (Triangle tri, Diagonal dt, DATASET(Layout\_Cell) A)
\end{flushleft}
\end{minipage}
\hyperlink{ecldoc:toc:PBblas}{Up}

\par
Extract the upper or lower triangle from the composite output from getrf (LU Factorization).
\par
\textbf{Parameter} : tri ||| Triangle type: Upper or Lower (see Types.Triangle) \\
\textbf{Parameter} : dt ||| Diagonal type: Unit or non unit (see Types.Diagonal) \\
\textbf{Parameter} : A ||| Matrix of cells. See Types.Layout\_Cell \\
\textbf{Return} : Matrix of cells in Layout\_Cell format representing a triangular matrix (upper or lower) \\
\textbf{See} : Std.PBblas.Types \\
