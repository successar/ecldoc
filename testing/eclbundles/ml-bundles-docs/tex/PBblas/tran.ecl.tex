\chapter*{PBblas.tran}
\hypertarget{ecldoc:toc:PBblas.tran}{}

\section*{\underline{IMPORTS}}
\begin{itemize}
\item PBblas
\item PBblas.Types
\item PBblas.internal
\item PBblas.internal.Types
\item PBblas.internal.MatDims
\item PBblas.internal.Converted
\item std.BLAS
\item std.system.Thorlib
\end{itemize}

\section*{\underline{DESCRIPTIONS}}
\subsection*{FUNCTION : tran}
\hypertarget{ecldoc:pbblas.tran}{}
\begin{minipage}[t]{\textwidth}
\begin{flushleft}
DATASET(Layout\_Cell) tran (value\_t alpha, DATASET(Layout\_Cell) A, value\_t beta=0, DATASET(Layout\_Cell) C=empty\_c)
\end{flushleft}
\end{minipage}
\hyperlink{ecldoc:toc:PBblas}{Up}

\par
Transpose a matrix and sum into base matrix result <== alpha * A**t + beta * C, A is n by m, C is m by n A**T (A Transpose) and C must have same shape
\par
\textbf{Parameter} : alpha ||| Scalar multiplier for the A**T matrix \\
\textbf{Parameter} : A ||| A matrix in DATASET(Layout\_Cell) form \\
\textbf{Parameter} : beta ||| Scalar multiplier for the C matrix \\
\textbf{Parameter} : C ||| C matrix in DATASET(Layout\_Call) form \\
\textbf{Return} : Matrix in DATASET(Layout\_Cell) form alpha * A**T + beta * C \\
\textbf{See} : PBblas/Types.layout\_cell \\
