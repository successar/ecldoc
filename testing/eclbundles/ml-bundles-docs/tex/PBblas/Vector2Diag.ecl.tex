\chapter*{PBblas.Vector2Diag}
\hypertarget{ecldoc:toc:PBblas.Vector2Diag}{}

\section*{\underline{IMPORTS}}
\begin{itemize}
\item PBblas
\item PBblas.internal
\item PBblas.internal.MatDims
\item PBblas.Types
\item PBblas.internal.Types
\item PBblas.Constants
\end{itemize}

\section*{\underline{DESCRIPTIONS}}
\subsection*{FUNCTION : Vector2Diag}
\hypertarget{ecldoc:pbblas.vector2diag}{}

{\renewcommand{\arraystretch}{1.5}
\begin{tabularx}{\textwidth}{|>{\raggedright\arraybackslash}l|X|}
\hline
\hspace{0pt}DATASET(Layout\_Cell) & Vector2Diag \\
\hline
\multicolumn{2}{|>{\raggedright\arraybackslash}X|}{\hspace{0pt}(DATASET(Layout\_Cell) X)} \\
\hline
\end{tabularx}
}

\hyperlink{ecldoc:toc:PBblas}{Up}

\par
Convert a vector into a diagonal matrix. The typical notation is D = diag(V). The input X must be a 1 x N column vector or an N x 1 row vector. The resulting matrix, in either case will be N x N, with zero everywhere except the diagonal.

\par
\begin{description}
\item [\textbf{Parameter}] X ||| A row or column vector (i.e. N x 1 or 1 x N) in Layout\_Cell format
\item [\textbf{Return}] An N x N matrix in Layout\_Cell format
\item [\textbf{See}] PBblas/Types.Layout\_cell
\end{description}

\rule{\textwidth}{0.4pt}
