\chapter*{\color{headfile}
Distributions
}
\hypertarget{ecldoc:toc:Distributions}{}
\hyperlink{ecldoc:toc:root}{Go Up}

\section*{\underline{\textsf{IMPORTS}}}
\begin{doublespace}
{\large
ML\_Core.Constants |
ML\_Core.Math |
}
\end{doublespace}

\section*{\underline{\textsf{DESCRIPTIONS}}}
\subsection*{\textsf{\colorbox{headtoc}{\color{white} MODULE}
Distributions}}

\hypertarget{ecldoc:Distributions}{}

{\renewcommand{\arraystretch}{1.5}
\begin{tabularx}{\textwidth}{|>{\raggedright\arraybackslash}l|X|}
\hline
\hspace{0pt}\mytexttt{\color{red} } & \textbf{Distributions} \\
\hline
\end{tabularx}
}

\par


\textbf{Children}
\begin{enumerate}
\item \hyperlink{ecldoc:distributions.normal_cdf}{Normal\_CDF}
: Cumulative Distribution of the standard normal distribution, the probability that a normal random variable will be smaller than x standard deviations above or below the mean
\item \hyperlink{ecldoc:distributions.normal_ppf}{Normal\_PPF}
: Normal Distribution Percentage Point Function
\item \hyperlink{ecldoc:distributions.t_cdf}{T\_CDF}
: Students t distribution integral evaluated between negative infinity and x
\item \hyperlink{ecldoc:distributions.t_ppf}{T\_PPF}
: Percentage point function for the T distribution
\item \hyperlink{ecldoc:distributions.chi2_cdf}{Chi2\_CDF}
: The cumulative distribution function for the Chi Square distribution
\item \hyperlink{ecldoc:distributions.chi2_ppf}{Chi2\_PPF}
: The Chi Squared PPF function
\end{enumerate}

\rule{\linewidth}{0.5pt}

\subsection*{\textsf{\colorbox{headtoc}{\color{white} FUNCTION}
Normal\_CDF}}

\hypertarget{ecldoc:distributions.normal_cdf}{}
\hspace{0pt} \hyperlink{ecldoc:Distributions}{Distributions} \textbackslash 

{\renewcommand{\arraystretch}{1.5}
\begin{tabularx}{\textwidth}{|>{\raggedright\arraybackslash}l|X|}
\hline
\hspace{0pt}\mytexttt{\color{red} REAL8} & \textbf{Normal\_CDF} \\
\hline
\multicolumn{2}{|>{\raggedright\arraybackslash}X|}{\hspace{0pt}\mytexttt{\color{param} (REAL8 x)}} \\
\hline
\end{tabularx}
}

\par
Cumulative Distribution of the standard normal distribution, the probability that a normal random variable will be smaller than x standard deviations above or below the mean. Taken from C/C++ Mathematical Algorithms for Scientists and Engineers, n. Shammas, McGraw-Hill, 1995

\par
\begin{description}
\item [\colorbox{tagtype}{\color{white} \textbf{\textsf{PARAMETER}}}] \textbf{\underline{x}} the number of standard deviations
\end{description}

\rule{\linewidth}{0.5pt}
\subsection*{\textsf{\colorbox{headtoc}{\color{white} FUNCTION}
Normal\_PPF}}

\hypertarget{ecldoc:distributions.normal_ppf}{}
\hspace{0pt} \hyperlink{ecldoc:Distributions}{Distributions} \textbackslash 

{\renewcommand{\arraystretch}{1.5}
\begin{tabularx}{\textwidth}{|>{\raggedright\arraybackslash}l|X|}
\hline
\hspace{0pt}\mytexttt{\color{red} REAL8} & \textbf{Normal\_PPF} \\
\hline
\multicolumn{2}{|>{\raggedright\arraybackslash}X|}{\hspace{0pt}\mytexttt{\color{param} (REAL8 x)}} \\
\hline
\end{tabularx}
}

\par
Normal Distribution Percentage Point Function. Translated from C/C++ Mathematical Algorithms for Scientists and Engineers, N. Shammas, McGraw-Hill, 1995

\par
\begin{description}
\item [\colorbox{tagtype}{\color{white} \textbf{\textsf{PARAMETER}}}] \textbf{\underline{x}} probability
\end{description}

\rule{\linewidth}{0.5pt}
\subsection*{\textsf{\colorbox{headtoc}{\color{white} FUNCTION}
T\_CDF}}

\hypertarget{ecldoc:distributions.t_cdf}{}
\hspace{0pt} \hyperlink{ecldoc:Distributions}{Distributions} \textbackslash 

{\renewcommand{\arraystretch}{1.5}
\begin{tabularx}{\textwidth}{|>{\raggedright\arraybackslash}l|X|}
\hline
\hspace{0pt}\mytexttt{\color{red} REAL8} & \textbf{T\_CDF} \\
\hline
\multicolumn{2}{|>{\raggedright\arraybackslash}X|}{\hspace{0pt}\mytexttt{\color{param} (REAL8 x, REAL8 df)}} \\
\hline
\end{tabularx}
}

\par
Students t distribution integral evaluated between negative infinity and x. Translated from NIST SEL DATAPAC Fortran TCDF.f source

\par
\begin{description}
\item [\colorbox{tagtype}{\color{white} \textbf{\textsf{PARAMETER}}}] \textbf{\underline{x}} value of the evaluation
\item [\colorbox{tagtype}{\color{white} \textbf{\textsf{PARAMETER}}}] \textbf{\underline{df}} degrees of freedom
\end{description}

\rule{\linewidth}{0.5pt}
\subsection*{\textsf{\colorbox{headtoc}{\color{white} FUNCTION}
T\_PPF}}

\hypertarget{ecldoc:distributions.t_ppf}{}
\hspace{0pt} \hyperlink{ecldoc:Distributions}{Distributions} \textbackslash 

{\renewcommand{\arraystretch}{1.5}
\begin{tabularx}{\textwidth}{|>{\raggedright\arraybackslash}l|X|}
\hline
\hspace{0pt}\mytexttt{\color{red} REAL8} & \textbf{T\_PPF} \\
\hline
\multicolumn{2}{|>{\raggedright\arraybackslash}X|}{\hspace{0pt}\mytexttt{\color{param} (REAL8 x, REAL8 df)}} \\
\hline
\end{tabularx}
}

\par
Percentage point function for the T distribution. Translated from NIST SEL DATAPAC Fortran TPPF.f source


\rule{\linewidth}{0.5pt}
\subsection*{\textsf{\colorbox{headtoc}{\color{white} FUNCTION}
Chi2\_CDF}}

\hypertarget{ecldoc:distributions.chi2_cdf}{}
\hspace{0pt} \hyperlink{ecldoc:Distributions}{Distributions} \textbackslash 

{\renewcommand{\arraystretch}{1.5}
\begin{tabularx}{\textwidth}{|>{\raggedright\arraybackslash}l|X|}
\hline
\hspace{0pt}\mytexttt{\color{red} REAL8} & \textbf{Chi2\_CDF} \\
\hline
\multicolumn{2}{|>{\raggedright\arraybackslash}X|}{\hspace{0pt}\mytexttt{\color{param} (REAL8 x, REAL8 df)}} \\
\hline
\end{tabularx}
}

\par
The cumulative distribution function for the Chi Square distribution. the CDF for the specfied degrees of freedom. Translated from the NIST SEL DATAPAC Fortran subroutine CHSCDF.


\rule{\linewidth}{0.5pt}
\subsection*{\textsf{\colorbox{headtoc}{\color{white} FUNCTION}
Chi2\_PPF}}

\hypertarget{ecldoc:distributions.chi2_ppf}{}
\hspace{0pt} \hyperlink{ecldoc:Distributions}{Distributions} \textbackslash 

{\renewcommand{\arraystretch}{1.5}
\begin{tabularx}{\textwidth}{|>{\raggedright\arraybackslash}l|X|}
\hline
\hspace{0pt}\mytexttt{\color{red} REAL8} & \textbf{Chi2\_PPF} \\
\hline
\multicolumn{2}{|>{\raggedright\arraybackslash}X|}{\hspace{0pt}\mytexttt{\color{param} (REAL8 x, REAL8 df)}} \\
\hline
\end{tabularx}
}

\par
The Chi Squared PPF function. Translated from the NIST SEL DATAPAC Fortran subroutine CHSPPF.


\rule{\linewidth}{0.5pt}


