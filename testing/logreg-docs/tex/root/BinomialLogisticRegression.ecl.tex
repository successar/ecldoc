\chapter*{\color{headfile}
BinomialLogisticRegression
}
\hypertarget{ecldoc:toc:BinomialLogisticRegression}{}
\hyperlink{ecldoc:toc:root}{Go Up}

\section*{\underline{\textsf{IMPORTS}}}
\begin{doublespace}
{\large
Constants |
\_versions.ML\_Core.V3\_1\_0.ML\_Core.Interfaces |
\_versions.ML\_Core.V3\_1\_0.ML\_Core.Types |
}
\end{doublespace}

\section*{\underline{\textsf{DESCRIPTIONS}}}
\subsection*{\textsf{\colorbox{headtoc}{\color{white} MODULE}
BinomialLogisticRegression}}

\hypertarget{ecldoc:binomiallogisticregression}{}

{\renewcommand{\arraystretch}{1.5}
\begin{tabularx}{\textwidth}{|>{\raggedright\arraybackslash}l|X|}
\hline
\hspace{0pt}\mytexttt{\color{red} } & \textbf{BinomialLogisticRegression} \\
\hline
\multicolumn{2}{|>{\raggedright\arraybackslash}X|}{\hspace{0pt}\mytexttt{\color{param} (UNSIGNED max\_iter=200, REAL8 epsilon=Constants.default\_epsilon, REAL8 ridge=Constants.default\_ridge)}} \\
\hline
\end{tabularx}
}

\par
Binomial logistic regression using iteratively re-weighted least squares.

\par
\begin{description}
\item [\colorbox{tagtype}{\color{white} \textbf{\textsf{PARAMETER}}}] \textbf{\underline{max\_iter}} maximum number of iterations to try
\item [\colorbox{tagtype}{\color{white} \textbf{\textsf{PARAMETER}}}] \textbf{\underline{epsilon}} the minimum change in the Beta value estimate to continue
\item [\colorbox{tagtype}{\color{white} \textbf{\textsf{PARAMETER}}}] \textbf{\underline{ridge}} a value to populate a diagonal matrix that is added to a matrix help assure that the matrix is invertible.
\end{description}

\textbf{Children}
\begin{enumerate}
\item \hyperlink{ecldoc:binomiallogisticregression.getmodel}{GetModel}
: Calculate the model to fit the observation data to the observed classes
\item \hyperlink{ecldoc:binomiallogisticregression.classify}{Classify}
: Classify the observations using a model
\item \hyperlink{ecldoc:binomiallogisticregression.report}{Report}
: Report the confusion matrix for the classifier and training data
\end{enumerate}

\rule{\linewidth}{0.5pt}

\subsection*{\textsf{\colorbox{headtoc}{\color{white} FUNCTION}
GetModel}}

\hypertarget{ecldoc:binomiallogisticregression.getmodel}{}
\hspace{0pt} \hyperlink{ecldoc:binomiallogisticregression}{BinomialLogisticRegression} \textbackslash 

{\renewcommand{\arraystretch}{1.5}
\begin{tabularx}{\textwidth}{|>{\raggedright\arraybackslash}l|X|}
\hline
\hspace{0pt}\mytexttt{\color{red} DATASET(Types.Layout\_Model)} & \textbf{GetModel} \\
\hline
\multicolumn{2}{|>{\raggedright\arraybackslash}X|}{\hspace{0pt}\mytexttt{\color{param} (DATASET(Types.NumericField) observations, DATASET(Types.DiscreteField) classifications)}} \\
\hline
\end{tabularx}
}

\par
Calculate the model to fit the observation data to the observed classes.

\par
\begin{description}
\item [\colorbox{tagtype}{\color{white} \textbf{\textsf{PARAMETER}}}] \textbf{\underline{observations}} the observed explanatory values
\item [\colorbox{tagtype}{\color{white} \textbf{\textsf{PARAMETER}}}] \textbf{\underline{classifications}} the observed classification used to build the model
\item [\colorbox{tagtype}{\color{white} \textbf{\textsf{RETURN}}}] \textbf{\underline{}} the encoded model
\item [\colorbox{tagtype}{\color{white} \textbf{\textsf{OVERRIDE}}}] \textbf{\underline{}} True
\end{description}

\rule{\linewidth}{0.5pt}
\subsection*{\textsf{\colorbox{headtoc}{\color{white} FUNCTION}
Classify}}

\hypertarget{ecldoc:binomiallogisticregression.classify}{}
\hspace{0pt} \hyperlink{ecldoc:binomiallogisticregression}{BinomialLogisticRegression} \textbackslash 

{\renewcommand{\arraystretch}{1.5}
\begin{tabularx}{\textwidth}{|>{\raggedright\arraybackslash}l|X|}
\hline
\hspace{0pt}\mytexttt{\color{red} DATASET(Types.Classify\_Result)} & \textbf{Classify} \\
\hline
\multicolumn{2}{|>{\raggedright\arraybackslash}X|}{\hspace{0pt}\mytexttt{\color{param} (DATASET(Types.Layout\_Model) model, DATASET(Types.NumericField) new\_observations)}} \\
\hline
\end{tabularx}
}

\par
Classify the observations using a model.

\par
\begin{description}
\item [\colorbox{tagtype}{\color{white} \textbf{\textsf{PARAMETER}}}] \textbf{\underline{model}} The model, which must be produced by a corresponding getModel function.
\item [\colorbox{tagtype}{\color{white} \textbf{\textsf{PARAMETER}}}] \textbf{\underline{new\_observations}} observations to be classified
\item [\colorbox{tagtype}{\color{white} \textbf{\textsf{RETURN}}}] \textbf{\underline{}} Classification with a confidence value
\item [\colorbox{tagtype}{\color{white} \textbf{\textsf{OVERRIDE}}}] \textbf{\underline{}} True
\end{description}

\rule{\linewidth}{0.5pt}
\subsection*{\textsf{\colorbox{headtoc}{\color{white} FUNCTION}
Report}}

\hypertarget{ecldoc:binomiallogisticregression.report}{}
\hspace{0pt} \hyperlink{ecldoc:binomiallogisticregression}{BinomialLogisticRegression} \textbackslash 

{\renewcommand{\arraystretch}{1.5}
\begin{tabularx}{\textwidth}{|>{\raggedright\arraybackslash}l|X|}
\hline
\hspace{0pt}\mytexttt{\color{red} DATASET(Types.Confusion\_Detail)} & \textbf{Report} \\
\hline
\multicolumn{2}{|>{\raggedright\arraybackslash}X|}{\hspace{0pt}\mytexttt{\color{param} (DATASET(Types.Layout\_Model) model, DATASET(Types.NumericField) observations, DATASET(Types.DiscreteField) classifications)}} \\
\hline
\end{tabularx}
}

\par
Report the confusion matrix for the classifier and training data.

\par
\begin{description}
\item [\colorbox{tagtype}{\color{white} \textbf{\textsf{PARAMETER}}}] \textbf{\underline{model}} the encoded model
\item [\colorbox{tagtype}{\color{white} \textbf{\textsf{PARAMETER}}}] \textbf{\underline{observations}} the explanatory values.
\item [\colorbox{tagtype}{\color{white} \textbf{\textsf{PARAMETER}}}] \textbf{\underline{classifications}} the classifications associated with the observations
\item [\colorbox{tagtype}{\color{white} \textbf{\textsf{RETURN}}}] \textbf{\underline{}} the confusion matrix showing correct and incorrect results
\item [\colorbox{tagtype}{\color{white} \textbf{\textsf{OVERRIDE}}}] \textbf{\underline{}} True
\end{description}

\rule{\linewidth}{0.5pt}


